
\chapter{Neue Befehle}

Mit jedem Schritt, den er sich dem Ratssaal näherte, stieg sein Unwille. Der Kommandant drehte und 
wendete nervös das Pergament. Die Einladung zur Ratsversammlung, persönlich vom König unterzeichnet. 
Er bog um die Ecke und steuerte auf die große, mit Gravuren verzierte, Eichentüre zu. Zwei Wachen, 
gekleidet in die roten Uniformen Saleicas, bewachten die Tür. Jozah zeigte flüchtig die Einladung 
vor und sie nickten gelangweilt, zum Zeichen, dass er weiter gehen sollte. Doch er zögerte.\\
 Sollte er einfach eintreten oder anklopfen und warten, bis er gerufen wurde? Er war ein angesehener 
Kommandant; wenn er klopfte, könnte das als Unsicherheit verkannt werden. Ginge er einfach hinein, 
könnte sich der König beleidigt fühlen. Er seufzte frustriert. Jozah war ein Mann des Schwertes. Ein 
Taktiker. Aber kein Politiker. Wie er dieses ganze Gerede verachtete. Er entschied sich für den 
Mittelweg, klopfte gegen die Tür und trat kurz darauf ein, ohne auf Antwort zu warten. Er war gut 
darin, innerhalb weniger Augenblicke die Lage einzuschätzen. Diese Fähigkeit hatte er auf dem 
Schlachtfeld gelernt oder vielleicht auch nur für sich entdeckt. Auch jetzt erfasste er also in 
wenigen Sekunden die Situation.\\
 Der Ratssaal war einst ein schlichter Raum gewesen, der Ernsthaftigkeit, Erhabenheit und Macht 
ausgedrückt hatte, erzählte man sich zumindest. Jetzt wirkte alles grell, die Dekorationen 
erschlugen einen praktisch. Vorhänge aus seidenem Tuch zierten die großen Fenster aus buntem Glas. 
Der Boden war aus hellem Holz, in der Mitte des Raumes bedeckt mit einem Teppich, der schon aus 
dieser Entfernung extrem teuer und flauschig aussah. An den Wänden hingen pure Kostbarkeiten aus 
allen möglichen Grafschaften und vermutlich etliches aus den Kolonien. An sich sah der Raum nicht 
anders aus, als der Rest des Schlosses es mittlerweile tat. In dem Prunk wirkte der ovale Tisch in 
der Mitte fehl am Platz. Er war alt, ein noch bestehendes Zeugnis vieler Jahrhunderte vergangener 
Geschichte. Es heißt, er war übersehen von Kratzern und Narben, die wütende, mächtige Männer bei 
Ratsversammlungen mit ihren Messern und schwungvollen Bewegungen hinterlassen hatten. Ein Blutfleck 
stammte von Garischa, dem Helden des Südens, der sich das Leben nahm, als in der Versammlung 
beschlossen wurde, den Vorstoß in die Länder hinter den jetzigen Kolonien aufzugeben. Sein 
Lebenswerk sei beendet, hatte Garischa verkündet und sich selbst die Kehle durchgeschnitten. Es soll 
eine scheußliche Sauerei gewesen sein, aber der damalige König hatte es ruhig mitangesehen und ihn 
erst wegbringen lassen, als der letzte Lebensfunkte aus Garischas Augen gewichen war. Seine Worte zu 
dieser Tat kannte jedes Kind in Brom-Dalar: ``Wartet, wir haben nicht das Recht, seinen letzten 
Willen zu stören.``\\ 
Vielleicht lag ja deshalb der Teppich dort. \textit{Wenn der Selbstmord Flecken im Tisch 
hinterlassen hat, dann vermutlich auch im Parkett}, überlegte der Kommandant.\\
Jozah wusste nicht, ob diese Geschichten stimmten, immerhin war es gerade das erste Mal, dass er das 
Schloss betrat, und dann gleich den Ratssaal!\\
Einst soll bei den Ratssitzungen König Kareens jeder Stuhl besetzt gewesen sein, jetzt waren es 
lediglich drei. König Semric – sein Sohn und Nachfolger - hatte nach dem Tod seines Vaters dessen 
Vertraute vom Hof jagen lassen. Jeden einzelnen, bis auf Lerin, den mittlerweile alt gewordenen 
Offizier. Ihm gegenüber saß ein Mann, dessen Alter Jozah nicht einschätzen konnte. Der Kopf war 
kahl rasiert, die Augen erinnerten an den durchdringenden Blick einer Schlange, das Gesicht war 
faltenlos. Kleine goldene Ohrringe zierten sein rechtes Ohr. Seine Haut war übersät mit schwarzen, 
blauen und violetten Tätowierungen. Er trug weite Gewänder, farblich angepasst an seine 
Tätowierungen. Zwischen ihnen, am Kopf der Tafel, saß der junge König und klopfte ungeduldig mit 
den Fingern auf die Tischplatte. \\
König Semric war kaum 23 Jahre alt, und schon mehr als zehn Jahre an der Macht. Dunkelblonde Locken 
fielen ihm in die Stirn. Er wirkte vertieft in Gedanken. Es war ebenfalls das erste Mal, dass Jozah 
den Herrscher des Landes, für den er seit Jahren im Militär diente und in den Kolonien sein Leben 
riskiert hatte, aus der Nähe sah. Und Jozah war enttäuscht. Ohne  prachtvolle Paradepferde, in 
glänzend polierter Rüstung und ohne die mit Juwelen besetzte Krone – passend zum hoch erhobenen 
Zepter oder wahlweise Schwert – wirkte er lediglich wie ein gelangweilter junger Mann, der gerade 
den Ansatz eines Bauches entwickelte. Keine Frage, König Semric hatte ein fein geschnittenes 
Gesicht und Haare, um die ihn vielleicht sogar manches Weib beneiden würde. Aber alles in allem 
wirkte er wie dieser Raum. Er sollte eine wichtige, ernsthafte Rolle ausfüllen und hatte die 
Grundlagen dafür, aber war vollgestopft mit unwichtigen Eigenschaften und Details. Was 
Ernsthaftigkeit, Erhabenheit und Macht ausdrücken sollte, strahlte dagegen Prunk, Reichtum und... 
zugegeben, auch Macht aus. Aber Jozah war sich nicht sicher, ob das eine gute Art von Macht war. 
Doch was wusste er schon. Er hatte sein halbes Leben in Kasernen und im Militär verbracht, die 
andere Hälfte als fünftgeborenes Kind eines Handwerkers.\\

Die Schlangenaugen des Priesters waren auf ihn gerichtet, das spürte Jozah sofort. Mit jedem 
verstreichenden Augenblick fühlte er sich unwohler denn je. Er hatte sich noch nie mehr in Gefahr 
gefühlt, als jetzt. Nicht einmal, als einer der Rebellen der Kolonie mit seiner erhobenen Axt über 
ihm stand. Das war immerhin eine Gefahr gewesen mit der man rechnen konnte. Und das hier? Das war 
unberechenbarer Wahnsinn!\\
Der Offizier Lerin, Jozah hatte schon viel von ihm gehört, aber ihn auch noch nicht persönlich 
getroffen, saß mit dem Rücken zu ihm. Seiner Haltung nach zu urteilen, war er eingeschlafen. Oder 
aber sein Rücken war generell dermaßen gebeugt. Auf jeden Fall konnte Jozah sich nicht vorstellen, 
dass dieser Mann einst auf dem Exerzierplatz marschiert und stramm gestanden hatte. Oder generell 
mit einer Waffe in der Hand Feinden gegenüber stand, auch wenn man einiges über ihn gehört 
hatte. Diese Geschichten waren aber so alt, Jozah war noch nicht einmal vom allmächtigen Osyma 
erdacht gewesen. Nein, er maß sich nicht an, über ihn zu urteilen. Er wusste nur, dass er hier so 
schnell wie möglich weg wollte und zurück zu seiner kleinen Truppe oder in seine Kammer in der 
Kaserne.\\
''Ha...``, kam es aus dem Mund des Königs und mit einer trägen Handbewegung deutete er auf den 
Besucher. ''Das ist er? Dieser Kommandant...``\\
''Mi'Kaé``, brummte eine Stimme, die Jozah dem Offizier zuschrieb. \\
Unwillkürlich richtete er sich auf und hob das Kinn. Seine Hand ruhte an seinem Gürtel, sehnte sich 
unbewusst nach dem Gewicht seines Säbels, welchen er beim betreten des Schlosses hatte abgeben 
müssen.\\
''Klopfte an die Türe der Kaserne, als er elf Jahre alt war, wurde weggeschickt. Eine Woche später 
stand plötzlich ein hagerer Knirps beim Exerzieren dabei. Bis heute weiß keiner, wie er in die 
Kaserne gekommen ist. Er wurde ausgebildet, war damals vier Jahre jünger als seine Kameraden. Als 
er 15 war, bürgte sein Ausbilder dafür, dass er als vollwertiger Soldat anerkannt wurde. In diesem 
Alter fingen die anderen erst an. Drei Jahre im Wachdienst hier in der Hauptstadt, einzelne 
Verfolgungen von Verbrechern durch das Land. Fünf Jahre Dienst in zwei verschieden Garnisionen, in 
denen er unter anderem die Ausbildung neuer Rekruten üernahm. In den Grafschaften Ringen und Kanto. 
Er kam mit Bewertungen voller Lob von den Generälen wieder zurück in die Hauptstadt. Verbrachte 
einige Monate wieder bei der Wache und meldete sich schließlich freiwillig zum Dienst in den 
Kolonien. Da war er... 26 Jahre alt.``\\
''24, Ser``, unterbrach Jozah ihn abrupt und biss sich anschließend auf die Zunge. Es war selten 
eine gute Idee, jemandem, der Macht, Ansehen und Respekt gewohnt war, zu unterbrechen. Aber das 
alles kam ihm vor wie ein Theaterstück und es bereitete ihm eine Gänsehaut, dass dieser Mann so 
viel über ihn zu wissen schien. Auch wenn er damit hätte rechnen müssen.\\
König Semric hatte sich gemächlich in einem Stuhl zurück gelehnt und die Hände auf dem Bauch 
gefaltet, während der Priester sich keinen Millimeter bewegt hatte und so aufrecht wie am Anfang 
saß. \\
''Nun ja... die Kolonien. Sie waren ein halbes Jahr dort, als die ersten Rebellen sich offen gegen 
Saleica stellten. Fünf Jahre insgesamt dienten Sie ihrem Land dort. Sie waren bei etlichen 
Schlachten dabei und so weit mir zugetragen wurde, haben sie das Kommando bei einer Belagerung 
geführt. Nach dem siegreichen Ende wurden sie zum Kommandanten erhoben und blieben noch eine Weile 
dort. Vor drei Monaten kehrten Sie nach Hause zurück.``\\
Es war immer noch eine Aufzählung für den König, aber jetzt hatte Lerin ihn angesprochen und 
angesehen. Jozah fühlte sich dadurch etwas weniger wie ein fremdartiges Tier, welches man 
desinteressiert anstarrte. \\
''Das heißt?``, fragte König Semric in einem gleichgültigen Ton. \\
Jozah konnte es ihm nicht verdenken. Seine Geschichte war nichts besonderes. Jeder Soldat konnte 
etwas dramatisches, spannendes oder humorvolles erzählen, wenn es um sein Leben ging. Und er selbst 
hätte auch gut auf diese Zusammenfassung verzichten können. Außerdem kamen dadurch so manche 
schlimmen Erlebnisse wieder hoch, die er gerne weiter verdrängt hätte. Die Formulierung der 
siegreichen Belagerung war dagegen das Lächerlichste, was man sagen konnte. Er und die Soldaten 
dort hatten nicht gesiegt. Sie hatten überlebt. Kaum mehr, als ein Drittel der ursprünglichen 
Besatzung. Aber sie lebten und das durch Jozah, auch wenn er nicht wirklich davon überzeugt war, 
dass seine Entscheidung irgendetwas dazu beigetragen hatte.\\
''Alles in allem von einem Jungen aus der unteren Mittelschicht zu einem Kommandanten 
aufgestiegen.``\\
''Innerhalb von zwanzig Jahren ist das etwas sehr geringes``, bemerkte der Priester, ohne die Mimik 
zu verziehen. Ha, da konnte Jozah mithalten. Ebenso unbeweglich starrte er in die Schlangenaugen 
zurück.\\ 
''Nun... für einen Adeligen vielleicht, für einen Jungen so niederer Geburt, ohne Geld und Freunde 
ist es beachtlich``, ergänzte der König und betrachtete währenddessen seine Fingernägel.\\
Jozah erwiderte weiterhin den Blick des Priesters, während er sich fragte, wie dieser junge König, 
der noch nie sein Schwert benutzt oder auch nur in einen Scheißhaufen getreten war, dazu außerdem 
jünger als er selbst, ihn als Jungen bezeichnete. Es war klar, worauf Semric hinauswollte. Soldaten 
seiner Art blieben meist die Fußsoldaten, die Frontschweine, die als erstes draufgingen. Immerhin 
hatte er selbst zehn Jahre nach seiner Ausbildung diesen Posten innegehabt. Aber Jozah war nicht 
draufgegangen. Er hatte überlebt und dafür eine Beförderung und den Befehl über 50 Mann bekommen. 
Adelige Söhne bekamen das als Einstiegsposten im Militär.\\
''Wie wäre es mit einem höheren Posten, Mi'Kaé?``\\
Er konnte nicht verhindern, dass er einen Moment überrascht aussah. Der Kommandant verbeugte sich 
rasch, auch wenn der König ohnehin nicht hinsah, um dies zu verbergen. ''Mit Verlaub, Ser, ich bin 
in meiner Position wunschlos glücklich. Außerdem schulde ich dem Land und meinem König bereits zu 
viel für die Beförderung zum Kommandanten.``\\
Ach... er war einfach kein Politiker. Diesen Satz hatte er geübt, seit er die Einladung erhalten 
hatte. Und trotzdem brachte er ihn nicht so überzeugend rüber, wie er es sich gewünscht hätte, 
obwohl es sogar die Wahrheit war. Jozah wollte zurück in die Kaserne, unter Männern und Frauen die 
einander mit Respekt begegneten und deutlich weniger redeten. Knappe, präzise Antworten waren ihm 
lieber, und die Sehnsucht nach seinem Säbel stieg.\\
Immerhin hatte die Antwort die Aufmerksamkeit des Königs geweckt. Zum ersten Mal sah er Jozah 
richtig an. Und das überaus deutlich. Sein Blick glitt über die zerzausten dunklen Haare, die 
verschwitzte Stirn und die Uniform hinweg, bis zu den ledernen Stiefeln. \\
''Nun, setzen Sie sich endlich``, sagte Semric: ''Ich habe keine Lust mehr, mich ständig zu 
verdrehen, um Sie anzublicken!``\\
Jozah trat zu dem Stuhl neben dem Offizier. Einer von vielen Stühlen, die dicht an den Tisch her 
angeschoben waren und seit langem mal wieder benutzt wurden. \\
Einen Moment herrschte Schweigen. Der Offizier starrte auf die Tischplatte, der König fummelte an 
seinem Jackett und der Priester lächelte vor sich hin. Jozahs Blick huschte am Tisch entlang. Die 
Oberfläche war mit Einkerbungen und Schnitten versehen, Macken und Dellen. Er erkannte den ein oder 
anderen Weinfleck, aber das hatte nichts mit den spannenden Geschichten gemein. Spuren von Grischas 
Suizid waren ebenfalls keine zu sehen. Seine Enttäuschung sah man ihm vermutlich an, zumindest 
wurde das Lächeln des Priesters noch süffisanter als zuvor. Hisio-Mahar, Hohepriester der 
Hauptstadt und Sprecher aller Priester Saleicas. Wenn die Priester und Gläubigen sich Osymas Kinder 
nannten, dann war Hisio-Mahar persönlich der Bruder des Gottes. Böse Zungen meinten, er stelle sich 
gleich mit dem Allmächtigen. Er selbst bezeichnete sich in den Andachten als Augen, Ohren und Stimme 
des Herrn. ''Sagt Ihnen der Name A'Rik etwas, Kommandant Mi'Kaé?``\\
Jozahs Finger verkrampften sich unter den Tisch, während er überlegte. Er konnte sich nicht 
entsinnen, diesen Namen schon einmal gehört zu haben. Aber definitiv klang er Adelig. Na, wenn man 
danach ging, klang sein eigener aber ebenfalls nach Adel. Irgendwann soll es wohl mal ein winziges, 
armes Adelshaus Mi'Kaé gegeben haben. Vor Jahrhunderten vielleicht, bis sie vollends in Ungnade 
fielen und ihren mickrigen Titel aberkannt bekamen.\\
''Es tut mir Leid, aber ich kann mich nicht entsinnen``, gestand Jozah.\\
''Evin A'Rik``, erklärte Hisio-Mahar: ''Graf Merandilas. Die nördlichste Grafschaft Saleicas und 
die Grenze zu unserem Nachbarland Kasir. Die Familie A'Rik hielt die Grenze, seit die Grafschaft 
in Saleica eingegliedert wurde.``\\
''Nun, aber jetzt ist der Graf verstorben``, fügte der Priester hinzu: ''Seine junge Braut schickte 
uns einen Brief, in dem sie uns von dem Ableben des Grafen berichtete.``\\
''Ein natürliches Ableben?``, wagte Jozah nachzufragen.\\
In diesem Fall wüsste er nämlich noch weniger, warum er hier war. Üblicherweise schickte man einen 
fähigen Adeligen in die Grafschaft, der heiratete die Witwe und übernahm die Aufgaben der 
Verwaltung für den König.\\
''Mit Evin ist das letzte, lebende Mitglied der Familie A'Rik von uns gegangen. Er hinterließ keine 
lebenden Erben und seine Witwe ist erst 15 Jahre alt. Tochter eines armen Kaufmanns. Sarimé 
Sil'Vera, vielleicht sagt der Name Ihnen etwas.``\\
Das tat er. Sil'Vera war einst ein Mann gewesen, der hier an diesem Tisch saß. Vielleicht 
sogar auf genau diesem Stuhl, auf welchem Jozah nun saß. Die Tochter war zu jung, als dass er sie 
gekannt hätte. Aber mit einem Sohn der Familie hatte er Wachdienst gehalten. Was aus ihm geworden 
war, wusste er jedoch nicht.\\ 
''Ich nehme an, es gibt ein Problem?``\\
''Wahrlich``, sprach Lerim: ''Damals, als Merandila erobert wurde, schrieben die damaligen Herren 
einen Vertrag. Die Führer Merandilas willigten nur zur Kapitulation ein, wenn bestimmte Bedingungen 
erfüllt würden. Dazu gehörte vor allem, dass einige Traditionen des Volkes beibehalten werden. Eine 
davon ist, dass das Land von einem Merandil regiert werden muss. Damals verheiratete man also einen 
unserer Adeligen mit einer angesehenen Merandil und das Problem war erst einmal erledigt. Sobald 
aus der Verbindung ein Kind entsteht, gehört die ganze Familie zu Merandila. A'Rik war zwar nicht 
geliebt, aber anerkannt. Seine junge Witwe stammt aus Brom-Dalar, sie ist Saleicanerin.``\\
''Das ist Jahrhunderte her, die Menschen bezeichnen sich als Saleicaner``, warf der König ein.\\
''Die Adeligen, mein König. Die Reichen und Gläubigen. Das Bauernvolk opfert heute noch ihrer 
kleinen Göttin und nennt sich selbst Merandil``, verbesserte Lerim: ''Das Volk ist die Grundlage 
des Herrschers. Man sollte es nicht unnötig verärgern.``\\
''Und? Dort ist die Grenze, das heißt, wir haben ein Drittel unseres stehenden Heeres in Merandila 
und den zwei angrenzenden Grafschaften verteilt. Wenn sie sich beschweren, zeigen wir unsere 
Waffen. Fertig.``\\
Dem König war anzusehen, dass er diese Besprechung gerne hinter sich bringen würde. Jozah war mit 
ihm einer Meinung. Er hatte immer noch nicht erkannt, weshalb er hier war.\\
''Worin genau besteht das Problem?``, erkundigte er sich: ''Es lässt sich doch bestimmt ein 
Adeliger finden, der in Merandila lebt und das Mädchen heiraten kann. Anscheinend ist es ja nicht 
schwierig, als Merandil anerkannt zu werden.``\\
Der Hohepriester lächelte ununterbrochen weiter. ''Doch, die Lage an sich ist etwas prekär. Das 
Mädchen ist schwanger. Sollte es überleben, ist die Witwe nach Merandilschem Gesetz anerkannt.``\\
''Dann könnte man sie trotzdem noch an einen Adeligen verheiraten, oder nicht?``, fragte Jozah, an 
den König gewandt.\\
Wieder antwortete Hisio-Mahar auf seine Frage. ''Nein... nicht so schnell, wie wir es brauchen. 
Unsere Grenzen zu Kasir werden lediglich von einem 15 jährigen, schwangeren Mädchen gehalten.``\\
\textit{Und lediglich von den vier Garnisonen innerhalb der Grafschaft}, dachte Jozah sarkastisch.\\
''Und es wird Krieg geben. Saleica braucht einen erfahrenen Mann oben, der uns von der Lage 
berichtet, den Zustand der Garnisonen prüft und aufpasst, dass das Mädchen keinen Unsinn anstellt. 
Bevor es wirklich zur ersten Schlacht kommen wird, werden wir einen geeigneten Kandidaten als 
Gemahl für sie ausgewählt haben. Wenn Osyma uns gnädig ist, stirbt das Kind, bevor es überhaupt 
geboren ist und die Merandil werden einsehen, dass sie dringend einen neuen Grafen benötigen. 
Reitet hoch und seht nach dem Rechten. Ich denke, zwei Tage müssten genügen um sich auf die Reise 
vorzubereiten. Wir wünschen eine angenehmen Aufenthalt, nehmen Sie einen warmen Mantel mit, die 
Winter dort sind kühl.``\\
Kaum hatte er ausgesprochen, stand König Semric schon auf und brummte: ''Schönen Tag noch.``
Das sollte wohl bedeuten, dass Jozah gehen durfte. Eilig stand er auf und verneigte sich tief in 
die Richtung des Königs. Zu seiner Überraschung erhob sich der ältere Offizier ebenfalls, neigte 
den Kopf und folgte dich hinter ihm aus dem Ratssaal. Die Tür fiel hinter ihnen ins Schloss und 
Jozah betrachtete sie noch einen Moment benommen. Das kam alles viel zu plötzlich. Und warum er? 
Gab es keinen anderen, der auf das Kind aufpassen konnte?\\
Lerim legte ihm eine schwere Hand auf die Schulter. ''Nun, ich begleite Sie ein Stück. Ich nehme 
an, Sie müssen ihre Waffe noch vom Wachposten abholen?``\\
''Äh... ja, stimmt``, murmelte Jozah, der den Säbel nun doch fast vergessen hatte.\\
Lerim machte eine Geste, die ihm zu verstehen gab, dass er losgehen sollte. Der Offizier blieb an 
seiner Seite und bemerkte Jozahs Zweifel. Er sprach mit leiser Stimme: ''Täuschen Sie sich nicht, 
die Sache ist ein ernstes Problem. Unser geliebter König unterschätzt das Volk Merandilas. Es ist 
stolz.``\\
''Bauern, Handwerker und Bettler?``, fragte Jozah zweifelnd.
Lerim lachte heißer. ''Natürlich. Schließlich sitzt hier vor mir auch ein Handerdwerkssohn, der es 
schafft, in der Hauptstadt zu überleben und sich einen Namen gemacht hat! Der Priester hat nicht 
nur seine Hände im Spiel, er steuert uns alle wie Marionetten.``\\
''Warum lässt der König das zu?``, flüsterte Jozah.\\
''Weil er ein Narr ist. Er ist noch der selbe Junge wie vor zehn Jahren, als die Priester ihm die 
Krone auf den Kopf gesetzt haben. Wenn es nach Hisio Ma'Har ginge, wäre schon längst ein 
Adeliger vor die Tore Merandilas spaziert und hätte sich als neuer Graf ausgerufen. Das Volk wird 
es niemals akzeptieren. Das Volk will A'Riks anerkannten Erben oder wenigstens seine Witwe. Es sind 
Traditionen, so alt wie das Leben selbst. Sie waren im Krieg, Mi'Kaé. Und Sie waren kein 
Schlächter, wie so viele Helden Saleicas. Sie lernten die Menschen dort kennen, weckten ihr 
Vertrauen. Sie sahen Sie als einen der ihren an oder zumindest als einen Menschen, dessen Ziel 
Friede und nicht Tod ist! Die Menschen in den Kolonien sind den Merandil ähnlich. Sie kapitulierten 
vor Saleica, um ihre Kinder zu schützen. Aber niemals wird ihr Wille gebrochen. Sie halten sich an 
Vereinbarungen, aber vergessen ihre Götter nicht. Ihr habt einmal das Vertrauen solcher Menschen 
erlangt, reitet nach Merandil und helft der Gräfin, dass sie es auch erlangt. Der Krieg gegen Kasir 
wird schneller kommen, als mir lieb ist und wir können keinen Bürgerkrieg gebrauchen.``\\
Sie waren mittlerweile bei den Posten, der Jozahs Säbel entgegen genommen hatte, angekommen. Der 
Kommandant merkte, wie Lerin aufmerksamer wurde, als er den Säbel entgegen nahm.\\
''Darf ich?``, fragte Lerim, aber wartete nicht auf eine Antwort.\\
Mit einer präzisen Bewegung zog er den Säbel aus der Scheide und schwang in mit lockerem Handgelenk 
durch die Luft. Er schnalzte mit der Zunge. ''Hübsches Ding. Ihrer?``\\
''Nein Ser. Aus der Waffenkammer der Kaserne.``\\
''Aber Sie haben eine eigene Waffe?``\\
Jozah zögerte und nickte. ''Ein Schwert, Ser.``\\
''Gut gut, ich verstehe schon. Die Mode!``, lachte Lerim: ''Unsere verehrten Adeligen zeigen sich 
momentan nur mit solchen Säbeln. Ich bevorzuge in einem Kampf ebenfalls lieber ein Schwert oder 
eine Axt.``\\
Lerim reichte ihm den Säbel und Jozah befestigte ihn wieder an seinem Gürtel. Aber eine Frage ließ 
ihn einfach nicht los und leise stellte er diese: ''Ser, woher wisst Ihr so viel über die 
Grafschaft?``\\
Lerim richtete sich auf und seine Augen funkelten jungenhaft. ''Die Helle möge Sie schützen, 
Mi'Kaé. Sagt, haben Sie ein gutes Pferd und gute Männer, die Sie begleiten  können? Lassen Sie 
jeden, an dem Sie zweifeln, hier. Sie können da oben keine Verräter gebrauchen.``\\
Der Offizier sprach in Rätseln, aber er wechselte so schnell das Thema, dass Jozah sich keine 
Gedanken darüber machen konnte. Verwirrt antwortete er: ''Ich würde für jeden meiner Männer durch 
das Feuer gehen und sie für mich. Wir waren gemeinsam im Krieg, ich glaube nicht, dass auch nur 
einer von ihnen zum Verrat fähig wäre. Und ein Pferd... nein, Ser.``\\
''Sie haben doch bestimmt eines, welches Sie in der Kaserne reiten?``\\
''Ja, Ser. Aber es gehört Saleica und dem Militär.``\\
''Betrachten Sie es als Abschiedsgeschenk. Ein Mann sollte mit seinem eigenem Schwert und seinem 
eigenem Pferd in den Krieg ziehen, wenn er schon sein eigenes Leben aufs Spiel setzt. Melden Sie 
sich, Mi'Kaé!`` \\


