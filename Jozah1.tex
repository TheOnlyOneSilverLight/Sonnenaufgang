
\chapter{Neue Befehle}

Mit jedem Schritt, den er sich dem Ratssaal näherte, stieg sein Unwille. Der Kommandant drehte und 
wendete nervös das Pergament. Die Einladung zur Ratsversammlung, persönlich vom König unterzeichnet. 
Er bog um die Ecke und steuerte auf die große, mit Gravuren verzierte Eichentüre zu. Zwei Wachen, 
gekleidet in die roten Uniformen Saleicas, bewachten die Tür. Jozah zeigte flüchtig die Einladung 
vor und sie nickten gelangweilt, zum Zeichen, dass er weiter gehen sollte. Doch er zögerte.\\
 Sollte er einfach eintreten oder anklopfen und warten, bis er gerufen wurde? Er war ein angesehener 
Kommandant; wenn er klopfte, könnte das als Unsicherheit verkannt werden. Ginge er einfach hinein, 
könnte sich der König beleidigt fühlen. Er seufzte frustriert. Jozah war ein Mann des Schwertes. Ein 
Taktiker. Aber kein Politiker. Wie er dieses ganze Gerede verachtete. Er entschied sich für den 
Mittelweg, klopfte gegen die Tür und trat kurz darauf ein, ohne auf Antwort zu warten. Er war gut 
darin, innerhalb weniger Augenblicke die Lage einzuschätzen. Diese Fähigkeit hatte er auf dem 
Schlachtfeld gelernt oder vielleicht auch nur für sich entdeckt. Auch jetzt erfasste er also in 
wenigen Sekunden die Situation.\\
 Der Ratssaal war einst ein schlichter Raum gewesen, der Ernsthaftigkeit, Erhabenheit und Macht 
ausgedrückt hatte, erzählte man sich zumindest. Jetzt wirkte alles grell, die Dekorationen 
erschlugen einen praktisch. Vorhänge aus seidenem Tuch zierten die großen Fenster aus buntem Glas. 
Der Boden war aus hellem Holz, in der Mitte des Raumes bedeckt mit einem Teppich, der schon aus 
dieser Entfernung extrem teuer und flauschig aussah. An den Wänden hingen pure Kostbarkeiten aus 
allen möglichen Grafschaften und vermutlich etliches aus den Kolonien. An sich sah der Raum nicht 
anders aus, als der Rest des Schlosses es mittlerweile tat. In dem Prunk wirkte der ovale Tisch in 
der Mitte fehl am Platz. Er war alt, ein noch bestehendes Zeugnis vieler Jahrhunderte vergangener 
Geschichte. Es heißt, er war übersehen von Kratzern und Narben, die wütende, mächtige Männer bei 
Ratsversammlungen mit ihren Messern und schwungvollen Bewegungen hinterlassen hatten. Ein Blutfleck 
stammte von Garischa, dem Helden des Südens, der sich das Leben nahm, als in der Versammlung 
beschlossen wurde, den Vorstoß in die Länder hinter den jetzigen Kolonien aufzugeben. Sein 
Lebenswerk sei beendet, hatte Garischa verkündet und sich selbst die Kehle durchgeschnitten. Es soll 
eine scheußliche Sauerei gewesen sein, aber König Kareen hatte es ruhig mitangesehen und ihn 
erst wegbringen lassen, als der letzte Lebensfunkte aus Garischas Augen gewichen war. Seine Worte zu 
dieser Tat kannte jedes Kind in Brom-Dalar: ``Wartet, wir haben nicht das Recht, seinen letzten 
Willen zu stören.``\\ 
Vielleicht lag ja deshalb der Teppich dort. \textit{Wenn der Selbstmord Flecken im Tisch 
hinterlassen hat, dann vermutlich auch im Parkett}, überlegte der Kommandant.\\
Jozah wusste nicht, ob diese Geschichten stimmten, immerhin war es gerade das erste Mal, dass er das 
Schloss betrat - und dann gleich den Ratssaal!\\
Einst soll bei den Ratssitzungen König Kareens jeder Stuhl besetzt gewesen sein, jetzt waren es 
lediglich drei. König Semric – sein Sohn und Nachfolger - hatte nach dem Tod seines Vaters dessen 
Vertraute vom Hof jagen lassen. Jeden einzelnen, bis auf Lerin, den mittlerweile alt gewordenen 
Offizier. Ihm gegenüber saß ein Mann, dessen Alter Jozah nicht einschätzen konnte. Der Kopf war 
kahl rasiert, die Augen erinnerten an den durchdringenden Blick einer Schlange, das Gesicht war 
faltenlos. Kleine goldene Ohrringe zierten sein rechtes Ohr. Seine Haut war übersät mit schwarzen, 
blauen und violetten Tätowierungen. Er trug weite Gewänder, farblich angepasst an seine 
Tätowierungen. Zwischen ihnen, am Kopf der Tafel, saß der junge König und klopfte ungeduldig mit 
den Fingern auf die Tischplatte. \\
König Semric war kaum 23 Jahre alt, und schon mehr als zehn Jahre an der Macht. Dunkelblonde Locken 
fielen ihm in die Stirn. Er wirkte vertieft in Gedanken. Es war ebenfalls das erste Mal, dass Jozah 
den Herrscher des Landes, für den er seit Jahren im Militär diente und in den Kolonien sein Leben 
riskiert hatte, aus der Nähe sah. Und Jozah war enttäuscht. Ohne prachtvolle Paradepferde, in 
glänzend polierter Rüstung und ohne die mit Juwelen besetzte Krone – passend zum hoch erhobenen 
Zepter oder wahlweise Schwert – wirkte er lediglich wie ein gelangweilter junger Mann, der gerade 
den Ansatz eines Bauches entwickelte. Keine Frage, König Semric hatte ein fein geschnittenes 
Gesicht und Haare, um die ihn vielleicht sogar manches Weib beneiden würde. Aber alles in allem 
wirkte er wie dieser Raum. Er sollte eine wichtige, ernsthafte Rolle ausfüllen und hatte die 
Grundlagen dafür, aber war vollgestopft mit unwichtigen Eigenschaften und Details. Was 
Ernsthaftigkeit, Erhabenheit und Macht ausdrücken sollte, strahlte dagegen Prunk, Reichtum und... 
zugegeben, auch Macht aus. Aber Jozah war sich nicht sicher, ob das eine gute Art von Macht war. 
Doch was wusste er schon. Er hatte sein halbes Leben in Kasernen und im Militär verbracht, die 
andere Hälfte als fünftgeborenes Kind eines Handwerkers.\\
Jozah verneigte sich tief vor seinem Herrscher und wartete auf irgendetwas, was ihm zeigte, wie es 
weitergehen sollte.
Doch es blieb still im Saal. Die Schlangenaugen des Priesters waren auf ihn gerichtet, das spürte 
Jozah sofort. Mit jedem verstreichenden Augenblick fühlte er sich unwohler denn je. Er hatte sich 
noch nie mehr in Gefahr gefühlt, als jetzt. Nicht einmal, als einer der Rebellen der Kolonie mit 
seiner erhobenen Axt über ihm stand. Das war immerhin eine Gefahr gewesen mit der man rechnen 
konnte. Und das hier? Das war unberechenbarer Wahnsinn!\\
Der Offizier Lerin, Jozah hatte schon viel von ihm gehört, aber ihn auch noch nicht persönlich 
getroffen, saß mit dem Rücken zu ihm. Seiner Haltung nach zu urteilen, war er eingeschlafen. Oder 
aber sein Rücken war generell dermaßen gebeugt. Auf jeden Fall konnte Jozah sich nicht vorstellen, 
dass dieser Mann einst auf dem Exerzierplatz marschiert und stramm gestanden hatte. Oder generell 
mit einer Waffe in der Hand Feinden gegenüber stand, auch wenn man einiges über ihn gehört 
hatte. Diese Geschichten waren aber so alt, Jozah war noch nicht einmal vom allmächtigen Osyma 
erdacht gewesen. Nein, er maß sich nicht an, über ihn zu urteilen. Er wusste nur, dass er hier so 
schnell wie möglich weg wollte und zurück zu seiner kleinen Truppe oder in seine Kammer in der 
Kaserne.\\
''Ha...``, kam es aus dem Mund des Königs und mit einer trägen Handbewegung deutete er auf den 
Besucher. ''Das ist er? Dieser Kommandant...``\\
''Mi'Kaé``, brummte eine Stimme, die Jozah dem Offizier zuschrieb. \\
Unwillkürlich richtete er sich auf und hob das Kinn. Seine Hand ruhte an seinem Gürtel, sehnte sich 
unbewusst nach dem Gewicht seines Säbels, welchen er beim betreten des Schlosses hatte abgeben 
müssen.\\
''Klopfte an die Türe der Kaserne, als er elf Jahre alt war, wurde weggeschickt. Eine Woche später 
stand plötzlich ein hagerer Knirps beim Exerzieren dabei``, sagte der Offizier tonlos: ''Bis 
heute weiß keiner, wie er in die Kaserne gekommen ist. Er wurde ausgebildet, war damals vier Jahre 
jünger als seine Kameraden. Als er 15 war, bürgte sein Ausbilder dafür, dass er als vollertiger 
Soldat anerkannt wurde. In diesem Alter fingen die anderen erst an. Drei Jahre im Wachdienst hier in 
der Hauptstadt, einzelne Verfolgungen von Verbrechern durch das Land. Fünf Jahre Dienst in zwei 
verschieden Garnisionen, in denen er unter anderem die Ausbildung neuer Rekruten üernahm. In den 
Grafschaften Ringen und Kanto. Er kam mit Bewertungen voller Lob von den Generälen wieder zurück in 
die Hauptstadt. Verbrachte einige Monate wieder bei der Wache und meldete sich schließlich 
freiwillig zum Dienst in den Kolonien. Da war er... 26 Jahre alt.``\\
''24, Ser``, unterbrach Jozah ihn abrupt und biss sich anschließend auf die Zunge. Es war selten 
eine gute Idee, jemandem, der Macht, Ansehen und Respekt gewohnt war, zu unterbrechen. Aber das 
alles kam ihm vor wie ein Theaterstück und es bereitete ihm eine Gänsehaut, dass dieser Mann so 
viel über ihn zu wissen schien. Auch wenn er damit hätte rechnen müssen.\\
König Semric hatte sich gemächlich in einem Stuhl zurück gelehnt und die Hände auf dem Bauch 
gefaltet.\\
''Nun ja... die Kolonien. Sie waren ein halbes Jahr dort, als die ersten Rebellen sich offen 
gegen Saleica stellten. Fünf Jahre insgesamt dienten Sie ihrem Land dort. Sie waren bei etlichen 
Schlachten dabei und so weit mir zugetragen wurde, haben sie das Kommando bei einer Belagerung 
geführt. Nach dem siegreichen Ende wurden sie zum Kommandanten erhoben und blieben noch eine Weile 
dort. Vor drei Monaten kehrten Sie nach Hause zurück.``\\
Es war immer noch eine Aufzählung für den König, aber jetzt hatte Lerin ihn angesprochen und 
angesehen. Jozah fühlte sich dadurch etwas weniger wie ein fremdartiges Tier, welches man 
desinteressiert anstarrte. \\
''Das heißt?``, fragte König Semric in einem gleichgültigen Ton. \\
Jozah konnte es ihm nicht verdenken. Seine Geschichte war nichts besonderes. Jeder Soldat konnte 
etwas dramatisches, spannendes oder humorvolles erzählen, wenn es um sein Leben ging. Und er selbst 
hätte auch gut auf diese Zusammenfassung verzichten können. Außerdem kamen dadurch so manche 
schlimmen Erlebnisse wieder hoch, die er gerne weiter verdrängt hätte. Die Formulierung der 
siegreichen Belagerung war dagegen das Lächerlichste, was man sagen konnte. Er und die Soldaten 
dort hatten nicht gesiegt. Sie hatten überlebt. Kaum mehr, als ein Drittel der ursprünglichen 
Besatzung. Aber sie lebten und das durch Jozah, auch wenn er nicht wirklich davon überzeugt war, 
dass seine Entscheidung irgendetwas dazu beigetragen hatte.\\
''Alles in allem von einem Jungen aus der unteren Mittelschicht zu einem Kommandanten 
aufgestiegen.``\\
''Innerhalb von fünfzehn Jahren ist das etwas sehr geringes``, bemerkte der Priester, ohne die 
Mimik zu verziehen. Da konnte Jozah mithalten. Ebenso unbeweglich starrte er zurück.\\ 
''Nun... für einen Adeligen vielleicht, für einen Jungen so niederer Geburt, ohne Geld und Freunde 
ist es beachtlich``, ergänzte der König.\\
Jozah erwiderte weiterhin den Blick des Priesters, während er sich fragte, wie dieser junge König, 
der noch nie sein Schwert benutzt oder auch nur in einen Scheißhaufen getreten war, dazu außerdem 
jünger als er selbst, ihn als Jungen bezeichnete. Es war klar, worauf Semric hinauswollte. Soldaten 
seiner Art blieben meist die Fußsoldaten, die Frontschweine, die als erstes draufgingen. Immerhin 
hatte er selbst zehn Jahre nach seiner Ausbildung diesen Posten innegehabt. Aber Jozah war nicht 
draufgegangen. Er hatte überlebt und dafür eine Beförderung und den Befehl über 50 Mann bekommen. 
Adelige Söhne bekamen das als Einstiegsposten im Militär.\\
''Wie wäre es mit einem höheren Posten, Mi'Kaé?``\\
Er konnte nicht verhindern, dass er einen Moment überrascht aussah. Der Kommandant verbeugte sich 
rasch, auch wenn der König ohnehin nicht hinsah, um dies zu verbergen. ''Mit Verlaub, Ser, ich bin 
in meiner Position wunschlos glücklich. Außerdem schulde ich dem Land und meinem König bereits zu 
viel für die Beförderung zum Kommandanten.``\\
Ach... er war einfach kein Politiker. Diesen Satz hatte er geübt, seit er die Einladung erhalten 
hatte. Und trotzdem brachte er ihn nicht so überzeugend rüber, wie er es sich gewünscht hätte, 
obwohl es sogar die Wahrheit war. Jozah wollte zurück in die Kaserne, unter Männern und Frauen die 
einander mit Respekt begegneten und deutlich weniger redeten. Knappe, präzise Antworten waren ihm 
lieber - und die Sehnsucht nach seinem Säbel stieg.\\
Immerhin hatte die Antwort die Aufmerksamkeit des Königs geweckt. Zum ersten Mal sah er Jozah 
richtig an. Und das überaus deutlich. Sein Blick glitt über die zerzausten dunklen Haare, die 
verschwitzte Stirn und die Uniform hinweg, bis zu den ledernen Stiefeln. \\
''Nun, setzen Sie sich endlich``, forderte Semric.\\
Jozah trat zu dem Stuhl neben dem Offizier. Einer von vielen Stühlen, die seit einem Jahrzehnt 
nicht mehr benutzt wurden.\\
Einen Moment herrschte Schweigen. Der Offizier starrte auf die Tischplatte, der König fummelte an 
seinem Jackett und der Priester lächelte vor sich hin. Jozahs Blick huschte am Tisch entlang. Die 
Oberfläche war mit Einkerbungen und Schnitten versehen, Macken und Dellen. Er erkannte den ein oder 
anderen Weinfleck, aber das hatte nichts mit den spannenden Geschichten gemein. Spuren von Grischas 
Suizid waren ebenfalls keine zu sehen. Seine Enttäuschung sah man ihm vermutlich an, zumindest 
wurde das Lächeln des Priesters noch süffisanter als zuvor. Hisio-Mahar, Hohepriester der 
Hauptstadt und Sprecher aller Priester Saleicas. Wenn die Priester und Gläubigen sich Osymas Kinder 
nannten, dann war Hisio-Mahar persönlich der Bruder des Gottes. Böse Zungen meinten, er stelle sich 
gleich mit dem Allmächtigen. Er selbst bezeichnete sich in den Andachten als Augen, Ohren und Stimme 
des Herrn. ''Sagt Ihnen der Name Mechir etwas, Mi'Kae?``\\
Jozahs Finger verkrampften sich unter dem Tisch, während er überlegte. Der Name kam ihm bekannt 
vor, aber er kam nicht drauf.\\
Der König klopfte ungeduldig mit den Fingern auf den Tisch. ''Arsas Mechir ist der kasirische 
König. Das einzige Königreich auf unserer Insel, das wir nicht eroberten. Jeden Krieg gegen 
Kasir haben wir verloren.``\\
Peinlich berührt kniff Jozah die Lippen fest zusammen. Er versuchte es zu überspielen und fragte 
direkt: ''Und was ist mit König Mechir?``\\
Semric atmete tief aus. ''Er hat mir einen Brief geschrieben. Persönlich. Er setzt ein 
Ultimatum.``\\
''Mein König``, unterbrach Hisio-Mahar den jungen Mann und legte ihm zur Bestärkung seiner Worte 
die Hand auf die Schulter: ''Einen einfachen Kommandanten interessieren keine politischen 
Details.``\\
Jozah traute seinen Sinnen kaum, aber der König setzte nicht erneut zum sprechen an. Etwas 
frustriertes und hilfloses lag in seinem Blick, ehe er sich wieder zurück lehnte und eine 
unbeteiligte Miene zur Schau trug.\\
''Einen General aber schon``, entschied Lerin und sah Hisio-Mahar herausfordernd an: ''Erläutert 
doch selbst den politischen Hintergrund, Priester. Schließlich ist das alles Eure Schuld.``\\
Der Hohepriester überging die Beleidigung und erwiderte in knappen Worten: ''Einer unserer Spione 
hat den Kontakt abgebrochen. Irgendwas ging schief. Er wollte mir... Pläne besorgen, musste dafür 
aber in der Gunst des Königs aufsteigen. Um zu Beweisen, dass er ihm treu ergeben war, 
organsisierte er einen Anschlag auf sich selbst, fasste den Attentäter und richtete ihn hin. In der 
Folter haben sie ihm entlockt, dass Saleica ihn geschickt hätte. Alles verlief nach Plan... und 
dann fand man seinen Erben mit aufgeschlitzter Kehle. Ich habe nichts mehr von Karkos gehört.``\\
''Und?``, fragte Jozah: ''Waren wir es?``\\
Alle - selbst der König - sahen Hisio-Mahar prüfend an.\\
''Das alles war Karkos eigene Idee. Aber offensichtlich nutzt er den Mord um endgültig seine 
Stellung am kasrischen Hof zu sichern. Er hat König Mechir offenbar gesagt, dass Saleica ihn als 
Spion rekrutieren wollte und auf seine Ablehnung hin erst einen Anschlag auf ihn versucht und 
schließlich seinen Sohn ermordet hätten.``\\
''Schlau``, bemerkte Lerin: ''Hättet Ihr bestimmt genauso getan, Priester.``\\
''Merich schickt Saleica ein Ultimatum. Entweder liefern wir den Drahtzieher aus oder er kappt alle 
Handelsbeziehungen und den Waffenstillstand. Er will...``, erklärte Semric.\\
''Er lügt``, entschied Hisio-Maharr gereizt, als müsste er etwas offensichtliches zum wiederholten 
Male aussprechen: ''Kasirs halber Handel läuft über uns. Noch mehr über unsere Häfen.``\\
''Er will eine Verhandlung. Einen fairen Prozess``, beendete der Königs ohne etwas zu der 
respektlosen Unterbrechung zu sagen.\\
Jozah blickte zum Offizier. Er wirkte noch als der Vernünftigste hier im Raum. Der Priester würde 
niemals gestehen und der König traute sich kaum zu reden. ''Dazu hat er jedes Recht.``\\
Lerin nickte zustimmend. ''Der Glaube und die Krone haben entschieden, darauf nicht einzugehen``, 
sagte der alte Mann: ''Da kommt Ihr nun ins Spiel, Mi'Kaé.``\\
Der Soldat lehnte sich unwillkürlich vor, hoffte, dass es nun endlich zu klaren Befehlen kommen 
würde.\\
''Evin A'Rik ist der Graf Merandilas und Wächter der Grenze. Er kennt die Gegend, die 
Besatzung der Garnisionen und den Feind. Wir haben zwei stehende Heere im Norden, die über die 
Grenze wachen. Kaum einer von denen da oben war je in einer Schlacht. Ihr dagegen habt euch in den 
Kolonien bewiesen. Reitet mit Eurer Truppe hoch, Mi'Kaé. Ihr unterstützt den Grafen, prüft den 
Zustand der Garnisonen, berichtet dem König und passt auf, dass keiner der Generäle da oben Mist 
baut.``\\
''Vorsichtsmaßnahme oder wird es zum Krieg kommen?``, fragte Jozah.\\
''Es wird immer irgendwann zum Krieg kommen.``\\

''Danke``, murmelte Jozah zu der Wache, die ihm seinen Säbel wieder gab.\\
Eilig hatte er den kürzesten Weg aus dem Schloss gesucht, während er immer noch versuchte, zu 
verstehen, was im Ratssaal passiert war.\\
''Darf ich?``, fragte plötzlich Lerin, der vor ihm aufragte.\\
\textit{Wie bei Osymas Namen ist der so schnell hier her gekommen?}, fragte Jozah sich und blickte 
zu dem Offizier auf. Er überagte ihn um einen halben Kopf. Von dem müden alten Mann aus dem Saal 
war kaum noch was zu erkennen. Widerstandslos hielt Jozah ihm den Griff der Waffe hin. Mit einer 
präzisen Bewegung zog er den Säbel aus der Scheide und schwang in mit lockerem Handgelenk 
durch die Luft. Er schnalzte mit der Zunge. ''Hübsches Ding. Ihrer?``\\
''Nein, Herr. Aus der Waffenkammer der Kaserne.``\\
''Aber Ihr habt eine eigene Waffe?``\\
Jozah zögerte und nickte. ''Ein Schwert, Herr.``\\
''Gut gut, ich verstehe schon. Die Mode!``, lachte Lerim: ''Unsere verehrten Adeligen zeigen sich 
momentan nur mit solchen Säbeln. Ich bevorzuge in einem Kampf ebenfalls lieber ein Schwert oder 
eine Axt.``\\
Lerim reichte ihm den Säbel und Jozah befestigte ihn wieder an seinem Gürtel. ''Ich bin etwas 
verwirrt``, gestand er leise.\\
''Wer ist das nicht``, entgegnete der Offizier und folgte Jozah die Stufen hinunter. ''Saleica 
braucht Euch da oben. Und nicht nur wegen Kasir. Merandila ist die Jüngste der saleicanischen 
Grafschaften. Vor wenigen Generationen noch waren sie ein erblühendes Königreich. Es sind 
Traditionen, so alt wie das Leben selbst. Ihr wart im Krieg, Mi'Kaé. Und Ihr wart kein 
Schlächter, wie so viele Helden Saleicas. Ihr lerntet die Menschen dort kennen, erlangten ihr 
Vertrauen. Sie sahen Euch als einen der ihren an oder zumindest als einen Menschen, dessen Ziel 
Friede und nicht Tod ist! Die Menschen in den Kolonien sind den Merandil ähnlich. Sie kapitulierten 
vor Saleica, um ihre Kinder und ihr Erbe zu schützen.``\\
''Ihr erwarten einen Bürgerkrieg``, schlussfolgerte Jozah: ''Aus welchem Grund? Der Wirtschaft geht 
es dort gut.``\\
''Glaube, Mi'Kaé. Es ging immer um den Glauben. Osyma ist nicht gnädig. Er akzeptiert keine anderen 
Götter neben sich. Die Merandil werden ihrer Göttin jedoch nicht abschwören. Sie sind zu 
überzeugt von ihr, auch wenn die Priester in der Grafschaft vieles versuchen. Unser König sieht 
diesen Konflikt nicht. Er ist noch immer der selbe Junge wie vor zehn Jahren, als die Priester ihm 
die Krone auf den Kopf setzten. Und Hisio-Maharr würde sich nur ins eigene Fleisch schneiden, wenn 
er zugäbe, welche religiösen Konflikte es mit den Merandil gibt. Solange niemand hinschaut kann er 
im Namen des Königs tun was er will. Habt Ihr ein schnelles Pferd?``\\
''Äh, ich reite ein Tier aus der Kaserne.``\\
''Betrachtet es als Abschiedsgeschenk. Ein Mann sollte mit seinem eigenem Schwert und seinem 
eigenem Pferd in den Krieg ziehen, wenn er schon sein eigenes Leben aufs Spiel setzt. Meldet 
Euch, Mi'Kaé.`` \\




Der Halbmond beschien die Straßen der Hauptstadt, hüllte sie in ein gespenstisches Licht. Aber die 
Nacht war keineswegs still. Die Mitte der Hauptstadt war fast schon als luftig zu bezeichnen. 
Breite, gepflasterte Straßen die immer wieder von ebenen Flächen abgelöst würden. Markt- oder 
Schauplätze die genügend Raum für die freizeitlichen Aktivitäten des Adels erübrigten. Entfernte 
man 
sich von der Stadtmitte, wurden die Häuser höher und schmaler, die Gassen enger, die Nacht stiller. 
Ein Volk, welches solch ein Geltungsbedürfnis hatte wie die Saleicaner, brauchte genügend Raum um 
sich auszuleben. Zumindest die Saleicaner mit Geld. Wer sich hier ein Anwesen leisten konnte, hatte 
entweder einen uralten Namen und es im Besitz seit der Gründung der Stadt, oder zu viel Geld um es 
auszugeben. Auch viele Soldaten der nahe gelegenen Kaserne gaben hier bei den regelmäßigen 
Feierlichkeiten und 
Wettbewerben ihren Sold aus.\\
Jozah hielt den Blick gesenkt, saß auf den Weg und grübelte nach. Er war kein typischer Soldat 
der saleicanischen Armee. Die Männer und Frauen Saleicas waren weither bekannt als Draufgänger, 
Schläger, Menschen die lieber zur Waffe griffen als den Mund auf zumachen. Mut, nannten sie es, und 
Ehre, wenn man blindlings in einen Kampf stürzte, täglich unsinnige Mutproben ausfocht und grölend 
Lobeshymnen auf den König, das Land und Osyma sang. So war Jozah nicht. \textit{Ich bin ein 
Feigling.}\\
Musik hallte durch die Nacht, Gelächter und das Kichern von Frauen. Einige fleißige Händler, die 
selbst zu dieser späten Stunde ihre Waren anpriesen und damit bei dem trunkenem Adel auch gute 
Chancen hatten.\\
Widerwillig musste Jozah sich jedoch auch eingestehen, dass er seit seiner Rückkehr aus den 
Kolonien das Nachtleben der Stadt genoss. Die temperamentvolle Musik, erzeugt von Trommeln, Flöten 
und Violinen, übertönte die Stimmen der Sterbenden, die ihn viel zu oft verfolgten. Der Wein und 
das süße Essen verdrängte die Erinnerung an den Schmerz des Hungers und die Gewissheit, elend zu 
verhungern, während die Belagerer einen saftigen Braten speisten. Die heiteren Lieder ließen die 
verzweifelten Gebete verklingen.\\
\textit{Ob es Osyma wohl gefallen hat, uns beim sterben zuzusehen?}\\
Jozah hatte damals fast mit seinem Leben abgeschlossen. Er hatte den Dolch schon in seiner Hand um 
die größte Sünde zu begehen. Es war nicht die Angst davor, was wohl Osyma mit ihm machen würde, 
wenn er sich wirklich selbst das Leben nahm, die ihn abhielt. Nein, nicht mal um sich 
die Kehle aufzuschlitzen war er mutig genug. Jozah verzog das Gesicht und spuckte auf die Straße um 
den schalen Geschmack loszuwerden.\\
Aber dies war auch ein Wendepunkt in seinem Glauben gewesen. War es wirklich Tapferkeit, die die 
saleicanischen Soldaten so tödlich machte? War es wirklich Mut gewesen, der die Männer und Frauen 
dazu gebracht hat, immer und immer wieder einen Ausbruch aus der belagerten Feste zu versuchen? Nur 
um mit Pfeilen beschossen zu werden oder in eine Grubenfalle zu geraten? Mittlerweile vermutete 
Jozah eher, dass es ihre Art war, dem nahendem Hungertod zu entkommen. \textit{Egal. In den 
Geschichtsbüchern wird stehen, dass sie alles gegeben haben um Osymas Flamme zu verbreiten. Dass 
sie kämpften wie Löwen und starben wie Löwen. Sie brüllten den Namen des Allmächtigen im Tod. Ist 
besser als ihn zu röcheln, während man verhungert.}\\
Jozah verzog das Gesicht in der Erinnerung daran und schüttelte den Kopf, aber die Bilder ließen 
ihn nicht los. Sein damaliger Kommandant konnte sich verletzt noch zurück in die Feste schleppen, 
erlag vier Tage später jedoch seiner Verwundung. Und dann begann das Chaos. Die letzten Vorräte 
wurden noch innerhalb einer Stunde verzehrt. Die Soldaten kämpften um den letzten Krümel 
verschimmelten Brots. Aus purem Wahnsinn verunreinigten sie den Brunnen mit ihrem Urin.\\
Es gab nur einen Grund, wieso die Rebellen die hölzerne Feste nicht längst abgebrannt haben. Die 
Gegend war zu trocken und sie wollten keinen Flächenbrand riskieren. Doch in diesen Stunden des 
Chaos kam irgendeiner der Soldaten auf die Idee. Erst war es nur Geflüstert. Dann wurden die Rufe 
immer lauter. Die Männer brüllten animalisch, wie sie wohl dachten, dass Löwen brüllen würden. 
Osymas Name wurde geschrien, seine Flammen vergöttert. \textit{Das letzte Gebet.}, erinnerte Jozah 
sich an den Namen, mit dem dieses Ereignis in die Geschichtsbücher eingetragen wurde. \textit{Das 
letzte Betteln trifft es besser. Das letzte Flehen, dass ein stolzer Gott sich nicht abwenden 
würde, auch wenn seine Streiter verloren haben.}\\
Nicht jeder war verzweifelt genug, diesem Aufschrei zu folgen. Zu widersprechen wagte jedoch 
niemand, zu schnell drang saleicanischer Stahl in die, die es versuchten. Es war Jozahs Idee 
gewesen, sich in die Gruft zurück zu ziehen und die Türen zu verrammeln. Elor, damals schon ein 
Freund, der still an seiner Seite stand, war dabei. Und noch zwanzig weitere Männer und Frauen. 
Jozah hatte Mishka noch tobend mit geschliffen. Stur wie er war, wollte der Adelige sterben wie es 
für einen Löwen würdig war. Aber auch sein Trommeln gegen die Wände hatte schließlich nachgelassen, 
als die Hitze selbst in der Gruft spürbar wurde und die Schreie erklangen. Die Dunkelheit hatte 
seltsamen Frieden gespendet. Erst als die Stille begann und auch nicht mehr aufhörte, hatten die 
verbliebenen Saleicaner die Steine wieder beiseite geschaufelt und sind aus der Gruft gestiegen. 
Völlig abgemagert, verdreckt und vom Chaos gezeichnet, aber lebend waren sie über das Feld aus 
Asche gestolpert und schon bald einer Truppe begegnet, die ihnen hatte zur Unterstützung kommen 
wollen.\\
Die ganze Geschichte endete damit, dass die Überlebenden Jozah für seine Idee sich zu verstecken 
feierten und vor den Vorgesetzten in den Garnisonen nur Lob von sich gaben. Es wurde so 
dargestellt, dass er das Feuer befohlen hatte um die Truppe der Rebellen zu vernichten und ihr Land 
für die Torheit zu bestrafen. Dem Chaos, dem Wahnsinn und den saleicanischen Feuertoten wurden 
keine weitere Beachtung geschenkt. Ehe er sich versah, war er zum Kommandanten befördert und hatte 
eine Truppe zugeteilt bekommen, die zum größten Teil aus den Zwanzig Überlebenden bestand.\\
\textit{Und nun General? Diesmal war's aber zu einfach.}, dachte er bitter.\\
Ein großgewachsener Mann kam auf die Straße getaumelt, am Arm eine hübsche Frau. Mishkas blondes 
Haar stand zerzaust in alle Richtungen, seine Kleidung war verrutscht und zerknittert. Ein seeliges 
Grinsen im Gesicht und der Geruch nach Alkohol begleitete ihn. Er hatte sich nicht verändert. 
Mishka war immer noch der verwöhnte vierte Sohn eines Adeligen, der mit dem Geld seiner Familie um 
sich warf. Soweit Jozah auf dem neusten Stand war, hatte aber seine älteste Schwester nun den Titel 
des Familienoberhaupts übernommen und schon einige ernste Gespräche mit Mishka hinter sich. Viel 
gebracht hat es nicht. Trotzdem wusste Jozah, dass das alles nur ein Versuch war, das Chaos 
zu vergessen. Mishka hatte es nicht ausgesprochen, aber er verdankte Jozah sein Leben und das war 
ihm deutlich bewusst.\\
``Heeey'', rief der Bondschopf aus: ``Hast gar nich erzählt, dass du auch noch raus wolltest! Wo 
warst'n den ganzen Nachmittag?''\\
``Jozah'', sprach auch seine Begleitung und löste sich aus der Berührung, um ihm um den Hals zu 
fallen. ``Und ich dachte schon, ich finde dich heute nicht mehr.''\\
Er schloss einen Moment die Augen und genoss die Berührung der Frau, während er den Duft ihres 
Parfüms einatmete. ``Ilia'', murmelte er: ``Schön dich zu sehen.''\\
Ilia Ma'Sah war eine dieser Frauen, die alle Aufmerksamkeit bekam, wenn sie einen Raum betrat. Die 
Männer folgten ihr mit Blicken und verstummten, selbst wenn sie gerade dabei waren, lautstark 
Wetteinsätze zu verhandeln. Sie war der Inbegriff einer saleicanischen Schönheit, fand Jozah. 
Blondes Haar, ansehnliche Rundungen und ein Temperament, vor dem einige Männer lieber Abstand 
hielten. Ihr Name war so alt wie Brom-Dalar, gehörte ihre Familie doch zu den Gründern der Stadt. 
Einige ihrer Ahnen haben wohl auch in die Königsfamilie eingeheiratet. Ihr Vater war ein General, 
der gleich mehrere Geschichtsbücher füllte und einst im Rat König Kareens saß, ehe er in Ruhestand 
ging. Er war freiwillig abgetreten, noch ehe die Priester Prinz Semric krönten. Nebenbei hatte er 
sein Vermögen vermehrt, in dem er in den Handel einstieg. Und Ilia war die einzige Erbin dieses 
Vermögens, dieses Namens. Und trotz ihrem Alter von 25 Sommer noch nicht verheiratet. Sie war zu 
frech, um einen Mann bisher lange genug zu halten. Und hatte wohl bisher auch noch nicht die 
Absicht gehabt, das Versprechen zu geben. Sie verbrachte ihre Nächte tanzend in der Stadt, die Tage 
bei Jagden, Fechten oder Plaudereien. Anfangs sah Jozah nur das und war entsprechend überrascht, 
als er entdeckte, dass Ilia noch vieles mehr war. Sie sprach Kasirisch, sowie zwei Sprachen aus den 
Kolonien. Las gerne in besagten Sprachen Bücher und Berichte und hatte eine Vorliebe für Schach.\\
Und dazu kam das Argument, dass sie die Tochter seines ehemaligen Mentors war. Ohne Vito Ma'Sah 
hätte Jozah die Zeit als Rekrut wohl nicht überstanden. Vor allem die anderen - teilweise adeligen 
- Rekruten hatten ihm das Leben schwer gemacht und auch einige der Ausbilder hatten ihn loswerden 
wollen. Einer der harmloseren Versuche war noch gewesen, ihn in die Kaserne irgendeines kleinen 
Kaffs an der Westküste zu versetzen. Natürlich wurden Soldaten in ganz Saleica geschätzt, aber in 
der Hauptstadt zählt der Familienname viel zu oft mehr als die Fähigkeiten.\\
``Wie geht es deinem Vater?'', erkundigte er sich höflich.\\
``Ach... wenn es nach ihm geht, soll ich in unser Strandhaus ziehen und sticken oder solche 
Dinge.''\\
``Er sorgt sich nur um dich'', erwiderte Jozah grinsend und küsste sie auf die Wange.\\
Ilia sah ihn einen Moment eindringlich an, wandte sich dann lachend um und deutete auf einen der 
Verkaufsstände. ``Ich will einen Gewürzwein. Darf ich dich einladen, Herr Kommandant?''\\
``General'', verbesserte Jozah.\\
Sein Freund und Ilia sahen ihn überrascht und fragend an. ``Offizier Lerin hat mich befördert.''\\
``War die Ernennung bereits? Und du hast mich nicht eingeladen?'', fragte Ilia entrüstet.\\
Jozah beeilte sich zu sagen: ``Nein, noch nicht.''\\
``Noch besser. Wenn mein Vater davon hört, wird er darauf bestehen, die Ernennung für dich 
abzuhalten! Oh, ich brauche noch ein neues Kleid. Und du eine annehmbarere Uniform!''\\
``Wen interessieren Stoffe, dass müssen wir feiern!'', entschied Mishka und hob seinen leeren 
Becher, legte den Kopf in den Nacken und versuchte noch einen letzten Tropfen mit der Zunge 
aufzufangen.\\
