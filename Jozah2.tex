\chapter{Schnelle Versprechen}

Jozah musste sich nun beeilen, um zu seiner Verabredung nicht zu spät zu kommen. Er eilte über den 
Palasthof und die Stufen hinab. Der Palast befand sich auf einem erhöhten Hügel, in welchem 
rundherum Stufenweise die Erde abgetragen und Gärten angelegt waren. An sich ähnelte der Standort 
also einer immensen Treppe auf deren Spitze der königliche Palast weilte. Während die Unterste der 
drei Stufen für das gesamte Volk zugänglich waren, bewachten Palastwachen die Höheren um die 
angesehenen Leute einzulassen. Die Zweite Ebene durfte die Mittelschicht betreten. Die dritte Ebene, 
unterhalb des Palastes, war nur für den hohen Adel, Priester und hochrangige Militärmitglieder 
vorbestimmt. \\
Jozah steuerte die zweite Ebene an. Er eilte über den Kiesweg und entdeckte eine Person am 
steinernen Geländer, von welcher man auf die dritte Ebene hinab sah. Die Sonne ging gerade unter und 
färbte den Himmel rötlich. Sie trug einen Sonnenschirm bei sich, wie stets. Selbst am Abend trug sie 
den blass rosafarbenen Schirm aufgespannt und hielt ihn über ihre goldenen Locken. \\
\textit{Vermutlich nicht mehr ein hübsches Ding für sie}, dachte Jozah. \textit{Genau wie dieser 
Säbel für mich. Jaja... man muss mit der Mode gehen.}\\
Die schöne Ilia Ma'Sah. Ihr Name sagte vermutlich jedem Jugensellen mit Rang etwas. Ihr Haar reichte 
bis zu ihrer Hüfte, wenn sie es offen trug. Augen, so blau wie das Meer an einem warmen Sommertag. 
Einladende Hüften, eine schmale Taille. Aber das lag vermutlich an ihrem eng geschnürtem Korsett. 
Die Locken wirkten etwas zu perfekt. Jozah hätte gerne gefragt, wie lange ihre Zofe wohl an diesen 
Locken gesessen hatte. Aber schöne reiche Frauen hörten so etwas vermutlich nicht gerne.\\
Seine Schritte knirschten auf dem Kies und die junge Frau wandte sich um. Die zarten, blassen Hände 
hielten den Griff des Sonnenschirms fest. Sie schlug die Augen auf und blickte ihn mit einem leisen 
Lächeln an. Jozah versuchte es zu erwidern. Der Tag hatte bisher solch unangenehmen Überraschungen 
enthalten, dass er auf die Verabredung eigentlich hätte verzichten können. Und das, obwohl er am 
Morgen noch die Stunden bis zu diesem Moment gezählt hatte. \\
Immerhin war Ilia eine fantastische Frau, schön wie gemalt, zart und sanft wie ein Schmetterling. 
Schon allein, wie sie ihn anblickte und ruhig darauf wartete, dass er näher kam. Ihre rechte Hand 
löste sich von dem Schirm und hielt sie ihm mit einer eleganten Bewegung entgegen. Jozah bückte sich 
und hauchte einen Kuss auf ihren Handrücken.\\
 ``Schade, dass Sie den Sonnenuntergang verpasst haben, Kommandant", sagte sie mit einer Stimme, 
wie zwitschernde Vögel die den Morgen begrüßten.\\
 ``Verzeiht. Ich hatte einen dringenden Termin."\\
Ihre Gesten hatten etwas faszinierendes. Aber nach all dem Prunk im Schloss, betrachtete Jozah sie 
plötzlich anders. Ilia erinnerte ihn an den König, wie er bei Paraden ritt und Reden hielt, aber 
dann in seinem eigenen Ratssaal nicht mehr als ein gelangweilter Junge war. In ihrem Aussehen, den 
Locken, der Kleidung, der Röte ihrer Wangen, erkannte Jozah deutlich, wie viel Mühe sie sich 
gegeben hat, um ihn zu beeindrucken. Oder einzuschüchtern? Aber das alles war falsch. Gestellt. 
Jede Geste ein geübtes Schauspiel. An sich hatte er das von Anfang an gewusst. Aber da war der Reiz 
noch gewesen, Ilias Geheimnis zu lüften. Ihr wahres Ich zu ergründen. Dieser Reiz war keineswegs 
weg, aber Jozah war müde. Er hatte keine Kraft mehr, noch weiter sich selbst zu verstellen. Er war 
ein Soldat. Und unter Soldaten gehörte er. \\
Ilia hatte sich noch nicht von der Stelle gerührt. Ihr Lächeln war eingefroren. Ein Kompliment, sie 
wartete auf ein Kompliment.\\
Jozah neigte leicht den Kopf.  ``Ihr strahlt schöner als die Sonne je könnte, Ilia. Warum sollte 
ich den Sonnenuntergang betrachten, wenn ich Euch ansehen könnte?"\\
Ihr Lachen war sanft und leise. Schon eher ein Kichern. \\
 ``Sie kamen aus dem Palast?", erkundigte sie sich.\\
 ``Der König bat mich in den Ratsaal", antwortete Jozah vorsichtig. Er war sich nie sicher, wie 
viel sie eigentlich wissen wollte. Ilia fragte viel nach seinem Alltag, seinen Vorlieben und 
Meinungen. Jede Frage, jeder Blick, wirkte wirklich interessiert. Aber kaum begann er zu sprechen, 
glitt ihr Blick zu etwas anderem und aus seiner Antwort entstand nie eine tiefere Unterhaltung.\\
\textit{Vielleicht traut sie sich nicht}, überlegte Jozah. \\
 ``Beim König?", wiederholte sie und sah ihn aus erstaunten blauen Augen an:  ``Welch Ehre!"\\
Und so ehrenvoll Ilias es nach ihrem Worten fand, anscheinend weckte es nicht genug Interesse um 
nachzufragen. Jozah besann sich und erwiderte ihr bezauberndes Lächeln. Vermutlich hatten ihre 
Reaktionen weniger mit Desinteresse als mit Unwissenheit zu tun. Oder aber es war Teil der 
Erziehung einer jungen Dame, dass sie den Mann reden ließ und sich selbst zurückhielt. Davon wusste 
Jozah nichts. \\
Sie schlenderten nebeneinander eine Weile durch den Garten, während die Sonne hinter dem Horizont 
verschwand. Laternen wurden am Wegesrand entzündet und der Garten leerte sich. Währenddessen 
plauderten sie über dies und das. Ilia lobte seinen Säbel und bat, ihn einmal schwingen zu dürfen. 
Er lehnte höflich ab und erwiderte, dass eine Waffe nicht in die zarten Hände einer Dame gehörten.
Natürlich gab es Soldatinnen und Kriegerinnen, aber so eine war Ilia nicht. Soldatinnen kicherten 
nicht, sie lachten laut wie ihre Kameraden. Sie zierten sich nicht vor Schmutz, sie prügelten sich 
darin. Sie betrachten nicht ihren Gegenüber um seine Mode zu beurteilen, sondern die eigenen 
Chancen in einem Kampf zu gewinnen. Zumindest die Soldatinnen, denen er bisher begegnet war. Ebenso 
wie nicht jeder Mann für den Kampf geschaffen war, war es auch nicht jede Frau. Und es gab auch 
genug männliche Adelige, die Kichern konnten und sich mit Mode auskannten. \\
Genaugenomen lehnte Jozah ihren Wunsch nur aus dem Grund ab, weil er fürchtete, ihr Vater würde 
etwas davon mitbekommen. Sie waren immerhin an einem öffentlichen Ort und Gerüchte verbreiteten 
sich schnell. Ilias Vater wusste von den Verabredungen und nahm sie an, aber er würde nicht hören 
wollen, dass seine einzige Tochter mit einem Säbel in die Luft stocherte.\\
Sie hatten sich auf einer Bank niedergelassen und Ilias Hand lag in seiner. Er überlegte die ganze 
Zeit schon, wie er ihr von seinem baldigen Aufbruch berichten sollte. Jozahs Blick glitt zu den 
ineinander verschlungenen Händen.  Plötzlich wurde ihm übel. Die Hand gehörte in seinen Gedanken 
nicht mehr der schönen Ilia, sondern einem verkrüppelten, vor Schmerzen schreienden Kameraden. 
Stunden hatte er diese Hand gehalten. Die Schreie wurden zu Schluchzen, zu Wimmern und schließlich 
endgültiger Stille. Das Blut rauschte in seinen Ohren und er spürte seinen hämmernden Herzschlag. 
Abrupt ließ Jozah ihre Hand los und stammelte.  ``Meine Liebe. Ihr wisst, ich stehe im Dienste des 
Königs und... ich weiß nicht, wie ich Euch das sagen soll..."\\
Ilias Augen weiteten sich wieder. Ein Ausdruck des freudigen Erstaunens legte sich auf ihr Gesicht. 
Nicht gespielt, sondern ehrlich, wie Jozah vermutete. Er unterdrückte seine plötzliche Panikattacke 
und wollte es hinter sich bringen.\\
 ``Oh Jozah", kam sie ihm zuvor und lachte. Ein richtiges Lachen!  ``Haben Sie schon mit meinem 
Vater gesprochen? Er hatte nichts von Ihren Plänen erwähnt. Bestimmt wollte er mir die Überraschung 
nicht verderben. Oh, ja! Ja ich will, Jozah!"\\
 ``Öh... was?", stotterte er verwirrt.\\
 ``Ich nehme Ihren Antrag an!"\\
Ilia flog ihm um den Hals. Er spürte ihre Haut an seine, ihr Haar kitzelte. In dieser Position sah 
sie wenigstens nicht seine Mimik, die Überraschung und Widerwillen zeigte. Jozah hatte Ilias Vater 
gegenüber gewisse Andeutungen gemacht, aber da war er bestimmt nicht der Einzige. Und zu einer 
richtigen Frage um die Hand seiner Tochter war es bisher aus rein zeitlichen Gründen noch überhaupt 
nicht gekommen. Allgemein passte das gerade überhaupt nicht. Der Befehl des Königs war eindeutig, 
er würde bald aufbrechen und Ilia könnte niemals mit. Sie würde gar nicht mit wollen. Die Reise 
würde Tage dauern, der Aufenthalt vermutlich Wochen, wenn nicht Monate. Er war Soldat und dem König 
verpflichtet. Eine Bitte, jemand anderem diesen Auftrag zu geben, wäre aussichtslos. Das 
schlussfolgerte Jozah aus dem Gespräch mit dem Offizier. \\
Und außerdem... er wollte Ilia nicht zu einer der vielen Soldatenfrauen machen, die jeden Tag in 
der Tür standen und darauf warteten, dass ihr Mann heimkehrte... oder die Botschaft seines Todes 
eintraf. Ja, er war kein einfacher Soldat mehr, er hatte sich in den Kolonien bewiesen und war 
sogar befördert worden. Seinen Namen nach waren seine Ahnen einst Angehörige des Adels, das änderte 
aber nichts daran, dass er weder Geld noch Ansehen besaß. Seine Eltern hockten vermutlich immer 
noch in ihrer kleinen Schusterei in einem Dorf, dessen Name nicht einmal auf Karten stand.\\
Ihm war das irgendwie immer klar gewesen. Aber Hoffnung und Sehnsucht konnten böse Spiele spielen. 
Wer weiß, vielleicht hätte er ihren Vater irgendwann gefragt und zu überzeugen versucht, warum er 
ein angemessener Ehemann wäre. Aber Ilia und der König kamen ihm gerade zuvor. Sie war 23 Jahre alt 
und hatte viele Verehrer. Jozah hätte nicht einmal damit gerechnet, dass er ihr Favorit war!\\
 ``Ilia", sagte er vorsichtig und löste sich von ihr:  ``Ich habe Euren Vater noch nicht gefragt."\\
Das blonde Mädchen blickte ihn einen Moment lang aus ihren himmelblauen Augen an. Dann kicherte sie 
wieder.  ``Sie sind verwegen, Kommandant. Oder trauen Sie sich nicht?"\\
 ``Es gab bisher keinen richtigen Moment. Aber ich wollte Euch etwas anderes mitteilen..."\\
 ``Aber ich muss gestehen, ich hatte auch nicht damit gerechnet, dass Sie mich vor dem ersten 
Herbstwind fragen würden", erklärte sie lächelnd:  ``Machen Sie sich keine Sorgen. Gehen Sie zu 
meinem Vater und fragen. Er hat geschworen, dass ich auf mein Herz hören darf, solange es ein Mann 
mit Namen ist. Und den haben Sie, Jozah. Mein Vater schätzt das Militär. Manchmal sogar mehr als 
den alten Adel. Er sagt immer, das Militär ist es, welches Saleica wachsen ließ, der Adel hat nur 
zugeschaut."\\
 ``Ilia... gerne werde ich Euren Vater fragen. Aber ich muss in den Norden. Auf Geheiß des Königs. 
Ich werde einige Monate nicht in der Hauptstadt sein."\\
 ``Aber noch ist kein Krieg, oder?", rief sie laut und blickte ängstlich zu ihm auf.\\
 ``Nein, meine Liebe. Das wird noch dauern. Nicht vor dem Frühling."\\
Jetzt lächelte Ilia wissend und ergriff wieder seine Hand.  ``Dann verstehe ich, warum Ihr mir 
genau jetzt die Frage gestellt habt. Sorgen Sie sich nicht, Jozah. Meine Gedanken werden nur bei 
Ihnen sein! Darum haben Sie mich jetzt gefragt, richtig? Weil Sie sich mit mir verloben wollten, 
damit kein anderer kommt." Sie strahlte ihn an und ihre Finger strichen sanft über seinen Oberarm. 
\\ ``Kümmern Sie sich um Ihre Vorbereitungen. Ich spreche mit meinem Vater. Er versteht, dass ein 
Soldat viel zu tun hat! Ich will Sie jetzt auch nicht weiter aufhalten. Würden Sie mich noch nach 
hause geleiten? Es ist spät und ich möchte ungern alleine durch die Stadt gehen."\\
 ``Ja... gewiss!", sagte Jozah und versuchte aus seiner Benommenheit zu gelangen. \\
\textit{So schnell ist man also verlobt}, dachte er überrumpelt.\\

Der Halbmond beschien die Straßen der Hauptstadt, hüllte sie in ein gespenstisches Licht. Aber die 
Nacht war keineswegs still. Besonders nicht in der Nähe der Kaserne. Hin und wieder hörte man das 
lockende Kichern einer Hure, auf der Suche nach Kunden. Die Gaststätte war selbst zu hören, als 
noch etliche Straßen zwischen ihr und Jozah lagen. Grölen, Klappern von Krügen und Poltern von 
Stühlen. Gesang und Musik. Ein Betrunkener überquerte vor Jozah stolpernd und lallend die Straße. 
Er hielt den Blick gesenkt, saß auf den Weg und grübelte nach. Jozah war kein typischer Soldat der 
saleicanischen Armee. Männer und Frauen Saleicas waren weither bekannt als Draufgänger, Schläger, 
Menschen die lieber zur Waffe griffen als den Mund auf zumachen. Mut, nannten sie es, und Ehre, 
wenn man blindlings in einen Kampf stürzte, täglich unsinnige Mutproben ausfocht und grölend 
Lobeshymnen auf den König, das Land und Osyma sang. So war Jozah nicht. Weder Streben nach Ehre 
noch Kampfeswut waren es, weshalb er jetzt hier stand. Er ging zur Armee, weil er keinen anderen 
Ausweg sah. Das hatte sich bis heute nicht geändert. Nichts anderes konnte er sich vorstellen. Was 
wäre ihm als Sohn eines Handwerkers für ein Leben beschieden gewesen? Entweder angestellt in der 
Werkstatt, die ein älterer Bruder erben würde, oder bei irgendjemand anderem, der vermutlich noch 
weniger für ihn übrig haben würde. \\
\textit{Priester}, dachte er und verzog angewidert das Gesicht. \\
Als Rekrut hatte er es schwer gehabt. Die Anderen sahen es nicht gerne, dass ein so junger Knabe 
unter ihnen war. Er wurde geprügelt, verflucht und ausgelacht. Aber er war standhaft geblieben. Die 
ein oder andere Mutprobe hatte Jozah damals mitgemacht, aber eher um akzeptiert zu werden anstatt 
aus ehrlichem Interesse. Standhaftigkeit, dass war vielleicht seine große Fähigkeit. Er war weder 
besonders selbstbewusst, noch mutig. \\
Die Belagerung kam ihm in den Sinn. Der Offizier hatte seine Taten so gelobt, aber Jozah glaubte 
nicht daran, dass er nicht die Wahrheit wusste. Wochenlang waren sie in der klapprigen 
burgähnlichen Stellung verharrt. Der einzige Grund, warum die Rebellen das hölzerne Gestell nicht 
abgefakelt haben war, dass sie keinen Flächenbrad riskieren wollten. Das war in den trockenen, 
heißem Land nicht unüblich und ein solcher kaum zu löschen. Sie hätten sich damit nur ins eigene 
Fleisch geschnitten. Nein, sie hatten die Taktik aushungern versucht. Wäre ihnen auch fast 
gelungen, wenn die Saleicaner nicht so stolz wären. Immer wieder schickte der Kommandant seine 
Männer vor das Tor, ließ Pfeile sinnlos durch die Luft surren, nur um dann wieder zurück hinter den 
Wall zu huschen. Bei einem dieser sinnlosen Ausfälle, die wohl mehr Mut darstellen sollten, als 
überhaupt vorhanden war, wurde der Kommandant selbst erwischt. Von einem Pfeil im Rücken wollte er 
sich erst nicht aufhalten lassen, bis die Wunde sich entzündete. Während die meisten der Soldaten 
einen endgültigen Ausfall planten, versuchte Jozah die Wunde seines Kommandanten so gut wie möglich 
zu versorgen. Er wusste natürlich was im Gange war. Es schien so, als hätten etliche der Rebellen 
sich zurückgezogen und nur noch ein Drittel der ursprünglichen Belagerer wäre anwesend. Oh, und wie 
Jozah sich noch an die Worte des Kommandanten erinnerte. Er befahl ihm, den Soldaten auszurichten, 
sie sollen sie alle fertig machen. Jeden einzelnen. Tja, somit hatte er das Kommando über die 
Soldaten erhalten, nachdem der Verantwortliche in eine friedliche Ohnmacht gefallen war. Es wäre 
nicht schwer gewesen diesen Befehl in die Tat umzusetzen. Ganz im Gegenteil, die Männer waren 
praktisch schon aus dem Thor heraus, als Jozah in den Hof trat und sie alle zurück pfiff. Ein 
knapper Bericht über den Zustand des Anführers folgte und Jozah hätte die Männer ziehen lassen. Wenn 
Elor, ein stiller Kamerad der dem Gespräch zeuge gewesen war, nicht erwähnte, dass Jozah nun die 
Entscheidungsgewalt übertragen bekommen hatte. Und augenblicklich fühlte Jozah sich, als stünde er 
mitten in einem Rudel junger Hunde, behangen mit Würsten und Schinken. Es war eine Sache, zuzusehen 
wie eine Gruppe Männer, getrieben von Blindheit und Siegeswillen, los stürmten. Er hatte ja selbst 
gesehen, dass die Belagerer sich teilweise zurückgezogen hatten und wäre vermutlich selbst mit 
gestürmt. Schon allein aus dem Grund, um endlich irgendetwas zu tun zu haben. Ja, er verstand die 
Soldaten. Wochenlang eingesperrt auf engsten Raum, die Vorräte erreichten ihr Ende, der Kommandant 
schwer verwunden und eine scheinbar geringe Anzahl an Feinden. Aber es war etwas anderes, besagte 
Männer in den Tod zu schicken. Selbst beim Schachspiel brachte er es kaum über sich, seine Bauern zu 
opfern um stärkeren Spielfiguren Angriffschancen zu bieten. Das empfand er als ein sicheres Zeichen, 
dass er sich einfach nicht dafür eignete, Verantwortung zu übernehmen. Vielleicht ein weiterer 
Grund, warum er als Junge zum Militär gegangen war. Er hatte jemanden gesucht, der ihm 
Entscheidungen abnahm, die Verantwortung trug, dem er sein Leben anvertrauen konnte. Und jetzt war 
er plötzlich in dieser Lage. Vermutlich hätte es ihm keiner nachgetragen, wenn er dem Plan des 
Kommandanten und der Männer nachgegangen wäre. Immerhin erwartete man das von Saleicanern. Aber 
stattdessen hatte er sich gedrückt, die Männer vertröstet. Er weiß noch, nur knappe Worte waren über 
seine Lippen gekommen. \\  ``Verstärkt die Wachen, die anderen ruhen sich heute Nacht aus. Morgen 
wird es unserem Kommandant wieder besser gehen!"\\
Aus einer Nacht wurden vier. Und mit jeder Nacht wurde das Murren und Fluchen der Männer lauter und 
grober. Jozah rechnete fest damit, dass bald einer sich erheben, ihm am Kragen packen und 
ordentlich durch schütteln würde. Er war sich heute noch sicher, es wäre unweigerlich dazu 
gekommen, wenn nicht vorher die Verstärkung eingetroffen wäre. Erlo war es, der ihm von der 
Sichtung berichtete. Und dann ging alles schnell. Jozah rief zu den Waffen, die Tore wurden geöffnet 
und sie fielen den Belagerern, die sich der Verstärkung zu gewandt hatten, in den Rücken. 
Letzenendes hatte sich herausgestellt, dass die Rebellen zahlreicher waren denn je. Sie kannten 
mittlerweile den Stolz der Eroberer, hatten darauf gewartet, dass sie auf die anscheinend geringe 
Anzahl reagieren würden. Hatten sie durch Jozahs Zögern jedoch nicht und völlig überrumpelt wurden 
sie nieder gemacht. Danach wurde er für seine Umsicht, seine Taktiken und Tarnmanöver in den Himmel 
gelobt. Keiner der Männer schien sich noch daran zu erinnern, dass sie beinahe gemeutert hätten. Der 
ursprüngliche Kommandant kam wieder zu sich, übergab Jozah einen Teil seiner Männer und beförderte 
ihm zum Kommandanten. Während den drei weiteren Jahren in den Kolonien hatte Jozah, seiner Meinung 
nach, ein gesundes Gleichgewicht zwischen Vorsicht und Dummheit erlangt. Wäre er noch einmal in der 
Situation, würde er vermutlich mehr riskieren. Wer weiß das jetzt schon. Was ihn noch mehr erstaunte 
war jedoch, dass die 50 Mann immer noch hinter ihm standen. Wenn man nur lange genug Seite an Seite 
auf dem Schlachtfeld stand, musste man einander wohl oder übel etwas bedeuten, vermutete Jozah. 
Erlo gehörte mittlerweile zu seinen engsten Vertrauen und Jozah entschied, dass er in ihm schon mal 
einen Reisegefährten für seinen momentanen Auftrag hatte. \\
Er hatte mittlerweile die Kaserne erreicht, steuerte aber nicht sofort auf die Unterkünfte zu. 
Stattdessen trat er in ein flaches, schmales Gebäude. Der Geruch nach Stroh und Pferd stieg ihm 
entgegen. Es war keine wirklich komfortable Unterkunft für die Tiere, aber die gab es für die 
Soldaten auch nicht. Man verließ sich anscheinend darauf, dass die Pferde genug Auslauf bekamen und 
sie somit ruhig in einem engen Stall stehen konnten. Jozah spähte den nur vom Mondschein 
beleuchteten Gang entlang. Irgendwo in der Dunkelheit, in der Nähe der Sattelkammer, schliefen 
vermutlich die Stallburschen. Möglichst leise um weder Tier noch Mensch zu erschrecken, trat er den 
Gang entlang. Es war nicht schwer, das bestimmte Pferd zu finden. Das Fell des Schimmels leuchtete 
wie der Mond selbst. \\
Das Tier hatte, wie die meisten anderen Pferde im Stall, bereits seine Schritte vernommen und 
blickte ihm abwartend mit gespitzten Ohren entgegen. Der Soldat trat an das Pferd heran und legte 
seine Hand auf das weiche Fell. Die Haltung des Tieres blieb weiterhin wartend. Jozah wusste genau, 
was der Wallach wollte. Stets erhielt er bei jeder Begrüßung ein Apfelstück, aber das hatte er 
jetzt nicht dabei. \\
 ``Morgen", vertröstete er den Schimmel und betrachtete die dunklen Augen des Tieres. \\
Jozah ritt das Pferd, seit er vor 3 Monaten mit seinen Leuten wieder in der Hauptstadt angekommen 
war. Es hatte weder einen offiziellen Namen noch einen zum Rufen. Jozah hatte bisher nie darüber 
nachgedacht, immerhin hat das Tier der Kaserne gehört und war ihm lediglich als Kommandanten 
zugeteilt worden. Aber Lerim hatte es ihm geschenkt. Jetzt gehörte der Schimmel Jozah. Bei den 
Gedanken musste er immer noch den Kopf schüttelnd. Er war kein Mann, der viel besaß. War er nie 
gewesen. Alles in allem, konnte er jetzt nur zwei Dinge sein eigen nennen. Darunter das Pferd. 
Selbst seine Uniformen waren von der Kaserne gestellt. Seinen Sold schickte er größtenteils an 
seine mittlerweile alt gewordenen Eltern, die er seid fünfzehn Jahren nicht mehr gesehen hatten. 
Sie konnten weder Lesen noch schreiben und somit herrschte auch keinen Kontakt. Immerhin tröstete 
er sich damit, dass sollten seine Eltern bereits verstorben sein, sein Bruder das Geld ebenfalls 
brauchen konnte. Es war nicht viel, aber Jozah bekam Unterkunft, Versorgung und alles nötige was er 
brauchte vom Militär. \\
 ``Schlaf gut", raunte er dem Schimmel zu, gab ihm noch einen kameradschaftlichen Klaps auf den 
Hals und verließ den Stall. Während seinem Besuch war es still geblieben. Manchmal wirkte die Stille 
auf ihm zu ungewöhnlich. Manchmal vergaß er, dass diese Tiere ebenso Soldaten waren wie er. Ihr 
Leben lang ausgebildet und auch schon in Schlachten aktiv geworden. Nur gehorsame und ruhige Pferde 
konnten in Saleica so weit kommen. Die Reiterei musste sich auf sie verlassen können. Mit diesen 
Gedanken noch im Kopf, machte Jozah sich auf den Weg, das Zweite zu Besuchen, welches nur ihm 
gehörte. \\

Seine Kammer war wie alle anderen in der Kaserne geschnitten. Klein, quadratisch und nur ein 
winziges Fenster ließ frische Luft herein. Rechts an der Wand befand sich ein schmales Bett mit 
einer strohgefütterten Matratze. Ein Tisch, zwei Stühle. Eine Truhe für Kleidung und persönliche 
Gegenstände. Die Wände waren grau, geziert von dem ein oder anderen Schmutzfleck. Die meisten 
Personen mit höheren militärischen Rängen bevorzugten, in ihren Wohnungen oder Häusern außerhalb 
der Kaserne zu wohnen. Es gab noch das ein oder andere komfortablere Zimmer für die Ausbilder, der 
Rest sah so aus wie Jozahs. Er hätte auf eine Wohnung sparen können. Das wäre vielleicht ein Jahr 
gewesen und schon hätte er sich eine gemütliche, nicht all zu exklusive leisten können. Aber er war 
immerhin erst seid wenigen Monaten hier, konnte sich nicht von dem Ort trennen, den er immer als 
Heimat betrachtet hatte. Er hatte sich als Kommandant lediglich den Luxus gegönnt, das Zimmer 
allein zu bewohnen. Die ranglosen Soldaten teilten sich den ähnlichen Raum zu zweit. \\
Wie er das alles lösen sollte, wenn er Ilia heiratete, wusste er nicht. Sie konnte nicht mit in die 
Kaserne und würde es auch nicht wollen. Sie war ihr großes Familienhaus gewohnt und würde auch 
nicht mit einer gemütlichen Wohnung zufrieden sein. Nun, ihre Mitgifte würde vermutlich für ein 
kleines Haus reichen, aber er war sich auch da nicht sicher, ob sie glücklich wäre. Ob sie sich 
darüber überhaupt schon Gedanken gemacht hatte?\\
Jozah seufzte. Er mochte sie. Mochte sie wirklich. Sie war so anders als die Menschen, die er 
gewohnt war. Er kannte die Kalten, die Zermürbten, die Grausamen, die Hassenden, die 
Hoffnungslosen, die ernsten Leute. Klar, seine Kameraden konnten auch zu herben Scherzen fähig 
sein, sangen Lieder oder erzählten Geschichten am Lagerfeuer. Aber ein jeder war gezeichnet von dem, 
was er erlebt hatte. Keinen Menschen konnte man mit einem Wort beschreiben, aber Ilia wirkte so 
gegensätzlich. 
Er mochte sie, mochte sie wirklich.\\
Jozah kniete sich vor die Truhe und öffnete das Schloss. Mit einem leisen Klick sprang es förmlich 
in seine Hand und der schwere Holzdeckel ließ sich anheben. Vorsichtig, fast ehrfurchtsvoll, nahm 
er den in ein schmutziges Tuch gewickeltes Laken heraus. Er legte den Gegenstand auf den Tisch und 
setzte sich auf den Hocker davor. Lange betrachtete Jozah es, ohne das Tuch beiseite zu schlagen. 
Es war wie ein Geheimnis. Sein eigenes, daher brauchte er es nicht zu lüften. Dies war der zweite 
Gegenstand, der ihm allein gehörte. Und diesen hatte er nicht geschenkt bekommen, sondern sich 
erkämpft und verdient. Das hob seinen Wert gleich an. Jozah seufzte und schlug das Tuch zurück. Es 
war ein schlichtes Schwert, was vor ihm lag. Ein Bastardschwert, wie man sie so nett nannte. Jozah 
brauchte es nicht zu schwingen um zu wissen, das es im Gleichgewicht war und perfekt in seine Hand 
passte. Ein Schwerte ohne Zierde, aber einem metallenen Schimmern welches einem Versprechen nahe 
kam. Ein Schwur auf seine eigene Schärfe und Tödlichkeit. \\
Sein ursprüngliches Schwert, welches er als einfacher Soldat getragen hatte und dem Land gehört 
hatte, war zerbrochen, als er damit die Axt eines Rebellen aufhalten wollte, die einem seiner 
Kameraden bedrohte. Am Ende der Schlacht hatte er in dem Feld voller Toten nach einer brauchbaren 
Ersatzwaffe gesucht und ein halbes Dutzend mit zum Schmied genommen. Keine von ihnen war sonderlich 
gepflegt und keine weckte den Anschein, dass sie nicht ebenfalls bei der nächst besten Gelegenheit 
brechen würde. Der Kamerad, den Jozah gerettet hatte, aber dafür in der nächsten Schlacht starb, 
bestand darauf für ihn den Schied, welcher nicht dem Militär angehörte, zu bezahlen. So war sein 
jetziges Schwert entstanden. Das Metall von den gesammelten Klingen wurde eingeschmolzen, möglichst 
gehärtet und geformt. \\
Jozah deckte das Tuch wieder darüber, verstaute die Waffe aber nicht. Morgen würde er sie wieder 
gründlich schärfen, denn seid er wieder in der Hauptstadt war, hatte er sie kaum angerührt. Selbst 
blinde wussten, dass man mit so einem Schwert höchstens Misstrauen unter Adel und Kaufleuten 
weckte. Die Menschen lebten gerne in ihrer Welt und wollten kein Zeichen des Krieges sehen. 
Zumindest kein unschönes Zeichen. Paraden, Uniformen, Turniere und Mutproben der Soldaten, diese 
Dinge waren gerne gesehen. Aber Waffen, die rein für den Zweck des Tötens und nicht des hübschen 
Anblicks geschaffen waren, die gehörten nicht zu dieser Vorstellungen der Menschen.\\

Obwohl er erst zu später Stunde in den Schlaf gefunden hatte, erhob sich Jozah wie gewohnt zum 
ersten Morgenapell. Die Zeit als er noch Rekrut war und nun eilig hinaus auf den Hof musste um die 
ersten zehn Runden zu drehen und anschließend in der Wäscherei zu arbeiten, bevor es Frühstük gab, 
war Osyma sei Dank schon lange vorbei. Trotzdem steckte es ihm noch wie jedem anderen Soldaten im 
Blut. \textit{Vermutlich gibt es diese Totur für die Knaben nur, damit alle anderen in Ruhe 
frühstücken können}, dachte er, während er seine Uniform richtete und nach seiner Rasierklinge 
griff.\\
Im Speißesaal dann setzt er sich neben einen breitschultrigen Blondschopf, der mit dem Kopf über 
der Schüssel Haferbrei hing.\\
``Lange Nacht, Herr Kommandant?'', grunzte Mishka: ``Bist so auffallend spät für deine Maßstäbe.''\\
``Ich musste das Herz einer Frau erobern'', erwiderte Jozah seinem langjährigen Freund.\\
Auch wenn Mishka der Inbegriff des Adelssohn beim Militär war - reich, gutaussehend, arrogant und 
ein schlagkräftiger Kämpfer - war er Jozah unterstellt. Wie jeder Adelige hatte er nach seiner 
Rekrutenzeit als Kommandant angefangen, jedoch durch einige freche Äußerungen, unüberlegte 
Handlungen oder verschlafene Tage den Status verloren. Es sollte damals nur eine Lehre für den 
verwöhnten Sohn sein, so hielten sie es hier immer. Vermutlich hätte Mishka die Nachfolge des 
Kommandanten, der damals ihre Truppe in den Kolonien führte, antreten sollen, ehe alles anders 
verlief. Es schien dem Blondschopf recht zu sein, so war er doch damals einer der ersten gewesen, 
die sich auf Jozahs Seite gestellt hatten. \textit{Wäre Mishka nicht gewesen, hätten sie vielleicht 
alle gemeuchelt.}\\
``Natürlich'', spottete Mishka: ``Deshalb habe ich überhaupt nicht geschlafen. Habe kein Weib 
gefunden und jeder Schlaf, den man nicht in den Armen eines hübschen Weibes verbringt, ist doch 
sinnlos.''\\
Jozah aß einige Löffel schweigend, ehe er sich räusperte und leise sprach: ``Ich meine es ernst... 
Ilia Ma'Har will mich heiraten...''\\
Mishka hielt in der Bewegung inne und sah seinen Kommandanten verblüfft an. ``Ilia Ma'Har? 
Verarscht du mich? Ihr Vater ist mit dir einverstanden? Ich meine... du weißt was ich meine.''\\
Jozah zog die Schultern hoch und sah sich flüchtig im Saal um. So vertraut er auch mit seinem 
Kameraden war, musste nicht jeder mitbekommen, wie ein Soldat seinen Kommandanten unterstellte, 
nicht wert genug für eine Frau zu sein.\\
``Ich habe ihn noch nicht gefragt. Eigentlich habe ich auch Ilia nicht gefragt...''\\
``Ist sie schwanger?'', flüsterte Mishka und seine Lippen verzogen sich zu einem breiten Grinsen.\\
``Nein!'', rief er empört aus und duckte sich gleich wieder über seine Schüssel um den fragenden 
Blicken der übrigen Anwesenden auszuweichen. ``Sie hat mich überrumpelt... oder... keine Ahnung. 
Hast du schon mal erlebt, dass eine Frau es will aber aus Anstand nicht den ersten Schritt machen 
darf und dich dann so mit Worten überhaupt, dass du denkst, dass du selbst den ersten Schritt getan 
hast...?''\\
Mishka hob kritisch eine Augenbraue: ``Fals du mit Es Sex meinst, ja. Eine Verlobung... nein, das 
noch nicht. Aber wieso auch nicht... Manche Frauen sind es gewohnt immer zu bekommen was sie wollen 
- egal was - und holen sich das dann auch. Ilia Ma'Har soll der Inbegriff solcher Frauen sein. 
Gerüchte zufolge.''\\
``Sie ist eine charmante, sehr hübsche junge Frau'', murrte Jozah verärgert und verteidigte die 
Ehre seiner inoffizellen Verlobten: ``Wir haben einige anregende Gespräche geführt und schöne 
Nachmittage miteinander verbracht. Ach... was solls, ich gehe heute zu ihren Vater. Schlimmer als 
mich aus den Haus jagen und mich in den höheren militärsichen Kreisen blamieren kann er ja nicht 
machen...''\\
Mishka leerte seine Schüssel. ``Wieso so eilig? Warum nicht erst noch ein paar weitere... anregende 
Gespräche?''\\
Jozah biss sich auf die Lippen. ``Weil wir morgen aufbrechen'', flüsterte er: ``Befehl vom König 
persönlich. Ruf die Truppe auf dem Hof zusammen, dann gebe ich euch die Befehle.''\\
Mishka schmiss seinen hölzernen Löffel auf den Tisch und rief mit abenteuerlustig funkelnden Augen: 
``Dein ernst, Jozah Mi'Kae? Du prahlst mit Weibergeschichten, noch ehe du mit einem Befehl direkt 
vom König herausrückst? So kennt man dich ja gar nicht!''\\
``Mach einfach was ich dir gesagt habe'', seufzte Jozah.\\

Er hielt die Ansprache an seine 50 Mann starke Truppe kurz und knapp, nannte den Grund ihrer Reise 
- die junge Witwe zu unterstützen - und dass sie am nächsten Morgen aufbrechen würden. Die 
notwendigen Aufgaben um die sich noch gekümmert werden musste, wie Proviant und das Überprüfen der 
Ausrüstung, verteilte er und gab ihnen ansonsten den Rest des Tages frei.\\
Es war später Vormittag, ehe er seine persönlichen Vorbereitungen für die Reise getroffen hatte und 
sich für die Aufgabe, die nun noch heute vor ihm lag, herausgeputzt hatte. Die Uniform war 
frisch ausgebürstet und die Manschetten poliert bis sie glänzten. Sein Pferd, ja, das klang 
wirklich gut... Sein Pferd war gestriegelt, Mähne und Schweif geflochten, als würde es gleich in 
einer Parade laufen. Das musste man den Stallburschen lassen, in dem herrichten der Pferden waren 
sie einmalig. Locker saß Jozah im Sattel, den Rücken gerade und den Kopf erhoben. Im Allgemeinen 
bevorzugte er es, übersehen zu werden, aber das war nicht vom Vorteil, wenn man einer schönen Frau 
den Hof machen wollte. Den Säbel hatte er nicht mitgenommen. Die Waffe lag wieder in der Kammer, 
dort wo sie hingehörte. Aber auch sein Schwert hatte er nicht mit. Es war einfach nichts für 
einen friedlichen Tag in Begleitung einer Dame. \\

 ``Werter Herr", grüßte ein Bediensteter, der ihm die Türe öffnete, und verneigte sich leicht:  
``Der Hausherr hat gehofft, dass Sie bald zu Besuch kommen würden. Sie sind herzlich eingeladen, 
einzutreten."\\
Damit hatte Jozah nicht gerechnet, aber er stieg aus dem Sattel und reichte dem Bediensteten die 
Zügel. Er stand etwas verloren neben dem Pferd und sah sich unschlüssig um. Nun, das war eben ein 
Problem mit den Stadtanwesen. Sie besaßen meist keine Ställe oder aber nur so kleine, dass kaum ein 
weiteres Pferd hinein passte.\\
 ``In der Nähe ist ein Gasthaus", erklärte Jozah und reichte ihm ein paar Münzen:  ``Stellt es dort 
unter und sagt dem Wirt, dass ich es bald abhole."\\
Als Jozah beschwingten Schritts die wenigen Treppenstufen hinauf eilte, stand dort bereits 
Ma'Har. Ilias Vater war geschätzte 60 Jahre. Sein Haar war licht, aber seine Wangen sorgfältig 
rasiert. Obwohl der Mann sich auf einen Stock stützte, wirkte seine Haltung stolz und aufrecht.\\
 ``Jozah Mi'Kaé", verkündete er fast schon feierlich und hob das Kinn:  ``Der Mann, der meiner 
Tochter nicht mehr aus dem Kopf geht."\\
Jozahs Wangen färbten sich rot. Er hätte nicht gedacht, das Ilia bereits so viel erzählt hatte. An 
sich waren sie doch nur wenige Male miteinander ausgegangen. Und... ja, da war natürlich noch der 
Antrag, den Ilia sich eingebildet hatte. \\
 ``Ich nehme an, Sie möchten Ilia zu einem Ausflug entführen?"\\
 ``Wenn Ihr es erlaubt", sagte Jozah respektvoll.\\
 ``Erst sollen wir uns etwas unterhalten. Tretet ein. Links die zweite Türe, dort ist der Salon."\\
Der Kommandant folgte Ma'Har und grübelte. Vielleicht würde in wenigen Minuten die ganze 
Angelegenheit erledigt sein. Er würde weiterhin als Jungesselle dieses Haus wieder verlassen und 
Ilia sich einem anderen, vermutlich adeligem, Mann zu wenden. Was konnte Jozah ihr schon bieten? 
All seine Zweifel kamen wieder an die Oberfläche seiner Gedanken. Am liebsten hätte er sich hastig 
für das Missverständnis entschuldigt und aus dem Haus gerannt.\\
Der Salon war minimalistisch eingerichtet. Ebenso wie der Flur, den sie durchquert hatten. Es 
standen dort lediglich zwei gepolsterte Sessel und ein dazu passendes Sofa in der selben, saftig 
grünen Farbe. Dazwischen ein kleiner Beistelltisch aus dunklem Holz, glatt poliert aber ohne 
Verzierungen. Ein gewobener Teppich nur lediglich verworrene Muster zeigte. Durch die farblosen 
Fenster fiel das Mittagslicht herein und zauberte Muster aus Sonnenstrahlen auf das Parkett. Ma'Har 
thronte bereits in einem Sessel und bedeutete ihm, sich auf den zweiten nieder zu lassen.\\
\textit{Dieser Raum wirkt herrschaftlicher als der Ratssaal des Königs.}\\
 ``Wollen Sie Tee oder können wir auf diese formellen Dinge verzichten?"\\
 ``Oh... nun, ich kann verzichten", antwortete Jozah unsicher, ob dies ein Test war.\\
Ma'Har lachte und stampfte dabei mit seinem Stock auf den Boden. Das Geräusch wurde vom Teppich 
verschluckt.  ``Wissen Sie, Mi'Kaé. Ich mag Sie! Sie sind so direkt. Ganz ein Soldat."\\
 ``Wart Ihr auch beim Militär?", fragte Jozah neugierig.\\
Ma'Har verzog bedauerlich das Gesicht.  ``Nein. Meine Eltern haben mir damals mit Enterbung und 
Verstoßung gedroht, sollte ich die Waffe wählen. Die haben das damals früh geplant. Der eine Sohn 
geht zum Militär, der andere in den Tempel und der Dritte bleibt zu hause und erbt. Und jeden haben 
sie nach diesen Plänen erzogen. Mein Bruder bekam schon ein Holzschwert in die Hand 
gedrückt, bevor er laufen konnte! Aber meine Bewunderung ist geblieben, dass können Sie mir 
glauben. Ilia sagte, Sie reiten morgen fort."\\
 ``Nach Merandila", erwiderte Jozah und nickte zustimmend:  ``Ich weiß nicht, wann ich wieder in 
der Hauptstadt sein werde. Deshalb... ich vermute Eure Tochter wird Euch etwas erzählt haben. 
Eigentlich ist das gestern alles etwas überstürzt passiert."\\
Ma'Har zwinkerte ihm zu.  ``Ich kenne meine Tochter. Solange Sie Ilia nicht geschwängert haben, ist 
es mir egal. Mein Anliegen ist hauptsächlich, dass ich Ihnen sagen möchte, was vermutlich geschehen 
wird. Ilia ist ein Schmetterling. Sie flattert von einem Ort zum nächsten und so sind auch ihre 
Gefühle. Will sie an einem Tag unbedingt Tanzen lernen und droht damit, in einen Essstreik zu 
verfallen, sollte ich ihr nicht sofort einen Tanzlehrer hinstellen und am nächsten will sie lieber 
Zeichenunterricht. Das hat in erster Linie nichts mit Verwöhnung zu tun, ich erfülle ihr auch 
selten genug Wünsche. Ich hoffte mit diesem Verhalten zu erreichen, dass sie etwas besonnener wird, 
aber anscheinend ist das einer ihrer Charaktermakel. Sie lässt sich schnell für Dinge begeistern. 
Auch wenn ich zugeben muss, dass sie bisher von keinem Mann dermaßen geschwärmt hat."\\
Jozah fühlte sich wie ein nutzloser Sack voller nassem Stroh.  ``Ich... ich kann Eurer Tochter 
nichts bieten, Herr. Ich habe kaum Geld, keinen Adelstitel und kann nicht einmal garantieren, dass 
ich länger als ein paar Wochen an einem Ort bleiben kann."\\
Ma'Har machte eine wegwerfende Geste.  ``Grübeln Sie doch nicht so viel. Ilia bringt genügend Geld 
mit in diese Verbindung, dazu einen Wohnsitz den ich bisher noch vermietet habe. Am Geld soll es 
nicht mangeln. Sie ist meine einzige Tochter, das Erbe ist gesichert und ich habe ihrem Bruder die 
vertragliche Pflicht auferlegt, auch nach meinem Ableben sicher zu gehen, dass Ilia genug für ein 
angemessenes Leben hat. Und es mag stimmen, dass Sie keinen Adelstitel tragen, aber Ihr Name klingt 
immerhin adelig und solange die Form gewahrt ist... Ich mag Sie, Mi'Kaé. Sie sind vernünftig und 
besonnen. Sie schätzen meine Ilia. Ich schlage Ihnen diese Vereinbarung vor. Ich stimme einer 
Verlobung zu und lasse sie verkünden. Sollte Ilias Meinung drei Monate darauf beruhen, wird die 
Hochzeit stattfinden. Sie wird, solange Sie im Dienst des Militärs nicht in der Hauptstadt sein 
werden und ich noch lebe, hier zuhause bleiben."\\
Ein Lächeln ergriff Jozahs Miene und er strahlte den alten Mann an. Ma'Har erwiderte die Geste mit 
einem grimmigen grinsen und nickte.  ``Ich denke, dies genügt als Antwort. Nun denn, nehmen Sie sie 
mit und verabschieden Sie sich. Doch heute hätte ich sie gerne wieder vor Sonnenuntergang im 
Haus."\\