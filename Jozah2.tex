\chapter{Schnelle Versprechen}

Jozah musste sich nun beeilen, um zu seiner Verabredung nicht zu spät zu kommen. Er eilte über den 
Palasthof und die Stufen hinab. Der Palast befand sich auf einem erhöhten Hügel, in welchem 
rundherum Stufenweise die Erde abgetragen und Gärten angelegt waren. An sich ähnelte der Standort 
also einer immensen Treppe auf deren Spitze der königliche Palast weilte. Während die Unterste der 
drei Stufen für das gesamte Volk zugänglich waren, bewachten Palastwachen die Höheren um die 
angesehenen Leute einzulassen. Die Zweite Ebene durfte die Mittelschicht betreten. Die dritte Ebene, 
unterhalb des Palastes, war nur für den hohen Adel, Priester und hochrangige Militärmitglieder 
vorbestimmt.\\
Jozah steuerte die zweite Ebene an. Er eilte über den Kiesweg und entdeckte eine Person am 
steinernen Geländer, von welcher man auf die dritte Ebene hinab sah. Die Sonne ging gerade unter und 
färbte den Himmel rötlich. Sie trug einen Sonnenschirm bei sich, wie oft, wenn sie alleine in der 
Stadt unterwegs war.\\
Zu Beginn ihrer Bekanntschaft hatte er es einmal gewagt, einen Scherz über den Schirm zu machen. 
Bis sie ihn eines Besseren belehrte und anhand einer praktischen Übung zeigte, dass er sehr wohl 
einen Nutzen hatte.\\
Die schöne Ilia Ma'Sah. Ihr Name sagte vermutlich jedem Jungessellen mit Rang etwas. Ihr Haar 
reichte bis zu ihrer Hüfte, wenn sie es offen trug. Augen, so blau wie das Meer an einem warmen 
Sommertag. Einladende Hüften, eine schmale Taille. Aber das lag vermutlich an ihrem eng geschnürtem 
Korsett. Die Locken wirkten etwas zu perfekt. Jozah hätte gerne gefragt, wie lange ihre Zofe wohl an 
diesen Locken gesessen hatte. Aber schöne reiche Frauen hörten so etwas vermutlich nicht gerne.\\
Seine Schritte knirschten auf dem Kies und die junge Frau wandte sich um. Die zarten, blassen Hände 
hielten den Griff des Sonnenschirms fest. Sie schlug die Augen auf und blickte ihn mit einem leisen 
Lächeln an. Jozah versuchte es zu erwidern. Der Tag hatte bisher solch unangenehmen Überraschungen 
enthalten, dass er auf die Verabredung eigentlich hätte verzichten können. Und das, obwohl er am 
Morgen noch die Stunden bis zu diesem Moment gezählt hatte.\\
Seit seiner Rückkehr aus den Kolonien traf er sich nun schon mit Ilia. Anfangs eher unfreiwillig. 
Was sollte eine junge Adelige mit seiner Gesellschaft? Sein ehemaliger Mentor, der mittlerweile in 
Ruhestand gegangene Offizier Ma'Sah hatte darauf bestanden, dass Jozah einige Ausflüge mit seiner 
reizenden Tochter unternahm. Jozah war schnell bewusst geworden, worauf der ehemalige Offizier 
hinaus wollte. Es war kein Geheimnis in Brom-Dallar, dass die einzige Erbin des Hauses Ma'Sah ihre 
Nächte lieber auf wilden Gesellschaften und ihre Tage auf dem Rücken eines Jagdpferdes verbrachte. 
Meistens in der Gesellschaft von Männern. Jozah befürchtete, Ilia wäre zu aufbrausend für sein 
eigentlich ruhiges Gemüt. Aber wie hätte er seinem alten Gönner diesen Wunsch abschlagen können?\\
Außerdem war Ilia eine fantastische Frau, schön wie gemalt, zart und sanft wie ein Schmetterling. 
Zumindest manchmal.\\
 ``Schade, dass Sie den Sonnenuntergang verpasst haben, Kommandant", sagte sie mit einem Lächeln 
in der Stimme.\\
 ``Verzeiht. Ich hatte einen dringenden Termin."\\
Sie betrachtete seinen Säbel und legte den Kopf schief. ''Der steht Ihnen nicht, Jozah. Mir würde 
er bestimmt besser passen.``\\
''Ach Ilia... ich werde nicht gegen Euch fechten!``\\
Sie warf den Kopf zurück und lachte spöttisch auf. ''Angst zu verlieren? Nein... sorgen Sie sich 
nicht, das Wort meines Vaters ist ja deutlich genug... Ich darf nur in unserem Salon üben. Meinen 
Sie es würde ihm auffallen, wenn ich meinem Schirm eine Klinge hinzufügen lassen? Immerhin weiß man 
als Dame in der Stadt nie...``\\
''Ich würde es nicht riskieren.``, murmelte Jozah.\\
''Nun fürchten Sie sich doch nicht so sehr vor meinem Vater. War er ein so strenger Mentor? Er mag 
Sie, Jozah. Sehr sogar.``\\
''Streng? Nun... er will, dass sein Wort geachtet wird...``\\
Die Frau legte den Kopf schief und verzog nachdenklich das Gesicht. ''Ich weiß gar nicht, was alle 
immer haben. Natürlich hat er mir immer schon Regeln vorgeschrieben, aber wenn ich mich nicht daran 
hielt, konnte er mir nie lange böse sein!``\\
Jozah lachte spöttisch und bot ihr den Ellbogen, um sich einzuhacken. Sie kam dem nach und dann 
spazierten sie an der Mauer des Gartens entlang. ''Natürlich kann er Euch gegenüber nicht lange 
böse sein. Ein Lächeln und man vergibt Euch alles!``\\
Ilia schüttelte lachend den Kopf. ''Nein, nein. Mir hat man schon viel verwehrt! Dann liegt es wohl 
doch eher daran, dass ich seine Tochter bin! Ganz Brom-Dalar erzittert, wenn Offizier Ma'Sah seine 
Stimme erhebt und sich über die versalzene Suppe im Gasthaus beschwert! Nur ich nicht. Nun sagen 
Sie schon, was war so wichtig, dass Sie mich vergessen haben?``\\
Jozah biss sich auf die Zunge und murmelte: ''Der König ließ mich zu sich rufen.``\\
Überrascht blieb Ilia stehen und sah den Kommandanten fragend an. Sie selbst hatte den König noch 
nie persönlich getroffen. Die Begrüßungen auf den Bällen, die es früher noch häufiger gegeben hatte 
als heute, war das Nähste gewesen. Und da lagen trotzdem noch etliche Meter zwischen dem König und 
seinen Gästen. Jozah sprach nicht weiter und Ilia blickte ihn besorgt an.\\
''Es hat sich plötzlich viel verändert``, begann er und überlegte, wie er ihr die Sorge nehmen 
konnte. ''Die letzten Monate mit Euch waren toll. Ich meine... ich habe Eure Gesellschaft sehr 
genossen! Also... Ihr habt mein Leben sehr erhellt!``\\
Er kniff die Lippen zusammen und suchte fieberhaft nach Worten. Der Dame würde es keinesfalls 
gefallen, dass er in zwei Tagen die Hauptstadt verlassen musste. Er konnte ihren Trotz schon vor 
sich sehen, wie sie die Arme vor der Brust verschränken und sich abwenden würde. Dabei stimmten 
seine Worte sogar. Jozah konnte sich nicht entsinnen, schon einmal eine friedlichere und schönere 
Zeit als die letzten Monate erlebt zu haben. Sie hatte die Albträume der Kolonien aus seinem 
Bewusstsein verdrängt. Ihre Geschichten von einerseits unwichtigen, aber erheiternden Geschehnisse 
im Adelskreis hatten die Schreie vergessen lassen, wenn er ein Leben beendete. Den Schmerz, wenn 
die feindliche Klinge seine Haut zerriss.\\
Ihre blauen Augen sahen ihn unbeirrt abwartend an. Mittlerweile war die Sonne hinterm Horizont 
verschwunden und die Laternen wurden entzündet. \textit{Wir sollten uns auf den Heimweg machen.}\\
Mit dem Vorhaben, die Verabredung nun doch an diesem Punkt zu beenden und sich davor zu drücken, 
der Dame von seinem Auftrag zu erzählen, holte er tief Luft und sagte: ''Ich weiß, dass kommt nun 
plötzlich für Euch. Vor allem, weil ich Euch auch noch habe warten lassen und die Verabredung alles 
andere als perfekt gelaufen ist... wollen wir...`` \textit{uns auf den Heimweg machen,} wollte er 
noch sagen, kam aber nicht dazu, den Satz zu beenden. Denn Ilia ließ überrascht seine Hand los und 
sah ihn erstaunt an, während sich ein strahlendes Lächeln auf ihre Lippen legte.\\
''Wirklich? Ich meine... nein, Sie haben schon recht, das ist kein angemessener Zeitpunkt für einen 
Antrag!``, lachte sie: ''Aber ich warte ja schon seit einigen Tagen darauf, dass Sie sich endlich 
trauen! Glauben Sie mir, ich war kurz davor, Ihnen einen zu machen! Aber mein Vater meinte, es wäre 
nicht recht, wenn ich als Adelige einen mache und dass ich mich gedulden soll``, plapperte Ilia 
aufgeregt.\\
''Wie bitte?``, stammelte Jozah überrumpelt.\\
Ilias Wangen färbten sich rot: ''Aber natürlich sage ich ja!``\\
Sie trat einen schnellen Schritt auf ihn zu und schlang ihre Arme um ihn. Jozah erwiderte die feste 
Umarmung unwillkürlich und nahm den Duft ihrer Haare wahr.\\
Seine Gedanken glichen einem tobenden Sturm, während er sich einerseits sorgte, was da gerade 
passiert war, aber andererseits sich ein Leben mit Ilia vorstellte. Sie hatte keinen Hehl daraus 
gemacht, dass es ihr keineswegs etwas ausmachte, als Erbin viel, viel Geld mit in eine mögliche 
Partnerschaft zu bringen, auch wenn er als Kommandant wenig dazu beitragen konnte.\\
''Ich habe Euren Vater noch nicht gefragt``, platzte es aus ihm heraus.\\
''Dann machen Sie das morgen. Er wird nicht nein sagen. Ich sagte ja, er mag Sie, Jozah. Ihr 
erinnert in an meinen Bruder, bevor er im Dienst fiel.``\\
Jozah löste sich sanft von ihr und fügte leise hinzu: ''Und das mit dem König... also er schickt 
mich nach Merandila.``\\
Ilia runzelte die Stirn. ''Wann?``\\
''In zwei Tagen.``\\
Wie erwartet sah sie ihn erst erschrocken an, dann verschwand das Lächeln und sie funkelte ihn 
grimmig an. ''Wie bitte? Für wie lange? Monate? Jahre? Weißt du eigentlich, wie kalt dort die 
Winter sind? Und die Sommer erst! Und die einzige Großstadt ist dieses Na'Rash, was fast nur aus 
Tempeln und Schulen besteht!``\\
Jozah wusste nicht, was er dazu sagen sollte und murmelte nur: ''Es tut mir Leid.``\\
''Und dann machst du jetzt einen Antrag?``, rief sie und war unbewusst in eine vertraute Anrede 
gefallen. Immer wieder schüttelte sie den Kopf. ''Ich meine... natürlich verstehe ich das. Du 
willst, dass ich auf dich warte. Wie ein Schoßhündchen!``\\
''Ilia...``, rief er: ''Magst du denn mit?``\\
''Nach Merandila? Da hoch in den Norden?`` Ilia lachte spöttisch: ''Vergiss es. Die Mode kommt da 
erst hin, wenn es ein Jahr zu spät ist. Und wie gesagt, es ist verflucht kalt! Und Brom-Dalar ist 
meine Heimat. Hier sind mein Vater und meine Freunde.``\\
Einen Moment herrschte Stille zwischen ihnen, dann sagte Jozah erklärend: ''Ich bin Soldat und habe 
Verpflichtungen gegenüber...``\\
''Dem König, dem Allmächtigen und dem Vaterland. Ich weiß. Ich bin die Tochter eines Offiziers, was 
glaubst du, wie oft ich schon in der Tür stand und wartete? Wie oft ich aus dem Fenster sah? Wie 
oft ich Briefe schrieb? Ich lernte schon mit vier Jahren schreiben und lesen, weil das der einzige 
Weg war Kontakt mit meinem Vater aufzunehmen.``\\
''Ilia...``\\
Sie hob die Hand und unterbrach ihn: ''Also gut. Ich nehme den Antrag trotzdem an. Du gehst nach 
Merandila und ich warte hier auf dich. Wenn du wieder kommst, heiraten wir? Aber ich verspreche 
dir, ich werde trotzdem tanzen gehen und bei den Treibjagden mitmachen. Ich werde kein 
Hausmütterchen, welches stickend im Salon sitzt und sich weinend ausmalt, in welcher Schlacht der 
Liebste gerade fällt!``\\
Jozah stellte sich die aufbrausende Dame stickend vor und musste laut lachen. ''Sicher. Anders 
könnte es auch nicht sein!``\\
Sie nickte und griff ihren Schirm fester. ''Dann gehe ich jetzt, es ist spät. Und ich erwarte, dass 
wir noch etwas romantische Zeit miteinander verbringen, ehe du nach Merandila aufbrichst und 
vermutlich erfrierst! Bis morgen.``\\
Sie hauchte ihm einen Kuss auf die Wange und ehe Jozah fragen konnte, ob er sie nicht doch besser 
nach Hause geleiten sollte, marschierte sie schon los. 
''Ich bin verlobt``, sagte er leise zu sich selbst und strich sich fahrig das Haar aus der Stirn.\\


Der Halbmond beschien die Straßen der Hauptstadt, hüllte sie in ein gespenstisches Licht. Aber die 
Nacht war keineswegs still. Besonders nicht in der Nähe der Kaserne. Hin und wieder hörte man das 
lockende Kichern einer Hure, auf der Suche nach Kunden. Die Gaststätte war selbst zu hören, als 
noch etliche Straßen zwischen ihr und Jozah lagen. Grölen, Klappern von Krügen und Poltern von 
Stühlen. Gesang und Musik. Ein Betrunkener überquerte vor Jozah stolpernd und lallend die Straße. 
Er hielt den Blick gesenkt, saß auf den Weg und grübelte nach. Jozah war kein typischer Soldat der 
saleicanischen Armee. Männer und Frauen Saleicas waren weither bekannt als Draufgänger, Schläger, 
Menschen die lieber zur Waffe griffen als den Mund auf zumachen. Mut, nannten sie es, und Ehre, 
wenn man blindlings in einen Kampf stürzte, täglich unsinnige Mutproben ausfocht und grölend 
Lobeshymnen auf den König, das Land und Osyma sang. So war Jozah nicht. Weder Streben nach Ehre 
noch Kampfeswut waren es, weshalb er jetzt hier stand. Er ging zur Armee, weil er keinen anderen 
Ausweg sah. Das hatte sich bis heute nicht geändert. Nichts anderes konnte er sich vorstellen. Was 
wäre ihm als Sohn eines Handwerkers für ein Leben beschieden gewesen? Entweder angestellt in der 
Werkstatt, die ein älterer Bruder erben würde, oder bei irgendjemand anderem, der vermutlich noch 
weniger für ihn übrig haben würde. \\
\textit{Priester}, dachte er und verzog angewidert das Gesicht. \\
Als Rekrut hatte er es schwer gehabt. Die Anderen sahen es nicht gerne, dass ein so junger Knabe 
unter ihnen war. Er wurde geprügelt, verflucht und ausgelacht. Aber er war standhaft geblieben. Die 
ein oder andere Mutprobe hatte Jozah damals mitgemacht, aber eher um akzeptiert zu werden anstatt 
aus ehrlichem Interesse. Standhaftigkeit, dass war vielleicht seine große Fähigkeit. Er war weder 
besonders selbstbewusst, noch mutig. \\
Die Belagerung kam ihm in den Sinn. Der Offizier hatte seine Taten so gelobt, aber Jozah glaubte 
nicht daran, dass er nicht die Wahrheit wusste. Wochenlang waren sie in der klapprigen 
burgähnlichen Stellung verharrt. Der einzige Grund, warum die Rebellen das hölzerne Gestell nicht 
abfackelten war, dass sie keinen Flächenbrand riskieren wollten. Das war in den trockenen, 
heißem Land nicht unüblich und ein solcher kaum zu löschen. Sie hätten sich damit nur ins eigene 
Fleisch geschnitten. Nein, sie hatten die Taktik aushungern versucht. Wäre ihnen auch fast 
gelungen, wenn die Saleicaner nicht so stolz wären. Immer wieder schickte der Kommandant seine 
Männer vor das Tor, ließ Pfeile sinnlos durch die Luft surren, nur um dann wieder zurück hinter den 
Wall zu huschen. Bei einem dieser sinnlosen Ausfälle, die wohl mehr Mut darstellen sollten, als 
überhaupt vorhanden war, wurde der Kommandant selbst erwischt. Von einem Pfeil im Rücken wollte er 
sich erst nicht aufhalten lassen, bis die Wunde sich entzündete. Während die meisten der Soldaten 
einen endgültigen Ausfall planten, versuchte Jozah die Wunde seines Kommandanten so gut wie möglich 
zu versorgen. Er wusste natürlich was im Gange war. Es schien so, als hätten etliche der Rebellen 
sich zurückgezogen und nur noch ein Drittel der ursprünglichen Belagerer wäre anwesend. Oh, und wie 
Jozah sich noch an die Worte des Kommandanten erinnerte. Er befahl ihm, den Soldaten auszurichten, 
sie sollen sie alle fertig machen. Jeden einzelnen. Tja, somit hatte er das Kommando über die 
Soldaten erhalten, nachdem der Verantwortliche in eine friedliche Ohnmacht gefallen war. Es wäre 
nicht schwer gewesen diesen Befehl in die Tat umzusetzen. Ganz im Gegenteil, die Männer waren 
praktisch schon aus dem Thor heraus, als Jozah in den Hof trat und sie alle zurück pfiff. Ein 
knapper Bericht über den Zustand des Anführers folgte und Jozah hätte die Männer ziehen lassen. Wenn 
Elor, ein stiller Kamerad der dem Gespräch zeuge gewesen war, nicht erwähnte, dass Jozah nun die 
Entscheidungsgewalt übertragen bekommen hatte. Und augenblicklich fühlte Jozah sich, als stünde er 
mitten in einem Rudel junger Hunde, behangen mit Würsten und Schinken. Es war eine Sache, zuzusehen 
wie eine Gruppe Männer, getrieben von Blindheit und Siegeswillen, los stürmten. Er hatte ja selbst 
gesehen, dass die Belagerer sich teilweise zurückgezogen hatten und wäre vermutlich selbst mit 
gestürmt. Schon allein aus dem Grund, um endlich irgendetwas zu tun zu haben. Ja, er verstand die 
Soldaten. Wochenlang eingesperrt auf engsten Raum, die Vorräte erreichten ihr Ende, der Kommandant 
schwer verwunden und eine scheinbar geringe Anzahl an Feinden. Aber es war etwas anderes, besagte 
Männer in den Tod zu schicken. Selbst beim Schachspiel brachte er es kaum über sich, seine Bauern zu 
opfern um stärkeren Spielfiguren Angriffschancen zu bieten. Das empfand er als ein sicheres Zeichen, 
dass er sich einfach nicht dafür eignete, Verantwortung zu übernehmen. Vielleicht ein weiterer 
Grund, warum er als Junge zum Militär gegangen war. Er hatte jemanden gesucht, der ihm 
Entscheidungen abnahm, die Verantwortung trug, dem er sein Leben anvertrauen konnte. Und jetzt war 
er plötzlich in dieser Lage. Vermutlich hätte es ihm keiner nachgetragen, wenn er dem Plan des 
Kommandanten und der Männer nachgegangen wäre. Immerhin erwartete man das von Saleicanern. Aber 
stattdessen hatte er sich gedrückt, die Männer vertröstet. Er weiß noch, nur knappe Worte waren über 
seine Lippen gekommen. \\  ``Verstärkt die Wachen, die anderen ruhen sich heute Nacht aus. Morgen 
wird es unserem Kommandant wieder besser gehen!"\\
Aus einer Nacht wurden vier. Und mit jeder Nacht wurde das Murren und Fluchen der Männer lauter und 
grober. Jozah rechnete fest damit, dass bald einer sich erheben, ihm am Kragen packen und 
ordentlich durch schütteln würde. Er war sich heute noch sicher, es wäre unweigerlich dazu 
gekommen, wenn nicht vorher die Verstärkung eingetroffen wäre. Erlo war es, der ihm von der 
Sichtung berichtete. Und dann ging alles schnell. Jozah rief zu den Waffen, die Tore wurden geöffnet 
und sie fielen den Belagerern, die sich der Verstärkung zu gewandt hatten, in den Rücken. 
Letzten Endes hatte sich herausgestellt, dass die Rebellen zahlreicher waren denn je. Sie kannten 
mittlerweile den Stolz der Eroberer, hatten darauf gewartet, dass sie auf die anscheinend geringe 
Anzahl reagieren würden. Hatten sie durch Jozahs Zögern jedoch nicht und völlig überrumpelt wurden 
sie nieder gemacht. Danach wurde er für seine Umsicht, seine Taktiken und Tarnmanöver in den Himmel 
gelobt. Keiner der Männer schien sich noch daran zu erinnern, dass sie beinahe gemeutert hätten. Der 
ursprüngliche Kommandant kam wieder zu sich, übergab Jozah einen Teil seiner Männer und beförderte 
ihm zum Kommandanten. Während den drei weiteren Jahren in den Kolonien hatte Jozah, seiner Meinung 
nach, ein gesundes Gleichgewicht zwischen Vorsicht und Dummheit erlangt. Wäre er noch einmal in der 
Situation, würde er vermutlich mehr riskieren. Wer weiß das jetzt schon. Was ihn noch mehr erstaunte 
war jedoch, dass die 50 Mann immer noch hinter ihm standen. Wenn man nur lange genug Seite an Seite 
auf dem Schlachtfeld stand, musste man einander wohl oder übel etwas bedeuten, vermutete Jozah. 
Erlo gehörte mittlerweile zu seinen engsten Vertrauen und Jozah entschied, dass er in ihm schon mal 
einen Reisegefährten für seinen momentanen Auftrag hatte. \\
Er hatte mittlerweile die Kaserne erreicht, steuerte aber nicht sofort auf die Unterkünfte zu. 
Stattdessen trat er in ein flaches, schmales Gebäude. Der Geruch nach Stroh und Pferd stieg ihm 
entgegen. Es war keine wirklich komfortable Unterkunft für die Tiere, aber die gab es für die 
Soldaten auch nicht. Man verließ sich anscheinend darauf, dass die Pferde genug Auslauf bekamen und 
sie somit ruhig in einem engen Stall stehen konnten. Jozah spähte den nur vom Mondschein 
beleuchteten Gang entlang. Irgendwo in der Dunkelheit, in der Nähe der Sattelkammer, schliefen 
vermutlich die Stallburschen. Möglichst leise um weder Tier noch Mensch zu erschrecken, trat er den 
Gang entlang. Es war nicht schwer, das bestimmte Pferd zu finden. Das Fell des Schimmels leuchtete 
wie der Mond selbst. \\
Das Tier hatte, wie die meisten anderen Pferde im Stall, bereits seine Schritte vernommen und 
blickte ihm abwartend mit gespitzten Ohren entgegen. Der Soldat trat an das Pferd heran und legte 
seine Hand auf das weiche Fell. Die Haltung des Tieres blieb weiterhin wartend. Jozah wusste genau, 
was der Wallach wollte. Stets erhielt er bei jeder Begrüßung ein Apfelstück, aber das hatte er 
jetzt nicht dabei. \\
 ``Morgen", vertröstete er den Schimmel und betrachtete die dunklen Augen des Tieres. \\
Jozah ritt das Pferd, seit er vor 3 Monaten mit seinen Leuten wieder in der Hauptstadt angekommen 
war. Es hatte weder einen offiziellen Namen noch einen zum Rufen. Jozah hatte bisher nie darüber 
nachgedacht, immerhin hat das Tier der Kaserne gehört und war ihm lediglich als Kommandanten 
zugeteilt worden. Aber Lerim hatte es ihm geschenkt. Jetzt gehörte der Schimmel Jozah. Bei den 
Gedanken musste er immer noch den Kopf schüttelnd. Er war kein Mann, der viel besaß. War er nie 
gewesen. Alles in allem, konnte er jetzt nur zwei Dinge sein eigen nennen. Darunter das Pferd. 
Selbst seine Uniformen waren von der Kaserne gestellt. Seinen Sold schickte er größtenteils an 
seine mittlerweile alt gewordenen Eltern, die er seid fünfzehn Jahren nicht mehr gesehen hatten. 
Sie konnten weder Lesen noch schreiben und somit herrschte auch keinen Kontakt. Immerhin tröstete 
er sich damit, dass sollten seine Eltern bereits verstorben sein, sein Bruder das Geld ebenfalls 
brauchen konnte. Es war nicht viel, aber Jozah bekam Unterkunft, Versorgung und alles nötige was er 
brauchte vom Militär. \\
 ``Schlaf gut", raunte er dem Schimmel zu, gab ihm noch einen kameradschaftlichen Klapps auf den 
Hals und verließ den Stall. Während seinem Besuch war es still geblieben. Manchmal wirkte die Stille 
auf ihm zu ungewöhnlich. Manchmal vergaß er, dass diese Tiere ebenso Soldaten waren wie er. Ihr 
Leben lang ausgebildet und auch schon in Schlachten aktiv geworden. Nur gehorsame und ruhige Pferde 
konnten in Saleica so weit kommen. Die Reiterei musste sich auf sie verlassen können. Mit diesen 
Gedanken noch im Kopf, machte Jozah sich auf den Weg, das Zweite zu Besuchen, welches nur ihm 
gehörte. \\

Seine Kammer war wie alle anderen in der Kaserne geschnitten. Klein, quadratisch und nur ein 
winziges Fenster ließ frische Luft herein. Rechts an der Wand befand sich ein schmales Bett mit 
einer strohgefütterten Matratze. Ein Tisch, zwei Stühle. Eine Truhe für Kleidung und persönliche 
Gegenstände. Die Wände waren grau, geziert von dem ein oder anderen Schmutzfleck. Die meisten 
Personen mit höheren militärischen Rängen bevorzugten, in ihren Wohnungen oder Häusern außerhalb 
der Kaserne zu wohnen. Es gab noch das ein oder andere komfortablere Zimmer für die Ausbilder, der 
Rest sah so aus wie Jozahs. Er hätte auf eine Wohnung sparen können. Das wäre vielleicht ein Jahr 
gewesen und schon hätte er sich eine gemütliche, nicht all zu exklusive leisten können. Aber er war 
immerhin erst seid wenigen Monaten hier, konnte sich nicht von dem Ort trennen, den er immer als 
Heimat betrachtet hatte. Er hatte sich als Kommandant lediglich den Luxus gegönnt, das Zimmer 
allein zu bewohnen. Die ranglosen Soldaten teilten sich den ähnlichen Raum zu zweit. \\
Wie er das alles lösen sollte, wenn er Ilia heiratete, wusste er nicht. Sie konnte nicht mit in die 
Kaserne und würde es auch nicht wollen. Sie war ihr großes Familienhaus gewohnt und würde auch 
nicht mit einer gemütlichen Wohnung zufrieden sein. Nun, ihre Mitgifte würde vermutlich für ein 
kleines Haus reichen, aber er war sich auch da nicht sicher, ob sie glücklich wäre. Ob sie sich 
darüber überhaupt schon Gedanken gemacht hatte?\\
Jozah seufzte. Er mochte sie. Mochte sie wirklich. Sie war so anders als die Menschen, die er 
gewohnt war. Er kannte die Kalten, die Zermürbten, die Grausamen, die Hassenden, die 
Hoffnungslosen, die ernsten Leute. Klar, seine Kameraden konnten auch zu herben Scherzen fähig 
sein, sangen Lieder oder erzählten Geschichten am Lagerfeuer. Aber ein jeder war gezeichnet von dem, 
was er erlebt hatte. Keinen Menschen konnte man mit einem Wort beschreiben, aber Ilia wirkte so 
gegensätzlich. 
Er mochte sie, mochte sie wirklich.\\
Jozah kniete sich vor die Truhe und öffnete das Schloss. Mit einem leisen Klick sprang es förmlich 
in seine Hand und der schwere Holzdeckel ließ sich anheben. Vorsichtig, fast ehrfurchtsvoll, nahm 
er den in ein schmutziges Tuch gewickeltes Laken heraus. Er legte den Gegenstand auf den Tisch und 
setzte sich auf den Hocker davor. Lange betrachtete Jozah es, ohne das Tuch beiseite zu schlagen. 
Es war wie ein Geheimnis. Sein eigenes, daher brauchte er es nicht zu lüften. Dies war der zweite 
Gegenstand, der ihm allein gehörte. Und diesen hatte er nicht geschenkt bekommen, sondern sich 
erkämpft und verdient. Das hob seinen Wert gleich an. Jozah seufzte und schlug das Tuch zurück. Es 
war ein schlichtes Schwert, was vor ihm lag. Ein Bastardschwert, wie man sie so nett nannte. Jozah 
brauchte es nicht zu schwingen um zu wissen, das es im Gleichgewicht war und perfekt in seine Hand 
passte. Ein Schwerte ohne Zierde, aber einem metallenen Schimmern welches einem Versprechen nahe 
kam. Ein Schwur auf seine eigene Schärfe und Tödlichkeit. \\
Sein ursprüngliches Schwert, welches er als einfacher Soldat getragen hatte und dem Land gehört 
hatte, war zerbrochen, als er damit die Axt eines Rebellen aufhalten wollte, die einem seiner 
Kameraden bedrohte. Am Ende der Schlacht hatte er in dem Feld voller Toten nach einer brauchbaren 
Ersatzwaffe gesucht und ein halbes Dutzend mit zum Schmied genommen. Keine von ihnen war sonderlich 
gepflegt und keine weckte den Anschein, dass sie nicht ebenfalls bei der nächst besten Gelegenheit 
brechen würde. Der Kamerad, den Jozah gerettet hatte, aber dafür in der nächsten Schlacht starb, 
bestand darauf für ihn den Schied, welcher nicht dem Militär angehörte, zu bezahlen. So war sein 
jetziges Schwert entstanden. Das Metall von den gesammelten Klingen wurde eingeschmolzen, möglichst 
gehärtet und geformt. \\
Jozah deckte das Tuch wieder darüber, verstaute die Waffe aber nicht. Morgen würde er sie wieder 
gründlich schärfen, denn seid er wieder in der Hauptstadt war, hatte er sie kaum angerührt. Selbst 
blinde wussten, dass man mit so einem Schwert höchstens Misstrauen unter Adel und Kaufleuten 
weckte. Die Menschen lebten gerne in ihrer Welt und wollten kein Zeichen des Krieges sehen. 
Zumindest kein unschönes Zeichen. Paraden, Uniformen, Turniere und Mutproben der Soldaten, diese 
Dinge waren gerne gesehen. Aber Waffen, die rein für den Zweck des Tötens und nicht des hübschen 
Anblicks geschaffen waren, die gehörten nicht zu dieser Vorstellungen der Menschen.\\

Obwohl er erst zu später Stunde in den Schlaf gefunden hatte, erhob sich Jozah wie gewohnt zum 
ersten Morgenapell. Die Zeit als er noch Rekrut war und nun eilig hinaus auf den Hof musste um die 
ersten zehn Runden zu drehen und anschließend in der Wäscherei zu arbeiten, bevor es Frühstück gab, 
war Osyma sei Dank schon lange vorbei. Trotzdem steckte es ihm noch wie jedem anderen Soldaten im 
Blut. \textit{Vermutlich gibt es diese Tortur für die Knaben nur, damit alle anderen in Ruhe 
frühstücken können}, dachte er, während er seine Uniform richtete und nach seiner Rasierklinge 
griff.\\
Im Speißesaal dann setzt er sich neben einen breitschultrigen Blondschopf, der mit dem Kopf über 
der Schüssel Haferbrei hing.\\
``Lange Nacht, Herr Kommandant?'', grunzte Mishka: ``Bist so auffallend spät für deine Maßstäbe.''\\
``Ich musste das Herz einer Frau erobern'', erwiderte Jozah seinem langjährigen Freund.\\
Auch wenn Mishka der Inbegriff des Adelssohn beim Militär war - reich, gutaussehend, arrogant und 
ein schlagkräftiger Kämpfer - war er Jozah unterstellt. Wie jeder Adelige hatte er nach seiner 
Rekrutenzeit als Kommandant angefangen, jedoch durch einige freche Äußerungen, unüberlegte 
Handlungen oder verschlafene Tage den Status verloren. Es sollte damals nur eine Lehre für den 
verwöhnten Sohn sein, so hielten sie es hier immer. Vermutlich hätte Mishka die Nachfolge des 
Kommandanten, der damals ihre Truppe in den Kolonien führte, antreten sollen, ehe alles anders 
verlief. Es schien dem Blondschopf recht zu sein, so war er doch damals einer der ersten gewesen, 
die sich auf Jozahs Seite gestellt hatten. \textit{Wäre Mishka nicht gewesen, hätten sie vielleicht 
alle gemeuchelt.}\\
``Natürlich'', spottete Mishka: ``Deshalb habe ich überhaupt nicht geschlafen. Habe kein Weib 
gefunden und jeder Schlaf, den man nicht in den Armen eines hübschen Weibes verbringt, ist doch 
sinnlos.''\\
Jozah aß einige Löffel schweigend, ehe er sich räusperte und leise sprach: ``Ich meine es ernst... 
Ilia Ma'Har will mich heiraten...''\\
Mishka hielt in der Bewegung inne und sah seinen Kommandanten verblüfft an. ``Ilia Ma'Har? 
Verarscht du mich? Ihr Vater ist mit dir einverstanden? Ich meine... du weißt was ich meine.''\\
Jozah zog die Schultern hoch und sah sich flüchtig im Saal um. So vertraut er auch mit seinem 
Kameraden war, musste nicht jeder mitbekommen, wie ein Soldat seinen Kommandanten unterstellte, 
nicht wert genug für eine Frau zu sein.\\
``Ich habe ihn noch nicht gefragt. Eigentlich habe ich auch Ilia nicht gefragt...''\\
``Ist sie schwanger?'', flüsterte Mishka und seine Lippen verzogen sich zu einem breiten Grinsen.\\
``Nein!'', rief er empört aus und duckte sich gleich wieder über seine Schüssel um den fragenden 
Blicken der übrigen Anwesenden auszuweichen. ``Sie hat mich überrumpelt... oder... keine Ahnung. 
Hast du schon mal erlebt, dass eine Frau es will aber aus Anstand nicht den ersten Schritt machen 
darf und dich dann so mit Worten überhaupt, dass du denkst, dass du selbst den ersten Schritt getan 
hast...?''\\
Mishka hob kritisch eine Augenbraue: ``Fals du mit Es Sex meinst, ja. Eine Verlobung... nein, das 
noch nicht. Aber wieso auch nicht... Manche Frauen sind es gewohnt immer zu bekommen was sie wollen 
- egal was - und holen sich das dann auch. Ilia Ma'Har soll der Inbegriff solcher Frauen sein. 
Gerüchte zufolge.''\\
``Sie ist eine charmante, sehr hübsche junge Frau'', murrte Jozah verärgert und verteidigte die 
Ehre seiner inoffiziellen Verlobten: ``Wir haben einige anregende Gespräche geführt und schöne 
Nachmittage miteinander verbracht. Ach... was solls, ich gehe heute zu ihren Vater. Schlimmer als 
mich aus den Haus jagen und mich in den höheren militärischen Kreisen blamieren kann er ja nicht 
machen...''\\
Mishka leerte seine Schüssel. ``Wieso so eilig? Warum nicht erst noch ein paar weitere... anregende 
Gespräche?''\\
Jozah biss sich auf die Lippen. ``Weil wir morgen aufbrechen'', flüsterte er: ``Befehl vom König 
persönlich. Ruf die Truppe auf dem Hof zusammen, dann gebe ich euch die Befehle.''\\
Mishka schmiss seinen hölzernen Löffel auf den Tisch und rief mit abenteuerlustig funkelnden Augen: 
``Dein ernst, Jozah Mi'Kae? Du prahlst mit Weibergeschichten, noch ehe du mit einem Befehl direkt 
vom König herausrückst? So kennt man dich ja gar nicht!''\\
``Mach einfach was ich dir gesagt habe'', seufzte Jozah.\\

Er hielt die Ansprache an seine 50 Mann starke Truppe kurz und knapp, nannte den Grund ihrer Reise 
- die junge Witwe zu unterstützen - und dass sie am nächsten Morgen aufbrechen würden. Die 
notwendigen Aufgaben um die sich noch gekümmert werden musste, wie Proviant und das Überprüfen der 
Ausrüstung, verteilte er und gab ihnen ansonsten den Rest des Tages frei.\\
Es war später Vormittag, ehe er seine persönlichen Vorbereitungen für die Reise getroffen hatte und 
sich für die Aufgabe, die nun noch heute vor ihm lag, herausgeputzt hatte. Die Uniform war 
frisch ausgebürstet und die Manschetten poliert bis sie glänzten. Sein Pferd, ja, das klang 
wirklich gut... Sein Pferd war gestriegelt, Mähne und Schweif geflochten, als würde es gleich in 
einer Parade laufen. Das musste man den Stallburschen lassen, in dem herrichten der Pferden waren 
sie einmalig. Locker saß Jozah im Sattel, den Rücken gerade und den Kopf erhoben. Im Allgemeinen 
bevorzugte er es, übersehen zu werden, aber das war nicht vom Vorteil, wenn man einer schönen Frau 
den Hof machen wollte. Den Säbel hatte er nicht mitgenommen. Die Waffe lag wieder in der Kammer, 
dort wo sie hingehörte. Aber auch sein Schwert hatte er nicht mit. Es war einfach nichts für 
einen friedlichen Ausritt in Begleitung einer Dame.\\

 
