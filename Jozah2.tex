\chapter{Schnelle Versprechen}


Obwohl er erst zu später Stunde in den Schlaf gefunden hatte, erhob sich Jozah wie gewohnt zum 
ersten Morgenapell. Die Zeit als er noch Rekrut war und nun eilig hinaus auf den Hof musste um die 
ersten zehn Runden zu drehen und anschließend in der Wäscherei zu arbeiten, bevor es Frühstück gab, 
war Osyma sei Dank schon lange vorbei. Trotzdem steckte es ihm noch wie jedem anderen Soldaten im 
Blut. \textit{Vermutlich gibt es diese Tortur für die Knaben nur, damit alle anderen in Ruhe 
frühstücken können}, dachte er, während er seine Uniform richtete und nach seiner Rasierklinge 
griff.\\
Er hatte nicht gut geschlafen. Ihn bedrückte eine Mischung aus freudiger Erwartung, Tatendrang und 
Sorge, die ihn wach hielt. Er genoss es, endlich wieder etwas zu tun zu haben. Die Eintönigkeit der 
Kaserne konnte ein Seegen sein, wenn man sich nah Struktur und Halt sehnte. Jozah stritt auch 
keinesfalls ab, dass es ihm und vermutlich auch seinen Leuten gut getan hatte. Trotz Wachdiensten 
oder anderen Verpflichtungen war es doch so viel anders als der Dienst in den Kolonien. Man war 
unter sich, musste nicht mit plötzlichen Rebellionen oder Aufständen rechnen. Man musste nicht 
hinterfragen, wer einem vielleicht etwas böses wolle. Nein, man stand auf zum Morgenapell, ging 
seinen eingeteilten Diensten nach und pünktlich stand das Essen auf dem Tisch. \\
Drei Monate würden niemals genügen um sich von den Erlebnissen in den Kolonien zu erholen. 
Vielleicht nicht einmal drei Jahre. Also verhielt Jozah sich wie jeder andere und wie es von 
Soldaten erwartet wurde, er verdrängte. Das half meistens, auch wenn das Bedürfnis, darüber zu 
sprechen, immer wieder kam. Seine ehemaligen Kameraden waren nun seine Truppe... er konnte es sich 
nur vor wenigen Auserwählten leisten, offen über seine Gedanken zu Gefühle zu sprechen und auch das 
wollte er nicht unnötig herausfordern. Er war der Kommandant, er trug die Verantwortung. Damit ging 
einher, dass Gespräche verstummten, wenn er dazu stieß, dass Blicke getauscht wurden. Keiner seiner 
Leute wollte vor ihrem Anführer Schwäche zeigen und sie wollten ebenso wenig, dass er das anders 
handhabte. \\
Trotz allem vermisste Jozah die Kolonien. Die Weite der Landschaft... etwas, was sich kein 
einfacher Bürger Saleicas vorstellen konnte. Diese Grenzenlosigkeit. Nicht nur die Landschaft, auch 
die Menschen die sie belebten. Es schien, als müsste man nur wenige Tage in eine Richtung gehen um 
immer neue Kulturen zu erleben. Vor den Meisten sollte man sich in acht nehmen. Aber die 
Geschichten... unzählige Legenden und Lieder. Grenzenlos.\\
Im Speißesaal dann setzt er sich neben Mishka, der mit dem Kopf über der Schüssel Haferbrei hing.\\
``Gut geschlafen, Herr General?'', grunzte Mishka: ``Bist so auffallend spät für deine Maßstäbe.''\\
``Ich schätze du hast es schon allen erzählt?'', mutmaßte Jozah und nippte an seinem Becher voll 
Wasser.\\
Der pochende Kopfschmerz ließ ihn die letzten Becher von in der Nacht bereuen und wiedermal konnte 
er sich nur fragen, wie intensiv Mishka wohl schon den Umgang mit Alkohol geübt hatte, dass man ihm 
am nächsten morgen kaum etwas anmerkte.\\
``Nur unseren Leuten.''\\
Jozah blickte sich im Saal nach den vertrauten Gesichtern um. ``Deshalb seh ich jetzt auch keinen 
von ihnen?''\\
``Die Schlafen noch.''\\
``Sag ihnen, dass sie genug gefeiert haben, Morgen will ich sie alle zur Zeit des Appells draußen 
auf dem Hof sehen'', entschied Jozah.\\
``Die offizielle Ernennung ist doch aber erst heute!'', entrüstete sich der Blondschopf.\\
``Dann hättet ihr eben nicht früher feiern sollen.''\\
``Das waren doch nur ein paar wenige Stunden...'', rechtfertigte sein Kamerad sich.\\
Jozah ließ ihn nicht ausreden. ``Es hat seinen Grund, warum die Beförderung jetzt kam. Wir haben 
einem Auftrag vom König persönlich und werden sobald wie möglich aufbrechen. Richte der Truppe aus, 
sie sollen ihre Angelegenheiten regeln, es geht nach Merandila.''\\
``Und das erzählst du erst jetzt?'', rief Mishka überrascht aus.\\
``Interessant'', murmelte es hinter ihnen.\\
Jozah rutschte kauend näher zu Mishka um Elor Platz zu schaffen. Die beiden sollten lieber nicht 
nebeneinander sitzen.\\
``Glückwunsch'', fügte Elor hinzu und nickte seinem General zu.\\
Der Mann war gute drei Jahre jünger als er selbst und wenn man ihn neben Mishka stellte, ungefähr 
nur ein Drittel von dessen Körpermasse. Dunkles, schwarzes Haar und ebenso dunkle, berechnende 
Augen. Elor war eher drahtig als muskulös, sprach meist wenig und lachen hörte man ihn noch 
seltener. In seinen Adern floss das Blut des Südens, wie man es höflich bezeichnete. Die 
respektlosere Variante wäre es, ihn Sklavensohn zu nennen. Die Sklaverei war offiziell bereits seit 
zwei Jahrzehnten abgeschafft worden. Ein vergeblicher Versuch König Kareens Friede mit den Rebellen 
der südlichen Kolonien zu schließen. Leichtsinnigerweise hatte er keinen Finger gekrümmt, um 
aufzupassen, ob dieses Gesetz auch die saleicanischem Adeligen in den Kolonien beachten.\\
Er war noch ein Knabe, als er sich den saleicanischen Eroberern anschloss und bereits vor Jozah 
dort im Krieg stationiert. Es hatte nie einen Grund gegeben Elor zu misstrauen. Trotzdem hatte er 
nicht mit nach Saleica kommen sollen. Soweit Jozah wusste, gab es nur eine Handvoll Ausländer in 
Brom-Dallar. Am Tag als die Schiffe in die Heimat segeln sollten, standen noch die Priester am 
Deck und zeterten, dass dies nicht erlaubt sei. Die Missionare hatten sich mit der falschen Truppe 
angelegt. Sie hatten zusammen das Chaos überlebt und Elor war dabei an ihrer Seite gewesen. Obwohl 
Mishka sonst eher lautstark verkündete, für was für einen dreckigen Hund er den Ausländer hielt, war 
er damals der Erste gewesen, der den Priestern gegenüber getreten war. Am Ende landeten drei von 
ihnen im Wasser. Einer war freiwillig gesprungen, dem Anderen hat Mishka das Vergnügen bereitet. Ein 
Dritter ist noch hinterher geplumbst, als er seine Glaubensgefährten hatte herausfischen wollen.\\
``Ich habe eine Nachricht von deiner Liebsten'', erklärte der Dunkelhaarige: ``Deine Ernennung 
beginnt heute nach der Entzündung des Nachmittagsfeuers.''\\
Jozah verschluckte sich und hustete. ``Liebsten?''\\
Seine beiden Freunde tauschten Blicke. ``Nicht?'', fragte Mishka: ``Also, wenn das so ist, nehm ich 
sie mir.''\\
``Untersteh dich!'', knurrte Jozah und wischte sich mit dem Ärmel über den Mund.\\
``Warum dann so überrascht? Ihr trefft euch doch seit Wochen'', stellte Elor fest und ließ eine 
Portion Brei von seinem Löffel zurück in die Schüssel tropfen. ``Ich vermisse das Essen'', seufzte 
er leise und verzog das Gesicht.\\
``Eure Rüben sind so widerlich wie eure Weiber '', entschied Mishka und griff nach Elors Schüssel: 
``Aber wie du willst, ich erlöse dich vor dieser Aufgabe.''\\

Vito Ma'sah hatte es sich nicht nehmen lassen, seine privaten finanziellen Mittel bei der 
Festlichkeit zu Jozahs Ernennung zum General einfließen zu lassen. Üblicherweise stellte die 
Familie des Ehrengastes diese. Oder aber, in Falle von mittellosen Soldaten, lief sie recht 
unspektakulär ab. Eine Urkunde des Königs, den Segen eines Priesters und noch die Überreichung 
eines symbolischen Geschenks in Form einer Klinge oder eines Pferdes und die Sache war erledigt.\\
Mit gemischten Gefühlen, was ihn wohl am heutigen Abend und in naher Zukunft erwarten würde, ließ 
Jozah seinen Schimmel durch die Straßen der Nobelviertel traben. Der rhythmische Hufschlag hallte 
in der vorabendlichen Stille wieder. Noch war die Stunde der Dämmerung, in der die adeligen Söhne 
und Töchter ihre Zeit damit verbrachten, sich für die geselligen Stunden in der Stadt herzurichten 
und den Ermahnungen ihrer Eltern kein Gehör schenkten.\\
Jozah hatte den Offizier Lerin bei Wort genommen und sich für den Schimmel entschieden. Der Wallach 
war bereits seit den Kolonien an seiner Seite. Ein treues, gehorsames Tier, welches selten scheute 
und ihn bereits mit einem Wiehern von der Weide aus grüßte.\\
Er zügelte sein Reittier vor den Stufen der Stadtvilla und glitt aus dem Sattel. Auch wenn er sich 
selten eitel zeigte was sein Äußeres anging, so trug Jozah heute seine neusten Lederstiefel, die 
sorgfältig ausgebürstete, neue Uniform eines Generals und hatte sich das Haar schneiden lassen. 
Immerhin war es sein erster und vorerst letzter Abend, den er in der Gesellschaft hochrangiger 
Militärs in der Hauptstadt verbringen würde. Er wollte seinem Mentor, dem er so viel zu verdanken 
hatte, auf keinen Fall Schande bringen.\\
Ein Bediensteter ergriff die Zügel mit einem Nicken und führte den Schimmel in den angrenzenden 
kleinen Stall. Jozah sah ihm hinterher, dann wanderte sein Blick zum Eingangsportal. Licht 
schimmerte durch den Türspalt hervor. Er straffte sich ein letztes Mal und trat dann die wenigen 
Stufen empor. Vor dieser Tür musste er nicht lange nachdenken, ob er klopfte oder eintrat. Vito 
Ma'Sah hatte ihm schon mehr als einmal zu verstehen gegeben, dass er ihn als Sohn sah. Außerdem war 
er der Ehrengast. Schwungvoll stieß er die Tür auf und trat mit einem breiten Lächeln auf den 
Gesicht ein. Ilia empfing - ganz die vorbildliche Dame des Hauses - eben noch die kurz vor ihm 
eingetretenen Gäste herzlich. Jozah hielt inne und beobachtete sie dabei. Ihr Blick huschte flüchtig 
über ihn hinweg, ehe sie sich wieder ganz auf das Paar vor ihr konzentrierte. Nichts in ihrer 
Haltung oder Mimik veränderte sich.\\
\textit{Leidenschaftliche Tänzerin oder berechnende Spielerin?}, grübelte Jozah amüsiert. Ilia war 
so viel und doch zeigte sie oft nur eine Seite ihres Charakters. Während sie am Abend zuvor noch 
das Mädchen mit viel zu viel Geld war, welches sich nur eine spannende Nacht gönnen wollte, so 
zeigte sie hier ihre Redegewandtheit. Jozah war sie immer noch nicht sicher, ob er bereits jemals 
hinter ihre zahlreichen Masken geblickt hatte. Er kannte sie schon seit ihrer Jugend und was das 
anging, hatte sie sich kaum verändert.\\ 
``Jozah'', begrüßte sie ihn schließlich und ergriff seine Hände. Sie hauchte ihm einen Kuss auf die 
Wange. ``Mein Vater ist aufgeregt wie ein kleines Kind. Er prahlt schon die ganze Zeit mit dir. Von 
Geschichten deiner Ausbildung bis hin zu deinen Heldentaten in den Kolonien!''\\
``Er hat viel zu viel Geld ausgegeben. Und wer sind diese ganzen Leute?'', flüsterte Jozah.\\
``Wenn es danach geht, besitzt er viel zu viel Geld. Und ein Großteil sind Freunde aus seiner 
aktiven Zeit beim Militär. Nutze lieber die Gelegenheit, neue Beziehungen zu knüpfen. Er wirft dir 
doch schon seit Jahren vor, dass du dem nicht genug Beachtung schenkst. Sonst wärst du schon viel 
früher zum Kommandanten befördert worden.''\\
Jozah lächelte als Antwort auf ihre Worte und verkniff sich die Aussage, dass dem ganz gewiss nicht 
so war. Er war trotz seinem adeligen Mentor nur der Sohn eines Handwerkers.\\
``Komm. Wir sollten niemanden länger warten lassen.'' Ilia zog ihn an einer Hand hinter sich her in 
den großzügigen Salon. Gut Eindutzend Gäste waren versammelt. Einige Bedienstete huschten 
unauffällig umher und servierten Getränke oder Speisen, welche auf einem Tisch entlang der Wand 
gerichtet waren. Davon gegenüber nahm der offene Kamin viel Platz ein. Kerzen und Lampen erhellten 
den Raum. Ein Musiker spielte auf einer Violine. Die gemütlich aussehenden Sitzgelegenheiten waren 
bereits belegt und die übrigen Gäste hatten sich zu kleinen Gruppen zusammengefunden und 
plauderten.\\
Kaum hatte Vito Ma'Sah ihn entdeckt, rief er zur Ruhe. Man traute der Gestalt des älteren Herrn so 
eine herrische, direkte Stimme auf den ersten Blick nicht zu. Aber immerhin war dies der 
Mann, der in seiner Jugend eine der Seeschlachten an der Monari-Küste für Saleica entschieden 
hatte. Jozah grauste schon die Vorstellung an einen Kampf auf dem Wasser und konnte nur Osyma 
danken, dass ihm das bisher erspart geblieben war. Er gehörte zwar nicht zu denen, die sofort über 
der Reling hingen um den Mageninhalt zu entleeren, aber den sicheren Stand für einen Kampf traute 
er sich auch nicht zu.\\
``Da ist er!'', verkündete Vito Ma'Sah: ``General Mi'Kae!''\\
Er klopfte ihm kameradschaftlich auf die Schulter und hob sein mit rotem Wein gefülltes Glas. ``Ich 
sah ihn damals auf dem Platz. Ein dünner Knabe der mich eher an einen Hund mit eingezogenen Schwanz 
erinnerte, als an einen Soldaten! Aber ich sag euch, kam man ihn zu nahe, biss der Köter plötzlich 
zu.''\\
Vito Ma'Sah nahm einen Schluck und lachte leise in sich hinein, ehe er mit seiner Rede fort fuhr: 
``Und Heute ist er fast schon ein passabler Löwe geworden! Nein nein, ich scherze ja nur. Jozah 
Mi'Kae ist ein pflichtbewusster Soldat, der im Namen der Krone und des Allmächtigen sein Leben in 
den Kolonien aufs Spiel setzte und siegreich heimkehrte! Er verdiente sich dort die Beförderung zu 
Kommandanten und führte seine Leute durch zahlreiche Schlachten. Und jetzt ist es so weit, dass 
unser geschätzter König Semric ihn zum General ernennt!''\\
Der Offizier überreichte ihm ein zusammengefaltetes Pergament mit dem goldenen Siegel des 
Königshauses. Den brüllenden, gekrönten Löwenkopf.\\
Jozah räusperte sich. ``Vielen Dank...''\\
Ein Priester trat zwischen den Gästen hervor und setzte dazu an, die feierlichen Worte der Segnung 
zu sprechen, aber Vito Ma'Sah winkte grob ab und packte Jozah ein weiteres Mal an der Schulter. 
``Später! Ich möchte noch etwas wichtiges und persönliches hinzufügen. General Mi'Kae war mir ein 
Sohn. Ich finanzierte seine Ausbildung. Ich schimpfe über seine faulen Vorgesetzten. Ich hieß ihn 
willkommen, als er aus den Kolonien heimkehrte. In den letzten Monaten verbrachte er viel Zeit in 
der Gesellschaft meiner reizenden Tochter Ilia und ich kam zu den Entschluss, dass es nun so weit 
ist, ihn vollends als Sohn und Erben in mein Haus aufzunehmen. Durch eine Verlobung mit meiner 
Tochter.''\\
Die Gäste applaudierten höflich und tauschten teils amüsierte, teils vielsagende Blicke. Es kam 
selten vor, dass jemand mit einem solchen Rang wie Vito Ma'Sah sich einen Schützling aus ärmlichen 
Verhältnissen holte. Ihn dann auch noch mit der einzigen Erbin des Familiennamens zu verloben war 
vermutlich eine Prämiere in der Hauptstadt Saleicas. Jozah starrte seinen Mentor sprachlos an. Er 
hatte in den letzten Wochen tatsächlich darüber nachgedacht, den alten Offizier unverfänglich zu 
fragen, wie er zu einer Heirat stehen würde. Ilia faszinierte Jozah und er war in dem Alter, in dem 
er sich langsam mal eine Frau suchen sollte. Bisher war sie die Einzige, die je auf Dauer sein 
Interesse gehalten hatte. Andererseits war Ilia auch in vielen Charakterzügen ihm gegensätzlich.\\
Und während Jozah noch seinen Mentor überrascht ansah, entging ihm der flüchtige Moment, in dem 
Ilias Maske zerbrach. Das Lächeln gefror. Der Blick wurde scharf und kalt wie Eis. Ihre Hand 
zitterte vor unterdrückten Gefühlen, als sie sich durch das blonde Haar strich. Sie lachte hell, 
eilte mit schwingenden Schritten an Jozahs Seite und umfasste seine Hüfte. ``Wie jede Frau malte 
ich mir bisher die romantischsten Anträge aus'', sagte sie zu den Gästen gewandt: ``Aber dass er 
schließlich von meinem Vater kommen würde, habe ich nicht erwartet!''\\
Die Männer und Frauen reagieren mit heiterem Gelächter auf den Witz und die Lage entspannte sich 
wieder. Im Gegensatz zu Ilias Griff. ``Ich denke, unsere geschätzten Gäste können verstehen, wenn 
ich ihnen einen Moment den Ehrengast stehle. Und den Gastgeber'', fügte sie hinzu und wandte sich 
abrupt um.\\
Sie wusste natürlich, warum ihr Vater das getan hatte. Niemals hätte Ilia vor adeligen Gästen 
abgelehnt. Und auch nicht in Jozahs Gegenwart, während er überrascht und stumm dabei stand.\\
``Habt ihr das geplant?'', fragte sie direkt, als die Tür hinter ihnen ins Schloss gefallen war.\\
``Nein'', beeilte Jozah sich zu sagen und blickte irritiert zu Vito Ma'Sah.\\
Dieser verzog nur das Gesicht und erwiderte: ``Wir haben oft genug darüber geredet, Ilia. Es wird 
Zeit für dich zu heiraten und den Familiennamen weiter zu tragen. Jozah ist der richtige. Ein 
ehrbarer Mann, steht in der Gunst des Königs und du hast in den letzten Wochen viel Zeit mit ihm 
verbracht. Es könnte keinen besseren Gatten für dich geben. Geld und Adelstitel bringst du selbst 
mit in die Verbindung. Oder hast du etwas an ihm auszusetzen?''\\
Ilia kniff die Lippen zusammen und lächelte Jozah kurz darauf an. ``Nein. Aber ich hatte es mir 
anders vorgestellt. Weniger überraschend. Weniger Zuschauer. Mehr Ehrlichkeit und echte Gefühle. 
Hattest du es vor, Jozah?''\\
General Mi'Kae räusperte sich. ``Nun... schon. Aber dann kam die Ernennung dazwischen und der 
Auftrag des Königs.''\\
``Auftrag des Königs?'', wiederholten Ilia und ihr Vater gleichzeitig.\\
Jozah zupfte sich nervös die Uniform glatt. ``Ich reite morgen schon nach Merandila. Um die junge 
Witwe bei den Kriegsvorbereitungen zu unterstützen.''\\
``Morgen schon'', sagte Ilia und warf ihrem Vater einen finsteren Blick zu: ``Du verlobst mich mit 
einem Soldaten, der gleich am nächsten Morgen in den Krieg zieht? Was stellst du dir vor, dass ich 
das Leben einer Witwe führe, obwohl wir noch nicht einmal geheiratet haben?''\\
``Dann geh mit nach Merandila'', entschied ihr Vater.\\
Ilia lachte laut auf. ``Nein. Da soll es so kalt sein, dass selbst die Hunde sich nicht vor die Tür 
wagen. Vergesst es.''\\
``Entschuldige Ilia'', murmelte Jozah: ``Das ist mir sehr peinlich und ich wollte dir nicht zu nahe 
treten...''\\
``Oh, deine Schuld ist es nicht'', entschied sie: ``Ich... habe es nur nicht erwartet.''\\
``Du sagst also nicht nein?'', fragte Jozah zögernd.\\
Ilia beugte sich vor und küsste ihn ein weiteres Mal auf die Wange. ``Nein. Aber verschiebe deinen 
Aufbruch noch um einen Tag. Ich will mich wenigstens kurz wie eine richtige Verlobte fühlen, 
ehe du reitest.''\\




 
