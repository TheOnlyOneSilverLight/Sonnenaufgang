\chapter{Schnelle Versprechen}


Der Halbmond beschien die Straßen der Hauptstadt, hüllte sie in ein gespenstisches Licht. Aber die 
Nacht war keineswegs still. Besonders nicht in der Nähe der Kaserne. Hin und wieder hörte man das 
lockende Kichern einer Hure, auf der Suche nach in der Innenstadt, die er gerade durchquerte. Die 
Mitte der Hauptstadt war fast schon als luftig zu bezeichnen. Breite, gepflasterte Straßen die 
immer wieder von ebenen Flächen abgelöst würden. Markt- oder Schauplätze die genügend Raum für die 
freizeitlichen Aktivitäten des Adels erübrigten. Entfernte man sich von der Stadtmitte, wurden die 
Häuser höher und schmaler, die Gassen enger, die Nacht stiller. Ein Volk, welches solch ein 
Geltungsbedürfnis hatte wie die Saleicaner, brauchten genügend Raum um sich auszuleben. Zumindest 
die Saleicaner mit Geld. Wer sich hier ein Anwesen leisten konnte, hatte entweder einen uralten 
Namen und es im Besitz seit der Gründung der Stadt, oder zu viel Geld um es auszugeben. Auch viele 
Soldaten der nahe gelegenen Kaserne gaben hier bei den regelmäßigen Feierlichkeiten und 
Wettbewerben ihren Sold aus. 
Er hielt den Blick gesenkt, saß auf den Weg und grübelte nach. Jozah war kein typischer Soldat der 
saleicanischen Armee. Männer und Frauen Saleicas waren weither bekannt als Draufgänger, Schläger, 
Menschen die lieber zur Waffe griffen als den Mund auf zumachen. Mut, nannten sie es, und Ehre, 
wenn man blindlings in einen Kampf stürzte, täglich unsinnige Mutproben ausfocht und grölend 
Lobeshymnen auf den König, das Land und Osyma sang. So war Jozah nicht. Weder Streben nach Ehre 
noch Kampfeswut waren es, weshalb er jetzt hier stand. Er ging zur Armee, weil er keinen anderen 
Ausweg sah. Das hatte sich bis heute nicht geändert. Nichts anderes konnte er sich vorstellen. Was 
wäre ihm als Sohn eines Handwerkers für ein Leben beschieden gewesen? Entweder angestellt in der 
Werkstatt, die ein älterer Bruder erben würde, oder bei irgendjemand anderem, der vermutlich noch 
weniger für ihn übrig haben würde. \\
Musik hallte durch die Nacht. Gelächter und das Kichern von Frauen. Einige fleißige Händler, die 
selbst zu dieser späten Stunde ihre Waren anpriesen und damit bei dem voll trunkenem Adel auch gute 
Chancen hatten.\\
Widerwillig musste Jozah sich jedoch auch eingestehen, dass er seit seiner Rückkehr aus den 
Kolonien das Nachtleben der Stadt genoss. Die temperamentvolle Musik, erzeugt von Trommeln, Flöten 
und Violinen, übertönte die Stimmen der Sterbenden, die ihn viel zu oft verfolgten. Der Wein und 
das süße Essen verdrängte die Erinnerung an den Schmerz des Hungers und die Gewissheit, elend zu 
verhungern, während die Belagerer einen saftigen Braten speisten. Die heiteren Lieder ließen die 
verzweifelten Gebete verklingen.\\
\textit{Ob es Osyma wohl gefallen hat, uns beim sterben zuzusehen?}\\
Jozah hatte damals fast mit seinem Leben abgeschlossen. Er hatte den Dolch schon in seiner Hand um 
die größte Sünde zu begehen. Es war nicht die Angst gewesen, was wohl Osyma mit ihm machen würde, 
wenn er sich wirklich selbst das Leben nahm, warum er nun noch hier stand. Nein, nicht mal um sich 
die Kehle aufzuschlitzen war er mutig genug. Jozah verzog das Gesicht und spuckte auf die Straße um 
den schalen Geschmack loszuwerden.\\
Aber dies war auch ein Wendepunkt in seinem Glauben gewesen. War es wirklich Tapferkeit, die die 
saleicanischen Soldaten so tödlich machte? War es wirklich Mut gewesen, der die Männer und Frauen 
dazu gebracht hat, immer und immer wieder einen Ausbruch aus der belagerten Feste zu versuchen? Nur 
um mit Pfeilen beschossen zu werden oder in eine Grubenfalle zu geraten? Mittlerweile vermutete 
Jozah eher, dass es ihre Art war, dem nahendem Hungertod zu entkommen. \textit{Egal. In den 
Geschichtsbüchern wird stehen, dass sie alles gegeben haben um Osymas Flamme zu verbreiten. Dass 
sie kämpften wie Löwen und starben wie Löwen. Sie brüllten den Namen des Allmächtigen im Tod. Ist 
besser als ihn zu röcheln, während man verhungert.}\\
Jozah verzog das Gesicht in der Erinnerung daran und schüttelte den Kopf, aber die Bilder ließen 
ihn nicht los. Sein damaliger Kommandant konnte sich verletzt noch zurück in die Feste schleppen, 
erlag vier Tage später jedoch seiner Verwundung. Und dann begann das Chaos. Die letzten Vorräte 
wurden noch innerhalb einer Stunde verzehrt. Die Soldaten kämpften um den letzten Krümel 
verschimmelten Brots. Aus purem Wahnsinn verunreinigten sie den Brunnen mit ihrem Urin.\\
Es gab nur einen Grund, wieso die Rebellen die hölzerne Feste nicht längst abgebrannt haben. Die 
Gegend war zu trocken und sie wollten keinen Flächenbrand riskieren. Doch in diesen Stunden des 
Chaos kam irgendeiner der Soldaten auf die Idee. Erst war es nur Geflüstert. Dann wurden die Rufe 
immer lauter. Die Männer brüllten animalisch, wie sie wohl dachten, dass Löwen brüllen würden. 
Osymas Name wurde geschrien, seine Flammen vergöttert. \textit{Das letzte Gebet.}, erinnerte Jozah 
sich an den Namen, mit dem dieses Ereignis in die Geschichtsbücher eingetragen wurde. \textit{Das 
letzte Betteln trifft es besser. Das letzte Flehen, dass ein stolzer Gott sich nicht abwenden 
würde, auch wenn seine Streiter verloren haben.}\\
Nicht jeder war verzweifelt genug, diesem Aufschrei zu folgen. Zu widersprechen wagte jedoch 
niemand, zu schnell drang saleicanischer Stahl in die, die es versuchten. Es war Jozahs Idee 
gewesen, sich in die Gruft zurück zu ziehen und die Türen zu verrammeln. Elor, damals schon ein 
Freund, der still an seiner Seite stand, war dabei. Und noch zwanzig weitere Männer und Frauen. 
Jozah hatte Mishka noch tobend mit geschliffen. Stur wie er war, wollte er sterben wie es für einen 
Löwen würdig war. Aber auch sein Trommeln gegen die Wände hatte schließlich nachgelassen, als die 
Hitze selbst in der Gruft spürbar wurde und die Schreie erklangen. Die Dunkelheit hatte seltsamen 
Frieden gespendet. Erst als die Stille begann und auch nicht mehr aufhörte, hatten die verbliebenen 
Saleicaner die Steine wieder beiseite geschaufelt und sind aus der Gruft gestiegen. Völlig 
abgemagert, verdreckt und vom Chaos gezeichnet, aber lebend waren sie über das Feld aus Asche 
gestolpert und schon bald einer Truppe begegnet, die ihnen hatte zur Unterstützung kommen wollen.\\
Die ganze Geschichte endete damit, dass die Überlebenden Jozah für seine Idee sich zu verstecken 
feierten und vor den Vorgesetzten in den Garnisonen nur Lob von sich gaben. Es wurde so 
dargestellt, dass er das Feuer befohlen hatte um das Heer der Feinde zu vernichten und ihr Land für 
die Torheit zu bestrafen. Dem Chaos, dem Wahnsinn und den saleicanischen Feuertoten wurden keine 
weitere Beachtung geschenkt. Ehe er sich versah, war er zum Kommandanten befördert und hatte eine 
Truppe zugeteilt bekommen, die zum größten Teil aus den Zwanzig überlebenden bestand.\\
\textit{Und nun General? Diesmal war's aber zu einfach.}, dachte er bitter.\\
Ein großgewachsener Mann kam auf die Straße getaumelt, am Arm eine hübsche Frau. Mishkas blondes 
Haar stand zerzaust in alle Richtungen, seine Kleidung war verrutscht und zerknittert. Ein seeliges 
Grinsen im Gesicht und der Geruch nach Alkohol begleitete ihn. Er hatte sich nicht verändert. 
Mishka war immer noch der verwöhnte vierte Sohn eines Adeligen, der mit dem Geld seiner Familie um 
sich warf. Soweit Jozah auf dem neusten Stand war, hatte aber seine älteste Schwester nun den Titel 
des Familienoberhaupts übernommen und schon einige ernste Gespräche mit Mishka hinter sich. Viel 
gebracht hat es nicht. Trotzdem wusste Jozah, dass das alles nur ein Versuch war, das Chaos 
zu vergessen. Mishka hatte es nicht ausgesprochen, aber er verdankte Jozah sein Leben und das war 
ihm deutlich bewusst.\\
``Heeey'', rief der Bondschopf aus: ``Hast gar nich erzählt, dass du auch noch raus wolltest! Wo 
warst'n den ganzen Nachmittag?''\\
``Jozah'', sprach auch seine Begleitung und löste sich aus der Berührung, um ihm um den Hals zu 
fallen. ``Und ich dachte schon, ich finde dich heute nicht mehr.''\\
Er schloss einen Moment die Augen und genoss die Berührung der Frau, während er den Duft ihres 
Parfüms einatmete. ``Ilia'', murmelte er: ``Schön dich zu sehen.''\\
Ilia Ma'Sah war eine dieser Frauen, die alle Aufmerksamkeit bekam, wenn sie einen Raum betrat. Die 
Männer folgten ihr mit Blicken und verstummten, selbst wenn sie gerade dabei waren, lautstark 
Wetteinsätze zu verhandeln. Sie war der Inbegriff einer saleicanischen Schönheit, fand Jozah. 
Blondes Haar, ansehnliche Rundungen und ein Temperament, vor dem einige Männer lieber Abstand 
hielten. Ihr Name war so alt wie Brom-Dalar, gehörte ihre Familie doch zu den Gründern der Stadt. 
Einige ihrer Ahnen haben wohl auch in die Königsfamilie eingeheiratet. Ihr Vater war ein General, 
der gleich mehrere Geschichtsbücher füllte und einst im Rat König Kareens saß, ehe er in Ruhestand 
ging. Er war freiwillig abgetreten, noch ehe die Priester Prinz Semric krönten. Nebenbei hatte er 
sein Vermögen vermehrt, in dem er in den Handel einstieg. Und Ilia war die einzige Erbin dieses 
Vermögens, dieses Namens. Und trotz ihrem Alter von 25 Sommer noch nicht verheiratet. Sie war zu 
frech, um einen Mann bisher lange genug zu halten. Und hatte wohl bisher auch noch nicht die 
Absicht gehabt, das Versprechen zu geben. Sie verbrachte ihre Nächte tanzend in der Stadt, die Tage 
bei Jagden, Fechten oder Plaudereien. Anfangs sah Jozah nur das und war entsprechend überrascht, 
als er entdeckte, dass Ilia noch vieles mehr war. Sie sprach Kasirisch, sowie zwei Sprachen aus den 
Kolonien. Las gerne in besagten Sprachen Bücher und Berichte und hatte eine Vorliebe für Schach.\\
Und dazu kam das Argument, dass sie die Tochter seines ehemaligen Mentors war. Ohne Vito Ma'Sah 
hätte Jozah die Zeit als Rekrut wohl nicht überstanden. Vor allem die anderen - teilweise adeligen 
- Rekruten hatten ihm das Leben schwer gemacht und auch einige der Ausbilder hatten ihn loswerden 
wollen. Einer der harmloseren Versuche war noch gewesen, ihn in die Kaserne irgendeines kleinen 
Kaffs an der Westküste zu versetzen. Natürlich wurden Soldaten in ganz Saleica geschätzt, aber in 
der Hauptstadt zählt der Familienname viel zu oft mehr als die Fähigkeiten.\\
``Wie geht es deinem Vater?'', erkundigte er sich höflich.\\
``Ach... wenn es nach ihm geht, soll ich in unser Strandhaus ziehen und sticken oder solche 
Dinge.''\\
``Er sorgt sich nur um dich'', erwiderte Jozah grinsend und küsste sie auf die Wange.\\
Ilia sah ihn einen Moment eindringlich an, wandte sich dann lachend um und deutete auf einen der 
Verkaufsstände. ``Ich will einen Gewürzwein. Darf ich dich einladen, Herr Kommandant?''\\
``General'', verbesserte Jozah.\\
Sein Freund und Ilia sahen ihn überrascht und fragend an. ``Offizier Lerin hat mich befördert.''\\
``War die Ernennung bereits? Und du hast mich nicht eingeladen?'', fragte Ilia entrüstet.\\
Jozah beeilte sich zu sagen: ``Nein, noch nicht.''\\
``Noch besser. Wenn mein Vater davon hört, wird er darauf bestehen, die Ernennung für dich 
abzuhalten! Oh, ich brauche noch ein neues Kleid. Und du eine annehmbarere Uniform!''\\
``Wen interessieren Stoffe, dass müssen wir feiern!'', entschied Mishka und hob seinen leeren 
Becher, legte den Kopf in den Nacken und versuchte noch einen letzten Tropfen mit der Zunge 
aufzufangen.\\


Seine Kammer war wie alle anderen in der Kaserne geschnitten. Klein, quadratisch und nur ein 
winziges Fenster ließ frische Luft herein. Rechts an der Wand befand sich ein schmales Bett mit 
einer strohgefütterten Matratze. Ein Tisch, zwei Stühle. Eine Truhe für Kleidung und persönliche 
Gegenstände. Die Wände waren grau, geziert von dem ein oder anderen Schmutzfleck. Die meisten 
Personen mit höheren militärischen Rängen bevorzugten, in ihren Wohnungen oder Häusern außerhalb 
der Kaserne zu wohnen. Es gab noch das ein oder andere komfortablere Zimmer für die Ausbilder, der 
Rest sah so aus wie Jozahs. Er hätte auf eine Wohnung sparen können. Das wäre vielleicht ein Jahr 
gewesen und schon hätte er sich eine gemütliche, nicht all zu exklusive leisten können. Aber er war 
immerhin erst seid wenigen Monaten hier, konnte sich nicht von dem Ort trennen, den er immer als 
Heimat betrachtet hatte. Er hatte sich als Kommandant lediglich den Luxus gegönnt, das Zimmer 
allein zu bewohnen. Die ranglosen Soldaten teilten sich den ähnlichen Raum zu viert. \\
Jozah kniete sich vor die Truhe und öffnete das Schloss. Mit einem leisen Klick sprang es förmlich 
in seine Hand und der schwere Holzdeckel ließ sich anheben. Vorsichtig, fast ehrfurchtsvoll, nahm 
er den in ein schmutziges Tuch gewickeltes Laken heraus. Er legte den Gegenstand auf den Tisch und 
setzte sich auf den Hocker davor. Lange betrachtete Jozah es, ohne das Tuch beiseite zu schlagen. 
Es war wie ein Geheimnis. Sein eigenes, daher brauchte er es nicht zu lüften. Dies war bisher der 
einzige Gegenstand, der ihm allein gehörte. Und diesen hatte er nicht geschenkt bekommen, sondern 
sich erkämpft und verdient. Das hob seinen Wert gleich an. Jozah seufzte und schlug das Tuch zurück. 
Es war ein schlichtes Schwert, was vor ihm lag. Ein Bastardschwert, wie man sie so nett nannte. 
Jozah brauchte es nicht zu schwingen um zu wissen, das es im Gleichgewicht war und gut in seine 
Hand passte. Ein Schwerte ohne Zierde, aber einem metallenen Schimmern welches einem Versprechen 
nahe kam. Ein Schwur auf seine eigene Schärfe und Tödlichkeit. \\
Sein ursprüngliches Schwert, welches er als einfacher Soldat getragen und dem Land gehört 
hatte, war zerbrochen, als er damit die Axt eines Rebellen aufhalten wollte, die einem seiner 
Kameraden bedrohte. Am Ende der Schlacht hatte er in dem Feld voller Toten nach einer brauchbaren 
Ersatzwaffe gesucht und ein halbes Dutzend mit zum Schmied genommen. Keine von ihnen war sonderlich 
gepflegt und keine weckte den Anschein, dass sie nicht ebenfalls bei der nächst besten Gelegenheit 
brechen würde. Der Kamerad, den Jozah gerettet hatte, aber dafür in der nächsten Schlacht starb, 
bestand darauf für ihn den Schied, welcher nicht dem Militär angehörte, zu bezahlen. So war sein 
jetziges Schwert entstanden. Das Metall von den gesammelten Klingen wurde eingeschmolzen, möglichst 
gehärtet und geformt. \\
So viele Schicksale ruhten also in dieser Klinge. So viele Kriegerleben. Jeder hatte eine eigene 
Geschichte mit einfließen lassen. Manchmal kam es Jozah so vor, als würden die Stimmen der 
Verstorbenen flüsternd erzählen. Worte in einer Sprache, die er nicht verstand, aber so voller 
Gefühl und Vorhersehung, dass es ihm Gänsehaut bereitete.\\
Jozah deckte das Tuch wieder darüber, verstaute die Waffe aber nicht. Er musste sie wieder 
gründlich schärfen, denn seid er wieder in der Hauptstadt war, hatte er sie kaum angerührt. 
Stattdessen hatte er versucht vor diesen Stimmen zu fliehen. \textit{Sie werden mich nie mehr 
loslassen}, stellte er schließlich fest und legte sich auf seine Matratze um den Rausch 
auszuschlafen.\\

Obwohl er erst zu später Stunde in den Schlaf gefunden hatte, erhob sich Jozah wie gewohnt zum 
ersten Morgenapell. Die Zeit als er noch Rekrut war und nun eilig hinaus auf den Hof musste um die 
ersten zehn Runden zu drehen und anschließend in der Wäscherei zu arbeiten, bevor es Frühstück gab, 
war Osyma sei Dank schon lange vorbei. Trotzdem steckte es ihm noch wie jedem anderen Soldaten im 
Blut. \textit{Vermutlich gibt es diese Tortur für die Knaben nur, damit alle anderen in Ruhe 
frühstücken können}, dachte er, während er seine Uniform richtete und nach seiner Rasierklinge 
griff.\\
Im Speißesaal dann setzt er sich neben Mishka, der mit dem Kopf über der Schüssel Haferbrei hing.\\
``Gut geschlafen, Herr General?'', grunzte Mishka: ``Bist so auffallend spät für deine Maßstäbe.''\\
``Ich schätze du hast es schon allen erzählt?'', mutmaßte Jozah und nippte an seinem Becher voll 
Wasser.\\
Der pochende Kopfschmerz ließ ihn die letzten Becher von in der Nacht bereuen und wiedermal konnte 
er sich nur fragen, wie intensiv Mishka wohl schon den Umgang mit Alkohol geübt hatte, dass man ihm 
am nächsten morgen kaum etwas anmerkte.\\
``Nur unseren Leuten.''\\
Jozah blickte sich im Saal nach den vertrauten Gesichtern um. ``Deshalb seh ich jetzt auch keinen 
von ihnen?''\\
``Die Schlafen noch.''\\
``Sag ihnen, dass sie genug gefeiert haben, Morgen will ich sie alle zur Zeit des Appells draußen 
auf dem Hof sehen'', entschied Jozah.\\
``Die offizielle Ernennung ist doch aber erst heute!'', entrüstete sich der Blondschopf.\\
``Dann hättet ihr eben nicht früher feiern sollen.''\\
``Das waren doch nur ein paar wenige Stunden...'', rechtfertigte sein Kamerad sich.\\
Jozah ließ ihn nicht ausreden. ``Es hat seinen Grund, warum die Beförderung jetzt kam. Wir haben 
einem Auftrag vom König persönlich und werden sobald wie möglich aufbrechen. Richte der Truppe aus, 
sie sollen ihre Angelegenheiten regeln, es geht nach Merandila.''\\
``Und das erzählst du erst jetzt?'', rief Mishka überrascht aus.\\
``Interessant'', murmelte es hinter ihnen.\\
Jozah rutschte kauend näher zu Mishka um Elor Platz zu schaffen. Die beiden sollten lieber nicht 
nebeneinander sitzen.\\
``Glückwunsch'', fügte Elor hinzu und nickte seinem General zu.\\
Der Mann war gute drei Jahre jünger als er selbst und wenn man ihn neben Mishka stellte, ungefähr 
nur ein Drittel von dessen Körpermasse. Dunkles, schwarzes Haar und ebenso dunkle, berechnende 
Augen. Elor war eher drahtig als muskulös, sprach meist wenig und lachen hörte man ihn noch 
seltener. In seinen Adern floss das Blut des Südens, wie man es höflich bezeichnete. Die 
respektlosere Variante wäre es, ihn Sklavensohn zu nennen. Die Sklaverei war offiziell bereits seit 
zwei Jahrzehnten abgeschafft worden. Ein vergeblicher Versuch König Kareens Friede mit den Rebellen 
der südlichen Kolonien zu schließen. Leichtsinnigerweise hatte er keinen Finger gekrümmt, um 
aufzupassen, ob dieses Gesetz auch die saleicanischem Adeligen in den Kolonien beachten.\\
Er war noch ein Knabe, als er sich den saleicanischen Eroberern anschloss und bereits vor Jozah 
dort im Krieg stationiert. Es hatte nie einen Grund gegeben, dem Melandar - wie seine Heimat hieß - 
zu misstrauen. Trotzdem hatte er nicht mit nach Saleica kommen sollen. Am Tag als die Schiffe in 
die Heimat segeln sollten, standen noch die Priester am Deck und zeterten, dass dies nicht erlaubt 
sei. Die Missionare hatten sich mit der falschen Truppe angelegt. Sie hatten zusammen das Chaos 
überlebt und Elor war dabei an ihrer Seite gewesen. Obwohl Mishka sonst eher lautstark verkündete, 
für was für einen dreckigen Hund er den Melandar hielt, so war er damals der Erste gewesen, der den 
Priestern gegenüber getreten war. Am Ende landeten drei von ihnen im Wasser. Einer war freiwillig 
gesprungen, dem anderen hat Mishka das Vergnügen bereitet. Ein Dritter ist noch hinterher 
geplumbst, als er seine Glaubensgefährten hatte herausfischen wollen. Immer noch erzählte die 
Truppe sich diese Geschichte und kam kaum aus dem Lachen heraus.\\
Soweit Jozah wusste, gab es nur eine Handvoll Ausländer in Brom-Dallar.\\
``Ich habe eine Nachricht von deiner Liebsten'', erklärte der Dunkelhaarige: ``Deine Ernennung 
beginnt heute nach der Entzündung des Nachmittagsfeuers.''\\
Jozah verschluckte sich und hustete. ``Liebsten?''\\
Seine beiden Freunde tauschten Blicke. ``Nicht?'', fragte Mishka: ``Also, wenn das so ist, nehm ich 
sie mir.''\\
``Unterstehe dich!'', knurrte Jozah und wischte sich mit dem Ärmel über den Mund.\\
``Warum dann so überrascht? Ihr trefft euch doch seit Wochen'', stellte Elor fest und ließ eine 
Portion Brei von seinem Löffel zurück in die Schüssel tropfen. ``Ich vermisse das Essen'', seufzte 
er leise und verzog das Gesicht.\\
``Eure Rüben sind so widerlich wie eure Weiber '', entschied Mishka und griff nach Elors Schüssel: 
``Aber wie du willst, ich erlöse dich vor dieser Aufgabe.''\\

Vito Ma'sah hatte es sich nicht nehmen lassen, seine privaten finanziellen Mittel bei der 
Festlichkeit zu Jozahs Ernennung zum General einfließen zu lassen. Üblicherweise stellte die 
Familie des Ehrengastes diese. Oder aber, in Falle von mittellosen Soldaten, lief sie recht 
unspektakulär ab. Eine Urkunde des Königs, den Segen eines Priesters und noch die Überreichung 
eines symbolischen Geschenks in Form einer Klinge oder eines Pferdes und die Sache war erledigt.\\
Mit gemischten Gefühlen, was ihn wohl am heutigen Abend und in naher Zukunft erwarten würde, ließ 
Jozah seinen Schimmel durch die Straßen der Nobelviertel traben. Der rhythmische Hufschlag hallte 
in der vorabendlichen Stille wieder. Noch war die Stunde der Dämmerung, in der die adeligen Söhne 
und Töchter ihre Zeit damit verbrachten, sich für die geselligen Stunden in der Stadt herzurichten 
und den Ermahnungen ihrer Eltern kein Gehör schenkten.\\
Jozah hatte den Offizier Lerin bei Wort genommen und sich für den Schimmel entschieden. Der Wallach 
war bereits seit den Kolonien auf seiner Seite. Ein treues, gehorsames Tier, welches selten scheute 
und ihn bereits mit einem Wiehern von der Weide aus grüßte.\\
Er zügelte sein Reittier vor den Stufen der Stadtvilla und glitt aus dem Sattel. Auch wenn er sich 
selten eitel zeigte, was sein Äußeres anging, so trug Jozah heute seine neusten Lederstiefel, die 
sorgfältig ausgebürstete, neue Uniform eines Generals und hatte sich das Haar schneiden lassen. 
Immerhin war es sein erster und vorerst letzter Abend, den er in der Gesellschaft hochrangiger 
Militärs in der Hauptstadt verbringen würde. Er wollte seinem Mentor, dem er so viel zu verdanken 
hatte, auf keinen Fall Schande bringen.\\
Ein Bediensteter ergriff die Zügel mit einem Nicken und führte den Schimmel in den angrenzenden 
kleinen Stall. Jozah sah ihm hinterher, dann wanderte sein Blick zum Eingangsportal. Licht 
schimmerte durch den Türspalt hervor. Er straffte sich ein letztes Mal und trat dann die wenigen 
Stufen empor. Vor dieser Tür musste er nicht lange nachdenken, ob er klopfte oder eintrat. Vito 
Ma'Sah hatte ihm schon mehr als einmal zu verstehen gegeben, dass er ihn als den Sohn sah, den er 
nie selbst hatte. Außerdem war er der Ehrengast. Schwungvoll stieß er die Tür auf und trat mit 
einem breiten Lächeln auf den Gesicht ein. Ilia empfing - ganz die vorbildliche Dame des Hauses - 
eben noch die kurz vor ihm eingetretenen Gäste mit einem herzlichen Händedruck. Jozah hielt inne 
und beobachtete sie dabei. Ihr Blick huschte flüchtig über ihn hinweg, ehe sie sich wieder ganz auf 
das Paar vor ihr konzentrierte. Nichts in ihrer Haltung oder Mimik veränderte sich.\
\textit{Leidenschaftliche Tänzerin oder berechnende Spielerin?}, grübelte Jozah amüsiert. Ilia war 
so viel und doch zeigte sie oft nur eine Seite ihres Charakters. Während sie am Abend zuvor noch 
das Mädchen mit viel zu viel Geld war, welches sich nur eine spannende Nacht gönnen wollte, so 
zeigte sie hier ihre Redegewandtheit. Jozah war sie immer noch nicht sicher, ob er bereits jemals 
hinter ihre zahlreichen Masken geblickt hatte.\\ 
``Jozah'', begrüßte sie ihn schließlich und ergriff seine Hände. Sie hauchte ihm einen Kuss auf die 
Wange. ``Mein Vater ist aufgeregt wie ein kleines Kind. Er prahlt schon die ganze Zeit mit dir. Von 
Geschichten deiner Ausbildung bis hin zu deinen Heldentaten in den Kolonien!''\\
``Er hat viel zu viel Geld ausgegeben. Und wer sind diese ganzen Leute?'', flüsterte Jozah.\\
``Wenn es danach geht, besitzt er viel zu viel Geld. Und ein Großteil sind Freunde aus seiner 
aktiven Zeit beim Militär. Nutze lieber die Gelegenheit, neue Beziehungen zu knüpfen. Er wirft dir 
doch schon seit Jahren vor, dass du dem nicht genug Beachtung schenkst. Sonst wärst du schon viel 
früher zum Kommandanten befördert worden.''\\
Jozah lächelte als Antwort auf ihre Worte und verkniff sich die Aussage, dass dem ganz gewiss nicht 
so war. Er war trotz seinem adeligen Mentor nur der Sohn eines Handwerkers.\\
``Komm. Wir sollten niemanden länger warten lassen.'' Ilia zog ihn an einer Hand hinter sich her in 
den großzügigen Salon. Gut Eindutzend Gäste waren versammelt. Einige Bedienstete huschten 
unauffällig umher und servierten Getränke oder Speisen, welche auf einem Tisch entlang der Wand 
gerichtet waren. Davon gegenüber nahm der offene Kamin viel Platz ein. Kerzen und Lampen erhellten 
den Raum. Ein Musiker spielte auf einer Violine. Die gemütlich aussehenden Sitzgelegenheiten waren 
bereits belegt und die übrigen Gäste hatten sich zu kleinen Gruppen zusammengefunden und 
plauderten.\\
Kaum hatte Vito Ma'Sah ihn entdeckt, rief er zur Ruhe. Man traute der Gestalt des älteren Herrn so 
eine herrische, direkte Stimme auf den ersten Blick nicht zu. Aber immerhin war dies der 
Mann, der in seiner Jugend eine der Seeschlachten an der Monari-Küste für Saleica entschieden 
hatte. Jozah grauste schon die Vorstellung an einen Kampf auf dem Wasser und konnte nur Osyma 
danken, dass ihm das bisher erspart geblieben war. Er gehörte zwar nicht zu denen, die sofort über 
der Reling hingen um den Mageninhalt zu entleeren, aber den sicheren Stand für einen Kampf traute 
er sich auch nicht zu.\\
``Da ist er!'', verkündete Vito Ma'Sah: ``General Mi'Kae!''\\
Er klopfte ihm kameradschaftlich auf die Schulter und hob sein mit rotem Wein gefülltes Glas. ``Ich 
sah ihn damals auf dem Platz. Ein dünner Knabe der mich eher an einen Hund mit eingezogenen Schwanz 
erinnere als an einen Soldaten! Aber ich sag euch, kam man ihn zu nahe, biss der Köter plötzlich 
zu.''\\
Vito Ma'Sah nahm einen Schluck und lachte leise in sich hinein, ehe er mit seiner Rede fort fuhr: 
``Und Heute ist er fast schon ein passabler Löwe geworden! Nein nein, ich scherze ja nur. Jozah 
Mi'Kae ist ein pflichtbewusster Soldat, der im Namen der Krone und des Allmächtigen sein Leben in 
den Kolonien aufs Spiel setzte und siegreich heimkehrte! Er verdiente sich dort die Beförderung zu 
Kommandanten und führte seine Leute durch zahlreiche Schlachten. Und jetzt ist es so weit, dass 
unser geschätzter König Semric ihn zum General ernennt!''\\
Der Offizier überreichte ihm ein zusammengefaltetes Pergament mit dem Siegel des Königshauses. Den 
brüllenden, gekrönten Löwenkopf.\\
Jozah räusperte sich. ``Vielen Dank...''\\
Ein Priester trat zwischen den Gästen hervor und setzte dazu an, die feierlichen Worte der Segnung 
zu sprechen, aber Vito Ma'Sah winkte grob ab und packte Jozah ein weiteres Mal an der Schulter. 
``Später! Ich möchte noch etwas wichtiges und persönliches hinzufügen. General Mi'Kae war mir ein 
Sohn. Ich finanzierte seine Ausbildung. Ich schimpfe über seine faulen Vorgesetzten. Ich hieß ihn 
willkommen, als er aus den Kolonien heimkehrte. In den letzten Monaten verbrachte er viel Zeit in 
der Gesellschaft meiner reizenden Tochter Ilia und ich kam zu den Entschluss, dass es nun so weit 
ist, ihn vollends als Sohn und Erben in mein Haus aufzunehmen. Durch eine Verlobung mit meiner 
Tochter.''\\
Die Gäste applaudierten höflich und tauschten teils amüsierte, teils vielsagende Blicke. Es kam 
selten vor, dass jemand mit einem solchen Rang wie Vito Ma'Sah sich einen Schützling aus ärmlichen 
Verhältnissen holte. Ihn dann auch noch mit der einzigen Erbin des Familiennamens zu verloben war 
vermutlich eine Prämiere in der Hauptstadt Saleicas. Jozah starrte seinen Mentor sprachlos an. Er 
hatte in den letzten Wochen tatsächlich darüber nachgedacht, den alten Offizier unverfänglich zu 
fragen, wie er zu einer Heirat stehen würde. Ilia faszinierte Jozah und er war in dem Alter, in dem 
er sich langsam mal eine Frau suchen sollte. Bisher war sie die Einzige, die je auf Dauer sein 
Interesse gehalten hatte. Andererseits war Ilia auch in vielen Charakterzügen ihm gegensätzlich.\\
Und während Jozah noch seinen Mentor überrascht ansah, entging ihm der flüchtige Moment, in dem 
Ilias Maske zerbrach. Das Lächeln gefror. Der Blick wurde scharf und kalt wie Eis. Ihre Hand 
zitterte vor unterdrückten Gefühlen, als sie sich durch das blonde Haar strich. Sie lachte hell, 
eilte mit schwingenden Schritten an Jozahs Seite und umfasste seine Hüfte. ``Wie jede Frau malte 
ich mir bisher die romantischsten Anträge aus'', sagte sie zu den Gästen gewandt: ``Aber dass er 
schließlich von meinem Vater kommen würde, habe ich nicht erwartet!''\\
Die Männer und Frauen reagieren mit heiterem Gelächter über den Witz und die Lage entspannte sich 
wieder. Im Gegensatz zu Ilias Griff. ``Ich denke, unsere geschätzten Gäste können verstehen, wenn 
ich ihnen einen Moment den Ehrengast stehle. Und den Gastgeber'', fügte sie hinzu und wandte sich 
abrupt um.\\
Sie wusste natürlich, warum ihr Vater das getan hatte. Niemals hätte Ilia vor adeligen Gästen 
abgelehnt. Und auch nicht in Jozahs Gegenwart, während er überrascht und stumm dabei stand.\\
``Habt ihr das geplant?'', fragte sie direkt, als die Tür hinter ihnen ins Schloss gefallen war.\\
``Nein'', beeilte Jozah sich zu sagen und blickte irritiert zu Vito Ma'Sah.\\
Dieser verzog nur das Gesicht und erwiderte: ``Wir haben oft genug darüber geredet, Ilia. Es wird 
Zeit für dich zu heiraten und den Familiennamen weiter zu tragen. Jozah ist der richtige. Ein 
ehrbarer Mann, steht in der Gunst des Königs und du hast in den letzten Wochen viel Zeit mit ihm 
verbracht. Es könnte keinen besseren Gatten für dich geben. Geld und Adelstitel bringst du selbst 
mit in die Verbindung. Oder hast du etwas an ihm auszusetzen?''\\
Ilia kniff die Lippen zusammen und lächelte Jozah kurz darauf an. ``Nein. Aber ich hatte es mir 
anders vorgestellt. Weniger überraschend. Weniger Zuschauer. Mehr Ehrlichkeit und echte Gefühle. 
Hattest du es vor, Jozah?''\\
General Mi'Kae räusperte sich. ``Nun... schon. Aber dann kam die Ernennung dazwischen und der 
Auftrag des Königs.''\\
``Auftrag des Königs?'', wiederholten Ilia und ihr Vater gleichzeitig.\\
Jozah zupfte sich nervös die Uniform glatt. ``Ich reite morgen schon nach Merandila. Um die junge 
Witwe bei den Kriegsvorbereitungen zu unterstützen.''\\
``Morgen schon'', sagte Ilia und warf ihrem Vater einen finsteren Blick zu: ``Du verlobst mich mit 
einem Soldaten, der gleich am nächsten Morgen in den Krieg zieht? Was stellst du dir vor, dass ich 
das Leben einer Witwe führe, obwohl wir noch nicht einmal geheiratet haben?''\\
``Dann geh mit nach Merandila'', entschied ihr Vater.\\
Ilia lachte laut auf. ``Nein. Da soll es so kalt sein, dass selbst die Hunde sich nicht vor die Tür 
wagen. Vergesst es.''\\
``Entschuldige Ilia'', murmelte Jozah: ``Das ist mir sehr peinlich und ich wollte dir nicht zu nahe 
treten...''\\
``Oh, deine Schuld ist es nicht'', entschied sie: ``Ich wartete ehrlich gesagt schon eine Weile 
darauf, dass du meinen Vater um meine Hand bitten würdest. Aber verstehe bitte, dass es wirklich 
sehr anders ist als das, was ich erwartet habe...''\\
``Du sagst also nicht nein?'', fragte Jozah zögernd.\\
Ilia beugte sich vor und küsste ihn ein weiteres Mal auf die Wange. ``Nein. Aber verschiebe deinen 
Aufbruch noch um ein paar Tage. Ich will mich wenigstens kurz wie eine richtige Verlobte fühlen, 
ehe du reitest.''\\




 
