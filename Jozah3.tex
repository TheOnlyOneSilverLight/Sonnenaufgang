\chapter{Aufbruch in den Norden}

Die Nacht vor dem Aufbruch seiner Mission verbrachte Jozah damit, seine Klinge zu schärfen, die 
Reiseroute zum zehnten Mal zu studieren und auf der Unterlippe zu kauen. Es war eine Mischung aus 
freudiger Erwartung, Tatendrang und Sorge, die ihn wach hielt. Er genoss es, endlich wieder etwas zu 
tun zu haben. Die Eintönigkeit der Kaserne konnte ein Seegen sein, wenn man sich nah Struktur und 
Halt sehnte. Jozah stritt auch keinesfals ab, dass es ihm und vermutlich auch seinen Leuten gut 
getan hatte. Trotz Wachdiensten oder anderen Verpflichtungen war es doch so viel anders als der 
Dienst in den Kolonien. Man war unter sich, musste nicht mit plötzlichen Rebellionen oder 
Aufständen rechnen. Man musste nicht hinterfragen, wer einem vielleicht etwas böses wolle. Nein, 
man stand auf zum Morgenapell, ging seinen eingeteilten Diensten nach und pünktlich stand das Essen 
auf dem Tisch. \\
Drei Monate würden niemals genügen um sich von den Erlebnissen in den Kolonien zu erholen. 
Vielleicht nicht einmal drei Jahre. Also verhielt Jozah sich wie jeder andere und wie es von 
Soldaten erwartet wurde, er verdrängte. Das half meistens, auch wenn das Bedürfnis, darüber zu 
sprechen, immer wieder kam. Seine ehemaligen Kameraden waren nun seine Truppe... er konnte es sich 
nur vor wenigen Außerwählten leisten, offen über seine Gedanken zu Gefühle zu sprechen und auch das 
wollte er nicht unnötig herausfordern. Er war der Kommandant, er trug die Verantwortung. Damit ging 
einher, dass Gespräche verstummten, wenn er dazu stieß, Dass Blicke getauscht wurden. Keiner seiner 
Leute wollte vor ihrem Kommandanten Schwäche zeigen und sie wollten ebenso wenig, dass ihr 
Kommandant das anders handhabt. \\
Trotz allem vermisste Jozah die Kolonien. Die Weite der Landschaft... etwas, was sich kein 
einfacher Bürger Saleicas vorstellen konnte. Diese Grenzenlosigkeit. Nicht nur die Landschaft, auch 
die Menschen die sie belebten. Es schien, als müsste man nur wenige Tage in eine Richtung gehen um 
immer neue Kulturen zu erleben. Vor den Meisten sollte man sich in acht nehmen. Aber die 
Geschichten... unzählige Legenden und Lieder. Grenzenlos.\\

Vor Morgengrauen verließ Jozah seine Kammer. Er wollte die kalte Morgenstunde noch einmal 
angemessen würdigen, ehe er die nächsten Tage keine Minute mehr alleine für sich haben würde. Sein 
Vorhaben ging jedoch nicht auf, denn Mishka befand sich ebenfals auf dem Hof und in der Ferne hörte 
man die harsche Stimme eines Ausbilders, der die Rekruten drillte.\\
Provozierend lässig lehnte der blonde Hühne an der Schulter seines Hengstes. Ein Tier, welches auf 
dem ersten Blick gleich zu Mishka passte. Ein Tier, welches direkt aus einem Albtraum entsprungen 
sein könnte. Riesig, so dass Jozah sich nicht einmal zutrauen würde, einigermaßen elegant in den 
Sattel gelangen zu können. Das ganze Geschöpf schien nur aus Muskeln und Kraft zu bestehen. Die 
Hufe hatten schon manchen, der auf dem Schlachtfeld zu Boden ging, sein Ende bereitet. Ein 
Schlachtross, schwerfällig und kraftvoll, langsam und ausdauernd. Die dunklen Augen funkelten 
ebenso abenteuerlich wie Mishkas. Die Vorfreude auf einen Kampf konnte man den beiden jeder Zeit 
ansehen. Jozah hatte noch nie ein Tier gesehen, welches Spaß an einer Schlacht gezeigt hatte. \\
\textit{Gab es nicht mal eine Legende über verfluchte Menschen, die in Tiergestalten verbannt 
wurden?}
Mishkas Hengst wirkte oft unheimlich menschlich. Unwillkürlich schüttelte Jozah den Kopf, um diese 
wirren Gedanken zu verscheuchen. Soweit er wusste, hatte Mishka das Tier als Abschiedsgeschenk 
seiner Familie erhalten, bevor er in die Kolonien ging.\\
``Na? Ists gestern spät geworden?", lachte Mishka.\\
Jozah verzog das Gesicht. ``Nicht später als bei dir. Du stinkst nach Alkohol und stehst trotzdem 
auf 
beiden Beinen."\\
Mishka zuckte mit den Schultern. ``Übung. Hab gehört, der Schimmel ist jetzt dein?"\\
Jozah blickte zum Stall, aus dem gerade ein Rekrut seinen Wallach brachte. Als Antwort auf Mishkas 
Frage nickte er nur und nahm die Zügel des Zelters entgegen. ``Wo sind die Anderen?"
``Werden bald kommen."\\
Jozah klopfte dem Wallach zur Begrüßung auf dem Hals und schnalzte mit der Zunge, um seine 
Aufmerksamkeit zu erlangen. Er führte das Tier zur kleinen Weide, die am Hof angrenzt. Das Gras war 
karg und und vertrocknet, bis auf wenige Zentimeter abgeknabbert. Außerhalb des Zaunes gab es noch 
das ein oder andere längere Halm und diese schnappte sich nun der Schimmel. 
``Elor versucht sein Biest einzufangen'', kommentierte Mishka und deutete auf das Padok.\\
Jozah hielt damit inne sein neues Eigentum zu verwöhnen und spähte in die Dämmerung. Der drathige 
Mann - der Zweite den Jozah noch als Freund bezeichnen konnte - stand nahezu reglos in der Mitte 
und seine von vielen verspottete Stute schlich um ihn herum. Sie glich eher einem Raubtier, welches 
seine Beute belauerte, als einem eher zur Flucht und Panik neigendem Pferd. Das dunkelbraune Fell 
wirkte in dem Morgenlicht ebenso schwarz wie die Mähne und die Augen auch bei Tage sind. Ihr 
Wiehern hallte über den Hof und Jozahs Wallach erwiderte den Ruf freudig, ohne zu bemerkten, dass 
er keineswegs gemeint war. \\
``Das Vieh gehört geschlachtet", murrte Mishka.\\
Seit die beiden einmal aneinander geraten waren und Mishka als Verlierer hervorkam, war er nicht 
gut auf die Stute zu sprechen. Das Zeichen über der Niederlage gegenüber dem Pferd zierte noch 
seine Hüfte. Eine Narbe, die Mishka mal ausnahmsweise nicht stolz herum zeigte. Es war wohl 
unehrenhaft von einem Pferd fast zerfetzt zu werden. \\
``Sind schon komische Typen, die aus den Kolonien."\\ 
``Und deren Pferde erst!", spottete Jozah und beobachtete Elor mit zusammengekniffenen Augen. 
Der Mann war gute drei Jahre jünger als er selbst und wenn man ihn neben Mishka stellte, ungefähr 
nur ein Drittel von dessen Körpermasse. Dunkles, schwarzes Haar und ebenso dunkle, berechnende 
Augen. Elor war eher drahtig als muskulös, sprach meist wenig und lachen hörte man ihn noch 
seltener. In seinen Adern floss das Blut des Südens, wie man es höflich bezeichnete. Die 
respektlosere Variante wäre es, ihn Sklavensohn zu nennen. Die Sklaverei war offiziell bereits seit 
zwei Jahrzehnten abgeschafft worden. Ein vergeblicher Versuch König Kareens Friede mit den Rebellen 
der südlichen Kolonien zu schließen. Leichtsinnigerweise hatte er keinen Finger gekrümmt, um 
aufzupassen, ob dieses Gesetz auch die saleicanischem Adeligen in den Kolonien beachten.\\
Jozah hatte bereits in etlichen Schlachten gekämpft, aber das Elend, welches er unter den Sklaven 
erlebte, war anders. Er war Soldat und würde niemals zögern, einen Feind in der Schlacht zu töten. 
Solange diese eine Waffe in der Hand hatten und den Willen, sie zu benutzen. Die Sklaven dagegen 
hingen in Ketten, waren abgemagert und nicht im Stande, sich zu wehren. Frauen und Kinder, 
missbraucht und angekettet. Männer und Jungen ihrer Hilflosigkeit vorgeführt. Elor war 
der Sklave eines hochrangigen Offiziers gewesen, der geplant hatte, bald in Ruhestand zu gehen und 
es sich in einer Strandvilla bequem zu machen. Er starb wie so viele im Krieg und Jozah entschied 
kurzerhand, dass er nicht zusehen würde, wie Menschen an verwöhnte Adelskinder vererbt werden 
würden. Er verkündete den Dutzend Sklaven, dass ihr Herr in der Schlacht gefallen war. Und dann 
wandte er sich demonstrativ ab. Es ging überraschend leise von statten und als er sich wieder 
umsah, waren sie weg, nur Elor nicht. Damals noch eher ein Junge als ein Mann hatte ihn mit 
Trotz angesehen. So schweigsam wie der junge Mann bis heute war, wusste Jozah kaum etwas über seine 
Beweggründe, sich den Eroberern anzuschließen. Aber Elor war nicht nur ein guter Kämpfer und 
Spurenleser, sondern auch ein treuer Freund. \\
Der seltsame Tanz zwischen der Stute und ihrem Reiter war scheinbar plötzlich beendet. Sie blieb 
stehen und funkelte ihn böse an, wich aber auch nicht zurück, als Elor auf sie zu ging und das 
Halfter befestigte. Jozah konnte nur den Kopf schütteln. \\
\textit{Der Krieg hat das Tier verbittert}, grübelte er. Elor hatte sie vom Blut durchdrängt 
Schlachtfeld geholt. Damals verstört, blutbesudelt und verletzt war sie sogar umgänglicher gewesen 
als jetzt. Der Dunkelhaarige befestigte den Sattel, welcher auf dem Gatter bereit hing und stieg 
auf. ``Wir sind fertig", verkündete Elor und nahm die Zügel auf. \\
``Weiber", murrte Mishka: ``Der eine kämpft mit der zickigen Stute, der andere mit 
der flatterhaften Blondine."\\
Jozah seufzte. ``Immer wieder schön, wenn der Tag und der Beginn einer Reise so angenehm harmonisch 
und friedlich verläuft."\\

Die ersten Stunden ritten sie schweigend nebeneinander her. Die Pferde trotteten an langen Zügeln 
den gepflasterten Weg entlang, der Brom-Dalar mit der nächst größerer Stadt Arr-Dero verband. Es 
waren nicht viele Reisende unterwegs. Jozah hatte aufgeschnappt, dass weiter westlich ein großer 
Markt abgehalten wurde. Dort trieben sich vermutlich die Leute herum, die sonst die Straßen 
blockierten. Jozah war es recht, so hatten sie ihre Ruhe. Er hatte aber doch erhofft, dass seine 
beiden Kameraden etwas unterhaltsamer sein würden. Stattdessen, was äußerst unüblich war, hielt 
sogar Mishka seine Klappe und blickte, was recht häufig vorkam, böse vor sich hin. \\
Jozah brach schließlich die Stille. ``Tut euch auch so der Arsch weh?"\\
Mishka gab einen zustimmenden Laut von sich und verzog das Gesicht. \\
``Wir werden alt", sagte Jozah scherzhaft.\\
``Nein. Nur faul", bemerkte Elor.\\
``Das ist das erste Anzeichen des Alters. Ah… früher haben wir den halben Tag auf dem Übungsplatz 
oder im Sattel verbracht und nun?\\"
``Ihr vielleicht", sagte Elor: ``Ich trainiere täglich mit Yu und bin am Abend auf dem Übungsplatz."
``Du interessierst dich ja weder für Frauen, Trinkspiele oder Wetten. Selbstverständlich hast du 
nichts Besseres zu tun, als die halbe Nacht auf Sandsäcke einzuprügeln."\\
Jozah griff ein und wechselte das Thema. Es kam schon selten genug vor, dass Elor Dinge über sich 
preisgab. ``Yu? Dein Pferd oder dein Schwert?"\\
Elor klopfte seiner Stute den Hals. ``Yu heißt Feuer."\\
``Welch passender Name!", rief Mishka, schüttelte den Kopf und schnalzte mit der Zunge. Sein Hengst 
setzte sich schwerfällig in Bewegung und trabte davon. Jozah sah sich flüchtig um und ließ sich die 
Chance nicht nehmen, ein paar private Worte mit Elor zu wechseln, solange keiner der anderen in 
Hörweite war.
``Die Kolonien… vermisst du sie?``, fragte er.
''Hast du Saleica nicht vermisst?``
Jozah überlegte kurz und zuckte dann mit den Schultern. ''Hier bin ich der Sohn eines Handwerkers. 
Die Kolonien haben mich zu einem Kommandanten gemacht.``\\
''In meiner Heimat war ich ein Sklave, hier bin ich ein Krieger``, entgegnete Elor.\\
''Also bist du glücklich hier?``\\
Er ließ sich Zeit mit seiner Antwort. ''Menschen sind geboren um zu dienen. Göttern, Naturgewalten, 
Ältesten, Königen. Ein jeder dient. Ich habe mir aussuchen können, wem ich diene, das reicht 
mir.``\\
''Du hast dich für den König Saleicas entschieden?``, Jozah musste ein Auflachen unterdrücken.\\
Elor sah ihn jedoch nur kritisch an. ''Wenn Sie meinen, Herr Kommandant.``\\
Dieses Kompliment machte Jozah sprachlos. Aber wieder war es Elor, der die Stille brach: ``Kennt 
Mishka den Weg?"\\
``Vermutlich nicht", überlegte Jozah, dankbar über den Themenwechsel.
``Da kommt eine Kreuzung."
``Ich weiß."\\
``Sollten wir ihn einholen?"\\
``Wir könnten auch an der Kreuzung warten, bis er zurückkommt."\\
``Ich könnte eine Pause vertragen." Elor grinste.\\
Jozah nickte zustimmend und lachte leise. \\


Mishka war längst aus ihrem Blickfeld verschwunden, als sie die Kreuzung erreichten. Jozah hob die 
Hand um die Aufmerksamkeit der restlichen Männer und Frauen zu erlangen. Mit wenigen Worten 
verkündete er, dass sie hier kurz ruhen würden und schon löste sich die Truppe in kleine, murmelnde 
Gruppen auf. \\ 
``Was ist der genaue Plan?", erkundigte sich Elor, während er ein Brotstück kaute. \\
Jozah reichte ihm etwas von dem Käse und sagte: ``Bis heute Abend erreichen wir Jash. Dort 
nächtigen wir in der Garnision. Am Morgen bekommen wir ein Packpferd und ausreichend Proviant. Den 
nächsten größeren Halt machen wir in Kar-Mas. Ich denke, das erreichen wir in fünf, sechs Tagen. 
Wenn der Weg weiterhin so ruhig bleibt. Von dort aus geht es nach Maju."\\
``Ich ahne was jetzt kommt", seufzte Elor.\\
``Ab da reisen wir mit einem Binnenschiff weiter. Ich schätze, zwei Wochen. Und dann noch einige 
wenige Tage zu Pferd."\\
``Und dann sitzen wir vor der Burg Merandilas. Und dann? Was will der König eigentlich?"\\
Jozah zögerte. ``Er will wissen, ob die junge Gräfin fähig ist, Merandila zu verwalten."\\
Elor beugte sich vor. ``Wozu? Warum nicht einfach einen seiner Schoßhunde hinschicken, der sie 
heiratet und fertig? Das Volk Merandilas wird ein bischen meckern… aber was soll es schon groß 
ausrichten? Die Mehrheit dort des stehenden Heeres stammt aus Merandila. Ich habe gesehen wir die 
Völker meines Kontinentes tapfer kämpfend untergingen... woher sollten die Bauern eines Grafschaft 
die Überzeugung erlangen, gegen ihren König siegen zu können?"\\
Sie aßen schweigend zu ende. Als Jozah das nächste Mal aufblickte, trottete Mishkas Hengst auf sie 
zu. Jozah konnte das Gesicht seines Kameraden nicht erkennen und erhob sich. Ernst blickte er ihm 
entgegen, weil er nicht einschätzen konnte, wie Mishka reagieren würde. Der Kerl war unberechenbar, 
wenn er sauer wurde. Das kannte Jozah gut genug.\\
Der hünenhafte Soldat stoppte sein müdes Pferd vor ihnen und funkelte Jozah an. Sein Gesicht verzog 
sich zu einer Grimasse, die Jozah schließlich als Grinsen erkannte. \\
``Gönnen Sie meinem Pferd noch fünf Minuten, Kommandant?"\\
``Iss, trink. Und dann reiten wir weiter. Wir werden erwartet."\\
``Nach meinen Erfahrungen ist es für die herzallerliebsten Gastgeber nicht sehr bedauerlich, wenn 
man später kommt und eher geht."\\
Dem konnte Jozah leider nur zustimmen. Er wollte nicht jeden Hochgeborenen als schlechten Gastgeber 
bezeichnen, aber es war auffallend häufig, dass sie nur darauf warten zu schienen, dass man endlich 
ging. Außer sie hatten eine Tochter im heiratsfähigen Alter, dann wurden ihre Bemühungen, einen so 
lange wie möglich zu halten und erst mit einem perfekten Eindruck über die eigene Familie und vor 
allem die besagte Tochter gehen zu lassen.\\
Sie ritten längst gemächlich weiter, während Jozah sich an seine Dienstjahre in den Kolonien 
entsann. Sie kamen als Eroberer und trotzdem saßen sie bei den Einwohnern am Lagerfeuer. Die 
Wenigstens begehrten auf. Die schlimmsten Schlachten waren schon zur Lebzeiten König Kareens 
geschlagen worden. Die Eroberer kämpften nicht mehr gegen ganze Völker, sondern nur noch gegen 
einzelne aufständische Gruppen.\\
Jozah hatte bei seiner Ankunft gleich Position in einem der ersten Lager bezogen. Er ritt mit einer 
Truppe von 15 Mann in ein Dorf. Kinder entdeckten sie zuerst. Sie rannten natürlich sofort zurück 
in ihre Häuser, laut rufend. Kurz darauf kamen Frauen und ältere Männer, freundlich lächelnd. Kaum 
waren sie aus den Sätteln gestiegen, wurden die Soldaten an den Armen gepackt und mit schnellen, 
unverständlichen Worten zur Dorfmitte gezogen. Essen wurde aufgetragen und ehe man sich 
versah, war man mitten in einem Fest gelandet. Jozah hatte kein Wort verstanden, aber dafür wurde 
viel mit den Händen gefuchtelt und Grimassen gezogen. \\
\textit{Wie können Menschen so unterschiedlich sein? Die Einen nehmen Fremde mit offenem Herzen 
auf, teilen ihr Hab und Gut und den Schutz ihrer Häuser. Die Anderen kämpfen aus Spaß. Töten für 
einen zornigen Gott.\\
Ich weiß, warum ich Soldat bin. Ich wollte meine Eltern stolz machen. Ich will etwas erreichen, was 
größer als mein Name ist, größer als mein Blut, größer als all meine Ahnen. Mishka ist Soldat, weil 
sein Vater Soldat war, seine Schwestern Soldaten sind. Er folgt dem Ruf seines Blutes, ich versuche 
das meine erst noch zum erklingen zu bringen. Eines dieser zwei Beweggründe betrifft vermutlich 
jeden meiner Leute, außer Elor.}\\


Nachdem die ersten Tage der Anspannung hinter ihnen lagen, begann Jozah die Reise zu genießen. Es 
tat gut, etwas zu tun zu haben. Die Wochen am Sterbebett seines Vater hatten ihn mehr ausgezerrt 
als er sich eingestehen konnte. Und es war einfach ein geniales Gefühl wieder mit den Männern und 
Frauen unterwegs zu sein, mit denen er vor wenigen Monaten noch in die Schlacht geritten war. Wie 
viele hatten damals ihren Schlachtruf gebrüllt? Wie viele hatten um ihre Toten geweint, um ihren 
eigenen Tod gefürchtet? Jeder. Und sie hatten das alle zusammen durchgemacht. Mit jedem, der damals 
schon dabei war, fühlt sich Jozah verbunden. Auch mit dem Arschloch Persene, der die meiste Zeit 
dreckige Witze macht oder dem heuchlerischen Oknaha, der bei jeder Gelegenheit in einem Bordell 
verschwindet. Jedem von ihnen würde er wieder sein Leben anvertrauen und er würde auch für sie in 
die Bresche springen.\\
\textit{Das ist der Unterschied!}, dachte er triumphierend, als er am Abend einen Blick über die 50 
Soldaten schweifen ließ. \textit{Kein König hat bestimmt, dass ich ihr Anführer werden sollte. 
Nein, sie haben mir die Gelegenheit gegeben, mich zu beweisen. Und ich habe es geschafft.}
Das war im Nachhinein betrachtet auch nicht schwierig. Viel schwieriger war es, dass es so blieb. \\
Die Soldaten genossen es ebenfalls außerhalb der Kaserne und den Städten zu sein. Die Meisten 
knabberten gerade am Proviant und führten sich auf, als wären sie betrunken. Einige hatten sich 
Stöcke gesucht und duellierten sich damit, obwohl ihre Waffen keine zwei Schritte entfernt lagen. 
Mutproben, Wetten, Scherze und Kämpfe. Jozah lächelte, als er sie betrachtete.   Er war stolz auf 
das, was er bis jetzt in seinen jungen Jahren – ohne jegliche Vorteile von Geburt oder Geld – 
erreicht hatte und er fand, dass durfte er auch sehr wohl sein. \\
Er entdeckte Elor bei den Pferden und trat zu ihm. ``Du bist verdammt still in den letzten Tagen. 
So kenne ich dich nicht."\\
Der dunkelhaarige Mann blickte nicht einmal auf, sondern striegelte mit ruhigen Bewegungen weiter 
das Fell seiner Stute. ``Ich passe mich nur an. Die Tage sind ebenfalls ruhig."\\
Jozah runzelte die Stirn und warf einen Blick zu der lärmenden Meute. Erzähl mir, über was du 
grübelst, Bruder", sagte der junge Kommandant leise. \\
Jetzt begegneten sich ihre Blicke. ``Yohir." erklärte er: ``Yohir bedeutet Bruder."
``Du hast mich schon einmal so genannt. In den Kolonien."
Elor seufzte und der berechnende Mann verwandelte sich wieder in den Freund, den Jozah kannte. 
Erleichterung erfüllte ihn. Die Fuchsstute bekam einen freundlichen Klaps zum Abschied und ihr Herr 
wandte sich Jozah nun zu. \\
``Es tut mir Leid um deinen Vater. Ich habe ihn nie kennengelernt."\\
Jozah zuckte mit den Schultern. ``Er war ein armer Mann. Sein Leben bestand aus Selbstmitleid und 
Neid."\\
``Und doch hat er dich geprägt. Jozah... wir sollten vorsichtiger sein. Die Zeiten ändern sich. Es 
gibt Krieg. Bald. Aber es wird für die Saleicaner noch zu lange dauern. Selbst in Brom-Dalar wurden 
sie unruhig, dabei sind so so weit weg von der erwarteten Front. Sie hassen Kasir seit unzähligen 
Jahren ohne einem bestimmten Grund und nun endlich gibt es Krieg! Das ist für viele praktisch eine 
Erlaubnis, durchzudrehen."\\
Wieder glitt der Blick des jungen Kommandanten über seine Leute. ``Nicht für uns."\\
Elor schnalzte mit der Zunge: ``Wenn du dich bemühst. Soll ich dir einen Rat geben?"
Nach einem kurzen zögern nickte Jozah und wartete gespannt. Doch der Blick seines Kameraden war auf 
den Boden gerichtet und es herrschte schweigen. Es vergingen zähe Sekunden, ehe ein Ruck durch 
seinen Körper ging und die dunklen Augen Jozah fixierten. ``Ruhm ist so zerbrechlich wie ein 
Genick. Ehre so schnell verraucht wie ein Feuer auf nassem Grund und Sicherheit trügerisch wie 
Schatten.''\\
``Du kennst mich. Ich strebe nicht nach Ruhm und Ehre. Ich bin ein einfacher Mann.''\\
Elor lachte spöttisch auf. ``Das sagst du oft. Vielleicht denkst du es meistens. Aber dem ist 
nicht so.''\\



Der Hafen lag in einem kleineren Dorf, in dem sich jetzt zur Mittagszeit mehr Menschen aufhielten 
als vermutlich wohnten. Die Arbeiter kamen von entfernteren Dörfern, da sie dort ihr Land besaßen. 
Jozah hatte bereits vor der Abreise einen Boten losgeschickt und veranlasst, dass eines der 
größeren Binnenschiffe sie hier mitnahm. Daher hatte er in den letzten zwei Tagen doch noch das 
Tempo seiner Truppe erhöhen müssen. Er ließ nicht gerne Leute waren. \\
Sie blieben auch nicht lange in dem Dorf, sondern verfrachteten gleich ihre Pferde über eine 
schmale Planke auf das Schiff. Die erfahrenen Soldatenpferde hatten wenig Probleme damit und 
gehorchten ihren Besitzern, die sie am Zaumzeug führten. Jozah selbst hatte einige Probleme mit 
seinem jungen Tier und sprach eindringlich auf ihn ein. Er wusste, dass Wetten liefen, wer wohl im 
Wasser landen würde und atmete erleichtert aus, als er seinen Schimmel an Deck hatte. Das Tier 
machte steife und stockende Bewegungen, die Ohren lagen flach am Schädel an und seine Augen waren 
nervös verdreht. 
``Ruhig", murmelte er ihm zu und beobachtete, wie das Tier einen Schritt nach dem anderen machte. 
Er sah es jetzt schon, dass der Schimmel sich komplett verspannen würde. Wenn sie wieder an Land 
wären, müsste er ein paar Übungen mit ihm machen, um seine Muskeln zu lockern. \\
Ein schrilles Wiehern ertönte. Jozah blickte zur Planke und spürte, wie sein Schimmel sich von dem 
panischen Schrei anstecken ließ. \\
``Schnell rein!", schimpfte er und ein Matrose half ihm, seinen Schimmel in die einfachen Ställe an 
Deck zu schieben.\\
``Elor", brüllte Jozah über das Deck: ``Bring den Gaul zur Ruhe oder ihr bleibt beide unten!"\\
Der Kommandant beugte sich über die Reling. Er sah noch, wie das Tier den Kopf hochriss, die Zügel 
durch die Hand flogen und Elor in den Fluss fiel. Flüche in einer Sprache, die Jozah nicht verstand 
und lautes Gelächter vom Rest der Truppe. Die Männer und Frauen gaben sich keine Mühe dabei, ihre 
Wetteinsätze unauffällig auszutauschen. Aber so zufriedenstellend war das Ergebnis wohl kaum, da 
niemand gewettet hatte, dass Elor es trocken auf das Schiff schaffen würde. \\
Jozah seufzte. Das alles half nicht dabei, seinen Zeitplan einzuhalten. Auch wenn er die letzten 
Tage außerhalb von Brom-Dalar selbst genossen hatte, wartete vermutlich viel Arbeit auf ihn. Er war 
noch nie in Merandila und muss sich das Gelände ansehen. Er wird die stehenden Heere abklappern und 
mit den Anführern der Regimente sprechen. Er wird einen Krieg planen und das alles vermutlich mit 
einem jungen Mädchen im Schlepptau. Bestenfalls würde sich die junge Witwe überzeugen lassen, in 
ihrer Burg zu bleiben und ihn die Arbeit machen zu lassen. Jozah glaubte generell nicht daran, dass 
die Gräfin lange Witwe bleiben wird. Vermutlich suchte der Hohepriester bereits einen nächsten 
Grafen und somit ihren Gemahl. Jozah konnte sich gut vorstellen, dass es sich nur so lange hinzog, 
weil der Hohepriester sich nicht entscheiden konnte, welchen seiner Günstlinge am erfolgreichsten 
sein würde.


