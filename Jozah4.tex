\chapter{Ein Mädchen (Anschluss an Sarimé 4)}

Trotz der triefenden Nässe seines Mantels, dem kalten Wind und den wartenden Blicken die auf ihm 
ruhten, konnte Jozah sich noch nicht dazu überwinden, aus dem Sattel zu steigen. Er war mit vielen 
Erwartungen hier her in den Norden geritten. Viele haben sich widersprochen. Das mit dem Wetter 
stimmte schon mal, es war scheußlich. Er konnte sich nicht entsinnen, jemals tagelang am Stück 
durchnässt gewesen zu sein. Auf der letzten Reisestrecke nach dem Schiff hatten sich auch kaum 
akzeptable Unterkünfte finden lassen, in denen seine Leute und ihre Pferde alle untergekommen 
wären. Aus Solidarität hatten sie sich abgewechselt, wer eine Nacht im trockenen verbringen durfte. 
Aber das widerlichste war ja dann doch, wieder in die nasse Kleidung steigen zu müssen. Nicht 
einmal das Fell der Pferde schien trocknen zu können.\\
Und dann waren da natürlich noch die Erwartungen, die er sich über die junge Witwe eingeprägt 
hatte. Jozah hatte bisher nicht viel Kontakt mit Frauen in Machtpositionen, außerhalb des Militärs. 
Ilia hatte solch eine Position, durch ihren Namen und ihr Geld. Sie ging spielerisch damit um, 
lachte charmant und wirkte stets, als ob sie alles im Griff hätte und nichts ihren Leichtmut stören 
könnte. Trotz dem Wissen, dass die Witwe erst 16 Jahre ist, hatte er sich ein ähnliches Verhalten 
vorgestellt. Immerhin war Sarimé Sil'Veras Name so alt wie die Hauptstadt selbst.\\
Nach dem ersten Anblick hatte er sich suchend im Hof umgesehen, aber die schlanke Gestalt war das 
einzige weibliche Wesen, welches jung genug war um besagte Gräfin zu sein. Jozah war sich seiner 
Unhöflichkeit bewusst, so war es weit über seinem Stand, hier einfach wartend im Sattel zu sitzen. 
Aber er konnte sich einfach nicht lösen. Das Mädchen war umringt von einer kleinen Gruppe Männern. 
Drei davon waren deutlich als Priester auszumachen, einer von ihnen dermaßen Tatoowiert, dass es 
sich nur um einen Hohepriester handeln konnte. Dann fielen ihm noch zwei Wachen auf und ein 
Bursche, der ihn entrüstet anfunkelte. Er stand nahe bei der Gräfin und Jozah entschied, dass diese 
beiden jungen Leute wohl vertraut miteinander waren.\\
\textit{Na... die junge Witwe wird sich doch nicht schon den nächsten ins Bett geholt haben?}\\
Er bereute diesen Gedanken doch jedoch schnell, als er sie musterte. Das rote Haar lockte sich 
vor Nässe und Wassertropfen schimmerten in den einzelnen Strähnen. Sie sah kränklich blass aus, 
hier unter den Regenwolken. Die warme Kleidung verhüllte ihre Gestalt, aber Anzeichen einer 
Schwangerschaft konnte er noch keine erkennen. Jozah bewunderte Ilia für ihre Schönheit, ihren 
Charme und ihr Selbstbewusstsein. Dieses Mädchen hier war unverkennbar hübsch, aber sie wirkte 
jetzt in diesem Moment auch nur wie Kind, das zu jung war für diese Verpflichtungen.\\
Der Hohepriester räusperte sich und wollte zu sprechen beginnen, da sprang unvermittelt der Bursche 
dazwischen. Sein Blick war intensiv und seine Worte laut, als er zu sprechen begann: ``Gräfin 
Sarimé Sil'Vera, Witwe des Grafen Evin A'Riks, Herrin über die Ländereien und Menschen Merandilas, 
erwartet etwas mehr Respekt!''\\
Jozahs Blick huschte zu dem Mädchen. Sie sah den Burschen ebenso überrascht an wie alle anderen 
anwesenden, dann änderte sich jedoch ihre Miene. Die Verunsicherung wich und ein Funkeln trat in 
ihre Augen.\\
\textit{Na bitte}, dachte Jozah: \textit{Noch ein paar Jährchen Übung und Erfahrung, dann könnte 
sie vielleicht eine brauchbare Führungskraft abgeben.}\\
Bis vor kurzem hatte er noch eher das Bedürfnis gehabt, dass Mädchen in den Arm zu schließen und 
trösten über den Kopf zu streichen. \textit{Vielleicht schaffen wir es ja, diese Jahre auf ein paar 
Monate zu reduzieren. Der Bursche scheint immerhin hilfreich zu sein.}\\
Jozah beschloss es ihr nicht schwerer als nötig zu machen und rutschte aus dem Sattel. Seine 
Glieder waren steif vor Kälte und der Anstrengung des Ritts, aber er verneigte sich tief und 
sprach: ``Verzeiht mir, Gräfin Sil'Vera. Die Kälte betäubt nicht nur die Muskeln, sondern 
anscheinend auch mein Gehirn.''\\
Er erwartete fast, dass der Bursche wieder sprechen würde, aber das Mädchen nickte und erwiderte: 
``Danke, dass Sie sich trotzdem auf den Weg gemacht haben um mich zu unterstützen. Ich bin mir 
sicher, im Winter werden wir vor dem Kamin sitzen und von der Wärme Brom-Dalars schwärmen.''\\
\textit{Ein Friedensangebot?}, versuchte er ihre versöhnlichen Worte zu interpretieren: \textit{Sie 
weiß, dass sie militärischen Rat braucht. Das ist ein Anfang.}\\
``Wir haben ein logistisches Problem'', fuhr sie fort und deutete mit einer wagen Handbewegung auf 
seine Truppe: ``Ich ahnte nicht, dass Sie mit so vielen Soldaten kommen. Ich werde den Stallmeister 
bitten, möglichst alle eurer Pferde zumindest heute Nacht einen trockenen und warmen Platz im Stall 
zu räumen.''\\
``So lange es nur das ist'', antwortete Jozah: ``Ich fürchtete schon, meine Leute müssten wieder 
unter freiem Himmel schlafen. Noch eine weitere Nacht und sie sind morgen nur noch Schlamm.''\\
Sein Scherz schien sie zu irritieren und er lächelte daraufhin nur verkniffen. \textit{Ich bin ein 
noch schlechterer Diplomat als Politiker!}\\

Am nächsten Morgen, gleich nachdem Jozah Briefe an Ilia und dem König abgeschickt hatte, um 
über seine Ankunft zu informieren, begann er damit sich einen Überblick zu schaffen. Am Mittag saß 
er immer noch am Schreibtisch der Gräfin und sichtete die Unterlagen und Briefkorrespondenzen mit 
den Generälen der Garnisonen, während das Mädchen still daneben saß und ihm über die Schulter sah. 
Ihre Anwesenheit störte ihn, aber er maß sie nicht an, sie aus ihrem eigenen Arbeitszimmer zu 
werfen. Außerdem empfand er es als einen Anfang, dass sie Interesse zeigte.\\
``Habt Ihr schon einen der Generäle persönlich getroffen?'', richtete er schließlich das Wort an 
sie.\\
Sarimé schüttelte den Kopf. ``Ich habe sie eingeladen, aber Sie haben die Briefe ja gelesen.''\\
``Und selbst reisen wolltet Ihr nicht?''\\
``Ich habe erst vor wenigen Wochen meinen Gatten zum Meer gebracht und viel zu tun. Außerdem 
Gäste...''\\
``Die Priester, ja, ich habe sie gesehen'', überlegte Jozah: ``Warum sind sie hier?''\\
Das Mädchen seufzte tief. ``Sie wollen mich kontrollieren. Wie Sie auch, Herr General. Ist das 
nicht offensichtlich? Sie schnüffelt herum und behandeln mich wie ein hysterisches, schwangeres 
Weib. Offiziell wollen sie die Segnung für mein Kind durchführen.''\\
Jozah lehnte sich im Stuhl zurück und dachte laut nach: ``Ich muss mit den Generälen sprechen. Es 
kann nicht sein, dass sie sich ihrer Gräfin verweigern. Außerdem gab es wohl ein paar Fälle von 
Plündereien in Kasir. Das gefährdet potentielle Verhandlungen.''\\
Sie runzelte die Stirn. ``Welche Verhandlungen? Die Priester wollen Krieg. Ich dachte, das Ziel ist 
es, Ländereien zu erobern.''\\
Jozah kratzte sich am Kinn. ``Was sich der Hohepriester Brom-Dalars auch gedacht haben mag, so 
leicht wird es bestimmt nicht. Unsere Kolonien sind kein vereintes Land gewesen, sondern kleine 
Völker die sich untereinander bekriegten. Kasir ist ein Königreich. Es hat einen Herrscher, 
einen Glauben und eine Geschichte, was es vereint. Saleica wird noch nicht verhandeln. Aber 
vielleicht könnte man es Euch zu lasten legen, wenn Ihr irgendwann an einen Punkt kommt, dass das 
feindliche Heer vor Euren Tor steht und Ihr um Gnade ersucht.''\\
Sie blickte ihn erschrocken an und legte schützend eine Hand auf ihren noch flachen Unterleib. 
``Sie meinen, die Kasira werden wirklich so weit ins Landesinnere kommen? Hier her?''\\
Jozah zuckte mit den Schultern. ``Woher soll ich das wissen? Aber was soll sie denn abhalten? 
Abgesehen davon, dass sie diese abgelegene Burg vielleicht gar nicht erst finden. Ich habe mich 
auch drei Stunden lang verirrt... Da kommen wir zum nächsten Punkt. Ihr solltet darüber nachdenken, 
Euch einen Wohnsitz zu wählen, der Euch näher an das Geschehen bringt.''\\
``Näher? Ich dachte, ich muss ohnehin in wenigen Monaten um mein Leben betteln!''\\
Jozah war sich nicht sicher, ob die junge Gräfin scherzte oder ihr Temperament mit ihr durch ging. 
Beschwichtigend murmelte er: ``Ich meinte ja nur, falls es so weit kommt...''\\
Sarimé richtete sich stur auf und verschränkte die Arme vor der Brust. ``Also, was meinen Sie dann? 
Ich soll die Burg verlassen. Wohin soll ich?''\\
``Na'Rash? Ist die größte Stadt in Merandila'', sagte Jozah wie zu sich selbst und studierte die 
Karte vor sich. ``Dann sieht Euer Volk Euch auch mal. Und um die Generäle werde ich mich 
kümmern.''\\
Das Mädchen lehnte sich an den Tisch und fügte hinzu: ``Verbinden wir es mit der Segnung meines 
Kindes. Sie können unmöglich die Einladung zu einer religiösen Feierlichkeit ablehnen, ohne auch 
den Hohepriester zu beleidigen. Immerhin besteht er darauf, es durchzuführen. Oh... er wird sich 
freuen, den ganzen Weg hier her um sonst gereist zu sein.''\\
``Gut. Dann lese ich mich hier noch etwas ein. Könnt Ihr die Vorbereitungen für den Umzug und die 
Segnung treffen?'', fragte Jozah direkt.\\
Skeptisch blickten ihn die grünen Augen an, während sie erwiderte: ``Natürlich. Oder erwarten Sie, 
dass ich das erst mit meinem Kindermädchen absprechen muss?''\\
\textit{Na, allein wird sie ja zu einer richtigen Wildkatze. Oder doch eher ein Raubvogel?}, dachte 
Jozah und verzog keine Miene. ``Nicht unbedingt Kindermädchen, aber vielleicht den Bastard.''\\
Ihre Miene wurde eisig, als sie sich abwandte und erhobenen Hauptes das Zimmer verließ. Mishka, der 
gerade eintreten wollte, drückte sich schnell zur Seite und sah ihr hinterher. Leise lachend ließ 
er die Tür ins Schloss fallen. ``Woah... du machst ihr aber schnell den Hof! Hat das so bei deiner 
reizenden Blondine auch geklappt?''\\
``Ich wollte wissen, wie sie reagieren würde.''\\
``Ich kann dir versichern, falls Tränen in ihren Augen waren, waren sie gefroren und spitz wie ein 
Dorn! Aber das hätte ich dir gleich sagen können. Auch wenn sie jung ist, ist sie eine Dame der 
Hauptstadt. Die versteinern eher, als dass sie weinen. Ich hab da so meine Erfahrungen. Kommt aber 
auch ganz darauf an, wie edel der Adelstitel ist. Ein schmaler Grad zwischen tränenreiche Flüche 
und dem da.''\\
Jozah tunkte die Feder in das Tintenfass. ``Ah... du hast viele Bekanntschaften mit den 
Damen gemacht, ich weiß. Und? Was ist mit ihr?''\\
Mishka lümmelte sich in einen der Sessel und legte die Stiefel auf den Kirschholztisch, der mit 
einer Porzellanvase bestückt war. ``Tja... ich habe mit Mibellra gesprochen, sie arbeitet in der 
Küche. Eine ganz süße Stupsnase und Sommersprossen...''\\
Jozah winkte ungeduldig mit der Feder. Ein Tintentropfen fiel auf das Blatt und verärgert legte er 
es zur Seite um ein frisches zu nehmen.\\
``Ja also sie hat erzählt, der Bastard hatte wohl was mit der verstorbenen Gräfin. Auch sehr 
hübsch, ich habe ein Gemälde gesehen.''\\
``Mishka!''\\
``Ist ja gut. Die hat sich umgebracht, das war ja kein Geheimnis. Aber ob er sich jetzt gleich an 
die neue rothaarige Gräfin ran gemacht hat, da erzählt jeder etwas anderes. Der Hauptmann der Wache 
ist wohl auch oft mit ihr allein...''\\
``Ist einer von den beiden relevant?'', fragte Jozah und begann den Titel eines der Generäle 
zuschreiben.\\
Mishka legte den Kopf schief und betrachtete seine Fingernägel. ``Ich glaube nicht. Die Leute mögen 
den Bastard sehr. Er ist für sie ein Merandil und Sohn ihres verstorbenen Grafen... Der Hauptmann? 
Pah... der hat doch keine Ahnung von einer richtigen Schlacht. Es wäre natürlich besser für das 
Mädchen, wenn sie mit dem Bastard das Bett teilt, dann sieht ihr Kind wenigstens so aus wie ihr 
Gatte und sie könnte eher einen Rechtsanspruch auf die Grafschaft gelten machen, anstatt wenn das 
Blag plötzlich blond ist...''\\
Jozah nickte zufrieden über die Informationen und fügte hinzu: ``Elor soll sie im Auge behalten. Er 
ist... unauffälliger als du. Und weniger direkt. Du kannst mir hier helfen. Finde mal raus, in 
welchem Maßstab die Karte geschrieben ist, ich kann die Legende nicht lesen. Wie kann man nur so 
eine furchtbare Handschrift haben.'' Seine letzten Worte ging in einem murmeln unter, während er 
stirnrunzelnd die Karte beiseite legte um sich nun dem Brief zu widmen.\\

