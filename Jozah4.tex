\chapter{Ein Mädchen (Anschluss an Sarimé 4)}

Trotz der tiefenden Nässe seines Mantels, dem kalten Wind und den wartenden Blicken die auf ihm 
ruhten, konnte Jozah sich noch nicht dazu überwinden, aus dem Sattel zu steigen. Er war mit vielen 
Erwartungen hier her in den Norden geritten. Viele haben sich widersprochen. Das mit dem Wetter 
stimmte schon mal, es war scheußlich. Er konnte sich nicht entsinnen, jemals tagelang am Stück 
durchnässt gewesen zu sein. Auf der letzten Reisestrecke nach dem Schiff hatten sich auch kaum 
akzeptable Unterkünfte finden lassen, in denen seine Leute und ihre Pferde alle untergekommen 
wären. Aus Solidarität hatten sie sich abgewechselt, wer eine Nacht im trockenen verbringen durfte. 
Aber das widerlichste war ja dann doch, wieder in die nasse Kleidung steigen zu müssen. Nicht 
einmal das Fell der Pferde schien trocknen zu können.\\
Und dann waren da natürlich noch die Erwartungen, die er sich über die junge Witwe eingeprägt 
hatte. Jozah hatte bisher nicht viel Kontakt mit Frauen in Machtpositionen, außerhalb des Militärs. 
Ilia hatte solch eine Position, durch ihren Namen und ihr Geld. Sie ging spielerisch damit um, 
lachte charmant und wirkte stets, als ob sie alles im Griff hätte und nichts ihren Leichtmut stören 
könnte. Trotz dem Wissen, dass die Witwe erst 16 Jahre ist, hatte er sich ein ähnliches Verhalten 
vorgestellt. Immerhin war Sarimé Sil'Veras Name so alt wie die Hauptstadt selbst.\\
Nach dem ersten Anblick hatte er sich suchend im Hof umgesehen, aber die schlanke Gestalt war das 
einzige weibliche Wesen, welches jung genug war um besagte Gräfin zu sein. Jozah war sich seiner 
Unhöflichkeit bewusst, so war es weit über seinem Stand, hier einfach wartend im Sattel zu sitzen. 
Aber er konnte sich einfach nicht lösen. Das Mädchen war umringt von einer kleinen Gruppe Männern. 
Drei davon waren deutlich als Priester auszumachen, einer von ihnen dermaßen Tatoowiert, dass es 
sich nur um einen Hohepriester handeln konnte. Dann fielen ihm noch zwei Wachen auf und ein 
Bursche, der ihn entrüstet anfunkelte. Er stand nahe bei der Gräfin und Jozah entschied, dass diese 
beiden jungen Leute wohl vertraut miteinander waren.\\
\textit{Na... die junge Witwe wird sich doch nicht schon den nächsten ins Bett geholt haben?}\\
Er bereute diesen Gedanken doch jedoch schnell, als er sie musterte. Das rote Haar kringelte sich 
vor Nässe und Wassertropfen schimmerten in den einzelnen Strähnen. Sie sah kränklich blass aus, 
hier unter den Regenwolken. Die warme Kleidung verhüllte ihre Gestalt, aber Anzeichen einer 
Schwangerschaft konnte er noch keine erkennen. Jozah bewunderte Ilia für ihre Schönheit, ihren 
Charm und ihr Selbstbewusstsein. Dieses Mädchen hier war unverkennbar schön, aber sie wirkte jetzt 
in diesem Moment auch nur wie Kind, das zu jung war für diese Verpflichtungen.\\
Der Hohepriester räusperte sich und wollte zu sprechen beginnen, da sprang unvermittelt der Bursche 
dazwischen. Sein Blick war intensiv und seine Worte laut, als er zu sprechen begann: ``Gräfin 
Sarimé Sil'Vera, Witwe des Grafen Evin A'Riks, Herrin über die Ländereien und Menschen Merandilas, 
erwartet etwas mehr Respekt!''\\
Jozahs Blick huschte zu dem Mädchen. Sie sah den Burschen ebenso überrascht an wie alle anderen 
anwesenden, dann änderte sich jedoch ihre Miene. Die Verunsicherung wich und ein Funkeln trat in 
ihre Augen.\\
\textit{Na bitte}, dachte Jozah: \textit{Noch ein paar Jährchen Übung und Erfahrung, dann könnte 
sie vielleicht eine brauchbare Führungskraft abgeben.}\\
Bis vor kurzem hatte er noch eher das Bedürfnis gehabt, dass Mädchen in den Arm zu schließen und 
trösten über den Kopf zu streichen. \textit{Vielleicht schaffen wir es ja, diese Jahre auf ein paar 
Monate zu reduzieren. Der Bursche scheint immerhin hilfreich zu sein.}\\
Jozah beschloss es ihr nicht schwerer als nötig zu machen und rutschte aus dem Sattel. Seine 
Glieder waren steif vor Kälte und der Anstrenung des Rittes, aber er verneigte sich tief und 
sprach: ``Verzeiht mir, Gräfin Sil'Vera. Die Kälte betäupt nicht nur die Muskeln, sondern 
anscheinend auch mein Gehirn.''\\
Er erwartete fast, dass der Bursche wieder sprechen würde, aber das Mädchen nickte und erwiderte: 
``Danke, dass Sie sich trotzdem auf den Weg gemacht haben um mich zu unterstützen. Ich bin mir 
sicher, im Winter werden wir vor dem Kamin sitzen und von der Wärme Brom-Dalars schwärmen.''\\
\textit{Ein Friedensangebot?}, versuchte er ihre versöhnlichen Worte zu interpretieren: \textit{Sie 
weiß, dass sie militärischen Rat braucht. Das ist ein Anfang.}\\
``Wir haben ein logistisches Problem'', fuhr sie fort und deutete mit einer wagen Handbewegung auf 
seine Truppe: ``Ich ahnte nicht, dass Sie mit so vielen Soldaten kommen. Ich werde den Stallmeister 
bitten, möglichst alle eurer Pferde zumindest heute Nacht einen trockenen und warmen Platz im Stall 
zu räumen.''\\
``So lange es nur das ist'', antwortete Jozah: ``Ich fürchtete schon, meine Leute müssten wieder 
unter freiem Himmel schlafen. Noch eine weitere Nacht und sie sind morgen nur noch Schlamm.''\\
Sein Scherz schien sie zu irritieren und er lächelte daraufhin nur verkniffen. \textit{Ich bin ein 
noch schlechterer Diplomat als Politiker!}\\



