\chapter{Ein Mädchen}

Trotz der triefenden Nässe seines Mantels, dem kalten Wind und den wartenden Blicken die auf ihm 
ruhten, konnte Jozah sich noch nicht dazu überwinden, aus dem Sattel zu steigen. Er war mit vielen 
Erwartungen hier her in den Norden geritten. Das Gerücht mit dem Wetter stimmte schon mal, es war 
scheußlich. Er konnte sich nicht entsinnen, jemals tagelang am Stück durchnässt gewesen zu sein. Auf 
der letzten Reisestrecke nach der Schiffsfahrt hatten sich auch kaum akzeptable Unterkünfte finden 
lassen, in denen seine Leute und ihre Pferde alle untergekommen wären. Aus Solidarität hatten sie 
sich damit abgewechselt, wer eine Nacht im trockenen verbringen durfte. Aber das widerlichste war 
ja dann doch wieder in die nasse Kleidung steigen zu müssen.\\
Und dann waren da natürlich noch die Erwartungen, die er sich über die junge Witwe eingeprägt 
hatte. Nach dem ersten Anblick hatte er sich suchend im Hof umgesehen, aber die schlanke Gestalt 
war das einzige weibliche Wesen, welches jung genug war um besagte Gräfin zu sein. Jozah war sich 
seiner Unhöflichkeit bewusst, so war es weit über seinen Stand, hier einfach wartend im Sattel zu 
sitzen. Aber er konnte sich noch nicht von diesem Anblick nicht lösen. Das Mädchen war umringt von 
einer kleinen Gruppe Männern. Drei davon waren deutlich als Priester auszumachen, einer von ihnen 
dermaßen tatoowiert, dass es sich nur um einen Hohepriester handeln konnte. Zu dieser imposanten 
Gruppe gesellten sich noch zwei Wachen ein Bursche, der ihn entrüstet anfunkelte. Er stand nahe bei 
der Gräfin und Jozah entschied, dass diese beiden jungen Leute wohl vertraut miteinander waren.\\
\textit{Na... die junge Witwe wird sich doch nicht schon den Nächsten ins Bett geholt haben?}\\
Er bereute diesen gehässigen Gedanken jedoch schnell, als er sie musterte. Das rote Haar lockte 
sich vor Nässe und Wassertropfen schimmerten in den einzelnen Strähnen. Sie sah kränklich blass 
aus, hier unter den Regenwolken. Die warme Kleidung verhüllte ihre Gestalt, aber Anzeichen einer 
Schwangerschaft konnte er noch keine erkennen. \textit{Sie ist so jung.}\\
Der Hohepriester räusperte sich und wollte zu sprechen beginnen, da sprang unvermittelt der Bursche 
dazwischen. Sein Blick war intensiv und seine Worte laut, als er zu sprechen begann: ``Gräfin 
Sarimé Sil'Vera, Witwe des Grafen Evin A'Riks, Herrin über die Ländereien und Menschen Merandilas, 
erwartet etwas mehr Respekt!''\\
Jozahs Blick huschte zu dem Mädchen. Sie sah den Burschen ebenso überrascht an wie alle anderen 
Anwesenden, dann änderte sich jedoch ihre Miene. Die Verunsicherung wich und ein Funkeln trat in 
ihre Augen.\\
\textit{Na bitte}, dachte Jozah: \textit{Noch ein paar Jährchen Übung und Erfahrung, dann könnte 
sie vielleicht eine brauchbare Führungskraft abgeben.}\\
Bis vor kurzem hatte er noch eher das Bedürfnis gehabt, dass Mädchen in den Arm zu schließen und 
trösten über den Kopf zu streichen. \textit{Vielleicht schaffen wir es ja, diese Jahre auf ein paar 
Monate zu reduzieren. Der Bursche scheint immerhin hilfreich zu sein.}\\
Jozah beschloss es ihr nicht schwerer als nötig zu machen und rutschte aus dem Sattel. Seine 
Glieder waren steif vor Kälte, aber er verneigte sich tief und sprach: ``Verzeiht mir, Gräfin 
Sil'Vera. Die Kälte betäubt nicht nur die Muskeln, sondern anscheinend auch mein Gehirn.''\\
Er erwartete fast, dass der Bursche wieder sprechen würde, aber das Mädchen nickte stattdessen und 
erwiderte: ``Danke, dass Ihr Euch trotzdem auf den Weg gemacht habt um mich zu unterstützen. Ich 
bin mir sicher, im Winter werden wir vor dem Kamin sitzen und von der Wärme Brom-Dalars 
schwärmen.''\\
\textit{Ein Friedensangebot?}, versuchte er ihre versöhnlichen Worte zu interpretieren: \textit{Sie 
weiß, dass sie militärischen Rat braucht. Das ist ein Anfang.}\\
``Wir haben ein logistisches Problem'', fuhr sie fort und deutete mit einer vagen Handbewegung auf 
seine Truppe: ``Ich ahnte nicht, dass Ihr mit so vielen Soldaten kommt. Ich werde den Stallmeister 
bitten, möglichst alle eurer Pferde zumindest heute Nacht einen trockenen und warmen Platz im Stall 
zu räumen.''\\
``So lange es nur das ist'', antwortete Jozah: ``Ich fürchtete schon, meine Leute müssten wieder 
unter freiem Himmel schlafen. Noch eine weitere Nacht und sie sind morgen nur noch Schlamm.''\\
Sein Scherz schien sie zu irritieren und er lächelte daraufhin nur verkniffen. \textit{Ich bin ein 
noch schlechterer Diplomat als Politiker!}\\

Am nächsten Morgen, gleich nachdem Jozah Briefe an Ilia und König Semric geschrieben hatte, begann 
er damit sich einen Überblick zu schaffen. Am Mittag saß er immer noch am Schreibtisch der 
Gräfin und sichtete die Unterlagen und Briefkorrespondenzen mit den Generälen der Garnisonen. Das 
Mädchen saß seit einigen Stunden schon still daneben und sah ihm über die Schulter. Auch wenn sie 
kaum redete, entging ihr keine seiner Bewegungen. Legte er einen Brief zur Seite, nahm sie ihn an 
sich und studierte abwechselnd das Schriftstück und seine Mimik. Ihre Anwesenheit störte Jozah, 
aber er maß sie nicht an, sie aus ihrem eigenen Arbeitszimmer zu werfen.\\
``Habt Ihr schon einen der Generäle persönlich getroffen?'', richtete er schließlich das Wort an 
sie.\\
Sarimé schüttelte den Kopf. ``Ich habe sie eingeladen, aber Ihr habt die Briefe ja gelesen.''\\
``Und selbst reisen wolltet Ihr nicht?''\\
``Ich habe erst vor wenigen Wochen meinen Gatten zum Meer gebracht und viel zu tun. Außerdem 
Gäste...'', rechtfertigte sie sich.\\
Jozah sah nickend über die Ausrede hinweg. Er konnte das Mädchen ja verstehen. Sie sah nicht so 
aus, als könnte sie gegen einen trotzigen General bestehen.\\
``Die Priester, ja, ich habe sie gesehen'', überlegte Jozah: ``Warum sind sie hier?''\\
Das Mädchen seufzte tief. ``Sie wollen mich kontrollieren. Wie Euch auch, Herr General. Ist das 
nicht offensichtlich? Sie schnüffelt herum und behandeln mich wie ein hysterisches, schwangeres 
Weib. Offiziell wollen sie die Segnung für mein Kind durchführen.''\\
Jozah lehnte sich im Stuhl zurück und dachte laut nach: ``Ich muss mit den Generälen sprechen. Es 
kann nicht sein, dass sie sich ihrer Gräfin verweigern. Außerdem gab es wohl ein paar Plündereien in 
Kasir. Ein Bote des Königs erreichte mich vor zwei Tagen und teilte mir mit, dass ich diesen 
Berichten nachgehen soll. Solche Verbrechen gefährden potentielle Verhandlungen.''\\
Sie runzelte die Stirn. ``Welche Verhandlungen? Ich dachte, das Ziel ist es, Ländereien zu 
erobern.''\\
Jozah kratzte sich am Kinn. ``Was sich der Hohepriester Brom-Dalars auch gedacht haben mag, so 
leicht wird es bestimmt nicht. Unsere Kolonien waren kein vereintes Land, sondern kleine 
Völker die sich untereinander aus dem Weg gingen. Kasir ist ein Königreich. Es hat einen Herrscher, 
einen Glauben und eine Geschichte, was es vereint. Die Plündereien könnte man Euch zu Lasten legen, 
wenn irgendwann das feindliche Heer vor Eurem Tor steht und Ihr um Gnade ersucht.''\\
Sarimé blickte ihn erschrocken an und legte schützend eine Hand auf ihren noch flachen Unterleib. 
``Sie meinen, die Kasira werden wirklich so weit ins Landesinnere kommen? Hier her?''\\
Jozah zuckte mit den Schultern. ``Woher soll ich das wissen? Aber was soll sie abhalten? 
Abgesehen davon, dass sie diese Burg vielleicht gar nicht erst finden. Ich habe mich 
auch drei Stunden lang verirrt... Da kommen wir zum nächsten Punkt. Ihr solltet darüber nachdenken, 
Euch einen Wohnsitz zu wählen, der Euch näher an das Geschehen bringt.''\\
``Näher? Ich dachte, ich muss ohnehin in wenigen Monaten um mein Leben betteln!''\\
Jozah war sich nicht sicher, ob die junge Gräfin scherzte oder ihr Temperament mit ihr durch ging. 
Beschwichtigend murmelte er: ``Ich meinte ja nur, falls es so weit kommt...''\\
Sarimé richtete sich stur auf und verschränkte die Arme vor der Brust. ``Ihr meint, ich soll die 
Burg verlassen. Habt Ihr auch einen Vorschlag, wohin?''\\
``Na'Rash? Ist die größte Stadt in Merandila'', sagte Jozah wie zu sich selbst und studierte die 
Karte vor sich. ``Dann sieht Euer Volk Euch auch mal. Und um die Generäle werde ich mich 
kümmern.''\\
Das Mädchen lehnte sich an den Tisch und fügte hinzu: ``Verbinden wir es mit der Segnung meines 
Kindes. Sie können unmöglich die Einladung zu einer religiösen Feierlichkeit ablehnen, ohne auch 
den Hohepriester zu beleidigen. Immerhin besteht er darauf, es durchzuführen. Oh... er wird sich 
freuen, den ganzen Weg hier her um sonst gereist zu sein.''\\
``Gut. Dann lese ich mich hier noch etwas ein. Könnt Ihr die Vorbereitungen für den Umzug und die 
Segnung treffen?'', fragte Jozah direkt.\\
Skeptisch blickten ihn die grünen Augen an, während sie erwiderte: ``Natürlich. Oder erwartet 
Ihr, dass ich das erst mit meinem Kindermädchen absprechen muss?''\\
\textit{Na, da wird sie ja zu einer richtigen Wildkatze. Oder doch eher ein Raubvogel?}, dachte 
Jozah und verzog keine Miene. ``Nicht unbedingt Kindermädchen, aber vielleicht mit Bastarden.''\\
Ihre Miene wurde eisig, als sie sich abwandte und erhobenen Hauptes das Zimmer verließ. Mishka, der 
gerade eintreten wollte, drückte sich schnell zur Seite und sah ihr hinterher. Leise lachend ließ 
er die Tür ins Schloss fallen. ``Woah... du machst ihr aber schnell den Hof! Hat das so bei deiner 
reizenden Blondine auch geklappt?''\\
``Ich wollte wissen, wie sie reagieren würde.''\\
``Ich kann dir versichern, falls Tränen in ihren Augen lagen, waren sie gefroren und spitz wie ein 
Dorn! Aber das hätte ich dir gleich sagen können. Auch wenn sie jung ist, ist sie eine Dame der 
Hauptstadt. Die versteinern eher, als dass sie weinen. Ich hab da so meine Erfahrungen. Kommt aber 
auch ganz darauf an, wie edel der Adelstitel ist. Ein schmaler Grad zwischen tränenreiche Flüche 
und dem da.''\\
Jozah tunkte die Feder in das Tintenfass. ``Ah... du hast viele Bekanntschaften mit den 
Damen gemacht, ich weiß. Und? Was ist mit ihr?''\\
Mishka lümmelte sich in einen der Sessel und legte die Stiefel auf den Kirschholztisch, der mit 
einer Porzellanvase bestückt war. ``Tja... ich habe mit Mibellra gesprochen, sie arbeitet in der 
Küche. Eine ganz süße Stupsnase und Sommersprossen...''\\
Jozah winkte ungeduldig mit der Feder. Ein Tropfen Tinte fiel auf das Blatt und verärgert legte er 
es zur Seite um ein Frisches zu nehmen.\\
``Ja also sie hat erzählt, der Bastard hatte wohl was mit der verstorbenen Gräfin. Auch sehr 
hübsch, ich habe ein Gemälde gesehen.''\\
``Mishka!''\\
``Ist ja gut. Die hat sich umgebracht, das war ja kein Geheimnis. Aber ob er sich jetzt gleich an 
die junge Witwe ran gemacht hat, da erzählt jeder etwas anderes. Der Hauptmann der Wache 
ist wohl auch oft mit ihr allein. Er heißt Samos und hat sein Leben hier in diesem kleinen 
Kaff verbracht. Die anderen Wachen haben ihn zu ihrem Hauptmann gewählt.''\\
``Ist einer von den Beiden relevant?'', fragte Jozah und setzte die Feder aufs Pergament.\\
Mishka legte den Kopf schief und betrachtete seine Fingernägel. ``Ich glaube nicht. Die Leute mögen 
den Bastard sehr. Er ist für sie ein Merandil und Sohn ihres verstorbenen Grafen... Der Hauptmann? 
Pah... der hat doch keine Ahnung von einer richtigen Schlacht. Es wäre natürlich besser für das 
Mädchen, wenn sie mit dem Bastard das Bett teilt, dann sieht ihr Kind wenigstens so aus wie ihr 
Gatte und sie könnte eher einen Rechtsanspruch auf die Grafschaft gelten machen, anstatt wenn das 
Blag plötzlich blond ist...''\\
Jozah nickte zufrieden über die Informationen und fügte hinzu: ``Elor soll sie im Auge behalten. Er 
ist... unauffälliger als du. Und weniger direkt. Außerdem trau ich den Priestern nicht. Es schaden 
nicht, wenn ein fähigerer Soldat als ihre Bauernwachen auf sie achtet. Du kannst mir hier helfen. 
Finde mal raus, in welchem Maßstab die Karte geschrieben ist, ich kann die Legende nicht lesen. Wie 
kann man nur so eine furchtbare Handschrift haben.'' Seine letzten Worte ging in einem Murmeln 
unter, während er stirnrunzelnd die Karte beiseite legte um sich den Briefen zu widmen.\\


Der graue Wallach riss den Kopf abrupt hoch und scharrte mit den Hufen, als Renec den Sattelgurt 
fest anzog. Er hob seinen Fuß in den Steigbügel und stemmte sich vom Boden ab. Mit Schwung landete 
er im Sattel und suchte sich eine bequeme Position. Renec nahm die Zügel auf und schnalzte mit der 
Zunge. Sofort spitzte der Wallach die Ohren.\\
``Reite vorsichtig, Junge.''\\
Der Bastard musste sich zu einem Lächeln zwingen, als er dem Stallmeister zu nickte. Sein 
Abschied fiel wortlos aus, denn eben diesen Ritt trat er an, um einen anderen Mann für seinen 
Dienst zu berufen. Es tat Renec im Herzen weh, denn er konnte sich die Ställe und Weiden nicht mit 
einem neuen Meister vorstellen.\\
``Bastard!''\\
Es war Samos und Renec runzelte die Stirn. ``Was willst du?''\\
Der blonde Hauptmann der Wache grinste, als er die Zügel des Grauen ergriff und zu Renec empor 
blickte. ``Ich werde dich begleiten. Warte kurz, ja?''\\
``Wieso sollte ich?''\\
Samos‘ Stimme wurde leiser, sein Grinsen breiter. ``Weil ich sonst nicht garantieren kann, dass die 
Worte die ich an dich richten werde, von Anderen ungehört bleiben.''\\
\textit{Mistkerl.}
Der Bastard nickte nur und sah ihm schweigend zu, wie er den Stall betrat und nach einem Pferd 
rief. Er schwieg auch weiterhin, als Samos zu Pferd aus dem Gebäude ritt. Stattdessen trieb 
Renec seinen Wallach in einen flotten Trab. Der Herbstwind zerrte an ihnen und Renec 
behielt den Himmel im Auge. Er hoffte, dass der heutige Tag ohne Regen verstreichen würde. Erst als 
sie den Wald erreichten und der Wind sich in den Baumkronen verfing, zügelten sie die Tiere.\\
``Also sprich'', forderte Renec ihn ungeduldig auf: ``Was willst du von mir?''\\
\textit{Vielleicht haut er danach wieder ab.}\\
Samos lachte und der Bastard beschloss, dass er den Mann wirklich nicht leiden konnte.\\
``Ist sie nicht bezaubernd, unsere junge Herrin?''\\
Er nickte nur und sah den Wachmann flüchtig von der Seite an.\\
``Ich hoffe, du findest sie nicht so bezaubernd wie Sieva. Ich meine ja nur… eine angeheiratete 
Mutter reicht doch, oder?''\\
Renec versteifte sich. ``Halt‘s Maul!''\\
``Ich glaube, unser Graf wusste es. Ich mein ja nur… die ganze Burg wusste es. Ich wette, er wollte 
dich umbringen.''\\
``Hat er aber nicht.''\\
``Ja. Also… das ist natürlich alles nur Spekulation, aber ich denke, Sieva hat dir dein 
Bastardleben gerettet. Hätte sie sich nicht umgebracht, dann hätte er nicht noch mehr Angst 
bekommen, niemals ein legitimes Kind als Erben zu haben. Obwohl sie ihm ja auch keines schenken 
konnte. So jedoch konnte unser Graf sich auch sicher sein, dass sein Bastard ihm nicht einen 
Bastard unterjubelte. Ist das nicht irre witzig? Stell mir mal vor, Sieva hätte ein Kind von dir 
bekommen! Ich muss sagen, deine Selbstbeherrschung ist bewundernswert. Ich erwartete schon nach dem 
zweiten Satz, dass du mir eine rein hauen würdest.''\\
``Ich habe es mir vorgestellt.''\\
``Ah…''\\
Renec schüttelte den Kopf. ``Mein Vater suchte sich eine neue Braut. Er hat nicht damit gerechnet, 
dass er so bald sterben würde. Wenn er mich hätte umbringen wollen, dann hätte er es getan.''\\
``Vielleicht hatte er nur keine Zeit mehr dafür. Vielleicht hat er es erst erfahren, als er schon 
zu krank war.''\\
``Woher willst du das wissen?''\\
``Ich habe es ihm gesagt.''\\
Renec funkelte ihn zornig an und biss sich auf die Zunge bis er Blut schmeckte.\\
``Oh'', rief Samos lachend: ``Nun schau nicht so. Ich wollte unserem Grafen nur anvertrauen, 
dass er sich nicht um seine junge Gemahlin sorgen solle. Schließlich bist du ja noch da um dich um 
sie zu kümmern. Und lass das lieber mit der Faust. Du bist kein Kämpfer.''\\
Er ließ Samos nicht aus den Augen. ``Sarimé war stets an seiner Seite.''\\
``Unsere Herrin hat ihn einmal verlassen und den Garten aufgesucht. Aber ist das denn wichtig?''\\
``Was willst du?''\\
``Sie vertraut dir.''\\
``Nein.''\\
Samos lachte und deutete in die östliche Richtung. ``Wir müssen diese Abzweigung nehmen, nicht? Und 
doch, sie vertraut dir Bastard. Zumindest mehr als anderen.''\\
``Sie vertraut ihrer Schwägerin.''\\
``Oh ja… ein lustiger Haufen… Aber das ist egal. Denn ihre Schwägerin hat nicht deine Macht.''
Nun musste Renec trocken auflachen. ``Macht?''\\
``Du kennst die richtigen Leute.''\\
``Welche Leute?''\\
``Leute mit Waffen. Leute mit Ehrgeiz. Leute, die die Helle im Herzen tragen und eine Königin 
wollen. Sie könnte diese Königin sein.''\\
``Sie ist jung.''\\
``Es wäre nicht das erste Mal, dass die Jugend zum Symbol der Veränderung wird.''\\
Renec dachte einen Moment schweigend nach. Dann fragte er misstrauisch. ``Warum jetzt? Warum planst 
du ausgerechnet jetzt Hochverrat?''\\
``Doch nicht jetzt!'', lachte Samos: ``Bis es so weit sein wird, werden noch einige Jahreszeiten 
vergehen. Aber jetzt ist die Zeit den Sturm der Revolution zu sähen.''\\
Der Bastard seufzte. ``Dann sprich. Was ist dein Plan?''\\
``Triff diese Leute. Rede zu ihnen. Erzähl ihnen von unserer Königin.''\\
``Nein. Es gibt bereits einen Geschichtenerzähler. Es braucht keinen weiteren'', erwiderte Renec 
misstrauisch und warf Samos einen vielsagenden Blick zu.\\
``Unsere - deine und meine - Aufgabe ist es, sie zu einer Herrin zu machen. Die Leute müssen ihr 
folgen wollen,'' ergänzte der Hauptmann.\\
``Sie ist jung und unerfahren.''\\
``Offensichtlich'', rief Samos verärgert aus: ``Aber wann, wenn nicht jetzt? Willst du warten, bis 
der nächste Adelige kommt? Em'Hir ist da. Und er hat viel zu viel Macht. Er regiert über Na'Rash 
und somit über die wichtigsten Punkte Merandilas. Es geht um unsere Heimat, Renec! Es geht um unsere 
Zukunft! Um unsere Freiheit!''\\
``Dann bring die Leute dazu, dass sie diese Freiheit wollen. Und zwar bestenfalls mit Sarimé als 
Königin. Das wäre ein Anfang'', erwiederte Renec: ``Und hoffe, dass die Helle die Gebete ihres 
Volks erhöhrt.''\\
Samos starrte verbissen den Weg entlang. ``Die Helle lebt'', entschied er.\\


Das Leben auf dem Hof des Saleicaners war friedlicher, als Lavay gedacht hatte. Sie fühlte sich zum 
ersten Mal seit Imurs Tod angekommen. Sie fand Ruhe in den täglichen Haushaltspflichten und der 
Versorgung der drei verbliebenen Pferde. Die Gespräche mit Rotan blieben oberflächlich, handelten 
nur davon, wo sie frisches Wasser her bekam und von seinen Pferden. Lavay fragte nicht nach, warum 
er so still war. Sie fragte nicht, wieso der Stall leer und der Hof einsam war. Ebenso wenig fragte 
er nach ihrer Geschichte. Nur eines Tages dann, als er ihr zum Frühstück eine gut gefüllte Schale 
süßen Haferbrei reichte und sich mit einem schweren Seufzen ihr gegenüber nieder ließ, erwähnte sie 
beiläufig: ``Der Prophet Eliras sprach, dass die Seelen der Gegangenen mit dem Wind reisen. Aber 
auch sie brauchen ein Zuhause, zu dem sie zurückkehren können um zu rasten. In den ersten Jahren 
ist das ihr Grab, aber der Körper verwest und wird wieder eins mit der Natur. Die Seele kann diesen 
Ort dann nicht mehr finden. Wir Kasira vergaben daher Dinge, die unseren Gegangenen sehr am Herzen 
lagen oder die ihnen gleichen. Diese Dinge finden die Seelen immer und dann können sie unsere 
Tränen sehen, wenn wir dort an sie denken.''\\
Rotan sah von seiner Schüssel auf und blickte ihr lange ins Gesicht. Schließlich legte er den Löffel 
nieder und seine Schultern sackten ein. ``Sie war zu gut für diese Welt. Nicht stark genug für die 
Geburt unseres Kindes.''\\
``Wie hieß sie?''\\
``Illiana. Wie kommst du darauf?''\\
Lavay deutete in den Raum. ``Sie ist immer noch hier. Überall. Darüber erzählte der Prophet Eliras 
auch. Wenn wir zu viele Dinge der Toten horten, dann können sie nicht frei sein. Das ist noch 
schlimmer, als wenn sie keinen Ort zur Rast haben. Zu viel hält sie gefangen. Ein paar wenige 
Gegenstände reichen um ihnen eine Heimat zu bieten, aber man darf sie nicht fesseln.''\\
Rotan verbarg sein Gesicht in den Händen. ``Aber ich will nicht, dass sie geht!''\\
Lavay senkte nachdenklich den Blick und erzählte leise: ``Mein Bruder starb. Er wurde nicht 
begraben. Und ich konnte keine Gegenstände begraben. Wir besaßen nicht viel und alles was 
ihn ausmachte trug er vor seinem Tod bei sich. Es gibt für mich keinen Ort zum Trauern. Ich 
glaube, der Prophet hatte nicht vollkommen recht. Ich denke, unsere Toten spüren auch die Liebe 
zu ihnen. Manchmal denke ich, dass Imur bei mir ist. Er kommt mit dem Wind und geht auch wieder 
mit ihm, aber einen winzigen Moment lang ist es, als würden wir uns meinen Herzschlag teilen.''\\
Lavay sah zur Seite und horchte in sich hinein. Es fiel ihr schwer die richtigen Worte zu finden um 
ihre Gefühle zu beschreiben: ``Ich denke, es war gut, dass er kein Grab bekam. So findet er mein 
Herz, auch so weit weg von meiner Heimat. Aber ich glaube auch, dass deine Frau ziehen 
will. Ihre Seele wird zurückkehren an den Ort, den du ihr schenken wirst.''\\
``Nur um dann wieder zu gehen?'', fragte er bitter.\\
``Was glaubt ihr Saleicaner über den Tod? Habt ihr auch Propheten?'', fragte Lavay nach.\\
Rotan schüttelte den Kopf. ``Wir haben Priester, die von unserem Gott erzählen. Und die leben noch. 
Unsere Toten kehren ein in Osymas Hallen, wenn sie es würdig sind. Wenn nicht, enden sie wohl eher 
in der göttlichen Besenkammer.''\\
``Wie wird man würdig?''\\
Rotans Augen funkelten grimmig. Das Geschirr schepperte, als er es grob in den Spühleimer fallen 
ließ. ``Krieg. Kampf. Morde. Für den Allmächtigen.''\\
Lavay betrachtete den wütenden Mann ruhig. ``Das erklärt die Geschichten über euch. Du bist nicht 
sehr gläubig?''\\
Der Gutsbesitzer schnaufte wütend und verkniff sich Flüche, die er in der Anwesenheit einer jungen 
Frau nicht laut aussprechen wollte.\\
``Ich weiß nicht, welche Geschichten wahr sind'', sagte Lavay nachdenklich, während sie das 
Geschirr abräumte: ``Aber ich weiß, welche mir mehr Trost spenden.''\\
Auf dem Hof wurden Stimmen laut. Jemand rief Rotans Nachname. Er sah auf, griff nach seiner Weste 
und schlüpfte hinein, während er über die Schwelle trat. Lavay vernahm Sätze auf Saleicanisch und 
konnte ihre Neugierde nicht verbergen. Schüchtern trat sie zur Tür und spähte hinaus. Es waren zwei 
Männer die auf Pferden saßen, die deutlich teurer aussahen als der klägliche Rest, den Rotan nicht 
hatte verkaufen können. Rotans Stute preschte am Gatter der Weide vorbei und wieherte auffordernd. 
Ihr schien der Besuch zu gefallen. Die beiden Wallache reagierten mit gespitzten Ohren und 
witternden Nüstern, aber blieben still.\\
Der Mann auf dem grauen Tier sprach mit Rotan höflich und sachlich. Sein Begleiter saß steif im 
Sattel und sah sich misstrauisch um.\\
\textit{Ein Soldat}, schoss es Lavay durch den Kopf.\\
Sie sah seine formelle Kleidung und das kurze Schwert an seinem Gürtel. Ihr forschender Blick 
entging dem Mann auch nicht. Er unterbrach den Dunkelhaarigen brüsk und deutete auf die junge 
Frau. Lavay sah den finsteren Blick aus den Augen des Unterbrochenen und musste kurz grinsen. Falls 
er der Herr des Soldaten war, würde er bestimmt Ärger für diese Unhöflichkeit bekommen. Sie wartete 
darauf, aber der Mann sah sie ebenfalls nur an. Rotan sprach etwas und Lavay entging das Wort 
Kasira nicht. Er wirkte niedergeschlagen und kam langsam auf sie zu. In ihrer Sprache murmelte er: 
``Es tut mir leid, Lavay. Aber ich kann einen Gesandten der Gräfin nicht belügen.''\\
``Warum sind sie hier? Was geschieht jetzt? Werde ich eingesperrt?'', flüsterte sie so hastig, dass 
Rotan Probleme damit hatte, ihre Worte zu verstehen.\\
Aber der Herr kam dem Gutsbesitzer zuvor. Er hatte sein Reittier ein paar Schritte näher gelenkt 
und sprach in einem klaren, beinahe akzentfreien Kasira: ``Wie ist dein Name?''\\
Sie senkte ängstlich den Blick. ``Lavay.''\\
``Der Herr Arell sagte, saleicanische Soldaten hätten dich aus Kasir entführt und hier abgesetzt.''
Es war nicht wie eine Frage formuliert, aber er wartete scheinbar auf eine Antwort. Lavay zögerte. 
Sie wusste nicht, was sie noch hinzufügen konnte und sprach schließlich nur: ``Es ist Krieg.''\\
Er schien mit der Tatsache, dass sie hier bei Rotan war, unzufrieden. Lavay trat einen scheuen 
Schritt zurück in das Haus. Es war ein dummer Versuch sich zu schützen. Aber vielleicht konnte sie 
durch die Hintertür in den Wald rennen und dabei schon einen guten Vorsprung erreichen, bis er vom 
Pferd abgestiegen und ins Haus getreten war.\\
Zu ihrer Überraschung schüttelte er plötzlich den Kopf. ``Ich bin nicht für militärische Belange 
verantwortlich.'' Der Mann musterte sie ein weiteres mal flüchtig und fügte hinzu: ``Du musst nicht 
fliehen.''\\
Dann sprach er wieder zu Rotan in der Landessprache und Lavay kehrte in das Haus zurück. Völlig 
verwirrt und noch geschockt ließ sie sich auf einen Stuhl fallen und starrte auf das Muster im 
Holz. Es dauerte einige Minuten bis sie Hufgetrappel vernahm. Rotan kam zu ihr. In 
einer ebenso erschöpften Bewegung ließ er sich ihr Gegenüber auf einen Stuhl nieder und schwieg. \\
``Was wollten die Kerle?'', fragte sie.\\
``Die Kerle.'' Er lachte trocken. ``Diese Kerle, wie du sie nennst, wurden von der Gräfin 
geschickt.''\\
``Welcher Gräfin?''\\
``Die Gräfin Merandilas. Unsere Herrin. Sie verwaltet diese Grafschaft für unseren König Semric.''\\
``Du klingst komisch'', sprach Lavay ihre Gedanken aus. Rotan hatte einen Ton in der Stimme, den 
sie nicht deuten konnte.\\
Er seufzte. ``Ach… diese Gräfin ist noch ein Kind. Unser Graf vor ein paar Wochen erst verstorben. 
Und dann kommt auch noch der Krieg hier her. Ein Kind soll unser Land schützen… Aber diese Kerle… 
Der Schwarzhaarige ist der Bastard des verstorbenen Grafen. Eigentlich ein guter Mann. Ich habe ihn 
schon öfters gesehen. Immer höflich. Immer hilfsbereit. Den Anderen kannte ich nicht, aber er 
stellte sich als Hauptmann der Leibwache vor.''\\
``Die Beiden mögen sich nicht'', sagte Lavay.\\
Rotan blickte sie erstaunt an. ``Meinst du?''\\
Die junge Frau zucke nur mit der Schulter. ``Und was wollten sie?''\\
Der Gutsherr lächelte schief. ``Die Gräfin will, dass ich für sie arbeite. Ich soll ihr 
Stallmeister werden.''\\
Er sah, dass Lavay mit diesem Begriff nicht viel anfangen konnte und erklärte: ``Ich soll mich um 
ihre Pferde kümmern. Sie einreiten und züchten.''\\
``Aber das ist doch gut. Das ist doch das, was du hier auch schon gemacht hast. Nur mit deinen 
Pferden.''\\
``Hm ja… vermutlich würde ich einige Tiere dort im Stall wiedererkennen. Ich habe früher auch mit 
dem Grafen verhandelt und ihnen Deckhengste und gute Stuten verkauft um deren Zucht zu verfeinern. 
Aber dann muss ich meinen Hof hier verlassen.''\\
``Er ist doch bereits verlassen'', sagte Lavay verärgert.\\
Rotan senkte den Blick als Reaktion auf diese wahren Worte.\\
``Und was ist mit mir?'', fragte sie schließlich.\\
``Er sagte, es kam ein General mit der Order, unerlaubte Plündereien zu untersuchen. Du wärst eine 
Zeugin und der General würde dich befragen. Aber seinetwegen kannst du auch abauen und 
untertauchen.''\\
Lavay schluckte. Sie hatte einen Kasira getötet. Würden die Saleicaner sie dafür verurteilen oder 
feiern? Nach den Worten über ihren Gott hätte letzteres Lavay gar nicht mal überrascht...\\
``Ich bleibe bei dir'', sagte sie: ``Ich habe keine Heimat. Und du sprichst immerhin meine 
Sprache. Wann brechen wir zu deiner Gräfin auf?''\\
Sie konnte nicht verstehen, warum Rotan noch zögerte. Lavay hatte genug getrauert und wollte die 
Vergangenheit endlich hinter sich lassen. Die Zukunft versprach so viel mehr als die trostlose Zeit 
die hinter ihr lag. ``Was spricht denn dagegen?'', fragte sie ungeduldig.\\
Rotan ließ sich Zeit mit seiner Antwort, murmelte dann aber: ``Nichts... Überhaupt nichts...''\\

