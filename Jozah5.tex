\chapter{Na'Rash}

Der Umzug der Gräfin gestaltete sich umfangreicher als Jozah gedacht hatte. Das sah er aber als 
seinen eigenen Fehler an. Überraschenderweise schaffte das Mädchen es trotzdem - mit Hilfe von 
Delegationen - alles innerhalb einer Woche zu organisieren. Die Priester schickte sie zu erst los, 
gleich mit ihrem persönlichen und ältlichen Hauspriester. Einen Teil ihrer Garde folgte den ersten 
Bediensteten um eine Stadtvilla, welche sich seit der Heirat mit dem Grafen nun in ihrem Besitz 
befand, herzurichten. Das Mädchen selbst ritt als eine der Letzten, gefolgt vom Rest ihrer Garde, 
dem Bastard und ihrer Familie. Jozah blieb zurück, las in der Zeit vermutlich jeden Brief, den es 
im Arbeitszimmer des verstorbenen Grafen gab und arbeitete sich durch die militärischen Unterlagen, 
welche einen weiteren Raum in Anspruch nahmen. Teilweise waren die Briefe schon Jahrzehnte alt, 
zeugten von Briefwechseln mit Kasir oder den jeweiligen Generälen der stehenden Heere. Er fand 
sogar ein Pergament, welches aus König Kareens Hand persönlich verfasst war.\\
Sorgfältig zog Jozah die Tür zu und drehte den Schlüssel im Schloss. ``Das einzige Exemplar?'', 
fragte er und blickte Mires Sil'Vera an. Er mochte den Mann nicht sonderlich. Seine ständige 
Nervosität war ansteckend und weckte auch in Jozah eine seltsame Unruhe.\\
``Meine Schwester hat noch einen Schlüssel'', stammelte der Kaufmann und kniff den Mund zusammen.\\
Jozah nickte nur und ließ den Schlüssel in seiner Weste verschwinden. So alt die Unterlagen auch 
waren, einige waren schon allein aus geschichtlicher Hinsicht unbezahlbar.\\
``Sie reiten dann auch bald los?'', fragte Mires Sil'Vera und verlagerte sein Gewicht von einem 
Bein auf das andere.\\
Jozah nickte bedächtig. ``Ja. Wir reiten schnell und sollten nicht viel später als der Reisezug 
eintreffen. Die Gespanne werden es nicht sehr leicht gehabt haben, bei diesen Witterungen.''\\
Der Kaufmann meinte wohl, die Aussage durch ein heißeres Lachen untermalen zu müssen.\\
``Ich lasse zwei meiner Männer hier'', fügte Jozah hinzu, ohne eine Erklärung abzugeben.\\
Vielleicht würde er sie später wieder zu sich holen, aber die beiden brüteten eh gerade eine 
Erkältung aus und Jozah wollte den verbliebenen Personal auf der Burg nicht die Unterlagen 
anvertrauen.\\
``Hey'', begrüßte er seine Truppe und schwang sich in den Sattel.\\
Die Soldaten erwiderten den Gruß mit einem leichten nicken und griffen die Zügel auf.\\
``Na?'', fragte Mishka verschwörerisch und lenkte sein Pferd näher an Jozahs Schimmel: 
``Dramatisches Angaloppieren um diesen Hosenscheißer zu beeindrucken?''\\
``Ach Mishka'', seufzte Jozah: ``Wir befinden uns auf einer alten Burg in der merandilischen 
Einöde. Solche Späße sollte man sich für die Hauptstadt aufbewahren. Vielleicht noch Na'Rash.''\\
Der Blonde verzog bedauernd das Gesicht. ``Aber die Leute aus der Hauptstadt sind militärische 
Paraden doch gewohnt. Da ist das nichts Besonderes.''\\
Jozah hob die Hand und gab das Zeichen zum Aufbruch. Die Soldaten ritten paarweise unter dem 
Eingangstor hindurch. Dieses war zwar nun vergrößert, aber es befanden sich weiterhin noch Stützen 
in der Innenseite des Bogens und einzelne Arbeiter tummelten sich herum.\\
``Wir haben dich die ganze Woche kaum gesehen. Weihst du uns ein, Herr General?'', murmelte 
Mishka.\\
Jozah blickte sich flüchtig um. ``Ich weihe dich ein. Hast du noch mit Elor gesprochen, bevor er 
aufbrach?''\\
``Ja. Er hatte keine Lust zu warten, bis du dich aus den Papieren herausgegraben hast. Er meinte, 
ich soll dir sagen, dass er das Mädchen mag.''\\
Jozah runzelte kritisch die Stirn. ``Tatsächlich?''\\
Sein Kamerad zuckte mit den Schultern. ``Ja, das fand ich aus sehr faszinierend. Einzelheiten, was 
genau er an ihr mag, wollte er aber nicht benennen. Zumindest schwieg er verbissen, als ich ihn 
fragte, ob es an dem Hintern oder der Oberweite liegt. Ich glaube, er steht einfach auf Rothaarige. 
Die gibt's in seiner Heimat vielleicht nicht.''\\
Jozah sah seinen alten Freund geduldig an und wartete darauf, dass er zum Punkt kam.\\
``Er fand es wohl amüsant, wie sie mit den Priestern umgesprungen ist. Zwar etwas schüchtern noch, 
aber sie hat keine Diskussion zugelassen, was den Aufbruch nach Na'Rash anging. Dafür hat sie 
einiges zurück gesteckt, als es um die Einzelheiten der Segnung ging. Elor verglich sie mit unserem 
geschätzten König, die Worte werde ich aber nicht wiederholen. Aber was diesen Vergleich angeht, 
hat sie wohl gut abgeschnitten. In Elors Augen. ''\\
Jozah nickte bedächtig. Die Aussage überraschte ihn dann jedoch schon. Er konnte das Mädchen noch 
nicht ganz einschätzen. \textit{Vielleicht einfach nur zu jung,} grübelte er: \textit{Vielleicht 
sähe alles ganz anders aus, wenn wir noch mehr Zeit hätten. Mehr Jahre.}\\
``Jetzt bist du dran.''\\
``Die letzten Briefe an die Generäle der stehenden Heere sind raus. Zusammenfassende Berichte an 
den König ebenso. Wenn die Generäle der Gräfin ihre Gunst verweigern und nicht zur Segnung kommen, 
werde ich sie ersetzen müssen.''\\
``Darfst du das?'', fragte Mishka überrascht.\\
Jozah zögerte kurz und antwortete dann: ``Ich habe mit dem Hohepriester Na'Rashs gesprochen. Em'Hir 
sprach, dass ich seinen Segen habe.''\\
``Was bedeutet, dass der Hohepriester aus der Hauptstadt es erlaubt und somit auch der König'', 
schlussfolgerte Mishka: ``Seit wann holen wir uns die Erlaubnis für militärische Aktionen von einem 
Priester?''\\
``Seit der König außer Reichweite ist und das Netzwerk der Priester deutlich schneller und 
ausgeprägter scheint. Wieso so tun, als würde man den formellen Weg gehen, während es am Ende doch 
eh beim Priester landet?'', zischte Jozah leise: ``Ich bin dieses Spiel leid. Wir sind im Krieg, 
ich habe keine Zeit den Stolz eines Mannes zu schonen, der die Krone trägt, wenn ein anderer das 
Zepter in den Händen hält. Wir sind Soldaten und stehen im Diensten des Landes!''\\
Mishka funkelte ihn an. ``Und der Krone!'', rief er kopfschüttelnd.\\
Seine laute Aussage zog Blicke auf sich. ``Für die Krone'', wiederholte der Adelssohn brüllend: 
``Für unser Vaterland.''\\
Die Soldaten begannen zu grinsen. Einige reckten ihre Fäuste in die Luft, andere stießen ihren 
Pferden die Fersen in die Flanken und ließen sie einen Satz springen. ``Alle, die 
unser Brüllen hören!'' riefen sie: ``Alles, was das Sonnenlicht berührt!''\\
Jozah sah zu, wie seine Soldaten jubelnd an Talsmund vorbei zogen und unwillkürlich kam ihm in den 
Sinn, wie viele von diesen Männern und Frauen wohl in einem oder drei Jahren noch an seiner Seite 
reiten würden.

``Na'Rash ist alt-merandil und bedeutet warmer Fels. Tja, traurigerweise ist es trotzdem nicht 
viel gemütlicher als im Rest der Grafschaft, wenn es erst mal regnet. Sie müssen unbedingt im 
Sommer noch hier sein. Die Weiden gleichen den Savannen, die es in den Kolonien geben soll. 
Zumindest, wenn es nicht regnet. Dann sind sie nämlich doch eher wie der Rest in Merandila'', 
plapperte der Mann, während die Soldaten ihre Pferde durch die Straßen der Stadt führten.\\
Jozah seufzte innerlich und versuchte den Mann nicht zu frustriert anzusehen. Er fasste es als Witz 
auf, dass die Gräfin ihm diesmal jemandem entgegen schickte, der ihm den Weg zum Anwesen zeigen 
sollte. Aber musste es unbedingt dieser sein?\\
``Wie war nochmal dein Name?'', fragte Jozah, nur um das Thema zu wechseln. Gespräche über das 
Wette - vor allem über Merandilas unleidlichen Niederschlag - frustrierten ihn nur noch mehr.\\
``Arham, Herr General'', erklärte er: ``Ich bin einer der Köche.''\\
``Und du lebst schon lange in Na'Rash?''\\
Er nickte. ``Ich gehörte zu den Bediensteten, die das Stadtanwesen des Grafen verwalteten. War ein 
ruhiges leben'', lachte er: ``Unser Graf schätzte das Leben auf dem Land ja deutlich mehr. Kann ich 
auch verstehen. Für jemanden, der es gewohnt ist, durch die Wälder zu reiten, wirkt die Stadt 
vermutlich sehr staubig.''\\
``Ein Koch der so viel redet?'', rief Mishka, der weiter hinten lief.\\
Arham drehte sich im laufen um und lachte über den Witz, der eigentlich keiner war. ``Ich bin auch 
bekannt als ein guter Geschichtenerzähler! Die Leute kommen von weit her, um mich erzählen zu 
hören! Im Herbst haben die Menschen lieber ein Dach über den Kopf anstatt sich Abendaktivitäten zu 
suchen. Da kommen sie lieber zu mir und hören zu.''\\
``Besonders wenn es regnet!'', pflichtete Jozah bei und nickte verstehend: ``Sind wir bald da?''\\
``Ein Stück ist es noch. Sie hätten besser den südlichen Eingang genommen, Herr General. Dann 
hätten sie über die Weiden reiten können und nicht hier durch die vollen Straßen.''\\
``Ja'', murmelte Jozah: ``Wie dumm von mir. Das wird mir bestimmt nie wieder passieren.''\\

Das Stadtanwesen stach bis auf seine Größe nicht von den restlichen Gebäuden hervor. Lediglich der 
ungewöhnlich große Stall zeugte davon, dass der verstorbene Graf sich diesen Ort selbst ausgesucht 
hatte. Jozahs Soldaten brachten ihre Tiere unter, aber auch hier gab es wieder Platzprobleme und 
die meisten Tiere befanden sich wohl außerhalb der Stadtmauern auf den Weiden. Jozah begnügte sich 
damit, die Zügel seines Schimmels einem braunhaarigem Mädchen in die Hand zudrücken. Er hatte sie 
nur flüchtig in der Burg gesehen, glaubte aber, dass sie mit dem neuen Stallmeister gekommen war.\\
``Herr General'', rief die junge Gräfin.\\
Sie stand auf den Stufen des Hintereingangs und breitete die Arme einladend aus. ``Ich hoffe, die 
Reise ging schneller als unsere. Ich bin erst vor zwei Tagen eingetroffen.''\\
Er nickte knapp und trat zu ihr, um sich respektvoll zu verneigen. Dann ging er aber doch noch 
einen Schritt weiter - immerhin würde er eine lange Zeit mit ihr zusammen arbeiten müssen: 
``Ich schätze, Ihr wolltet es mir heimzahlen für die Art und Weise unserer ersten Begegnung?'', 
murmelte er.\\
Sarimé blinzelte mehrmals irritiert und deutete an, dass er doch eintreten solle. ``Ich weiß nicht 
recht, was Sie meinen, Herr General.''\\
``Der äußerst gesprächige Führer. Euer Koch. Er hat es fast geschafft, dass ich zurück in die 
Hauptstadt galoppiert wäre. Ich muss später mal nachzählen, ob meine Truppe noch vollständig 
ist.''\\
Sarimé führte ihn den Gang entlang und schüttelte schmunzelnd den Kopf. ``Verzeiht, aber was Sie 
anscheinend als scherzhaft empfunden haben, ging nicht auf meine Kosten. Ich bat Renec, jemandem 
vom Personal loszuschicken, der Ihnen den Weg zeigen würde. So weit ich es mitbekommen habe, hat 
sich der Koch freiwillig gemeldet. Aber ich hatte daran keinen weiteren Gedanken gerichtet.''\\
``Oh...'', murmelte Jozah: ``Dann muss ich wohl damit leben, dass einem solche Personen jederzeit 
unerwartet über den Weg laufen können. Ich verstehe immer besser, warum manche Menschen sich bei 
solchen Bedrohungen in den Glauben und Schutz des Allmächtigen flüchten.''\\
``War es denn so schlimm?'', lachte das Mädchen.\\
``Ihr solltet Euch einen anderen Koch suchen. Wenn er anfängt zu reden, hört er nicht mehr auf. Das 
habe ich bisher nur von Priestern erlebt.''
``Vielleicht ist das der Grund, wieso Evin ihn in die Küche gesteckt hat'', kicherte Sarimé.\\
``Der Termin für die Segnung ist geblieben?'', wechselte Jozah das Thema.\\
Die junge Gräfin nickte. ``Und die Priester sehr beschäftigt. Ich habe ihnen den Großteil der 
Planungen überlassen.''\\
Skeptisch runzelte der General die Stirn, aber ehe er etwas sagen konnte, fügte sie hinzu: ``Ich 
möchte mit lieber mit anderen Dingen beschäftigen. Wichtigeren. Sie sagten, dass wir damit rechnen 
müssten, dass Kasir weit vorstoßen könnte...''\\
Jozah räusperte sich. ``Das kann immer sein. Ich meine... es ist eher unwahrscheinlich, betrachtet 
man unsere stehenden Heere an der Grenze. Aber eine Garantie gibt es nicht.''\\
``Ich habe viel nachgedacht und denke, ich muss mich auf die Gespräche mit den anderen Generälen 
vorbereiten. Außerdem einen Überblick über unsere Ressourcen erhalten - das Volk soll so wenig 
leiden wie möglich. Merandila steht einer realen Gefahr gegenüber... versteht mich nicht falsch, 
General Mi'Kaé, ich zweifle nicht daran, dass Saleica diesen Krieg gewinnen wird. Aber wie wird 
meine Grafschaft danach aussehen?''\\
``Nun... das wollte ich auch eher mit den restlichen Generälen besprechen, aber der Plan des Königs 
ist es selbstverständlich, die Schlachten aus seinem Land herauszuhalten und bestenfalls das Chaos 
Kasir überlassen. Deshalb tendiere ich auch dazu, unsere Heere so schnell wie möglich ein gutes 
Stück in den Norden zu schicken, damit Kasir überhaupt keine Chance hat, den Krieg zu uns zu 
bringen.''\\
Jozah beobachtete die Mimik der Gräfin genau und erwartete die Reaktion. Sie nickte jedoch nur 
bedächtig. ``Dann gibt es vieles in kurzer Zeit zu besprechen. Nach der Zeremonie der Segnung ist 
vermutlich ein guter Zeitpunkt. Die Priester werden noch beschäftigt sein.''\\
Überrascht sagte er: ``Aber Herrin... das ist Euer Festtag. Und der Eures Kindes. Ihr erhaltet 
Osymas Segen!''\\
In ihren grünen Augen blitzte der Schalk. ``Was bringt meinem Kind der Segen eines Gottes, wenn 
sein Erbe zu Schlamm und Ruinen wird? Entschuldigt mich jetzt bitte, ich habe eine Verabredung in 
der Stadt. Bleiben Sie hier oder besitzen Sie ein eigenes Anwesen in Na'Rash?''\\
Jozah fuhr sich durch das Haar. ``Ich habe viel zu tun und würde es vorziehen den Kontakt zu Euch 
nicht zu verlieren'', wich er aus.\\
Sie nickte erneut und winkte einen der Bediensteten zu sich, um ihm zu bitten, Jozah ein Zimmer zu 
zeigen.\\
Doch deutlich erschöpfter, als er zunächst gedacht hatte, ließ er sich in einen der gepolsterten 
Sessel fallen und schloss einen Moment die Augen. \textit{Ich bin ein General.}\\

``Viel Aufwand oder wenig Aufwand?'', fragte Sarimé und starrte in den Spiegel.\\
Renec lümmelte auf dem Bett und blätterte durch einen er zerlesenen Romane, die Sarimé für die 
Reise wieder einmal ausgegraben hatte.\\
``Mittel'', brummte er: ``Stadthalter Ga'Leor ist vom Adel. Pedan ist... Pedan. Und der Dritte ein 
Priester, die kann man eh nicht zufrieden stellen.''\\
Sarimé seufzte und steckte sich eine bronzene Nadel ins Haar um die vorderen Haarsträhnen aus ihrem 
Gesicht zu halten. Sie machte sich nicht die Mühe, ihre Sommersprossen mit Puder zu bedecken und 
ließ ihre restlichen Locken offen über den Rücken fallen. Das warme Stoffkleid behielt sie an und 
legte sich nur noch ein Fell über die Schultern.\\
``Das Mondlicht fing sich in ihrem goldnen Haar. Wie gern hätte er hinein gegriffen und sie an sich 
gezogen. Aber er sah nur auf ihre glänzenden Lippen und der Moment verstrich. Die Ersehnte warf ihm 
einen letzten, leidenden Blick zu und folgte ihrem frisch angetrautem Mann hinaus aus dem 
Festsaal'', zitierte Renec: ``Ernsthaft? Ich hätte von dir anspruchsvollere Literatur erwartet.''\\
Sarimé wandte sich ihm zu und warf die Haarbürste nach ihm, welche er mit dem Buchrücken abfing. 
Grinsend tauchte er hinter der Deckung hervor.\\
``Meine Stiefmutter hat die wenigen Bücher, die wir besaßen, ausgesucht. Mein Vater erlaubte nur 
zwei neue im Jahr.''\\
``Wieso?'', fragte er neugierig: ``Hatte er kein Interesse an Literatur?''\\
``Er hatte mehr Interesse daran, Essen auf den Tisch zu bringen und dabei den Schein zu waren'', 
murmelte Sarimé und warf einen letzten Blick in den Spiegel: ``Wir können gehen.''\\
Doch Renec rührte sich nicht und sah sie nur überrascht an.\\
``Was ist denn?'', fragte sie ungeduldig und wurde zunehmend nervöser: ``Soll ich lieber doch etwas 
anderes ansehen? Sieht man den Bauch zu sehr? Du hast doch vorhin gesagt, man sieht nichts.''\\
``Dein Bauch ist noch zu klein, als das man etwas sehen könnte.''\\
``Nackt schon!''\\
Auch wenn Suja immer wieder sagte, dass sie doch stolz auf die bisher noch recht kleine Rundung 
ihres Bauches sein sollte. Sarimé fand es schrecklich. Sie gestand sich ein, dass man noch kaum 
etwas sah, aber die Angst davor, in wenigen Wochen wie eine Kugel auszusehen, ließ sie jetzt schon 
Panik bekommen. 
``Dazu kann ich nichts sagen'', spottete er und schüttelte den Kopf: ``Nein. Ich dachte nur, dass 
du... solche Umgebungen gewöhnt bist.''\\
Er deutete auf die Einrichtung des Gemachs, welches schicker als die Burg war. Auch sah man im 
ganzen Gebäude immer wieder vergoldete Bilderrahmen oder teuer aussehende Vasen. Das Anwesen war 
das, was Sarimés Stiefmutter vergeblich versucht hatte nachzustellen. ``Natürlich'', spottete 
Sarimé: ``Meine Stiefmutter sagte einmal zu mir, dass es kein Problem ist, so würde ich niemals in 
die Gefahr geraten, dass mir mein einziges, ansehnliches Kleid nicht mehr passt.''\\
Sie biss sich auf die Lippen und verstummte. Niemandem hatte sie bisher erwähnt, dass sie aus 
ärmlichen Verhältnissen kam und hatte es eigentlich auch nicht vorgehabt. Nicht einmal Suja und 
Mires wussten, wie schlimm es in den letzten Jahren noch geworden war.\\
Renec rappelte sich auf und öffnete ihr die Tür. ``Schade. Hätte man dir mehr zu Essen gegeben, 
wärst du vielleicht doch noch größer geworden und man müsste nicht ständig auf dich runter 
schauen.''\\
Die Gräfin funkelte ihn entrüstet an, war aber insgeheim dankbar darüber, dass er es mit einem 
Scherz abtat und nicht weiter darauf einging. ``Ich will Lavay dabei haben.''\\
``Wozu?'', fragte er und lief neben ihr durch die Gänge.\\
``Ich will mich weiter mit ihr unterhalten. Ich will ihre Geschichte erfahren. Eine Kasira spaziert 
nicht einfach so nach Saleica'', erklärte sie und dachte an die simplen Gespräche, die sie mit dem 
etwas gleichaltrigen Mädchen geführt hatte. Mehr ein Stammeln und auf Dinge deuten, als eine 
Unterhaltung, aber amüsant.\\
``Die kann ich dir auch erzählen'', erwiderte Renec.\\
``Als ob sie dir die Wahrheit sagen würde. Lass mich doch von ihr Kasirisch lernen.''\\
Lachend schüttelte er den Kopf. ``Ich kann mir nicht vorstellen, dass auch nur einer von euch 
beiden wirklich etwas dabei lernt, aber wenn du meinst... bei Fragen, kannst du dich ja gerne an 
mich wenden.''
Die kleine Stadtkutsche wartete schon Im Hof. Sie war Überdacht, aber zu drei Seiten hin offen. Der 
Kutscher saß auf einem schmalen Sitz und hielt die Zügel eines einzelnen Braunen in der Hand. Ehe 
sie sich von Renec in die Kutsche helfen ließ, grüßte sie mit einem nicken eine Gruppe von Mi'Kaès 
Soldaten und Rotan. Sie bat ihm um die Gesellschaft seiner Helferin, bekam aber gleich eine 
verneinende Antwort. ``Verzeiht, Herrin, aber wir haben viel zu tun. Die Pferde der Soldaten müssen 
versorgt werden! Und das sind mehr als fünfzig.''\\
Sarimé sah das ein und seufzte enttäuscht. ``Dann heute Abend, wenn sie Zeit und Lust hat.''\\
Renec wartete, bis sie den Hof verlassen hatten, ehe er neckend sagte: ``Von wegen, Sprache lernen. 
Du willst nur eine Freundin.''\\
Sie zog die Schultern hoch. ``Ist das verkehrt? Suja redet ständig von dem Kind. Samos von Verrat. 
Du... beleidigst mich meistens.''\\
``An ihr wird dich auch etwas stören. Du wirst es herausfinden, sobald sie saleicanisch kann'', 
spottete der Bastard.\\
``Bis es so weit ist, genieße ich ihre Gesellschaft.''\\
``Herrin!'', rief es und Samos trieb sein Reittier neben die Kutsche. Verärgert darüber, dass sie 
ohne eine Wache aufgebrochen war, blickte er sie an.\\
``Du schläfst nie, hm?'', fragte sie.\\
``Nicht in einer Stadt voller Priester'', zischte der Hauptmann ihrer Wache und wich auf dem Weg 
nicht mehr von der Seite ihrer Kutsche.

Im Salon warteten schon ein mit Tee und Gebäck gedeckter Tisch und ein freier Sessel auf die Gräfin 
Merandilas. Die amtierenden Stadthalter Na'Rash erhoben sich, als sie eintrat und neigten 
respektvoll ihre Köpfe. Pedan trat dabei einen Schritt auf sie zu und umfasste in einer innigen 
Geste ihren Unterarm. ``Es tut gut Euch zu sehen, Herrin'', murmelte er.\\
Sie schenkte ihm ein kurzes Lächeln und begrüßte dann die anderen beiden Männer mit den Worten: 
``Es freut mich sehr, endlich Eure Bekanntschaft zu machen.''\\
Ga'Leor, der gewählt Stadthalter des Volkes setzte sich als erster und schenkte ihr eine Tasse Tee 
ein. ``Ehrlich gesagt überraschte es mich etwas, dass Ihr es so eilig hattet. Ich ging davon aus, 
dass wir uns zum festlichen Anlass der Segnung kennenlernen.''\\
Renec hatte sich wie Samos zur Wand zurückgezogen. Sarimé warf einen flüchtigen Blick durch den 
Raum. Jeder von ihnen hatte zwei persönliche Wachen mitgebracht, die sich am Rand hielten und 
finster blickten. Das selbst der Priester darauf zurückgegriffen hatte, überraschte sie jedoch.\\
``Ich empfand es als wichtig, keine weitere Zeit zu verschwenden. Ich habe mich noch nicht dazu 
entschieden, wie lange ich hier in Na'Rash bleiben, angesichts des nahenden Krieges. Und für eben 
jenen gibt es einiges zu besprechen.''\\
``Mit Euren Generälen, gewiss'', warf der Priester ein: ``Aber was sollen wir schon tun können?''\\
Anders als die meisten Priester, denen sie bisher begegnet war, hatte er keine säuselnde Stimme. 
Stattdessen wirkte er ebenso arrogant und abschätzend wie der Vertreter des Adels. Gut möglich, 
dass er ein solcher war, bevor er in den Dienst Osymas trat.\\
``Einiges. Wir müssen das Volk schützen. Und da Na'Rash eine der größten Städte Merandilas ist, 
befindet sich hier eine große Anzahl der Bevölkerung.''\\
``Nicht so viele, wie das klingen mag'', warf Ga'Leor ein.\\
``Hier sind die Schulen. Die Bibliotheken. Jeder Bauer, der es sich leisten kann, schickt sein Kind 
hier her. Die Jugend Merandilas befindet sich hier in diesen Mauern. Sollten diese Mauern fallen, 
fällt unsere Zukunft!'', sprach Sarimé entschieden und strafte den Adeligen mit einem strengen 
Blick, weil er sie unterbrochen hatte.\\
``Wir sollten jemanden schicken, der unsere Mauern überprüft'', sagte Pedan nachdenklich und 
kratzte sich am Kinn: ``Und wir sollten Vorräte anlegen. Die Stadtwache... ist nicht sonderlich 
groß.''\\
``Es wird aber auch schwierig werden, weitere Wachen anzuheuern'', warf der Priester Sakan ein: 
``Das Geld der Grafschaft sollte in die Armeen und deren Versorgung fließen. Unser Blick sollte 
sich nach Norden richten, Gräfin!''\\
``Wieso schwierig?'', fragte Sarimé.\\
``Weil jeder der einen Hauch von Patriotismus empfindet, sich den Armeen anschließt um für Osyma zu 
kämpfen!'', rief Sakan: ``Es reicht schon, dass die ehrlosen Kämpfer sich als Wachen bei 
Privatpersonen anheuern lassen.''\\
``So wie Eure eigenen Männer?'', fragte das Mädchen und kostete den Tee.\\
Der Priester starrte sie zornig an und presste seine Lippen fest aufeinander, schwieg aber.\\
``Gut'', erwiderte sie: ``Ich werde mit meinen Generälen sprechen, ob sie eine Idee haben, wie die 
Sicherheit der Stadt gewährleistet werden kann. Vielleicht lassen sich ja einige ehrbare Soldaten 
finden, die vom Heer bezahlt und abgestellt werden, Merandilas Hauptstadt zu beschützen.''\\
``Warum fragt Ihr uns überhaupt?'', warf der Priester entrüstet ein.\\
Sarimé lächelte charmant. ``Weil ich die Unterstützung der Stadthalter benötige. Das wisst Ihr 
doch.''\\
Ga'Leor schien zu grübeln und musterte sie eingehend. Sie ließ den drei Männern Zeit zu überlegen 
und probierte einen der Kekse. Er war ihr zu trocken und sie legte ihn neben den Teller auf die 
bestickte Tischdecke. \textit{Der Mann hat einen seltsamen Geschmack.}\\
``Doch'', sprach der Adelige nun: ``Ich stimme dieser Überlegung zu. Hier in der Stadt befinden 
sich einige geschichtliche Kostbarkeiten. In Form von Büchern, aber auch Gebäuden. Es wäre eine 
Schande, wenn etwas zerstört werden würde.''\\
\textit{Und erst recht seine Gebäude und Kostbarkeiten.}
``Kasir wird nicht bis hier her kommen'', unterbrach Sakan: ``Deshalb ist es unnötig, über 
irgendwelche alten Gebäude zu reden.''\\
``Aber wenn die Gräfin doch anbietet, es aus ihrer Kasse zu bezahlen?'', warf Pedan ein: ``Die 
Stadt verliert dadurch nichts. Und ihre Chance, diese Sache möglichst unbeschadet zu überstehen, 
steigt.''\\
``Der Sold der Soldaten kann über die Verwaltung der Grafschaft laufen'', ergänzte Sarimé: ``Kost 
und Unterkunft müsste Na'Rash den Männern und Frauen gewähren.'' Sie schenkte Sakan ein weiteres 
Lächeln: ``Ebenso den heiligen Beistand des Tempels. Es müsste auch genügend Häuser als Unterkunft 
geben. Ich hörte, der Adel verlässt die Stadt und zieht sich in die südlicheren Gegenden Saleicas 
zurück?''\\
Ga'Leor rutschte unruhig herum. ``Hm ja. Aber das hat nicht unbedingt etwas mit den Krieg zu 
tun.''\\
``Natürlich nicht. Viele Kinder der reichen Häuser haben sich auch jetzt dem Militär angeschlossen, 
um Osyma und ihr Land zu ehren'', ergänzte Sakan.``\\
''Ganz unüblich ist die Abreise generell aber auch nicht``, fuhr Ga'Leor nach der Unterbrechung 
fort: ''Das liegt am Wetter hier. Kaum jemand, der nicht muss, verbringt den Herbst und den Winter 
hier im Norden.``\\
''Und der Rest schafft sein Geld weg, damit die Grafschaft - wenn es eng wird - nicht doch darauf 
zurückgreifen kann``, sagte Pedan spöttisch.\\
''Sollen sie ziehen``, murmelte Sarimé: ''Dann befinden sich weniger Nörgler in der Stadt, wenn es 
ernst werden sollte. Ach ja, da fällt mir ein: Ich möchte die Schulen schließen.``\\
Alle drei Männer - und auch einige der Wachen, einschließlich Renec - starrten sie sprachlos an.\\
''Aber``, fasste sich Pedan als erster: ''Die Schulen sind die Einzigen in Merandila! Ihr spracht 
doch von der Zukunft und dem Volk... wieso wollt Ihr die kostenlose Bildung verbieten?``\\
''Nicht verbieten``, korrigierte Sarimé: ''Aber ist es klug, die Schüler hier in der Stadt zu 
behalten, in Kriegszeiten? Schlimmstenfalls kommt es zu einer Belagerung. Durch die Schulen ist die 
Stadt überfüllt und das durchgehend. Die Vorräte für Notfälle werden jedoch nach der Anzahl der 
Erwachsenen mit festem Wohnsitz in der Stadt errechnet. Ich schätze mal, ein Drittel - wenn nicht 
mehr - der Menschen, die sich in Na'Rash befinden, sind Schüler ohne Einkommen und festem Wohnsitz 
innerhalb der Mauern. Dieser Anzahl können die Lagerhallen aber nicht gerecht werden. Also schicken 
wir die Schüler innerhalb der nächsten Wochen nach Hause.``\\
''Auf keinen Fall``, empörte sich Sakan und stand auf: ''Die Schulen unterstehen dem Tempel. Der 
Hohepriester wird damit niemals einverstanden sein.``\\
''Ja, aber er wird kaum gegen die Einstimmigkeit der Stadthalter widersprechen können, oder? 
Schließlich untersteht ihr dem Volk und nicht dem Tempel``, erklärte die junge Gräfin.\\
Sakan ließ sich nicht zu einer weiteren Antwort herab, warf ihr nur einen vernichtenden Blick zu 
und stürmte aus dem Salon.\\
Schweigend blickten die Verbliebenen zur offenen Tür. Pedan räusperte sich. ''War das Euer Plan?``\\
''Mein Plan ist das Volk zu schützen``, wiederholte Sarimé.\\
Ga'Leor sagte: ''Also ich kann mit dem Kerl persönlich auch nichts anfangen. Viel zu stur und 
verbohrt und er rennt doch immer erst zum Hohepriester, bevor er etwas sagt. Aber ganz so sinnvoll 
war das vermutlich auch nicht, denn wir müssen eine Einstimmigkeit erreichen.``\\
Sarimé lächelte ihm zu. ''Dann fangen wir damit an, dass Ihr mir Eure Meinung sagt.``\\
Er deutete mit einem Nicken in Pedans Richtung. ''Warum nicht seine?``\\
Sie war sich Pedans Loyalität gewiss und erwiderte weiterhin lächelnd: ''Wir sind uns einig.``\\
Ihr ehemaliger Haushofmeister nickte zustimmend. ''Ihre Argumente sind schlüssig und ich finde, wir 
haben uns bisher viel zu wenig Gedanken darüber gemacht, wie wir in diesen Kriegszeiten agieren. 
Sakan ist doch nur so aufgebracht, weil er vorhatte, wieder möglichst viel Geld der Stadtkassen in 
das nächste, religiöse Fest zu investieren.``\\
Ga'Leor blickte von einem zum anderen. ''Verstehe. Naja... was soll ich dann noch sagen. Ich bin 
hier offensichtlich der Einzige, den sich das Volk aussuchen konnte. Und daher sehe ich es in 
Gewissen Maßen als meine Pflicht, wenigstens zu versuchen, der Sache gerecht zu werden.``\\
''Erst jetzt?``, fragte Sarimé skeptisch.\\
''Ich habe schon länger mit dem Gedanken gespielt``, antwortete er grinsend: ''Aber es war nicht 
leicht, gegen Sakan und den vorherigen Stadthalter anzukommen. Er war kein Priester, aber die 
beiden waren stets einer Meinung und hätte ich mich quer gestellt, wäre gar nichts voran gekommen. 
Oder man hätte mich vergiftet. Warum denkt Ihr, ist Na'Rash für seine hohe Religiosität 
bekannt? Ich traue mich gar nicht zu sagen, wie viel Geld aus der Kasse zu Gunsten des Tempels 
floss. Jedoch auch inklusive der Schulen.``\\
''Trotzdem zu viel``, murmelte Pedan.\\
''Nun, das hat sich ja jetzt geändert. Traut Euch also``, schlug Sarimé vor: ''Die Einstimmigkeit 
werden wir noch erreichen.``\\
''Die Schulen zu schließen ist hart``, griff Ga'Leor das Thema wieder auf: ''Ich verstehe Euer 
Argument. Aber damit ist viel organisatorischer Aufwand verbunden.``\\
''Deshalb sollten wir möglichst früh damit beginnen``, rief sie.\\
''Ich muss über Eure Vorschläge nachdenken.``\\
''Wie lange?``, fragte Sarimé.\\
Er zuckte mit den Schultern. ''Sagen wir es so. Ich werde mich nicht öffentlich zu Euren Vorhaben 
bekennen, wenn Ihr mir keine Einstimmigkeit bewirken könnt.`` Er kniff seine Augen zusammen. 
''Ohne, dass jemand plötzlich verstirbt.``\\
Sarimé nickte nachdenklich. ''Einverstanden. Ich werde mich melden. Pedan? Mögt Ihr mich noch 
zurück zu meinem Anwesen begleiten? Es gibt bestimmt einige, die sich freuen würden, Euch mal 
wieder zu sehen.``\\
Der ehemalige Haushofmeister schmunzelte und nahm die Einladung dankend an.\\
''Vielen Dank, dass Ihr Zeit für diese spontane Sitzung hattet, Stadthalter``, verabschiedete die 
Gräfin sich: ''Und danke für den Tee.``\\

