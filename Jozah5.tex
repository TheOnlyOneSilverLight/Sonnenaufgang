\chapter{Na'Rash}

Der Umzug der Gräfin gestaltete sich umfangreicher als Jozah gedacht hatte. Das sah er aber als 
seinen eigenen Fehler an. Überraschenderweise schaffte das Mädchen es trotzdem - mit Hilfe von 
Delegationen - alles innerhalb einer Woche zu organisieren. Die Priester schickte sie zu erst los, 
gleich mit ihrem persönlichen und ältlichen Hauspriester. Einen Teil ihrer Garde folgte den ersten 
Bediensteten um eine Stadtvilla, welche sich seit der Heirat mit dem Grafen nun in ihrem Besitz 
befand, herzurichten. Das Mädchen selbst ritt als eine der Letzten, gefolgt vom Rest ihrer Garde, 
dem Bastard und ihrer Familie. Jozah blieb zurück, las in der Zeit vermutlich jeden Brief, den es 
im Arbeitszimmer des verstorbenen Grafen gab und arbeitete sich durch die militärischen Unterlagen, 
welche einen weiteren Raum in Anspruch nahmen. Teilweise waren die Briefe schon Jahrzehnte alt, 
zeugten von Briefwechseln mit Kasir oder den jeweiligen Generälen der stehenden Heere. Er fand 
sogar ein Pergament, welches aus König Kareens Hand persönlich verfasst war.\\
Am Tag seines Aufbruchs zog er sorgfältig die Tür zu und drehte den Schlüssel im Schloss. 
``Das einzige Exemplar?'', fragte er und blickte Mires Sil'Vera an. Er mochte den Mann nicht 
sonderlich. Seine ständige Nervosität war ansteckend und weckte auch in Jozah eine seltsame 
Unruhe.\\
``Meine Schwester hat noch einen Schlüssel'', stammelte der Kaufmann und kniff den Mund zusammen.\\
Jozah nickte nur und ließ den Schlüssel in seiner Weste verschwinden. So alt die Unterlagen auch 
waren, einige waren schon allein aus geschichtlicher Hinsicht unbezahlbar.\\
``Sie reiten dann auch bald los?'', fragte Mires Sil'Vera und verlagerte sein Gewicht von einem 
Bein auf das andere.\\
Jozah nickte bedächtig. ``Ja. Wir reiten schnell und sollten nicht viel später als der Reisezug 
eintreffen. Die Gespanne werden es nicht sehr leicht gehabt haben, bei diesen Witterungen.''\\
Der Kaufmann meinte wohl, die Aussage durch ein heißeres Lachen untermalen zu müssen.\\
``Ich lasse zwei meiner Männer hier'', fügte Jozah hinzu, ohne eine Erklärung abzugeben.\\
Vielleicht würde er sie später wieder zu sich holen, aber die beiden brüteten eh gerade eine 
Erkältung aus und Jozah wollte den verbliebenen Personal auf der Burg nicht die Unterlagen 
anvertrauen.\\
``Hey'', begrüßte er seine Truppe und schwang sich in den Sattel.\\
Die Soldaten erwiderten den Gruß mit einem leichten nicken und griffen die Zügel auf.\\
``Na?'', fragte Mishka verschwörerisch und lenkte sein Pferd näher an Jozahs Schimmel: 
``Dramatisches Angaloppieren um diesen Hosenscheißer zu beeindrucken?''\\
``Ach Mishka'', seufzte Jozah: ``Wir befinden uns auf einer alten Burg in der merandilischen 
Einöde. Solche Späße sollte man sich für die Hauptstadt aufbewahren. Vielleicht noch Na'Rash.''\\
Der Blonde verzog bedauernd das Gesicht. ``Aber die Leute aus der Hauptstadt sind militärische 
Paraden doch gewohnt. Da ist das nichts Besonderes.''\\
Jozah hob die Hand und gab das Zeichen zum Aufbruch. Die Soldaten ritten paarweise unter dem 
Eingangstor hindurch.\\
``Wir haben dich die ganze Woche kaum gesehen. Weihst du uns ein, Herr General?'', murmelte 
Mishka.\\
Jozah blickte sich flüchtig um. ``Ich weihe dich ein. Hast du noch mit Elor gesprochen, bevor er 
aufbrach?''\\
``Ja. Er hatte keine Lust zu warten, bis du dich aus den Papieren herausgegraben hast. Er meinte, 
ich soll dir sagen, dass er das Mädchen mag.''\\
Jozah runzelte kritisch die Stirn. ``Tatsächlich?''\\
Sein Kamerad zuckte mit den Schultern. ``Ja, das fand ich aus sehr faszinierend. Einzelheiten, was 
genau er an ihr mag, wollte er aber nicht benennen. Zumindest schwieg er verbissen, als ich ihn 
fragte, ob es an dem Hintern oder der Oberweite liegt. Ich glaube, er steht einfach auf Rothaarige. 
Die gibt's in seiner Heimat vielleicht nicht.''\\
Jozah sah seinen alten Freund geduldig an und wartete darauf, dass er zum Punkt kam.\\
``Er fand es wohl amüsant, wie sie mit den Priestern umgesprungen ist. Zwar etwas schüchtern noch, 
aber sie hat keine Diskussion zugelassen, was den Aufbruch nach Na'Rash anging. Dafür hat sie 
einiges zurück gesteckt, als es um die Einzelheiten der Segnung ging. Elor verglich sie mit unserem 
geschätzten König, die Worte werde ich aber nicht wiederholen. Aber was diesen Vergleich angeht, 
hat sie wohl gut abgeschnitten. In Elors Augen. ''\\
Jozah nickte bedächtig. Die Aussage überraschte ihn dann jedoch schon. Er konnte das Mädchen noch 
nicht ganz einschätzen. \textit{Vielleicht einfach nur zu jung,} grübelte er: \textit{Vielleicht 
sehe alles ganz anders aus, wenn wir noch mehr Zeit hätten. Mehr Jahre.}\\
``Jetzt bist du dran.''\\
``Die letzten Briefe an die Generäle der stehenden Heere sind raus. Zusammenfassende Berichte an 
den König ebenso. Wenn die Generäle der Gräfin ihre Gunst verweigern und nicht zur Segnung kommen, 
werde ich sie ersetzen müssen.''\\
``Darfst du das?'', fragte Mishka überrascht.\\
Jozah zögerte kurz und antwortete dann: ``Ich habe mit dem Hohepriester Na'Rashs gesprochen. Em'Hir 
gab mir seinen Segen.''\\
``Was bedeutet, dass der Hohepriester aus der Hauptstadt es erlaubt und somit auch der König'', 
schlussfolgerte Mishka: ``Seit wann holen wir uns die Erlaubnis für militärische Aktionen von einem 
Priester?''\\
``Seit der König außer Reichweite ist und das Netzwerk der Priester deutlich schneller und 
ausgeprägter scheint. Wieso so tun, als würde man den formellen Weg gehen, während es am Ende doch 
eh beim Priester landet?'', zischte Jozah leise: ``Ich bin dieses Spiel leid. Wir sind im Krieg, 
ich habe keine Zeit den Stolz eines Mannes zu schonen, der die Krone trägt, wenn ein anderer das 
Zepter in den Händen hält. Wir sind Soldaten und stehen im Diensten des Landes!''\\
Mishka funkelte ihn an. ``Und der Krone!'', rief er kopfschüttelnd.\\
Seine laute Aussage zog Blicke auf sich. ``Für die Krone'', wiederholte der Adelssohn brüllend: 
``Für unser Vaterland.''\\
Die Soldaten begannen zu grinsen. Einige reckten ihre Fäuste in die Luft, andere stießen ihren 
Pferden die Fersen in die Flanken und ließen sie einen Satz springen. ``Jeder, der unser Brüllen 
hört!'' riefen sie: ``Alles, was das Sonnenlicht berührt!''\\
Jozah sah zu, wie seine Soldaten jubelnd an Talsmund vorbei zogen und unwillkürlich kam ihm in den 
Sinn, wie viele von diesen Männern und Frauen wohl in einem oder drei Jahren noch an seiner Seite 
reiten würden.\\

``Na'Rash ist alt-merandil und bedeutet warmer Fels. Tja, traurigerweise ist es trotzdem nicht 
viel gemütlicher als im Rest der Grafschaft, wenn es erst mal regnet. Sie müssen unbedingt im 
Sommer noch hier sein. Die Weiden gleichen den Savannen, die es in den Kolonien geben soll. 
Zumindest, wenn es nicht regnet. Dann sind sie nämlich doch eher wie der Rest in Merandila'', 
plapperte der Mann, während die Soldaten ihre Pferde durch die Straßen der Stadt führten.\\
Jozah seufzte innerlich und versuchte den Mann nicht zu frustriert anzusehen. Er fasste es als Witz 
auf, dass die Gräfin ihm diesmal jemandem entgegen schickte, der ihm den Weg zum Anwesen zeigen 
sollte. Aber musste es unbedingt dieser sein?\\
``Wie war nochmal dein Name?'', fragte Jozah, nur um das Thema zu wechseln. Gespräche über das 
Wette - vor allem über Merandilas unleidlichen Niederschlag - frustrierten ihn nur noch mehr.\\
``Arham, Herr General'', erklärte er: ``Ich bin einer der Köche.''\\
``Und du lebst schon lange in Na'Rash?''\\
Er nickte. ``Ich gehörte zu den Bediensteten, die das Stadtanwesen des Grafen verwalteten. War ein 
ruhiges leben'', lachte er: ``Unser Graf schätzte das Leben auf dem Land ja deutlich mehr. Kann ich 
auch verstehen. Für jemanden, der es gewohnt ist, durch die Wälder zu reiten, wirkt die Stadt 
vermutlich sehr staubig.''\\
``Ein Koch der so viel redet?'', rief Mishka, der weiter hinten lief.\\
Arham drehte sich im laufen um und lachte über den Witz, der eigentlich keiner war. ``Ich bin auch 
bekannt als ein guter Geschichtenerzähler! Die Leute kommen von weit her, um mich erzählen zu 
hören! Im Herbst haben die Menschen lieber ein Dach über den Kopf anstatt sich Abendaktivitäten zu 
suchen. Da kommen sie lieber zu mir und hören zu.''\\
``Besonders wenn es regnet!'', pflichtete Jozah bei und nickte verstehend: ``Sind wir bald da?''\\
``Ein Stück ist es noch. Sie hätten besser den südlichen Eingang genommen, Herr General. Dann 
hätten sie über die Weiden reiten können und nicht hier durch die vollen Straßen.''\\
``Ja'', murmelte Jozah: ``Wie dumm von mir. Das wird mir bestimmt nie wieder passieren.''\\

Das Stadtanwesen stach bis auf seine Größe nicht von den restlichen Gebäuden hervor. Lediglich der 
ungewöhnlich große Stall zeugte davon, dass der verstorbene Graf sich diesen Ort selbst ausgesucht 
hatte. Jozahs Soldaten brachten ihre Tiere unter, aber auch hier gab es wieder Platzprobleme und 
die meisten Tiere befanden sich wohl außerhalb der Stadtmauern auf den Weiden. Jozah begnügte sich 
damit, die Zügel seines Schimmels einem braunhaarigem Mädchen in die Hand zudrücken. Er hatte sie 
nur flüchtig in der Burg gesehen, glaubte aber, dass sie mit dem neuen Stallmeister gekommen war.\\
``Herr General'', rief die junge Gräfin.\\
Sie stand auf den Stufen des Hintereingangs und breitete die Arme einladend aus. ``Ich hoffe, die 
Reise ging schneller als unsere. Ich bin erst vor zwei Tagen eingetroffen.''\\
Er nickte knapp und trat zu ihr, um sich respektvoll zu verneigen. Dann ging er aber doch noch 
einen Schritt weiter - immerhin würde er eine lange Zeit mit ihr zusammen arbeiten müssen: 
``Ich schätze, Ihr wolltet es mir heimzahlen für die Art und Weise unserer ersten Begegnung?'', 
murmelte er.\\
Sarimé blinzelte mehrmals irritiert und deutete an, dass er doch eintreten solle. ``Ich weiß nicht 
recht, was Ihr meint, Herr General.''\\
``Der äußerst gesprächige Führer. Euer Koch. Er hat es fast geschafft, dass ich zurück in die 
Hauptstadt galoppiert wäre. Ich muss später mal nachzählen, ob meine Truppe noch vollständig 
ist.''\\
Sarimé führte ihn den Gang entlang und schüttelte schmunzelnd den Kopf. ``Verzeiht, aber was Ihr 
anscheinend als scherzhaft empfunden habt, ging nicht auf meine Kosten. Ich bat Renec, jemandem 
vom Personal loszuschicken, der Euch den Weg zeigen sollte. Aber ich hatte daran keinen weiteren 
Gedanken gerichtet.''\\
``Oh...'', murmelte Jozah: ``Dann muss ich wohl damit leben, dass einem solche Personen jederzeit 
unerwartet über den Weg laufen können. Ich verstehe immer besser, warum manche Menschen sich bei 
solchen Bedrohungen in den Glauben und Schutz des Allmächtigen flüchten.''\\
``War es denn so schlimm?'', lachte das Mädchen.\\
``Ihr solltet Euch einen anderen Koch suchen. Wenn er anfängt zu reden, hört er nicht mehr auf. Das 
habe ich bisher nur von Priestern erlebt.''
``Vielleicht ist das der Grund, wieso Evin ihn in die Küche gesteckt hat'', kicherte Sarimé.\\
``Der Termin für die Segnung ist geblieben?'', wechselte Jozah das Thema.\\
Die junge Gräfin nickte. ``Und die Priester sind sehr beschäftigt. Ich habe ihnen den Großteil der 
Planungen überlassen.''\\
Skeptisch runzelte der General die Stirn, aber ehe er etwas sagen konnte, fügte sie hinzu: ``Ich 
möchte mich lieber mit anderen Dingen beschäftigen. Wichtigeren. Ihr sagtet, dass wir damit rechnen 
müssten, dass Kasir weit vorstoßen könnte...''\\
Jozah räusperte sich. ``Das kann immer sein. Ich meine... es ist eher unwahrscheinlich, betrachtet 
man unsere stehenden Heere an der Grenze. Aber eine Garantie gibt es nicht.''\\
``Ich habe viel nachgedacht und denke, ich muss mich auf die Gespräche mit den anderen Generälen 
vorbereiten. Außerdem einen Überblick über unsere Ressourcen erhalten - das Volk soll so wenig 
leiden wie möglich. Merandila steht einer realen Gefahr gegenüber... versteht mich nicht falsch, 
General Mi'Kaé, ich zweifle nicht daran, dass Saleica diesen Krieg gewinnen wird. Aber wie wird 
meine Grafschaft danach aussehen?''\\
``Nun... das wollte ich auch eher mit den restlichen Generälen besprechen, aber der Plan des Königs 
ist es selbstverständlich, die Schlachten aus seinem Land herauszuhalten und bestenfalls das Chaos 
Kasir überlassen. Deshalb tendiere ich auch dazu, unsere Heere so schnell wie möglich ein gutes 
Stück in den Norden zu schicken, damit Kasir überhaupt keine Chance hat, den Krieg zu uns zu 
bringen.''\\
Jozah beobachtete die Mimik der Gräfin genau und erwartete die Reaktion. Sie nickte jedoch nur 
bedächtig. ``Dann gibt es vieles in kurzer Zeit zu besprechen. Nach der Zeremonie der Segnung ist 
vermutlich ein guter Zeitpunkt. Die Priester werden noch beschäftigt sein.''\\
Überrascht sagte er: ``Aber Herrin... das ist Euer Festtag. Und der Eures Kindes. Ihr erhaltet 
Osymas Segen!''\\
In ihren grünen Augen blitzte der Schalk. ``Was bringt meinem Kind der Segen eines Gottes, wenn 
sein Erbe zu Schlamm und Ruinen wird? Entschuldigt mich jetzt bitte, ich habe eine Verabredung in 
der Stadt. Bleibt Ihr hier oder besitzt Ihr ein eigenes Anwesen in Na'Rash?''\\
Jozah fuhr sich durch das Haar. ``Ich habe viel zu tun und würde es vorziehen den Kontakt zu Euch 
nicht zu verlieren'', wich er aus.\\
Sie nickte erneut und winkte einen der Bediensteten zu sich, um ihm zu bitten, Jozah ein Zimmer zu 
zeigen.\\
Doch deutlich erschöpfter, als er zunächst gedacht hatte, ließ er sich in einen der gepolsterten 
Sessel fallen und schloss einen Moment die Augen. \textit{Ich bin ein General.}\\

``Viel Aufwand oder wenig?'', fragte Sarimé und starrte in den Spiegel.\\
Renec lümmelte auf dem Bett und blätterte durch einen der zerlesenen Romane, die Sarimé für die 
Reise wieder einmal ausgegraben hatte.\\
``Mittel'', brummte er: ``Stadthalter Ga'Leor ist vom Adel. Pedan ist... Pedan. Und der Dritte ein 
Priester, die kann man eh nicht zufrieden stellen.''\\
Sarimé seufzte und steckte sich eine bronzene Nadel ins Haar um die vorderen Haarsträhnen aus ihrem 
Gesicht zu halten. Sie machte sich nicht die Mühe, ihre Sommersprossen mit Puder zu bedecken und 
ließ ihre restlichen Locken offen über den Rücken fallen. Das warme Stoffkleid behielt sie an und 
legte sich nur noch ein weißes Luchsfell über die Schultern.\\
``Das Mondlicht fing sich in ihrem goldnen Haar. Wie gern hätte er hinein gegriffen und sie an sich 
gezogen. Aber er sah nur auf ihre glänzenden Lippen und der Moment verstrich. Die Ersehnte warf ihm 
einen letzten, leidenden Blick zu und folgte ihrem frisch angetrautem Mann hinaus aus dem 
Festsaal'', zitierte Renec: ``Ernsthaft? Ich hätte von dir anspruchsvollere Literatur erwartet.''\\
Sarimé wandte vom Spiegel ab und warf die Haarbürste. Mit einem dumpfen Geräusch fing der Buchrücken des Romans sie ab. 
Grinsend tauchte Renec hinter der Deckung hervor.\\
``Meine Stiefmutter hat die wenigen Bücher, die wir besaßen, ausgesucht. Mein Vater erlaubte nur 
zwei neue im Jahr.''\\
``Wieso?'', fragte er neugierig: ``Hatte er kein Interesse an Literatur?''\\
``Er hatte mehr Interesse daran, Essen auf den Tisch zu bringen und dabei den Schein zu waren'', 
murmelte Sarimé und warf einen letzten Blick in den Spiegel: ``Wir können gehen.''\\
Doch Renec rührte sich nicht und sah sie nur überrascht an.\\
``Was ist denn?'', fragte sie ungeduldig und wurde zunehmend nervöser: ``Soll ich lieber doch etwas 
anderes ansehen? Sieht man den Bauch zu sehr? Du hast doch vorhin gesagt, man sieht nichts.''\\
``Dein Bauch ist noch zu klein, als das man etwas sehen könnte.''\\
``Nackt schon!''\\
Auch wenn Suja immer wieder sagte, dass sie doch stolz auf die bisher noch recht kleine Rundung 
ihres Bauches sein sollte - Sarimé fand es schrecklich. Sie gestand sich ein, dass man noch kaum 
etwas sah, aber die Angst davor, in wenigen Wochen wie eine Kugel auszusehen, ließ sie jetzt schon 
Panik bekommen.\\
``Dazu kann ich nichts sagen'', spottete er und schüttelte den Kopf: ``Nein. Ich dachte nur, dass 
du... solche Umgebungen gewöhnt bist.''\\
Er deutete auf die Einrichtung des Gemachs, welches schicker als die Burg war. Auch sah man im 
ganzen Gebäude immer wieder vergoldete Bilderrahmen oder teuer aussehende Vasen. Das Anwesen war 
das, was Sarimés Stiefmutter vergeblich versucht hatte nachzustellen. ``Natürlich'', spottete 
Sarimé: ``Meine Stiefmutter sagte einmal zu mir, dass es kein Problem ist, so würde ich niemals in 
die Gefahr geraten, dass mir mein einziges, ansehnliches Kleid nicht mehr passt.''\\
Sie biss sich auf die Lippen und verstummte. Nicht einmal Suja und Mires wussten, wie schlimm es in 
den letzten Jahren im Hause Sil'Vera geworden war.\\
Renec rappelte sich auf und öffnete ihr die Tür. ``Schade. Hätte man dir mehr zu Essen gegeben, 
wärst du vielleicht doch noch größer geworden und man müsste nicht ständig auf dich runter 
schauen.''\\
Die Gräfin funkelte ihn entrüstet an, war aber insgeheim dankbar darüber, dass er es mit einem 
Scherz abtat und nicht weiter darauf einging. ``Ich will Lavay dabei haben.''\\
``Wozu?'', fragte er und lief neben ihr durch die Gänge.\\
``Ich will mich weiter mit ihr unterhalten. Ich will ihre Geschichte erfahren. Eine Kasira spaziert 
nicht einfach so nach Saleica'', erklärte sie und dachte an die simplen Gespräche, die sie mit der 
nur wenige Jahre älteren Frau geführt hatte. Mehr ein Stammeln und auf Dinge deuten, als eine 
Unterhaltung, aber amüsant.\\
``Die kann ich dir auch erzählen'', erwiderte Renec.\\
``Als ob sie dir die Wahrheit sagen würde. Lass mich doch von ihr Kasirisch lernen.''\\
Lachend schüttelte er den Kopf. ``Ich kann mir nicht vorstellen, dass auch nur einer von euch 
beiden wirklich etwas dabei lernt, aber wenn du meinst... bei Fragen, kannst du dich ja gerne an 
mich wenden.''
Die kleine Stadtkutsche wartete schon im Hof. Sie war überdacht, aber zu drei Seiten hin offen. Der 
Kutscher saß auf einem schmalen Sitz und hielt die Zügel eines einzelnen Braunen in der Hand. Ehe 
sie sich von Renec in die Kutsche helfen ließ, grüßte sie mit einem nicken eine Gruppe von Mi'Kaès 
Soldaten und Rotan. Sie bat ihm um die Gesellschaft seiner Helferin, bekam aber gleich eine 
verneinende Antwort. ``Verzeiht, Herrin, aber wir haben viel zu tun. Die Pferde der Soldaten müssen 
versorgt werden! Und das sind mehr als fünfzig.''\\
Sarimé sah das ein und seufzte enttäuscht. ``Dann heute Abend, wenn sie Zeit und Lust hat.''\\
Renec wartete, bis sie den Hof verlassen hatten, ehe er neckend sagte: ``Von wegen, Sprache lernen. 
Du willst eine Freundin.''\\
Sie zog die Schultern hoch. ``Ist das verkehrt? Suja redet ständig von dem Kind. Samos von Verrat. 
Du... beleidigst mich meistens.''\\
``An ihr wird dich auch etwas stören. Du wirst es herausfinden, sobald sie Saleicanisch kann'', 
spottete der Bastard.\\
``Bis es so weit ist, genieße ich ihre Gesellschaft.''\\
``Herrin!'', rief es und Samos trieb sein Reittier neben die Kutsche. Verärgert darüber, dass sie 
ohne eine Wache aufgebrochen war, blickte er sie an.\\
``Du schläfst nie, hm?'', fragte sie.\\
``Nicht in einer Stadt voller Priester'', zischte der Hauptmann ihrer Wache und wich auf dem Weg 
nicht mehr von der Seite ihrer Kutsche.\\

Im Salon warteten schon ein mit Tee und Gebäck gedeckter Tisch und ein freier Sessel auf die Gräfin 
Merandilas. Die amtierenden Stadthalter Na'Rash erhoben sich, als sie eintrat und neigten 
respektvoll ihre Köpfe. Pedan trat dabei einen Schritt auf sie zu und umfasste in einer innigen 
Geste ihren Unterarm. ``Es tut gut Euch zu sehen, Herrin'', murmelte er.\\
Sie schenkte ihm ein kurzes Lächeln und begrüßte dann die anderen beiden Männer mit den Worten: 
``Es freut mich sehr, endlich Eure Bekanntschaft zu machen.''\\
Ga'Leor, der gewählt Stadthalter des Volkes setzte sich als erster und schenkte ihr eine Tasse Tee 
ein. ``Ehrlich gesagt überraschte es mich etwas, dass Ihr es so eilig hattet. Ich ging davon aus, 
dass wir uns zum festlichen Anlass der Segnung kennenlernen.''\\
Renec hatte sich wie Samos zur Wand zurückgezogen. Sarimé warf einen flüchtigen Blick durch den 
Raum. Jeder von ihnen hatte zwei persönliche Wachen mitgebracht, die sich am Rand hielten und 
finster blickten. Das selbst der Priester darauf zurückgegriffen hatte, überraschte sie jedoch.\\
``Ich will keine weitere Zeit verschwenden, angesichts des nahenden Krieges. Und für diesen gibt 
es einiges zu besprechen.''\\
``Mit Euren Generälen, gewiss'', warf der Priester ein: ``Aber was sollen wir schon tun können?''\\
Anders als die meisten Priester, denen sie bisher begegnet war, hatte er keine säuselnde Stimme. 
Stattdessen wirkte er ebenso arrogant und abschätzend wie der Vertreter des Adels. Gut möglich, 
dass er ein solcher war, bevor er in den Dienst Osymas trat.\\
Sarimé biss sich auf die Zunge. Er hatte sie aus ihrem Konzept gebracht. ``Was wollt Ihr damit 
sagen?'', fragte sie.\\
``Wir verwalten die Stadt, Herrin'', fuhr der Priester Sakan fort: ``Wir verteilen die Finanzen, wir 
veranstalten Feste, wir organisieren die Stadtwache.''\\
Sie unterbrach ihn. ``Die Stadtwache! Wie groß ist sie?''\\
Pedan war es, der ihre Frage beantwortete, während er sich nachdenklich am Kinn kratzte. ``Bisher 
genug um in drei Schichten die Tore zu bewachen, sowie zu den Marktzeiten präsent zu sein.''\\
``Genug um im Falle einer Belagerung den Frieden in der Stadt zu wahren?'', fragte Sarimé.\\
Ga'Leor lachte.\\
Sakan warf ihm einen tadelnden Blick zu und erklärte: ``Bisher reichte die Anzahl um die gängigen 
Verbrechen wie Totschlag, Raub und Ketzerei einzudämmen. Durch den nahenden Krieg sieht das anders 
aus. Es befinden sich einige Soldatentruppen in der Stadt. Die ersten Bauern kommen aus den Dörfern 
um Schutz hinter unseren Mauern zu suchen. Die Schulen sind sowieso überfüllt, da die Erntezeit 
vorbei ist und jede merandilische Familie versucht, ihr Kind hier unter zu bringen. Besonders 
durch die Bauern steigt das Verbrechen.''\\
Diesmal machte der Adelige sich mit einem tiefen Seufzer bemerkbar. ``Ach bitte, ich hatte gehofft, 
dass diese Diskussion erst bei der zweiten Tasse aufkommt. Muss es denn jedes mal so enden?''\\
Sakan setzte zu einer zornigen Erwiderung an, da kam Sarimé ihm leise zuvor und fragte vorsichtig: 
``Welche Diskussion?''\\
``Ketzerei!'', spuckte Sakan aus, während Ga'Leor die Augen verdrehte und einen weiteren Schluck 
trank.\\
``Das Gesinde baut ihr Altäre! Sie feiern Osymas Feste nicht und flüstern ihre falschen Gebete!''\\
``Sie haben Angst'', murmelte Pedan: ``Sie sehen die Soldaten. Sie hören die Gerüchte. Sie wissen 
ganz genau, dass Merandila zu einem Schlachtfeld wird. Und sie ahnen auch, wer das veranlasst 
hat.''\\
Sakan sah ihn herausfordernd an, aber Pedan wandte den Blick ab.\\
Sarimé räusperte sich. ``Ketzerei wirkt im Angesicht dieser Tatsache zweitrangig...''\\
``Was ist mit den Bauern'', warf Ga'Leor ein um das Thema zu wechseln: ``Der Krieg ist noch fern 
und bereits jetzt reisen viele an. Wir haben nicht genügend Unterkünfte.''\\
Pedan nickte nachdenklich. ``Klingt es zu hart, sie nicht herein zu lassen?''\\
Sarimé sah ihn fassungslos an. ``Ja!'', kam sie jedem anderen zuvor.\\
``Jeder ist in den Tempeln willkommen'', erklärte Sakan: ``Solange es keine Ketzer sind.''\\
``Und die Schule?'', fragte Sarimé: ``Man könnte den Schulbetrieb einstellen und die Gebäude zu 
Unterkünften umfunktionieren.''\\
``Die Schule untersteht dem Tempel'', entgegnete Ga'Leor und blickte Sakan an: ``Und somit unserem 
Hohepriester und der Rektorin Silhe Basra. Aber unabhängig von der Unterkunft besitzen wir nicht 
genügend Ressourcen um halb Merandila in Na'Rash zu ernähren.''\\
``Unser König sagte, dass Kasir nicht so weit kommen wird!'', spuckte Sakan aus: ``Also sind diese 
Überlegungen unnütz. Wir sollten uns lieber Gedanken darüber machen, wie wir unsere Armeen 
unterstützen können und die Ketzerei eindämmen. Osyma wird uns nicht beistehen, wenn diese 
Banden weiter falsche Gebete sprechen.''\\
``Unser König ist in Brom-Dallar'', sagte Sarimé: ``Das ist eine weite Reise. Aber für die 
militärischen Belange schickte er mir einen seiner Generäle als Unterstützung. Für alles weitere 
verlässt sich König Semric darauf, dass wir die Vorkehrungen treffen um das Volk zu schützen. 
Wir haben nur noch wenige Monate. Wintermonate.''\\
``Nach der offiziellen Bekanntmachung werden unsere Soldaten durch Lebensmittelnachschub aus den 
umliegenden Grafschaften versorgt'', erklärte Pedan und blätterte einige Unterlagen durch: ``Für 
Merandila müssen wir sorgen.''\\
Sarimé fuhr sich frustriert durch das Haar. ``Wie kann es eine Grafschaft geben, die nur eine 
einzige Großstadt besitzt? Wie sollen die Dörfer und Höfe geschützt werden?''\\
Ga'Leor zuckte mit den Schultern und funkelte Sakan an. ``Tja...''\\
Pedan runzelte die Stirn. ``Es gibt einige Burgen wie die Eure, Herrin. Sie gehören einigen adeligen Familien. Ihr könntet befehlen, dass sie Bevölkerung aus einem größeren Radius aufnehmen. Wir müssten auf der Karte festlegen, welche Dörfer sich an welche Burg wenden sollen und Kundgebungen rausgeben. Das sollte möglichst bald geschehen, dass wir im Falle des Falles...''\\
Er wurde unterbrochen, als ein brennender Lumpen durch das offene Fenster flog. Er landete rechts 
von der Sitzgruppe, doch zwei, drei, vier weitere folgten. Hastig sprangen sie auf und stolperten 
zur Seite. Sarimés Knöchel streifte einen der Lumpen und hektisch gewann sie Abstand. Ein Gestank 
nach verkohltem Leder und Haar breitete sich aus. Renec zog sie hinter sich und Samos schlich sich seitlich an eines der Fenster heran, um hinauszuspähen. Sakans Wachen stürmten aus dem Zimmer, während 
Ga'Leors sich eilig daran machten, die Lumpen zu löschen. Sie kippten Tee darüber oder schlugen mit 
den seidenen Kissen darauf ein. Ga'Leor schwieg verbissen und beobachtete das Treiben in stiller 
Wut. Pedan hatte sich hinter einem Schrank versteckt und spähte bebend hervor. Und alles wurde 
überlagert von dem Gezetter des Priesters.\\
``Da habt ihr eure Ketzer!'', fauchte er und deutete auf das Fenster und die Lumpen.\\
``So etwas kam schon öfters vor?'', fragte Sarimé unsicher.\\
Ga'Leor nahm einen seiner Wachen das Kissen ab und betrachtete die schwarzen Flecken auf dem Muster 
aus Rosen und Efeu. ``Unter anderem ein Grund, wieso der Adel die Stadt verlässt. Eine Bande aus 
Kindern, die Unruhe stiftet. Aber sie sind nicht auffindbar, um sie zur Rechenschaft zu ziehen. Dem 
Gericht fehlen die Beweise.''\\
``Beweise gibts genug!'', entschied Sakan und trat gegen einen vor sich hin kohlenden Lumpen: 
``Schreiend ziehen sie durch die Gassen und beleidigen Osyma und den Tempel! Sie haben in der 
Schule angefangen. Bücher zerstört. Heilige Flammen gelöscht. Den Jüngeren ketzerische Lieder 
beigebracht!''\\
``Wie alt sind sie?'', fragte sie.\\
``Vermutlich zwischen vierzehn und siebzehn, Herrin'', murmelte Pedan, der sich vorsichtig aus 
seiner Deckung wagte: ``Aber wir wissen nicht, wie viele.\\
Ga'Leor verzog das Gesicht. ``Nur dumme Kinder. Kein Grund sie der Ketzerei anzuklagen. Die Ursache liegt doch ganz woanders.''\\
``Sie zerstören das Eigentum ehrbarer Saleicaner! Beschmutzen das Andenken unserer Ahnen! Verjagen 
sogar Bürger Na'Rashs.'' Sakan deutete auf die ruinierten Teppiche und Kissen.\\
Ga'Leor zuckte nur mit den Schultern und ließ das Kissen wieder fallen.\\
``Wie lange geht das schon so?'', fragte Sarimé.\\
Stille kehrte ein. Pedan wirkte, als hätte er sich lieber wieder versteckt. Unruhig verlagerte er 
das Gewicht von einem Bein auf das Andere. Ga'Leor sah seine Gräfin skeptisch mit hochgezogener Augenbraue 
an, während in Sakans Blick der blanke Hass durchkam. Der Adelige räusperte sich. ``Einige... unzufriedene und
weniger gläubige Menschen nahmen wohl die Tatsache, dass unser geschätzter und 
leider verstorbener Graf dem Meer übergeben wurde, als Anlass, an Osymas Allmacht zu zweifeln. 
''Und nun?``, flüsterte Sarimé leise.\\
Unschlüssig sah sie auf den Boden. Es hatte für sie damals so viel Sinn gemacht, diesen Kompromiss 
zwischen den Göttern einzugehen. Damals, als ihre Welt noch die Burg und Talsmund war.\\
\textit{Ich bin wirklich nur ein dummes Kind.}\\
Aber schlau genug um zu wissen, dass es keinen Kompromiss zwischen den Glauben mehr geben durfte. 
Sonst würde sie noch der Ursprung eines Bürgerkriegs werden.\\
''Verhaltet Euch Eurem Stand entsprechend!``, spuckte Sakan aus: ''Feiert die Segnung. Preiset 
Osyma. Und wagt nicht mehr, ketzerische Gedanken auszusprechen!``\\
\textit{Ich bin keine Ketzerin!}, dachte Sarimé zornig, schwieg aber.\\
''Und als vom Volk gewählt, werde ich eine Rede vorbereiten``, erklärte Ga'Leor und sah von einem 
zum anderen: ''Ich glaube, es wäre vom Nachteil, wenn sich eine der beiden anderen Parteien 
einmischt. Wir brauchen die Einheit und das Vertrauen des Volks um diesen Krieg zu überstehen.``\\

