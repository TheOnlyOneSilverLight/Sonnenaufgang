\chapter{Ziellos}

``Weißt du'', nuschelte er ohne den Becher abzusetzen: ``Sie ist wie so eine... also diese 
Wasserweiber da. Im Meer. Nein... schlimmer. Wie diese Nebelwesen. Warst du schon mal am Meer? Ach 
verdammt, ihr Merandil kommt ja nicht zu nem gescheiten Ozean!''\\
Die Magd lächelte nur liebenswürdig und füllte sein Trinkgefäß erneut mit Schnaps. Das hübsche 
Mädchen ließ sich neben Jozah auf die Bank fallen und tat so, als würde sie ein Haar von seiner 
Uniform wischen. ``Nebelwesen, Herr General?'', murmelte sie: ``Nein, davon habe ich wirklich noch 
nicht gehört.''\\
Jozah nahm einen weiteren Schluck. Mittlerweile schmeckte der Schnaps gar nicht mehr so schlecht. 
Zumindest deutlich besser als die ersten drei Becher, die er möglichst schnell herunter gewürgt 
hatte, damit der Schmerz nachließ. Das hatte nur nicht sonderlich gut geholfen.\\
``Sie nähern sich im Mondlicht den Schiffen'', flüsterte er verschwörerisch: ``Man spürt sie an der 
Kälte! Das Meer gefriert unter ihnen, der Wind verstummt und die Segel fallen in sich zusammen. Als 
würde die Zeit stehen bleiben. Alles ist wie erstarrt.''\\
``Und das hat eine Frau mit Euch getan?'', schlussfolgerte die Magd, stahl ihm mit geschickten 
Fingern den Becher aus der Hand und nahm ebenfalls einen Schluck.\\
``Dieses verlogene Biest'', knurrte er und schlug mit der Faust auf den Tisch. ``Verdammt, aua!''\\
``Wie tragisch'', sprach sie und winkte dem Wirt, damit er einen weiteren Krug brachte.\\
Jozah konzentrierte sich noch auf seine schmerzende Hand, da entging ihm der skeptische Blick des 
Wirts. Ebenso wie die Reaktion der Frau, die mit schnellen Fingern weitere Münzen aus Jozahs Beutel 
holte und auf den Tisch legte. ``Das ist ausreichend'', fügte sie lächelnd hinzu und tätschelte 
dann Jozah tröstend die Schulter. ``Wie ist ihr Name?''\\
Jozah kämpfte mit den Tränen, griff dann jedoch schnell zum Schnaps und nahm einen weiteren tiefen 
Schluck. Er hustete. ``Ilia'', versuchte er mit größtmöglicher Verachtung zu sagen, aber es klang 
eher wie ein erbärmliches Winseln. Er trug ihren Brief immer noch bei sich. Noch war keine offizielle Bekanntmachung in Merandila angekommen, aber Ilia bekam doch immer was sie wollte. ``Bald die Königin Saleicas.''\\
Die Magd runzelte die Stirn. ``Aber Herr General. Wollen Sie tatsächlich sagen, dass die Braut des 
Königs Ihnen das Herz gebrochen hat?''\\
``König'', spie Jozah aus und diesmal gelang es ihm, seinen Zorn Ausdruck zu verleihen: ``Dieser 
Welpe! Dieser Feigling! Was kann er denn, außer auf seinem hübschen Ross zu sitzen und zu winken? 
Brav zu nicken, wenn die Priester reden? Kriege zu veranlassen ohne jemals selbst gekämpft zu 
haben? Ja, was er kann? Dass kann ich dir sagen! Er kann meine Braut ficken!''\\
Jozah sprang auf die Beine, schwankte und musste sich am Tisch festhalten um nicht zu stürzen. 
``Semric der Zweite ist ein Wurm! Ein Häufchen Dreck!''\\
Die übrigen Gäste, die zu dieser späten Stunde noch die Taverne besuchten, starrten ihn nun 
unverhohlen an. Einer räusperte sich. ``Dafür, dass du die saleicanische Uniform trägst, laberst du 
ziemlichen Müll.''
``Saleica!'', fluchte Jozah und riss an dem gestickten Löwenkopf auf seiner 
Schulter. Da sich nichts tat, griff er zu dem Dolch an seinem Gürtel - sein Säbel war irgendwie 
verschwunden - und trennte den Stoff ab. Nicht ohne einen blutigen Schnitt in seiner Haut zu 
hinterlassen. ``Mist'', murmelte er leise, warf dann aber den Löwenkopf auf den Boden. ``Ich scheiß 
auf Saleica! Soll der Wurm sie doch ficken!''\\
Jozah hielt es für ziemlich aussagekräftig, seinen Dolch im Holz des Tisches stecken zu lassen, 
holte also aus, verfehlte den Tisch und fiel der Klinge hinterher zu Boden.\\

Sein Kopf schien zu explodieren. Er griff sich mit beiden Händen an die Schläfe, wie um sich zu 
vergewissern, dass sein Haupt noch da war. Mühsam rollte Jozah sich auf die Seite, schlug die Augen 
auf und versuchte herauszufinden, was er da sah. Er konnte nur die Beine eines Tisches erkennen. 
Und spürte die klamme Feuchtigkeit seiner Uniform. Ein lautes Stöhnen entwich ihm. Nichts auf der 
Welt wäre ein Grund, zu versuchen aufzustehen. Nicht einmal, dass der Boden übersät war von 
Schmutz und Scherben oder die Luft erfüllt von dem Gestank nach Alkohol und Urin.\\
``Und Ihr habt ihn hier liegen lassen?'', fragte eine weibliche Stimme skeptisch.\\
``Er hatte nicht mehr genug Geld für ein Zimmer'', kam die gebrummte Antwort.\\
``Oder genug Geld um jemanden dazu zu bringen, ihn die Treppe hinauf zu tragen'', fügte 
eine Frau spottend hinzu.\\
Jozah kniff die Augen fest zusammen und versuchte einzuordnen, wieso nun ein brennende Schmerz in 
seiner Schulter noch hinzukam. Ebenso die plötzliche Übelkeit.\\
``Herr... Kessla'', begann die skeptische Stimme wieder: ``Sie und ihre Tochter wollen mir also 
sagen, dass ein saleicanischer General die ganze Nacht hier verbracht hat um Schnaps zu trinken und 
dann auf dem Boden zu schlafen?''\\
``Genau genommen kam er nachmittags schon'', erklärte die Tochter: ``Und warum sollten wir lügen?''\\
``Nun... vielleicht wurde er ja niedergeschlagen und entführt.''\\
Jozah konnte dem Gespräch nur Bruchstückhaft folgen, versuchte sich aber zu erinnern, ob er 
angegriffen wurde. Das würde immerhin seine Kopfschmerzen erklären.\\
``Klar. Und dann hat jemand mindestens zwei Schnapsflaschen über ihn ausgekippt.''\\
Es folgte ein Moment der Stille und dann ein hastig gemurmeltes: ``Verzeiht, Herrin.''\\
\textit{Herrin...}\\
``Scheiße!'', keuchte Jozah und versuchte ruckartig aufzustehen. Sein Kopf knallte gegen den Tisch 
und er sank erneut zurück auf den Boden. Eine Hand packte ihn am Knöchel und zog ihn grob unter dem 
Tisch hervor. Jozah rollte sich auf den Rücken und blinzelte empor. Direkt in das von roten Locken 
umrahmte Gesicht seiner Gräfin.\\
Wieder Stille, in der Sarimé Sil'Vera ihn mit ausdrucksloser Miene musterte.\\
``Was soll ich nun dazu sagen'', sprach sie schließlich: ``Mein wichtigster General und Berater liegt 
betrunken in einer zwielichtigen Taverne. Meine Hoffnung, diesen Krieg zu überstehen, ohne das 
meine gesamte Grafschaft zu Grunde geht, stinkt nach Alkohol und... und... was ist das 
eigentlich?!''\\
``Urin'', warf Elor ein, der jetzt in Jozahs Blickfeld trat und ihn mit dem Stiefel gegen die Wade 
stieß.\\
``Urin?'', wiederholte Sarimé und ihre steinerne Mine zeigte einen kurzen Moment Ekel und 
Widerwillen, ehe sie sich wieder im Griff hatte. ``Urin!'', sagte sie fassungslos.\\
``Jaa... das war aber von Hunden'', fügte der Wirt hinzu: ``Also... da bin ich mir ziemlich
sicher.''\\
``Mein General wurde von Hunden angepinkelt?'', rief Sarimé: ``Mi'Kae, nun stehen Sie endlich 
auf!''\\
``Ähm'', brachte er mühsam hervor, setzte zu einem erneuten Versuch an und schaffte es immerhin in 
eine sitzende Position. Elor streckte ihm hilfsbereit seine Hand entgegen. Die Welt begann sich 
erneut zu drehen. Jozah fühlte sich wie in der Schwebe und konnte sich noch abwenden, um der Gräfin 
Sil'Vera zu ersparen, ihr direkt vor die Füße zu kotzen.\\
``Mi'Kae!'', donnerte ihre Stimme: ``Was in Osymas Namen ist in Euch gefahren?!''\\
Jozah, noch am würgen und husten, versuchte sich krampfhaft eine Ausrede einfallen zu lassen.\\
Er schwankte noch zwischen \textit{Mein Vater ist verstorben!} und \textit{Ich wurde unter Drogen 
gesetzt!}\\
``Herzschmerz'', erklärte die Tochter des Wirts: ``Hat ständig von einer Frau geredet. Und dann 
unseren König beleidigt. Sich sogar die Uniform zerrissen.''\\
Sarimé schüttelte den Kopf. ``Das reicht. Soldat, danke für die Information. Ich denke, es wäre für 
uns alle von Vorteil, wenn niemand General Mi'Kaes Zeitvertreib mitbekommt. Sobald er nicht mehr 
dabei ist sich zu übergeben, schafft ihn zurück zum Anwesen. Und steckt ihn in eine Badewanne.''\\
``Sein Pferd ist übrigens weg'', warf der Wirt etwas nervös ein: ``Er hat's einfach draußen stehen 
lassen. Wir wussten ja nicht... ich meine... das Tier war doch hoffentlich nicht in Eurem Besitz, 
Herrin?''\\
Sarimé lachte kurz auf. ``Nein. War es nicht.''\\
Gefolgt von zwei Wachen ihrer Garde und der jungen Frau, die sie zur Zeit immer begleitete, verließ 
die Gräfin Merandilas das heruntergekommene Gebäude. Jozah funkelte Elor böse an. ``Warum verdammt 
noch mal!''\\
``Ich fand's amüsant'', erklärte sein Freund und zuckte mit den Schultern. ``Und ich hatte gehofft, 
dass ihre Anwesenheit reichen würde um dich zu Verstand zu bringen. Ich war doch gestern Abend 
schon da und du hast mir nicht zugehört.''\\
Jozah biss die Zähne fest zusammen und versuchte den Geschmack nach Magensäure zu verdrängen. Er 
konnte sich überhaupt nicht daran erinnern, dass Elor überhaupt hier gewesen war.\\
``Was genau hat er denn gegen unseren geliebten König gesagt?'', fragte Elor grinsend den Wirt.\\

Jozah musste bei seinem Sturz hart auf das rechte Knie gefallen sein, denn es schmerzte, sobald er 
das Bein beugte. Daher kamen sie nur langsam voran. Elor schlenderte mit desinteressierten Blick 
neben seinen humpelnden General her, bis Jozah abrupt stehen blieb. Mitten in dem Gewühl einer 
Kreuzung. Menschen drängten aneinander vorbei, ohne sich eines Blickes zu würdigen. Oder aber sie 
riefen sich Beleidigungen und Flüche entgegen. Einzelne Fuhrwerke, ab und an ein Reiter teilte den 
Fluss aus Leibern.\\
Elor bemerkte erst nach drei weiteren Schritten, dass sein Freund stehen geblieben war und wandte 
sich ihm ungeduldig zu. ``Was ist los?''\\
``Warum bist du Soldat geworden?'', fragte Jozah leise, ohne den Blick von der Straße zu wenden.\\
Seine Brust hob sich, während er tief einatmete. ``Weil ich ein Sklave war und man mich nicht 
gefragt hat?'', entgegnete er, als wäre das offensichtlich: ``Oder weil ich nichts anderes kann, 
als kämpfen? Koch wäre ja was, aber da habe ich kein Händchen für.''\\
``Ich mein's ernst'', brummte Jozah und starrte auf einen Schuhabdruck im Staub. ``Warum hast du 
dich für diesen Weg entschieden? Du hättest damals gehen können. Du warst frei.''\\
``Deinetwegen.''\\
Jozah blickte ihn überrascht an.\\
``Mein Vater sagte: es gibt Männer, die führen und Männer, die folgen.''\\
Elor legte ihm eine Hand auf die Schulter. ``Soldaten machen alles, was ihnen befohlen wird. Sie 
töten. Sie sterben. Sie vergessen Moral und Ideale, wenn ihr Herr das verlangt. Sie sind wie Hunde. 
Sie gehorchen.''\\
``In der saleicanischen Kultur wird der Dienst für Osyma ruhmvoller empfunden'', murmelte Jozah.\\
``Ich weiß. Ihr lügt euch selbst an. Ich war ein Sklave und dann war ich Soldat. Es läuft auf das 
Selbe hinaus.''\\
``Wieso bist du nicht gegangen?'', wiederholte Jozah verständnislos. Er hatte nicht gedacht, dass 
sein Freund seinen Dienst so empfand.\\
Elor zuckte mit den Schultern. ``Es ist leichter zu folgen. Und ich folge gerne einem Mann, den ich 
vertraue. Ein Wort von dir genügt.''\\
``Warum?''\\
``Du bist ein guter Mann. Ich habe mich damals entschlossen dir zu folgen. Nicht dem König. Nicht 
den Priestern. Nicht Osyma.''\\
``Das ist ein Schwur und Verrat'', zischte Jozah.\\
``Nur in deinem Land'', entgegnete Elor. Er wollte noch etwas hinzufügen, runzelte dann jedoch die 
Stirn und deutete hinter seinen General. ``Ist das nicht dein Pferd?''\\
Jozah konnte seinen Augen nicht trauen. Ruckartig riss er den Arm empor. ``Heh! Stehn geblieben!''\\
So schnell er konnte, humpelte er zu der Seitengasse, drängte sich grob an den Leuten vorbei und 
rief wiederholt: ``Sofort stehen bleiben!''\\
Elor, der ihn mühelos einholte, zeigte eine grüblerische Miene und blieb an Jozahs Seite, auch wenn 
seine Chancen deutlich besser standen, den Reiter in der engen Gasse einzuholen. Schließlich hielt 
er Jozah an der Schulter fest. ``Warte. Das ist doch ein Trick. Er würde abhauen, wenn er nicht 
erwischt werden will und gehört hat er uns bestimmt.''\\
Widerwillig stellte er die schmerzhafte Verfolgung ein und musste seinem Freund recht geben. Das 
entging dem Reiter auch nicht. Er warf einen Blick über die Schulter und rief: ``Auf der Straße zu 
reden ist etwas unvorsichtig, General Mi'Kae. Besonders bei Ihrer auffälligen Erscheinung.''\\
Elor zog seine Klinge und lief auf Jozahs Nicken hin einen Schritt vor ihm die Gasse entlang. Ohne 
Waffe und angeschlagen wäre er nicht gerade in einer guten Kampfposition.\\
Der Reiter deutete die Gasse entlang. ``Es ist nicht mehr weit.''\\ 
Dann trieb er den Schimmel wieder an und bog um die nächste Ecke.\\


``Tee?'', fragte der Mann ihm gegenüber sympathisch lächelnd.\\
Jozah musterte ihn grimmig, nachdem er der einladende Geste gefolgt und sich an den Tisch gesetzt 
hatte. Der Mann war älter als er, wirkte gepflegt und trug die Kleidung eines einfachen 
Handwerkers. Er kam Jozah bekannt vor, aber so sehr er auch grübelte, es fiel ihm nicht ein. Der 
Reiter war eine völlig gegensätzliche Erscheinung. Groß, muskulös, das Gesicht eines Kampfhundes. 
Er stand in der Nähe der Tür und starrte vor sich hin.\\
``Himmelkraut'', schlussfolgerte Elor nachdem er sich schnüffelnd eine der Becher unter die Nase 
gehalten hatte: ``Oder?''\\
``Soll bei Kopfschmerzen helfen, habe ich gehört'', erklärte der Mann und schenkte lächelnd einen 
dritten Becher für sich selbst ein. Er warf einen fragenden Blick auf seinen Kampfhund, zuckte dann 
bedauernd mit der Schulter und stellte den Krug zurück. ``Schade.''\\
``Wer bist du und was willst du?'', fragte Jozah und ignorierte den dampfenden Tee vor sich.\\
``Ich will mich nur unterhalten'', erklärte er: ``Aber Ihr haben natürlich recht, werter Herr 
General. Es wäre unhöflich, wenn ich mich nicht vorstellen würde. Mein Name ist Arham und ich bin 
Holzfäller. Ich hatte die Ehre, Eure Truppe hier in Na'Rash willkommen zu heißen und zum Anwesen der Gräfin zu führen.''\\
Jozah sah den Mann ungläubig an. \textit{Der Kerl, der so viel geredet hat...}\\
Bis auf die Art seiner Kleidung passte nichts zu seiner Behauptung ein Holzfäller zu sein. Weder seine Wortwahl, noch der Tee.\\
``Und was soll ich mit einem Holzfäller besprechen? Und ist Tee nicht das Getränkt kasirischer 
Adelsdamen?''\\
Arhams Haltung blieb ihm offen und zugewandt, während er breit lächelnd erklärte: ``Genau genommen 
bevorzugten schon unsere merandilischen Ahnen Tee. Das ging bei der Eroberung etwas verloren... das 
lag vielleicht daran, dass die saleicanischen Eroberer Tee mit dem Norden und somit Kasir in 
Verbindung brachten, weshalb es als Zeichen der Sympathie mit dem Feind galt. Wie schade, nicht? 
Ein billiges, wohlschmeckendes und gesundes Getränk. Aber ich hörte, dass Gräfin Sieva regelmäßig 
von ihren kasirischen Verwandten Teeimporte erhielt und auch selbst nach Kräutern zum trocknen 
suchte.''\\
Jozah bemühte sich nicht, seine Ungeduld zu verbergen. Finster sah er drein, versuchte den 
pochenden Schmerz seines Kopfes zu ignorieren und wartete auf brauchbare Informationen. Er hatte 
keine Lust den Mann zu drängen. Stattdessen wartete er darauf, dass seine Geduldsgrenze erreicht 
war und er einfach aufstehen und aus dem Raum humpeln würde. Der General musste sich jedoch 
eingestehen, dass die Atmosphäre seltsam war. Dieser Mann strahlte eine Gelassenheit und Harmonie 
aus, die nicht zu dem zwielichtigem Raum passte. Oder dem Schläger in der Ecke.\\
``Wer bist du?'', fragte Jozah knurrend.\\
Arham antwortete erst nach einem weiteren Schluck. ``Ich? Niemand. Nur ein Zuhörer. Ein Tröster. 
Ein Prediger.''\\
``Ein Priester'', schlussfolgerte Elor interessiert.\\
Lächelnd schüttelte er den Kopf. ``Es ist mehr eine Freizeitbeschäftigung. Die Helle hat keine Priester nötig. Niemand, der ihr Wort 
verkündet, denn sie spricht selbst. Niemand, der ihre Taten beschreibt, denn die Helle zeigt sie 
uns allen. Niemand braucht an die Helle zu glauben, denn sie ist unter uns.''\\
Jozah atmete tief ein und hielt einen Moment die Luft an. Er hatte Priester noch nie leiden können. 
Aber dieser Mann war ein ganz anderer Fall als jeder Priester, dem er bisher begegnet war. Sie alle 
waren arrogant und fast schon hysterisch gewesen, wenn sie vom Allmächtigen sprachen. Dieser... 
Holzfäller sprach in einem plaudernden Tonfall, als würde er vom Wetter der letzten Tage 
berichten.\\
``Und? Was hat diese Helle so getan und gesagt in letzter Zeit?''\\
``Ich sagte doch, ich brauche es nicht zu erzählen. Macht selbst die Augen auf und Ihr werdet sie 
hören und sehen.''\\
``Vielleicht bin ich ja blind?''\\
Arham lachte. ``Oh, Ihr achtet nur auf das Falsche, Herr General. Ihre Stimme ist leise und 
besonnen, nicht wie die des Feuergottes. Sie ist im Flüstern in den Gassen.''\\
``Und? Was wird da geflüstert?'', fragte Jozah und war kurz davor aufzustehen.\\
Zu seiner Überraschung antwortete Elor. ``Sie wollen eine Königin.''\\
Jozahs Miene verfinsterte sich. ``Ilia wird die Krone stehen. Die Menschen werden schon sehen, was 
sie davon haben!''\\
Arham griff über den Tisch und tätschelte kurz Jozahs Faust. ``Oh, es geht nicht um eine ferne 
Adelige in Brom-Dallar. Es geht um das Versprechen, dass die letzte Königin Meradilas gab. Die 
Helle lebt. Merandila lebt. Und das merandilische Volk wird leben. Als ein Land unter einer 
Königin. Und diese Königin ist gekommen.''\\
Jozah biss sich auf die Zunge und sah ihn ungläubig an. ``Sie ist ein Kind.''\\
``Sie ist nicht allein. Ganz Merandila wird hinter ihr stehen.''\\
``Du sprichst für jeden Mann, jede Frau und jedes Kind in der gesamten Grafschaft?''\\
Arham überlegte kurz. ``Ich spreche für den Glauben, für die Tradition und die Hoffnung dieses 
Landes.''\\
``Wir gehen'', entschied Jozah, blieb aber ratlos sitzen. Er wusste immer noch nicht, was dieser 
Mann eigentlich von ihm wollte.\\
Der Prediger erriet seinen Gedanken und erklärte: ``Ich will nur Euer Wort, Herr General. Euren 
Schwur. Ich hörte, Ihr seit etwas... frustriert über Saleica. Jetzt ist die Gelegenheit, sich eine 
Herrin zu wählen. Eine neue Krone, unter der Ihr dienen wollt. Eine neue Zukunft.''\\ Arham erhob 
sich, rückte seinen Stuhl zurück an den Tisch und fügte im hinausgehen hinzu: ``Ihr wärt nicht der 
erste General, der vor Sarimé Sil'Vera auf die Knie fällt.''\\