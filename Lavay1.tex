\chapter{Kasira}

Imur kauerte an der Wand und horchte auf die knirschenden Schritte der Wache auf dem Kiesweg. 
Lautlos zählte er bis zehn. Ein dumpfes Geräusch erklang, gefolgt von einem leisen Ächzen und die 
Schritte verklangen. Imur erhob sich in einer fließenden Bewegung und bog um die Ecke. Haska 
sah flüchtig zu ihm auf und murrte: ``Der Kerl ist verdammt schwer. Hilf mir!''\\
Er bückte sich, packte den Mann an den Waden und zerrte ihn mit Haskas Hilfe ein gutes Stück in 
die Gartenanlage des Gebäudes hinein. Ein außgesprochen hübscher und großflächiger Garten, wenn man 
bedachte, dass er sich mitte in einer kasirischen Großstadt befand. Die Hälfte der Einwohner lebten 
in erbärmlichen Verhältnisses, in den Gassen und verschimmelten Häusern. Ganze Großfamilien in 
einzelnen Zimmern und trotzdem konnten sie die Miete kaum bezahlen. \textit{Ob er von seinen 
Fenstern aus unser Viertel sehen kann?}, überlegte Imur flüchtig. Ein keuchendes Geräusch weckte 
jedoch wieder seine Aufmerksamkeit. ``Er lebt noch.``\\
Seine Hand verschwand unter seinem Mantel und kam mit einer blanken Klinge wieder hervor.\\
``Warte! Der Blutgeruch wird die Wachhunde aufschrecken! Wir lassen ihn einfach liegen, sind eh 
schon außerhalb des Zeitplans!''\\
\textit{Du und deine Zeitpläne}, dachte Imur spöttisch, sprach es jedoch nicht aus. Meistens hatte 
Haska ja auch recht behalten, einen Mord erledigte man nicht einfach so spontan. Tage hatten sie 
sich auf den heutigen Abend vorbereitet. Stunden stand er auf der Straße vor dem Haus und hatte 
hinauf zu den Glasscheiben gestarrt. Haska hatte sogar die Baupläne aus dem Archiv geklaut und 
ordentlich Bestechungsgeld vorgeschossen. Auch er selbst hatte seine letzten Münzen investiert. 
Erspartes, welches nicht existierte. Es war, als habe er seiner Schwester und seiner Mutter das 
Brot vom Teller geklaut. Egal. Morgen schon wird er ihnen einen saftigen Braten auf den Tisch 
stellen können!\\
Hintereinander schlichen sich zurück zu dem mehrstöckigen Gebäude. Haska ging voraus, wie stets. Er 
war schon länger in diesem Geschäft, hatte schon mehr Aufträge erfüllt und nur durch ihn hatte sich 
diese Möglichkeit für Imur überhaupt ergeben. Aber heute, hatte Imur sich vorgenommen, würde es 
anders enden als sonst. Heute würde er seinem Kameraden beweisen, dass sich dessen Mühen ihn zu 
lehren gelohnt hatten. \\ 
Ihnen würden nur wenige Minuten bleiben, bis die zweite Wache bemerken würde, dass etwas nicht 
stimmte. Die Route des Bewusstlosen ging am Wohngebäude entlang und ein gutes Stück in die 
Gartenanlage hinein, anschließend um das Gebäude herum und traf dann am Tor seinen Kollegen. Ein 
Rundgang dauerte exakt elf Minuten. Imur hatte ihn genug Nächte dabei beobachtet. \\
Haska deutete nach oben und sah ihn abwartend an. Schnell wurde er ungeduldig, als Imur nicht 
wie abgesprochen in die Knie ging und die Hände ineinander verschränkte, um ihm als Leiter zu 
dienen. Stattdessen schüttelte er den Kopf und deutete auf sich. Haska zögerte, warf einen 
schnellen Blick über die Schulter und schwieg verbissen. Einen langen Moment blickten die 
beiden jungen Männer sich an. Eine stumme Diskussion, die Haska schließlich dadurch beendete, 
dass er sich in Imurs Position stellte und hecktisch andeutete, er solle hochklettern. Mit 
einem siegreichen Grinsen setzte Imur seinen Fuß auf Haskas Hände und stieß sich federnd vom 
Boden ab. \\
``Hmpf!'', entwich es Haska und schnaufte vor Anstrengung. ``Du warst auch mal leichter!''\\
``Ach, halt doch dein Maul!'', presste Imur hervor und klammerte sich an den schmalen Fenstersims. 
Sie hatten ein Dienstmädchen bestochen, ein Fenster im Erdgeschoss nur anzulehnen. Da das Gelände 
hinter dem Haus stark abfiel, war der Garten im Vergleich zur Straße an der Vorderseite des 
Gebäudes tiefer gelegt und das Fenster befand sich in einer problematischen Höhe. Imur zog sich 
unter Mühen hoch und stieß mit den Ellbogen das Fenster auf. Er biss sich vor Anstrengung auf die 
Zunge, als er sich hochhievte und schließlich Halt fand. Bedacht möglichst lautlos zu sein, 
schlüpfte er durch das Fenster und fand sich in einem dunklen Zimmer wieder. Imur beugte sich aus 
dem Fenster und streckte eine Hand aus. Wenn Haska sich streckte, könnte er sie gerade erreichen. 
Doch dazu kam es gar nicht.
``Trödel nicht so!'', blaffte eine barsche Stimme.\\
Haska rannte in den Garten und tauchte hinter dichtem Buschwerk unter. Imur zog sich ins Zimmer 
zurück und lehnte sich schwer atmend gegen die Wand. \textit{Verdammt, wir waren zu langsam!}\\
``Hey! Wo treibst du dich herum?'', brüllte die zweite Wache stocksauer: ``Faules Pack!''\\
Imur unterdrückte einen Fluch und beschloss, alleine weiter zu machen. Es kostete ihm Überwindung, 
auch wenn er vor wenigne Augenblicken noch überzeugt war, bereit genug dafür zu sein. Aber da 
dachte er auch noch, Haska würde ihm Rückendeckung geben können. Er hatte zu viel Arbeit in 
die Sache gesteckt, um jetzt noch umzukehren. Schlimmer noch, wenn er jetzt aufgab, wäre seine 
gerade erst begonnene Kariere erledigt, Haska würde ihn nie wieder eines Blickes würdigen und seine 
Schwester vermutlich nie wieder ein Wort mit ihm wechseln, weil er jetzt nicht einmal mehr für die 
Miete der Kammer aufkommen können würde. Sie würden wieder auf der Straße landen...\\
Er trat hinaus auf den Flur und schloss leise die Türe hinter sich. Zögernd blickte er sich um und 
rief sich den Grundriss des Gebäudes vor Augen. Langsam konnte Imur seine Nervosität nicht mehr 
verdrängen und zwang sich, ruhig zu atmen. Er tastete sich den dunklen Flur entlang, bis er auf 
eine steinerne Treppe stieß. Vorsichtig setzte er einen Fuß vor den anderen und stieg die Stufen 
hinauf. Er lauschte auf verräterische Geräusche vom Dienstpersonal oder möglichen Wachen. Der 
Hausherr schlief ganz oben, im dritten Stock, in vermeintlicher Sicherheit.\\
Das dritte Stockwerk unterschied sich im Aufbau nicht vom vorherigen. Links der Treppe führte 
der Weg weiter, dann bog er noch einmal ab und endete schließlich, wieder auf der Höhe, auf der die 
Stufen begannen. Ein schmiedeeisernes Geländer, welches Imur bis zum Bauchnabel reichte, 
verhinderte, dass man das Treppenhaus hinab stürzte.\\
Doch er wusste es natürlich besser. Das Stockwerk unterschied sich schon alleine dadurch, dass es 
komplett von einem Mann bewohnt wurde. Was unten weitere Schlafzimmer waren, hatte man hier zu 
einem Büro und einer privaten Vorratskammer umgebaut. Den Abort musste er sich mit niemandem teilen 
und selbst ein eigenes Bad hatte er sich gegönnt, wofür das warme Wasser in den dritten Stock 
befördert werden musste.\\
Imur schlich an noch ein paar Türen entlang, doch was deren genaue Bedeutung anging, war er sich 
auch nach intensivem Ausspionieren noch nicht sicher. Er hatte in Erfahrung gebracht, dass sein 
Opfer diverse Kunstschätze sammelte, was auch immer das bedeutete, und das musste wohl auch 
irgendwo zu finden sein. \\
Unbeirrt tappte er zu der einen Tür, auf die es ankam: Das Schlafzimmer des Hausherren.\\
Entschlossen, aber vorsichtig, drückte er die Klinke herab. Sie machte keinen Laut, bis auf das 
Klicken, als das Schloss den Weg frei gab. Innerlich atmete er auf. Er hätte keine Angst davor 
gehabt, ein Schloss knacken zu müssen, doch es kostete Zeit, besonders, wenn man es leise und im 
dunkeln tun musste.\\
Langsam und gleichmäßig schob er die schwere Tür auf, doch sie knarzte nicht einmal. Geradezu 
einladend schwang sie auf. Das Licht des Mondes drang an den schweren Vorhängen vorbei und 
beleuchtete die Mitte des Raumes. Ein großer Kleiderschrank nahm eine komplette Wand in Beschlag, 
an der anderen Stand das Bett. Imur atmete ruhig und leise, während er sich dem Federbett näherte. 
Sein Opfer atmete überhaupt nicht. Böses ahnend streckte er die linke Hand aus und tastete nach 
ihm, mit der Rechten ließ er sein Messer aus der Scheide gleiten.
Doch er griff nach nichts, als einer weichen Decke, die unter seinen Fingern zusammen fiel.
Hier schlief niemand. Sein Herz begann zu rasen.\\
``Was? Nur einer?'', rief eine Stimme: ``Hatte ich nicht genug Lohn in Aussicht gestellt um eine 
ganze Gruppe Halunken zu bezahlen? Oder wolltest du nicht teilen, hm?''\\
Imur starrte auf die Tür. Der Gang hinter dem Mann wurde von Fackeln erhellt. Sprachlos stand er 
dem Hausherrn gegenüber und konnte nur zusehen, wie seine Wachen sich ihm mit gezückten Waffen 
näherten. In seinem Kopf tobte ein Sturm aus Gedanken. Wovon sprach der Kerl? Wie konnte das 
passieren? Was hatte ihn verraten? Wo war Haska?\\

Es war bereits Nachmittag und Lavay saß lustlos auf dem Hocker am Tisch, den Kopf mit den Händen 
gestützt und beobachtete den vergeblichen Versuch ihrer Mutter, die Kammer halbwegs wohnlich zu 
gestalten. Ihr Magen schmerze vor Hunger, aber sie versuchte es wie so oft zu ignorieren. 
Irgendwann musste Imur ja wieder kommen... sie wartete schon seit einer Woche auf ihren Bruder 
und hatte keine Ahnung, wohin er plötzlich verschwunden sein könnte. Sie wusste nur von seiner 
Bekanntschaft mit diesem zwielichten Kerl namens Haska. Aber trotzdem war er nie länger als ein 
paar Tage weggeblieben ohne ein Lebenszeichen von sich zu geben. Meistens begleitet von etwas 
Geld oder Nahrung. Nach den ersten beiden Tagen hatte sie noch selbst versucht etwas Geld 
aufzutreiben. Aber es war nie leicht gewesen hier in diesen Vierteln der Stadt. Dazu kam, dass 
es ihrer Mutter immer schlechter ging, Lavay traute sich kaum sie aus den Augen zu lassen.\\
Also saß sie nun hier auf einen der beiden Hocker und beobachtete ihre Mutter, wie sie die 
Lumpen ihrer Betten aufschüttelte und durch den Staub einen Hustenanfall bekam. Noch ehe er sich 
gelegt hatte, rückte sie den zweiten Hocker auf die andere Seite des Zimmers und arrangierte 
konzentriert den vertrockneten Blumenstraß auf dem Tisch. Als sie schließlich zu dem alten Besen 
griff und zaghaft zu kehren begann, fragte Lavay skeptisch: ``Was hast du vor?''\\
Ihre Mutter strich sich eine Haarsträhne aus dem Gesicht und stützte sich schwer atmend auf den 
Besen. Schweißtropfen sammelten sich auf ihrer Stirn. Die Temperatur war in den vergangenen Tagen 
deutlich gestiegen und in dieser steinernen Kammer mit dem winzigen Fenster, kaum groß genug um den 
Kopf hinaus zu strecken, fühlte man sich wie in einem Backofen. Es war grässlich. Aber nicht ganz 
so grässlich wie den fast schon beißenden, grellen Strahlen der Mittagssonne auf dem Marktplatz 
ausgesetzt zu sein. \\
``Dein Vater arbeitet hart und wenn er nach Hause kommt…''\\
Lavays Faust schlug schmerzhaft auf dem Holztisch ein. Der alte Tonbecher fiel um und die Blumen 
verteilten sich auf der zerkratzten Oberfläche. Die ganze Woche ertrug Lavay nun schon das Gerede 
ihrer Mutter mit stillschweigen. Aber noch ein weiteres Wort über die baldige Rückkehr ihres Vaters 
konnte und wollte sie nicht mehr hören. Sie stand auf und trat auf ihre Mutter zu. Lavay legte ihr 
die Hände auf die Schultern und fixierte sie streng. Der erschrockene und wirre Blick der älteren 
Frau tat ihr fast schon weh. Trotzdem sprach sie leise und eindringlich: ``Er ist tot, Mutter. Er 
kommt nicht mehr zu uns zurück. Wach endlich auf und tu etwas! Irgendetwas Sinnvolles!''\\
Der Mund ihrer Mutter öffnete und schloss sich, ohne dass ein Laut über ihre Lippen kam. Sie ließ 
den Besen fallen und griff sich in einer verzweifelten, hilflosen Geste an den Kopf. Tränen stiegen 
ihr in die Augen. \\
``Wovon sprichst du?'', stammelte sie und wich von ihrer Tochter zurück: ``Dein Vater…''\\
Lavay schloss die Augen und unterdrückte einen Wutschrei. ``Das ist so lächerlich!'', fluchte sie: 
``Du bist lächerlich! Ich sollte einfach…''\\
\textit{Scheiße Imur, komm endlich wieder!}\\
Es klopfte an der Kammertür. Lavay und ihre Mutter blickten gleichzeitig auf. Niemand klopfte je an 
diese Tür. Es kam kein Besuch. Nur der Vermieter, aber der klopfte nicht, der stieß sie auf und 
erwartete sofort sein Geld. Ihre Mutter reagierte wieder nicht, also trat die siebzehnjährige vor 
und drückte die Klinke hinab. \\
``Was tust du hier?'', fragte sie leise und spähte an Haska vorbei, ob Imur wohl hinter ihm war. 
Aber sie sah nur ein paar der Gestalten, die nicht mal eine Kammer mieten konnten und mit vielen 
anderen auf dem schmalen Gang hausten. Und selbst dafür Unsummen zahlen mussten. \\
Lavay wusste was Haska für ein Mensch war, womit er sein Geld verdiente. Und trotzdem kam sie nicht 
umhinn sich selbst einzugestehen, dass er viel zu unschuldig aussah. Braune Locken, die ihm in die 
Stirn vielen, Augen in der Farbe des Honigs. Vielleicht war er deshalb in diesem Geschäft so gut. 
Die Leute erwarteten einfach nicht, dass er zuschlagen konnte, bis sie mit einer gebrochenen Nase 
am Boden lagen. \\
``Wo ist Imur?'' Sie versuchte streng zu klingen, aber die Sorge überwog. Lavay hatte diesen Mann 
vor ihr noch nie so zögernd erlebt. Sonst war er immer das arrogante Arschloch mit dem hübschen 
Grübchen, welches sein freches Grinsen begleitete. Dagegen wirkte Imur mit seinen breiten 
Schultern, den unrasiertem Gesicht und großen Händen wie der fieße Schlägertyp von nebenan. \\
Haska schien nach Worten zu suchen und wirkte immer verzweifelter. ``Es tut mir leid'', stammelte 
er schließlich: ``Irgendwas ging schief... ich weiß nicht was... ich...''\\
``Wieso stehst du hier und er nicht?'', fragte Lavay eisig.\\
``Ich hab's versucht. Ehrlich! Ich hab mich überall umgehört, aber keiner wusste was... bisher. Ich 
schwöre dir, ich habs versucht... ich bin sogar in's Gefängnis eingestiegen, da war er nicht...''\\
Seine Worte entgingen auch nicht den unfreiwilligen Zuhörern auf dem Gang. Einige beugten sich 
interessiert näher vor um Einzelheiten mitzubekommen. Es ging ihnen nicht darum, dass sie Haska 
verpfeiffen würden. Es ging lediglich um die Geschichte, eine Abwechslung zu ihrem öden Leben.\\
``Bisher. Und jetzt?'', flüsterte Lavay.\\
Haska schluckte und wich ihrem Blick aus. ``Ich habe vorhin mitbekommen, wie ein Ratsmitglied des 
Königs verkündete, dass jemand ihn habe ermorden wollen... und er den Attentäter festgenommen 
hat....''\\
``Dann weiß du jetzt, wo Imur ist? Dann hol ihn raus!''\\
``Und ihn heute noch hinrichten lassen will.''\\
Lavay erblasste. Ihre Finger krallten sich in das weiche Holz des Türrahmens. ``Heute.''\\
``Gleich.''\\
``Warum kommst du erst jetzt?''\\
Er hob in einer hilflosen Geste die Schultern. ``Ich hab's versucht... ich dachte, ich würde alles 
wieder hinbekommen.''\\
Sie sah in die Kammer, in der ihre Mutter gerade die Tonscherben aufsammelte und schloss mit einem 
Ruck dir Tür hinter sich. ``Bring mich hin. Sofort!''\\
