\chapter{Maras Gnade}

Imur kauerte an der Wand und horchte auf die knirschenden Schritte der Wache auf dem Kiesweg. 
Lautlos zählte er bis zehn. Ein dumpfes Geräusch erklang, gefolgt von einem leisen Ächzen und die 
Schritte verklangen. Imur erhob sich in einer fließenden Bewegung und bog um die Ecke. Haska 
sah flüchtig zu ihm auf und murrte: ``Der Kerl ist verdammt schwer. Hilf mir!''\\
Er bückte sich, packte den Mann an den Waden und zerrte ihn mit Haskas Hilfe ein gutes Stück in 
die Gartenanlage des Gebäudes hinein. Ein ausgesprochen hübscher und großflächiger Garten, wenn man 
bedachte, dass er sich mitten in einer kasirischen Großstadt befand. Die Hälfte der Einwohner 
lebten in erbärmlichen Verhältnisses, in den Gassen und verschimmelten Häusern. Ganze Großfamilien 
in einzelnen Zimmern und trotzdem konnten sie die Miete kaum bezahlen. \textit{Ob er von seinen 
Fenstern aus unser Viertel sehen kann?}, überlegte Imur flüchtig. Ein keuchendes Geräusch weckte 
jedoch wieder seine Aufmerksamkeit. ``Er lebt noch.``\\
Seine Hand verschwand unter seinem Mantel und kam mit einer blanken Klinge wieder hervor.\\
``Warte! Der Blutgeruch wird die Wachhunde aufschrecken! Wir lassen ihn einfach liegen, sind eh 
schon außerhalb des Zeitplans!''\\
\textit{Du und deine Zeitpläne}, dachte Imur spöttisch, sprach es jedoch nicht aus. Meistens hatte 
Haska ja auch recht behalten, einen Mord erledigte man nicht einfach so spontan. Tage hatten sie 
sich auf den heutigen Abend vorbereitet. Stunden stand er auf der Straße vor dem Haus und hatte 
hinauf zu den Glasscheiben gestarrt. Haska hatte sogar die Baupläne aus dem Archiv geklaut und 
ordentlich Bestechungsgeld vorgeschossen. Auch er selbst hatte seine letzten Münzen investiert. 
Erspartes, welches nicht existierte. Es war, als habe er seiner Schwester und seiner Mutter das 
Brot vom Teller geklaut. Egal. Morgen schon wird er ihnen einen saftigen Braten auf den Tisch 
stellen können!\\
Hintereinander schlichen sich zurück zu dem mehrstöckigen Gebäude. Haska ging voraus, wie stets. Er 
war schon länger in diesem Geschäft, hatte schon mehr Aufträge erfüllt und nur durch ihn hatte sich 
diese Möglichkeit für Imur überhaupt ergeben. Aber heute, hatte Imur sich vorgenommen, würde es 
anders enden als sonst. Heute würde er seinem Kameraden beweisen, dass sich dessen Mühen ihn zu 
lehren gelohnt hatten. \\ 
Ihnen würden nur wenige Minuten bleiben, bis die zweite Wache bemerken würde, dass etwas nicht 
stimmte. Die Route des Bewusstlosen ging am Wohngebäude entlang und ein gutes Stück in die 
Gartenanlage hinein, anschließend um das Gebäude herum und traf dann am Tor seinen Kollegen. Ein 
Rundgang dauerte exakt elf Minuten. Imur hatte ihn genug Nächte dabei beobachtet. \\
Haska deutete nach oben und sah ihn abwartend an. Schnell wurde er ungeduldig, als Imur nicht 
wie abgesprochen in die Knie ging und die Hände ineinander verschränkte, um ihm als Leiter zu 
dienen. Stattdessen schüttelte er den Kopf und deutete auf sich. Haska zögerte, warf einen 
schnellen Blick über die Schulter und schwieg verbissen. Einen langen Moment blickten die 
beiden jungen Männer sich an. Eine stumme Diskussion, die Haska schließlich dadurch beendete, 
dass er sich in Imurs Position stellte und hektisch andeutete, er solle hoch klettern. Mit 
einem siegreichen Grinsen setzte Imur seinen Fuß auf Haskas Hände und stieß sich federnd vom 
Boden ab. \\
``Hmpf!'', entwich es Haska und schnaufte vor Anstrengung. ``Du warst auch mal leichter!''\\
``Ach, halt doch dein Maul!'', presste Imur hervor und klammerte sich an den schmalen Fenstersims. 
Sie hatten ein Dienstmädchen bestochen, ein Fenster im Erdgeschoss nur anzulehnen. Da das Gelände 
hinter dem Haus stark abfiel, war der Garten im Vergleich zur Straße an der Vorderseite des 
Gebäudes tiefer gelegt und das Fenster befand sich in einer problematischen Höhe. Imur zog sich 
unter Mühen hoch und stieß mit den Ellbogen das Fenster auf. Er biss sich vor Anstrengung auf die 
Zunge, als er sich hoch hievte und schließlich Halt fand. Bedacht möglichst lautlos zu sein, 
schlüpfte er durch das Fenster und fand sich in einem dunklen Zimmer wieder. Imur beugte sich aus 
dem Fenster und streckte eine Hand aus. Wenn Haska sich streckte, könnte er sie gerade erreichen. 
Doch dazu kam es gar nicht.
``Trödle nicht so!'', blaffte eine barsche Stimme.\\
Haska rannte in den Garten und tauchte hinter dichtem Buschwerk unter. Imur zog sich ins Zimmer 
zurück und lehnte sich schwer atmend gegen die Wand. \textit{Verdammt, wir waren zu langsam!}\\
``Hey! Wo treibst du dich herum?'', brüllte die zweite Wache stocksauer: ``Faules Pack!''\\
Imur unterdrückte einen Fluch und beschloss, alleine weiter zu machen. Es kostete ihm Überwindung, 
auch wenn er vor wenigen Augenblicken noch überzeugt war, bereit genug dafür zu sein. Aber da 
dachte er auch noch, Haska würde ihm Rückendeckung geben können. Er hatte zu viel Arbeit in 
die Sache gesteckt, um jetzt noch umzukehren. Schlimmer noch, wenn er jetzt aufgab, wäre seine 
gerade erst begonnene Kariere erledigt, Haska würde ihn nie wieder eines Blickes würdigen und seine 
Schwester vermutlich nie wieder ein Wort mit ihm wechseln, weil er jetzt nicht einmal mehr für die 
Miete der Kammer aufkommen können würde. Sie würden wieder auf der Straße landen...\\
Er trat hinaus auf den Flur und schloss leise die Türe hinter sich. Zögernd blickte er sich um und 
rief sich den Grundriss des Gebäudes vor Augen. Langsam konnte Imur seine Nervosität nicht mehr 
verdrängen und zwang sich, ruhig zu atmen. Er tastete sich den dunklen Flur entlang, bis er auf 
eine steinerne Treppe stieß. Vorsichtig setzte er einen Fuß vor den anderen und stieg die Stufen 
hinauf. Er lauschte auf verräterische Geräusche vom Dienstpersonal oder möglichen Wachen. Der 
Hausherr schlief ganz oben, im dritten Stock, in vermeintlicher Sicherheit.\\
Das dritte Stockwerk unterschied sich im Aufbau nicht vom vorherigen. Links der Treppe führte 
der Weg weiter, dann bog er noch einmal ab und endete schließlich, wieder auf der Höhe, auf der die 
Stufen begannen. Ein schmiedeeisernes Geländer, welches Imur bis zum Bauchnabel reichte, 
verhinderte, dass man das Treppenhaus hinab stürzte.\\
Doch er wusste es natürlich besser. Das Stockwerk unterschied sich schon alleine dadurch, dass es 
komplett von einem Mann bewohnt wurde. Was unten weitere Schlafzimmer waren, hatte man hier zu 
einem Büro und einer privaten Vorratskammer umgebaut. Den Abort musste er sich mit niemandem teilen 
und selbst ein eigenes Bad hatte er sich gegönnt, wofür das warme Wasser in den dritten Stock 
befördert werden musste.\\
Imur schlich an noch ein paar Türen entlang, doch was deren genaue Bedeutung anging, war er sich 
auch nach intensivem Ausspionieren noch nicht sicher. Er hatte in Erfahrung gebracht, dass sein 
Opfer diverse Kunstschätze sammelte, was auch immer das bedeutete, und das musste wohl auch 
irgendwo zu finden sein. \\
Unbeirrt tappte er zu der einen Tür, auf die es ankam: Das Schlafzimmer des Hausherren.\\
Entschlossen, aber vorsichtig, drückte er die Klinke herab. Sie machte keinen Laut, bis auf das 
Klicken, als das Schloss den Weg frei gab. Innerlich atmete er auf. Er hätte keine Angst davor 
gehabt, ein Schloss knacken zu müssen, doch es kostete Zeit, besonders, wenn man es leise und im 
Dunkeln tun musste.\\
Langsam und gleichmäßig schob er die schwere Tür auf, doch sie knarrte nicht einmal. Geradezu 
einladend schwang sie auf. Das Licht des Mondes drang an den schweren Vorhängen vorbei und 
beleuchtete die Mitte des Raumes. Ein großer Kleiderschrank nahm eine komplette Wand in Beschlag, 
an der anderen Stand das Bett. Imur atmete ruhig und leise, während er sich dem Federbett näherte. 
Sein Opfer atmete überhaupt nicht. Böses ahnend streckte er die linke Hand aus und tastete nach 
ihm, mit der Rechten ließ er sein Messer aus der Scheide gleiten.
Doch er griff nach nichts, als einer weichen Decke, die unter seinen Fingern zusammen fiel.
Hier schlief niemand. Sein Herz begann zu rasen.\\
``Was? Nur einer?'', rief eine Stimme: ``Hatte ich nicht genug Lohn in Aussicht gestellt um eine 
ganze Gruppe Halunken zu bezahlen? Oder wolltest du nicht teilen, hm?''\\
Imur starrte auf die Tür. Der Gang hinter dem Mann wurde von Fackeln erhellt. Sprachlos stand er 
dem Hausherrn gegenüber und konnte nur zusehen, wie seine Wachen sich ihm mit gezückten Waffen 
näherten. In seinem Kopf tobte ein Sturm aus Gedanken. Wovon sprach der Kerl? Wie konnte das 
passieren? Was hatte ihn verraten? Wo war Haska?\\

Es war bereits Nachmittag und Lavay saß lustlos auf dem Hocker am Tisch, den Kopf mit den Händen 
gestützt und beobachtete den vergeblichen Versuch ihrer Mutter, die Kammer halbwegs wohnlich zu 
gestalten. Ihr Magen schmerze vor Hunger, aber sie versuchte es wie so oft zu ignorieren. 
Irgendwann musste Imur ja wieder kommen... sie wartete schon seit einer Woche auf ihren Bruder 
und hatte keine Ahnung, wohin er plötzlich verschwunden sein könnte. Sie wusste nur von seiner 
Bekanntschaft mit diesem zwielichtigen Kerl namens Haska. Aber trotzdem war er nie länger als ein 
paar Tage weggeblieben ohne ein Lebenszeichen von sich zu geben. Meistens begleitet von etwas 
Geld oder Nahrung. Nach den ersten beiden Tagen hatte sie noch selbst versucht etwas Geld 
aufzutreiben. Aber es war nie leicht gewesen hier in diesen Vierteln der Stadt. Dazu kam, dass 
es ihrer Mutter immer schlechter ging, Lavay traute sich kaum sie aus den Augen zu lassen.\\
Also saß sie nun hier auf einen der beiden Hocker und beobachtete ihre Mutter, wie sie die 
Lumpen ihrer Betten aufschüttelte und durch den Staub einen Hustenanfall bekam. Noch ehe er sich 
gelegt hatte, rückte sie den zweiten Hocker auf die andere Seite des Zimmers und arrangierte 
konzentriert den vertrockneten Blumenstrauß auf dem Tisch. Als sie schließlich zu dem alten Besen 
griff und zaghaft zu kehren begann, fragte Lavay skeptisch: ``Was hast du vor?''\\
Ihre Mutter strich sich eine Haarsträhne aus dem Gesicht und stützte sich schwer atmend auf den 
Besen. Die Temperatur war in den vergangenen Tagen deutlich gesunken. Der Sommer wich schnell und 
Lavay fürchtete bereits die eisige Kälte des Winters. Es war drei Jahre her, dass sie den letzten 
Winter auf der Straße verbringen mussten und sie hatte fast gedacht, dass es nicht mehr so weit 
kommen würde. Aber nun war Imur verschwunden...\\
``Dein Vater arbeitet hart und wenn er nach Hause kommt…''\\
Lavays Faust schlug schmerzhaft auf dem Holztisch ein. Der alte Tonbecher fiel um und die Blumen 
verteilten sich auf der zerkratzten Oberfläche. Die ganze Woche ertrug Lavay nun schon das Gerede 
ihrer Mutter mit Stillschweigen. Aber noch ein weiteres Wort über die baldige Rückkehr ihres Vaters 
konnte und wollte sie nicht mehr hören. Sie stand auf und trat auf ihre Mutter zu. Lavay legte ihr 
die Hände auf die Schultern und fixierte sie streng. Der erschrockene und wirre Blick der älteren 
Frau tat ihr fast schon weh. Trotzdem sprach sie leise und eindringlich: ``Er ist tot, Mutter. Er 
kommt nicht mehr zu uns zurück. Wach endlich auf und tu etwas! Irgendetwas Sinnvolles!''\\
Der Mund ihrer Mutter öffnete und schloss sich, ohne dass ein Laut über ihre Lippen kam. Sie ließ 
den Besen fallen und griff sich in einer verzweifelten, hilflosen Geste an den Kopf. Tränen stiegen 
ihr in die Augen. \\
``Wovon sprichst du?'', stammelte sie und wich von ihrer Tochter zurück: ``Dein Vater…''\\
Lavay schloss die Augen und unterdrückte einen Wutschrei. ``Das ist so lächerlich!'', fluchte sie: 
``Du bist lächerlich! Ich sollte einfach…''\\
\textit{Scheiße Imur, komm endlich wieder!}\\
Es klopfte an der Kammertür. Lavay und ihre Mutter blickten gleichzeitig auf. Niemand klopfte je an 
diese Tür. Es kam kein Besuch. Nur der Vermieter, aber der klopfte nicht, der stieß sie auf und 
erwartete sofort sein Geld. Ihre Mutter reagierte wieder nicht, also trat die siebzehnjährige vor 
und drückte die Klinke hinab. \\
``Was tust du hier?'', fragte sie leise und spähte an Haska vorbei, ob Imur wohl hinter ihm war. 
Aber sie sah nur ein paar der Gestalten, die nicht mal eine Kammer mieten konnten und mit vielen 
anderen auf dem schmalen Gang hausten. Und selbst dafür Unsummen zahlen mussten. \\
Lavay wusste was Haska für ein Mensch war, womit er sein Geld verdiente. Und trotzdem kam sie nicht 
umhin sich selbst einzugestehen, dass er viel zu unschuldig aussah. Braune Locken, die ihm in die 
Stirn vielen, Augen in der Farbe des Honigs. Vielleicht war er deshalb in diesem Geschäft so gut. 
Die Leute erwarteten einfach nicht, dass er zuschlagen konnte, bis sie mit einer gebrochenen Nase 
am Boden lagen. \\
``Wo ist Imur?'' Sie versuchte streng zu klingen, aber die Sorge überwog. Lavay hatte diesen Mann 
vor ihr noch nie so zögernd erlebt. Sonst war er immer das arrogante Arschloch mit dem hübschen 
Grübchen, welches sein freches Grinsen begleitete. Dagegen wirkte Imur mit seinen breiten 
Schultern, den unrasiertem Gesicht und großen Händen wie der fieße Schlägertyp von nebenan. \\
Haska schien nach Worten zu suchen und wirkte immer verzweifelter. ``Es tut mir leid'', stammelte 
er schließlich: ``Irgendwas ging schief... ich weiß nicht was... ich...''\\
``Wieso stehst du hier und er nicht?'', fragte Lavay eisig.\\
``Ich hab's versucht. Ehrlich! Ich hab mich überall umgehört, aber keiner wusste was... bisher. Ich 
schwöre dir, ich hab's versucht... ich bin sogar in's Gefängnis eingestiegen, da war er nicht...''\\
Seine Worte entgingen auch nicht den unfreiwilligen Zuhörern auf dem Gang. Einige beugten sich 
interessiert näher vor um Einzelheiten mitzubekommen. Es ging ihnen nicht darum, dass sie Haska 
verpfeifen würden. Es ging lediglich um die Geschichte, eine Abwechslung zu ihrem öden Leben.\\
``Bisher. Und jetzt?'', flüsterte Lavay.\\
Haska schluckte und wich ihrem Blick aus. ``Ich habe vorhin mitbekommen, wie ein Ratsmitglied des 
Königs verkündete, dass jemand ihn habe ermorden wollen... und er den Attentäter festgenommen 
hat....''\\
``Dann weiß du jetzt, wo Imur ist? Hol ihn raus!''\\
``Und ihn heute noch hinrichten lassen will.''\\
Lavay erblasste. Ihre Finger krallten sich in das weiche Holz des Türrahmens. ``Heute.''\\
``Gleich.''\\
``Warum kommst du erst jetzt?''\\
Er hob in einer hilflosen Geste die Schultern. ``Ich hab's versucht... ich dachte, ich würde alles 
wieder hinbekommen.''\\
Sie sah in die Kammer, in der ihre Mutter gerade die Tonscherben aufsammelte und schloss mit einem 
Ruck dir Tür hinter sich. ``Bring mich hin. Sofort!''\\

Nach den Lehren der Prophetin Mara war es ein Frevel, Verurteilte innerhalb der Stadt zu töten. Die 
Seelen brauchten die Natur um verblassen zu können. Bäume, Gräser, der Wind. Sie leiteten ihnen den 
Weg. Auch die Körper wurden der Natur übergeben. Meist begrub man sie, denn so heilig der Vorgang 
auch war, verlor der Verstorbene seine Ehre, wenn man den Körper bei seiner Verwesung beobachten 
konnte. Verurteilte verdienten keine Ehre. Es war ein Verbrechen, die Leichen zu begraben.\\
Lavay rannte an abgeernteten Feldern vorbei. An Kindern, die auf den Ästen der alten Bäume saßen 
und Äpfel sammelten. Als einziger Weg diente eine Spur aus blanker Erde. So viele Füße liefen 
täglich darüber, dass kein Gras mehr zu wachsen wagte. Der Henkersbaum war einer von vielen alten, 
krummen Bäumen auf dem Feld. Und an vielen baumelten die verwesenden, ehrlosen Überreste. Lavay 
hielt kurz inne und nahm diesen Anblick in sich auf. Sie war noch nie hier gewesen. Wozu auch? 
Der Herbst hatte hier im Norden längst seinen Anfang genommen und die Blätter der Bäume färbten 
sich in den Farben des Feuers. Äpfel lagen verstreut auf dem feuchten Boden und verströmten einen 
süßlichen, fauligen Geruch. Und inmitten dieser Obstwiese wankten die Toten mit ihrem ausgefranzten 
Halsschmuck. Die Kinder mit ihren Sammelkörben schienen sich nicht daran zu stören. Aufgrund des 
Anlasses dieses Ortes nahm keiner das Obst, der es nicht bitter nötig hatte. \\
``Dahinten'', murmelte Haska und deutete auf eine kleine Versammlung. \\
Er packte sie grob am Arm um zu verhindern, dass sie wieder losrannte. Nebeneinander näherten sie 
sich langsam der kleinen Menge. Klagende Angehörige standen neben fluchendem Pöbel. Hier starb 
jeder Verbrecher, dessen Titel oder Geldbeutel nichts anderes zuließen. Im Tode sind alle gleich - 
sprach einst die Prophetin Mara - und daher ist es bedeutungslos, weshalb Verurteilte dem Tod 
übergeben werden. Schenkte man ihrer Schuld Beachtung, schenkte man ihnen Ehre. Und das verdienten 
sie nicht. \\
Also starben hier kleine Diebe mit der selben Gleichgültigkeit der Leute wie Mörder oder Verräter. 
Ebenso schweigend verlief die Prozedur. Der Henker hievte einen Verurteilten auf sein Podest, legte 
ihm den Strick um und stieß ihn hinab. Tanzend zuckten die Füße auf der Suche nach Grund, den sie 
nie wieder erreichen würden. Lavays Augen glitten suchend über die kleine Ansammlung Verurteilter. 
Imur war nicht dabei. Also wanderte ihr Blick zu den Tanzenden. Drei hingen bereits. Einer kämpfte 
noch. Und Imur sah ins Nichts.\\
Lavay sah in das ausdruckslose Gesicht ihres Bruders und spürte den Herbstwind nicht mehr. Die 
Nässe in ihren löchrigen Schuhen, die Tränen auf ihren Wangen, all das verblasste. \\
``Wieso?'', hauchte sie, ohne den Blick abzuwenden: ``Wieso musste er sterben.''\\
Haska, der vor wenigen Minuten aufgelöst vor ihrer Tür stand und nur gestammelt hatte, war nun 
wieder so gelassen und ruhig wie sie ihn kennengelernt hatte. ``Wir sollten jemanden umbringen'', 
antwortete er sachlich.\\
``Wieso?''\\
``Weil es jemand wollte.''\\
``Nein! Wieso Imur?'', krächzte Lavay.\\
``Alles was er tat, tat er für dich'', sagte er: ``Er stieg für dich in dieses Haus. Er hob für 
dich sein Messer. Er sprach mich damals an... weil er nicht zusehen konnte, wie du verhungerst.''\\
Lavay wirbelte herum und schlug Haska kräftig gegen die Brust. Der junge Mann trat einen Schritt 
zurück um sein Gleichgewicht zu halten, betrachtete sie aber nur skeptisch. \\
``Und wieso hast du nicht nein gesagt?'', schrie sie ihn an: ``Du hattest keinen Grund ihn 
mitzunehmen! Er war damals miserabel.''\\
Lavay senkte betroffen ihre Stimme, als sie die Blicke der Menschen bemerkte. Sie sah ihn voller 
Hass an und in ihren Augen lagen die Worte, die sie nicht aussprechen konnte. Imur war damals 
anders gewesen. Ein guter Mensch. Er jammerte viel. Er konnte nicht einmal einen Apfel klauen, ohne 
dass er erwischt wurde. Sie war damals dabei gewesen, als ihr Bruder den drahtigen Jungen 
angesprochen hatte. Gebettelt hatte er. Und Haska hatte gelacht. Sie konnte immer noch den Spott 
hören, als er antwortete, dass das kleine Mädchen sich besser eigenen würde. Imur dachte vermutlich, 
dass Haska ihn einfach beleidigen wollte. Lavay nicht. Sie konnte ihm nur zustimmen. Aber ihr Bruder 
konnte sich nicht vorstellen, dass Frauen für solche Tätigkeiten geeignet waren. Erst recht nicht 
seine kleine Schwester. Und er hatte ja auch genug von Haska gelernt um sein Partner zu werden. Und 
nun hing er. \\
``Ich hatte Mitleid'', antwortete Haska leise: ``Er hatte nicht aufgehört zu betteln.''\\
``Willst du ihn heute Nacht runter holen und begraben?'', fragte sie flüsternd.\\
``Ich hätte mein Leben für seines riskiert'', antwortete er: ``Aber nicht für seine Leiche.''\\
Lavay nickte. Das war nur logisch. ``Lass uns gehen'', murmelte sie und wandte sich ab. Imur war 
tot. \\

``Danke, dass du es mir gesagt hast.''\\
``Ich ging davon aus, dass du weinen würdest'', bemerkte Haska und sah sie prüfend von der Seite 
an, während sie das Stadttor durchquerten.\\
``Ich werde nicht vor dir weinen'', entgegnete Lavay eisig. \\
``Er war auch mir ein Bruder, Lavay. Ich kannte ihn vermutlich besser als du.''\\
Lavay sagte darauf nichts, aber gab ihm recht. Imur war oft tagelang verschwunden, mit Haska. Zu 
ihr und ihrer Mutter war er immer fürsorglich und rücksichtsvoll. Fast schon nervend. Er hatte sie 
stets wie ein kleines Kind behandelt. Ihre Augen begannen zu brennen und sie biss sich auf die 
Zunge, um die Tränen zu verdrängen.\\
``Das heißt, wir sind Geschwister?'', brachte sie dann doch sarkastisch hervor.\\
Haska hielt an einer Kreuzung inne und blickte sie ernst an. ``Ich weiß, wie schlecht es euch geht. 
Ich kann euch helfen.''\\
Lavay ließ ihn ohne eine Antwort stehen und eilte durch ihr vertrautes Viertel. Während dem Rückweg 
beschäftigte Lavay die Frage, wie sie ihrer Mutter erklären sollte, das Imur hingerichtet worden 
war. Sie verdrängte ihre Trauer und das Entsetzten und zwang sich, objektiv zu denken. Haska hatte 
recht. Sie konnte weder die Miete noch den Lebensunterhalt zahlen. Mit hängendem Kopf betrat Lavay 
das heruntergekommene Wohnhaus, schlich wortlos an den zahlreichen Nachbarn vorbei und betrat ihre 
winzige Kammer. Ihre Mutter saß am Tisch und kauerte über einer Handvoll Gegenständen. Lavay trat 
vorsichtig zu ihr und strich ihrer Mutter fürsorglich eine Haarsträhne hinter das Ohr. Während Lavay 
die ältere Frau betrachtete, fragte sie sich, wann sie die Rollen vertauscht hatten. Hätte ihre 
Mutter sie nicht trösten müssen? Lavay machte Anstalten, von Imurs Tod zu berichten, als ihre Mutter 
sie anlächelte und auf den Tisch zeigte. "Sieh nur.``\\
Lavay Blick folgte dem Fingerzeig und sie musterte die Gegenstände. Ein alter Holzkamm, geschwärzt 
aber noch gut erhalten. Ein flacher, glatter Stein lag daneben. Außerdem eine ausgefranzte, 
blassrosane Haarschleife, eine Kette aus Holzperlen und eine ausgemergelte Holzfigur, die wohl 
einst einen Soldaten dargestellt hat.\\
"Mama'', begann Lavay stockend.\\
Ihre Mutter hielt ihr strahlend den Holzkamm entgegen. "Schau, den habe ich von deiner Großmutter 
zum Hochzeitstag geschenkt bekommen. Und hier, als du noch ganz klein warst, hast du diese 
Haarschleife getragen. Dein Vater hat sie damals für dich gekauft. Und Imur hat die Spielfigur 
bekommen.''\\
Lavay seufzte und griff nach dem flachen Stein. Sie tastete über die glatte Oberfläche. Imur hatte 
ihn damals, als sie beide noch Kinder waren, von der Tochter eines reisenden Händlers bekommen. 
Lavay, Imur und das Mädchen hatten in der kurzen Zeit viel gespielt und Unruhe gestiftet. Doch nach 
den zwei Wochen musste ihr Vater weiterreisen und als Abschied hatte sie ihnen den Stein geschenkt. 
Sie hatte ihn in ihrer Heimatstadt an der Küste gefunden und als Erinnerungsstück während der 
langen Reise mitgenommen.\\ 
Damals hatten Imur und Lavay diesen Stein gehütet wie einen Schatz. Das war bereits so viele Jahre 
her. Lavay dachte an Imur und seinen verzerrten Körper, der steif und kalt am Strick hing und vom 
Wind sanft hin und her geschaukelt wurde. Bald würden die Krähen und Aasfresser kommen und…
Nein, den Gedanken wollte sie nicht beenden. Lavays Finger verkrampften sich um den Stein.
"Imur ist tot'', sagte sie atemlos. \\

Lavay hätte erwartet, dass ihre Mutter schreien und toben würde. Dass sie die wenigen Möbel 
zerschlagen und sich selbst verletzen würde. Stattdessen hockte sie auf dem Schlaflager, vergrub 
ihre zitternden Finger im langen Haar und wippte stumm hin und her. Lavay hatte gedacht, dass sie 
eine Ewigkeit mit ihr darüber würde diskutieren müssen, bis sie es endlich einsah. So wie sie 
ihre Mutter jeden Tag daran erinnern musste, dass ihr Vater längst von Maden zerfressen in der Erde 
lag. Aber die ältere Frau hatte es wortlos hingenommen. Keine einzige Frage. Nicht wie oder warum. 
Der Zustand ihrer Mutter verunsichere Lavay zutiefst. Mit einer schreienden Wahnsinnigen hätte sie 
klar kommen können, damit hatte sie gerechnet. Aber das hier? Ein gebrochenes Wesen, mit dem nichts 
anzufangen war. Weder aß noch trank sie. Wenn Lavay ihre Mutter ansprach, folgte keinerlei 
Reaktion. Lavay wusste nicht, was sie tun sollte. Am Abend war sie dann so verzweifelt, dass sie 
sich neben ihre Mutter nieder ließ, ihr über den Rücken strich und sie fest in den Arm nahm. 
Krampfhaft versuchte sie, dem stetigen Wiegen ihrer Mutter Einhalt zu gebieten. \\
Irgendwann sah sie schließlich zu ihrer Tochter auf. Die Augen rot unterlaufen, das Gesicht 
durchzogen von Kummerfalten, die Lippen zitterten und die Tränen durchliefen ihr Gesicht. Aber das 
Erschütterndste war die Leere in ihren Augen. \\
\textit{Ich muss sie zurücklassen}, dachte Lavay entsetzt über den Zustand ihrer Mutter 
gleichermaßen wie über diesen Gedanken. War sie nun wirklich schon so weit gesunken, dass sie ihre 
Mutter, die Frau, die ihr das Leben geschenkt hatte, die sie geliebt hatte, bevor sie 
geboren war, bereit war, zu zurück zu lassen? Das wäre grausam und undankbar, aber hatte sie eine 
andere Wahl? \\
Ihre Mutter vergrub ihr Gesicht an Lavays Schulter. Monoton strich Lavay ihr durch 
das Haar und begann leise zu summen. Erst planlos und spontan, dann festigte sich die Melodie und 
sie sang leise das alte Wiegenlied, welches Generationen von armen und reichen Kindern von ihren 
Eltern, Großeltern, Ammen und Geschwistern vorgesungen wurde:\\
"Leg dich nieder, mein Kind\\
Flieg wie ein Vogel im Wind.\\
Blicke in den Himmel, strecke die Arme aus,\\
Erreiche die Sterne und tanze auf dem Mond.\\
Und Morgen, wenn die Sonne erwacht,\\
wird wieder gescherzt und gelacht. \\
Leg dich nieder, mein Kind.\\
Flieg wie ein Vogel im Wind.''\\
Lavay betrachtete zärtlich ihre Mutter, die tatsächlich eingeschlafen war. Sie hauchte ihr einen 
Kuss auf die Stirn, mehr wagte sie nicht. Sie wollte sie schlafen lassen, sich ausruhen und 
erholen. \\
Vorsichtig stand Lavay auf und sah sich in dem Zimmer um. Sie holte tief Luft und griff zu einem 
Stoffbeutel. Sie überlegte nicht lange. Essen, viel mehr könnte sie nicht mitnehmen. Geld besaßen 
sie keines und auch nicht viel mehr Kleidung. Den Rest der Nacht verbrachte sie auf dem Hocker neben 
dem winzigen Fenster und beobachtete ihre Mutter. Das Mondlicht bahnte sich seinen Weg durch das 
Fensterchen und beleuchtete den staubigen Boden. Reglos saß Lavay da, drehte und wendete den 
flachen Stein in ihren Händen. Immer wieder schwor sie sich, jetzt gleich aufzustehen und zu gehen. 
Aber etwas hielt sie zurück. Ihre Mutter sah so friedlich aus. Wenn sie sie hier zurück lassen 
würde, was dann? Der Vermieter würde sein Geld einfordern und da sie keines hatte, würde er sie 
rauswerfen. Und dann? Auf der Straße? Ihre Mütter würde keine drei Tage überleben…\\
Lavay gab sich schließlich geschlagen und seufzte resigniert. Den Beutel ließ sie auf dem Tisch 
liegen und legte sich vorsichtig neben ihre Mutter. Sie kuschelte sich an sie, schloss die Augen und 
atmete den vertrauten Geruch des unbequemen Lagers und ihrer Mutter ein. \\
Am Morgen, kaum als die Sonne sich über den Horizont erhoben hatte, brachen sie auf. Lavay und ihre 
Mutter wechselten kaum ein Wort und sie erklärte der älteren Frau weder Grund des Aufbruchs noch 
Reiseziel. Lavay vergaß es schlicht, da ihre Mutter nicht fragte und stumm den knappen 
Aufforderungen nachging und das Nötigste zusammenpackte. Als sie das Stadttor mit seinen 
zahlreichen wehenden Fahnen erreichten, atmete sie unwillkürlich auf. Das tiefe Blau der 
kasirischen Flagge schien den Herbsthimmel mit seiner Farbenpracht herausfordern zu wollen. Die 
junge Frau durchquerte das Tor, ohne sich umzublicken. Warum auch? Diese Stadt hatte ihr 
alles genommen. So wenig sie auch besessen hatte. Den Vater, den Bruder und ihre Sicherheit. Ein 
Neuanfang, dass war, was Lavay wollte und ihre Mutter brauchte.\\