\chapter{Die falsche Wahl}


''Können…'', stammelte Lavays Mutter: ''Können wir rasten?''\\
Lavay seufzte und sah in den Himmel. Die Sonne hatte ihren Zenit längst überschritten und sie 
hatten in den letzten zwei Tagen ein gutes Stück zurückgelegt. Doch trotzdem war der Weg noch weit. 
Lavay wollte heute noch die nächste Stadt erreichen und dort eine billige Unterkunft finden. Sie 
waren auf den Beinen, seit das Sonnenlicht es erlaubte. In den Nächten blieb Lavay schlaflos. Ab 
und an dämmerte sie ein, bis das Heulen eines Wolfs sie wieder weckte. Während ihre Mutter unruhig 
schlief und sich auf ihrem Lager hin und her wällzte, kämpfte Lavay damit wach zu bleiben. Es war 
zu gefährlich. Sie suchte zwar jeden Abend einen abgeschiedenen Ort, fern der Straße, aber wenn sie 
keine menschlichen Räuber fanden, dann die Raubtiere. Es hieß, die Wölfe kämen mit dem Schnee. Noch 
war es etwas zu früh für Schnee in der südlichen Province Kasirs, aber Lavay kannte sich mit Tieren 
und Wetter nicht sonderlich aus und wollte nichts riskieren.\\
Lavay nickte ihrer Mutter zu und ließ sich am Straßenrand in den Staub nieder. Sie gab ihr einen 
Teil des Proviants und nahm sich selbst ein Stück Brot und Käse. Verträumt kaute Lavay und malte 
sich ihr zukünftiges Leben aus.\\
Ein kompletter Neuanfang. Niemand wusste woher sie kam und wer sie war. Sie ließ die Elendsviertel 
hinter sich und streifte die Fesseln ihrer Geburt ab. Lavay würde sich saubere Kleidung suchen, 
vielleicht sogar ein Kleid und sich in den ehrbaren Gasthäusern vorstellen. Ihre Mutter könnte 
vielleicht als Küchenhilfe etwas finden. Kochen konnte sie gut, erinnerte sich Lavay an die Zeiten, 
in denen ihre Mutter noch täglich etwas zubereiten konnte. An die Zeiten, in denen genug Essen 
vorhanden war um anständige Mahlzeiten daraus zu machen. Wenn der Wirt großzügig war, könnten sie 
vielleicht in einer Kammer des Gasthauses leben und dafür auf einen Teil des Lohns verzichten.\\
Lavay setzte gerade dazu an, ihre Mutter in diese Pläne einzuweihen, als sie inne hielt. Die ältere 
Frau saß in sich zusammengesunken auf dem feuchten Gras und beugte sich über ihr Stück Brot, 
welches sie zitternd in der Hand hielt. Sie hatte die Augen geschlossen und Schweißtropfen 
bildeten sich auf ihrer Stirn. Die linke Hand presste sie fest gegen ihren Körper. Hastig rutschte 
Lavay an die Seite ihrer Mutter und legte prüfend die Hand auf ihre Stirn. Ihre Mutter glühte vor 
Fieber.\\
''Mama'', entwich es Lavay entsetzt.\\
''Müssen wir schon weiter?''\\
Lavay schüttelte den Kopf und streckte die Hand aus. ''Zeig mir deine Hand!''\\
Ihre Mutter wich ihrem Blick aus und Lavay griff nach der verkrampften Fingern. Sanft strich sie 
den langen Ärmel des Kleids zurück. Die Hand war in einen schmutzigen Stoffstreifen gehüllt. Lavay 
kniff ihre Lippen fest aufeinander. Angst erfüllte sie. Und Wut. Den provisorischen Verband löste 
sie mit wenigen, groben Bewegungen und starrte auf die entzündete Wunde. Sie hatte genug gesehen, 
ließ ihre Mutter los und sprang auf die Beine. Unruhig lief sie einige Schritte hin und her.\\
''Warum hast du nichts gesagt?'', rief Lavay und ballte die Hände zu Fäusten. \\
Ihre Mutter hob in einer hilflosen Geste den unverletzten Arm. ''Ich wollte nicht… du hast zur Eile 
gedrängt und ich… schreie mich nicht an…'', stammelte sie.\\
''Zeig!'', forderte Lavay barsch und packte den Arm erneut. Sie schob den Ärmel des Kleides zurück 
um zu überprüfen, wie weit die Entzündung bereits vorangeschritten war. Viel zu weit.\\
Lavay kannte sich nicht mit Heilkunde aus, aber sie hatte solche Verletzungen schon oft gesehen. In 
ihrem Viertel kam es nicht selten vor. Bei manchen Bettlern konnte man es sogar riechen.\\
''Das kann einfach nicht wahr sein'', murmelte Lavay, während sie den unruhigen Gang fortsetzte.
Ihre Mutter kauerte zitternd am Boden. Tränen liefen ihr über die Wangen und sie starrte stumm auf 
ihre verletzte Hand. Verzweifelt verabschiedete die junge Frau sich von ihren Plan und überlegte, 
was sie nun mit ihrer Mutter anstellen sollte. Sie brauchte einen Arzt oder noch besser, eine 
Kräuterfrau und Heilkundige. Das wäre um einiges billiger. Außerdem hielt Lavay nichts von den 
fettleibigen Ärzten, denen sie bisher begegnet war.\\ 
Lavay kehrte zu ihrer Mutter zurück und streckte die Hand auf, um ihr aufzuhelfen. 
''Komm. Ich helfe dir. Bis zur nächsten Stadt müssen wir es schaffen und dann können wir jemanden 
suchen, der dir hilft. Bestimmt.''\\
''Ich…'', krächzte ihre Mutter und schüttelte den Kopf: ''Ich schaffe das nicht. Es tut so weh… Wie 
Feuer in meinen Adern.''\\
Lavay sah sich ratlos um. Sie waren alleine auf der Straße und nirgends war eine Menschenseele zu 
sehen.\\
''Ein bisschen Schlaf'', murmelte ihre Mutter: ''Dann geht es mir sicher besser!''\\
Lavay glaubte nicht daran, aber sie hatte selbst keine andere Idee. Also ließ sie ihre Mutter sich
hinlegen, stützte ihren Kopf mit dem Stoffbeutel und deckte sie mit ihrer fadenscheinigen Wolldecke 
zu. Fürsorglich kniete sie sich nieder und gab der Verletzten etwas zu trinken aus dem 
Wasserbeutel. Nach wenigen Augenblicken war ihre Mutter eingeschlafen und Lavay streckte seufzend 
ihre Glieder aus. Sie hatte keine Ahnung, wie schlimm die Verletzung war und wie sie sie versorgen 
sollte. In ihrer jugendlichen Naivität redete sie sich ein, dass ein bisschen Schlaf und Erholung 
ihrer Mutter ausreichend helfen müsste, um am nächsten Tag in der Stadt eine Heilkundigen zu suchen. 
Die Frage, wie sie diese bezahlten sollte, verdrängte Lavay aber. Ebenso wie die leise Stimme der 
Vernunft in ihr, die ihr böse zuflüsterte, dass das Schicksal ihrer Mutter längst besiegelt war und 
sie lieber beginnen sollte, ein Grab auszuheben. Lavay hätte sich gerne auf die Suche nach einem 
Bachlauf gemacht, aber sie fürchtete sich davor, ihre Mutter allein zu lassen. Momentan war zwar 
keine Seele auf der Straße zu sehen, aber das Gelände war unübersichtlich und schon hinter der 
letzten Bergkuppe könnten Fremde sein. Nein, sie wollte nicht, dass irgendwelche Leute ihre 
wehrlose Mutter am Wegesrand fanden.\\
Ihrer Mutter entwich ein gequältes Stöhnen. Lavay richtete sich auf. Der Schlaf der Verletzten war 
plötzlich alles andere als ruhig. Sie wand sich im Fieberwahn, als kämpfte sie gegen eine wilde 
Bestie. Sie jammerte und weinte, rief um Hilfe und schlug um sich. Lavay rüttelte verzweifelt an 
ihren knochigen Schultern. Sie riss die Augen auf und sah ihrer Tochter direkt ins Gesicht ohne sie 
zu erkennen.\\
''Mama'', sprach Lavay sie zaghaft an. Sie sahen einander einen Moment in die Augen. Sie hatte 
schöne Augen, wurde Lavay plötzlich bewusst, als hätte sie noch nie vorher genau hingesehen. Augen 
in einem sanften Braun. Augen, die niemandem etwas zu leide tun konnten. Augen, die solch grausames 
Schicksal wie das, welches ihre Mutter erleiden musste, nicht verdient hatten! Der Tod des Gatten, 
Leben in der Armut, ein Sohn der sich zu den schlimmsten Verbrechen gezwungen sah und dafür 
gerichtet wurde. Und nun dieser weiterer Schlag des Schicksals. ''Mama!'', wiederholte Lavay und 
Tränen liefen ihr über die Wangen.\\
Ihre Mutter sank zurück auf den Boden und blickte starr in den Himmel. Ihre Lippen bewegten sich, 
aber Lavay verstand die gebrabbelten Worte nicht.\\
''Hier'', sagte sie: ''Trink etwas!''\\
Ihre Mutter reagierte nicht auf ihre Worte und beachtete auch den dargebotenen Wasserbeutel nicht. 
Lavay ließ die Hände sinken und beobachtete hilflos, wie ihre Mutter vom Fieber gepeinigt wurde. 
Die junge Frau zwang sich, ruhig zu bleiben und zu überlegen. Das Wasser ging ihnen aus, ebenso der 
karge Proviant. Und die Aussicht, die Nacht hier am Wegesrand zu verbringen, war alles andere als 
verlockend. Sie waren alles und jedem ausgeliefert.\\

Stunden verstrichen und der Abend dämmerte bereits. Ihre Mutter war nicht mehr ansprechbar und 
Lavay hatte die Versuche auch aufgegeben. Ein kleiner Gedanke war immer lauter geworden. Es 
wäre besser, wenn sie jetzt starb. Hier. Sie könnte ihre Mutter liegen lassen. Wenn niemand Lavay 
dabei sah, dann könnte niemand sie dafür anklagen. Imurs Seele würde sie finden. Sie wären wieder 
beisammen und Lavay könnte endlich weit genug fliehen um alles hinter sich zu lassen. Aber noch 
atmete die Sterbende rasselnd und stockend. Noch zuckten ihre Augenlieder. Noch hielt sie die Hand 
ihrer Tochter festumklammert.\\
Und als das Rattern von Wagenrädern und die trottenden Schritte eines Pferdes näher kamen, wusste 
Lavay, dass sich das alles erledigt hatte. Es gab Zeugen. Sie blieb der Straße den Rücken zugekehrt 
sitzen und schloss die Augen. Eine Träne rann Lavay über die Wange und sie biss sich auf die 
Zunge, bis sie Blut schmeckte.\\
Die Geräusche wurden immer lauter, kamen immer näher, bis sie schließlich verstummten. ``He, 
Fräulein!'', rief eine Männerstimme: ``Was ist los? Brauchst Hilfe?''\\
Lavay drehte sich schweren Herzens um und sah zu dem Sprecher auf. Auf einem klapprigen, 
zweirädrigen Wagen hockte ein graubärtiger Mann. Unter den ausgefransten Strohhut blickten ihr zwei 
freundliche Augen fragend entgegen. Er schnalzte mit der Zunge und riss an den Zügeln. Der 
ebenso graue Wallach mit dürren Beinen und zerzauster Mähne zog die Kutsche an den Straßenrand, 
stoppte abrupt vor Lavay und blies ihr warmen Atem ins Gesicht.\\
''Na?'', wiederholte der Wagenlenker und zeigte etliche Zahnlücken, als er sie freundlich 
angrinste: ''Brauchst du Hilfe? Oder biste der hübsche Lockvogel einer Räuberbande, die mir alles 
nehmen will?'' Lachend breitete er die Arme aus. ''Ich hab aber nix. Darfst gerne nachgucken. Nur 
Hut und Pferd!''\\
Ernst blickte sie zu ihm auf. ``Meine Mutter hat Fieber'', antwortete sie nur.\\
Sein Lachen verfloss und jetzt erst schien er den reglosen Körper am Boden wahrzunehmen. Er 
schirmte seine Augen mit der Hand vor dem Licht der Abendsonne ab und blinzelte mehrmals. ``Dann 
auf, ich nehm euch mit nach Irsil. Ist nur noch ein kurzes Stück. Dann kannst du auf dem Markt nach 
einem Heiler suchen.''\\
Lavay regte sich nicht. Er sah doch, dass sie nichts bei sich trug. Jemand mit genügend Geld für 
einen Heiler reiste nicht zerlumbt und ohne Gepäck. ``Ich glaube'', sagte sie leise: ``Sie wird 
es nicht mehr schaffen.''\\
Er überlegte kurz, dann verfinsterte sich seine Miene und Lavays letzte Hoffnung verfloss. ``Dann 
trotzdem auf. Willst sie hier ja wohl nicht liegen lassen wie einen Verurteilten?''\\
Er würde sie nicht gehen lassen. Sie würde mit nach Irsil kommen, einen Bestatter finden, sich 
verschulden und dafür auch im Stadtregister eintragen lassen müssen. \textit{Und die nächsten 
Fesseln.}\\
Kurz flackerte der Wunsch in ihr auf, sich einfach abzuwenden. Zu rennen und sich nie wieder 
umzudrehen. Aber noch war ihre Mutter nicht gegangen. Noch konnte Lavay sie nicht im Stich lassen. 
\textit{Wie Haska sagte. Für eine Leiche lohnt es sich nicht.}\\
Sie ließ den Kopf hängen und nickte. Der Kutscher stieg ab und räumte seine Transportfläche frei. 
Es war nur wenig Platz. Lavay stieg zuerst darauf und nahm den schlaffen Oberkörper entgegen. Es 
war anstrengend und ihre Arme zitternden bald. Während der Fahrt hielt sie ihre Mutter fest 
umklammert und starrte zu den Sternen. Sie sah Irlies Stern, Jerics, Elrims und Maras. Und so viele 
mehr. Und irgendwann, während sie sich in den Geschichten dieser Männer und Frauen verlor, schlug 
das Herz ihrer Mutter zum letzten Mal.\\

Die Nacht verbrachte sie neben ihrer Mutter. Die Bestattungshalle Irsils lag im westlichen 
Viertel der Stadt. Der Kutscher hatte sie bis zum Tor gefahren und ihr dann noch geholfen eine 
provisorische Trage zu finden. Er stellte keine weiteren Fragen, aber behielt Lavay im Auge. Erst 
als sie mit einem der Bestatter gesprochen und einen Schuldvertrag aufgesetzt hatte, verließ er sie 
mit einem nicken.\\
``Morgen früh wird ein Stadtverwalter kommen um dich in das Verzeichnis einzutragen'', erklärte der 
Bestatter in einem müden, leiernden Tonfall und setzte seine Unterschrift unter den Schuldvertrag 
neben Lavays Gekritzel. Sie konnte weder Lesen noch Schreiben, aber die Zahl ihrer Schulden sah 
beunruhigend lang aus. ``Bis dahin verlass die Halle nicht. Dort drüben findest du alles um die 
Rituale durchzuführen.''\\
Lavay nickte nur und kniete sich neben die Leiche. Sie sah dem Mann zu, wie er zur nächsten Trage 
ging. Eine Familie, heulend und schluchzend, kauerte neben dem Körper eines alten Mannes. Einige 
der wenigen, die sich nicht für die Bestattung verschuldeten. Lavay wandte sich ab. Eine Zeit lang 
saß sie tatenlos da und betrachtete das erschöpfte Gesicht ihrer Mutter. Es tat weh. Sie war 
allein. Imur war gegangen. Ihre Mutter war gegangen. Ihr Vater nicht mehr als eine Erinnerung. Aber 
der Hunger war geblieben.\\
Die junge Frau vollführte die Rituale nicht. Sie wusch weder die Haut noch salbte sie die Leiche 
oder sang die Lieder.\\
Sie grübelte und grübelte. Suchte den Punkt in ihrem Leben, an dem alles den Bach runter gegangen 
war. An dem sie den Zenit des Glücks überschritten hatte und nur noch Leid auf sie wartete. Ihr 
Magen knurrte und erinnerte sie schmerzhaft daran, dass sie schon viel zu lange nichts mehr 
gegessen hatte. Lavay biss sich auf die Lippen. Sie dachte an die Lieder ihrer Kindheit. Die Tänze 
mit den anderen Kindern. Die Prügeleien und Mutproben. Damals, als sie noch dachte, irgendwann 
würde alles besser werden. War es mutig oder feige so ein Leben nicht führen zu wollen?\\
Etwas nasses berührte sie flüchtig an der Hand und Lavay zuckte zusammen. Der Hund leckte sich über 
die Nase und blickte sie aus blauen Augen an. Ein verlaustes, mageres Vieh mit zerupften Fell und 
nur drei Beinen. ``Verzieh dich!'', zischte Lavay: ``Ich habe nichts zum teilen!''\\
``Hey,'' tönte es hinter ihr und der Hund zog den Schwanz ein. Er winselte kläglich, als ein Schuh 
geflogen kam und ihn traf. Das Tier war so schwach, dass ihm davon die Beine einknickten. Mühsam 
versuchte er sich wieder aufzurichten. Lavay betrachtete ihn skeptisch. Sie sah keine Angst in 
seinen ungewöhnlichen Augen. Auch nicht die übliche Naivität eines bettelnden Köters. Aber 
Trotz.\\
``Schafft das Vieh aus der Halle der Gegangenen!'', verlangte der Familienvater neben dem alten 
Mann.\\
Einer der Bestatter kam gemächlich näher, wog seinen Stock in den Händen und nickte nur als 
Antwort. Lavay begegnete den Blick des Hundes. Er war ruhig. Wartete. Und sah dabei nur sie an.
\textit{Du wirst nicht sterben}, beschloss Lavay: \textit{Nicht heute Nacht.}\\
Sie erhob sich und innerhalb dieser Geste ließ das Tier sich neben ihrer Mutter nieder. Mit einem 
Grummeln legte er seine Schnauze auf die Pfoten. Trotz der entspannten Haltung verfolgte sein Blick 
sie wachsam. ``Nein!'', rief Lavay und fügte kleinlauter hinzu: ``Bitte. Er gehört zu meiner 
Familie.''\\
Dem Bestatter war es egal, dass sah sie ihm an. Der Mann wartete auf die Reaktionen der Trauernden. 
Kaum einer sah von den Ritualen auf und schließlich seufzte der Familienvater, umschlang seine 
Tocher und winkte ab, ehe er sich wieder dem Gegangenen zuwandte.\\
Erschöpft und verwirrt setzte die junge Frau sich nieder. Anscheinend schlief der Köter schon. 
Hatte sich dicht an den kalten Körper ihrer Mutter gedrängt.\\
Als sie am Morgen erwachte, war das Tier weg und eine ungeduldige Stadtverwalterin sah auf sie 
herab.\\


''Verzeihung”, murmelte Lavay und wurde rot, als Tropfen der cremigen Kartoffelsuppe auf dem 
Wams des Kunden landeten. Der Mann blickte zweifelnd zu ihr hinauf und runzelte die Stirn.
''Junges Fräulein, seid Ihr nicht in der Lage, einen  Suppenteller zu tragen?”\\
Er klang sehr amüsiert und tauschte einen Blick mit seinen Kameraden.\\
''Doch, ich meine... es tut mir sehr Leid... Ein Versehen...”\\
''Na, ich hoffe doch, dass es keine Absicht war. Wenn, hättet Ihr mir besser gleich den gesamten 
Teller über den Kopf kippen sollen. Nun bring mir noch einen Krug Bier, Mädchen.”\\
''Gewiss... vielen Dank”, stammelte sie und machte sich eilig auf den Weg zurück zum Tresen. 
Zumindest nahm der Mann es mit Humor. Die Wirtin aber nicht. Sie stand am Tresen und funkelte Lavay 
aus kleinen Augen böse an. Sie war eine großgewachsene, hagere Frau. Die Nase war zu groß für das 
Gesicht und ihre Stirn schien stets skeptisch gerunzelt.\\
''Was sollte das schon wieder?”, schnappte sie Lavay an, kaum dass sie in Hörweite war.\\
''Ein Versehen...”, wiederholte Lavay beschämt.\\
''Das wievielte heute?”\\
Lavay fürchtete, dass die Wirtin ihr gleich wieder eine Ohrfeige geben würde, aber dafür sahen zu 
viele Gäste herüber. Vermutlich würde sie es sich bis zum nächsten Morgen aufheben oder aber ihr 
einfach noch mehr Geld vom Lohn abziehen. Er war eh schon wenig genug. Von den wenigen 
Münzen, die sie in der Woche bekam, musste sie fast die Hälfte abgeben, um die kleine Kammer und 
das karge Mahl zu bezahlen. Außerdem hatte der Wirt verlangt, dass sie ein Kleid trug, wenn sie die 
Gäste bediente. Damit hatte Lavay bereits gerechnet. Fies war es jedoch gewesen, dass das Ehepaar 
ihr ein gebrauchtes Kleid und eine Haube für das Haar gab und den Preis dafür als Schulden 
anrechnete. Lavay würde schon zwei Wochen arbeiten müssen, um das Kleid, die Haube und das paar 
ausgetretene Schuhe zu bezahlen. Es würde kaum etwas bleiben, um ihre Schulden für die Beerdigung 
ihrer Mutter zu zahlen. Schon zwei Mal in den letzten Wochen kam einer der Stadtverwalter um sie an 
die lange, lange Zahl zu erinnern. Und dreimal hatte sie sich vor ihnen versteckt.\\
Lavay nahm den Krug und brachte ihn an den Tisch der drei Herren. Während sie zu einer Gruppe 
Neuankömmlinge eilte, um deren Bestellung aufzunehmen, spürte sie deutlich die Blicke der Männer 
auf sich ruhen. Ihr lief ein Schauer über den Rücken. Das war das widerlichste an dieser Arbeit. 
Die Männer an sich, ihre Blicke und spottenden Worte speziell. Manch einer, der schon etliche Krüge 
geleert hatte, blieb auch nicht bei Blicken und Worten, sondern griff zu. Lavay hatte dem ersten 
einen ordentlichen Stoß gegeben, sodass er von der Bank plumpste und dort benommen liegen blieb. 
Noch dazu hatte sie ihm eine ordentliche Welle von Schimpfwörtern und Flüche entgegen geworfen. 
Woraufhin die Wirtin ihr deutlich klar gemacht hatte, dass sie so ein Verhalten von Angestellten 
nicht dulden konnte. Bei weiteren ähnlichen Vorfällen reagierte Lavay also wie ihre zwei 
Kolleginnen. Sie schwieg und ging weiter. Es viel ihr schwer. Aber sie brauchte Geld.\\
Es wurde später und die Gäste, vorwiegend Männer, betrunkener. Lavay sehnte sich danach, dass auch 
bald der letzte Gast gehen würde. Heute war nämlich sie diejenige, die von den Mädchen bis zum 
Schluss bleiben musste. Die Wirtin würde sich bald verabschieden und der Wirt, der meistens in der 
Küche blieb war schon längst gegangen. Es wurde auch nichts anderes als Bier, Wein und Met 
mehr ausgeschenkt. Trotz der späten Stunde, waren noch viele Gäste da. Lavay befürchtete, dass es 
weit bis in die Nacht dauern würde, jeden loszuwerden. Und trotzdem musste sie am nächsten Morgen 
wieder im Gasthaus stehen, den Boden fegen und die Tische schrubben.\\
Lavay registrierte aus den Augenwinkeln, wie ein deutlich betrunkener Mann sie schmierig angrinste 
und eine Hand ausstreckte. Sie versuchte, ihm auszuweichen, doch da griff er blitzschnell nach dem 
Rock ihres Kleides und Lavay verlor das Gleichgewicht. Die Teller fielen scheppernd zu Boden und 
zerplatzten zu aberhunderten Scherben. Ehe Lavay sich versah, baute sich schon die hagere Wirtin 
vor ihr auf und schimpfte lautstark: ''Jetzt reicht es mir mit dir! Schon wieder eines deiner 
Versehen? Das ist doch nicht zu fassen. Du lernst es wohl nie!''\\
''Aber... ich kann doch gar nichts... das...'' Lavay traten Tränen in die Augen. Sie wusste schon, 
was jetzt kommen würde.\\
''Nein, keine Ausreden du faules Stück! Kehr die Scherben auf und dann verzieh dich in deine 
Kammer. Eine der Anderen wird deinen Dienst heute übernehmen. Sonst legst du vermutlich noch die 
ganze Einrichtung in Schutt und Asche! Deinen Lohn morgen kannst du streichen. Der geht dafür 
drauf, die ganzen Teller zu bezahlen und die Gäste zu entschädigen!''\\
Lavay verkniff sich das Weinen und bückte sich, um die Scherben aufzusammeln. Dabei schnitt sie 
sich an einer scharfen Kante und ihr entwich ein leises Schluchzen.\\

''Sieh es so, eigentlich hast du es ganz gut getroffen. Das Gasthaus zum steinernen Tor ist nichts 
im Vergleich zu meiner früheren Arbeitsstelle... da gab es jeden Tag Schlägereien und nur die 
widerlichsten Gäste aus den Armenviertel kamen. Sie betranken sich jeden Tag mit dem billigen 
Gesöff, dass der Wirt dort anbot und brachten ihre Flöhe und weitere Ungeziefer mit! Es war 
schrecklich, sage ich dir. Ich meine, die Gäste vom steinernen Tor sind ja schon praktisch reich. 
Und wenn man freundlich ist, geben sie gutes Trinkgeld...''\\
''Freundlich'', wiederholte Lavay bitte und vergrub ihr Gesicht in dem dünnen Kissen. Sie hoffte 
ihre Kollegin würde endlich den Mund halten. Sie konnte es nicht mehr hören. Freundlich war sie, 
ohja. Sie ließ sich begrapschen und lächelte dabei.\\
''Gute Nacht!'', knurrte Lavay und hoffte, dass würde sie endlich zum Schweigen bringen.\\
''Pfff'', gab sie von sich und fügte hinzu: ''Wenn du so weiter machst, brauchst du dich nicht 
wundern, wenn du bald wieder auf der Straße sitzt.''\\
Lavay war sich ziemlich sicher, dass das wirklich bald geschehen würde. Also müsste sie vorher noch 
so viel Geld wie möglich auftreiben.\\
\textit{Beobachte.}\\
An sich, wenn sie nicht gerade Suppenteller trug, hatte sie im Gastraum stets die Gelegenheit zu 
beobachten. Aber nicht um zu handeln. Das wäre am leichtesten, wenn sie die Zimmer putzte. 
Was aber an sich nur recht selten geschah oder auf ausdrücklichen Wunsch eines Gastes. Da blieb die 
Möglichkeit während dem Nachtdienst. Aber die Betrunkenen, die so spät noch da waren, konnten 
meistens kaum genügend Geld zusammenkratzen, um zu zahlen.\\
Somit begnügte sie sich anfangs damit, den letzten Gästen einen höheren Preis zu nennen und das 
Geld für sich einzustecken. Die waren dann nur selten in der Lage überhaupt zu zählen, wie viele 
Krüge sie geleert hatten und Lavay vermutete, dass sie nicht die Einzige war, die großzügig 
abrechnete. Aber es reichte nicht.\\
So kam der Tag, an dem sie ihren freien Nachmittag in der Woche hatte, und wieder stand sie mit 
fast leeren Taschen da. Heute würde man sie im Rathaus erwarten. Wenn sie nicht kämen, dann würden 
sie sie suchen und ihre Erinnerungen wurden immer strenger. Mit hängendem Kopf putzte sie über 
einen der Tische. \textit{Vielleicht stecken sie mich gleich in die Zellen}, dachte sie 
beklommen.\\ 
Lavay hatte darüber nachgedacht, es darauf ankommen zu lassen. Aber es gab niemanden, 
der sie frei kaufen könnte. Sie würde ihr Leben dort verbringen und vermutlich früh sterben. Wenn 
sie weglief würde man sie suchen. Wie es für Imur wohl war, den Strick zu spüren? Die Leere unter 
den Füßen?\\\
''Mädchen'', wurde sie angesprochen.\\
Lavay sah auf und wandte dann sofort überrascht den Blick wieder ab. Betreten musterte sie die 
Schuhe des Sprechers. Es war lange her, dass sie einem Adeligen begegnet war. Wenn, dann nur aus 
der Ferne. In ihr Viertel hatten sie sich selten verirrt und erst recht nicht in das Gasthaus zum 
steinernen Tor. Dort kehrten neben den Stammgästen des Virtels eher Kaufleute auf Reisen ein oder 
wohlhabendere Bauern, die ihre Waren in der Stadt verkauften.\\
Sie kam nicht dazu, den Mann zu begrüßen. Die Wirtin kam aus der Küche geschossen und rief laut 
durch den gesamten Schankraum: ''Willkommen, mein Herr. Was darf ich Euch und Euren Männern 
anbieten?''\\
Die Männer traten an die Schänke und Lavay hatte die Gelegenheit, sie zu mustern. Der Adelige an 
sich war recht klein, kleiner gar als Lavay. Sie schätzte ihn auf Anfang Zwanzig, kurzes braunes 
Haar, glattrasiertes Gesicht. Ohrringe, Armbänder und eine glänzende Brosche. Seine Kleidung war 
bunt und wirkte fremdländisch. \textit{Ein Schnösel}, entschied Lavay.\\Das sah sie auch seiner 
Mimik an, als er mit der Wirtin sprach. Verstehen konnte sie nichts, darum trat sie unauffällig 
näher und beobachtete die Begleiter des Jungen. Groß, robuste Kleidung und bewaffnet.\\
''Eines der Mädchen wird Euch ein Zimmer richten, verehrter Herr!'', säuselte die Wirtin.\\
Lavay entdeckte Marga und Terim, die tuschelnd beieinander standen und dem Adeligen schmachtende 
Blicke zuwarfen. \\
''Nein'', entgegnete er: ''Das ist unnötig. Es regnet in Strömen und ich werde lediglich 
solange ihr bleiben, bis es aufhört. Ich konnte auf die Schnelle nichts Besseres finden.''\\
Die hagere Wirtin wurde rot. Schwer zu sagen ob vor Zorn oder Scham. Dann kicherte sie nervös wie 
ein junges Mädchen. ''Dann darf ich Euch Trunk anbieten?''\\
Der Adelige runzelte die Stirn. ''Ist er denn genießbar?''\\
Lavay konnte nicht mehr an sich halten und rief halblaut: ''Gestorben ist noch keiner. Zumindest 
hat sich keiner von denen beschwert.''\\
Der junge Mann sah mit einem Lächeln auf den Lippen zu ihr. ''Dann bring uns etwas.''\\
Sie ließen sich an einem Tisch nieder und Lavay nahm drei Krüge von der Wirtin entgegen. Sie sah 
sie so böse an, dass Lavay damit rechnete, die hagere Frau würde sie gleich anfauchen und spitze, 
giftige Zähne entblößen.\\
Ohne auch nur einen Tropfen zu verschütten, Lavay hatte mittlerweile etwas Übung, trug sie das 
Tablett zu den Männern. Seine Leibwächter murmelten auf ihn ein. ``Es ist keine gute Idee, 
Herr!''\\
``Man könnte Euch erkennen! Denkt daran, was Eurem Vater passiert ist.''\\
Lavay erkannte Geflüster, was nicht für Fremde Ohren bestimmt war. Eilig stellte sie die Krüge in 
die Tischmitte und wollte nur weg. Der Griff des jungen Mannes war sanfter als die Hände der 
Betrunkenen. Aber das funkelnde Gold gaben dieser Handlung mehr Gewicht. Lavay erstarrte. Ihr Blick 
wanderte langsam von dem goldenen Ring den Arm hinauf und schließlich sah sie in seine braunen 
Augen. Marga und Terims Lachen war längst verstummt, aber Lavay spürte, dass ihre Kolleginnen sie 
nicht aus den Augen ließen. ``Habt Ihr noch einen Wunsch, Herr?'', krächzte Lavay und spürte 
überdeutlich die Stelle, an der er sie festhielt.\\
``Nur eine Frage'', erklärte er mit gesenkter Stimme und nickte seinen Männern zu: ``Kommst du 
von hier?''\\
Lavay biss sich auf die Lippen und schüttelte den Kopf. ``Nein. Aus Morach'', log sie.\\
``Dann bist du ja durch das ganze Land gereist'', bemerkte er.\\
Sie hatte keine Ahnung wo dieses Kaff lag, nickte aber bestätigend und fragte sich, wieso er sie 
immer noch nicht losließ.``\\
''Dann kamst du auch durch Janka?``\\
Alle befestigten Reiserouten führten durch Janka, also nickte sie nur.\\
''Sagt dir der Name Karkos etwas?``\\
Lavay verneinte Kopfschüttelnt, entschlüpfte seinem Griff und wandte sich eilig einem rufenden Gast 
zu. Angespannt stand sie am Nachbarstisch, versuchte gleichzeitig der Bestellung des Gastes und den 
Worten des Adeligen zu lauschen. Keines von beidem gelang ihr.\\
Die Herren blieben nicht lange. Sie tranken und als der Regen nachließ, zahlten sie und verließen 
die Gaststätte. Erst als die junge Frau an den Tisch trat und die drei Krüge ergriff, drang das 
Wort zu ihr durch.\\
\textit{Karkos}\\
Lavay wurde schlecht. Ihre Hand wanderte tastent über ihre Stirn und sie ließ sich auf die Bank 
nieder. Tief holte sie Luft und bemühte sich, nicht in Tränen auszubrechen. Haska hatte diesen 
Namen ausgesprochen. Sie hatte nicht nachgefragt. Wozu auch? Imur war tot.\\
Aber wenn dieser junge Mann der Sohn war. Und Karkos noch lebte, weil Imur versagt hat... Diese 
Familie war reich. Das bedeutete, sie war auch mächtig. Sie hatte ihr Leben zerstört.\\
''Hey``, giftete die Wirtin.\\
Als Lavay nicht einmal zu ihr aufsah, senkte sie die Stimme. ''Ist die schlecht? Oh, nicht schon 
wieder eine schwanger!``\\
Lavay blinzelte sie stumm an und wusste nicht, was sie darauf antworten sollte. Ehe sie sich 
versah, drückte ihr die Wirten drei Münzen in die Hand und schob sie aus dem Schankraum. ''Geh in 
die Kräutergasse und frag nach Maras Gnade. Terim war erst letzten Monat da. Das Geld müsste 
reichen. Morgen hast du frei. Und dann passt du gefälligst auf, wenn du dich das nächste mal mit 
einem Kerl ins Heu verziehst! Selbstverständlich geht das Geld von deinem Lohn ab.``\\
Sprachlos stand Lavay im Nieselregen und sah noch zu, wie ihr die Wirtin mit einem letzten Blick in 
dem eine Spur von Anteilname lag, zu nickte. 

Die ersten Stunden verbrachte Lavay damit, ziellos durch die Straßen zu wandern. Vorbei an den 
üblichen Bettlern, den bekannten Gesichtern der Straßenkindern und der Stadtwache. Mit jedem 
Schritt wurde sie überzeugter. Imur hatte stets behauptet, dass ein Großteil seiner Arbeit die 
Planung war. Lavay war anderer Meinung. Manchmal kam es auch einfach auf Schnelligkeit an. Die 
Münzen der Wirtin gab sie dafür aus, nach und nach den ein oder anderen Bettler nach dem Adeligen 
zu fragen. Mancher wusste etwas, mancher nicht. Bei manchen bedankte sie sich mit Münzen, andere 
winkten ab. Trotz des Gasthofkleides sah man ihr an, wo sie herkam, erklärte eine zahnlose Frau und 
gab ihr den letzten entscheidenden Hinweis mit einem flüchtigen Kopfnicken hin zu einem hoch 
aufragenden Gasthaus. Der Name des Hauses war mit vergoldeten Buchstaben über den breiten Türrahmen 
gepinselt. Das Holz glänzte poliert. Sogar Glasscheiben konnte es vorweisen. Lavay betrachtete 
nachdenklich die Schrift, die sie nicht lesen konnte, und nickte. \textit{Was habe ich zu 
verlieren?} 
Ihre Entscheidung war gefallen.\\
Sie wartete bis tief in die Nacht. Der Regen ließ nach und einige Stunden gesellte sie sich zu der 
zahnlosen Frau. Ab und an warfen Passanten ihnen Münzen hin und Lavay griff ohne zögern zu. Sie 
konnte jede gebrauchen. Reiche gingen an ihnen vorbei, würdigten den Bettlern keines Blickes. Die 
meisten traten ein in das Gasthaus mit der Goldschrift. Lavay sah auch den jungen Adeligen und 
seine beiden Söldner. Es war schwer, sie nicht zu offensichtlich anzustarren, deshalb zählte die 
junge Frau konzentriert ihr Erbetteltes. Sie wartete weitere Stunden, bis ihre Glieder ganz klamm 
vor Kälte waren und die meisten Lichter im Gasthaus erloschen. Um ihre Muskeln aufzuwecken rieb 
Lavay sich die Waden und Oberschenkel, die Arme und Hände. Ihre Haut prickelte. Als Abschied nickte 
sie der Bettlerin nur stumm zu, dann erhob sie sich und warf einen kurzen Blick auf die leere 
Straße, ehe sie den Hintereingang suchte. Lavay kam an einigen verschlossenen Fenstern vorbei, ehe 
sie die Tür fand. Nicht halb so prunkvoll wie das Eingangsportal, aber fest verschlossen. 
Unschlüssig lief sie die Strecke wieder zurück und hielt schließlich an einem der schmalen Fenster 
vor ihren Füßen inne. \textit{Selbst die Kellerfenster sind aus Glas}, dachte sie lächelnd: 
\textit{Und Glas bricht.}\\
Tastent suchte sie nach einem brauchbaren Stein in ihrer Nähe und schlug ohne weiter zu überlegen 
gegen die Scheibe. Mit einem Klirren brach das Glas. Lavay zog ihre Hand zu schnell zurück und 
schnitt sich an einer scharfen Kannte. Hektisch sah sie sich um, aber es blieb ruhig. Den schmalen 
Schnitt an ihrem Handrücken irgnorierend rappelte sie sich auf und trat die Glasreste mit dem 
Stiefel aus der Fassung. Die Öffnung war schmal. Aber Lavay ebenso.\\
Mit den Füßen zuerst kletterte sie durch das eingeschlagene Fenster. Lavay hielt den Atem an und 
versuchte sich so schlank wie möglich zu machen. Den Bauch fest eingezogen und die Arme über den 
Kopf gestreckt schaffte sie es mit windenden Bewegungen durch zu schlüpfen. Glas klitzerte auf 
ihrer Kleidung und der Stoff an ihrer Schulter hatte einen klaffenden Riss. Lavay wischte sich ihre 
blutende Hand am Rock ab und legte den Kopf in den Nacken um tief durchzuatmen.\\
Sie war in einem Weinkeller gelandet. Durch das wenige Licht des Mondes erkannte sie nur grobe 
Umrisse. Tastend strich sie über die Rundungen der Fässer. Die Ungetüme waren größer als sie 
selbst. Irgendwoher erklang ein tropfendes Geräusch und die leisen Stimmen des Gasthausbetriebs. 
Lavay folgte den Stimmen, setzte sorgfältig einen Fuß vor den anderen und presste ihre 
blutende Hand gegen den Rock. Wie in Trance schritt die junge Frau auf die vom Kerzenlicht 
beleuchteten Holzstufen zu. Nicht einmal ein Knarzen erklang, als sie hinauf schritt. Vorsichtig 
steckte sie den Kopf durch die angelehnte Tür und sah sich um. Der Schankraum grenzte an den Flur. 
Er war nur noch spärlich beleuchtet und leises Murmeln erklang. Anders als der Raum den Flur entlang 
zu ihrer rechten. \textit{Die Küche}, vermutete sie und lauschte den Gesprächen des Personals. Lavay 
wartete keinen weiteren Moment, sondern trat über den Flur und huschte die nächste Treppe hinauf. Zu 
den Gästezimmern, wie sie hoffte. Ein weiterer Flur lag vor ihr. Bereits zur Hälfte abgebrannte 
Kerzen zeigten ihr, dass die Nacht schon vorran geschritten war. Die Türen sahen alle gleich aus, 
aber die Abstände dazwischen waren unterschiedlich. Lavay vermutete, dass es wie in ihrer 
Arbeitsstelle bedeutete, dass die ersten Zimmer mit größeren Abstand zur nächsten Tür die 
billigeren Sammelunterkünfte waren. Lavay beschloss zu raten und wählte die Mittlere der drei in 
Frage kommenden Zimmer. Ihre Hand lag schon auf dem Türgriff, da erstarrte sie. Zwei Frauen 
kicherten hinter der dünnen Holztür. Lavay zögerte und beschloss an der nächsten zu lauschen. Sie 
kam nicht dazu, denn tiefe Stimmen und Schritte kamen aus der Richtung der Treppe. Hektisch 
versuchte Lavay die Tür zu öffnen, doch sie war abgeschlossen. Sie wirbelte herum und griff in 
aufsteigender Panik zur Dritten. Vor Überraschung, dass das Schloss sich wirklich öffnen ließ, hätte 
Lavay fast aufgelacht. Sie huschte hindurch, schloss die Tür schnell aber leise und sah sich 
hektisch in dem stockdunklen Raum um. Es war nicht das perfekte Versteckt, eine andere Wahl sah 
Lavay jedoch nicht. Hastig rollte sie sich unter das Bett und klaubte den Stoff ihres Rocks 
zusammen, damit auch ja kein Fetzen hervor ragte. Im selben Moment, als Lavay die Luft anhielt, 
öffnete sich die Tür und eine Öllampe erhellte den Raum.\\
''Ich habe das Zimmer überprüft, Herr``, erklärte einer der Stimmen.\\
Karkos - seine Stimme hatte sich eingebrannt, Lavay konnte sogar sein verschmitztes Lächeln vor 
sich sehen - lachte nur. ''Ihr übertreibt!``\\
Stoffe raschelten. Ein Kettenhemd. ''Solange Euer Vater so gut zahlt, übertreiben wir gerne, 
Herr.``\\
''Wie du meinst. Aber ihr verzieht euch jetzt trotzdem in die Sammelunterkünfte. Ich werde mein 
Zimmer nicht mit Wachen teilen und wagt es bloß nicht, meine Geduld noch weiter auf die Probe zu 
stellen.``\\
Er sagte es bestimmt, aber seine Stimme klang eher amüsiert als genervt. Möglichst leise versuchte 
Lavay auszuatmen. Ihre Lunge brannte bereits und es war ihr, als würde sie den Geruch des frischen 
Blutes wahrnehmen. Die beiden Wachen zogen brummend ab, die Tür fiel ins Schloss. Einen Moment 
geschah gar nichts und Lavay fürchtete, das Atmen hätte sie verraten. Aber Karkos stand mit dem 
Blick zur Tür und rührte sich einen Moment lang nicht. Ehe er doch auf sie zutrat und abschloss.\\
Es erfüllte Lavay mit Genugtuung, dass der Mann Angst hatte. Und niemals würde er darauf kommen, 
dass sie es war, deren Rache er spüren würde.\\
Sie hörte das Rascheln der Kleider, als er sich auszog. Das Klirren der Ringe. Das Bett knarzte 
leise, als er sich nieder legte und die Lampe löschte. Lavay verharrte regungslos und versuchte so 
wenig wie möglich zu atmen. Sie wartete eine gefühlte Ewigkeit und wagte keine Bewegung. Ihre 
Gedanken tobten dafür. Es war ihr egal, dass der junge Mann nett klang. Es war ihr egal, dass sein 
Lachen sympathisch war. Es war ihr egal, dass er sterben würde.\\
\textit{Haska hätte wirklich mich, anstatt Imur wählen sollen.}\\
Lavay holte ein letztes Mal tief Luft, dann rollte sie sich unter dem Bett hervor. Ihre Augen 
hatten sich schon lange an die Dunkelheit gewöhnt. Ihre Schritte erzeugten keinen Ton. Wie ein 
Schatten erhob sie sich und betrachtete den schlafenden Körper. Sie war kalt. Jede vergossene Träne 
über ihre Gegangenen hatte mehr und mehr Wärme aus ihrem Herzen gezogen, bis nichts mehr übrig 
war. Sein Messer lag neben der Öllampe und den Ringen auf seinem Nachttisch. Es war so leicht.\\
Lavay ergriff es, zog die Klinge aus der Scheide und betrachtete die Waffe andächtig. Die junge 
Frau zögerte. Nicht aus Skruppel, sie überlegte nur noch, wo genau sie zustechen sollte. Er könnte 
noch schreien und dann wäre eine Flucht schwer. Bisher hatte sie sich zwar noch keine Gedanken über 
das \textit{danach} gemacht, aber ein Rückzug stand ebenso außer Frage.\\
\textit{Ich muss ihm die Kehle durchschneiden}, überlegte Lavay und tippte mit der Fingerspitze 
gegen die Klinge. Sie hatte schon gesehen, wie Menschen auf diese Weise starben. Es kam nicht 
selten vor, dass die Stadtwache in ihrem Viertel an Ort und Stelle gleichermaßen Riechter wie Henker 
war. \textit{Den Kopf zurück ziehen. Und dann schnell sein.}\\
Ihre Hand schloss sich fest um den Griff. Lavay zwang sich, nicht weiter nachzudenken, überwand die 
letzte Distanz und packte seine Haare. Mit einem Ruck zog sie ihn zu sich und die Klinge glitt über 
die verletztliche Stelle. Lavay fluchte und befürchtete, sie hätte nicht tief genug geschnitten. 
Sie ließ ihn los, hörte sein Röcheln und das Geräusch des Blutes, welches stoßweise aus der 
Verletzung schwabte. Aber sonst blieb es still. Er schrie nicht. Als Lavay nach ihm tastete spürte 
sie die warme Feuchtigkeit, die sich in die Decke sog. Mit der Stille hörte sie auch wieder ihre 
Gedanken. Als wäre Imur es, der ihr sagte, was nun zu tun war. Sie musste raus hier. Aber war mit 
Blut übersäht. Schon ihr eigenes war zu verräterisch. Lavay zog sich hastig aus, stopfte das Kleid 
wahllos in eine leere Schublade und schlüpfte in Karkos Hose und Wams. Auch bei den Stiefeln konnte 
sie sich nicht zurück halten. Das Leder wirkte viel zu bequem. Das mittlerweile eingetrocknete Blut 
an ihren Händen fühlte sich seltsam an, aber Lavay bemühte sich es zu ignorieren. Hektisch wischte 
sie das Messer an der Decke grob ab, packte ihr langes braunes Haar und schnitt es so kurz wie es 
ihr gelang. Auch das Haar versteckte sie in der Schublade. Es war egal, wenn sie es finden würden. 
Sie musste dann schon weit genug weg sein. Alles war egal. Sie würde laufen und laufen, bis es 
irgendwann nicht mehr ging.\\
Das Messer steckte sie sich in den Stiefel und als letztes Griff sie noch wahllos nach zwei seiner 
Ringe, ehe sie sich zum Fenster wandte. Es knarzte, als sie es öffnete und sich hinauslehnte. 
\textit{Erster Stock. Das schaff ich.}\\
Der Morgenhimmel färbte sich bereits violett. Aber Lavay rechnete sich gute Chancen aus. Karkos war 
so spät zu Bett gegangen, dass seine Wachen ihn bestimmt nicht vor dem späten Vormittag erwarten 
würden. Sie zog sich hoch und kniete auf dem schmalen Fenstersims, sah hinüber zur aufgehenden 
Sonne.\\
\textit{Du hättest mich wählen sollen, Haska.}\\


