\chapter{xxx}


"Können…``, stammelte Lavays Mutter: "Können wir rasten?``\\
Lavay seufzte und sah in den Himmel. Die Sonne hatte ihren Zenit längst überschritten und sie 
hatten in den letzten zwei Tagen ein gutes Stück zurückgelegt. Doch trotzdem war der Weg noch weit. 
Lavay wollte heute noch die nächste Stadt erreichen und dort eine billige Unterkunft oder 
zumindest ein windgeschütztes Fleckchen finden. Der Plan war, sich dann gleich am nächsten Morgen 
eine Arbeit zu suchen. Vielleicht könnte sie in einer Gaststätte nachfragen. Kellnern oder die 
Reittiere versogen, das würde sie schon schaffen. Lavay nickte ihrer Mutter zu und ließ sich am 
Straßenrand in den Staub nieder. Sie gab ihr einen Teil des Proviants und nahm sich selbst ein Stück 
Brot und Käse. Verträumt kaute Lavay und malte sich ihr zukünftiges Leben aus. \\
Ein kompletter Neuanfang. Niemand wusste woher sie kam und wer sie war. Sie ließ die Elendsviertel 
hinter sich und streifte die Fesseln ihrer Geburt ab. Lavay würde sich saubere Kleidung suchen, 
vielleicht sogar ein Kleid und sich in den ehrbaren Gasthäusern vorstellen. Ihre Mutter könnte 
vielleicht als Küchenhilfe etwas finden. Kochen konnte sie gut, erinnerte sich 
Lavay an die Zeiten, in denen ihre Mutter noch täglich etwas zubereiten konnte. An die Zeiten, in 
denen genug Essen vorhanden war um anständige Mahlzeiten daraus zu machen. Wenn der Wirt großzügig 
war, könnten sie vielleicht in einer Kammer des Gasthauses leben und dafür auf einen Teil des Lohns 
verzichten.\\
Lavay setzte gerade dazu an, ihre Mutter in diese Pläne einzuweihen, als sie inne hielt. Die ältere 
Frau saß in sich zusammengesunken auf dem feuchten Gras und beugte sich über ihr Stück Brot, 
welches sie zitternd in der Hand hielt. Sie hatte die Augen geschlossen und Schweißtropfen 
bildeten sich auf ihrer Stirn. Die linke Hand presste sie fest gegen ihren Körper. Hastig rutschte 
Lavay an die Seite ihrer Mutter und legte prüfend die Hand auf ihre Stirn. Ihre Mutter glühte vor 
Fieber.\\
"Mama``, entwich es Lavay entsetzt.\\
"Wollen wir weiter?``\\
Lavay schüttelte den Kopf und streckte die Hand aus. "Zeig mir deine Hand!``\\
Ihre Mutter wich ihrem Blick aus und Lavay griff nach der verkrampften Hand. Sanft strich sie den 
langen Ärmel des Kleids zurück. Die Hand war in einen schmutzigen Stoffstreifen gehüllt. Lavay 
kniff ihre Lippen fest aufeinander. Angst erfüllte sie. Und Wut. Den provisorischen Verband löste 
sie mit wenigen, groben Bewegungen und starrte auf die entzündete Wunde. Sie hatte genug gesehen, 
ließ ihre Mutter los und sprang auf die Beine. Unruhig lief sie einige Schritte hin und her.\\
"Warum hast du nichts gesagt?``, rief Lavay und ballte die Hände zu Fäusten. \\
Ihre Mutter hob in einer hilflosen Geste den unverletzten Arm. "Ich wollte nicht… du hast zur Eile 
gedrängt und ich… schrei mich nicht an…``, stammelte sie.\\
"Zeig!``, forderte Lavay barsch und packte den Arm ernaut. Sie schob den Ärmel des Kleides zurück 
um zu überprüfen, wie weit die Blutvergiftung bereits vorangeschritten war. Viel zu weit.\\
Lavay kannte sich nicht mit Heilkunde aus, aber sie hatte solche Verletzungen schon oft gesehen. In 
ihrem Viertel kam es nicht selten vor. Sie vermutete, dass es damit zu tun hatte, dass die Wunden 
nicht sauber genug gehalten werden konnten, wenn man mitten im Dreck lebte. Aber im Vergleich 
hatten sie es mit ihrer Kammer doch gut gehabt und sauberer als so viele andere.
"Das kann einfach nicht wahr sein'', murmelte Lavay, während sie den unruhigen Gang fortsetzte.
Ihre Mutter kauerte zitternd am Boden. Tränen liefen ihr über die Wangen und sie starrte stumm auf 
ihre verletzte Hand. Verzweifelt verabschiedete die junge Frau sich von ihren Plan und überlegte, 
was sie nun mit ihrer Mutter anstellen sollte. Sie brauchte einen Arzt oder noch besser, eine 
Kräuterfrau und Heilkundige. Das wäre um einiges billiger. Außerdem hielt Lavay nichts von den 
fettleibigen Ärzten, denen sie bisher begegnet war.\\ 
Lavay kehrte zu ihrer Mutter zurück und streckte die Hand auf, um ihr aufzuhelfen. 
"Komm. Ich helfe dir. Bis zur nächsten Stadt müssen wir es schaffen und dann können wir jemanden 
suchen, der dir hilft. Bestimmt.``\\
"Ich…``, krächzte ihre Mutter und schüttelte den Kopf: "Ich schaffe das nicht. Es tut so weh… Wie 
Feuer in meinen Adern.``\\
Lavay sah sich ratlos um. Sie waren alleine auf der Straße und nirgends war eine Menschenseele zu 
sehen. \\
"Ein bisschen Schlaf``, murmelte ihre Mutter: "Dann geht es mir sicher besser!``\\
Lavay glaubte nicht daran, aber sie hatte selbst keine andere Idee. Also ließ sie ihre Mutter 
hinlegen, stützte ihren Kopf mit dem Stoffbeutel und deckte sie mit ihrer fadenscheinigen Wolldecke 
zu. Fürsorglich kniete sie sich nieder und gab der Verletzten zu trinken aus dem Wasserbeutel. Nach 
wenigen Augenblicken war ihre Mutter eingeschlafen und Lavay streckte seufzend ihre Glieder aus. 
Sie hatte keine Ahnung, wie schlimm die Verletzung war und wie sie sie versorgen sollte. In ihrer 
jugendlichen Naivität redete sie sich ein, dass ein bischen Schlaf und Erholung ihrer Mutter 
ausreichend helfen müsste, um am nächsten Tag in der Stadt eine Kräuterkundige zu suchen. Die 
Frage, wie sie diese bezahlten sollte, verdrängte Lavay aber. Ebenso wie die leise Stimme der 
Vernunft in ihr, die ihr böse zuflüsterte, dass das Schicksal ihrer Mutter längst besiegelt war und 
sie lieber beginnen sollte, ein Grab auszuheben. Lavay hätte sich gerne auf die Suche nach einem 
Bachlauf gemacht, aber sie fürchtete davor, ihre Mutter allein zu lassen. Momentan war zwar keine 
Seele auf der Straße zu sehen, aber das Gelände war unübersichtlich und schon hinter der letzten 
Bergkuppe könnten Fremde sein. Nein, sie wollte nicht, dass irgendwelche Leute ihre wehrlose, 
kranke Mutter am Wegesrand fanden.\\
Ihrer Mutter entwich ein gequältes Stöhnen. Lavay richtete sich auf. Der Schlaf der Verletzten war 
plötzlich alles andere als ruhig. Sie wand sich im Fieberwahn, als kämpfte sie gegen eine wilde 
Bestie. Sie jammerte und weinte, rief um Hilfe und schlug um sich. Lavay rüttelte verzweifelt an 
ihren knochigen Schultern. Sie riss die Augen auf und sah ihrer Tochter direkt ins Gesicht ohne sie 
zu erkennen.\\
"Mama``, sprach Lavay sie zaghaft an. Sie sahen einander einen Moment in die Augen. Sie hatte 
schöne Augen, wurde Lavay plötzlich bewusst, als hätte sie noch nie vorher genau hingesehen. Augen 
in einem sanften Braun. Augen, die niemandem etwas zu leide tun konnten. Augen, die solch grausames 
Schicksal wie das, welches ihre Mutter erleiden musste, nicht verdient hatten! Der Tod des Gatten, 
Leben in der Armut, ein Sohn der sich zu den schlimmsten Verbrechen gezwungen sah und dafür 
gerichtet wurde. Und nun dieser weiterer Schlag des Schicksals. "Mama!``, wiederholte Lavay und 
Tränen liefen ihr über die Wangen.\\
Ihre Mutter sank zurück auf den Boden und blickte starr in den Himmel. Ihre Lippen bewegten sich, 
aber Lavay verstand die gebrabbelten Worte nicht.\\
"Hier``, sagte sie: "Trink etwas!``\\
Ihre Mutter reagierte nicht auf ihre Worte und beachtete auch den dargebotenen Wasserbeutel nicht. 
Lavay ließ die Hände sinken und beobachtete hilflos, wie ihre Mutter vom Fieber gepeinigt wurde. 
Lavay zwang sich, ruhig zu bleiben und zu überlegen. Das Wasser ging ihnen aus, ebenso der karge 
Proviant. Und die Aussicht, die Nacht hier am Wegesrand zu verbringen, war alles andere als 
verlockend. Sie wären wilden Tieren, gesetzlosen Räubern und den Soldaten des Königs schutzlos 
ausgeliefert. \\

Lavay schreckte aus ihren Überlegungen auf, als sie das Rattern von Wagenrädern und die trottenden 
Schritte eines Pferdes vernahm. Einen Moment sah sie in die Richtung, aus der die Geräusche kamen, 
dann rappelte sie sich auf und lief an der Straße entlang. Sie riss die Arme hoch und winkte, um 
Aufmerksamkeit zu erregen. Die Sorge, dass zwielichtige Gestalten über die Bergkuppe kamen, hatte 
sie mittlerweile verlassen. Sie brauchten Hilfe und es war ihr nun egal von wem diese kam. Die 
Sorge wäre auch unbegründet gewesen. Auf einem klapprigen, zweirädrigen Wagen hockte ein 
graubärtiger Mann. Unter den ausgefransten Strohhut blickten ihr zwei freundliche Augen fragend 
entgegen. Er schnalzte mit der Zunge und riss an den Zügeln. Der graue Wallach mit dürren Beinen 
und zerzauster Mähne stoppte abrupt vor Lavay und blies ihr den warmen Atem ins Gesicht. Vor 
Erleichterung, dass der Wagen tatsächlich gehalten hatte, schlag sie ihre Arme um den Pferdehals 
und drückte einen Kuss in das stupfte Fell. Der Wallach blickte sie etwas verwundert an und 
schnaufte.\\
"Na, Mädel?``, fragte der Wagenlenker und zeigte etliche Zahnlücken, als er sie freundlich 
angrinste: "Brauchst du Hilfe? Oder biste der hübsche Lockvogel einer Räuberbande, die mir alles 
nehmen will?`` Lachend breitete er die Arme aus. "Ich hab aber nix. Darfst gerne nachgucken, Mädel. 
Nur Hut und Pferd!``\\
Sein Lachen klang herzlich. Lavay schüttelte den Kopf und deutete den Weg entlang. "Meine Mutter 
liegt da vorne. Sie ist verletzt und hat starkes Fieber…``\\
Lavay verstummte und schluchzte leise aus Angst vor einer Ablehnung. Der Blick des Mannes folgte 
ihrer ausgestreckten Hand und nickte sofort. Wortlos lenkte er den Wagen zu der Verletzten, stieg ab 
und bückte sich, um Lavays Mutter auf die leere Ladefläche seines Wagens zu hiefen.\\

Es war bereits zu spät, als sie die nächste größere Stadt erreichten. Lavay saß reglos neben den 
erkaltenden Körper ihrer Mutter. Wie gerne hätte sie ihre Arme um sie gelegt, das lange Haar 
gestreichelt und sie gewiegt wie ein Kleinkind. Aber sie tat es nicht. Ihr Kopf war leer. Sie 
schien die Kontrolle über ihre Glieder verloren zu haben. Lavay starrte sie einfach an.
Irgendwann, die Sonne machte bereits Anstalten, sich vor der Dunkelheit zurückzuziehen, legte 
jemand ihr eine Hand auf die Schulter. Lavay sah blinzelnd auf und erkannte den graubärtigen Mann, 
der sie und ihre Mutter mitgenommen hatte. Er wollte bestimmt seinen Wagen nehmen und nachhause 
fahren. Lavay räusperte sich und wollte sich bedanken, dass er angehalten hatte. Dass er es 
versucht hatte. Aber kein Wort kam über ihre Lippen. Stattdessen traten ihr die Tränen in die 
Augen.\\
"Ich habe mich erkundigt. Sie kann hier begraben werden. Das nötige Geld habe ich dir ausgelegt. 
Zahl es mir zurück, sobald du kannst. Ich komme jede Woche hier her. Komm, ich fahr den Wagen zu 
der Stelle.``\\
Lavay verbarg ihr Gesicht in ihren Händen und zitterte. Sie musste daran denken, dass Imur kein 
Grab bekam. Er hing vermutlich immer noch dort am Galgen, mit der Schlinge um den Hals. Kalt und 
leblos. Seine Haut vom Tod verfärbt. Krähen und Fliegen würden sich über ihn her machen. Kein Grab 
für Imur. Kein Segen. Kein Gebet und keine Blumen. Lavay schluchzte leise. Wen hatte sie denn jetzt 
noch? Ihr Bruder hingerichtet, als er versuchte, sie aus dem Armenviertel zu holen. Ihre Mutter 
grausig vom Fieber verbrannt worden, weil sie ihrer Tochter ihre Schwäche nicht zumuten wollte.\\
"Alles meine Schuld``, heulte Lavay.\\
Der Mann umarmte sie zögernd und fragte überfordert: "Hast du Verwandte, zu denen du gehen 
kannst?``\\
Lavay schüttelte stumm den Kopf. \\
"Komm, Mädchen. Wir beerdigen deine Mutter und dann suchen wir uns ein Gasthaus. Morgen muss ich 
gehen und du suchst dir eine Arbeit. Es wird alles gut. Du bist alt genug, um auf eigenen Beinen zu 
stehen.``\\
Er sprach sehr sanft, aber bestimmt und zwang Lavay schließlich, ihn anzublicken. "Du bist kein 
kleines Kind, sondern eine hübsche junge Frau. Und ich denke, du wirst gut zurechtkommen. Nun 
kümmern wir uns um deine Mutter und dann kannst du anfangen, Geld zu verdienen um deine Schulden 
bei mir zu begleichen, ja?``\\
Lavay nickte und wischte sich die Tränen ab. Er hatte recht. Tränen brachten nun nichts. Trotzdem 
war sie sich sicher, dass sie heute nur weinend einschlafen würde.\\


``Verzeihung”, murmelte Lavay und wurde rot, als Tropfen der cremigen Kartoffelsuppe auf dem 
Wams des Kunden landeten. Der Mann blickte zweifelnd zu ihr hinauf und runzelte die Stirn.
``Junges Fräulein, seid Ihr nicht in der Lage, einen  Suppenteller zu tragen?”\\
Er klang sehr amüsiert und tauschte einen Blick mit seinen Kameraden.\\
``Doch, ich meine... es tut mir sehr Leid... Ein Versehen...”\\
``Na, ich hoffe doch, dass es keine Absicht war. Wenn, hättet Ihr mir besser gleich den gesamten 
Teller über den Kopf kippen sollen. Nun bring mir noch einen Krug Bier, Mädchen.”\\
``Gewiss... vielen Dank”, stammelte sie und machte sich eilig auf den Weg zurück zum Tresen. 
Zumindest nahm der Mann es mit Humor. Die Wirtin aber nicht. Sie stand am Tresen und funkelte Lavay 
aus kleinen Augen böse an. Sie war eine großgewachsene, hagere Frau. Die Nase war zu groß für das 
Gesicht und ihre Stirn schien stets skeptisch gerunzelt.\\
``Was sollte das schon wieder?”, schnappte sie Lavay an, kaum dass sie in Hörweite war.\\
``Ein Versehen...”, wiederholte Lavay beschämt.\\
``Das wievielte heute?”\\
Lavay fürchtete, dass die Wirtin ihr gleich wieder eine Ohrfeige geben würde. Aber dafür sahen zu 
viele Gäste zu. Das würde sie sich vermutlich für den nächsten Morgen aufheben. Oder aber, sie 
würde ihr einfach noch mehr Geld vom Lohn abziehen. Er war eh schon wenig genug. Von den wenigen 
Münzen, die sie in der Woche bekam, musste sie fast die Hälfte abgeben, um die kleine Kammer und 
das karge Mahl zu bezahlen. Außerdem hatte der Wirt verlangt, dass sie ein Kleid trug, wenn sie die 
Gäste bediente. Damit hatte Lavay bereits gerechnet. Fies war es jedoch gewesen, dass das Ehepaar 
ihr ein gebrauchtes Kleid und eine Haube für das Haar gab und den Preis dafür als Schulden 
anrechnete. Lavay würde schon zwei Wochen arbeiten müssen, um das Kleid, die Haube und das paar 
ausgetretene Schuhe zu bezahlen. Es würde nichts bleiben, um ihre Schulden für die Beerdigung ihrer 
Mutter zu zahlen.\\
Lavay nahm den Krug und brachte ihn an den Tisch der drei Herren. Während sie zu einer Gruppe 
Neuankömmlinge eilte, um deren Bestellung aufzunehmen, spürte sie deutlich die Blicke der Männer 
auf sich ruhen. Ihr lief ein Schauer über den Rücken. Das war das widerlichste an dieser Arbeit. 
Die Männer an sich, ihre Blicke und spottenden Worte speziell. Manch einer, der schon etliche Krüge 
geleert hatte, blieb auch nicht bei Blicken und Worten, sondern griff zu. Lavay hatte dem ersten 
einen ordentlichen Stoß gegeben, sodass er von der Bank plumpste und dort benommen liegen blieb. 
Noch dazu hatte sie ihm eine ordentliche Welle von Schimpfwörtern und Flüche entgegen geworfen. 
Woraufhin die Wirtin ihr deutlich klar gemacht hatte, dass sie so ein Verhalten von Angestellten 
nicht dulden konnte. Bei weiteren ähnlichen Vorfällen reagierte Lavay also wie ihre zwei 
Kolleginnen. Sie schwieg und ging weiter. Es viel ihr schwer. Aber sie brauchte Geld. Schon die 
letzten drei Wochen hatte sie den freundlichen Bauern, der ihr damals geholfen hatte, vertrösten 
müssen. Aber morgen, wenn sie ihren Lohn für die vergangene Woche bekommen würde, dann könnte sie 
dem Bauern endlich eine Anzahlung geben.\\
Es wurde später und die Gäste, vorwiegend Männer, betrunkener. Lavay sehnte sich danach, dass auch 
bald der letzte Gast gehen würde. Heute war nämlich sie diejenige, die von den Mädchen bis zum 
Schluss bleiben müsste. Die Wirtin würde sich bald verabschieden und der Wirt, der meistens in der 
Küche blieb war schon längst gegangen. Es wurde auch nichts anderes als Bier, Wein und Met 
ausgeschenkt. Trotz der späten Stunde, waren noch viele Gäste da. Lavay befürchtete, dass es weit 
bis in die Nacht dauern würde, jeden loszuwerden. Und trotzdem musste sie am nächsten Morgen wieder 
im Gasthaus stehen, den Boden fegen und die Tische schrubben.\\
Lavay registrierte aus den Augenwinkeln, wie ein deutlich betrunkener Mann sie schmierig angrinste 
und eine Hand ausstreckte. Sie versuchte, ihm auszuweichen, doch da griff er blitzschnell nach dem 
Rock ihres Kleides und Lavay verlor das Gleichgewicht. Die Teller fielen scheppernd zu Boden und 
zerplatzten zu aberhunderten Scherben. Ehe Lavay sich versah, baute sich schon die hagere Wirtin 
vor ihr auf und schimpfte lautstark: "Jetzt reicht es mir mit dir! Schon wieder eines deiner 
Versehen? Das ist doch nicht zu fassen. Du lernst es wohl nie!``\\
"Aber... ich kann doch gar nichts... das...`` Lavay traten Tränen in die Augen. Sie wusste schon, 
was jetzt kommen würde.\\
"Nein, keine Ausreden und faules Stück! Kehr die Scherben auf und dann verzieh dich in deine 
Kammer. Eine der Anderen wird deinen Dienst heute übernehmen. Sonst legst du vermutlich noch die 
ganze Einrichtung in Schutt und Asche! Deinen Lohn morgen kannst du streichen. Der geht dafür 
drauf, die ganzen Teller zu bezahlen und die Gäste zu entschädigen!``\\
Lavay verkniff sich das Weinen und bückte sich, um die Scherben aufzusammeln. Dabei schnitt sie 
sich an einer scharfen Kante und ihr entwich ein leises Schluchzen.\\

"Sieh es so, eigentlich hast du es ganz gut getroffen. Das Gasthaus zum steinernen Tor ist nichts 
im Vergleich zu meiner früheren Arbeitsstelle... da gab es jeden Tag Schlägereien und nur die 
widerlichsten Gäste aus den Armenviertel kamen. Sie betranken sich jeden Tag mit dem billigen 
Gesöff, dass der Wirt dort anbot und brachten ihre Flöhe und weitere Ungeziefer mit! Es war 
schrecklich, sage ich dir. Ich meine, die Gäste vom steinernen Tor sind ja schon praktisch reich. 
Und wenn man freundlich ist, geben sie gutes Trinkgeld...``\\
"Freundlich``, wiederholte Lavay bitte und vergrub ihr Gesicht in dem dünnen Kissen. Sie hoffte 
ihre Kollegin würde endlich den Mund halten. Sie konnte es nicht mehr hören. Freundlich, war sie, 
ohja. Sie ließ sich begrapschten und lächelte dabei.\\
"Gute Nacht!``, knurrte Lavay und hoffte, dass würde sie endlich zum Schweigen bringen.\\
"Pfff``, gab sie von sich und fügte hinzu: "Wenn du so weiter machst, brauchst du dich nicht 
wundern, wenn du bald wieder auf der Straße sitzt.``\\

Die Idee kam ihr spät nachts. So spät, dass es an sich fast schon wieder Morgen war. Lavay 
schätzte, dass es nur noch eine knappe Stunde bis zum Morgengrauen sein würde. Dann müsste sie sich 
schon wieder an die Arbeit machen. Für die heutige Nacht war also nicht mehr genügend Zeit. Und wer 
weiß, vielleicht könnte sie auch während des Tages die ein oder andere freie Minute finden.\\
Lavay hasste betteln. Und sie hasste es, Schulden zu haben.\\
Der nette Bauer hatte ihr so geholfen. Allein durch sein Geld hatte sie ihre Mutter beerdigen 
können. Wenn auch ein karges Grab, geteilt mit einigen anderen und ohne jegliche Zeremonie. Nicht 
einmal die Götter wurden angerufen, ihr gnädig zu sein und sie sicher zu sich zu geleiten.\\
Die Wirtin würde ihr das Leben schwer machen, dass war Lavay klar. Und sie würde, wenn das so 
weiter ging, ewig brauchen, ihre Schulden zu begleichen. Sobald ein anderes Mädchen kam, würde die 
Wirtin sie bestimmt rauswerfen. Lavay rechnete zumindest damit. Schließlich war sie zwar zu den 
anderen beiden auch streng, aber nicht so ungerecht und fies.\\
"An sich lässt sie mir gar keine Wahl!``, entschied Lavay und überlegte, welche Schätze die Kunden 
des Gasthauses wohl bei sich trugen. Es war eine Weile her, seit Lavay etwas geklaut hatte. Sie 
glaube zwar nicht, dass man so etwas verlernen konnte, aber sie wollte vorsichtig sein.\\ 
Marga und Terim, ihre Kolleginnen, regten sich grummelnd. Marga ob den Kopf und gähnte: "Einer muss 
in den Stall.``\\
Der Stalldienst war der verhassteste bei den Mädchen. Man musste am frühsten aufstehen, weil 
einige Reisende bei Sonnenaufgang los wollten. Und natürlich das ausmisten. Lavay hatte diesen 
Dienst bisher noch nicht gemacht. Sie hatte nicht wirklich Erfahrung mit Tieren. Es lag aber auch 
daran, dass die Wirtin ihr meist den Nachtdienst auftrug und sie somit vom Stall am nächsten Morgen 
verschont wurde. An sich waren es manchmal gar vier Stunden, die man später ins Bett kam, aber nur 
wenige Minuten länger schlafen konnte. Aber da sie nun eh schon wach war, erbot sie sich und trat 
aus dem Zimmer.\\

\textit{Plane, wie du beobachtest.}\\
An sich, wenn sie nicht gerade Suppenteller trug, hatte sie im Gastraum stets die Gelegenheit, zu 
beobachten. Aber nicht wirklich zu handeln. Das wäre am leichtesten, wenn sie die Zimmer putzte. 
Was aber an sich nur recht selten geschah oder auf ausdrücklichen Wunsch eines Gastes. Da blieb die 
Möglichkeit während dem Nachtdienst. Aber die Betrunkenen, die so spät noch da waren, konnten 
meistens kaum genügend Geld zusammenkratzen, um zu zahlen.\\
Somit begnügte sie sich anfangs damit, den letzten Gästen einen höheren Preis zu nennen und das 
Geld für sich einzustecken. Die waren dann nur selten in der Lage, überhaupt zu zählen, wie viele 
Krüge sie geleert hatten und Lavay vermutete, dass sie nicht die einzige war, die so handelte. 
Reich wurde sie damit trotzdem nicht. Es reichte nicht einmal, ihre Schulden zu begleichen.\\
So kam der Tag, an dem sie ihren freien Nachmittag in der Woche hatte, und wieder stand sie mit 
fast leeren Taschen da. Sie fürchtete sich, ihren Helfer wieder vertrösten zu müssen. Er hatte ihr 
so viel Gutes getan und ihre ständigen Ausreden mussten auf ihn einen sehr undankbaren Eindruck 
machen. Mit hängendem Kopf putzte sie über einen der Tische. In wenigen Minuten hätte sie frei und 
müsste sich auf den Weg zum Marktplatz machen. Sie überlegte sich schon die Wortwahl ihrer 
ausschweifenden Entschuldigung.\\
"Mädchen``, wurde sie angesprochen.
Lavay sah auf und wandte dann sofort überrascht den Blick wieder ab. Betreten musterte sie die 
Schuhe des Sprechers. Es war lange her, dass sie einen Adeligen begegnet war. Wenn, dann nur aus 
der Ferne. In ihr Viertel hatten sie sich selten verirrt. Oder gar in das Gasthaus zum steinernen 
Thor. Dort kehrten eher Kaufleute auf Reisen ein oder wohlhabendere Bauern, die ihre Waren in der 
Stadt verkauften. Und die üblichen Stammgäste des Viertels.\\
Sie kam nicht dazu, den Mann zu begrüßen. Die Wirtin kam aus der Küche geschossen und rief laut 
durch den gesamten Schankraum: "Willkommen, mein Herr. Was darf ich Euch und Euren Männern 
anbieten?``\\
Die Männer traten an die Schänke und Lavay hatte die Gelegenheit, sie zu mustern. Der Adelige an 
sich war recht klein, kleiner gar als Lavay. Sie schätzte ihn auf Anfang Zwanzig, kurzes braunes 
Haar, glattrasiertes Gesicht. Ohrringe, Armbänder und eine glänzende Brosche. Seine Kleidung war 
bunt und wirkte fremdländisch. \textit{Ein Schnösel}, entschied Lavay.\\Das sah sie auch seiner 
Mimik an, als er mit der Wirtin sprach. Verstehen konnte sie nichts, darum trat sie unauffällig 
näher und beobachtete die Begleiter des Jungen. Groß, robuste Kleidung und bewaffnet.\\
"Eines der Mädchen wird Euch ein Zimmer richten, verehrter Herr!``, säuselte die Wirtin.\\
Lavay entdeckte Marga und Terim, die tuschelnd beieinander standen und dem Adeligen schmachtende 
Blicke zuwarfen. \\
"Nein``, entgegnete er: "Das ist unnötig. Es regnet in Strömen und ich werde lediglich 
solange ihr bleiben, bis es aufhört. Ich konnte auf die Schnelle nichts Besseres finden.``\\
Die hagere Wirtin wurde rot. Schwer zu sagen ob vor Zorn oder Scham. Dann kicherte sie nervös wie 
ein junges Mädchen. "Dann darf ich Euch Trunk anbieten?``\\
Der Adelige runzelte die Stirn. "Ist er denn genießbar?``\\
Lavay konnte nicht mehr an sich halten und rief halblaut: "Gestorben ist noch keiner. Zumindest hat 
sich keiner von denen beschwert.``\\
Der junge Mann sah mit einem Lächeln auf den Lippen zu ihr. "Dann bring uns etwas.``\\
Sie ließen sich an einem Tisch nieder und Lavay nahm drei Krüge von der Wirtin entgegen. Sie sah 
sie so böse an, dass Lavay damit rechnete, die hagere Frau würde sie gleich anfauchen und spitze, 
giftige Zähne entblößen.\\
Ohne auch nur einen Tropfen zu verschütten, Lavay hatte mittlerweile etwas Übung, trug sie das 
Tablett zu den Männern. Sie stellte den ersten Krug zu dem Leibwächter. Das missfiel dem jungen 
Adelige, denn er räusperte sich und fragte scharf: "Bedient man nicht erst den Herr?``\\
Lavay lächelte möglichst charmant. "Nun, ich ginge davon aus, dass wenn Eure Männer sich beleidig 
fühlen, weil sie ihren Krug zuletzt bekommen, sie mir mit ihren Schwertern mehr Schaden zufügen 
können, als Ihr.``\\
Die Leibwächter grinsten breit, während ihr Herr eher überrascht wirkte. Er deutete auf die zwei 
und sagte: "Nun… ich bezahlte die Schwerter und ich bezahle die Männer. Also bin ich der Besitzer 
dieser Schwerter.``\\
"Aber könnt Ihr sie auch benutzen?``\\
Lavay trat vorsichtshalber einen Schritt zurück und setzte zur Entschuldigung an. Jetzt war sie zu 
weit gegangen, fürchtete sie. Aber der junge Mann lachte und deutete auf die Bank neben sich. "Setz 
dich!``\\
Das war keine Bitte. Steif ließ sie sich auf die Bank nieder.\\
"Wie heißt du?``\\
"Lavay.``\\
"Mein Name ist Aras Irimk. Was macht so ein hübsches Mädchen in so einem heruntergekommenen 
Schuppen?``\\
"Hier kehren Männer ein, die meistens zahlen können. So heruntergekommen ist er nicht``, murmelte 
sie: "Es gibt viel schlimmere…``\\
"Was aber nicht meine Frage beantwortet``, sagte er und lächelte.\\
"Ich brauche Geld``, antwortete sie schlicht.\\
"Hm…``, kam es von Aras und er nippte an seinem Trunk. Er verzog das Gesicht. "Zumindest weckt es 
keinen Brechreiz… Onar, schau nach, ob es noch regnet!``\\
Einer seiner Begleiter erhob sich und schlürfte zur Tür.\\
"Ich bin nur zu Besuch in der Stadt``, erklärte Aras: "Ich muss meiner zukünftigen Verlobten 
Aufwartung machen und brauche ein Geschenk. Jedoch taugen weder ich noch meine beiden Begleiter 
wirklich dazu, etwas zu finden, was einem hübschen, jungen Mädchen gefällt.``\\
Lavay betrachtete seine Ohrringe, die Armbänder und seine Kleider. Ihm zu glauben viel ihr 
schwer. "Mit Schmuck liegt man wohl nie verkehrt``, sagte Lavay.\\
"Aber sie ist ebenfalls von Hoher Geburt. Ich denke, sie wird genügend Schmuck haben.``\\
"Dann ein Tier vielleicht.``\\
"Onar schlug einen Falken vor, aber ich glaube nicht, dass dies etwas für eine junge Dame ist.``\\
Lavay fand die Raubvögel faszinierend, aber sie war ja auch keine Dame. ''Ich habe auf dem Markt 
letztens einen Stand gesehen. Der Händler bot saleicanische Falken an. Wie die Rasse genau heißt, 
weiß ich nicht mehr, aber sie waren sehr hübsch und er sagte, sie hören aufs Wort. Eine sehr kluge 
und gelehrige Rasse. Wie ein Hund mit Flügeln.``\\
Aras schüttelte nachdenklich den Kopf.\\
"Dann vielleicht ein Pferd?``, schlug sie vor: "Hübsch und elegant.``
Aras stützen seinen Kopf mit der Hand. "Ich weiß nicht… das sind schreckhafte Tiere und sie 
stinken.``\\
\textit{Und teuer}, dachte Lavay.\\
"Dann einen Welpen?``, sagte sie nach einigen Sekunden überlegen: "Mit weichem Fell und treuen 
Augen. Bindet eine Schleife darum, gebt ihm den Namen Herzchen und die edle Dame wird Euch auf ewig 
verfallen.``\\
Es war mehr ein Scherz gewesen, aber Aras riss die Augen auf und klatschte in die Hände. "Gute 
Idee! Weiß du, wo ich so ein Vieh herbekomme?``\\
"Ich lebe noch nicht sehr lange hier``, entschuldigte sie sich.\\
"Hm…``, kam es enttäuscht von ihm: "Vielleicht doch dann Schmuck.``\\
Lavay runzelte die Stirn. Zu faul ein schönes Geschenk zu suchen, war der edle Aras also auch. 
"Begleite mich auf den Markt und such etwas für sie aus, ja? Die Wirtin wird dir bestimmt für ein 
paar Stunden frei geben, wenn ich sie darum bitte.``\\
Lavay nickte nur lahm. Sie blieb brav sitzen, während die Männer ihre Krüge leerten und 
plauderten.\\

Lavay war natürlich klar, dass er mehr von ihr wollte, als die Hilfe bei der Auswahl eines 
Geschenks. Es kam ihr ganz recht. Denn sie wollte seinen Geldbeutel und vielleicht das ein oder 
andere Schmuckstück. Mit den kleinen Diebstählen im Gasthaus würde sie nicht weit kommen. Die beiden 
Leibwächter Onar und Mascha blieben stets drei Schritte hinter ihnen. Die Riesen waren Brüder, wie 
Lavay beiläufig erfahren hatte. \\
Jetzt, da der Himmel aufklarte, füllten sich die Straßen und der Markt. Leute priesen lautstark 
ihre Waren an. Von bunten Stoffen bis zu Lebensmitteln. Sie kamen an einem Stand vorbei, an dem 
Dutzende Tauben in kleine Käfige gesperrt waren und zum Kauf angeboten wurden. Ein erbärmlicher 
Gestank ging von den Tieren aus und sie machten beinahe einen noch ungepflegteren Eindruck als ihr 
Verkäufer.\\
Lavay entdeckte sogar den Karren ihres Gönners, dem sie so viel zu verdanken hatte. Er war wie 
immer an ihrem Freien Tag auf dem Markt und würde am Nachmittag warten, dass sie zu ihm kam.
"Wie wäre es mit einem Vögelchen?``, fragte Lavay und deutete auf die Tauben.\\
Aras machte eine wegwerfende Handbewegung. Höchstens auf den Teller. Aber so wie die stinken würde 
ich das auch lieber vermeiden.``\\
Der Verkäufer warf ihm einen vernichtenden Blick zu, was Lavay zum kichern brachte. Ganz Nebenbei 
hackte Aras sich bei ihr ein und sie schlenderten über den Mark. Immer wieder wandte Lavay sich wie 
zufällig um, nur um festzustellen, dass Onar und sein Bruder nicht von ihrer Seite wichen.\\
"Wie ist Eure Verlobte?``, erkundigte sich Lavay.\\
"Oh, schön wie ein Frühlingsmorgen. Und ebenso jung``, sagte er: "Lieblich anzusehen wie eine edle 
Vase, aber dumm wie Stroh.``\\
"Soso.``\\
Er blieb stehen und sah ihr in die Augen. "Weißt du… mich interessieren viel eher Frauen die auch 
etwas in ihrem hübschen Köpfchen haben. Die nicht nur brav nicken und lächeln.``\\
Er grinste sie an.\\
"Verstehe``, entgegnete Lavay: "Ihr liebt die Gefahr.``\\
"Gefahr?``, lachte er: "Wie könntest du eine Gefahr für mich darstellen?``\\
Lavay erwiderte sein Lachen nur. Aras sah zu seinen Leibwächtern und warf ihnen einen Beutel zu. 
"Schaut, ob ihr etwas Hübsches auf dem Markt findet.``\\
Grinsend zogen die Brüder ab. Aras stolzierten weiter und führte Lavay - er versuchte es wohl 
möglichst unauffällig - in eine Gasse.\\
"Du bist ein sehr schönes Mädchen``, sagte er und strich ihr über die Wange.\\
"Das sagtet Ihr bereits``, erwiderte sie und schob seine Hand fort.\\
"Na na, was schaust du plötzlich so zornig?``\\
"Vielleicht solltet Ihr besser dem Beispiel Eurer Männer folgen und zu einer Hure gehen. Ich werde 
Euch nicht das geben, was Ihr Euch gerade ersehnt.``\\
Sie spürte seine Hand schwer auf ihrer Schulter. Ruckartig stieß er sie gegen die Mauer und blickte 
sie böse an. "Vorsicht… ich mag es, wenn sie frech sind. Aber übertreibe es nicht.``\\
"Wie war das noch? Wie kann ein Mädchen wie ich Euch gefährlich werden?``, fragte sie und sah ihn 
aus unschuldigen blauen Augen an.\\
Er beugte sich herab. Vielleicht um sie zu küssen, aber wer weiß. Er hielt inne, als der das kalte 
Metall an seiner Kehle fühlte. Er schnappte nach Luft und glotzte sie an wie ein Fisch. "Euer 
Geldbeutel bitte, gnädigster Herr.``\\
"Woher…``, krächzte er.\\
"Euer Geld``, wiederholte Lavay geduldig.\\
Langsam kam er in Bewegung und fummelte mit der Hand an seinem Gürtel herum. Lavay behielt ihm im 
Auge, aber sie glaubte nicht, dass er eine Waffe haben würde. Dafür hatte er ja, wie er im 
Wirtshaus deutlich gemacht hatte, seine Männer. Ein Säckchen fiel zu Boden.\\
"Bestimmt nicht alles``, sagte sie: "Aber ich bin bescheiden. Außer… den Armreif will ich noch.``\\
Den streifte sie von seiner zitternden Hand. Schnell duckte Lavay sich fort, ergriff das Säckchen 
und ließ den zitternden Adeligen in der Gasse stehen. Als sie zurück auf die belebte Straßen kam, 
hörte sie seine Rufe. Sie taucht ein in die Menschenmenge, aber noch ehe sie um die nächste Ecke 
bog, warf sie einen Blick hinter sich und entdeckte Aras, wie er taumelnd auf die Straße lief und 
laut brüllte. Die Stadtwachen wurde auf ihm aufmerksam. Lavay wurde blass vor Schreck und 
versuchte, möglichst unauffällig zu sein. Sie tat, als würde sie die Stände beiläufig betrachten, 
blieb aber nirgends stehen.\\
"Ah…. Du``, rief plötzlich eine bekannte Stimme.\\
Lavay war bei dem Karren des freundlichen Bauern angelangt. "Na?``, fragte er verschmitzt: "Willst 
du mich wieder vertrösten?``\\
Lavay schüttelte den Kopf und reichte ihm den Beutel. "Nimm dir das Geld heraus und lass den Beutel 
am besten verschwinden``, riet sie ihm: "Lebwohl!``\\
An seinem Blick erkannte sie, dass er verstanden hatte. Er setzte wohl zu einem Wiederspruch an, 
doch da hatte sie sich schon abgewandt. Zum Glück des Bauern hatte sie bereits zwei Kreuzungen 
hinter sich gelassen, ehe die Rufe erklangen.\\
"Da ist sie!``\\
Es war Onar und sein Bruder, der sich einen Weg durch die Menge bahnte. Gefolgt von drei weiteren 
Männern. Sie trugen nicht die Uniform der Stadtwache. Lavay schluckte. Also hatte Aras mehr als nur 
zwei Männer in seinem Dienst. Ohne weiteres Zögern stieß sie den Korb einer Gemüsehändlerin um und 
Rüben kullerten auf die Straße. Die Menschen blieben stehen und wichen zurück. Lavay hoffte, dass 
sie ihre Verfolger so etwas aufhalten würden. Schnell rannte sie los, stieß Menschen in ihrem Weg 
zur Seite und versuchte so viel Abstand wie nur möglich zu gewinnen.\\
\textit{Wenn Haska mich jetzt sehen könnte...}\\