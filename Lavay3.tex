\chapter{Die Klauen des Löwen}

Der rote Himmel des Sonnenuntergangs verblasste langsam. Mit der zunehmenden Dunkelheit erwachten 
die Tiere des Waldes. Der kühle Wind trug deren Stimmen zu Lavay. Sie schauderte, als das schaurige 
Heulen eines Wolfes erklang. Lavay legte den Kopf in den Nacken und erhaschte einen flüchtigen 
Blick, auf den blassen Mond, ehe dieser von dunklen Wolken verschluckt wurde. Sie schloss die Augen 
und versuchte ihren rasenden Herzschlag zu beruhigen.\\
Sie war ein Stadtkind, hatte in den ersten Jahren ihres Lebens nicht anderes als ihre ärmliches 
Viertel gekannt. Was war es schon für ein großer Schritt gewesen, in die besser gestellten 
Siedlungen der Stadt zu treten? Und nun hatte sie selbst die Straßen verlassen und für sie somit 
auch jede Möglichkeit sich zu orientieren. Anfangs war ihr die Idee besser vorgekommen, im Wald 
Schutz zu suchen. Ihre Finger glitten nach Halt suchend über die raue Rinde des Baumes. 
\textit{Werde ich je wieder auf eine Straße stoßen und oder bis an mein Lebensende hier herum 
irren?}\\
Ohne das Licht des Mondes war es stockdunkel. Lavay dachte kurz darüber nach, einfach hier zu 
bleiben, anstatt sich wie blind durch den Wald zu tasten. Aber da erklang wieder das Heulen des 
Wolfes. Schon im die Mittagszeit herum hatte sie diesen Laut gehört. Mal näher, mal weiter in der 
Ferne. Stets hatte sie sich in eine andere Richtung gewandt, nur fort von dem Ruf des Jägers. Ein 
weiterer Grund, wieso sie nun keine Ahnung mehr hatte, wo sie sich befand.\\ 
Lavay tastete nach dem schmalen Messer an ihrem Gürtel. Mit der kleinen Klinge könnte sie es niemals 
gegen die Jäger des Waldes aufnehmen, aber trotzdem beruhigte die junge Frau sich etwas. Während 
sie einen Fuß vor dem anderen setzte, lauschte sie den Geräuschen in ihrer Umgebung. Die Tiere 
waren nicht das Einzige, was sie fürchten musste. Nach einige Minuten, in denen sie sich fühlte wie 
ein junges Kaninchen, dass jederzeit damit rechnete gleich vom Bussard gepackt zu werden, sah sie 
tatsächlich wie die Finsternis des Waldes sich auftat. Lavay grinste und freute sich schon darauf, 
sich an einem trockenen Plätzchen zum Schlafen niederzulegen, als jemand ihr eine Hand über den Mund 
legte. Sie stank nach Dreck und Schweiß. Vor Schreck biss sich Lavay die Zunge blutig und der 
metallische Geschmack breitete sich in ihrem Mundraum aus.\\
„Na, was suchst du nachts im dunklen, dunklen Wald?", spottete der Mann.\\
Sie erkannte die Stimme gleich, war sie doch eine derer, vor denen sie seit Tagen floh. Onar, die 
Wache des eingebildeten Adeligen. Das war das Problem damit, wenn man dem Adel bestahl. Er war 
hartnäckiger. Ein einfacher Mann hätte sich niemals die Mühe gemacht eine kleine Diebin durch das 
Land zu jagen. Ein einfacher Mann hätte auch nicht die finanziellen Mittel dafür besessen.\\
Das Selbstbewusstsein, welches sie sich hartnäckig eingeredet hatte, zersprang zu abertausenden 
Scherben. Sie konnte es förmlich in ihren Ohren klingeln hören, wie die Splitter übereinander zu 
Boden fielen und sich wie ein gefährlich glitzerndes Meer zu ihren Füßen ausbreiteten. Lavay wurde 
schmerzlich bewusst, wie schwach und klein und dünn sie doch war. In einer Geste der Verzweiflung 
bemühte sie sich, ihren Arm aus der Klammer des Mannes zu entwinden, doch er merkte vermutlich kaum 
mehr als die Anspannung ihrer Muskeln und wie diese zu zittern begannen.\\
Am liebsten hätte sie ihm Beleidigungen entgegen geschleudert. Ihn als räudigen Hund und dreckiges 
Schwein betitelt, aber sie hatte schon damit zu kämpfen, ihre Lungen mit Luft zu füllen. Wie muss 
es Imur ergangen sein, als das Gewicht seines eigenen Körpers ihm die Luft nahm?\\
Onars raues Lachen ertönte ganz nah an ihrem Ohr. "Weißt du? Eigentlich kann er doch gar nichts mit 
dir anfangen. Er will diese Frechheit nur nicht auf sich sitzen lassen. Kann ich verstehen. Dich 
ungeschoren laufen zu lassen, würde den Ruf meines Herren schaden. Auf diese Verbrechen folgt der 
Tod, dass verstehst du ja, stimmt's? Man legt sich nicht mit dem kasirischen Adel an."\\
Lavay starrte mir offenen Augen in die Dunkelheit und ihr Widerstand war nur noch das Zittern ihrer 
Glieder. Ja, sie verstand es. Und sie ahnte, was er vorhatte. Und sie wusste, dass sie keine Chance 
hatte, sich zu retten. Wie betäubt nahm sie wahr, wie er seinen Griff veränderte. Das zerreißende 
Geräusch ihrer Kleidung hallte einem Echo gleich in ihren Ohren wieder. Sein Griff war grob und 
schmerzhaft. Er trat ihr die Beine weg und ließ sie auf den Waldboden fallen. Dort blieb sie liegen, 
denn kurz darauf war er schon über ihr. Ein Wolf, der seine Zähne um ihre Kehle schloss, wäre ihr 
willkommener gewesen.\\
Lavay hatte nicht die Kraft um ihre Augen zu schließen. Aber die Grausamkeit von dem, was gerade 
geschah, nahm sie mit fort. Ihre blauen Augen fixierten einen einzelnen Stern am Nachthimmel, den 
sie durch eine kleine Lücke in den Kronen des Bäume erspähen konnte. Sie sah Imur vor sich, wie er 
ihr lachend durchs Haar strich. Wie er mit dem Wind sacht hin und her schwang. Wie die Krähen 
sich auf ihm nieder ließen und zu picken begannen. Ihre Mutter, wie sie weinend in den Armen ihrer 
Tochter lag. Ihr Lächeln, dass zumindest Lavays Welt erhellen konnte. Und ihre Stimme, die leise 
sang: \textit{Leg dich nieder, mein Kind\\
Flieg wie ein Vogel im Wind.\\
Blicke in den Himmel, strecke die Arme aus,\\
Erreiche die Sterne und tanze auf dem Mond.\\
Und Morgen, wenn die Sonne erwacht,\\
wird wieder gescherzt und gelacht.\\ 
Leg dich nieder, mein Kind.\\
Flieg wie ein Vogel im Wind.\\
Flieg wie ein Vogel im Wind...}\\
Tränen liefen Lavay über ihre Wangen. Ein Schluchzen entwich ihr, als der Schmerz stärker wurde. 
Oder wurde sie wacher? \\
Imur hatte schon die andere Welt gesehen, als sie mit Haska ankam. Er ist mit dem Gedanken 
gestorben, dass keiner von den beiden Menschen, die ihm am wichtigsten waren, ihn hängen sehen 
würden. Was war besser? In die Augen von Fremden sehen, während man Maras Gnade empfing? Oder die 
Liebenden weinen sehen? Lavay wandte den Kopf und blickte in ein Augenpaar. Einen winzigen Moment, 
von Hoffnung getragen, dachte sie wahrhaftig, dass Imur dort saß. Aber die Augen waren Gelb.\\
Der Wolf hatte sie gefunden. Sah sie bedauern in seinen Augen? Weil ein anderer Jäger vor ihm die 
Beute erwischte? Das Tier wandte sich ruckartig ab und verschmolz wieder im Schatten des Waldes.\\
Hass entflammte, als hätte ein Blitz in ihrem Herzen eingeschlagen. Sie starrte dem Mann, 
der über ihr lag, ins Gesicht. Seine Bewegungen wurden immer ruckartiger, sein Gesicht verzog sich 
und die Augen kniff er zusammen, während er seinen stinkenden Atem keuchend ausstieß. Es war Lavay, 
als würde jemand anderes ihre Hand ergreifen, sie ziehen und zu dem Gürtel führen. Ihre Finger 
schlossen sich um den Griff, ohne dass sie selbst dazu in Stande war, die nötige Kraft für diese 
Tat aufzubringen. Und dann, in dem Moment als sich ein zufriedenes Grinsen auf seinem Gesicht 
abzuzeichnen begann, stieß sie zu. Die klinge verschwand in seinem Körper. Lavay begriff erst, als 
ihr warmes Blut über die Hände sprudelte. Oras dagegen bäumte sich auf, stieß das Mädchen von sich 
und rollte sich so von ihr herunter. Er schrie und riss den Dolch aus seiner Brust. Die Waffe hob 
sich drohend und fiel... Dumpf war der Ton, als sie im Laub aufkam. Ebenso dumpf, aber etwas lauter, 
als Oras wieder nieder sank und sich vor Schmerz wimmernd krümmte. Lavay sah ihm zu, wie sein Leben 
schwand. Zitternd, aber unfähig etwas zu empfinden, schlang sie ihre Arme um die Knie und 
wartete.\\
\textit{Sieht man es, wenn die Seele entweicht? Sieht man den Todesgott, der seine Arme um einen 
legt?}\\
Sie sah nichts, außer wie sein Blick glasig wurde und in die andere Welt saß. \textit{Aber 
vielleicht bin ich diejenige, die gerade gestorben ist.}\\
Und Lavay wünschte sich so sehr, dass dieser Gedanke Wirklichkeit wäre.\\

Die meisten Bewohner des kleinen Dorfes schienen auf dem Feld zu sein und die letzten Körner 
zu ernten, ehe der Herbst weiter voran schritt. Lavay sah eine Frau, die ihren Sohn eine Ohrfeige 
gab. Drei Kinder scheuchten dürre Schafe durch das Dorf und hinaus auf die feuchten Wiesen. 
„Was willst du hier, Fremde?“, fragte eine dürre Frau barsch. Der Hunger des letzten Jahres 
sah man ihr deutlich an. Das brachte Lavay zum stutzen. Sie hatte gedacht, dass die Bauern weniger 
hungern mussten als die Arbeitslosen in der Stadt. Immerhin pflanzten sie alles an. Nachfragen 
brauchte sie jedoch auch nicht, sie war klug genug es sich zusammen zu reimen. Irgendwoher mussten 
alle Menschen, jenseits der Armenviertel und der Bauern, ja ihr Getreide bekommen.\\
Lavay lächelte bemüht freundlich. „Ich wollte mich erkundigen, ob einer der Leute hier mir 
vielleicht einen Platz zum Schlafen geben könnte. Nur für die nächsten paar Stunden. Ich bin bald 
wieder weg. Und ich kann zahlen.“\\
Sie würde den Armreif hergeben. Immerhin war er einer der Gründe für ihr Unglück. Die Frau musterte 
sie misstrauisch von oben bis unten. „Was macht ein junges Mädchen wie du alleine hier? Kommst du 
etwa aus Salaica, um uns auszuspionieren?“\\
Lavay sah sie überrascht an und schüttelte schnell den Kopf. \textit{Saleica? Ich kann doch 
unmöglich schon so nahe der Grenze sein... Vielleicht sollte ich doch wieder nach Norden...}\\
„Ich bin mit meinem Vater gereist. Aber wir haben einander verloren. Wir hatten ausgemacht, dass 
wir uns in der nächst größeren Stadt treffen.“\\
„Du kannst Baram fragen, ob er ein Plätzchen für dich hat. Der lässt fast jeden in seine Hütte. Aber 
du solltest bis Morgen warte, wenn du weiter wandern willst. Hier in der Nähe ist ein Wolfsrudel, 
das schon einige Schafe gerissen hat. Sie sind genauso hungrig wie wir.“\\
„Vielen Dank“, sagte Lavay und ging zu dem Haus, auf welches die Frau gezeigt hatte.\\

Baram verkaufte ihr etwas Brot und gab ihr sogar einen Schluck Milch von seiner Kuh zu trinken. 
Schlafen ließ er sie kostenlos in einer Ecke seiner Hütte. Lavay hatte nicht mit so viel 
Freundlichkeit gerechnet und wagte kaum sich zu entspannen. Sie wich aus, sobald eines der 
Familienmitgliedern ihr zu nahe kamen, selbst bei dem jüngsten Mädchen, welches auf wackeligen 
Beinen erste Laufversuche machte. Früher hätte sie Baram vielleicht gemocht. Jetzt ließ sie nicht 
zu, sicher überhaupt einer Gefühlsregung hin zu geben und lächelte nur still und freundlich vor 
sich hin. 
Zum Frühstück lud Barams Weib sie herzlich ein und es sollte wohl zufällig wirken, dass sie gleich 
zwei ihrer Söhne neben sie Platz nehmen ließ. Baram reichte ihr strahlend eine Holzschüssel voller 
Haferbrei und fragte: „Verreist du viel mit deinem Vater?“\\
„Manchmal“, log Lavay: „Wir haben Verwandte besucht, aber einander verloren.“\\
Barams Frau runzelte die Stirn. „Dann ist er nicht sehr aufmerksam.“\\
„Ich kann mir gar nicht vorstellen, jemanden wie dich aus den Augen zu verlieren“, murmelte Barams 
Sohn und wurde rot. Lavay lächelte schief und starrte in ihre Schüssel. Aber ihr entging nicht, wie 
Baram dem Jungen auffordernd zu zwinkerte.\\
„Und woher stammst du nun, Lavay?“\\
„Wir leben in der Nähe von Majaku… Aber seid meine Mutter tot ist, reist mein Vater viel… ich denke, 
unser Haus erinnert sie zu sehr an sie“, sagte Lavay und blickte betreten in ihre Schüssel. Ihre 
Augen füllten sich mit Tränen, als sie an ihre Mutter dachte. Schnell wischte sie sie fort und 
lächelte schief. Die Bauersleute sahen sie stumm an, erwiderten dann ihr unerwartetes Lächeln und 
das Essen wurde erst einmal schweigend fortgesetzt.\\
„Willst du noch was?“, fragte Barams Weib, riss Lavay die halbvolle Schüssel aus der Hand und füllte 
sie erneut.\\
 „Nein“, rief Lavay: „Das muss doch nicht sein. Ich will euch doch nicht euer gesamtes Essen 
nehmen.“\\
Auf diese Aussage, bekam sie ein fast schon gütiges Lächeln von der Mutter und Baram schien sehr 
zufrieden. „Natürlich nicht“, sagte er: „Du bist ein anständiges Mädchen.“\\
Lavay nickte nur Gedanken verloren. Sie konzentrierte sich auf den warmen Brei und verpasste den 
Anfang des plötzlichen Streits. Baram stand aufrecht einem seiner Söhne gegenüber und hatte ein 
zornrotes Gesicht. ``Wir haben darüber schon gesprochen und ich habe nein gesagt!''\\
``Das ist viel zu gefährlich!'', fiel Barams Weib mit zitternder Stimme ein.\\
``Ich werde hier doch nicht verhungern'', rief der junge Mann und schüttelte energisch den Kopf: 
``Wartet es ab, mit dem Sold kann ich euch einen Braten kaufen!''\\
``Fals dich die Saleicaner nicht häuten und selbst zum Braten machen'', feixte der Älteste des 
Nachwuchses.\\
Lavay machte sich auf ihrem Hocker klein und wagte sich kaum zu fragen, aber die Neugierde siegte: 
``Sold? Willst du in die Armee eintreten?''\\
Stolz nickte er ihr zu und lächelte charmant. ``Klar. Wie sollte ich sonst eine Familie ernähren 
können? Und einer Frau das bieten können, was sie verdient?''\\
``Gibt es Krieg?'', fragte sie überrascht. So lange war sie doch nicht außerhalb der Zivilisation 
gewesen, oder doch?\\
Baram seufzte und setzte sich kraftlos wieder an den Tisch. ``Noch nicht. Aber man weiß bei diesen 
saleicanischen Teufeln ja nie... Nen Händler hat vor ein paar Tagen gemeint, dass die großen 
Kaufleute ihre Handelsrouten ändern und Schutzverträge eingehen. Vielleicht ist das alles nur 
Gerede, aber mein Bruder im nächsten Dorf hat erzählt, dass die Rekrutierer schon dort waren und 
nach Freiwilligen suchen.''\\
``Zu denen du nicht gehören wirst''; schimpfte Barams Weib an ihren Sohn gerichtet fort und kurz 
darauf war ihr Gatte schon wieder auf den Beinen und redete auf den jungen Mann ein.\\
„Also…“, stammelte Lavay und rappelte sich auf: „Ich muss dann auch mal weiter… mein Vater wartet 
bestimmt. Vielen, vielen Dank für die Gastfreundschaft.“\\
„Aber, du wolltest doch noch etwas bleiben… wegen den Wölfen!“, stammelte Baram enttäuscht. 
Vielleicht hatte er sich erhofft, dass sie seinen Sohn davon abhalten könnte, sich freiwillig 
rekrutieren zu lassen. Vermutlich hatte er schon Hochzeitspläne im Hinterkopf gehabt. Bei dieser 
Vorstellung hielt sie nun gar nichts mehr hier und sie wiederholte hastig: „Vielen, vielen Dank für 
das Essen und eure Güte, aber ich muss zu dem Treffpunkt, den ich mit Vater verabredet hatte. Sonst 
sorgt er sich noch weitere Stunden unnötig.“\\
Barams Frau umarmte sie trotzdem herzlich und drückte ihr einen Kuss auf die Wange. Lavay erstarrte 
bei der Geste, obwohl sie von einer Frau stammte. Baram nickte mit enttäuschten Blick, während die 
jüngsten Kinder sie nur aus großen Augen anblickten.\\
Und wieder hörte sie das Heulen des Wolfs.\\
Nachdenklich zupfte sie an Grashalmen und überlegte, in welche Richtung sie weiter gehen sollte. Sie 
war alleine, musste auf niemanden Rücksicht nehmen. Genau genommen konnte sie überall hin. Kasir 
war groß. Und warum nicht das Land irgendwann einmal verlassen? Kasir grenzte nur an ein Land, 
Saleica. Einst sollen es wohl viele Länder dort im Süden gewesen sein, aber das aufbraußende 
Volk hatte sie alle erobert und unterworfen. Teufel nannte man sie, Dämonen, die nur für das 
Töten und Siegen geboren waren.\\
Lavay war alleine. Nichts hielt sie mehr. Alle die ihr am Herzen lagen waren Tod und aus ihrer 
Heimat, wer konnte sich da noch an ihren Namen erinnern? Außer Haska vielleicht. Nun konnte sie 
sein wer sie wollte, ihre eigene Geschichte neu schreiben! Sie benötigte lediglich ein paar Lügen 
und Fantasie. An einem Tag war sie eine Kaufmannstocher, die vor ihrem grässlichen Verlobten davon 
lief, in der nächsten eine Pilgerin. Weit fern waren die trostlosen Tage in Janka. Der staubigen 
Stadt in Süd-Saronne weinte sie keiner Träne nach.\\
Sie runzelte nachdenklich die Stirn. Aber sie konnte ja nicht für den Rest ihres Lebens so weiter 
machen, oder? Immer nur von einem Dorf zu nächsten und ab und zu was klauen? Lavay schüttelte den 
Kopf. Darüber wollte sie jetzt nicht nachdenken, sie war zu sehr damit beschäftigt die Geister der 
Vergangenheit zu verdrängen.\\

Der Weg war nicht mehr ein Trampelpfad und wohin er führte, darüber dachte Lavay mittlerweile nicht 
mehr nach. Sie konnte sich gerade gut vorstellen, wie ihre Mutter sich in den letzten Tagen vor 
ihrem Tod gefühlt haben musste. Trotz der Schwäche die ihren Körper wie Gewichte nieder drückte, 
setzte sie einen Fuß vor den Anderen. Manchmal schwankte sie leicht, hielt den Blick aber stur auf 
das bisschen blanke Erde. Lavay hatte nie viele Nahrungsmittel gehabt, aber Schlaf. Sie konnte 
nachts in der Wildnis kaum zur Ruhe finden, egal wie erschöpft sie war. Kaum schloss sie die Augen, 
meinte sie den fauligen Atem eines Wolfs zu spüren, die Stiche von Insekten, der Ruf der Raubtiere 
erklang. In Dörfer ging sie auch nicht mehr, denn sie konnte sich nicht mehr dazu überwinden. Dort 
waren Männer. Und so freundlich sie auch vielleicht sein könnten, die Begegnung mit Barams Familie 
hat ihr auch wieder gezeigt, dass die Männer stets an mehr interessiert waren. \\
So taumelte sie also schon seit Tagen, die sie nicht zu zählen vermochte, in den südlichen Gefilden 
des Landes umher. Manchmal fand sie Beeren oder Früchte des Herbstes, einmal sogar einen zappelnden 
Hasen in einer Falle. Es tat ihr im Herzen weh, das Tier zu töten, aber der Hunger zwang sie zu 
dieser Tat. Sie weinte bittere Tränen, als sie das Blut des Tieres an ihren Händen sah. Das erste 
Mal, dass sie selbst ein Tier getötet hatte...\\
\textit{Imur hat mich vor so viel geschützt}, dachte sie und versuchte sich nicht wieder vor Kummer 
die Zunge blutig zu beißen.\\
Lautes Wiehern erklang und Lavay blieb stehen. Seit Stunden hatte sie keine Zivilisation gesehen, 
aber sie war auch nicht sehr schnell gelaufen. Die Erschöpfung ließ keine Panik zu, stattdessen sah 
sie sich nur um und erblickte auf der nächsten Hügelkuppe fünf Reiter. Sie blieb stehen wo sie war, 
denn es gab in dieser Wiese nichts, was ihr hätte Schutz geben können. Vielleicht würde es ja 
schnell gehen. Der letzte Keim der Hoffnung wurde für sie zerstört, als de Reiter nähe kamen und 
sie erkannte, dass es keine offiziellen Soldaten waren. Sie hatte viele in den letzten Tagen aus 
der Ferne gesehen und man erkannte sie gleich an ihren blauen Uniformen und dem Wolf darauf.\\ 
Lavay hätte schon fliegen können muss, um den schnellen Pferden der Männer zu entkommen. Sie 
galoppierten auf die junge Frau zu und umkreisten sie lachend. Lavay versuchte ihnen in die 
Gesichter zu blicken, aber sie bewegten sich zu schnell und hektisch. Plötzlich fand sie sich auf 
dem Boden wieder, hatte es gerade noch geschafft sich mit den Händen und Knien aufzufangen. Einer 
der Männer hatte den Knauf seines Schwertes gegen ihre Schulter gestoßen und der Schmerz 
explodierte regelrecht in ihrem Körper. Ihr wurde kurz schwarz vor Augen und als sie Anstalten 
machte, sich wieder aufzurappeln, schwebte eine Klinge über ihrem Gesicht. Aus weit aufgerissenen 
Augen starrte sie zu dem Soldaten hinauf. Er grinste. Seine Kameraden blieben auf ihren Pferden 
sitzen und lachten. \\
„Wie ein Hase ist sie gerannt“, spottete einer. \\
„Kleine Diebin, dachtest du wirklich, du könntest einfach so weg rennen?“, fragte einer der anderen
Mascha, er war derjenige, dessen Schwert nur wenige Zentimeter von Lavays Haut entfernt war, 
schüttelte den Kopf. „Nicht nur Diebin. Sie hat meinen Bruder getötet!“\\
Lavay konnte in diesem Moment überhaupt nicht denken. Sie beobachtete nur das Schwert über sich und 
zuckte vor der Berührung des kalten Stahls zurück. \textit{Was schmerzt mehr? Der Strick? Das 
Fieber? Die Klinge? Oh Mara... ist es wahr? Ist der Tod Erlösung? Geleitet mich der Wind ins 
Jenseits?}\\
„Mascha, reiß dich zusammen! Unser Herr will sie haben. Sie wird ihre Strafe bekommen.“\\
Lavay zitterte und kniff die Augen fest zusammen, während Mascha sie verfluchte und sein Schwert 
nicht fort nahm. Sie flehte stumm, dass die Männer sie hier und jetzt töten würden. Wäre das nicht 
perfekt? Ihr Blut würde das Gras nähren und der prasselnde Herbstregen würde es fortschwämmen. Die 
Stürme würden ihre Seele mit sich nehmen, bis zum Mond, bis zu den Sternen. Das alles klang 
erfüllender als das Ungewisse, was ansonsten geschehen würde. Nein, es war nicht ungewiss. Die 
Männer würden das gleiche tun was Maschas Bruder bereits getan hatte. Sie konnte sich die Folter 
denken, die Qualen einer einsamen, finsteren Kellerzelle. Erniedrigende Gerichtsverhandlungen, 
deren Ergebnis bereits schon fest stand. Der Adelige würde sie und alles was sie ausmachte 
zerfetzten und damit bei der nächsten Abendgesellschaft prahlen.\\
Mehrere Pferde wieherten schrill und die Männer brüllten einander etwas zu. Lavay schlug die Augen 
wieder auf und sah, wie Mascha das Schwert aus der Hand glitt. Es fiel dich neben ihr zu Boden. Er 
drehte sich langsam um und Lavay zählte drei Pfeile in seinem Rücken. Gurgelnd ging er in die Knie 
und sackte zusammen. Seine Kameraden griffen nach ihren Schwertern. Ein Dutzend Reiter trabten fast 
schon gelassen auf sie zu. Ein goldener Löwe der sich hoch zur Sonne streckte, prang auf ihren 
roten Gewändern. Das Wappen hatte Lavay noch nie in ihrem Leben gesehen, aber sie kannte es 
trotzdem. Jedes Kind in Kasir kannte den Löwen, der die Sonne herausforderte. Jedes Kind kannte die 
Geschichten der barbarischen Krieger Saleicas, welche Dörfer plünderten und Menschen umbrachten, um 
damit ihren Gott zu ehren. Aber was taten sie in Kasir? Seit Generationen hatten sie sich nicht 
weiter in den Norden gewagt, hieß es in den Geschichten. Der Norden sei standhaft, darum hatten sie 
ihre gierigen Löwenklauen nach Süden über das Meer ausgestreckt.\\
Als Aras Männer die Überzahl sahen, rissen sie an den Zügeln und brüllten ihre Pferde an. Hufe 
donnerten dich an Lavay vorbei. Sie hörte den Anführer der Gruppe etwas rufen, konnte es aber nicht 
einordnen. Benommen blieb sie stehen und sah zu, wie die Löwen über Maschas Kameraden herfielen. 
Ihre Klauen zerfetzen Muskeln und Sehnen, ihre Kiefer durchtrennten Gliedmaßen und ihre Pranken 
brachen Knochen. Lavay betrachtete das Blut, welches Maschas Herz aus seinen wunden pumpte und kam 
nicht umhin, dieses tiefe Rot als schön zu empfinden. Der Lärm des Kampfes, den sie als das 
Brüllen von Löwen wahrnahm, trat in den Hintergrund bei diesem Anblick. Irgendwann dann spürte sie 
eine schwere Hand auf ihrer Schultern. Sie blickte auf. \textit{Die Löwen rasieren sich...}, 
stellte sie verwundert den offensichtlichsten Unterschied zwischen den kasirischen und 
saleicanischen Männern fest. Der Soldat sagte etwas in seiner ihr fremden Sprache und sie sah ihn 
nur stumm an.\\
Ein weiterer Löwe lenkte sein geschecktes Tier näher. Lavay erkannte den nachlässig geflochtenen 
Zopf und runzelte die Stirn. Offensichtlich gab es auch weibliche Löwen in der saleicanischen 
Armee. Die Frau musterte sie abschätzend und sprach in undeutlichem Kasirisch: ``Er will wissen, ob 
du die Hure dieser Männer bist.''\\
Lavay war sprachlos und schaffte es nur den Kopf zu schütteln. Zu spät kam ihr der Gedanke, ob 
diese Geste bei den Saleicanern wohl die selbe Bedeutung hatte, aber sie schienen es richtig zu 
verstehen.\\
Die Soldaten begannen zu reden und auf Lavay wirkte es so, als ob sie über sie und den Kampf 
scherzten. Der Mann stieg wieder auf sein Reittier und warf der Frau ein Seil zu. Die Kameraden 
lachten grölend, während die Frau es geschickt auffing und dann zu Lavay ging und ihr die Hände 
fesselte. Lavay ließ es widerstandslos geschehen und fragte stattdessen: ``Warum seit ihr hier?''\\
``Krieg'', antwortete die Soldatin einsilbig: ``Wir wollten Spaß haben. Hatten wir. Wir wollten 
plündern. Haben wir nicht. Also bist du jetzt meine Beute. Es ist ehrlos ohne Beute heim zu 
kehren.''\\
``Krieg?'', wiederholte Lavay überrascht. Sie hatte gehört, dass die kasirische Armee dringend 
Rekruten suchte, aber dass der Krieg bereits ausgebrochen war, wusste sie nicht.\\
Die Soldatin zeigte grinsend ihre Zähne. ``Mein Name ist Sokra A'Rell und du bist nun mein 
Besitz!''\\

Sokra schwang sich wieder in den Sattel und ihr Reittier machte einen überraschen Satz, als sie ihm 
brüsk die Fersen in die Flanken stieß. Das Pferd warf den Kopf hoch, die Mähne flog durch die Luft. 
Triumphierend warf sie noch einen letzten Blick auf das Schlachtfeld, welches sie und ihre 
Kameraden hinterlassen hatten, und reckte die Faust in den Himmel. Der Siegesgruß, begleitet von 
ihren lauten Worten: ``Alle die unser Brüllen hören!''\\
Die anderen Männer und Frauen fielen mit ein: ``Alles was das Sonnenlicht berührt!''\\
Helix und Farun wendeten ihre Pferde und galoppierten mit einem lauten Jolen über die Leichen 
hinweg. Einige andere taten es ihnen noch nach, während Sokra einer der Jüngsten in der Truppe mit 
einem Kopfnicken zu verstehen gab, dass die Pferde der Toten eingesammelt werden sollten. 
Zumindest, solange sie noch brauchbar waren. Sie grinste schief, während sie ihre Kameraden 
beobachtete. Obwohl sie offiziell auf dem selben Rang stand wie sie, war sie bei dieser eher 
inoffiziellen Mission die Anführerin.\\
\textit{Verdammt, es ist Krieg!}, dachte sie und lachte still. \textit{Und so sollte ein Krieg auch 
aussehen!}
In den letzten Wochen lag das Gefühl des nahenden Krieges in den Garnisonen in der Luft. Und 
trotzdem saßen alle nur herum, schärften Klingen, polierten Rüstungen, flickten Zaumzeug. Sokra 
wusste ganz genau, dass sie und ihre Freunde nicht die einzigen waren, die sich fort stahlen um 
schon einmal mit dem Krieg zu beginnen, während die Offiziere und der Adel noch an den gedeckten 
Tischen saßen und leise plapperten! Krieg ist laut, blutig und wild!\\
Sokra sah ruckartig zu der schmächtigen Gestalt auf dem viel zu großen Pferd neben sich. Der junge 
Priester war ihrer Einheit erst seit wenigen Wochen zugeteilt, aber er war es gewesen, der ihnen 
erzählte, dass der Krieg sicher war. Osyma sehnt sich nach Heldentaten, nach sterben zu seinen 
Ehren. Blut, dass für den Allmächtigen vergossen wird. Todesschreie, die in seinem Namen erklingen. 
Sokra war es egal ob Gott oder König es befahl. Sie war Soldatin, eine Kriegerin Saleicas. Seit 
mehr als 15 Jahren diente sie im Heer und eine richtige Schlacht hatte sie noch nie erlebt. Sie 
hätte alles gegeben um in die Kolonien versetzt zu werden, aber sie war in der saleicanischen 
Grafschaft Merandila geboren und verpflichtet die Grenze gen Norden zu schützen.\\
``Na?'', fragte sie den Priester, dessen Tätowierungen sich auf seine Arme beschränkten: ``Ist 
Osyma glücklich?''\\
Jetzt, da sie ihn direkt angesprochen hatte, schien er seinen Mut zusammen zu nehmen und mit 
trotzig erhobenem Kinn erwiderte er: ``Sie hatten kein Recht ohne Befehl ins Feindesland 
einzudringen.''\\
``Du bist auch eingedrungen'', erwiederte Sokra abwertend.\\
``Das waren Zivilisten.''\\
``Sie hatten Waffen. Unschuldige tragen keine Waffen'', spottete Sokra: ``Außerdem haben wir einen 
richtigen Zivilisten gerettet.''\\
``Wieso dann die Fesseln?'', fragte der Priester und betrachtete die junge Frau, die reglos da 
stand und mit erschrockenen Augen das Spiel der Reiter betrachtete. Sie verstand vermutlich kein 
einziges Wort der Unterhaltung. ``In Saleica gibt es keine Sklaverei mehr'', sprach er.\\
Sokra lachte auf. ``Das sagte unser König. Was sagt der Gott dazu?''
Die Soldatin hob die Finger an die Lippen und stieß einen schrillen Pfiff aus. Nicht nur ihr Pferd 
spitzte die Ohren, sondern auch ihre Kameraden sammelten sich wieder. Ihre Pferde hinterließen 
blutige Hufabdrücke. Sokra winkte lässig mit der Hand und deutete den Rückzug nach Süden an. Zurück 
in die Heimat. Zu der Gefangenen herrschte sie in kurzen, kasirischen Worten: ``Steig auf.''\\
Ihre Kameraden lachten, als die Frau ungeschickt versuchte sich mit den gefesselten Händen auf das 
stämmige, kasirische Pferd zu ziehen. 

Sie ritten fast zwei Tage durch, nur mit wenige Stunden Rast. Jeder einzelne von den Soldaten war 
ein geübter Reiter und keiner von ihnen hatte ein Problem damit, in der langsamen Gangart im Sattel 
zu schlafen. Es beschwerte sich auch keiner, außer der Priester. Sein Jammern dämpfte den Triumph, 
den sie empfunden hatten. Er wurde nicht müde, ihnen mögliche Konsequenzen aufzuzählen. Sokra war 
bewusst, dass man ihnen gehörig Ärger machen konnte, aber streng genommen hatten sie gegen keinen 
Befehl verstoßen. Außerdem, wer sollte sie richten? Die Offiziere waren abhängig vom König oder 
dessen Vertretern den Grafen. Der König Saleicas könnte doch niemals seine Soldaten verurteilen, 
die für den Seegen des Allmächtigen in den Kampf gezogen sind! Und den alten Grafen hatte Sokra 
sogar schon mal gesehen. Er hatte das Wesen eines Kriegers, auch von ihm erwartete Sokra keine 
ernst zu nehmende Bestrafung. Und die Priester? Sie würden sich doch ins eigene Fleisch schneiden, 
wenn sie Soldaten dafür bestraften, dass sie dem Allmächtigen ehrten.\\
Überraschend bemerkte Sokra aber auch, dass die Gefangene keinen Ton von sich gab. Sie hing sehr 
schief im Sattel, sodass die Soldatin erst befürchtete, sie würde irgendwann einfach herunter 
fallen. Deshalb hatte sie sie mit zusätzlichen Seilen fest am Sattel festgebunden und das Problem 
war schon einmal erledigt. Sokra sah der jungen Frau an, dass sie vermutlich noch nie auf einem 
Pferd gesessen hatte und sie durch eine falsche Haltung Schmerzen plagten. Aber daran war das Weib 
selbst schuld. Zu Fuß würde sie nur aufhalten und Sokra war nicht bereit, ihre Beute einfach so im 
Staub liegen zulassen. Sie hatte auch schon eine Idee, wo sie ihre Gefangene unterbringen konnte, 
ohne dass ihr Offizier davon erfahren musste.\\

Rotan streckte sich um seine verkrampften Muskeln zu lockern. Mutlos legte er sich zurück und seine 
Finger tasteten durch das taufrische Gras. Er stöhnte, weil er keine Worte fand seinen Schmerz zu 
beschreiben. Wenn er wenigstens auf jemanden wütend sein könnte, wenn er wenigstens jemanden die 
Schuld geben könnte! Aber es gab niemanden.\\
Niemanden außer ihm selbst. Egal wie sehr er es sich auch einredete, dass er nichts für sie hätte 
tun können. Seine Hände ballten sich zu Fäusten. \textit{Ich durfte nicht einmal zu ihr. Nicht 
einmal, als klar wurde, dass ihr nur noch wenige Atemzüge bleiben.}\\
Stattdessen hatte er an der Wand gekauert und zuhören müssen, wie seine kleine Tochter das erste 
und 
letzte Mal schrie. Es hatte sich anders angehört, als die ersten Schreie seines Sohnes. Damals war 
die Stimme des Säuglings kraftvoll, trotzig und gesund durch das Gut gehallt. Aber die kränkliche 
und schwache Stimme seiner kleinen Tochter nahm alle Hoffnung mit sich fort. Er hatte nicht dabei 
sein dürfen, als seine Frau starb. Er hatte nicht dabei sein dürfen, als seine kleine Tochter für 
wenige Minuten das Leben kennenlernte und dann der Mutter folgte.\\
Rotan richtete sich auf und betrachtete die aufgewühlte Erde unter dem Kirschbaum. Dort ruhten Frau 
und Tochter. „Verdammte Priester!“, zischte er leise.\\
Rotan sah sich zögernd um, ob irgendjemand seine wütenden Worte gehört hatte. Er entspannte sich 
wieder, als er niemanden erblickte. Betrübt betrachtete er den alten Baum, den sein Vater einst 
gepflanzt hatte. Unter diesem Baum hatte er Illiana kennengelernt. Unter diesem Baum und dem 
grimmigen Blick eines Priesters hatten sie einander ewige Verbundenheit geschworen. Unter diesem 
Baum lag sie nun begraben.\\
„Rotan“, sagte sein Knecht, der sich leise genährt hatte: „Illiana war einfach zu schwach um die 
Geburt zu überleben. Und eure Tochter ebenso.“\\
Rotan schüttelte den Kopf. „Ich war nicht für Illiana da. Und meine kleine Tochter war nicht 
schwach. Sie hat nur früher als wir anderen erkannt, dass diese Welt eisig und leer ist. Sie hat 
erkannt, dass es hier weder Recht noch Wärme gibt. Dass es nichts gibt, wofür es sich zu leben 
lohnt.“\\
Das folgende Schweigen brach der Knecht nach wenigen Minuten mit den zögernden Worten: „Ich wollte 
mich nur verabschieden. Die Pferde sind alle verkauft. Die letzten werden morgen abgeholt.“\\
„Alle?“, fragte Rotan heißer.\\
Er hatte in den letzten Wochen kein einziges Mal den Stall betreten. Und er wagte sich zu erinnern, 
er habe dem Knecht aufgetragen alle fortzujagen. Pferd wie Arbeiter.\\
„Bis auf den alten Braunen, die junge Falbe und dem Rappen.“\\
„Warum?“\\
„Wer will schon einen alten Gaul? Und der Rappe lahmt.“\\
„Die Stute?“\\
„Ist zu wild. Keiner wagte sich in ihre Nähe. Und jeder der es wagte, konnte sich keine Minute auf 
ihren Rücken halten. Soll ich noch dem Schlachter Bescheid sagen, bevor ich gehe?“\\
Rotan zog die Schultern hoch. „Nein. Das mach ich selbst.“\\
„Und dann?“\\
Rotan ignorierte seine Frage und blieb reglos sitzen, bis sein jahrelanger Freund sich schließlich 
seufzend abwandte, seine Reisetasche schulterte und ging. Rotan holte tief Luft und antwortete in 
die Stille hinein: „Ich weiß es nicht.“\\

Der Besitzer des leeren Guts schlürfte mit hängenden Schultern die Wiese, auf denen vor wenigen 
Wochen noch mehr als zwei Dutzend reinrassiger Pferde standen, zu seinem verwaisten Hof. Er 
meinte noch das vielstimmige Wiehern zu hören und trampelnde Geräusch, welches die Hufe auf dem 
Gras verursachte. Und die Fohlen erst, wie sie um die Stuten herum sprangen und sich gegenseitig 
zum wilden Spiel aufforderten.\\
„Bruder!“, rief eine tiefe Frauenstimme.\\
Rotan sah sich um und entdeckte ein Dutzend Soldaten die quer über seine Wiese kamen. Wäre er nicht 
so müde, hätte er ihnen die Hölle heiß gemacht. Schließlich trampelten sie unnötig das Gras 
zusammen, welches er als Heu im Winter brauchen würde. Ach nein. Er benötigte es ja nicht mehr. 
\textit{Der erste Winter seit Jahrzehnten, in dem kein Hufescharren den Stall erfüllen wird.}\\ 
Während Rotan beobachtete, wie seine ältere Schwester und ihre Kameraden näher kamen, fürchtete er 
einen Moment, dass sie ihm den letzten Rest nehmen könnte. Eigentlich war Sokra die Erstgeborene 
und 
Erbin. Aber sie hatte ihm Land, Gut und Tiere überlassen um zum Heer zu gehen. Sie hatte ihm die 
Grundlagen für ein gutes Leben gewährt. Er stand auf Ewig in ihrer Schuld. Er sollte ihr dankbar 
sein. Aber stattdessen empfand er nur Abscheu. Sokra vergötterte König Semrik als wäre er wirklich 
der Sohn des Allmächtigen. Rotan dagegen war das gleichgültig. Was ging ihn Politik an, wenn er 
genug Probleme mit den Fohlen hatte. Aber jetzt hatte er diese Probleme nicht mehr.\\
Als sie nahe genug waren, verneigte sich Rotan widerwillig vor seiner Schwester. Zwar hatte sie ihm 
das Land überlassen, trug aber noch den Titel des Familienoberhauptes.\\
„Ich habe nicht viel Zeit.“\\
\textit{Warum bist du dann hier?!}\\
Ihr Pferd warf den Kopf hoch und tänzelte. Sie griff die Zügel hart und Rotan schenkte dem Tier 
einen mitleidigen Blick. Das war ein gutes Pferd. Mit einem guten Reiter wäre es fantastisch. 
Aber Sokra hatte nie ein Händchen für Pferde besessen. Jedoch für große beängstigende Waffen und 
daher sagte er nichts dazu.\\
„Warum so betrübt? Deine Frau und dein Kind weilen jetzt an der Seite des Allmächtigen. Du solltest 
dich darüber freuen.“\\
Rotan schaffte es zu lächeln und nickte. „Ich weiß. Ich freue mich natürlich, dass er sie 
auserwählt 
hat um so früh schon an seinem Tisch zu sitzen. Aber ich hätte sie doch gerne noch etwas an meinem 
gehabt.“\\
\textit{Verlogene Scheiße}, dachte er finster und nicht zum ersten Mal fragte er sich, wie die 
Leute nur an solch einen Unsinn glauben konnten.\\
„Nun, ich habe ein Geschenk für dich.“\\
Rotan schloss einen Moment die Augen. Dann fasste er sich und zwang sich zu einem weiteren Lächeln. 
„Du hast mir in der Vergangenheit Geschenke gemacht, für die ich ewig in deiner Schuld stehen 
werde, 
Schwester. Ein weiteres ist nicht nötig.“\\
„Ich will es aber los werden.“\\
\textit{Genau wie den Hof.}\\
Sokra zerrte ein einem Strick und erst jetzt fiel Rotan das Mädchen auf. Sie stolperte und fiel 
hin. 
Ihr entwich ein leises Ächzen und sie blieb zitternd sitzen. Sie starrte vor Schmutz und das Haar 
glänzte fettig. „Aus den Kolonien?“, fragte er zögernd.\\
„Aus Kasir“, antwortete Sokra: „Also fast.“\\
Die Soldaten lachten. \\
„Was soll ich mit ihr?“\\
„Mach was du willst. Sie kann dir im Stall helfen.“\\
Rotan schwieg. Er wollte ihr nicht sagen, dass es nichts mehr im Stall zu tun gab. Ihr war nie viel 
an den Tieren oder am Haus gelegen, trotzdem wollte er nicht, dass sie es als Zeichen seiner 
Schwäche auffasste. Sie würde ihn nur auslachen. Also nickte er stumm. Sie drückte ihm den Strick 
in 
die Hand und wiederholte: „Ich habe es eilig.“\\
``Sag mir erst, was das alles soll'', forderte er und tatsächlich hielt sie inne und blickte einen 
Moment nachdenklich auf ihn herab. Leise antwortete sie: ``Sie ist mein Besitz.''\\
``In Saleica gibt es keine Sklaverei mehr!'', entschied Rotan.\\
``Genau deswegen soll sie hier bei dir bleiben.''\\
``Und was hast du davon? Willst du alle paar Wochen kommen und dir von ihr die Stiefel putzen 
lassen?'', fauchte Rotan zornig. Diesmal ging Sokra viel zu weit.\\
Sie lachte jedoch nur. ``Mal sehen. Mein Pferd putzt ja schon du!''\\
Rotan ballte die Hände zur Faust. ``Ich mein es ernst! Ich gehe persönlich zu deiner Garnison und 
klage dich an, Sklaverei zu betreiben.''\\
Ungeduldig und wütend verzog sie das Gesicht. ``Was willst du eigentlich? Wir haben sie gerettet. 
Gauner waren hinter ihr her, sie war mitten in der Wildnis und allein. Hätten wir sie dort liegen 
lassen sollen, wäre dir das lieber gewesen? Dann mach du halt mit ihr was du willst. Ich hab keine 
Lust mehr mich mit dem Weib zu beschäftigen!''\\
Rotan erkannte hinter der zornigen Mimik seiner Schwester, dass sie Angst hatte. Mittlerweile war 
wohl genügend Zeit verstrichen und sie sich ihrer Tat bewusst. Sie hatte Angst vor Konsequenzen. 
Diese Frau war der Beweis, dass sie in Kasir gewesen war. Er würde ihr zutrauen, dass sie diesen 
Beweis verschwinden lassen würde, wenn sie ihn nicht anderweitig los würde. Er nickte also nur.\\
Ohne einen Abschiedsgruß rammte sie ihrem Reittier die Fersen in die Flanken und preschte gefolgt 
von ihren Kameraden los. Rotan sah ihr lange hinterher, bis er schließlich ein kleines Messer 
zückte und die Fesseln durch schnitt. „Sie haben dich wie einen alten Köter festgebunden“, stellte 
er auf Kasirisch fest.\\
Sie rappelte sich ruckartig auf und starrte ihn böse an. Rotan musste gestehen, dass sie doch nicht 
so hilflos war, wie er sie erst wahrgenommen hatte. Ihre blauen Augen funkelten zornig und ihre 
gesamte Körperhaltung zeigte, dass sie mit einem Kampf rechnete. Er ließ den Strick fallen und sah 
sie nachdenklich an. Rotan dachte an den Kirschbaum unter dem seine Frau und seine Tochter begraben 
lagen.\\
``Ich heiße Lavay'', sagte sie schließlich und sah ihn ratlos an. \\
Sie verstand nicht recht, was gerade eben geschehen war, aber Rotan wirkte freundlicher auf sie als 
die Soldaten. Und er sprach Kasirisch deutlicher. ``Woher sprichst du meine Sprache?''\\
Er kratzte sich über die kurzen Barthaare. ``Rotan A'Rel. Meine reizende Schwester hast du 
kennengelernt. Ich hatte früher öfters Kontakt mit Kasira. Habe viele Pferde in den Norden 
verkauft.'' Er stockte kurz und fügte hinzu: ``Und meine Frau hatte entfernte, kasirische 
Verwandte. Du siehst als, als hättest du Schmerzen, Mädchen.''\\
Lavay senkte den Blick. Sie wollte es nicht gerne zugeben, aber sie verharrte unter anderem deshalb 
reglos, weil die Innenseite ihrer Schenkel durch die Reibung des Sattels wund waren. Ihr ganzer 
Körper fühlte sich verkrampft an und sie sehnte sich nach erholsamen Schlaf. Der Gutsbesitzer 
musterte sie eingehend und nickte nur verstehen. ``Es war ein langer Ritt für einen schlechten 
Reiter, nehme ich an.''\\
Er führte die junge Frau auf das Gut zu und hielt ihr die Tür zum Wohnbereich auf. Rotan sah sich 
um, versuchte diese Zimmer mit den Augen einer Fremden zu betrachten. Die Teller mit Essensresten 
standen immer noch auf dem Tisch. Seit vor wenigen Tagen die Wehen einsetzten. Die Blumen, die 
Illiana noch vor der Mahlzeit in eine tönerne Vase stellte, begannen zu verwelken. Sogar die bunte 
Decke, die er ihr damals reichte, weil sie während der Schwangerschaft so schnell fror, hing noch 
über dem Stuhl.\\
Rotan wühlte in einer Schublade und reichte der jungen Frau ein kleines Gefäß mit einer 
entzündungshemmenden Salbe. Illiana hatte sie hergestellt, nach einem Rezept ihres Großvaters.\\
``Dort ist sauberes Wasser'', erklärte Rotan und deutete anschließend auf den Schrank. ``Und da... 
kannst du dir frische Kleidung herausholen.'' Er würde es nicht schaffen, selbst in Illianas 
Kleidung zu wühlen. Ihren Duft noch einmal zu riechen. Lavay sah sich immer noch wortlos um, als 
Rotan sich ruckartig abwandte und den Wohnraum fluchtartig verließ. Er war nie ein guter Gastgeber 
gewesen, das hatte Illiana übernommen. Eigentlich war Rotan stets ein wortkarger Mensch gewesen, 
außer er hatte über Pferde reden können. Und zu diesen trugen seine Füße ihn nun. Der Stall nahm 
die größte Innenfläche des Gebäudes ein. Diese vielen Boxen nun leer zu sehen, war fast noch 
schmerzlicher als das Grab seiner Familie. Seit Rotan sich erinnern konnte, hatte es nach Mist, 
Pferd und Heu gerochen. Sein Knecht hatte noch den letzten Strohhalm ausgefegt. Das Zaumzeug war 
mit den Tieren verkauft worden, aber einige Ballen Stroh und Heu türmten sich noch im Lager auf. An 
den drei sonnigsten Boxen waren die letzten Tiere seiner Zucht übrig geblieben. Rotan trat zu den 
alten Grauen und klopfte ihn sanft den Hals. Das Tier hatte er seinem Sohn geschenkt, aber auf die 
lange Reise nach Na'Rash hatte er sich für die Mitfahrt in einer Kutsche entschlossen. Der Wallach 
war alt und das Leben in einer Stadt wollten sie ihm nicht mehr zumuten. Er hat immerhin viele 
Jahre treu gedient und seinen Sohn sicher getragen, kaum das er laufen konnte.\\
\textit{Ob er es schon weiß? Ob der Bote ihm schon die Nachricht überbracht hat, dass er nun 
Halbwaise ist?}, grübelte Rotan. Aber es war zu früh, selbst wenn der Knecht die Nachricht gleich 
fortgeschickt hätte. Vielleicht war es besser so. Er genoss dort den kostenlosen Unterricht der 
Priester. Die Frage war nur, wie lange noch. Auch wenn die Lehre finanziert wurde, waren die 
Unterkünfte in Na'Rash teuer und Rotan nun in seiner Trauer verarmt. Er wollte nicht darüber 
nachdenken, wie sein Sohn toben würde, wenn er das Ausmaß erfuhr. Immerhin war er die 15 Jahre 
seines Lebens mit dem Wissen aufgewachsen, dass als dies hier - das Gut, das Land, die Pferde - 
einst ihm gehören würden. Jetzt könnte er nur noch als Knecht bei einem anderen Gut oder beim Adel 
enden. Rotan schüttelte seufzend den Kopf und kehrte zurück in das Wohnhaus. Vorsichtig spähte er 
hinein, um die Frau nicht in einer unschicklichen Situation zu überraschen. Immerhin saß er selbst 
im Sattel, ehe er laufen konnte. Er wusste welche Wunden das hinterlassen konnte. Und an welchen 
Stellen. Lavay lag schlafend auf dem Bett. Zu seiner Überraschung trug sie Hose und Hemd welches 
seinen Sohn gehörte und keines der Kleider seiner verstorbenen Frau. Er ließ sie Schlafen und 
machte sich das erste Mal seit Tagen Gedanken, was er kochen könnte. Immerhin hatte er jetzt einen 
Gast.\\