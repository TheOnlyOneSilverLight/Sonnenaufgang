\chapter{Die Klauen des Löwen}

Der rote Himmel des Sonnenuntergangs verblasste langsam. Mit der zunehmenden Dunkelheit erwachten 
die Tiere des Waldes. Der kühle Wind trug deren Stimmen zu Lavay. Sie schauderte, als das schaurige 
Heulen eines Wolfes erklang. Lavay legte den Kopf in den Nacken und erhaschte einen flüchtigen 
Blick, auf den blassen Mond, ehe dieser von dunklen Wolken verschluckt wurde. Sie schloss die Augen 
und versuchte ihren rasenden Herzschlag zu beruhigen.\\
Sie war ein Stadtkind, hatte in den ersten Jahren ihres Lebens nicht anderes als ihre ärmliches 
Viertel gekannt. Was war es schon für ein großer Schritt gewesen, in die besser gestellten 
Siedlungen der Stadt zu treten? Und nun hatte sie selbst die Straßen verlassen und für sie somit 
auch jede Möglichkeit sich zu orientieren. Seit Tagen irrte sie durch das Nirgendwo. Anfangs war 
ihr die Idee besser vorgekommen, im Wald Schutz zu suchen. Mittlerweile sah sie so zerzaust und 
dreckig aus, dass jeder Reisender auf der Straße sie verjagen würde. Ihre Finger glitten nach Halt 
suchend über die raue Rinde des Baumes. \textit{Werde ich je wieder auf eine Straße stoßen und oder 
bis an mein Lebensende hier herum irren?}\\
Ohne das Licht des Mondes war es stockdunkel. Lavay dachte kurz darüber nach, einfach hier zu 
bleiben, anstatt sich wie blind durch den Wald zu tasten. Aber da erklang wieder das Heulen des 
Wolfes. Schon im die Mittagszeit herum hatte sie diesen Laut gehört. Mal näher, mal weiter in der 
Ferne. Stets hatte sie sich in eine andere Richtung gewandt, nur fort von dem Ruf des Jägers. Ein 
weiterer Grund, wieso sie nun keine Ahnung mehr hatte, wo sie sich befand.\\ 
Lavay tastete nach dem schmalen Messer an ihrem Gürtel. Mit der kleinen Klinge könnte sie es niemals 
gegen die Jäger des Waldes aufnehmen, aber trotzdem beruhigte die junge Frau sich etwas. Während 
sie einen Fuß vor dem anderen setzte, lauschte sie den Geräuschen in ihrer Umgebung. Die Tiere 
waren nicht das Einzige, was sie fürchten musste. Nach einige Minuten, in denen sie sich fühlte wie 
ein junges Kaninchen, dass jederzeit damit rechnete gleich vom Bussard gepackt zu werden, sah sie 
tatsächlich wie die Finsternis des Waldes sich auftat. Lavay grinste und freute sich schon darauf, 
sich an einem trockenen Plätzchen zum Schlafen niederzulegen, als jemand ihr eine Hand über den Mund 
legte. Sie stank nach Dreck und Schweiß. Vor Schreck biss sich Lavay die Zunge blutig und der 
metallische Geschmack breitete sich in ihrem Mundraum aus.\\
„Na, was suchst du nachts im dunklen, dunklen Wald?", spottete der Mann.\\
Sie erkannte die Stimme gleich, war sie doch eine derer, vor denen sie seit Tagen floh - Karkos 
Wache. Ihre Tat war selbst in den abgelegenen Dörfern bekannt geworden. Geflüster, dass eine 
saleicanische Spionin es auf die engsten Berater des kasirischen Königs abgesehen hatte. Geflüster 
über Krieg. Junge Männer, die sich rekrutieren ließen um es den saleicanischen Schweinen endlich zu 
zeigen. Die Lieder von Säufern über die zahlreich gewonnenen Schlachten. Und immer waren sie Lavay 
eicht auf der Spur gewesen. Und jetzt stand er hinter ihr.\\
Das Selbstbewusstsein, welches Lavay sich hartnäckig eingeredet hatte, zersprang zu abertausenden 
Scherben. Sie konnte es förmlich in ihren Ohren klingeln hören, wie die Splitter übereinander zu 
Boden fielen und sich wie ein gefährlich glitzerndes Meer zu ihren Füßen ausbreiteten. Lavay wurde 
schmerzlich bewusst, wie schwach und klein und dünn sie doch war. In einer Geste der Verzweiflung 
bemühte sie sich, ihren Arm aus der Klammer des Mannes zu entwinden, doch er merkte vermutlich kaum 
mehr als die Anspannung ihrer Muskeln und wie diese zu zittern begannen.\\
Am liebsten hätte sie ihm Beleidigungen entgegen geschleudert. Ihn als räudigen Hund und dreckiges 
Schwein betitelt, aber sie hatte schon damit zu kämpfen, ihre Lungen mit Luft zu füllen. Wie muss 
es Imur ergangen sein, als das Gewicht seines eigenen Körpers ihm die Luft nahm? Als der Strick 
sich fest in seine Haut grub?\\
``Du kleine saleicanische Schlampe'', knurrte er und zerrte an ihr: ``Weißt du eigentlich, wie 
sehr du mich genervt hast? Wart ab, das ist nur ein Vorgeschmack auf das, was die Folterer des 
Königs mit dir anstellen werden!''\\
Lavay starrte mir offenen Augen in die Dunkelheit und ihr Widerstand war nur noch das Zittern ihrer 
Glieder. Sie ahnte was er vorhatte. Und sie wusste, dass sie keine Chance hatte, sich zu retten. Wie 
betäubt nahm sie wahr, wie er seinen Griff veränderte. Das zerreißende Geräusch ihrer Kleidung 
hallte einem Echo gleich in ihren Ohren wieder. Sein Griff war grob und schmerzhaft. Er trat ihr die 
Beine weg und ließ sie auf den Waldboden fallen. Dort blieb sie liegen, denn kurz darauf war er 
schon über ihr. Ein Wolf, der seine Zähne um ihre Kehle schloss, wäre ihr willkommener gewesen.\\
Lavay hatte nicht die Kraft um ihre Augen zu schließen. Aber die Grausamkeit von dem, was gerade 
geschah, nahm sie mit fort. Ihre blauen Augen fixierten einen einzelnen Stern am Nachthimmel, den 
sie durch eine kleine Lücke in den Kronen des Bäume erspähen konnte. Sie sah Imur vor sich, wie er 
ihr lachend durchs Haar strich. Wie er mit dem Wind sacht hin und her schwang. Wie die Krähen 
sich auf ihm nieder ließen und zu picken begannen. Ihre Mutter, wie sie weinend in den Armen ihrer 
Tochter lag. Ihr Lächeln, dass zumindest Lavays Welt erhellen konnte. Und ihre Stimme, die leise 
sang: \textit{Leg dich nieder, mein Kind\\
Flieg wie ein Vogel im Wind.\\
Blicke in den Himmel, strecke die Arme aus,\\
Erreiche die Sterne und tanze auf dem Mond.\\
Und Morgen, wenn die Sonne erwacht,\\
wird wieder gescherzt und gelacht.\\ 
Leg dich nieder, mein Kind.\\
Flieg wie ein Vogel im Wind.\\
Flieg wie ein Vogel im Wind...}\\
Tränen liefen Lavay über ihre Wangen. Ein Schluchzen entwich ihr, als der Schmerz stärker wurde. 
Oder wurde sie wacher?\\
Imur hatte schon die andere Welt gesehen, als sie mit Haska ankam. Er ist mit dem Gedanken 
gestorben, dass keiner von den beiden Menschen, die ihm am wichtigsten waren, ihn hängen sehen 
würden. Was war besser? In die Augen von Fremden sehen, während man Maras Gnade empfing? Oder die 
Liebenden weinen sehen? Lavay wandte den Kopf und blickte in ein Augenpaar. Einen winzigen Moment, 
von Hoffnung getragen, dachte sie wahrhaftig, dass Imur dort saß. Aber die Augen waren gelb.\\
Der Wolf hatte sie gefunden. Sah sie Bedauern in seinen Augen? Weil ein anderer Jäger vor ihm die 
Beute erwischte? Das Tier wandte sich ruckartig ab und verschmolz wieder im Schatten des Waldes.\\
Hass entflammte, als hätte ein Blitz in ihrem Herzen eingeschlagen. Sie starrte dem Mann ins 
Gesicht. Seine Bewegungen wurden immer ruckartiger, sein Gesicht verzog sich 
und die Augen kniff er zusammen, während er seinen stinkenden Atem keuchend ausstieß. Es war Lavay, 
als würde jemand anderes ihre Hand ergreifen, sie ziehen und zu dem Gürtel führen. Ihre Finger 
schlossen sich um den Griff, ohne dass sie selbst dazu in Stande war, die nötige Kraft für diese 
Tat aufzubringen. Und dann, in dem Moment als sich ein zufriedenes Grinsen auf seinem Gesicht 
abzuzeichnen begann, stieß sie zu. Die Klinge tauchte tief in seinen Bauch ein und kraftlos fiel 
ihre Hand zurück ins Laub. Die Wache bäumte sich auf, stieß das Mädchen von sich und rollte sich so 
von ihr herunter. Er schrie und riss den Dolch aus sich. Blut sprudelte aus der Wunde. Das 
Messer schwebte drohend über ihr und fiel... Dumpf war der Ton, als sie im Laub aufkam. Ebenso 
dumpf, aber etwas lauter, als ihr Vergewaltiger nieder sank und sich vor Schmerz wimmernd 
krümmte. Lavay sah ihm zu, wie sein Leben schwand. Zitternd, aber unfähig etwas zu empfinden, 
schlang sie ihre Arme um die Knie und wartete.\\
\textit{Sieht man es, wenn die Seele entweicht? Sieht man den Todesgott, der seine Arme um einen 
legt?}\\
Sie sah nichts, außer wie sein Blick glasig wurde und in die andere Welt sah. \textit{Aber 
vielleicht bin ich diejenige, die gerade gestorben ist.}\\
Und Lavay wünschte sich so sehr, dass dieser Gedanke Wirklichkeit wäre.\\
Die Gedanken im ihrem Kopf befahlen ihr aufzustehen, wegzulaufen. Er war nicht allein. Andere 
würden kommen. Bald! Aber sie schaffte es nicht aufzustehen. Bis ein Knurren anschwoll. Der Wolf 
schob sich beinahe Lautlos durch das Unterholz, duckte sich drohend zum Sprung und zeigte seine 
Zähne. \textbf{Du wirst nicht sterben,} wisperten die Baumkronen.\\
Der Wolf sprang ab, riss sein Maul auf und schnappte in die Luft. Lavay rappelte sich auf und 
rannte, so schnell ihre Muskeln es noch schafften. Sie spürte die Erschöpfung nicht mehr, kein 
Schmerz und keine Wunde. Nur Angst. \textbf{Nicht heute Nacht!}\\


Der Weg war nicht mehr als ein Trampelpfad und wohin er führte, darüber dachte Lavay mittlerweile 
nicht mehr nach. Sie konnte sich gerade gut vorstellen, wie ihre Mutter sich in den letzten Tagen 
vor ihrem Tod gefühlt haben musste. Trotz der Schwäche die ihren Körper wie Gewichte nieder 
drückte, setzte sie einen Fuß vor den Anderen. Manchmal schwankte sie leicht, hielt den Blick aber 
stur auf das bisschen blanke Erde. Lavay hatte nie viele Nahrungsmittel gehabt, aber Schlaf. Sie 
konnte nachts in der Wildnis kaum zur Ruhe finden, egal wie erschöpft sie war. Kaum schloss Lavay 
die Augen, meinte sie den Atem eines Wolfs zu spüren, die Stiche von Insekten, das Schleichen der 
Raubtiere. In Dörfer ging sie auch nicht mehr, denn sie konnte sich nicht mehr dazu 
überwinden. Es wurde immer noch nach ihr gesucht, dem war sie sich sicher.\\
So taumelte sie also schon seit Tagen, die sie nicht zu zählen vermochte, in den südlichen Gefilden 
des Landes umher. Manchmal fand sie Beeren oder Früchte des Herbstes, einmal sogar einen zappelnden 
Hasen in einer Falle. Es tat ihr im Herzen weh, das Tier zu töten, aber der Hunger zwang sie zu 
dieser Tat. Sie weinte bittere Tränen, als sie das Blut des Tieres an ihren Händen sah. Das erste 
Mal, dass sie selbst ein Tier getötet hatte... und es fiel ihr so viel schwerer als das Leben eines 
Menschen zu nehmen.\\
Die Erschöpfung ließ keine Panik zu, als das Wiehern eines Pferdes erklang. Stattdessen sah Lavay 
sich nur um und erblickte auf der nächsten Hügelkuppe fünf Reiter. Die junge Frau blieb stehen, 
denn es gab in dieser Wiese nichts, was ihr hätte Schutz geben können. Die Wache hatte etwas vom 
König gesagt. Vielleicht würden sie sie erstmal nur festnehmen. Bis zur Hauptstadt war es ein 
weiter weg. Vielleicht würde der König ihr glauben. Oder vielleicht war eine Flucht möglich, bevor 
die Folterer sie in die Finger bekamen. Der letzte Keim der Hoffnung wurde jedoch zerstört, als die 
Reiter nähe kamen. Sie trugen nicht die offiziellen blauen Uniformen Kasirs. Kopfgeldjäger, 
Söldner. Man brauchte keine gerechte Behandlung von ihnen zu erwarten.\\ 
Lavay hätte schon fliegen muss, um den schnellen Pferden der Männer zu entkommen. Sie 
galoppierten auf die junge Frau zu und umkreisten sie lachend. Lavay versuchte ihnen in die 
Gesichter zu blicken, aber sie bewegten sich zu schnell und hektisch. Plötzlich fand sie sich auf 
dem Boden wieder, hatte es gerade noch geschafft sich mit den Händen und Knien aufzufangen. Einer 
der Männer hatte den Knauf seines Schwertes gegen ihre Schulter gestoßen und der Schmerz 
explodierte regelrecht in ihrem Körper. Ihr wurde kurz schwarz vor Augen und als sie Anstalten 
machte, sich wieder aufzurappeln, schwebte eine Klinge über ihrem Gesicht. Aus weit aufgerissenen 
Augen starrte sie zu dem Söldner hinauf. Er grinste. Seine Kameraden blieben auf ihren Pferden 
sitzen und lachten.\\
„Wie ein Hase ist sie gerannt“, spottete einer. \\
„Du saleicanische Schlampe! Dachtest du wirklich, du könntest fliehen?“, schrie derjenige, dessen 
Schwert nur wenige Zentimeter von Lavays Haut entfernt war.\\
Lavay konnte in diesem Moment überhaupt nicht denken. Sie beobachtete nur das Schwert über sich und 
zuckte vor der Berührung des kalten Stahls zurück. \textit{Was schmerzt mehr? Der Strick? Das 
Fieber? Die Klinge? Oh Mara... ist es wahr? Ist der Tod Erlösung? Geleitet mich der Wind ins 
Jenseits? Nein. Ich bin eine Mörderin.}\\
Lavay zitterte und kniff die Augen fest zusammen. Ein stummes Flehen, dass die Männer sie hier und 
jetzt töten würden. Wäre das nicht perfekt? Ihr Blut würde das Gras nähren und der prasselnde 
Herbstregen würde es fortschwämmen. Die Stürme würden ihre Seele mit sich nehmen, bis zum Mond, bis 
zu den Sternen. Das alles klang erfüllender als das Ungewisse, was ansonsten geschehen würde. Nein, 
es war nicht ungewiss. Die Männer würden das gleiche tun was Maschas Bruder bereits getan hatte. Sie 
konnte sich die Folter denken, die Qualen einer einsamen, finsteren Kellerzelle. Erniedrigende 
Gerichtsverhandlungen, deren Ergebnis bereits schon fest stand. Der König würde sie und alles was 
sie ausmachte zerfetzten und damit bei der nächsten Abendgesellschaft prahlen.\\
\textit{Flieg wie ein Vogel im Wind...}\\
Mehrere Pferde wieherten schrill und die Männer brüllten einander etwas zu. Lavay schlug die Augen 
 auf und sah, wie dem Söldner über ihr das Schwert aus der Hand glitt. Es fiel dich neben ihr zu 
Boden. Er drehte sich langsam um und Lavay zählte drei Pfeile in seinem Rücken. Gurgelnd ging er in 
die Knie und sackte zusammen. Seine Kameraden griffen nach ihren Schwertern. Ein Dutzend Reiter 
trabten fast schon gelassen auf sie zu. Ein goldener Löwe der sich hoch zur Sonne streckte, prang 
auf ihren roten Gewändern. Das Wappen hatte Lavay noch nie in ihrem Leben gesehen, aber sie kannte 
es trotzdem. Jedes Kind in Kasir kannte den Löwen, der die Sonne herausforderte. Jedes Kind kannte 
die Geschichten der barbarischen Krieger Saleicas, welche Dörfer plünderten und Menschen umbrachten, 
um damit ihren Gott zu ehren. \textit{Unmöglich.}\\
``Die Schlampe hat Verstärkung!'', fluchte einer der Söldner und riss an den Zügeln.\\
Seine Kameraden brüllten ihre Pferde an. Hufe donnerten dich an Lavay vorbei. Sie hörte den Anführer 
der Gruppe etwas rufen, konnte es aber nicht einordnen. Ein Pferd trampelte an ihr vorbei, 
während sein Reiter es zur mit Schlägen zur Flucht antrieb. Benommen blieb die Gejagte nur stehen 
und sah zu, wie die Löwen über die Männer herfielen. Ihre Klauen zerfetzen Muskeln und Sehnen, ihre 
Kiefer durchtrennten Gliedmaßen und ihre Pranken brachen Knochen. Lavay betrachtete das Blut, 
welches das Herz aus einer Wunden pumpte und kam nicht umhin, dieses tiefe Rot als schön zu 
empfinden. Der Lärm des Kampfes, den sie als das Brüllen von Löwen wahrnahm, trat in den Hintergrund 
bei diesem Anblick. Irgendwann dann spürte sie eine schwere Hand auf ihrer Schultern. Sie blickte 
auf. \textit{Die Löwen rasieren sich...}, stellte sie verwundert den offensichtlichsten Unterschied 
zwischen den kasirischen und saleicanischen Männern fest. Der Soldat sagte etwas in seiner ihr 
fremden Sprache und sie sah ihn nur stumm an.\\
Ein weiterer Löwe lenkte sein geschecktes Tier näher. Lavay erkannte den nachlässig geflochtenen 
Zopf und runzelte die Stirn. Offensichtlich gab es auch weibliche Löwen in der saleicanischen 
Armee. Die Frau musterte sie abschätzend und sprach in undeutlichem Kasirisch: ``Er hat dich 
Saleicanerin genannt.''\\
Lavay war sprachlos und schaffte es nur den Kopf zu schütteln. Zu spät kam ihr der Gedanke, dass 
das vielleicht ein Fehler war. Aber wie hätte sie sich ohne Sprachkenntnisse als eine Bürgerin 
Saleicas ausgeben sollen?\\
Die Soldaten begannen zu reden und auf Lavay wirkte es so, als ob die Unterhaltung größtenteils aus 
Scherzen bestand. Der Mann stieg wieder auf sein Reittier und warf der Frau ein Seil zu. 
Die Kameraden lachten grölend, während die Frau es geschickt auffing und dann zu Lavay ging und ihr 
die Hände fesselte. Lavay ließ es widerstandslos geschehen und fragte stattdessen: ``Warum seid ihr 
hier?''\\
``Krieg'', antwortete die Soldatin einsilbig: ``Wir wollten Spaß haben. Hatten wir. Wir wollten 
plündern. Haben wir nicht. Also bist du jetzt meine Beute. Es ist ehrlos ohne Beute heim zu 
kehren.''\\
``Krieg?'', wiederholte Lavay überrascht. Sie hatte gehört, dass die kasirische Armee dringend 
Rekruten suchte, aber dass der Krieg bereits ausgebrochen war, wusste sie nicht.\\
Die Soldatin zeigte grinsend ihre Zähne. ``Mein Name ist Sokra A'Rell und du bist nun mein 
Besitz!''\\

Sokra schwang sich wieder in den Sattel und ihr Reittier machte einen überraschten Satz, als sie 
ihm brüsk die Fersen in die Flanken stieß. Das Pferd warf den Kopf hoch, die Mähne flog durch die 
Luft und es schnaufte entrüstet. Triumphierend warf Sokra noch einen letzten Blick auf das 
Schlachtfeld, welches sie und ihre Kameraden hinterlassen hatten, und reckte die Faust in den 
Himmel. Der saleicanische Siegesgruß, begleitet von ihren lauten Worten: ``Jeder, der unser Brüllen 
hört!''\\
Die anderen Männer und Frauen fielen mit ein: ``Alles was das Sonnenlicht berührt!''\\
Helix und Farun wendeten ihre Pferde und galoppierten mit einem lauten Jolen über die Leichen 
hinweg. Einige andere taten es ihnen noch nach, während Sokra einer der Jüngsten in der Truppe mit 
einem Kopfnicken zu verstehen gab, dass die Pferde der Toten eingesammelt werden sollten. 
Zumindest, solange sie noch brauchbar waren. Sie grinste schief, während sie ihre Kameraden 
beobachtete. Obwohl sie offiziell auf dem selben Rang stand wie sie, war sie bei dieser eher 
inoffiziellen Mission die Anführerin.\\
\textit{Verdammt, es ist Krieg!}, dachte sie und lachte still. \textit{Und so sollte ein Krieg auch 
aussehen!}
In den letzten Wochen lag das Gefühl des nahenden Krieges in den Garnisonen in der Luft. Und 
trotzdem saßen alle nur herum, schärften Klingen, polierten Rüstungen, flickten Zaumzeug. Sokra 
wusste ganz genau, dass sie und ihre Freunde nicht die einzigen waren, die sich fort stahlen um 
schon einmal mit dem Krieg zu beginnen, während die Offiziere und der Adel noch an den gedeckten 
Tischen saßen und leise plapperten! Krieg ist laut, blutig und wild!\\
Sokra sah ruckartig zu der schmächtigen Gestalt auf dem viel zu großen Pferd neben sich. Der junge 
Priester war ihrer Einheit erst seit wenigen Wochen zugeteilt, aber er war es gewesen, der ihnen 
erzählte, dass der Krieg sicher war. Osyma sehnt sich nach Heldentaten, nach sterben zu seinen 
Ehren. Blut, dass für den Allmächtigen vergossen wird. Todesschreie, die in seinem Namen erklingen. 
Sokra war es egal ob ein Gott es befahl. Sie war Soldatin, eine Kriegerin Saleicas. Seit 
mehr als 15 Jahren diente sie im Heer und eine richtige Schlacht hatte sie noch nie erlebt. Sie 
hätte alles gegeben um in die Kolonien versetzt zu werden, aber sie war in der saleicanischen 
Grafschaft Merandila geboren und verpflichtet die Grenze gen Norden zu schützen.\\
``Na?'', fragte sie den Priester, dessen Tätowierungen sich auf seine Arme beschränkten: ``Ist 
Osyma glücklich?''\\
Jetzt, da sie ihn direkt angesprochen hatte, schien er seinen Mut zusammen zu nehmen und mit 
trotzig erhobenem Kinn erwiderte er: ``Eure Taten waren ehrvoll. Aber nichts von Dauer. Kein Sieg 
über Ländereien oder kasirische Lords. Lediglich ein Anfang.''
``Immerhin haben wir angefangen!'', murrte Sokra und wendete wieder ihr Pferd um einen Überblick 
über ihre Kameraden zu erhalten. Skeptisch musterte sie die junge Frau, die reglos da stand und mit 
erschrockenen Augen das Spiel der Reiter betrachtete. Sie verstand vermutlich kein einziges Wort der 
Unterhaltungen. Sokra war sich nicht sicher, ob die Toten die Frau wirklich als Saleicaner 
bezeichnet hatten oder doch nur ihre Truppe schon gesehen hatte. Aber da die Fremde offensichtlich 
kein Wort verstand, konnte es ja nicht zutreffen.\\
Sokra A'Rell hob die Finger an die Lippen und stieß einen schrillen Pfiff aus. Nicht nur ihr 
Pferd spitzte die Ohren, sondern auch ihre Kameraden sammelten sich wieder. Die Pferde hinterließen 
blutige Hufabdrücke im schlammigen Grund. Ein lässiges Winken deutete den Rückzug nach Süden an.\\



Sie ritten fast zwei Tage durch, nur mit wenige Stunden Rast. Jeder einzelne von den Soldaten war 
ein geübter Reiter und keiner von ihnen hatte ein Problem damit, in der langsamen Gangart im Sattel 
zu schlafen. Sie waren immerhin eine der elitären Einheiten der saleicanischen Armee. Die 
Löwen-Reiter waren es, die die Kolonien eroberten. Sie waren es, die vor so vielen Generationen 
Saleica zu dem machten, was es nun war. Wie ein Sturm ergossen sich die Reiter Saleicas mit 
Säbeln und Lanzen über die Feinde. Die Hufe der Pferde machten keinen Halt vor am Boden 
Liegenden. Schnell kamen die Klingen über Männer und Frauen und blutig schimmerten sie am Ende 
der Schlacht. Es gab in der Geschichte Saleicas nur wenige Niederlagen. Und die hatten ausnahmslos 
alle mit Kasir zu tun. Sokra warf wie so oft in den Stunden des Ritts einen Blick hinter sich, in 
den Norden. Sie wusste, wenn sie weit genug der Kälte entgegen reiten würde, würde sie an die 
Saronne gelangen. Den mächtigsten Fluss in Kasir. Und noch weiter, dann würde sie die Bergkette der 
Amshoh am Horizont erblicken. Aber Karten zu studieren war kein Vergleich dazu, jemals diese Berge 
wirklich zu sehen. \textit{Osyma möge uns leiten}, dachte sie mit verbissenem Stolz.\\
Vor Hundert Jahren hatten es Saleicaner zuletzt versucht, auch den Norden zu erobern. Ein geeintes 
Land, das war König Esmers Ziel gewesen. Seine Reiter waren als Die Sensen bekannt. Sie mähten die 
Kasira nieder, wie reifes Getreide. Bis sie an den steinernen Festungen scheiterten. \textit{Aber 
wir haben dazu gelernt!}\\
Sokra dachte an die Belagerungswaffen, die täglich auf dem Gelände getestet wurden, jedoch 
konnte sie sich nicht vorstellen, wie die Baumeister die schweren Ungetüme in den Norden bringen 
wollten. Sie kraulte ihrem Schecken neckend den Hals. Ihrer Meinung nach würde niemals eine 
Belagerungswaffe einen guten Reiter ersetzen.\\
``Was hast du mit der Kasira vor?'', fragte Joshk und kam näher.\\
Die Soldatin zuckte mit den Schultern. ``Mal sehen.''\\
``Du kannst sie nicht mit in die Garnison bringen. Unser Auftrag war die Grenze zu überprüfen... 
nicht zu überqueren.''\\
``Und wer soll den Vorgesetzten sagen, dass wir nicht der Grenze gefolgt sind?''\\
Er wirkte unzufrieden mit ihrer Antwort. ``Im wahrscheinlichsten Fall der Priester.''\\
``Ich kann sie ja im Tempel absetzen. Vielleicht wollen sie mal wieder was opfern?'', spottete 
Sokra.\\
Joshk würdigte dem Witz nur eine Grimasse und ließ sein Reittier in eine flottere Gangart fallen.
Sokra war bewusst, dass man ihnen gehörig Ärger machen konnte, aber streng genommen hatten sie gegen 
keinen Befehl verstoßen. Außerdem, wer sollte sie richten? Die Offiziere waren abhängig vom König. 
Und der König könnte doch niemals seine Soldaten verurteilen, die für seine Krone in 
den Kampf gezogen sind! Und die Priester würden sich doch ins eigene Fleisch schneiden, wenn sie 
Soldaten dafür bestraften, dass sie dem Allmächtigen ehrten.\\
Die Soldatin biss sich auf die Zunge. Sie liebte ihre Heimat. Und sie glaubte fest daran, dass 
Saleica etwas höheres, größeres war als alle anderen Länder. Und mit ihm auch dessen Volk. 
Das Reich ist seit je her gewachsen, wieso sollte diese junge Frau nicht auch eine Saleicanerin 
werden können? Die kasirische Bastard hatte es sogar geschrien.\\
``Ich weiß, was ich mit dir mache'', murmelte Sokra in ihrer Muttersprache.\\
