\chapter{Vergewaltigung, Unterschlupf im Dorf}

Der rote Himmel des Sonnenuntergangs verblasste langsam. Mit der zunehmenden Dunkelheit erwachten 
die Tiere des Waldes. Der kühle Wind trug deren Stimmen zu Lavay. Sie schauderte, als das schaurige 
Heulen eines Wolfes erklang. Lavay legte den Kopf in den Nacken und erhaschte einen flüchtigen 
Blick, auf den blassen Mond, ehe dieser von dunklen Wolken verschluckt wurde. Sie schloss die Augen 
und versuchte ihren rasenden Herzschlag zu beruhigen.\\
Sie war ein Stadtkind, hatte in den ersten Jahren ihres Lebens nicht anderes als ihre ärmliches 
Viertel gekannt. Was war es schon für ein großer Schritt gewesen, in die besser gestellten 
Siedlungen der Stadt zu treten? Und nun hatte sie selbst die Straßen verlassen und für sie somit 
auch jede Möglichkeit sich zu orientieren. Anfangs war ihr die Idee besser vorgekommen, im Wald 
Schutz zu suchen. Ihre Finger glitten nach Halt suchend über die raue Rinde des Baumes. 
\textit{Werde ich je wieder auf eine Straße stoßen und oder bis an mein Lebensende hier herum 
irren?}\\
Ohne das Licht des Mondes war es stockdunkel. Lavay dachte kurz darüber nach, einfach hier zu 
bleiben, anstatt sich wie blind durch den Wald zu tasten. Aber da erklang wieder das Heulen des 
Wolfes. Schon im die Mittagszeit herum hatte sie diesen Laut gehört. Mal näher, mal weiter in der 
Ferne. Stets hatte sie sich in eine andere Richtung gewandt, nur fort von dem Ruf des Jägers. Ein 
weiterer Grund, wieso sie nun keine Ahnung mehr hatte, wo sie sich befand.\\ 
Lavay tastete nach dem schmalen Messer an ihrem Gürtel. Mit der kleinen Klinge könnte sie es niemals 
gegen die Jäger des Waldes aufnehmen, aber trotzdem beruhigte die junge Frau sich etwas. Während 
sie einen Fuß vor dem anderen setzte, lauschte sie den Geräuschen in ihrer Umgebung. Die Tiere 
waren nicht das Einzige, was sie fürchten musste. Nach einige Minuten, in denen sie sich fühlte wie 
ein junges Kaninchen, dass jederzeit damit rechnete gleich vom Bussard gepackt zu werden, sah sie 
tatsächlich wie die Finsternis des Waldes sich auftat. Lavay grinste und freute sich schon darauf, 
sich an einem trockenen Plätzchen zum Schlafen niederzulegen, als jemand ihr eine Hand über den Mund 
legte. Sie stank nach Dreck und Schweiß. Vor Schreck biss sich Lavay die Zunge blutig und der 
metallische Geschmack breitete sich in ihrem Mundraum aus.\\
„Na, was suchst du nachts im dunklen, dunklen Wald?", spottete der Mann.\\
Sie erkannte die Stimme gleich, war sie doch eine derer, vor denen sie seit Tagen floh. Onar, die 
Wache des eingebildeten Adeligen. Das war das Problem damit, wenn man dem Adel bestahl. Er war 
hartnäckiger. Ein einfacher Mann hätte sich niemals die Mühe gemacht eine kleine Diebin durch das 
Land zu jagen. Ein einfacher Mann hätte auch nicht die finanziellen Mittel dafür besessen.\\
Das Selbstbewusstsein, welches sie sich hartnäckig eingeredet hatte, zersprang zu abertausenden 
Scherben. Sie konnte es förmlich in ihren Ohren klingeln hören, wie die Splitter übereinander zu 
Boden fielen und sich wie ein gefährlich glitzerndes Meer zu ihren Füßen ausbreiteten. Lavay wurde 
schmerzlich bewusst, wie schwach und klein und dünn sie doch war. In einer Geste der Verzweiflung 
bemühte sie sich, ihren Arm aus der Klammer des Mannes zu entwinden, doch er merkte vermutlich kaum 
mehr als die Anspannung ihrer Muskeln und wie diese zu zittern begannen.\\
Am liebsten hätte sie ihm Beleidigungen entgegen geschleudert. Ihn als räudigen Hund und dreckiges 
Schwein betitelt, aber sie hatte schon damit zu kämpfen, ihre Lungen mit Luft zu füllen. Wie muss 
es Imur ergangen sein, als das Gewicht seines eigenen Körpers ihm die Luft nahm?\\
Onars raues Lachen ertönte ganz nah an ihrem Ohr. "Weißt du? Eigentlich kann er doch gar nichts mit 
dir anfangen. Er will diese Frechheit nur nicht auf sich sitzen lassen. Kann ich verstehen. Dich 
ungeschoren laufen zu lassen, würde den Ruf meines Herren schaden. Auf diese Verbrechen folgt der 
Tod, dass verstehst du ja, stimmt's? Man legt sich nicht mit dem kasirischen Adel an."\\
Lavay starrte mir offenen Augen in die Dunkelheit und ihr Widerstand war nur noch das Zittern ihrer 
Glieder. Ja, sie verstand es. Und sie ahnte, was er vorhatte. Und sie wusste, dass sie keine Chance 
hatte, sich zu retten. Wie betäubt nahm sie wahr, wie er seinen Griff veränderte. Das zerreißende 
Geräusch ihrer Kleidung hallte einem Echo gleich in ihren Ohren wieder. Sein Griff war grob und 
schmerzhaft. Er trat ihr die Beine weg und ließ sie auf den Waldboden fallen. Dort blieb sie liegen, 
denn kurz darauf war er schon über ihr. Ein Wolf, der seine Zähne um ihre Kehle schloss, wäre ihr 
willkommener gewesen.\\
Lavay hatte nicht die Kraft um ihre Augen zu schließen. Aber die Grausamkeit von dem, was gerade 
geschah, nahm sie mit fort. Ihre blauen Augen fixierten einen einzelnen Stern am Nachthimmel, den 
sie durch eine kleine Lücke in den Kronen des Bäume erspähen konnte. Sie sah Imur vor sich, wie er 
ihr lachend durchs Haar strich. Wie er mit dem Wind sacht hin und her schwang. Wie die Krähen 
sich auf ihm nieder ließen und zu picken begannen. Ihre Mutter, wie sie weinend in den Armen ihrer 
Tochter lag. Ihr Lächeln, dass zumindest Lavays Welt erhellen konnte. Und ihre Stimme, die leise 
sang: \textit{Leg dich nieder, mein Kind\\
Flieg wie ein Vogel im Wind.\\
Blicke in den Himmel, strecke die Arme aus,\\
Erreiche die Sterne und tanze auf dem Mond.\\
Und Morgen, wenn die Sonne erwacht,\\
wird wieder gescherzt und gelacht.\\ 
Leg dich nieder, mein Kind.\\
Flieg wie ein Vogel im Wind.\\
Flieg wie ein Vogel im Wind...}\\
Tränen liefen Lavay über ihre Wangen. Ein Schluchzen entwich ihr, als der Schmerz stärker wurde. 
Oder wurde sie wacher? \\
Imur hatte schon die andere Welt gesehen, als sie mit Haska ankam. Er ist mit dem Gedanken 
gestorben, dass keiner von den beiden Menschen, die ihm am wichtigsten waren, ihn hängen sehen 
würden. Was war besser? In die Augen von Fremden sehen, während man Maras Gnade empfing? Oder die 
Liebenden weinen sehen? Lavay wandte den Kopf und blickte in ein Augenpaar. Einen winzigen Moment, 
von Hoffnung getragen, dachte sie wahrhaftig, dass Imur dort saß. Aber die Augen waren Gelb.\\
Der Wolf hatte sie gefunden. Sah sie bedauern in seinen Augen? Weil ein anderer Jäger vor ihm die 
Beute erwischte? Das Tier wandte sich ruckartig ab und verschmolz wieder im Schatten des Waldes.\\
Hass entflammte, als hätte ein Blitz in ihrem Herzen eingeschlagen. Sie starrte dem Mann, 
der über ihr lag, ins Gesicht. Seine Bewegungen wurden immer ruckartiger, sein Gesicht verzog sich 
und die Augen kniff er zusammen, während er seinen stinkenden Atem keuchend ausstieß. Es war Lavay, 
als würde jemand anderes ihre Hand ergreifen, sie ziehen und zu dem Gürtel führen. Ihre Finger 
schlossen sich um den Griff, ohne dass sie selbst dazu in Stande war, die nötige Kraft für diese 
Tat aufzubringen. Und dann, in dem Moment als sich ein zufriedenes Grinsen auf seinem Gesicht 
abzuzeichnen begann, stieß sie zu. Die klinge verschwand in seinem Körper. Lavay begriff erst, als 
ihr warmes Blut über die Hände sprudelte. Oras dagegen bäumte sich auf, stieß das Mädchen von sich 
und rollte sich so von ihr herunter. Er schrie und riss den Dolch aus seiner Brust. Die Waffe hob 
sich drohend und fiel... Dumpf war der Ton, als sie im Laub aufkam. Ebenso dumpf, aber etwas lauter, 
als Oras wieder nieder sank und sich vor Schmerz wimmernd krümmte. Lavay sah ihm zu, wie sein Leben 
schwand. Zitternd, aber unfähig etwas zu empfinden, schlang sie ihre Arme um die Knie und 
wartete.\\
\textit{Sieht man es, wenn die Seele entweicht? Sieht man den Todesgott, der seine Arme um einen 
legt?}\\
Sie sah nichts, außer wie sein Blick glasig wurde und in die andere Welt saß. \textit{Aber 
vielleicht bin ich diejenige, die gerade gestorben ist.}\\
Und Lavay wünschte sich so sehr, dass dieser Gedanke Wirklichkeit wäre.\\

Die meisten Bewohner des kleinen Dorfes schienen auf dem Feld zu sein und die letzten Körner 
zu ernten, ehe der Herbst weiter vorran schritt. Lavay sah eine Frau, die ihren Sohn eine Ohrfeige 
gab. Drei Kinder scheuchten dürre Schafe durch das Dorf und hinaus auf die feuchten Wiesen. 
„Was willst du hier, Fremde?“, fragte eine dürre Frau barsch. Der Hunger des letzten Jahres 
sah man ihr deutlich an. Das brachte Lavay zum stutzen. Sie hatte gedacht, dass die Bauern weniger 
hungern mussten als die Arbeitslosen in der Stadt. Immerhin pflanzten sie alles an. Nachfragen 
brauchte sie jedoch auch nicht, sie war klug genug es sich zusammen zu reimen. Irgendwoher mussten 
alle Menschen, jenseits der Armenviertel und der Bauern, ja ihr Getreide bekommen.\\
Lavay lächelte bemüht freundlich. „Ich wollte mich erkundigen, ob einer der Leute hier mir 
vielleicht einen Platz zum Schlafen geben könnte. Nur für die nächsten paar Stunden. Ich bin bald 
wieder weg. Und ich kann zahlen.“\\
Sie würde den Armreif hergeben. Immerhin war er einer der Gründe für ihr Unglück. Die Frau musterte 
sie misstrauisch von oben bis unten. „Was macht ein junges Mädchen wie du alleine hier? Kommst du 
etwa aus Salaica, um uns auszuspionieren?“\\
Lavay sah sie überrascht an und schüttelte schnell den Kopf. \textit{Saleica? Ich kann doch 
unmöglich schon so nahe der Grenze sein... Vielleicht sollte ich doch wieder nach Norden...}\\
„Ich bin mit meinem Vater gereist. Aber wir haben einander verloren. Wir hatten ausgemacht, dass 
wir uns in der nächst größeren Stadt treffen.“\\
„Du kannst Baram fragen, ob er ein Plätzchen für dich hat. Der lässt fast jeden in seine Hütte. Aber 
du solltest bis Morgen warte, wenn du weiter wandern willst. Hier in der Nähe ist ein Wolfsrudel, 
das schon einige Schafe gerissen hat. Sie sind genauso hungrig wie wir.“\\
„Vielen Dank“, sagte Lavay und ging zu dem Haus, auf welches die Frau gezeigt hatte.\\

Baram verkaufte ihr etwas Brot und gab ihr sogar einen Schluck Milch von seiner Kuh zu trinken. 
Schlafen ließ er sie kostenlos in einer Ecke seiner Hütte. Lavay hatte nicht mit so viel 
Freundlichkeit gerechnet und wagte kaum sich zu entspannen. Sie wich aus, sobald eines der 
Familienmitgliedern ihr zu nahe kamen, selbst bei dem jüngsten Mädchen, welches auf wackeligen 
Beinen erste Laufversuche machte. Früher hätte sie Baram vielleicht gemocht. Jetzt ließ sie nicht 
zu, sicher überhaupt einer Gefühlsregung hin zu geben und lächelte nur still und freundlich vor 
sich hin. 
Zum Frühstück lud Barams Weib sie herzlich ein und es sollte wohl zufällig wirken, dass sie gleich 
zwei ihrer Söhne neben sie Platz nehmen ließ. Baram reichte ihr strahlend eine Holzschüssel voller 
Haferbrei und fragte: „Verreist du viel mit deinem Vater?“\\
„Manchmal“, log Lavay: „Wir haben Verwandte besucht, aber einander verloren.“\\
Barams Frau runzelte die Stirn. „Dann ist er nicht sehr aufmerksam.“\\
„Ich kann mir gar nicht vorstellen, jemanden wie dich aus den Augen zu verlieren“, murmelte Barams 
Sohn und wurde rot. Lavay lächelte schief und starrte in ihre Schüssel. Aber ihr entging nicht, wie 
Baram dem Jungen auffordernd zu zwinkerte.\\
„Und woher stammst du nun, Lavay?“\\
„Wir leben in der Nähe von Majaku… Aber seid meine Mutter tot ist, reist mein Vater viel… ich denke, 
unser Haus erinnert sie zu sehr an sie“, sagte Lavay und blickte betreten in ihre Schüssel. Ihre 
Augen füllten sich mit Tränen, als sie an ihre Mutter dachte. Schnell wischte sie sie fort und 
lächelte schief. Die Bauersleute sahen sie stumm an, erwiderten dann ihr unerwartetes Lächeln und 
das Essen wurde erst einmal schweigend fortgesetzt.\\
„Willst du noch was?“, fragte Barams Weib, riss Lavay die halbvolle Schüssel aus der Hand und füllte 
sie erneut.\\
 „Nein“, rief Lavay: „Das muss doch nicht sein. Ich will euch doch nicht euer gesamtes Essen 
nehmen.“\\
Auf diese Aussage, bekam sie ein fast schon gütiges Lächeln von der Mutter und Baram schien sehr 
zufrieden. „Natürlich nicht“, sagte er: „Du bist ein anständiges Mädchen.“\\
Lavay nickte nur Gedanken verloren. Sie konzentrierte sich auf den warmen Brei und verpasste den 
Anfnag des plötzlichen Streits. Baram stand aufrecht einem seiner Söhne gegenüber und hatte ein 
zornrotes Gesicht. ``Wir haben darüber schon gesprochen und ich habe nein gesagt!''\\
``Das ist viel zu gefährlich!'', fiel Barams Weib mit zitternder Stimme ein.\\
``Ich werde hier doch nicht verhungern'', rief der junge Mann und schüttelte energisch den Kopf: 
``Wartet es ab, mit dem Sold kann ich euch einen Braten kaufen!''\\
``Fals dich die Saleicaner nicht häuten und selbst zum Braten machen'', feixte der Älteste des 
Nachwuchses.\\
Lavay machte sich auf ihrem Hocker klein und wagte sich kaum zu fragen, aber die Neugierde siegte: 
``Sold? Wilst du in die Armee eintreten?''\\
Stolz nickte er ihr zu und lächelte charmant. ``Klar. Wie sollte ich sonst eine Familie ernähren 
können? Und einer Frau das bieten können, was sie verdient?''\\
``Gibt es Krieg?'', fragte sie überrascht. So lange war sie doch nicht außerhalb der Ziwilisation 
gewesen, oder doch?\\
Baram seufzte und setzte sich kraftlos wieder an den Tisch. ``Noch nicht. Aber man weiß bei diesen 
saleicanischen Teufeln ja nie... Nen Händler hat vor ein paar Tagen gemeint, dass die großen 
Kaufleute ihre Handelsrouten ändern und Schutzverträge eingehen. Vielleicht ist das alles nur 
gerede, aber mein Bruder im nächsten Dorf hat erzählt, dass die Rekrutierer schon dort waren und 
nach Freiwilligen suchen.''\\
``Zu denen du nicht gehören wirst''; schimpfte Barams Weib an ihren Sohn gerichtet fort und kurz 
darauf war ihr Gatte schon wieder auf den Beinen und redete auf den jungen Mann ein.\\
„Also…“, stammelte Lavay und rappelte sich auf: „Ich muss dann auch mal weiter… mein Vater wartet 
bestimmt. Vielen, vielen Dank für die Gastfreundschaft.“\\
„Aber, du wolltest doch noch etwas bleiben… wegen den Wölfen!“, stammelte Baram enttäuscht. 
Vielleicht hatte er sich erhofft, dass sie seinen Sohn davon abhalten könnte, sich freiwillig 
rekrutieren zu lassen. Vermutlich hatte er schon Hochzeitspläne im Hinterkopf gehabt. Bei dieser 
Vorstellung hielt sie nun gar nichts mehr hier und sie wiederholte hastig: „Vielen, vielen Dank für 
das Essen und eure Güte, aber ich muss zu dem Treffpunkt, den ich mit Vater verabredet hatte. Sonst 
sorgt er sich noch weitere Stunden unnötig.“\\
Barams Frau umarmte sie trotzdem herzlich und drückte ihr einen Kuss auf die Wange. Lavay erstarrte 
bei der Geste, obwohl sie von einer Frau stammte. Baram nickte mit enttäuschten Blick, während die 
jüngsten Kinder sie nur aus großen Augen anblickten.\\
Und wieder hörte sie das Heulen des Wolfs.\\
Nachdenklich zupfte sie an Grashalmen und überlegte, in welche Richtung sie weiter gehen sollte. Sie 
war alleine, musste auf niemanden Rücksicht nehmen. Genau genommen konnte sie überall hin. Kasir 
war groß. Und warum nicht das Land irgendwann einmal verlassen? Kasir grenzte nur an ein Land, 
Saleica. Einst sollen es wohl viele Länder dort im Süden gewesen sein, aber das aufbreißende 
Volk hatte sie alle erobert und unterworfen. Teufel nannte man sie, Dämonen, die nur für das 
Töten und Siegen geboren waren.\\
Lavay war alleine. Nichts hielt sie mehr. Alle die ihr am Herzen lagen waren Tod und aus ihrer 
Heimat, wer konnte sich da noch an ihren Namen erinnern? Außer Haska vielleicht. Nun konnte sie 
sein wer sie wollte, ihre eigene Geschichte neu schreiben! Sie benötigte lediglich ein paar Lügen 
und Fantasie. An einem Tag war sie eine Kaufmannstocher, die vor ihrem grässlichen Verlobten davon 
lief, in der nächsten eine Pilgerin. Weit fern waren die trostlosen Tage in Janka. Der staubigen 
Stadt in Süd-Saronne weinte sie keiner Träne nach.\\
Sie runzelte nachdenklich die Stirn. Aber sie konnte ja nicht für den Rest ihres Lebens so weiter 
machen, oder? Immer nur von einem Dorf zu nächsten und ab und zu was klauen? Lavay schüttelte den 
Kopf. Darüber wollte sie jetzt nicht nachdenken, sie war zu sehr damit beschäftigt die Geister der 
Vergangenheit zu verdrängen.\\