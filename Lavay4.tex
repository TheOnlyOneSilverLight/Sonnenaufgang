\chapter{Inhalt:, Renec Gespräch mit Samos; Einleben in Saleica, Treffen mit Renec}

Der graue Wallach riss den Kopf abrupt hoch und scharrte mit den Hufen, als Renec den Sattelgurt 
fest anzog. Gleich nach seinem ersten Ritt auf diesem Tier hatte er gelernt, dass dem Wallach was 
das Satteln anging nicht zu vertrauen war und man den Gurt jedes Mal nachprüfen sollte. Er hob 
seinen Fuß in den Steigbügel und stemmte sich vom Boden ab. Mit Schwung landete er im Sattel und 
suchte sich eine bequeme Position. \\
\textit{Ich brauche einen neuen Sattel}, überlegte er und strich über das raue Leder.\\
Renec nahm die Zügel auf und schnalzte mit der Zunge. Sofort spitzte der Wallach seine Ohren und 
drehte sie ihm zu.\\
``Reite vorsichtig, Junge.''\\
Der junge Mann musste sich zu einem Lächeln zwingen, als er dem Stallmeister zu nickte. Sein 
Abschied fiel wortlos aus, denn eben diesen Ritt trat er an, um einen anderen Mann zum Stallmeister 
zu berufen. Es tat Renec im Herzen weh, denn er konnte sich die Ställe und Weiden nicht ohne ihn 
vorstellen.\\
``Bastard!''\\
Es war einer der Wächter und Renec runzelte die Stirn. ``Was willst du?''\\
Samos grinste, als er die Zügel des Grauen ergriff und zu ihm empor blickte. ``Ich werde dich 
begleiten. Warte kurz, ja?''\\
``Wieso sollte ich?''\\
Samos‘ Stimme wurde leiser, sein Grinsen breiter. ``Weil ich sonst nicht garantieren kann, dass die 
Worte die ich an dich richten werde, von Anderen ungehört bleiben.''\\
\textit{Mistkerl.}
Renec nickte nur und sah ihm schweigend zu, wie er den Stall betrat und nach einem Pferd rief. Er 
schwieg auch weiterhin, als Samos auf einem Braunen aus dem Gebäude kam. Stattdessen trieb er 
seinen Wallach in einen flotten Trab und ritt vom Hof. Der Herbstwind zerrte an ihnen und Renec 
behielt den Himmel im Auge. Er hoffte, dass der heutige Tag ohne Regen verstreichen würde. Erst als 
sie den Wald erreichten, schritten die Pferde langsam dahin.\\
``Also sprich'', forderte Renec ihn ungeduldig auf: ``Was willst du von mir?''\\
\textit{Vielleicht haut er danach wieder ab und nervt mich nicht den ganzen Weg lang.}\\
Samos lachte und der Bastard beschloss, dass er den Mann wirklich nicht leiden konnte.\\
``Ist sie nicht bezaubernd, unsere junge Herrin?''\\
Er nickte nur und sah den Wachmann flüchtig von der Seite an.\\
``Ich hoffe, du findest sie nicht so bezaubernd wie Sieva. Ich meine ja nur… eine angeheiratete 
Mutter reicht doch, oder?''\\
Renec versteifte sich. ``Halt‘s Maul!''\\
``Ich glaube, unser Graf wusste es. Ich mein ja nur… die ganze Burg wusste es. Ich wette, er wollte 
dich umbringen.''\\
``Hat er aber nicht.''\\
``Ja. Also… das ist natürlich alles nur Spekulation, aber ich denke, Sieva hat dir dein 
Bastardleben gerettet. Hätte sie sich nicht umgebracht, dann hätte er nicht noch mehr Angst 
bekommen, niemals ein legitimes Kind als Erben zu haben. Obwohl sie ihm ja auch keines schenken 
konnte. So jedoch konnte unser Graf sich auch sicher sein, dass sein Bastard ihm nicht einen Bastard 
unter jubelte. Ist das nicht irre witzig? Stell mir mal vor, Sieva hätte ein Kind von dir bekommen! 
Ich muss sagen, deine Selbstbeherrschung ist bewundernswert. Ich erwartete schon nach dem zweiten 
Satz, dass du mir eine rein hauen würdest.''\\
``Ich habe es mir vorgestellt.''\\
``Ah…''\\
Renec schüttelte den Kopf. ``Mein Vater suchte sich eine neue Braut. Wenn er mich hätte umbringen 
wollen, dann hätte er es getan.''\\
``Vielleicht hatte er nur keine Zeit mehr dafür. Vielleicht hat er es erst erfahren, als er schon 
zu krank war.''\\
``Woher willst du das wissen?''\\
``Ich habe es ihm gesagt.''\\
Der Bastard funkelte ihn zornig an.\\
``Oh'', rief Samos lachend: ``Nun schau nicht so. Ich wollte unserem Grafen nur anvertrauen, dass 
er sich nicht um seine junge Gemahlin sorgen solle. Schließlich bist du ja noch da um sich um sie 
zu kümmern. Und lass das lieber mit der Faust. Du bist kein Kämpfer.''\\
Er ließ Samos nicht aus den Augen. ``Sarimé war stets an seiner Seite.''\\
``Unsere Herrin hat ihn einmal verlassen und den Garten aufgesucht. Aber ist das denn wichtig?''\\
``Was willst du?''\\
``Sie vertraut dir.''\\
``Nein.''\\
Samos lachte und deutete in die östliche Richtung. ``Wir müssen diese Abzweigung nehmen, nicht? Und 
doch, sie vertraut dir Bastard. Zumindest mehr als anderen.''\\
``Sie vertraut ihrer Schwägerin.''\\
``Oh ja… ein lustiger Haufen… Aber das ist egal. Denn ihre Schwägerin hat nicht deine Macht.''
Nun musste Renec trocken auflachen. ``Macht?''\\
``Du kennst die richtigen Leute.''\\
``Welche Leute?''\\
``Leute mit Waffen. Leute mit Ehrgeiz. Leute, die die Helle im Herzen tragen und eine Königin 
wollen. Sie könnte diese Königin sein.''\\
``Sie ist jung.''\\
``Es wäre nicht das erste Mal, dass die Jugend zum Symbol der Veränderung wird.''\\
Renec dachte einen Moment schweigend nach. Dann fragte er misstrauisch. ``Warum jetzt? Warum planst 
du ausgerechnet jetzt Hochverrat?''\\
``Doch nicht jetzt!'', lachte Samos: ``Bis es so weit sein wird, werden noch einige Jahreszeiten 
vergehen. Hochverrat sollte sorgfältig geplant werden. Aber ich glaube, das meinst du gar nicht, 
stimmt‘s?''\\
Der Bastard seufzte. ``Dann sprich. Was ist dein Plan?''\\
``Triff diese Leute. Rede zu ihnen. Erzähl ihnen von unserer Königin.''\\
``Ich denke, ich sollte lieber bei ihr bleiben'', erwiderte Renec misstrauisch und warf Samos einen 
vielsagenden Blick zu.\\
``Wir helfen ihr zu einer Herrin zu werden, der die Leute folgen wollen.''\\
``Sie ist jung und unerfahren.''\\
``Offensichtlich'', rief Samos verärgert aus: ``Aber willst du ein Land erobern und sein Volk 
unterdrücken, wie die Saleicaner? Nein! Es geht um unsere Freiheit!''\\
``Dann bring die Leute dazu, dass sie diese Freiheit wollen. Und zwar bestenfalls mit Sarimé als 
Königin. Den Respekt der Leute muss sie sich verdienen.''\\

Das Leben auf dem Hof des Saleicaners war friedlicher, als Lavay gedacht hatte. Sie fühlte sich zum 
ersten Mal seit Imurs Tod angekommen. Sie fand Ruhe in den täglichen Haushaltspflichten und der 
Versorgung der drei verbliebenen Pferde. Die Gespräche mit Rotan blieben oberflächlich, handelten 
nur davon, wo sie frisches Wasser her bekam und von seinen Pferden. Lavay fragte nicht nach, warum 
er so still war. Sie fragte nicht, wieso der Stall leer und der Hof einsam war. Ebenso wenig fragte 
er nach ihrer Geschichte. Nur eines Tages dann, als er ihr zum Frühstück eine gut gefüllte Schale 
süßen Haferbrei reichte und sich mit einem schweren Seufzen ihr gegenüber nieder ließ, erwähnte sie 
beiläufig: ``Der Prophet Eliras sprach, dass die Seelen der Gegangenen mit dem Wind reisen. Aber 
auch sie brauchen ein Zuhause, zu dem sie zurückkehren können um zu rasten. In den ersten Jahren 
ist das ihr Grab, aber der Körper verwest und wird wieder eins mit der Natur. Die Seele kann diesen 
Ort dann nicht mehr finden. Wir Kasira vergaben daher Dinge, die unseren Gegangenen sehr am Herzen 
lagen oder die ihnen gleichen. Diese Dinge finden die Seelen immer und können unsere Tränen sehen, 
wenn wir dort um sie trauern.''\\
Rotan sah von seiner Schüssel auf und blickte dem unscheinbaren Mädchen lange ins Gesicht. 
Schließlich legte er den Löffel nieder und seine Schultern sackten ein. ``Sie war zu gut für diese 
Welt. Nicht stark genug für die Geburt unseres Kindes.''\\
``Wie hieß sie?''\\
``Illiana. Wie kommst du darauf?''\\
Lavay deutete in den Raum. ``Sie ist immer noch hier. Überall. Darüber erzählte der Prophet Eliras 
auch. Wenn wir zu viele Dinge der Toten horten, dann können sie nicht frei sein. Das ist noch 
schlimmer, als wenn sie keinen Ort zu rasten haben. Zu viel hält sie gefangen. Ein paar wenige 
Gegenstände reichen um ihnen eine Heimat zu bieten, aber man darf sie nicht fesseln.''\\
Rotan verbarg sein Gesicht in den Händen. ``Aber ich will nicht, dass sie geht!''\\
Lavay senkte nachdenklich den Blick und erzählte leise: ``Mein Bruder starb. Er wurde nicht 
begraben. Und ich konnte keine Gegenstände von ihm begraben. Wir besaßen nicht viel und alles was 
ihn ausmachte trug er vor seinem Tod bei sich. Es gibt für mich keinen Ort zum Trauern. Ich 
glaube, der Prophet hatte nicht vollkommen recht. Ich denke, unsere Toten spüren auch die Liebe 
zu ihnen. Manchmal denke ich, dass er bei mir ist. Er kommt mit dem Wind und geht auch wieder 
mit ihm, aber einen winzigen Moment lang, ist es, als würden wir uns meinen Herzschlag teilen.''\\
Lavay sah zur Seite und horchte in sich hinein. Es fiel ihr schwer die richtigen Worte zu finden um 
ihre Gefühle zu beschreiben: ``Ich denke, es war gut, dass er kein Grab bekam. So findet er mein 
Herz, auch so weit weg von dem Ort, an dem wir lebten. Aber ich glaube auch, dass deine Frau ziehen 
will. Ihre Seele wird zurückkehren an den Ort, den du ihr schenken wirst.''\\
``Nur um dann wieder zu gehen?'', fragte er bitter.\\
``Was glaubt ihr Saleicaner über den Tod? Habt ihr auch Propheten?'', fragte Lavay nach.\\
Rotan schüttelte den Kopf. ``Wir haben Priester, die von unserem Gott erzählen. Und die leben noch. 
Unsere Toten kehren ein in Osymas Hallen, wenn sie es würdig sind. Wenn nicht, enden sie wohl eher 
in der göttlichen Besenkammer.''\\
``Wie wird man würdig?''\\
Rotans Augen funkelten grimmig. Das Geschirr schepperte, als er es grob in den Spüleimer fallen 
ließ. ``Krieg. Kampf. Morde. Für den Allmächtigen.''\\
Lavay betrachtete den wütenden Mann ruhig. ``Das erklärt die Geschichten über euch. Du bist nicht 
sehr gläubig?''\\
Der Gutsbesitzer schnaufte wütend und verkniff sich Flüche, die er in der Anwesenheit einer jungen 
Frau nicht laut aussprechen wollte.\\

Sie räumte geraden den Frühstückstisch ab, als sie Stimmen auf dem Hof vernahm. Anscheinend hatten 
sie Besuch, denn Rotan sprach Saleicanisch. Lavay konnte ihre Neugierde nicht verbergen und trat 
zur Tür. Wenn die Besucher in bösen Absichten gekommen wären, dann hätten die Geräusche anders 
geklungen, also stellte sich die junge Frau offen in die Türe und musterte die Gäste. Es waren zwei 
Männer die auf Pferden saßen, die deutlich teurer aussahen als der klägliche Rest, den Rotan nicht 
hatte verkaufen können. Höchstens die wilde Fuchsstute konnte an die beiden Wallache heran reichen. 
Eben diese Stute preschte auch am Gatter der Weide vorbei und wieherte auffordert. Ihr schien der 
Besuch zu gefallen. Die beiden Wallache reagierten mit gespitzten Ohren und witternden Nüstern auf 
sie, aber blieben still.\\
Der Mann auf dem grauen Tier sprach mit Rotan in einem höflichen, sachlichen Tonfall. Sein Begleiter 
saß steif im Sattel und sah sich wachsam um.\\
\textit{Ein Soldat}, schoss es Lavay durch den Kopf.\\
Sie sah seine formelle Kleidung und das kurze Schwert an seinem Gürtel. Ihr forschender Blick 
entging dem Mann auch nicht. Er unterbrach den schwarzhaarigen brüsk und deutete auf die junge 
Frau. Lavay sah den finsteren Blick aus den Augen des Unterbrochenen und musste kurz grinsen. Falls 
er der Herr des Soldaten war, würde er bestimmt Ärger für diese Unhöflichkeit bekommen. Sie wartete 
darauf, aber der Mann sah sie ebenfalls nur an. Rotan sprach etwas und Lavay entging das Wort Kasira 
nicht. Er wirkte niedergeschlagen und kam langsam auf sie zu. In ihrer Sprache murmelte er: ``Es 
tut mir leid, Lavay. Aber ich kann einen Gesandten der Gräfin nicht belügen.''\\
``Warum sind sie hier? Was geschieht jetzt? Werde ich eingesperrt?'', flüsterte sie so hastig, dass 
Rotan Probleme damit hatte, ihre Worte zu verstehen.\\
Aber der Herr kam dem Gutsbesitzer zuvor. Er hatte sein Reittier ein paar Schritte näher gelenkt 
und sprach in einem klaren, beinahe akzentfreien Kasira: ``Wie ist dein Name?''\\
Sie senkte ängstlich den Blick. ``Lavay.''\\
``Der Herr Arell sagte, saleicanische Soldaten hätten dich aus Kasir entführt und hier abgesetzt.''
Es war nicht wie eine Frage formuliert, aber er wartete scheinbar auf eine Antwort. Lavay zögerte. 
Sie wusste nicht, was sie dazu noch hinzufügen konnte und sprach schließlich nur: ``Es ist 
Krieg.''\\
Er schien mit der Tatsache, dass sie hier bei Rotan war, unzufrieden. Lavay trat einen scheuen 
Schritt zurück in das Haus. Es war ein dummer Versuch sich zu schützen. Aber vielleicht konnte sie 
durch die Hintertür in den Wald rennen und dabei schon einen guten Vorsprung erreichen, bis er vom 
Pferd abgestiegen und ins Haus getreten war.\\
Zu ihrer Überraschung schüttelte er plötzlich den Kopf. ``Ich bin nicht für militärische Belange 
verantwortlich.'' Der Mann musterte sie ein weiteres mal flüchtig und fügte hinzu: ``Du musst nicht 
weg rennen.''\\
Dann sprach er wieder zu Rotan in der Landessprache und Lavay kehrte in das Haus zurück. Völlig 
verwirrt und noch geschockt ließ sie sich auf einen Stuhl fallen und starrte auf die Muster und 
Schnitte des Holzes. Es dauerte einige Minuten bis sie Hufgetrappel vernahm. Rotan kam zu ihr. In 
einer ebenso erschöpften Bewegung ließ er sich ihr Gegenüber auf einen Stuhl nieder und schwieg. \\
``Was wollten die Kerle?'', fragte sie.\\
``Die Kerle.'' Er lachte trocken. ``Diese Kerle, wie du sie nennst, wurden von der Gräfin 
geschickt.''\\
``Was für eine Gräfin?''\\
``Die Gräfin Merandilas. Unsere Herrin. Sie verwaltet diese Grafschaft für unseren König Semric.''\\
``Du klingst komisch'', sprach Lavay ihre Gedanken aus. Rotan hatte einen Ton in der Stimme, den 
sie nicht deuten konnte.\\
Er seufzte. ``Ach… diese Gräfin ist noch ein Kind. Unser Graf vor ein paar Wochen erst verstorben. 
Und dann kommt auch noch der Krieg hier her. Ein Kind soll unser Land schützen… Aber diese Kerle… 
Der Schwarzhaarige ist der Bastard des verstorbenen Grafen. Eigentlich ein guter Mann. Ich habe ihn 
schon öfters gesehen. Immer höflich. Immer hilfsbereit. Den anderen kannte ich nicht, aber er 
stellte sich als Hauptmann der Leibwache vor.''\\
``Die beiden mögen sich nicht'', sagte Lavay.\\
Rotan blickte sie erstaunt an. ``Meinst du?''\\
Die junge Frau zucke nur mit der Schulter. ``Und was wollten sie?''\\
Der Gutsherr Lächelte schief. ``Die Gräfin will, dass ich für sie arbeite. Ich soll ihr 
Stallmeister werden.''\\
Er sah, dass Lavay mit diesem Begriff nicht viel anfangen konnte und erklärte: ``Ich soll mich um 
ihre Pferde kümmern. Sie einreiten und züchten.''\\
``Aber das ist doch gut. Das ist doch das, was du hier auch schon gemacht hast. Nur mit deinen 
Pferden.''\\
``Hm ja… vermutlich würde ich einige Tiere dort im Stall wiedererkennen. Ich habe früher auch mit 
dem Grafen verhandelt und ihnen Deckhengste und gute Stuten verkauft um deren Zucht zu verfeinern. 
Aber dann muss ich meinen Hof hier verlassen.''\\
``Er ist doch bereits verlassen'', sagte Lavay verärgert.\\
Rotan senkte den Blick als Reaktion auf diese wahren Worte. ``Und um dich habe ich mir auch Sorgen 
gemacht.''\\
``Den Bastard hat es nicht gestört.''
``Der Bastard ist ein Bastard und kein Graf oder Berater oder Kommandant. Seinen Begleiter hat es 
sehr gestört.''\\
Lavay schüttelte verärgert den Kopf. ``Dann bleiben wir halt hier auf diesem verlassenem Hof und 
trauern den Geistern der Vergangenheit hinterher!''\\
Sie konnte nicht verstehen, warum Rotan noch zögerte. Lavay hatte genug getrauert und wollte die 
Vergangenheit endlich hinter sich lassen. Die Zukunft versprach so viel mehr als die trostlose Zeit 
die hinter ihr lag. ``Was spricht denn dagegen?'', fragte sie ungeduldig.\\
Rotan ließ sich Zeit mit seiner Antwort, murmelte dann aber: ``Nichts... Überhaupt nichts...''\\

