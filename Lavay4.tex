\chapter{Die Klauen des Löwen}


Der Weg war nicht mehr ein Trampelpfad und wohin er führte, darüber dachte Lavay mittlerweile nicht 
mehr nach. Sie konnte sich gerade gut vorstellen, wie ihre Mutter sich in den letzten Tagen vor 
ihrem Tod gefühlt haben musste. Trotz der Schwäche die ihren Körper wie Gewichte nieder drückte, 
setzte sie einen Fuß vor den Anderen. Manchmal schwankte sie leicht, hielt den Blick aber sturr auf 
das bischen blanke Erde. Lavay hatte nie viele Nahrungsmittel gehabt, aber Schlaf. Sie konnte 
nachts in der Wildnis kaum zur Ruhe finden, egal wie erschöpft sie war. Kaum schloss sie die Augen, 
meinte sie den fauligen Atem eines Wolfs zu spüren, die Stiche von Insekten, der Ruf der Raubtiere 
erklang. In Dörfer ging sie auch nicht mehr, denn sie konnte sich nicht mehr dazu überwinden. Dort 
waren Männer. Und so freundlich sie auch vielleicht sein könnten, die Begegnung mit Barams Familie 
hat ihr auch wieder gezeigt, dass die Männer stets an mehr interessiert waren. \\
So taumelte sie also schon seit Tagen, die sie nicht zu zählen vermochte, in den südlichen Gefilden 
des Landes umher. Manchmal fand sie Beeren oder Früchte des Herbstes, einmal sogar einen zappelnden 
Hasen in einer Falle. Es tat ihr im Herzen weh, das Tier zu töten, aber der Hunger zwang sie zu 
dieser Tat. Sie weinte bittere Tränen, als sie das Blut des Tieres an ihren Händen sah. Das erste 
Mal, dass sie selbst ein Tier getötet hatte...\\
\textit{Imur hat mich vor so viel geschützt}, dachte sie und versuchte sich nicht wieder vor Kummer 
die Zunge blutig zu beissen.\\
Lautes Wiehern erklang und Lavay blieb stehen. Seit Stunden hatte sie keine Zivilisation gesehen, 
aber sie war auch nicht sehr schnell gelaufen. Die Erschöpfung ließ keine Panik zu, stattdessen sah 
sie sich nur um und erblickte auf der nächsten Hügelkuppe fünf Reiter. Sie blieb stehen wo sie war, 
denn es gab in dieser Wiese nichts, was ihr hätte schutz geben können. Vielleicht würde es ja 
schnell gehen. Der letzte Keim der Hoffnung wurde für sie zerstört, als de Reiter nähe kamen und 
sie erkannte, dass es keine offiziellen Soldaten waren. Sie hatte viele in den letzten Tagen aus 
der Ferne gesehen und man erkannte sie gleich an ihren blauen Uniformen und dem Wolf darauf.\\ 
Lavay hätte schon fliegen können muss, um den schnellen Pferden der Männer zu entkommen. Sie 
gallopierten auf die junge Frau zu und umkreisten sie lachend. Lavay versuchte ihnen in die 
Gesichter zu blicken, aber sie bewegten sich zu schnell und hektisch. Plötzlich fand sie sich auf 
dem Boden wieder, hatte es gerade noch geschafft sich mit den Händen und Knien aufzufangen. Einer 
der Männer hatte den Knauf seines Schwertes gegen ihre Schulter gestoßen und der Schmerz 
explodierte regelrecht in ihrem Körper. Ihr wurde kurz schwarz vor Augen und als sie Anstalten 
machte, sich wieder aufzurappeln, schwebte eine Klinge über ihrem Gesicht. Aus weit aufgerissenen 
Augen starrte sie zu dem Soldaten hinauf. Er grinste. Seine Kameraden blieben auf ihren Pferden 
sitzen und lachten. \\
„Wie ein Hase ist sie gerannt“, spottete einer. \\
„Kleine Diebin, dachtest du wirklich, du könntest einfach so weg rennen?“, fragte einer der anderen
Mascha, er war derjenige, dessen Schwert nur wenige Zentimeter von Lavays Haut entfernt war, 
schüttelte den Kopf. „Nicht nur Diebin. Sie hat meinen Bruder getötet!“\\
Lavay konnte in diesem Moment überhaupt nicht denken. Sie beobachtete nur das Schwert über sich und 
zuckte vor der Berührung des kalten Stahls zurück. \textit{Was schmerzt mehr? Der Strick? Das 
Fieber? Die Klinge? Oh Mara... ist es wahr? Ist der Tod Erlösung? Geleitet mich der Wind ins 
Jenseits?}\\
„Mascha, reiß dich zusammen! Unser Herr will sie haben. Sie wird ihre Strafe bekommen.“\\
Lavay zitterte und kniff die Augen fest zusammen, während Mascha sie verfluchte und sein Schwert 
nicht fort nahm. Sie flehte stumm, dass die Männer sie hier und jetzt töten würden. Wäre das nicht 
perfekt? Ihr Blut würde das Gras nähren und der prasselnde Herbstregen würde es fortschwämmen. Die 
Stürme würden ihre Seele mit sich nehmen, bis zum Mond, bis zu den Sternen. Das alles klang 
erfüllender als das Ungewisse, was ansonsten geschehen würde. Nein, es war nicht ungewiss. Die 
Männer würden das gleiche tun was Maschas Bruder bereits getan hatte. Sie konnte sich die Folter 
denken, die Qualen einer einsamen, finsteren Kellerzelle. Erniedrigende Gerichtsverhandlungen, 
deren Ergebniss bereits schon fest stand. Der Adelige würde sie und alles was sie ausmachte 
zerfetzten und damit bei der nächsten Abendgesellschaft prahlen.\\
Mehrere Pferde wieherten schrill und die Männer brüllten einander etwas zu. Lavay schlug die Augen 
wieder auf und sah, wie Mascha das Schwert aus der Hand glitt. Es fiel dich neben ihr zu Boden. Er 
drehte sich langsam um und Lavay zählte drei Pfeile in seinem Rücken. Gurgelnd ging er in die Knie 
und sackte zusammen. Seine Kameraden griffen nach ihren Schwertern. Ein Dutzend Reiter trabten fast 
schon gelassen auf sie zu. Ein goldener Löwe der sich hoch zur Sonne streckte, prang auf ihren roten 
Gewändern. Das Wappen hatte Lavay noch nie in ihrem Leben gesehen, aber sie kannte es trotzdem. 
Jedes Kind in Kasir kannte den Löwen, der die Sonne herausforderte. Jedes Kind kannte die 
Geschichten der barbarischen Krieger Saleicas, welche Dörfer plünderten und Menschen umbrachten, um 
damit ihren Gott zu ehren. Aber was taten sie in Kasir? Seit Generationen hatten sie sich nicht 
weiter in den Norden gewagt, hieß es in den Geschichten. Der Norden sei standhaft, darum hatten sie 
ihre gierigen Löwenklauen nach Süden über das Meer ausgestreckt.\\
Als Aras Männer die Überzahl sahen, rissen sie an den Zügeln und brüllten ihre Pferde an. Hufe 
donnerten dich an Lavay vorbei. Sie hörte den Anführer der Gruppe etwas rufen, konnte es aber nicht 
einordnen. Benommen blieb sie stehen und sah zu, wie die Löwen über Maschas Kameraden herfielen. 
Ihre Klauen zerfezten Muskeln und Sehnen, ihre Kiefer durchtrennten Gliedmaßen und ihre Pranken 
brachen Knochen. Lavay betrachtete das Blut, welches Maschas Herz aus seinen wunden pumpte und kam 
nicht umhinn, dieses tiefe Rot als schön zu empfinden. Der Lärm des Kampfes, den sie als das 
Brüllen von Löwen wahrnahm, trat in den Hintergrund bei diesem Anblick. Irgendwann dann spürte sie 
eine schwere Hand auf ihrer Schultern. Sie blickte auf. \textit{Die Löwen rasieren sich...}, 
stellte sie verwundert den offensichtlichsten Unterschied zwischen den kasirischen und 
saleicanischen Männern fest. Der Soldat sagte etwas in seiner ihr fremden Sprache und sie sah ihn 
nur stumm an.\\
Ein weiterer Löwe lenkte sein geschecktes Tier näher. Lavay erkannte den nachlässig geflochtenen 
Zopf und runzelte die Stirn. Offensichtlich gab es auch weibliche Löwen in der saleicanischen 
Armee. Die Frau musterte sie abschätzend und sprach in undeutlichem Kasirisch: ``Er will wissen, ob 
du die Hure dieser Männer bist.''\\
Lavay war sprachlos und schaffte es nur den Kopf zu schütteln. Zu spät kam ihr der Gedanke, ob 
diese Geste bei den Saleicanern wohl die selbe Bedeutung hatte, aber sie schienen es richtig zu 
verstehen.\\
Die Soldaten begannen zu reden und auf Lavay wirkte es so, als ob sie über sie und den Kampf 
scherzten. Der Mann stieg wieder auf sein Reittiert und warf der Frau ein Seil zu. Die Kameraden 
lachten grölend, während die Frau es geschickt auffing und dann zu Lavay ging und ihr die Hände 
fesselte. Lavay ließ es widerstandslos geschehen und fragte stattdessen: ``Warum seit ihr hier?''\\
``Krieg'', antwortete die Soldatin einsielbig: ``Wir wollten Spaß haben. Hatten wir. Wir wollten 
plündern. Haben wir nicht. Also bist du jetzt meine Beute. Es ist ehrlos ohne Beute heim zu 
kehren.''\\
``Krieg?'', wiederholte Lavay überrascht. Sie hatte gehört, dass die kasirische Armee dringend 
Rekruten suchte, aber dass der Krieg bereits ausgebrochen war, wusste sie nicht.\\
Die Soldatin zeigte grinsend ihre Zähne. ``Mein Name ist Sokra A'Rell und du bist nun mein 
Besitz!''\\

Sokra schwang sich wieder in den Sattel und ihr Reittier machte einen überraschen Satz, als sie ihm 
brüsk die Fersen in die Flanken stieß. Das Pferd warf den Kopf hoch, die Mähne flog durch die Luft. 
Triumphierend warf sie noch einen letzten Blick auf das Schlachtfeld, welches sie und ihre 
Kameraden hinterlassen hatten, und reckte die Faust in den Himmel. Der Siegesgruß, begleitet von 
ihren lauten Worten: ``Alle die unser Brüllen hören!''\\
Die anderen Männer und Frauen fielen mit ein: ``Alles was das Sonnenlicht berührt!''\\
Helix und Farun wendeten ihre Pferde und galoppierten mit einem lauten Jolen über die Leichen 
hinweg. Einige andere taten es ihnen noch nach, während Sokra einer der Jüngsten in der Truppe mit 
einem Kopfnicken zu verstehen gab, dass die Pferde der Toten eingesammelt werden sollten. 
Zumindest, solange sie noch brauchbar waren. Sie grinste schief, während sie ihre Kameraden 
beobachtete. Obwohl sie offiziell auf dem selben Rang stand wie sie, war sie bei dieser eher 
inoffiziellen Mission die Anführerin.\\
\textit{Verdammt, es ist Krieg!}, dachte sie und lachte still. \textit{Und so sollte ein Krieg auch 
aussehen!}
In den letzten Wochen lag das Gefühl des nahenden Krieges in den Garnisionen in der Luft. Und 
trotzdem saßen alle nur herum, schärften Klingen, pollierten Rüstungen, flickten Zaumzeug. Sokra 
wusste ganz genau, dass sie und ihre Freunde nicht die einzigen waren, die sich fort stohlen um 
schon einmal mit dem Krieg zu beginnen, während die Offiziere und der Adel noch an den gedeckten 
Tischen saßen und leise plapperten! Krieg ist laut, blutig und wild!\\
Sokra sah ruckartig zu der schmächtigen Gestalt auf dem viel zu großen Pferd neben sich. Der junge 
Priester war ihrer Einheit erst seit wenigen Wochen zugeteilt, aber er war es gewesen, der ihnen 
erzählte, dass der Krieg sicher war. Osyma sehnt sich nach Heldentaten, nach sterben zu seinen 
Ehren. Blut, dass für den Allmächtigen vergossen wird. Todesschreie, die in seinem Namen erklingen. 
Sokra war es egal ob Gott oder König es befahl. Sie war Soldatin, eine Kriegerin Saleicas. Seit 
mehr als 15 Jahren diente sie im Heer und eine richtige Schlacht hatte sie noch nie erlebt. Sie 
hätte alles gegeben um in die Kolonien versetzt zu werden, aber sie war in der saleicanischen 
Grafschaft Merandila geboren und verpflichtet die Grenze gen Norden zu schützen.\\
``Na?'', fragte sie den Priester, dessen Tattoowierungen sich auf seine Arme beschränkten: ``Ist 
Osyma glücklich?''\\
Jetzt, da sie ihn direkt angesprochen hatte, schien er seinen Mut zusammen zu nehmen und mit 
trotzig erhobenem Kinn erwiderte er: ``Sie hatten kein Recht ohne Befehl ins Feindesland 
einzudringen.''\\
``Du bist auch eingedrungen'', erwiederte Sokra abwertend.\\
``Das waren Ziwilisten.''\\
``Sie hatten Waffen. Unschuldige tragen keine Waffen'', spottete Sokra: ``Außerdem haben wir einen 
richtigen Ziwilisten gerettet.''\\
``Wieso dann die Fesseln?'', fragte der Priester und betrachtete die junge Frau, die reglos da 
stand und mit erschrockenen Augen das Spiel der Reiter betrachtete. Sie verstand vermutlich kein 
einziges Wort der Unterhaltung. ``In Saleica gibt es keine Sklaverei mehr'', sprach er.\\
Sokra lachte auf. ``Das sagte unser König. Was sagt der Gott dazu?''
Die Soldatin hob die Finger an die Lippen und stieß einen schrillen Pfiff aus. Nicht nur ihr Pferd 
spitzte die Ohren, sondern auch ihre Kameraden sammelten sich wieder. Ihre Pferde hinterließen 
blutige Hufabdrücke. Sokra winkte lässig mit der Hand und deutete den Rückzug nach Süden an. Zurück 
in die Heimat. Zu der Gefangenen herrschte sie in kurzen, kasirischen Worten: ``Steig auf.''\\
Ihre Kameraden lachten, als die Frau ungeschickt versuchte sich mit den gefesselten Händen auf das 
stämmige, kasirische Pferd zu ziehen. 

Sie ritten fast zwei Tage durch, nur mit wenige Stunden Rast. Jeder einzelne von den Soldaten war 
ein geübter Reiter und keiner von ihnen hatte ein Problem damit, in der langsamen Gangart im Sattel 
zu schlafen. Es beschwerte sich auch keiner, außer der Priester. Sein Jammern dämpfte den Triumph, 
den sie empfunden hatten. Er wurde nicht müde, ihnen mögliche Konsequenzen aufzuzählen. Sokra war 
bewusst, dass man ihnen gehörig Ärger machen konnte, aber streng genommen hatten sie gegen keinen 
Befehl verstoßen. Außerdem, wer sollte sie richten? Die Offiziere waren abhängig vom König oder 
dessen Vertretern den Grafen. Der König Saleicas könnte doch niemals seine Soldaten verurteilen, 
die für den Seegen des Allmächtigen in den Kampf gezogen sind! Und den alten Grafen hatte Sokra 
sogar schon mal gesehen. Er hatte das Wesen eines Kriegers, auch von ihm erwartete Sokra keine 
ernst zu nehmende Bestrafung. Und die Prieste? Sie würden sich doch ins eigene Fleisch schneiden, 
wenn sie Soldaten dafür bestraften, dass sie dem Allmächtigen ehrten.\\
Überraschend bemerkte Sokra aber auch, dass die Gefangene keinen Ton von sich gab. Sie hing sehr 
schief im Sattel, sodass die Soldatin erst befürchtete, sie würde irgendwann einfach herunter 
fallen. Deshalb hatte sie sie mit zusätzlichen Seilen fest am Sattel festgebunden und das Problem 
war schon einmal erledigt. Sokra sah der jungen Frau an, dass sie vermutlich noch nie auf einem 
Pferd gesessen hatte und sie durch eine falsche Haltung Schmerzen plagten. Aber daran war das Weib 
selbst schuld. Zu Fuß würde sie nur aufhalten und Sokra war nicht bereit, ihre Beute einfach so im 
Staub liegen zulassen. Sie hatte auch schon eine Idee, wo sie ihre Gefangene unterbringen konnte, 
ohne dass ihr Offizier davon erfahren musste.\\

Rotan streckte sich um seine verkrampften Muskeln zu lockern. Mutlos legte er sich zurück und seine 
Finger tasteten durch das taufrische Gras. Er stöhnte, weil er keine Worte fand seinen Schmerz zu 
beschreiben. Wenn er wenigstens auf jemanden wütend sein könnte, wenn er wenigstens jemanden die 
Schuld geben könnte! Aber es gab niemanden.\\
Niemanden außer ihm selbst. Egal wie sehr er es sich auch einredete, dass er nichts für sie hätte 
tun können. Seine Hände ballten sich zu Fäusten. \textit{Ich durfte nicht einmal zu ihr. Nicht 
einmal, als klar wurde, dass ihr nur noch wenige Atemzüge bleiben.}\\
Stattdessen hatte er an der Wand gekauert und zuhören müssen, wie seine kleine Tochter das erste und 
letzte Mal schrie. Es hatte sich anders angehört, als die ersten Schreie seines Sohnes. Damals war 
die Stimme des Säuglings kraftvoll, trotzig und gesund durch das Gut gehallt. Aber die kränkliche 
und schwache Stimme seiner kleinen Tochter nahm alle Hoffnung mit sich fort. Er hatte nicht dabei 
sein dürfen, als seine Frau starb. Er hatte nicht dabei sein dürfen, als seine kleine Tochter für 
wenige Minuten das Leben kennenlernte und dann der Mutter folgte.\\
Rotan richtete sich auf und betrachtete die aufgewühlte Erde unter dem Kirschbaum. Dort ruhten Frau 
und Tochter. „Verdammte Priester!“, zischte er leise.\\
Rotan sah sich zögernd um, ob irgendjemand seine wütenden Worte gehört hatte. Er entspannte sich 
wieder, als er niemanden erblickte. Betrübt betrachtete er den alten Baum, den sein Vater einst 
gepflanzt hatte. Unter diesem Baum hatte er Illiana kennengelernt. Unter diesem Baum und dem 
grimmigen Blick eines Priesters hatten sie einander ewige Verbundenheit geschworen. Unter diesem 
Baum lag sie nun begraben.\\
„Rotan“, sagte sein Knecht, der sich leise genährt hatte: „Illiana war einfach zu schwach um die 
Geburt zu überleben. Und eure Tochter ebenso.“\\
Rotan schüttelte den Kopf. „Ich war nicht für Illiana da. Und meine kleine Tochter war nicht 
schwach. Sie hat nur früher als wir anderen erkannt, dass diese Welt eisig und leer ist. Sie hat 
erkannt, dass es hier weder Recht noch Wärme gibt. Dass es nichts gibt, wofür es sich zu leben 
lohnt.“\\
Das folgende Schweigen brach der Knecht nach wenigen Minuten mit den zögernden Worten: „Ich wollte 
mich nur verabschieden. Die Pferde sind alle verkauft. Die letzten werden morgen abgeholt.“\\
„Alle?“, fragte Rotan heißer.\\
Er hatte in den letzten Wochen kein einziges Mal den Stall betreten. Und er wagte sich zu erinnern, 
er habe dem Knecht aufgetragen alle fortzujagen. Pferd wie Arbeiter.\\
„Bis auf den alten Braunen, die junge Falbe und dem Rappen.“\\
„Warum?“\\
„Wer will schon einen alten Gaul? Und der Rappe lahmt.“\\
„Die Stute?“\\
„Ist zu wild. Keiner wagte sich in ihre Nähe. Und jeder der es wagte, konnte sich keine Minute auf 
ihren Rücken halten. Soll ich noch dem Schlachter Bescheid sagen, bevor ich gehe?“\\
Rotan zog die Schultern hoch. „Nein. Das mach ich selbst.“\\
„Und dann?“\\
Rotan ignorierte seine Frage und blieb reglos sitzen, bis sein jahrelanger Freund sich schließlich 
seufzend abwandte, seine Reisetasche schulterte und ging. Rotan holte tief Luft und antwortete in 
die Stille hinein: „Ich weiß es nicht.“\\

Der Besitzer des leeren Guts schlürfte mit hängenden Schultern die Wiese, auf denen vor wenigen 
Wochen noch mehr als zwei Dutzend reinrassiger Pferde standen, zu seinem verwaisten Hof. Er 
meinte noch das vielstimmige Wiehern zu hören und trampelnde Geräusch, welches die Hufe auf dem 
Gras verursachte. Und die Fohlen erst, wie sie um die Stuten herum sprangen und sich gegenseitig 
zum wilden Spiel aufforderten.\\
„Bruder!“, rief eine tiefe Frauenstimme.\\
Rotan sah sich um und entdeckte ein Dutzend Soldaten die quer über seine Wiese kamen. Wäre er nicht 
so müde, hätte er ihnen die Hölle heiß gemacht. Schließlich trampelten sie unnötig das Gras 
zusammen, welches er als Heu im Winter brauchen würde. Ach nein. Er benötigte es ja nicht mehr. 
\textit{Der erste Winter seit Jahrzehnten, in dem kein Hufescharren den Stall erfüllen wird.}\\ 
Während Rotan beobachtete, wie seine ältere Schwester und ihre Kameraden näher kamen, fürchtete er 
einen Moment, dass sie ihm den letzten Rest nehmen könnte. Eigentlich war Sokra die Erstgeborene und 
Erbin. Aber sie hatte ihm Land, Gut und Tiere überlassen um zum Heer zu gehen. Sie hatte ihm die 
Grundlagen für ein gutes Leben gewährt. Er stand auf Ewig in ihrer Schuld. Er sollte ihr dankbar 
sein. Aber stattdessen empfand er nur Abscheu. Sokra vergötterte König Semrik als wäre er wirklich 
der Sohn des Allmächtigen. Rotan dagegen war das gleichgültig. Was ging ihn Politik an, wenn er 
genug Probleme mit den Fohlen hatte. Aber jetzt hatte er diese Probleme nicht mehr.\\
Als sie nahe genug waren, verneigte sich Rotan widerwillig vor seiner Schwester. Zwar hatte sie ihm 
das Land überlassen, trug aber noch den Titel des Familienoberhauptes.\\
„Ich habe nicht viel Zeit.“\\
\textit{Warum bist du dann hier?!}\\
Ihr Pferd warf den Kopf hoch und tänzelte. Sie griff die Zügel hart und Rotan schenkte dem Tier 
einen mitleidigen Blick. Das war ein gutes Pferd. Mit einem guten Reiter wäre es fantastisch. 
Aber Sokra hatte nie ein Händchen für Pferde besessen. Jedoch für große beängstigende Waffen und 
daher sagte er nichts dazu.\\
„Warum so betrübt? Deine Frau und dein Kind weilen jetzt an der Seite des Allmächtigen. Du solltest 
dich darüber freuen.“\\
Rotan schaffte es zu lächeln und nickte. „Ich weiß. Ich freue mich natürlich, dass er sie auserwählt 
hat um so früh schon an seinem Tisch zu sitzen. Aber ich hätte sie doch gerne noch etwas an meinem 
gehabt.“\\
\textit{Verlogene Scheiße}, dachte er finster und nicht zum ersten Mal fragte er sich, wie die 
Leute nur an solch einen Unsinn glauben konnten.\\
„Nun, ich habe ein Geschenk für dich.“\\
Rotan schloss einen Moment die Augen. Dann fasste er sich und zwang sich zu einem weiteren Lächeln. 
„Du hast mir in der Vergangenheit Geschenke gemacht, für die ich ewig in deiner Schuld stehen werde, 
Schwester. Ein weiteres ist nicht nötig.“\\
„Ich will es aber los werden.“\\
\textit{Genau wie den Hof.}\\
Sokra zerrte ein einem Strick und erst jetzt fiel Rotan das Mädchen auf. Sie stolperte und fiel hin. 
Ihr entwich ein leises Ächzen und sie blieb zitternd sitzen. Sie starrte vor Schmutz und das Haar 
glänzte fettig. „Aus den Kolonien?“, fragte er zögernd.\\
„Aus Kasir“, antwortete Sokra: „Also fast.“\\
Die Soldaten lachten. \\
„Was soll ich mit ihr?“\\
„Mach was du willst. Sie kann dir im Stall helfen.“\\
Rotan schwieg. Er wollte ihr nicht sagen, dass es nichts mehr im Stall zu tun gab. Ihr war nie viel 
an den Tieren oder am Haus gelegen, trotzdem wollte er nicht, dass sie es als Zeichen seiner 
Schwäche auffasste. Sie würde ihn nur auslachen. Also nickte er stumm. Sie drückte ihm den Strick in 
die Hand und wiederholte: „Ich habe es eilig.“\\
``Sag mir erst, was das alles soll'', forderte er und tatsächlich hielt sie inne und blickte einen 
moment nachdenklich auf ihn herab. Leise antwortete sie: ``Sie ist mein Besitz.''\\
``In Saleica gibt es keine Sklaverei mehr!'', entschied Rotan.\\
``Genau deswegen soll sie hier bei dir bleiben.''\\
``Und was hast du davon? Willst du alle paar Wochen kommen und dir von ihr die Stiefel putzen 
lassen?'', fauchte Rotan zornig. Diesmal ging Sokra viel zu weit.\\
Sie lachte jedoch nur. ``Mal sehen. Mein Pferd putzt ja schon du!''\\
Rotan ballte die Hände zur Faust. ``Ich mein es ernst! Ich gehe persönlich zu seiner Garnision und 
klage dich an, Sklaverei zu betreiben.''\\
Ungeduldig und wütend verzog sie das Gesicht. ``Was willst du eigentlich? Wir haben sie gerettet. 
Gauner waren hinter ihr her, sie war mitten in der Wildnis und allein. Hätten wir sie dort liegen 
lassen sollen, wäre dir das lieber gewesen? Dann mach du halt mit ihr was du willst. Ich hab keine 
Lust mehr mich mit dem Weib zu beschäftigen!''\\
Rotan erkannte hinter der Mimik seiner Schwester, dass sie Angst hatte. Mittlerweile war wohl 
genügend Zeit verstrichen und sie sich ihrer Tat bewusst. Sie hatte Angst vor Konsequenzen. Diese 
Frau war der Beweis, dass sie in Kasir gewesen war. Er würde ihr zutrauen, dass sie diesen Beweis 
verschwinden lassen würde, wenn sie ihn nicht anderweitig los würde. Er nickte also nur.\\
Ohne einen Abschiedsgruß rammte sie ihrem Reittier die Fersen in die Flanken und preschte gefolgt 
von ihren Kameraden los. Rotan sah ihr lange hinterher, bis er schließlich ein kleines Messer zückte 
und die Fesseln durchschnitt. „Sie haben dich wie einen alten Köter festgebunden“, stellte er auf 
Kasirisch fest.\\
Sie rappelte sich ruckartig auf und starrte ihn böse an. Rotan musste gestehen, dass sie doch nicht 
so hilflos war, wie er sie erst wahrgenommen hatte. Ihre blauen Augen funkelten zornig und ihre 
gesamte Körperhaltung zeigte, dass sie mit einem Kampf rechnete. Er ließ den Strick fallen und sah 
sie nachdenklich an. Rotan dachte an den Kirschbaum unter dem seine Frau und seine Tochter begraben 
lagen.\\
