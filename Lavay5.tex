\chapter{Tempel/Schule in Na'Rash}

Lavay saß auf den weichen Fellen des Bettes und schwieg ebenso wie die anderen Anwesenden. Ihre 
Fingerspitzen schwebten über das weiche, weiße Fell hinweg. Nur als einen Hauch, als ein Kitzeln, 
nahm sie es wahr. Ihre Linke dagegen krallte sich in das graue, rauere Fell, während sie seine 
Muster betrachtete. Lavay hatte noch nie so viele Felle auf einem Bett gesehen. Im Besitz auch nur 
einer Person. Hier in diesem Gemach fühlte sie sich selbst wie eine Königin. War sie wirklich in 
einer Kammer aufgewachsen, die kaum größer als dieses Bett war? Sie schlief immer noch auf dem 
Boden, aber das Stroh war sauber, die Decke warm und der stetige Geruch von Pferden umgab sie im 
Stall.\\
Das alles hier war so fremd. Und doch war sie seltsamerweise willkommen. Das Mädchen hatte ihr 
Aufmerksamkeit geschenkt. Lachen und Zeit. Seit dem wandelte sie durch das Anwesen, ohne dass auch 
nur eine Tür vor Lavay verschlossen blieb. Einige Bedienstete neigten sogar den Kopf leicht vor 
ihr. Nur redete keiner mit ihr, außer Rotan. Die anderen Stallburschen wichen ihr aus, flüsterten 
und fluchten über sie. Aber keiner wagte, ihr offene Abneigung entgegen zu bringen. Und das nur, 
weil die Gräfin sie anlächelte.\\                                          
\textit{Sie starrt zu oft aus dem Fenster.}\\
Immerhin hatte Lavay dadurch gelernt, was 'Fenster' und 'Dach' auf Saleicanisch hieß.\\
Die Gräfin sagte etwas, aber Lavay hatte sich zu sehr auf die Felle konzentriert um den vielen 
unbekannten Wörtern ihren Sinn zu entlocken.\\                                                   
Ihr Berater oder Freund oder was auch immer er wohl war, übersetzte. ``Unsere Gräfin fragt, was du 
auf den Straßen gesehen hast.''\\
Lavay kniff die Augen zusammen. Sie mochte Renec nicht besonders. Er wirkte überheblich auf sie. 
Aber er war nun mal einer der zwei Männer hier, die ihre Sprache beherrschten.\\
``Nichts besonderes'', antwortete sie und zuckte mit den Schultern.\\
Das verstand Sarimé bereits. Sie wandte sich ihr zu. Ihre grünen Augen waren umschattet. ``Sag 
alles'', murmelte sie. Ihre Worte hatten einen harten Dialekt.\\
Lavay sah von grünen Augen in Graue und wieder zurück. ``Ich war auf dem Markt. Die Händler riefen 
laut. Es war sehr voll'', erzählte die junge Frau in ihrer Muttersprache: ``Es gab Streit mit ein 
paar Jugendlichen. Die Wachen haben sie verfolgt, aber nicht erwischt.''\\
``Was haben die Jugendlichen getan?''\\
Lavay runzelte die Stirn. ``Laut gerufen. Ich weiß nicht was. Dann bin die die Straße weiter. 
Vorbei an dieser Pferdestatue mit der Frau. Auf den Stufen standen viele abgebrannte Kerzen. Ein 
Priester hat wütend geschrien. Und eine verkohlte Flagge lag noch dabei.''\\
Renec übersetzte fließend und blickte Sarimé abwartend an. Sie schloss die Augen. Es überraschte 
Lavay, die Rothaarige so voller Schuld zu sehen.\\
``Was bedeutet das?'', fragte sie hartnäckig.\\
Renec antwortete: ``Die Merandil stehen zwischen zwei Göttern.''\\
Lavay schüttelte den Kopf. Sie verstand nicht, wieso das ein Problem darstellte. Es gab so viele 
Propheten, da war es natürlich, dass ihre Lehren nicht immer übereinstimmten. Trotzdem gab es 
in Kasir darüber keinen Streit.\\
``Und wo ist das Problem?'', fragte Lavay nach.\\
Renec zögerte. Er hatte keine Lust zu antworten, dass sah sie ihm an. Aber auf einen Wink der 
Gräfin hin, öffnete sich sein zusammengekniffener Mund. ``Die Menschen streiten nicht nur. Sie 
begehen Verbrechen. Wenn das so weiter geht, wird es zum Krieg kommen. Der gewählte Stadthalter 
bemüht sich, aber sein Haus wurde letzte Nacht angesteckt. Einige Jugendliche von Priestern 
verprügelt und verhaftet, die in den Tempel eingebrochen sind. Die heuern auch immer mehr 
persönliche Wachen an, weil sie der Organisation der Stadt nicht mehr vertrauen.''\\
``Dann reicht es wohl nicht, dass sie diesen Osyma so ehrt'', schlussfolgerte Lavay und deutete auf 
Sarimé.\\
Sie dachte an all die vielen Dinge, die die Gräfin in den letzten Tagen im Namen des einen Gottes 
getan hatte. Selbst zu Gottesdiensten gerufen. Spenden ausgeteilt und stets 'dank Osyma' 
gepriesen, über heilige Feuer gewacht und Andachten gehalten.\\
``Dann wollen die Leute nicht den Feuergott'', schlussfolgerte Lavay.\\
``Aber er ist der Allmächtige!'', rief Sarimé im schweren Kasira und schüttelte den Kopf.\\
``Nein,'' warf Renec ein: ``Seine Priester sind allmächtig.''\\
Lavay sah auf das gewebte Muster des Teppichs zu ihren Füßen. Sie sachte an Janka, ihre 
Heimatstadt. In Staub und Armut war sie geboren und viel zu viele Menschen würden darin sterben. 
Und warum? Weil Mächtige sie übersahen. Die Elendsviertel waren eine eigene kleine Welt geworden. 
Wäre Imur nicht gestorben, hätte sie ihn nicht sterben sehen, dann wäre sie heute noch dort in 
dieser Welt aus Schmutz und Hunger. Stattdessen saß sie auf Fellen in einem Zimmer aus Licht.\\
\textit{Na'Rash hat kein Elendsviertel. Aber sie sind auch in einer Welt, in der sie nicht leben 
wollen.}\\
Sarimé fehlten die Worte und sie sah Renec an, während sie auf Saleicanisch sprach: ``Und du willst 
sie entmachten.''\\
Der Bastard deutete auf das Fenster: ``Die Merandil wollen sie entmachten. Und früher oder später 
werden sie es versuchen. Mit dir oder ohne dich.''\\

Silhe Basra deutete eine Verneigung vor der Gräfin an, ehe sie eine ausladende Geste über das 
Schulgelände machte. ``Was darf ich Euch zeigen, meine Herrin?''\\
Sarimés Blick wanderte über das grüne Gras, zu dem glitzernden Wasser des Springbrunnens und den 
steinernen Säulen vor der Haupthalle. Kinder und Jugendliche saßen auf den Stufen und der Wiese, 
genossen ihre Pause und die letzten herbstlichen Sonnenstrahlen.\\
``Ich will den Unterricht nicht stören'', erwiderte Sarimé ausweichend.\\
Dieses Treffen hatte die Direktorin veranlasst und nur aus der Neugierde heraus, wieso, war Sarimé 
deren Bitte gefolgt. Also wartete sie und blickte die Frau, die nahezu doppelt so alt war wie sie, 
fragend an.\\
Sie trug nicht das übliche Gewand der Priester, obwohl ihre Tätowierungen sie als eine der Ihren 
aufwies. \textit{Sie träg die Schuluniform}, fiel der jungen Gräfin auf und verglich die weiten 
Hosen der Schüler und Lehrer.\\
Silhe Basra strich sich das lange schwarze Haar, welches zumindest eine ihrer Kopfhälften zierte, 
hinter das Ohr. Die andere Seite war kahl rasiert und übersät mit Mustern aus Violett und 
Blautönen.\\
Die Direktorin nickte nur und fuhr fort. ``Dreiviertel meiner Schüler stammen aus dem Umland 
Na'Rashs. Die restlichen sind in der Stadt geboren. Als Kinder von Handwerkern, Marktschreiern, 
Stadtwachen. Die Kaufleute, die Ärzte, der Adel... sie leisten sich private Lehrer. Wie Eure 
Familie in Brom-Dallar bestimmt auch für Euch.''\\
Sarimé nickte lächelnd und biss sich dabei auf die Zunge. Es wäre zu blamabel gewesen, wenn eine 
Sil'Vera in eine öffentliche Schule gegangen wäre. Ihr Vater brachte ihr das Rechnen, Lesen und 
Schreiben bei. Ihre Magd das Nähen, Sticken und Pflanzenkunde. Ihre Stiefmutter lehrte sie 
tanzen und Umgangsformen. Den Rest hatte sie aus den stillen Stunden in Brom-Dallars Bibliotheken.\\
Silhe Basra begann am Schulgebäude entlang zu schlendern und Sarimé folgte ihr abwartend. Sie 
nickten grüßend den Schülern zu, sowie auch den Lehrern, die beisammen standen und sich leise 
unterhielten.\\
``Ich bin im Norden geboren, Herrin'', fuhr die Rektorin fort: ``Weit im Norden, an der Ostküste. 
Meine Familie war mehr Kasira als Saleicaner. Die Winter wurden hart dort oben in dem 
abgeschiedenen Tal. Die Siedlungen lösten sich auf. Manchen zogen nach Norden. Manche nach 
Süden.''\\
``Und Eure Familie kam hier her?''\\
Silhe Basra schüttelte den Kopf. ``Nein. Sie entschieden sich für Kasir. Meine Eltern und Onkel 
hofften, dort eine sichere Arbeit zu finden. Ich habe viele Jahre nichts mehr von ihnen gehört.''\\
``Ihr wollt, dass ich frage, wieso Ihr nach Saleica kamt?'', sagte Sarimé ungeduldig.\\
Die Frau blieb stehen und blickte über die Dächer Na'Rashs. ``In beiden Ländern ist ein Leben ohne 
Geld schwierig, meine Herrin. Aber in Saleica hat man immer noch eine Chance, mächtig zu werden, 
ohne Titel und ohne Beziehungen. Ich bin Priesterin. Ich bin Lehrerin. Ich gebe diese Kinder nicht 
auf.''\\
``Einige dieser Kinder begehen Verbrechen gegen Bürger und den Glauben'', erwiderte Sarimé.\\
Sarimé folgte ihrem Blick und entdeckte den Stadthalter Sakan, wie er umgeben seiner persönlichen 
Wachen und in Begleitung Em-Hirs auf sie zusteuerten. ``Das sind die Kinder Merandilas'', 
entgegnete Silhe Basra leise: ``Ganz gleich, an welchen Gott sie ihre Gebete richten.''\\
``Geschätzte Gräfin'', säuselte Em-Hir und nickte Sarimé beiläufig zu. Er streckte der Rektorin 
seine Hand entgegen und sie küsste erst seine, dann die des Stadthalters.\\
``Welch Überraschung euch zu sehen'', bemerkte Sakan: ``Dann hat unsere Gräfin dich bereits 
informiert, Schwester Basra?''\\
Die Frau schwieg und versuchte ihre Verwirrung nicht zu deutlich zu zeigen.\\
Sarimé nahm den Vorwurf auf. Welchen Grund die Rektorin auch immer gehabt haben mag, sie zu diesem 
Gespräch zu bitten. ``Es hat sich noch nichts entschieden, Stadthalter. Unsere Unterhaltung wurde 
sehr grob beendet. Aber ich wollte mir die Meinung der Direktorin einholen, ja. Es geht immerhin um 
ihre Schüler.''\\
``Kein Grund ohne die Einwilligung das Stadtrates Gerüchte zu verbreiten'', zischte Sakan.\\
Sarimé zögerte einen flüchtigen Moment, dann erwiderte sie: ``Silhe Basra ist die Direktorin, sie 
hat mitzuentscheiden, ob Flüchtlinge im Zeiten des Krieges in den Schulgebäuden untergebracht 
werden können.''\\
Sie sah die Priesterin nicht an und wartete unsicher auf deren Antwort.\\
``Ich bin kein Mitglied des Stadtrates'', erwiderte diese: ``Was auch immer die weisen Herren und 
die Herrin entscheiden, dem werde ich folgen.''\\
Die beiden Männer wirkten sehr zufrieden. Im Gehen fügte Sakan noch hinzu: ``Und wegen der 
Gewaltverbrechen lasse ich die Stadtwachen mit meinen Wächtern aufstocken. Es geht um die 
Sicherheit unserer Stadt und es wurde vor wenigen Stunden zwischen den Stadthaltern einstimmig 
entschieden.''\\
Sarimé sah ihnen nicht hinterher, während sie auf den Tempel zu gingen und eintraten.\\
``Es reicht nicht'', sagte die Direktorin unvermittelt: ``Es wird nichts reichen. Außer, Ihr gebt 
Euren Titel ab und werdet Novizin. Und auch dann wird es Zweifler geben. Ihr habt diese Macht 
geerbt Herrin. Ein Geschenk, wofür ich viele Jahre hart arbeiten musste. Aber ohne das Volk seid 
Ihr machtlos.''\\
``Wie soll ich allein mit der Unterstützung von Bauern und Handwerkern überleben?'', fragte Sarimé 
leise.\\
Tief war die Verneigung, während Silhe Basra noch flüsterte: ``Macht nicht die selben Fehler wie 
unser König.''\\

Es war das erste Mal, dass Sarimé sein Zimmer betrat. Ihre Wache, die sie stumm von dem Augenblick 
an, als sie ihr eigenes Gemach verließ, begleitete, blieb auf dem Gang zurück. Ihr entging nicht 
das Zucken in der Augenbraue, ehe der Mann sich umdrehte und sehr beschäftigt dem dunklen Flur 
entlang sah. Behutsam schloss das Mädchen die Tür hinter sich und wartete reglos, bis ihre Augen 
sich an das karge Mondlicht im Zimmer gewöhnten. Der Raum war klein. Eine Truhe und ein Regal nahm 
die Seite rechts von ihr ein. Das Fenster befand sich ihr gegenüber. Darunter ein schräger kleiner 
Tisch, auf dem die Flamme eines Kerzenstummels um ihr Leben zuckte. Und dann war da nur noch er.\\
Renec lag von ihr abgewandt. Sie sah nur seine vom Schlaf zerzausten Haare. Den Abend hatte er 
nicht im Anwesen verbracht, hatte Samos ihr erzählt. Und immer noch trug der Bastard seine 
unscheinbare Kleidung aus dunkelblauen Stoff und der Weste.\\
Leise trat sie näher und nutzte den Moment, ihn zu betrachten. Sein Atem ging regelmäßig. Auch, als 
er sich raschelnd drehte und zu ihr aufblickte. Einen Moment war Sarimé sich unschlüssig, was sie 
eigentlich tat. Dann setzte sie sich auf die Bettkante und faltete die Hände im Schoß. Wagte es 
nicht, ihn weiter anzusehen. Die Kerzenflamme verglomm und Sarimé spürte seine warmen Hände an 
ihren Schultern. Renec zog sie zu sich auf das Bett, umschlang sie fest und verbarg sein Gesicht in 
ihrem Haar.\\
``Ist es Ga'Leor?'', fragte sie leise.\\
Er schwieg.\\
``Was plant ihr? Warum schleichst du dich aus dem Anwesen? Mit wem triffst du dich? Was plant 
ihr? Über mich...''\\
Sie hörte sein Herz schlagen. Ruhig und Stark.\\
``Antworte mir!'', rief sie lauter als beabsichtigt.\\
Renec küsste ihr Haar. ``Ich war bei einer Andacht. Lavay war auch dabei, falls du mir nicht 
glaubst.''\\
``Klug. Eine Taube mitzunehmen an einen Ort, an dem viele Menschen sich versammeln. Du hättest 
meinen Tod planen können und sie hätte es nicht mitbekommen.''\\
Renec lachte leise auf. ``Unterschätze die Kasira nicht.''\\
``Also nicht Ga'Leor?''\\
``Du brauchst keine Angst zu haben. Er braucht dich.''\\
Sarimé versteifte sich. ``Wer?''\\
Sein Griff lockerte sich, ließ ihr den Freiraum aufzustehen. Da sie es jedoch nicht tat, setzte er 
das Streicheln ihres Arms fort. ``Ich weiß kaum etwas'', flüsterte Renec in ihr Ohr: ``Hab 
vertrauen.''\\
``Es ist meine Aufgabe, alles zu wissen!'', zischte Sarimé: ``Ich bin die Gräfin!''\\
``Nein. Es ist deine Aufgabe zu überleben. Und dass dein Kind überlebt.''\\
``Ich bin keine Zuchtsau.'' Ihre Worte waren schwächer, als sie es beabsichtigt hatte. Hastig 
blinzelte sie die Tränen fort.\\
Seine nächsten Worte flüsterte er wieder in ihr Ohr. Sein warmer Atem kitzelte auf der Haut. 
``Nein. Du bist die Gräfin. Aber du könntest Königin sein.''\\
``Wieso verheimlichst du mir dann alles?'', fragte Sarimé.\\
``Weil ich es nicht weiß.''\\
Sie schob seine Arme von sich und setzte sich wieder auf die Kante des schmalen Bettes. Kühl 
blickte Sarimé zu ihm herab. ``Wo warst du?''\\
Ärger huschte über Renecs  Gesicht. Das sonst so jugendliche, schiefe Lächeln blieb aus. Er holte 
tief Luft, ehe er wiederholte: ``Bei einer Andacht.''\\
Sarimé erhob sich und strich in einer beiläufigen Geste ihren Rock glatt. ``Gut. Zur nächsten 
nimmst du mich mit.''\\
Sie ließ ihm nicht die Zeit zu protestieren. ``Ich oder Samos, such es dir aus.''\\








