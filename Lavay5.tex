\chapter{Tempel/Schule in Na'Rash}

``Du sagtest, du wärst in einer Stadt aufgewachsen'', fragte Renec.\\
Lavay schlenderte neben ihm über den kurz geschnittenen Rasen der Tempel- und Schulanlage Na'Rashs. 
Es war ihr freier Tag und eigentlich hatte sie sich darauf eingestellt, ihn wie jeden anderen im 
Stall zu verbringen und den Saleicanern auszuweichen. Die wenigsten behandelten sie mit der 
gleichen Höflichkeit wie dieser Renec oder gar die Freundlichkeit der Gräfin. Für letzteres 
versuchte Lavay noch den Grund herauszufinden. Warum fand das rothaarige Mädchen sie so 
interessant? Sie hatte bisher noch keinerlei Gemeinsamkeiten feststellen können, abgesehen davon, 
dass sie sich kaum verstanden. Trotzdem war ihr Lächeln oft genug der einzige Lichtblick ihres 
Tages. Die anderen Knechte behandelten sie grob, schubsten sie aus ihrem Weg oder rissen ihr die 
Zügel eines Pferdes aus der Hand. Lavay konnte schon verstehen, dass die anderen keine Lust hatten, 
sich zu bemühen sich ihr verständlich zu machen. Wenn Rotan in der Nähe war, behandelte man Lavay 
mit kühler Verachtung und erst am letzten Abend stritten die Knechte mit ihrem Stallmeister über 
sie. Lavay vermutete es ging darum, dass sie nicht mit ihr - einer Kasira und Feindin - zusammen 
arbeiten wollten. Sie schlief auch nicht mit den anderen in einer Kammer, sondern zog sich zu den 
Pferden zurück. Ab und an wechselte sie, bei welchem Pferd sie sich ins Stroh kroch, aber meistens 
schlief sie bei der zierlichen Rappstute. Bei ihr hatte Lavay die geringste Angst, im Schlaf 
niedergetrampelt zu werden. Sie war ja nur halb so groß wie manch eines der anderen Tiere.\\
``Janka'', murmelte Lavay: ``Ich kam da zwar auch nicht weit herum, aber ich glaube nicht, dass es 
in Janka auch nur annähernd eine so große Fläche Gras gibt.''\\
Er nickte abwesend. ``Hier finden immer mal wieder Sportveranstaltungen statt. Meistens von den 
Schülern ausgeführte Wettkämpfe in unterschiedlichsten Disziplinen. Alle paar Jahre veranstaltet 
die Grafschaft auch Turniere für Soldaten des Heers. Mein Vater nahm früher an solchen angeblich 
auch selbst teil. Es muss hart gewesen sein, in die Wildnis aufzubrechen, wenn man nur das 
Stadtleben kennt.''\\
Lavay warf ihm einen finsteren Blick zu und sah sich um. Sie war immerhin mit der Gräfin hier her 
gekommen und nun hatte man sie mit deren Berater allein gelassen. Sarimé war plaudern mit einigen 
Kuttenträgern in das Schulgebäude verschwunden. Anscheinend wäre Lavays Anwesenheit dort nicht 
wünschenswert gewesen. Das war ihr aber auch gleich. Was Lavay jedoch deutlich mehr störte, war, 
dass sie eben so ziemlich alleine war mit diesem Mann. Kaum hatten sie die anderen Menschen 
verlassen, stieg die Nervosität in ihr hoch. Sie ließ Renec keine Sekunde aus den Augen und zuckte 
bei plötzlichen Bewegungen zusammen.\\
``Ich erzähle dir meine Geschichte nicht'', erwiderte sie: ``Es geht dich nichts an.''\\
Er sah sie kritisch an. ``Ich dachte nur, du würdest dich gerne unterhalten. Spricht ja nicht jeder 
hier deine Sprache.''\\
Lavay schwieg und starrte auf das Gras vor ihr. Eine lange Zeit schlenderten sie - mit deutlichem 
Abstand - nebeneinander her, dann entdeckte Lavay das rote Haar der Herrin. Ihr Schritt 
beschleunigte sich unwillkürlich und als Sarimé die Beiden entdeckte, wandte sie sich ihnen ganz 
zu. Sie lächelte die Kasira an und sagte in einem fragenden Ton auf Kasirisch: ``Hier schön?''\\
Lavay vermutete, dass die Herrin wohl fragen wollte, ob es ihr hier gefiel und sie nickte lächelnd. 
Schnell bückte sie sich, riss einzelne Grashalme heraus und zeigte sie Sarimé. ``Sehr schön!'', 
sagte sie um zu zeigen, was ihr hier gefiel. Dann jedoch fiel ihr auch die Begleitung der Gräfin 
auf. Eine Frau mit langem dunklen Haar. Ihr Kopf war jedoch teilweise rasiert und von den 
verschlungenen Mustern der Priester geziert. Ihr Blick war berechnend, während sie Lavay musterte 
und sich dann wieder der Gräfin zu wandte. Die restliche Unterhaltung konnte Lavay durch die 
unterschiedlichen Sprachen nicht folgen.\\

Sarimé legte der Direktorin vertrauensvoll die Hand auf die Schulter, während sie zu Renec sagte: 
``Das ist Silhe Basra, die oberste Rektorin der Schule. Wir hatten eben ein interessantes Gespräch 
und ich durfte mir die Räumlichkeiten anschauen.''\\
Der jungen Gräfin entging sein misstrauischer Blick nicht, aber sie lächelte nur noch strahlender. 
``Und was man so alles erfährt. Wusstest du, dass die Stellen, an denen die Diener Osymas 
Tätowiert sind, eine Bedeutung haben? Ihre Muster auf dem Kopf zeichnet sie als Studierte aus!''\\
``Erfreut Euch kennenzulernen'', murmelte Renec nur und zwang sich zu einem Lächeln.\\
Sarimé kicherte. ``Ich weiß, du denkst bestimmt, warum ich euch einander vorstelle. Nun...''\\
Die junge Gräfin tauschte einen Blick mit der Rektorin, welche vielleicht doppelt so alt war die 
das Mädchen. Diese fuhr fort: ``Ihr solltet auch eine heitere Miene zur Schau tragen. Hier sind so 
viele Augen die uns sehen, auch wenn die Ohren außer Reichweite sind. Lasst uns etwas plaudern. 
Einen Witz erzählen.''\\
``Um was geht es?'', fragte Renec und zog die Mundwinkel höher.\\
``Um die Schließung der Schule'', erklärte Silhe Basra: ``Natürlich finde ich das als Rektorin 
nicht sehr erstrebenswert, schließlich gehört sie zu meinem Lebenswerk. Tagtäglich stecke ich all 
meine Energie in sie hinein. Eigentlich bin ich wohl die lauteste Gegnerin gegen das Vorhaben 
unserer geschätzten Gräfin.''\\
``Aber?'', fragte er und versuchte herauszufinden, woher diese Vertrautheit zwischen den zwei 
Frauen kam.\\
``Die Schule wird ja nicht für immer geschlossen. Immerhin ist sie eines der wichtigsten 
Institutionen in Merandila'', warf Sarimé ein: ``Und die Zeit der Schließung kann gut dadurch 
überbrückt werden, dass das Gebäude renoviert wird.''\\
Die Rektorin nickte zustimmend. ``Ich versuche schon seit Jahren hier einiges zu ändern, aber es 
ist schwierig, Baumaßnahmen während dem Schulbetrieb durchzubringen. Vor allem da ein Großteil 
unserer Finanzen in den Unterricht und Materialien für den täglichen Gebrauch fließt.''\\
``Die Gräfin finanziert also die Renovierungen?'', fasste Renec zusammen.\\
``Nicht nur Renovierungen. Eventuell wird auch ein neues Gebäude entstehen'', erklärte Sarimé: 
``Silhe hat schon zahlreiche Zeichnungen.''\\
``In Kriegszeiten können wir uns keine extra Ausgaben leisten'', zischte Renec leise und kniff den 
Mund zusammen.\\
``Es sind keine zusätzlichen Kosten. Wir sparen ja am Unterricht. Fast alle Lehrer sind Priester, 
die nach dem Verlust ihrer Stelle hier an der Schule Obdach in den Tempeln finden werden'', sagte 
Sarimé.\\
``Das heißt, der König zahlt. Nicht die Gräfin'', warf Silhe Basra ein.\\
Renec funkelte sie an. ``Und was habt Ihr davon?''\\
Die Rektorin lächelte breit und warf einen vielsagenden Blick zu Sarimé. ``Ich werde in den 
nächsten Jahren des Krieges mit Freude meiner Gräfin und Na'Rash dienen!''\\
Zweifelnd schüttelte der Bastard den Kopf. ``Ich verstehe nicht...''\\
Sarimé sagte: ``Liebe Silhe... sei doch nicht so zuverlässig. Erst einmal muss eine Stelle frei 
werden, die du antreten könntest.''\\
``Ich muss bald los, die Glocke ruft gleich zur Mittagsversammlung im großen Saal'', erwiderte die 
Rektorin: ``Ich werde mich bei Euch melden, Herrin, wenn ich aussagekräftige Argumente habe. Osyma 
segne Euch!''\\
Sarimé neigte den Kopf zu einem Abschiedsgruß. ``Und möge er zufrieden mit uns sein'', fügte sie 
die religiöse Phrase hinzu.\\
Mit großen Schritte und wehender Kutte verließ die Rektorin sie und eilte auf das Mittlere der 
Schulgebäude zu. ``Was war das eben?'', fragte Renec.\\
``Unsere neue Stadthalterin. Vertreterin der Priester.''\\
``Intrigen? Du? Das hätte ich nicht erwartet'', murmelte er überrascht.\\
``Ich weiß nicht wovon du sprichst. Wir begegneten gerade nur einer ehrgeizigen Priesterin, welche 
alles daran setzen will, dass Na'Rash diesen Krieg möglichst unbeschadet überstehen wird'', 
antwortete Sarimé und nickte Lavay freundlich zu, während sie über die Wiese zurück in die Richtung 
ihrer Kutsche spazierte.\\
``Wie soll der jetzige Stadthalter abgesetzt werden?'', fragte Renec neugierig.\\
``Woher soll ich das wissen? Wieso sollte ich es wissen? Es ist ja nicht so, als ob ich einem 
Verrat zustimmen würde!'', sagte Sarimé entschieden: ``Aus welchen Gründen auch immer der Vertreter 
in Ungnade fallen wird, meine Aufgabe als Gräfin ist es darüber zu richten. Und schnellstmöglich 
dafür zu sorgen, dass ein Ersatz eingesetzt wird. Die Rektorin hat einen ausgezeichneten Ruf, ist 
fromm und beliebt bei der Stadtbevölkerung. Welche Argumente könnte der Hohepriester gegen sie 
liefern?''\\
``Vielleicht, dass du betrügst?''\\
Sarimé zog die Schultern hoch. ``Ich doch nicht. Ich weiß von nichts. Ich habe eben nur der 
Rektorin versichert, dass - sollte es zu einer Schließung der Schule kommen - ich diese durch 
finanzielle Mittel für Baumaßnahmen unterstützen werde. Es gibt keine Beweise und keine Zeugen für 
unser Gespräch. Sollte sie meinen, sich doch lieber in den Schutz des Hohepriesters zu flüchten, 
steht ihre Aussage gegen meine. Und die Priester können mich nicht loswerden...'' Sarimé warf einen 
Blick auf ihren Bauch. ``Nicht, solange ich schwanger bin.''\\
``Sei vorsichtig'', flüsterte Renec besorgt: ``Ein Kind braucht nicht unbedingt seine leibliche 
Mutter...''\\
``Ich weiß. Darüber werde ich mir Sorgen machen, wenn es so weit ist. Bis dahin genieße ich den 
Schutz, den mir mein Kind im Bauch bringt. Bis dahin habe ich hoffentlich meine Feinde möglichst 
weit weg von mir.''\\
``Und Kasir?''\\
Sarimé blickte ihn ernst an und schwieg dann doch. Stattdessen bückte sie sich, entriss weitere 
Grashalme und hielt sie Lavay hin. ``Gras'', sprach sie deutlich und wartete auf Lavays Übersetzung.

