\chapter{Diener}

Die vergangenen Tage hatten Lavay eine Erkenntnis beschert. Sie wusste, woher der Stolz der 
Saleicaner kam. Sie waren frei. Sie konnten einander prügeln, ohne dass es die Stadtwachen störte. 
Sie konnten in Häuser einbrechen, ohne dass sich ein offizieller Richter einmischte. Sie konnten 
lieben, ohne dass Verträge unterzeichnet werden mussten. In Saleica galt das Gesetz der Tradition 
und das des Stärkeren. Der Adel wusste sich vor Dieben zu schützen, in dem seine angeheuerten 
Wachen keinerlei Einschränkungen genossen. Er konnte Gnade walten lassen oder hinrichten, ohne dass 
ein Richter nachfragen würde. Der Mensch war lediglich seinem Herrn verpflichtet, 
Lavay war sich unschlüssig, wie sie dieses System finden sollte. 
Trotz den starren Strukturen in Kasir konnte sie nicht behaupten, dass sie sich in der Heimat 
sicherer gefühlt hatte. \textit{Ich stand auf der falschen Seite des Gesetzes.}\\
Als Kasira waren die Gesetze ihre Fesseln gewesen. Kaum ein Beruf, den sie durch ihren Stand 
ausüben durfte. Viele Orte, die sie nie betreten durfte. Und trotzdem blieb ihr der Schutz, den das 
Gesetz versprach, verwehrt. Hier, in Saleica, teilten sich reich und arm den selben Markt, die 
selben Schulen und den selben Tempel. \textit{Zuhause war ich gefesselt. Hier treibe ich hilflos im 
Strom...}\\
Es gab weniger Bettler. Weniger zwielichtige Gestalten. Weniger Arbeitslose. Es schien nur ein 
Gesetz zu geben. Das Wort des Allmächtigen. Nur einen obersten Herrn, den König selbst.\\
Und vielleicht war es dieser Stolz, der die Saleicaner blind machte. Aber nicht taub. Das 
Geflüstert war unüberhörbar geworden.\\
Über Generationen hatten die Saleicaner sich eingebildet, ein Reich zu sein. Na'Rash teilte nicht 
in arm und reich, aber in Saleicaner und Merandil. Lavay hatte Übung darin bekommen, zu erkennen. 
Sie sah es in den Augen der Menschen. Selbst unter den Priestern gab es Blicke, die sich beschämt 
senkten, wenn die Predigen über Osyma durch die Straßen hallten. Und das taten sie oft in den 
letzten Wochen. Die Priester hatten es sich zu ihrer höchsten Pflicht ernannt, die Ketzerei zu 
zerschlagen. Verbrennen sollen sie in den heiligen Flammen des Allmächtigen. Lavay hatte diese 
Worte so oft gehört, dass sie sie selbst in Saleicanisch komplett verstand. Predigten konnte sie 
fehlerfrei zitieren, während sie beim Einkaufen auf dem Markt noch nach Worten rang.\\
\textit{Er ist ein brüllender Gott}\\
Als Ausländerin folgten ihr stets die Blicke. Aber die Menschen waren unschlüssig. Sie war eine 
Kasira. Eine Feindin. Aber der Groll des Volkes richtete sich gegen die Priester, die mit ihren 
Worten Kasir zum Feind berufen hatten. Gegen den Adel, der feige in den Süden floh. Gegen die 
Soldaten, die feiernd und betrunken durch die Straßen zogen. Und die Priester hatten nur 
Ohren für das Geflüster in den Straßen. Wie Meeresrauschen schwoll es an.\\
\textit{Die Flut kommt.}\\
Ihr Leben war zu einem reißenden Fluss geworden. Sie saß auf einem löchrigen Floss und war der 
Strömung ausgesetzt. Lavay hatte keine Vorstellung, wo dieser Fluss enden würde, aber sie hatte 
aufgegeben gegen die Wassermassen zu kämpfen. Treibend schritt sie also durch ihr Leben, sah sich 
aufmerksam um und ließ sich kein Anblick, kein Gespräch, kein Geheimnis entgehen. Viele dieser 
Geheimnisse teilte sie mit der jungen Gräfin.\\
Lavay verharrte an einer Kreuzung und betrachtete eine kleine, verwackelte Kreidezeichnung. Ein 
simpler Stern. Oft schon hatte die junge Frau solche Markierungen gesehen und beobachtet. Sie 
wusste, dass dies ein Zeichen der Anhänger des alten Glaubens waren. Dadurch kommunizierten sie, 
vereinbarten Treffen oder zeigten ihre Zugehörigkeit.\\
Die Kasira trat einen Schritt zurück und sah sich um. Ihr Blick begegnete der einer Tempelwache. Zu 
Dritt marschierten sie auf die Kreuzung zu. Lavay befeuchtete ihre Lippen und entschied schnell. 
Hastig tauchte sie in der Seitengasse ab. Zwar brauchte sie zwei Anläufe, aber dann fand sie Halt 
und zog sich mit zitternden Armen hinauf auf das niedrige Dach. Flach ließ sie sich auf die Ziegeln 
rollen und hielt den Atem an. Die Tempelwache - nichts mehr als vom Hohepriester bezahlte Wachleute 
- bogen in die Gasse ein. Seit Sil'Veras Ankunft in Na'Rash hatte der Hohepriester Em'Hir Wachen 
eingestellt. Sie trugen nicht wie die Stadtwache Uniformen, sondern mit Stahl besetzte 
Lederrüstungen. Und Amulette, in denen ein Stück Kohle der Heiligen Flammen aufbewahrt wurde. Nach 
Lavays Geschmack waren diese Männer und Frauen viel zu versessen auf ihren Dienst. Sie hörte, dass 
einige sogar vom Dienst aus den stehenden Heeren abgetreten waren um den Priestern zu dienen. Lavay 
hatte versucht die Gräfin zu warnen. Aber in ihrem Stadtanwesen, umgeben von ihrer Garde und der 
Politik des nahenden Krieges war sie blind dafür, was auf den Straßen geschah. Selbst nachdem sie 
durch die Gassen, durch die Pfützen und über die Märkte gegangen war.\\
\textit{Sie denkt nur an sich. An ihr Kind. An ihren Krieg und ihre Macht.}\\
Aber Lavay hörte die Stimmen der alten Religion. Sie sah die Klingen Osymas. Lavay biss sich auf die Zunge 
und lauschte auf die Schritte der Tempelwachen. Die Männer hatten den Kreidestern entdeckt.\\
Lavay konnte es der Gräfin nicht böse nehmen. Nicht, seit das Mädchen sie nachts zu sich rufen 
ließ. \textit{Sie ist so jung.}\\
Ihre Freundschaft lebte in der Stille der Nacht. Nur ihre Garde hörten die junge Frau und das 
Mädchen flüstern. Kommen und gehen sahen sie Lavay nie, denn sie nahm einen Weg an der Hausfassade entlang. Trotz 
so vielen Unterschieden waren sie einander doch nicht fremd. Ob Seide oder Wolle. Ob Schmuck oder 
Narben. Sie kannten den Hunger. Sie kannten das Gewicht eines Mannes, den sie nicht wollten. Lavay 
bewunderte Sarimé für ihre Tapferkeit. Die Kasira wäre geflohen. Und sie dankte allen Wesenheiten, dass ihre Niederlage im Wald kein Kind in ihr hat wachsen 
lassen.\\
Die Stimmen der Wachen verklangen und Lavay wagte einen Blick hinunter auf die Straßen. Die 
Marktzeit war angebrochen und jeder schien auf den Weg dorthin. Geübt schob Lavay ihre Füße von der 
Dachrinne und ließ sich in die Gasse fallen. Sie mied die überfüllten Straßen und wandte 
sich nach Osten. Endlich kehrte der Winter ein. In ihrer Heimatstadt erstarrte der Fluss 
vermutlich bereits zu Eis. Das war immer eine gute Gelegenheit Geld zu verdienen. Damit der 
Schiffsverkehr nicht für die langen Wintermonate lahm gelegt wurde, bezahlte die Stadt Männer dafür, 
das Eis zu brechen. Bevor Imur Haska kennenlernte, hat er Lavay und ihre Mutter so über den Winter gebracht. 
Dafür wäre er aber auch 
einmal fast ertrunken und es grenzte an ein Wunder, dass ihm alle Glieder blieben. Die meisten 
verloren recht schnell Finger und Zehen an die Kälte.\\
Es begann wieder zu schneien. Lavay hob lächelnd den Blick zum Himmel. Nur eine kleine Ansammlung von grauen Wolken, 
der Rest erstrahlte im kühlen Licht der Wintersonne.
Die Kasira streckte ihre Hände empor, um die tanzenden Flocken zu spüren. \textit{Heimat.}\\
Ein harter Griff schloss sich um ihre Schulter. Lavay wirbelte herum, holte zum Schlag aus, da 
packte eine weitere Hand ihren Unterarm. ``Hab noch eine!'', rief die Tempelwache über ihren Kopf 
hinweg. Lavay zerrte und wandte sich, aber der Griff wurde nur fester. Hektisch sah sie über ihre 
Schulter, suchte nach Hilfe. Doch sie befand sich auf einen kleinen Hof abseits der belebten 
Straßen. \textit{Ich bin so dumm.}\\
``Ich habe nichts getan!'', brachte sie mühsam und mit starkem Akzent hervor.\\
``Eine Ausländerin'', stellte eine der näher kommenden Wachen fest und musterte sie kritisch.\\
Lavay hielt in ihren verzweifelten Versuchen inne und hob trotzig das Kinn. Sie sah ihm in die 
Augen, während er hinzufügte. ``Eine Kasira.''\\
Feindseligkeit lag in der Luft. ``Wir nehmen sie trotzdem mit. Kasira sind genauso gottlos wie die 
Ketzer.''\\
``Das dürft ihr nicht!'', rief Lavay: ``Die Gräfin...!''\\
Sie wollte so viel sagen, aber ihr fehlten die Worte. Und die Wachen hatten kein Interesse daran, 
ihr zuzuhören. Der Mann, der sie festhielt, schlug ihr fest gegen die Schläfen und ihr versagten 
die Beine. Nach Luft ringend ging sie in die Knie und wagte nicht erneut etwas zu sagen oder auch 
nur aufzublicken. Die Angst saß zu tief. Die Erinnerung - an den heißen Atem, den Gestank des Mannes 
und dem Waldboden - war zu lebendig.\\

``Hast du schon einmal darüber nachgedacht, doch lieber Kämmerer zu werden?'', unterbrach Sarimé 
den Kommandanten ihrer Garde. Er eilte, ebenso wie zwei weitere Kämpfer, neben ihr her durch die 
marmornen Gänge des Anwesens.\\
``Ihr hört nicht zu!'', fluchte Samos.\\
Sarimé blieb ruckartig stehen. Sie kam bereits zu spät zum Gericht. Die Frau ihrer Garde schaffte 
es gerade noch zu reagieren und trat eilig einen Schritt zur Seite, um ihre Herrin nicht anzurempeln. 
Zorn lag in Sarimés Stimme. ``Ich höre. Ich höre viel. Du bist nicht der Einzige, der mich mit 
Informationen überhäuft. Mi'Kae, Solvan, Arton. Der Bastard, meine Verwandtschaft, die Hebamme. 
Briefe vom König. Briefe von Kanto und Ringen. Briefe aus Kasir! Briefe von Händlern, vom 
Hafenmeister, von Buchhaltern! Und dann der Kommandant meiner Garde, der keine andere Aufgabe hat, 
als mich zu beschützen! Nicht, mich auf mögliche Fehler hinzuweisen.''\\
``Das ist dumm'', spuckte er aus.\\
Die beiden Gardenmitglieder tauschten unruhige Blicke.\\
Sarimés stimmte wurde leise. ``Du hast mir einen Eid geschworen, Samos. Was auch immer du dir dabei 
erhofft hast zu erreichen und dir nicht gelang. Was auch immer ich sage oder tue. Welche 
Entscheidungen ich auch treffen mag. Du hast mir einen Eid geschworen.'' Ihr Blick war kalt, als 
sie hinzufügte: ``Es hat seinen Grund, wieso man nur dem König sein Leben schwören sollte.''\\
Samos sah sie finster an, legte seine Faust auf sein Herz und kniete nieder. ``Ihr 
\textit{seid} meine Königin!''\\
Stille herrschte in dem vom Sonnenlicht erfüllten Flur. Die Gräfin blickte auf den blonden Haarschopf
hinab. Die Worte, die sie seit Monaten umschwebten, waren gefallen. Trotz des Eides und der Schwüre 
ihrer Garde, war dieses eine Wort stets zum Greifen nah und doch verborgen im Nebel geblieben.\\
\textit{Meine Königin.}\\
``Mein Ziel ist es, diesen Krieg zu gewinnen'', sagte sie und lief weiter.\\
Samos rappelte sich auf und rannte einige Schritte um wieder gleich auf neben ihr zu sein. ``Nein! 
Euer Ziel sollte sein, Merandila zu retten! Eine würdige... Gräfin zu sein. Und dafür dürft Ihr 
nicht zulassen, dass Merandila zum Schlachtfeld für Kasir und Saleica wird. Lasst nicht zu, dass 
unser Boden vom Blut getränkt wird, dass Jahrzehnte keine Ernte reifen wird. Lasst nicht zu, dass 
das Lebenswerk der Bauern und Handwerker innerhalb von Wochen und Monaten zerstört wird. Lasst 
nicht zu, dass unsere Kinder sterben, unsere Männer und Frauen in einen Krieg ziehen, der nicht der 
ihre ist!''\\
``Das ist nicht der richtige Ort für solche Worte!'', zischte Sarimé. Nach einem Moment der Stille 
fuhr sie fort: ``Außerdem versuche ich das bereits.''\\
Sie erreichten das geschlossene Portal. Auf der hölzernen Tür tobte ein eigener Krieg. Welche 
Geschichte auch immer diese Schnitzereien erzählen mögen, es war eine blutige. Samos trat näher. 
Schwer spürte sie seine Hände auf ihren Schultern. Sarimé musste den Kopf heben, um ihm in die 
Augen blicken zu können. ``Der Feind ist nicht im Norden, Herrin!'' Seine Worte waren leise wie ein 
Hauch, sein Blick um so fester. ``Er ist hier. Hier in dieser Stadt, auf der Straße, in den 
Anwesen. Im Tempel!''\\
Sarimé wandte den Blick ab, ohne ihm eine Antwort zu geben. Sie stieß die Tür auf und trat einen 
Schritt in den Raum. Alle Blicke richteten sich auf sie.\\
Auf einem Podium saßen die drei Stadthalter in ihren weißen Kitteln. Pedan ganz links, neben einem 
leeren Stuhl. Sein Äußeres war zerstreut wie eh und je. Ga'Leor behangen mit Schmuck und 
Edelsteinen. Sakan thronte in der Mitte. Das weiße Gewand hob seine blauen Tattoos hervor. Einige 
der Generäle, einschließlich Mi'Kae, saßen an der weiß verputzten Wand. Gegenüber waren einige Bänke 
für Besucher aufgestellt. Überwiegend Priester saßen dort. Sarimé entdeckte Silhe Basra und zu 
ihrer flüchtigen Überraschung auch ihren Stallmeister. Kein Schmuck zierte den Raum. Das 
Sonnenlicht fiel durch die Glasfenster und ließ den Boden brennen. Der rote Marmor glich einem 
Flammenmeer. An den Türen waren Wachen positioniert. Einige trugen das Kohleamulett aus dem heiligen 
Feuer im Tempel.\\
``Wir können anfangen'', beschloss Sarimé und trat auf das Podium zu den drei Stadthaltern und 
Richtern empor.\\
Pedan raschelte mit einigen Papieren und begann den Ablauf vorzulesen. ``Wir sitzen heute über 34 
Leben zu Gericht. Fünfzehn Männer und Frauen standen im Dienste der Garnison Effenberg unter 
General Leinos Kommando. Die genaue Anklage gilt es heute zu formulieren. Dann zwei weitere 
Rekruten, ebenfalls aus Effenberg, denen Mord an einem Ausbilder vorzuwerfen ist. Fünf Diebe. Drei 
Steuerbetrüger. Drei Scheidungsgesuche der Eheschließung. Fünf Vertragsstreitigkeiten.''\\
Pedan blätterte durch die Unterlagen und überflog die geschriebenen Zeilen. ``Und vier Randalierer, 
die Eigentum des Tempels, der Stadt und Privatbesitzt zerstörten. Sowie Schlägereien 
anzettelten.''\\
``Unsere Generäle haben einen Krieg zu führen'', befahl Sarimé: ``Also beginnen wir mit den 
militärischen Belangen, um ihre wertvolle Zeit nicht mit Banalitäten zu verschwenden.''\\
Ga'Leor las fünfzehn Namen vor. Fünfzehn Männer und Frauen in den roten Uniformen des Heeres 
marschierten in den Raum. Sarimé musterte jeden von ihnen genau. Das erste Wort gebührte ihr, auch 
wenn Sakan schien, als wäre er anderer Meinung. Die Soldaten blickten trotzig zum Podium empor. 
Ihre Uniformen waren sauber, ihre Stiefel poliert. Hätte man es ihnen erlaubt, wären sie mit 
Säbeln in den Gerichtssaal getreten. Der Inbegriff des saleicanischen Stolzes.\\
``Was trug sich zu?'', fragte Sarimé in die Stille hinein.\\
Ga'Leor war es, der den Bericht vor sich hatte und zusammenfasste: ``Die Truppe unter der Führung 
Sokra Árells erhielt den Auftrag, an der Grenze zu Kasir zu patrouillieren. Sie waren sieben Tage 
länger unterwegs als angedacht. Sie wurden dabei gesehen, wie sie nach Norden ritten. Über die 
Grenze hinaus. Der Truppe wird Befehlsverweigerung vorgeworfen.''\\
``Mehr nicht?'', fragte Sarimé und hob eine Augenbraue: ``Sie haben im Namen Osymas, Saleicas und 
König Semrics gemordet und geplündert.''\\
``Wir haben Krieg, Gräfin'', knurrte General Leinos: ``Da tötet man den Feind. Da plündert man 
die Dinge des Feindes.'' Er spuckte aus. ``Man vergewaltigt die Frauen des Feindes. Und die 
Kinder...''\\
``Genug!'', unterbrach Ga'Leor und funkelte den General wütend an.\\
Doch der General verschränkte nur die Arme vor der Brust. ``Graf A'Rik war sich diesen Tatsachen 
bewusst. Er war ein Krieger.''\\
``Ein Soldat ist kein Krieger.''\\
Die Stimme hatte ihre ruhige Besonnenheit inne, für der General aus dem Süden in Na'Rash bereits 
bekannt wurde. Sarimé presste die Lippen fest aufeinander. Sie war es, die diese fünfzehn Menschen 
vor Gericht stellte. Hätte sie es nicht verlangt, wären sie mit einer Rüge ihres Vorgesetzten davon 
gekommen. Nun wartete vielleicht der Tod auf jeden Einzelnen.\\
``Sprecht, General Mi'Kae'', gewährte Pedan ihm das Wort.\\
Jozah Mi'Kae wirkte einen Moment unschlüssig, ehe er sachlich ausführte: ``Ein Soldat befolgt 
Befehle. Er ist kein Held. Er ist kein Krieger. Er macht das, was er machen soll. Und in diesem 
Fall war der Befehl, an der Grenze zu patrouillieren. Dieser Befehl wurde missachtet.''\\
``Das Wort des Allmächtigen zu verbreiten, kann kein Verbrechen sein!'', fauchte Sakan.\\
``Aber auf das Wort seiner Generäle zu scheißen, schon'', sagte Ga'Leor entschieden und nickte seiner Gräfin zu.\\
Seinen Worten folgte Schweigen. Sarimé fuhr unbeirrt fort: ``Und wer auf das Wort seines Generals 
scheißt, scheißt auf das Wort seiner Gräfin. Also sprecht, Sokra Arell. Was brachte Sie zu dieser 
Entscheidung, Ihr Vaterland, Ihren General und mich zu verraten?''\\
Die breitschultrige Frau blickte zu dem Mädchen empor und fragte sich, ob das alles hier ein Witz 
auf ihre Kosten sein sollte. Das Gör war so zerbrechlich, Sokra könnte ihr vermutlich mit den 
Händen das Genick brechen. Sie räusperte sich und tauschte Blicke mit ihren Kameraden. Sie wusste 
nicht viel von dem Mädchen, dass sich Gräfin nannte. Aber sie wusste, dass die Rothaarige wohl 
mit dem alten Glauben der Merandil in Verbindung gebracht wurde, auch wenn sie bei der Segnung 
ihres Kindes ein ehrbares Vorbild dargestellt hatte. Also entschied Sokra sich für die einzige 
Antwort, von der sie hoffte, dass es ihren Kopf retten konnte. ``Der Priester sagte, wir sollen 
Osymas Wort verbreiten. Bevor der erste Schnee fällt.''\\
``Welcher Priester?'', fragte Ga'Leor nach.\\
Sokra hob die Hand auf ihre Brusthöhe. ``So ein kleiner... keine Ahnung wie der heißt.'' Sie zuckte 
mit den Schultern und sah ihren General an. ``Der war schon ne Weile in der Garnison und unserer Truppe zugeteilt.''\\
Stimmen wurden laut. Die Priester riefen. Die Generäle schimpften. Pedan raschelte in Unterlagen. 
Ga'Leor rief Sakan zu, er solle endlich seine tätowierte Schnauze halten.\\
Sarimé nickte Samos zu und der Kommandant stieß ein Brüllen aus, dass einem Löwen gerecht werden 
konnte. Die Stille kam nur widerwillig zurück. Immer wieder hörte man noch hastig gemurmelte Flüche 
und trotzige Antworten.\\
``Sie sind eine gestandene Frau'', fuhr Sarimé ihre Anklage fort: ``Seit Jahren im Dienste des 
Königs. Wollen Sie mir wirklich sagen, dass Sie nicht fähig sind, Ihrer Stellung angemessene 
Entscheidungen zu treffen? Sie sind keine Novizin des Tempels.''\\
``Wir wollten nur unserem König dienen'', antwortete Sokra und ihre Kameraden nickten zustimmend. 
``Und unserem Gott'', fügte sie hastig hinzu.\\
``Wie viele Zivilisten sind gestorben?'', fragte Ga'Leor und Pedan begann eilig zu blättern.\\
Wieder zuckte Sokra mit den Schultern und grinste sarkastisch. ``Wir haben nicht gezählt.''\\
Die Antwort war nicht gut, dass bemerkte die Soldatin gleich. Die junge Gräfin lehnte sich hinter 
ihrem Tisch zurück, ihre Miene blieb ausdruckslos.\\
\textit{Ihre Entscheidung ist gefallen.}\\
Die Verzweiflung stieg. Hilfesuchend sah Sokra zu den Priestern und Generälen.\\
``Ich habe viele Jahre meinem König treu gedient!'', rief Sokra laut. Furcht und Zorn lag in ihrer 
Stimme: ``Auch das war ein Dienst!''\\
``Ein Dienst, den er nicht befohlen hat'', entschied Ga'Leor.\\
Auch er hatte sich entschieden.\\
``Dann lasst sie im Tempel sühnen'', rief Sakan: ``Es liegt an Osyma ihre Buße anzuerkennen oder 
sie zu verurteilen.''\\
``Verehrter Stadthalter Sakan'', erinnerte Sarimé ihn sanft: ``So funktioniert unser Rechtssystem 
nicht. Noch nicht.''\\
General Leinos stand auf. ``Wir können es uns nicht leisten, tapfere und gute Soldaten 
hinzurichten! Sie sind nicht desertiert, sie haben für Saleica gekämpft!''\\
Die Gräfin blickte zu General Mi'Kae. Er räusperte sich. ``Vermutlich ist eine Hinrichtung 
hilfreich um weitere ähnliche Verbrechen und Missetaten zu verhindern.''\\
Sokras Hände begannen zu zittern. Sie hatte sich oft ihren Tod ausgemalt. Regelrecht geprahlt hatte 
sie bei abendlichen Trinkereien, dass es ihr nicht bestimmt war, alt zu werden. Sie würde für den 
König sterben. Ruhmreich und Geschichte schreibend. Oder vielleicht auch nur lautlos verklingen, 
während nur ihr König wusste, dass sie ihm sein Leben schenkte.\\
``Herrin!'', rief Sokra: ``Ihr sagtet es. Wir dienen dem König. Also soll er über uns richten!''\\
``Ich verwalte Merandila in König Semrics Namen'', erwiderte die Gräfin ruhig.\\
Sokra fiel auf die Knie. ``Wir haben unsere Namen unserem König gegeben. Wir haben ihm einen Eid 
geschworen!''\\
Die Züge der Gräfin wurden weicher. Mitleidig sprach sie: ``Aber unser König ist nicht hier.''\\
``Es ist Krieg. Er wird doch kommen'', sagte Sokra verblüfft: ``Er wird seine Heere anführen...''\\
\textit{So wie jeder König bisher.}\\
Jeder Anwesende schwieg betreten. Selbst die Gräfin senkte den Blick. Nur Sakans Lachen hallte 
durch den Raum. Sarimé Sil'Vera starrte den Stadthalter an und wartete, bis er verstummte. Es 
dauerte. Und jeder wartete mit ihr, gespannt auf ihre Worte, die ihr deutlich schon auf der Zunge 
lagen. Sakan schien es nicht zu stören, dass er Sokras Worte als einziger lächerlich fand. 
Entspannt lehnte er sich zurück, kicherte noch einmal und nahm einen Schluck aus seinem Becher voll 
Wein.\\
``Stadthalter Pedan'', bat die Gräfin: ``Schreibt einen weiteren Punkt auf die Liste.''\\
Pedan tauchte die Feder in schwarze Tinte und ließ sie über dem Pergament schweben, während er auf 
ihre folgenden Worte wartete. ``Wir sitzen heute über 35 Leben zu Gericht. Stadthalter Sakan, 
geweihter Priester Osymas, muss sich vor dem gewählten Vertretern Na'Rash, sowie der Gräfin Sarimé 
Sil'Vera und dem Volk Merandilas rechtfertigen.''\\
Sakan knallte seinen Becher auf das Podium. Blutroter Wein ergoss sich auf seine Papiere. ``Was 
wollt Ihr mir vorwerfen?''\\
Sarime beachtete ihn nicht, sondern wartete, bis Pedan geschrieben hatte und fuhr fort: ``Die 
Anklage lautet, Verrat an König Semric, Herr über Saleicas und Sohn Osymas, durch Verleumdung.''\\
Sokra entging nicht, wie die Krieger des Tempels einen Schritt auf die Mitte zu traten. Die Wachen 
der Gräfin zogen ihre Klingen. Ebenso drei der Generäle.\\
``Ich bin mir sicher, Osyma wird Euch als treuen Diener verzeihen, nachdem wir eine passende Form 
der Buße gefunden haben'', beendete Sarimé das Thema und wandte sich Sokra zu: ``Aber bringen wir 
nichts durcheinander. Sokra Arell. Sie sprachen im Namen Ihrer Truppe. Sie sagten, Sie wollten nur 
König Semric dienen.''\\
Die Soldatin sah, wie Sakan aufsprang und den Gerichtssaal verlassen wollte. Die blonde Wache an der 
Seite der Gräfin trat ihm in den Weg und legte ihm fast schon kameradschaftlich eine Hand auf die 
Schulter. Widerwillig zogen sich die Schläger des Tempels einen Schritt zurück.\\
``Ich spreche im Namen des Königs. Und ich bezweifle, dass er solche Diener wünscht.''\\
Sokra horchte auf. Nun würde ihr Todesurteil kommen.\\
``Die Angeklagten wünschten, sich vor unserem König zu rechtfertigen. Bis sie die Gelegenheit 
dazu haben, werden sie in den Zellen eingesperrt.''\\
Die Gräfin nickte Pedan zu und wartete darauf, dass das Kratzer der Schreibfeder verstummte. 
Niemand im Saal wagte zu sprechen. ``Und ich enthebe sie ihres Dienstes. Unser König kann keine 
ungehorsamen Diener gebrauchen.''\\
Und da kam dieses junge Gör und nahm Sokra das Ziel ihres Lebens.\\
Stumm und fassungslos sah sie auf den Boden. Hörte nicht, wie Stadthalter Ga'Leor die nächsten 
Angeklagten ausrief und mit einem Seufzer hinzufügte, dass dies noch ein langer Nachmittag werden 
würde.``\\
