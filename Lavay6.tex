\chapter{Die Helle lebt}

Die vergangenen Tage hatten Lavay eine Erkenntnis beschert. Sie wusste, woher der Stolz der 
Saleicaner kam. Sie waren frei. Sie konnten einander prügeln, ohne dass es die Stadtwachen störte. 
Sie konnten in Häuser einbrechen, ohne dass sich ein offizieller Richter einmischte. Sie konnten 
lieben, ohne dass Verträge unterzeichnet werden mussten. In Saleica galt das Gesetz der Tradition 
und das des Stärkeren. Der Adel wusste sich vor Dieben zu schützen, in dem seine angeheuerten 
Wachen keinerlei Einschränkungen genossen. Er konnte Gnade walten lassen oder hinrichten, ohne dass 
ein Richter nachfragen würde. Lavay war sich unschlüssig, wie sie dieses System finden sollte. 
Trotz den starren Strukturen in Kasir konnte sie nicht behaupten, dass sie sich in der Heimat 
sicherer gefühlt hatte. \textit{Ich stand auf der falschen Seite des Gesetzes.}\\
Als Kasira waren die Gesetze ihre Fesseln gewesen. Kaum ein Beruf, den sie durch ihren Stand 
ausüben durfte. Viele Orte, die sie nie betreten durfte. Hier, in Saleica, teilten sich reich und 
arm den selben Markt, die selben Schulen und den selben Tempel. Die Heimat war gefährlich, weil 
Lavay in Fesseln lag. Saleica war gefährlich, denn sie trieb hilflos in einem Fluss.\\
Es gab weniger Bettler. Weniger zwielichtige Gestalten. Weniger Arbeitslose. Es schien nur ein 
Gesetz zu geben. Das Wort des Allmächtigen. Nur einen Herrn, den König selbst.\\
Und vielleicht war es dieser Stolz, der die Saleicaner blind machte. Aber nicht taub. Das 
Geflüstert war unüberhörbar geworden.\\
Über Generationen hatten die Saleicaner sich eingebildet, ein Reich zu sein. Na'Rash teilte nicht 
in arm und reich, aber in Saleicaner und Merandil. 