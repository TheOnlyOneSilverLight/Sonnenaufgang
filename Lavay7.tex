\chapter{Die Helle lebt}

Noch keine Stunde war vergangen, seit die Wachen der Stadt Sokra und ihre Kameraden in die Zellen 
gebracht hatte. Demütigend wurden ihnen die Uniformen und Abzeichen entwendet. Es stank nach Pisse 
und Erbrochenem. Fauliges Stroh lag in den Ecken und die Ratten machten sich nicht einmal die Mühe, 
auf den Einbruch der Nacht zu warten. Es blieb nur ein kahlköpfiger Gefängniswärter, von dem der 
strengste Alkoholgestank ausging und ein hageres, weißhaariges Ding von Mädchen, welches immer und 
immer wieder mit einem Stein gegen einen Holzscheit klopfte. Die ehemaligen Soldaten wurden alle in 
eine der Sammelzellen geschubst. Sokra lehnte trotzig an der Wand, neben dem einzigen Fenster durch 
dem ein schmaler Lichtschein hereinfiel. Das schmutzig blonde Haar stand ihr wirr in alle Richtungen 
ab. Ihre Arme hielt sie provokant verschränkt vor ihrem Brustkorb und ihre Augen waren zu zwei 
Schlitzen zusammengekniffen. Keiner wagte es, ihr den schwachen Luftzug streitig zu machen. Im 
Gegenteil. Ihre Kameraden saßen im Dreck. Einige starrten finster vor sich hin, andere sahen aus, 
als würden sie gleich jammernd zusammenbrechen.\\
\textit{Wie sähen sie aus, wenn wir wirklich zum Tode verurteilt wären?}\\
Nun, dass würde Sokra in einigen Monaten ja noch herausfinden. Sie schüttelte zornig den Kopf. Sie 
hatte mit allem gerechnet. Hängen. Köpfen. Sogar den Scheiterhaufen. Bußestunden im Tempel. 
Degradierung. Aber völliger Ausschluss? Diese rothaarige Göre hatte ihr Leben vorerst verschont, 
aber ihre Ehre in den Dreck geworfen. Sokra hoffte, dass dieses Mädchen ihr Wort hielt, denn bis 
der König kommen würde, hätte sie die leidenschaftlichsten Worte gefunden, sie sie als Soldatin 
aufbringen kann. Er musste sie begnadigen! Er war ihr König! Der Mann, für den sie ihren Säbel 
hob. Sokra hatte König Semric noch nie gesehen. Und natürlich hörte sie die Beleidigungen und 
Geschichten. Doch ihr war es gleich. Er war ihr König. Scheiß auf Osyma. Scheiß auf die Priester.\\
Jetzt saßen sie hier, umgeben von Dieben und Trinkern.\\
\textit{Der König wird kommen...}\\
Sie hätte damals mit in die Kolonien ziehen sollen. So wie dieser mickrige General aus dem Süden, 
der vom König persönlich nach Merandila geschickt wurde. Dann hätte sie, Sokra Arell, Heldentaten 
vollbracht. Sie wäre vor ihrem König in die Knie gegangen, hätte seine Hand geküsst und wäre von 
ihm zum General befördert worden. Vielleicht würde sie dann der rothaarigen Schnepfe zur Seite 
stehen. Aber die Liebe hatte sie zögern lassen.\\
Sie spuckte auf den Boden, in Erinnerung an den Mann. Ein Mann, der ihr 
deutlich gelehrt hatte, dass das andere Geschlecht nur zum saufen und ficken gut war. Das Herz 
sollte man stets bei sich behalten. Dummerweise war diese Erkenntnis erst gekommen, als die Schiffe 
längst in See gestochen waren und eine weitere Versetzung in die Kolonien gestrichen war. Zu viele 
ihrer Landsleute waren ihr zuvorgekommen.\\


Lavays Kopf fühlte sich an, als wolle er zerspringen. Sie hustete trocken, was ihre raue Kehle 
schmerzhaft reizte. Lichter tanzten vor ihren Augen, während sie versuchte aufzustehen. Ihre 
Arme zitterten unter der Anstrengung und ihr gelang es lediglich, sich zu setzen. Sie saß in Mitten 
einer keuchenden und schluchzenden Menschenmenge. Hauptsächlich Frauen, Kinder und Alte. Ein 
Säugling schrie gleich rechts von ihr. Schniefend und blutend sahen die umstehenden einander an. 
Nach einigen Sekunden krochen ein junges Mädchen zu dem Baby und nahm es unsicher in die Arme. Das 
Schreien verstummte jedoch nicht.\\
``Was ist passiert?'', krächzte Lavay.\\
Das aufgeregte und angsterfüllte Gemurmel war zu unverständlich für sie. Niemand reagierte auf ihre 
Worte. Die Kasira rappelte sich mühsam weiter auf um sich umzublicken. Sie sah so viele Köpfe 
und Leiber, dicht an dicht gedrängt. Die Wände waren aus dunklem Holz, die Fenster hoch und 
schmal. Kaum Licht fiel herein. Sie erkannte den Ort kaum wieder, aber es konnte gar nicht 
anders sein. Dies musste eine der Markthallen sein, die Lavay vor wenigen Tagen an Sarimés 
Seite betreten hatte. Die Gräfin veranlasste, alles zu Räumen um es als Lebensmittellager für 
den Fall einer Belagerung zu verwenden.\\
Das Mädchen mit dem Säugling schob sich durch die Menge auf sie zu. Die Kasira biss sich auf 
die Zunge. Sie hatte den Fehler begangen, den flehenden Blick aus den rehbraunen Augen zu 
begegnen. In ihrem Versuch, dass ungeschehen zu machen, starrte Lavay auf den Boden und bemerkte 
trockenes Stroh und Lumpen. Und die öligen Flecken, welche selbst ihre Kleidung zierten. Flecken 
auf ihrer Haut. Unter ihren Fingernägeln, nach dem vergeblichen Versuch, das Öl abzuwischen. 
Selbst das farblose Haar des Säuglings war schwarz verschmiert. Es war überall.\\
Das Flüstern der Straßen war verstummt. Stattdessen trug der Wind Schreie und Flehen mit sich fort. 
Lavay legte den Kopf in den Nacken und starrte auf die hohe Decke der Halle. Immer wieder hörte sie 
den Namen der alten Göttin, aber sie selbst dachte an einen völlig anderen. \textit{Sarimé...}\\
Aber die Gräfin war unerreichbar. Sie war auf dem Hügel im Norden der Stadt. Dort, wo der Tempel 
Osymas thronte, gleich neben dem Schulgelände und dem Gericht. Dort, wo heute jede wichtige Person 
Na'Rashs versammelt war. Jeder, der die Schläger des Tempels hätte aufhalten können. Und jeder, der 
ein Alibi brauchte um nicht mit ihnen im Zusammenhang zu stehen.\\
Die Halle würde brennen. Vielleicht nicht genug um auf die Stadt über zu greifen. Aber die Flammen 
würden sich von Öl und Zunder nähren, zwischen den gestapelten Holz züngeln und an den Menschen 
lecken. Vielleicht nicht genug um sie alle zu Asche zu verbrennen. Aber genug um sie im Qualm zu 
ersticken. Und dann würden die Balken brechen. Funken sprühen. Das brennende Dach in sich 
zusammenstürzen.\\
Das Mädchen zog den Säugling fest an ihre Brust. Ihre freie Hand krallte sich schmerzhaft in Lavays 
Linke. Die Männer und Frauen Osymas brauchten keine Worte zu verschwenden. Sie entzündeten 
einfach nur die Feuer und lauschten, ob das Flüstern nun für immer aus den Straßen Na'Rashs 
verklingen würde.\\

Sokra sah auf. Suchte nach dem einen Etwas, was sich verändert hatte. Ihr Blick fand das magere 
Mädchen, welches mit dem Rücken zu ihr saß. Sie war erstarrt. Glich einer grotesken Skulptur. Als ob 
der Künstler einen Windgeist hätte meißeln wollen. Farblos und statt Fleisch und Leben nur Skelett 
und Leid.\\
Ohne ihr Klopfen herrschte gespenstische Stille. Nicht einmal der besoffene Wachmann schnarchte 
mehr. Auch er sah auf das Mädchen. Jeder sah auf das Mädchen, stellte Sokra fest. Auch ihr gelang 
es nicht mehr, weg zu sehen. Sie hielt den Atem an.\\
Ein Zittern ging durch den knochigen Körper. Die Hand schloss sich um den Stein und er zersprang. 
Sie stand auf, drehte sich um. Der graue Kittel reichte ihr kaum bis zu den Knien. Fasern des 
Stoffes standen ab und das ganze Kleid schien bei der nächsten Bewegung auseinander zu fallen. Ihr 
weißes Haar hing fettig und gerade an ihrem eingefallenen Gesicht. Aber etwas hatte sich geändert. 
Ihre Haut war nicht mehr kränklich grau. Sie schimmerte blass wie ein Königskind, welches noch 
keinen Tag in der Sonne war. Die Wunden und Kratzer heilten. Und ihre Augen...\\
Sokra fand keine Worte, um dieses Blau zu beschreiben.\\
Sie strahlte Kälte aus. Ruhe, fast schon Gleichgültigkeit lag in ihrem Blick, der von einem 
Gefangenen zum nächsten wanderte. Dann fand sie Sokra.\\
Das Mädchen hob die rechte Hand und einen Moment geschah nichts. Niemand in den Zellen schien sich 
zu bewegen. Zu atmen. Ein Knall brach den Zauber und die Männer und Frauen zuckten zusammen. Auch 
Sokra wäre einen Schritt zurück gewichen, wenn sie nicht schon die rauen Steine der Mauer an ihrer 
Schulter spüren würde.\\
Erstaunt öffnete sie den Mund, aber keine Worte kamen heraus. Sie ließ die Klinge, die lautlos aus 
dem Nebenraum durch den Türbogen schwebte, nicht aus den Augen. Ihr Säbel.\\
Die klauenähnlichen Finger des Mädchens schlossen sich um den mit schwarzem Leder umwickelten 
Griff. Einen einzelnen Schritt trat sie näher auf die Tür zu. Das Schloss knackte und schwang mit 
einem schrillen Quietschen auf.\\
``Du hast dieses Schwert dem König Saleicas geweiht'', sagte sie plötzlich.\\
Ihre Stimme war so rein und klar, wie ein Regen bei strahlendem Sonnenschein. Zumindest war es das 
Bild, was Sokra zu erst in den Sinn kam. Und so kalt wie Nebel, der sich auf die Haut legte und 
jede Pore benetzte.\\
Ihr Mund war trocken, aber sie schaffte es nach einer gefühlten Ewigkeit, ihr Gewicht auf beide 
Beine zu verlagern und ihre Arme zu lösen. Schnell huschten ihre Augen durch den Raum, aber keiner 
ihrer Kameraden schien ansprechbar. Sie alle starrten gebannt in die eisigen Augen.\\
Das Mädchen hielt den Säbel senkrecht vor sich und betrachtete die Spieglungen im polierten 
Stahl.\\
``Heute'', sagte sie bedächtig: ``Wirst du mir dienen.''\\
Sokra musste sich räuspern, ehe sie fragen konnte: ``Und wer bist du?''\\
Erstaunen lag in dem Eisaugen. Dann legte das Kind den Kopf schief und blickte tadelnd drein. ``Wir 
spielen kein Spiel, Sokra Arell. Es geht um das Leben deines Volkes. Um dein Heimatland.''\\
Die Soldatin musste all ihren Mut zusammen nehmen, um die nächsten Worte zu sprechen. ``Ich diene 
meinem König. Keinem Gott.''\\
``Ach? Ich höre sie schreien, Sokra Arell. Ich höre sie flehen und beten. Ich höre sie husten und 
sterben. Ich höre auch dein Flehen. Diene mir heute und dein Sehnen wird sich im Süden erfüllen.''\\
``Also doch ein Spiel. Einsatz gegen Einsatz.''\\
Das Mädchen legte die Stirn in Falten und schien über diesen Vergleich nachzudenken. Sie zuckte mit 
den Schultern und statt einer Antwort, hielt sie ihr den Säbel entgegen.\\
``Ich bin kein Spieler'', fügte sie dann doch hinzu: ``Ich habe nichts zu verlieren und nichts zu 
gewinnen.''\\
``Warum bist du dann hier?'', flüsterte Sokra.\\
``Ich habe es versprochen.''\\
Sokra ergriff ihren Säbel. Leicht und vertraut lag die Waffe in ihrer Hand.\\
``Diene mir heute, Sokra Arell'', wisperte das Mädchen: ``Und die Götter werden es nicht 
vergessen.''

Die Soldatin hob die Hand vor das Gesicht und sah sich um. Blenden stand die Mittagssonne in Mitten 
des blauen Himmels. Die Straßen und Gassen waren leer. Zögernd blickte Sokra zurück. Die Tür war 
halb geöffnet und zeigte den dunklen Raum dahinter, der zu den Zellen führte. Dort an einem kleinen 
Tischchen starrte eine junge Stadtwache blicklos an die Wand. Sie hatte dem Drang widerstehen 
müssen, ihm auf die Schulter zu tippen und zu erklären, wer sie befreite und warum sie jetzt mit 
einem Säbel bewaffnet an ihm vorbei spazierte. Sokra schüttelte den Kopf.\\
\textit{Ein dummes Spiel.}\\
Aber sie hatte einen großen Schluck von seinem Met genommen und den Dolch aus seinem Gürtel. Der 
steckte mittlerweile in ihrem Stiefel.\\
Sokra lauschte auf das rhythmische Klopfen, was wieder begonnen hatte, sobald die Zellen hinter ihr 
lagen. Anscheinend hatte das Kind wieder einen neuen Stein gefunden. Hatte sie vielleicht alles nur 
geträumt?\\
Unschlüssig sah die Frau sich um. Blickte von rechts nach links und überlegte, was genau ihre 
Aufgabe eigentlich war. Dann wanderte ihr Blick erneut in den Himmel und sie sah die Rauchschwaden. 
Schwer und dunkel hingen sie über den östlichen Vierteln Na'Rashs. Dort, wo die Handelskarawanen 
durch das Binnentor kamen um ihre Waren zu lagern.\\
Ihre Füße schienen den richtigen Weg zu finden, ohne dass Sokra weiter nach dem Weg suchen musste. 
Gespenstisch kam ihr das Geräusch ihrer Schritte vor, als sie durch die leeren Gassen rannte. Es 
war falsch. Eine Stadt wie Na'Rash sollte keine einzige leere Straße haben. Nicht zur 
Mittagsstunde, während die Herbstsonne alles in Gold und Orange tauchte. Sie sah Bewegungen hinter 
Fensterscheiben. Türen, die schnell zu geschlagen wurden. \textit{Wo bleibt der berüchtigte Mut der 
Saleicaner?}\\
Als sie Stimmen hörte, verlangsamte Sokra ihren Lauf und drückte sich an die Wand eines Hauses. 
Vorsichtig schob sie sich näher an die Kreuzung und spähte um die Ecke. Sie sah zahlreiche Menschen 
in den Kitteln der Priester. Sie kehrten ihr den Rücken zu, hielten einander an den Händen und 
sangen eine Preisung an ihren Allmächtigen. Einen Moment verlor Sokra sich in dem Anblick. Hinter 
den Singenden stand die Lagerhalle in Flammen. Das Feuer fraß sich gierig durch das Holz, knabberte 
an den Balken, sprengte die Ziegeln des Dachs. Blinzelnd riss sie sich davon los, atmete 
tief ein und aus um dann einen weiteren Blick auf das Szenario zu werfen.\\
Die Schläger versperrten die komplette Straße. Vermutlich sah es an anderen Kreuzungen nicht besser 
aus. \textit{Und wenn diese bleiche Göttin jetzt noch erklären würde, wie ich das anstellen soll?}\\
Sokra biss sich auf die Zunge. Schlagartig wurde ihr bewusst, um was es eigentlich ging. Da drin 
starben Menschen.\\
Ein leiser Pfiff weckte ihre Aufmerksamkeit. Wachsam sah sie sich um und suchte den Ursprung des 
Geräusches in der Gasse hinter sich. Ein drahtiger Mann winkte er flüchtig zu und bedeutete Sokra, 
ihm zu folgen. Sie warf einen letzten, zweifelnden Blick auf die Flammen, ehe sie sich abwandte und 
zurück schlich. Er wartete an der Mauer einer mit der Zeit alt und brüchig gewordenen Stadtvilla. 
Trotz der Eile, die sie hatte, trat sie nur langsam näher und musterte ihn genau. Er war ein 
winziges Stück kleiner als sie, in dunklen, einfachen Stoffen gekleidet und sein Haar hing wirr und 
schwarz in alle Richtungen. Sie konnte keine Waffe entdecken.\\
``Am Tor sind die Meisten dieser Idioten'', erklärte er leise und schnell: ``Die Nordöstliche Seite 
grenzt an den Hinterhof des Anwesens. Dort sind nur eine Handvoll.''\\
``Woher weißt du das?'', fragte sie misstrauisch und blickte an der Mauer entlang. Etliche Meter 
entfernt sah sie das Portal aus eisernen Stäben.\\
``Dort sind einige schmale Fenster des Lagers. Vielleicht schaffen wir es, ein paar Leute 
rauszubekommen...''\\
``Die verdammte Stadtwache sollte hier sein. Die Soldaten des Königs...''\\
``Wir sind die Soldaten des Königs'', sagte er entschieden, ging in die Knie und verschränkte die 
Finger um eine Tritthilfe zu bilden: ``Los, du zuerst.''\\
Sokra kniff die Lippen zusammen und warf einen letzten Blick hinter sich. Vielleicht würde sie ja 
doch plötzlich eine andere Alternative entdecken. Oder noch besser, die aufmarschierende 
Stadtwache. Das Einzige was sie entdeckte, war ein grauer Haarschopf hinter einem Fenster, der sich 
eilig wieder verbarg.\\
Sokra nahm Schwung, stützte ihren Fuß auf Die Hände des Fremden und stieß sich ab. Ihre Hände 
erreichten die obere Kante der Mauer. Mühelos zog sie sich empor, genoss das kurze Zittern in ihren 
Armmuskeln und stieß die Luft aus ihren Lungen. Die Arbeit auf dem Gut hatte sie schon als Kind 
ihrem Bruder überlassen. Sie war gerannt, geklettert, mit Stöcken gekämpft und mit Fäusten 
ausgeteilt. Es war für niemanden in der Familie Arell eine Überraschung gewesen, dass Sokra mit 
fünfzehn Jahren in Na'Rash vor der Tür der Kaserne stand. Über die Mauern und Dächer der 
merandilischen Großstadt war sie schon als Rekrutin des Nachts geklettert. Für Wetten und 
Mutproben, die sie selten verlor. Einen Herzschlag lang kauerte Sokra im Mittagslicht und 
betrachtete die Stadt, die sie seit ihrer Jugend nicht mehr aus dieser Perspektive gesehen hatte. 
Die ihres Ranges enthobene Soldatin sah hinunter zur Straße und begegnete das flüchtige Zucken 
eines schiefen Grinsens. Der Mann hielt ihr die Hand entgegen. Sokra setzte sich rittlings auf die 
Mauer und klammerte sich fest. Die freien Hände streckte sie ihm entgegen. Seine Füße stemmten sich 
gegen die Mauer und mit einem leises Keuchen schafften sie es beide auf die Mauer und dort zu 
bleiben. Stumm deutete er in eine Richtung und lief kauernd voraus. Sie folgte ihm, auf den Rauch 
und die Flammen zu.\\

An einen baufälligen Turm der Villa vorbei, erkannte Sokra schon den Garten, von dem der 
Dunkelhaarige gesprochen hatte. Sie hielt inne um die Umgebung zu beobachten. Fünf Kuttenträger. 
Drei davon größer als sie, alle bewaffnet. Der Boden war größtenteils eben, ein niedriges 
Dornengestrüpp zierte den Rand des Gartens. Qualm stieg aus den mit Holz versperrten Fenstern der 
Lagerhalle.\\
``Da ist deine Stadtwache'', wisperte er und deutete mit einem Nicken zu einem der Männer. Auch 
wenn sie sein Gesicht nicht sah, konnte Sokra sich denken, worauf er hinaus wollte.\\
``Los'', wisperte der Fremde und sprang von der Mauer.\\
Sokra zog ihren Säbel und folgte ihm. Der Sprung blieb nicht ungehört. Eine befehlsgewohnte Stimme 
rief zu den Waffen und die Umhänge raschelten. Sokra verzog das Gesicht, als ein stechender Schmerz 
von ihrem Knöchel aus in die Wade zog, aber da blockte sie auch schon ein Kurzschwert ab und 
leitete den Schwung des Schlags zur Seite fort.\\
Da war es wieder. Das Gefühl, als könnte sie Brüllen. Als könnte sie das Blut schmecken und den 
Schmerz nicht spüren. Die rothaarige Schnepfe konnte befehlen so viel sie wollte. Ihr Männer 
konnten Sokras Uniform und die Dienstabzeichen in den Müll werfen. Sie war trotzdem eine Löwin. Ihr 
Säbel trotzdem König Semric geweiht. Sie kämpfte nicht für verfluchte Götter! Nicht für 
tätowierte Priester oder weißhaarige Hexen. Nicht für Merandila. Nicht für Ehre. Sokra kämpften, 
weil sie eine Löwin war. Weil es das war, was sie war.\\
Sie trat einen schnellen Schritt vor, hackte ihren Fuß hinter den Knöchel ihres Gegners und zog ihn 
von den Beinen. Noch ehe er auf dem Boden aufkam, stieß ihr Säbel schnell wie der Kiefer einer 
Raubkatze zu und durchbohrte seine Organe. Die Schläger des Tempels hatten nicht mit Widerstand 
gerechnet, nicht einmal Leder trugen sie zum Schutz. So liebte Sokra den Kampf. Zwei Klingen und 
sonst nur Fleisch und Blut. Wenn noch deine Zähigkeit und Schnelligkeit mehr zählte als die Dicke 
deiner Rüstung.\\
Während sie sich mit dem Zweiten anlegte, war ihre Hand nass von ihrem Blut, aber sie zögerte 
nicht. Ihr Säbel schwang durch die Luft. Täuschte und zuckte. Sie hatte keinen Blick für den 
Fremden, der mit Fäusten und Tritten einen der Kuttenträger in den seligen Schlaf befördert hatte. 
Er hatte dem Vierten von hinten in der Mangel. Sokra beendete es schnell und sah sich nach dem 
letzten um.\\
``Er ist weg'', keuchte der Dunkelhaarige und wischte sich mit dem Arm den Schweiß von der Stirn: 
``Schon als wir sprangen.''\\
``Er wird Alarm schlagen'', rief Sokra.\\
``Er war nur ein Kind'', erwiderte er und untersuchte die Bretter. Der Qualm war dichter geworden. 
Das Gebäude strahlte Hitze aus. Und jetzt, wo der Rausch des Kampfes, der nur wenige Minuten 
gedauert hatte, vorbei war, hörte Sokra die Schreie.\\

Lavay zerrte das Mädchen hinter sich her. Sie rempelte Leute an, schob Kinder zur Seite und stieg 
über Kniende hinweg. Sie musste irgendetwas tun! Sie konnte nicht hier flehen und beten zu Göttern, 
die nicht die ihren waren und darauf warten, dass der Qualm sie erstickte. Der raue Stoff ihres 
Ärmels presste sich gegen ihre trockenen Lippen, aber Schutz gewährte es kaum. Sie war blind. Blind 
in einem Meer aus Menschen, die wogend und brausend all ihre Sinne betäubten. Es stank nach Öl, 
verkohlten Stroh, Urin und Erbrochenem. Schweiß lief ihr übers Gesicht. Vielleicht waren auch 
Tränen darunter, Lavay wusste es nicht.\\
Und dann sah sie es. Das Licht. Es fiel durch eine eckige Öffnung. Alle Menschen im Umkreis wandten 
sich in der selben Bewegung diesem Licht zu. Lavay senkte den Arm vor ihrem Gesicht um sich einen 
Weg zu bahnen.\\
``Die Helle kommt!'', schrie eine weibliche Stimme neben ihr.\\
Die Menschen drängten auf das Fenster zu. Wenige Meter daneben öffnete sich ein weiteres und noch 
eins und noch eins. Leute gingen zu Boden und wurden überrannt. Erstickte Schreie, die noch lauter 
zu werden schienen. Der Qualm überlagert das Licht und doch ließ niemand es aus den Augen. Nur der 
pochende Schmerz an ihrem linken Handgelenk zeigte Lavay, dass das Mädchen und der Säugling noch 
hinter ihr waren. Die Kasira schaffte es an die Wand der Lagerhalle. Sie müsste sich strecken um 
die Öffnung zu erreichen. Sie wirbelte herum, packte das Mädchen an der Schulter und schüttelte 
sie. ``Lass mich los!'', schrie Lavay sie auf kasirisch an.\\
Die Finger des Mädchens glitten beim nächsten Ruck ab. Ehe sie nach greifen konnte, sprang Lavay 
und bekam die Kante zu greifen. Sie spürte Luft an ihren Fingern. Reine und saubere Luft. Das 
Husten derer, die es schon raus geschafft hatten. Das Husten derer, die hinter ihr starben.\\
Das Mädchen krallte sich in den Stoff ihrer Hose. Doch Lavay trat um sich, während ihre zitternden 
Arme versuchten, sie in das Leben zu ziehen. Sie spürte einen Widerstand, der ihr den letzten 
nötigen Schwung gab. Eine Hand packte sie an den Haaren und zerrte.\\
Lavay kam in den Dornen auf, rollte sich zusammen und rang nach Atem. Gierig schnappte sie nach 
Luft und hustete den Qualm aus ihren Lungen. Ihre Augen tränten. Ihre Haut brannte. Erst jetzt 
zeigte ihr Körper durch Schmerzen die vielen Kratzer und Prellungen auf.\\
``Geh zur Seite!'', befahl jemand und schob sie grob von dem Fenster. Lavay schleppte sich 
kriechend und zitternd durch das Gras. Sie starrte auf ihre dreckigen Hände und schluchzte leise. 
Es war unnötig sich um zudrehen. Das Mädchen würde nicht kommen. Ihr Tritt hatte sie zu Boden gehen 
lassen. Und viele andere waren nachgerückt.\\

