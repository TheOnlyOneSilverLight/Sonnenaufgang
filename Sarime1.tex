\chapter{Ankunft}

Das Rucken und Stolpern der Kutsche war für Sarimé mittlerweile vertraut geworden. Sie wollte gar 
nicht wissen, wie viele blaue Flecken ihre Haut nun zierten, aber immerhin wurde ihr nach Wochen 
dieser Fahrt nicht mehr übel. Sie sehnte sich jedoch nach Gesellschaft. Der Kutscher war ein 
schweigsamer Mann, der vermutlich diese Reise ebenso verdammte wie Sarimé selbst.
\textit{Vielleicht wartet eine Familie auf ihn. Auf mich wartet keiner mehr.}\\
Sarimé musste bei diesem Gedanken seufzen. Seit Tagen hatte sie mit niemandem mehr gesprochen und 
die Einsamkeit nagte an ihr. Und das war mittlerweile, sie gestand es sich nur ungerne ein, ihre 
größte Furcht. Würde sie sich je wieder in einer Gesellschaft befinden, in der man sie mochte, wie 
sie war? Zu Beginn der Reise hatte ewige Stunden damit verbracht, sich auszumalen wie grausig ihr 
Leben werden würde. Ein fremder Gatte, den sie noch nie gesehen hatte, und der vermutlich drei mal 
so alt war wie sie selbst; eine fremde Grafschaft, weit weg von Familie und Bekannten. Die Ängste 
eines jungen Mädchens, welches in wenigen Tagen seinen sechzehnten Geburtstag erleben würde.\\
Der Aufbruch kam so unerwartet, obwohl Sarimé schon seit etlichen Jahren darauf gewartet hatte. Ihre 
Familie war verarmt, der König hatte sie vertrieben und somit den Ruf der Sil'Veras ruiniert. Aber 
ihr Vater und dessen zweite Frau hatten es geschafft, nach außen hin den Schein zu wahren. Sie 
lebten immer noch in der alten Stadtvilla im Kaufmannsviertel Brom-Dalars, aber lediglich ein Raum 
war für Gäste ihres Standes vorzeigbar. Der Rest bestand aus morschen Holzwänden. Das Parkett 
bestand aus Löchern. Wandbehänge und Bilder waren längst verkauft. Heimlich besorgte die einzige, 
alternde Bedienstete Lebensmittel auf dem Gesindemarkt.\\
Es war schwierig, die Familienverhältnisse klar zu deuten. Ihr ältester Bruder starb im Dienst des 
Militärs, hieß es. Genaugenomen hatte er sich auf eine dieser dummen Mutproben der Soldaten 
eingelassen und war betrunken vom Dach der Kaserne gefallen. Genickbruch. \\
Der zweite Bruder hatte der Familie vor zwei Jahren den Rücken gekehrt und war mit seiner Frau in 
die Grafschaft Ringen gezogen. Dort hatte er sich nach Sarimés Erkenntnis als Mitarbeiter eines 
kleinen Kaufmanns verdingt, der vermutlich ebenso arm war wie die Sil'Veras. Mires hatte Vaters 
zweite Frau nicht mehr ertragen. Oder besser gesagt, Mires' Weib hatte beschlossen, dass ihre 
Familie die Hauptstadt verlassen wird. Seitdem war Sarimé das letzte Kind aus erster Ehe. Die 
Stiefmutter hatte noch zwei weiteren Mädchen das Leben geschenkt, welche wohl einst das Erbe der 
Sil'Veras erhalten würden. \\
Sarimé lehnte sich zurück und ihr Kopf stieß gegen das harte Holz der Rückwand. Man könnte meinen, 
sie hätte eine Ausbildung erhalten, die über ihrem finanziellen Stand war. Ihr Vater hatte sie früh 
die Pflichten und Fähigkeiten eines Kaufmanns gelehrt. Sie konnte gut rechnen, lesen und schreiben. 
Sie wusste Finanzen zu regeln, sollten denn welche vorhanden sein. Ihre Stiefmutter hatte aus 
halbwegs ansehnlichen Stoffen Kleider selbst genäht, die zwei Schönsten wurden nur zu Spaziergängen 
und Tempelbesuchen getragen, um sie nicht abzunutzen. Sie bekam Schmuck aus Glas und Blech, gefärbt, 
um auf den ersten Blick hübsch und teuer anzusehen. \\
\textit{Eine junge Dame muss auf ihren Ruf achten.}, zitierte sie gedanklich die Worte iher 
Stiefmutter. Ihr Vater hätte niemals zugelassen, dass Sarimé so vergrault werden würde wie Mires. 
Dafür erinnerte sie ihn wohl zu sehr an ihre Mutter. Daher wählte Sarimés Stiefmutter einen anderen 
Weg. Sie bemühte sich, Sarimé mit den vorhandenen Mitteln so weit auszubilden und zu lehren, dass 
sie möglichst schnell verheiratet werden konnte. Und sie hatte es offensichtlich geschafft. \\
Der Aufbruch war so plötzlich, dass Sarimé kaum protestieren konnte. Ihr weniges Gepäck befand sich 
bereits in der Kutsche, sie selbst wurde grob hinterher geschoben. Der Kutscher schnalzte mit der 
Peitsche und schon konnte sie zu ihrem Vater nur noch irritiert zurück blicken. Sarimé wusste von 
den Schulden der Familie. Und sie wusste auch, dass ihre Stiefmutter jedem auf Brautsuche sofort 
Sarimés Bild zeigte und Lobpreisung auf sie sang. Aber welch Adeliger in der Hauptstadt wollte sie 
schon? \\
Sie waren verarmt, aber die Sil'Veras konnten sich immer noch des alten Namens und des Blutes in 
ihren Adern rühmen. Soweit Sarimé es beurteilen konnte, war es wohl genau dieser alte Name gewesen, 
der ihren Zukünftigen überzeugt hatte. Sie wusste nur wenig über ihn. Er war deutlich älter als sie 
und reich. Der Herr über eine Grafschaft, was himmelhoch über Sarimés momentanem Stand war. Die 
Angst war längst Verbitterung gewichen. Sarimé kratzte sich in hilfloser Wut über den rechten 
Unterarm. Als der Schmerz kam, bereute sie es sofort und rieb mit Tränen in den Augen die gerötete 
Haut. \\

Das einzige Buch, welches sie dabei hatte, war nach den letzten Wochen völlig zerlesen. Sie kannte 
es jetzt praktisch auswendig und das, obwohl sie die Geschichte nicht einmal mochte. Eine weitere 
Erzählung darüber, wie mutige Männer und Frauen in Osymas Namen Länder eroberten. Nebenbei geschahen 
ein paar Intrigen, Verräter wurden enttarnt und Liebe tauchte auch hin und wieder auf. Das störte 
Sarimé am meisten an der Geschichte. Sie glaubte nicht wirklich daran, dass es etwas wie Liebe gab. 
Als Kind hatte sie gemeint, es in dem Lächeln zu sehen, welches ihr Vater ihrer Mutter schenkte. 
Oder in deren sorgenvollen Blicken, wenn sie wiederum ihren Gatten ansah. \\
\textit{Einbildung}\\
Seine zweite Ehe war etwas völlig anderes. Die Beiden benutzten einander, um in der Gesellschaft 
möglichst gut dazustehen. Das letzte Paar, welches Sarimé als Vergleich heranziehen konnte, waren 
Mires und seine Frau. Diese Beziehung war schwierig. Ihre Schwägerin hatte offensichtlich das Sagen 
und Mires war ein Träumer. Aber diese Ehe war nicht arrangiert worden. Er stand plötzlich eines 
Tages mit ihr vor der Türe und erklärte, dass sie heiraten würden. Irgendetwas muss also vorhanden 
gewesen sein. \textit{Mehr als bloße Nützlichkeit?}\\
Die Kutsche blieb stehen und das abrupte Halten riss Sarimé aus ihren Überlegungen. Das Mädchen 
musste sich festhalten, um nicht vom schmalen, gepolsterten Sitz zu rutschen. Sie holte überrascht 
Luft. Das Hufgetrappel der beiden Kutschpferde war verstummt, stattdessen vernahm sie eifrige 
Schritte und Worte von Männern. \\
\textit{Ist es schon Abend? Nein, das kann nicht sein. Vielleicht blockiert etwas die Straße.}\\
Sie richtete sich auf und ihr entwich ein gequälter Laut. Ihr Rücken schmerzte fürchterlich von der 
unbequemen Fahrt. Fahrig strich sie sich eine gelockte rote Haarsträhne hinter das Ohr und atmete 
bewusst tief ein und aus. Langsam dämmerte ihr der Grund für das Anhalten. Sie hatten das Ziel 
erreicht. Sie hörte ihren eigenen Puls rasen und schloss einen Moment lang die Augen, um sich zur 
Ruhe zu zwingen. Jetzt gab es auch keinen Grund mehr, Panik zu bekommen. Dafür war es längst zu 
spät.\\
Jemand vor der Türe räusperte sich. ``Herrin?'', tönte eine junge Männerstimme.\\
Sarimé schlug die Augen auf und ihre Fingerspitzen strichen zögernd über den Vorhang. Sie hielt 
sich davon ab, ihn zur Seite zu schieben und einen Blick zu wagen. Das würde einen zweifelnden 
Eindruck hinterlassen und obwohl sie unsicher war, wollte sie das nicht gleich beim ersten Treffen 
mit ihrem Gatten zeigen. Sie erhob sich, so gut wie möglich, - die Decke der Kutsche war bei weitem 
nicht hoch genug um ihr einen aufrechten Stand zu ermöglichen - und öffnete die Türe von innen. Das 
Sonnenlicht floss herein und blendete sie einen Augenblick. Angestrengt blinzelnd erkannte sie eine 
dargebotene Hand, die ihr aus der Kutsche helfen wollte. Sie nahm diese Hilfe an, bis sie Boden 
unter ihren Füßen spürte. Eilig ließ sie die Hand fallen und blickte sich um.\\
Das Mädchen, welches den Prunk, die Enge und den Gestank der Hauptstadt gewöhnt war, musterte die 
Landschaft mit großen Augen. Während der Reise hatte sie öfters Gelegenheit gehabt, sich umzusehen. 
Aber meist war es spät abends, der Kutscher drängte zur Eile oder sie war schlicht völlig 
erschöpft. Außerdem was es anders, einen Wald zu betrachten, den man nie wiedersehen würde oder 
aber das Land, welches nun gezwungenermaßen ihr Zuhause sein würde. Die Festung Merandilas, Sitz 
des Grafen A'Rick und ihr zukünftiges Heim war im Licht der Nachmittagssonne fast schon malerisch 
anzusehen.
Fast wie einem kitschigen Roman entnommen.\\
Bei genauerem Hinsehen wirkte das eindrucksvoll große Gebäude weniger spektakulär. Die Festung war 
von einer drei Mann hohen Mauer umgeben. An manchen Stellen war sie von Ranken oder Moos bewachsen, 
an anderen zeugte helleres Gestein von Reparaturarbeiten. Und nach ihrem Eindruck hatten noch 
deutlich mehr Stellen eine Reparatur nötig. Dahinter sah man das ebenfalls steinerne Hauptgebäude. 
Winzige Fenster, ein mit Ziegeln gedecktes Dach. Der Rest der Häuser, die sie erahnen konnte, 
bestand aus Holz. \\
``Warum sind wir nicht bis hinein gefahren?'', fragte Sarimé.\\
Der junge Mann trug die Kleidung eines Bediensteten und verneigte sich ein weiteres Mal. ``Verzeiht 
Herrin, aber das Tor ist nicht groß genug für Kutschen.''\\
``Junge Herrin'', sprach eine weitere Stimme sie an.\\
Sarimé wandte sich dieser zu. Sie hatte bis jetzt ihre Aufmerksamkeit lediglich auf das Gebäude 
gerichtet gehabt. Der Mann war älter als sie. Vielleicht Anfang zwanzig. Er hatte schwarzes Haar, 
welches ihm bis zum Nacken reichte und dunkle, graue Augen. Sein Blick ruhte beunruhigend fest auf 
ihr und sein Lächeln blieb durchgehend.\\
``Seid willkommen, Sarimé Sil'Vera. Wir haben Eure Ankunft sehnsüchtig erwartet.''\\
Einen Moment lang erwiderte Sarimé das Lächeln. Einen Moment lang flüchtete sie sich in die 
Vorstellung, dass ihr baldiger Gatte gar kein alter, hässlicher und verbitterter Greis war, sondern 
dieser junge, charmante Mann. Er verbeugte sich und hauchte einen Kuss auf ihren Handrücken. Sie 
spürte lediglich einen flüchtigen Atemzug.\\
``Gestatten, Renec. Ich bin der Bastard des Grafen.''\\
\textit{Renec. Der Bastard}, dachte sie enttäuscht. \\ 
``Wann... lerne ich den Grafen kennen?'', fragte sie leise.\\
``Der Graf ist heute Morgen zu einer der Garnisonen aufgebrochen. Er wird mit Euch zu Abend 
speisen, Herrin. In Eurem Gemach wartet ein heißes Bad auf euch.''\\
``Sehr freundlich'', murmelte Sarimé.\\
 
Sie hätte jetzt am Liebsten einfach geschlafen; die Augen geschlossen und sich vorgestellt, all dem 
zu entrinnen. Der Bastard wandte sich an den Kutscher, lud ihn in die Küche ein und überreichte ihm 
seinen Lohn. Grummelnd wurde das Angebot abgelehnt und der Kutscher erklärte, er würde sich auf den 
Heimweg machen und in einer Gaststätte übernachten. Sarimé stand auf der gepflasterten Straße und 
blickte der Kutsche hinterher. Da verlor sich die letzte Verbindung zu ihrem alten Leben. Irgendwie 
hatte sie sich mehr erhofft. Während des Aufbruchs hatte sie nur ein müdes Winken von ihrem Vater 
erhalten, ehe die Kutsche hinter der Kurve verschwand. Mit diesem Kutscher war sie nun mehr als 
drei Wochen gereist und nun verschwand er, ohne sich ihr noch einmal zuzuwenden. Sarimés Augen 
begannen zu brennen. \textit{Ich bin eben doch nur ein kleines, dummes Mädchen.}\\
``Herrin?'', räusperte sich der Bastard respektvoll. \\
Sarimé straffte sich ruckartig und nickte ihm leicht zu. Sie blickte an sich herab und zupfte in 
einem kläglichen Versuch, ihr zerknittertes Reisekleid zu richten, an dem Stoff ihres Rockes. \\
``Vielleicht ist es besser, dass der Graf mich nicht in diesem Zustand sieht'', bemerkte sie. Es 
war mehr laut ausgesprochener Gedanke und sie schalt sich gleich eine dumme Kuh dafür. Mit der 
alten Magd ihrer Familie war sie sehr vertraut gewesen. Zeitweise vertrauter als mit ihrem Vater. 
Aber durfte eine baldige Gräfin sich das auch leisten? Sarimé befürchtete ein Leben in Einsamkeit 
auf sich zukommen und sie senkte ertappt die Lieder. Der Bastard ging auf diese Peinlichkeit nicht 
ein, als hätte er kein Wort gehört. Stattdessen schickte er den Bediensteten, der sich anscheinend 
um die Pferde des Kutschers hatte kümmern sollen, wieder in die Festung. Sarimé war erleichtert, 
dass der Bastard ihr nicht die Hand anbot. Einerseits war es unter seinem Stand, andererseits konnte 
sie gut auf Körperkontakt zu fremden Personen verzichten.\\
Die Beiden traten nebeneinander durch das Tor, welches zwar breit war, aber in der Höhe gerade groß 
genug um einen Reiter im aufrechten Sitz hindurch zu lassen.\\
``Warum ist das Tor zu klein?'', fragte sie neugierig nach.\\
``Die Mauer steht schon seit etlichen Generationen. Der Grundriss bestand schon, als Merandila noch 
ein eigenständiges Land war. Als es zur Grafschaft wurde, erweiterten die Verwalter des neuen 
Königs die Mauer nicht. Für die damaligen Anforderungen war er ausreichend, außerdem weigerten sich 
die Bauarbeiter, die Mauern zu verändern.''\\
``Warum?''\\
Der Bastard zuckte mit den Schultern. ``Menschen sehen nicht gerne, wie alles, was sie kennen und 
lieben, zerstört wird. Für das Volk hat der Machtwechsel gereicht und die Arbeiter haben 
gestreikt.''\\
``Aber jetzt würden sie die Befehle ihres Grafen befolgen?'', fragte Sarimé vorsichtig.\\
``Gewiss, Herrin. Habt keinen Grund zur Sorge.'' Er lächelte bei diesen Worten.\\
Der Hof war vollständig gepflastert. Fast eine komplette Mauerseite wurde von einem länglichen 
Stall eingenommen. Wenige Meter daneben erhob sich ein gemauerter Brunnen. Zwei Frauen standen dort 
und kurbelten den Wassereimer wieder an die Oberfläche. Renec bemerkte ihren Blick und erklärte mit 
sichtlichem Stolz: ``Die Stallungen des Grafen. Exzellente Pferde aus den besten Zuchten 
Merandilas. Schnell und ausdauernd, gehorsam und intelligent.''\\
``Klingt mehr nach den Pferden von Jägern, als nach denen von Kriegern.''\\
``Die Merandil waren nie große Krieger, daher haben sie damals auch die Eroberung der Saleicaner 
nicht aufhalten können.''\\
``Und Jäger?'', fragte Sarimé.\\
Wieder lächelte er. Ein arrogantes Lächeln, welches seinem Stand gewiss nicht entsprach. Aber er 
zeigte es nur den Bruchteil einer Sekunde, sodass Sarimé sich nicht sicher war, ob sie es wirklich 
gesehen hatte. 
``Jäger lauern im Schatten des Waldes auf ihre Beute. Sie stellen eher Fallen, als dass sie 
brüllend und mit fuchtelnder Keule auf ein Reh zustürzen.''\\
``Das Reh würde ja auch weg rennen.''\\
``Nur wenn es merkt, dass es schwächer ist. Der Löwe Saleicas würde dem Jäger ebenfalls brüllend 
entgegen stürzen und ihn in Fetzen reisen.''\\
``Für Ehre und Osyma'', fügte Sarimé seinen Worten hinzu. Einen Moment sahen sie sich schweigend an 
und sie empfand einen Einklang zwischen ihnen. Immer noch war sie zweifelnd und unsicher, aber 
immerhin schien dieser Renec kein fanatischer Anhänger Osymas zu sein. \\
Sarimé würde viel für einen angenehmen Gesprächspartner in ihrem Alter geben. \textit{Er ist ein 
Bastard. Der Bastard meines baldigen Gatten. Ich sollte ihn nicht einmal anlächeln.}\\
``Interessiert Ihr Euch für Pferde?'', fragte er höflich, da ihr Blick wieder am Stall hängen 
blieb.\ 
``Zuhause hatte ich Unterricht. Aber ich ritt noch nie außerhalb Brom-Dalars.''\\
An seinem Blick bemerkte sie, dass er nur höfliche Konversation betrieb. Es interessierte ihn nicht 
wirklich. Er war höflich, immerhin stand die baldige Gattin seines Vaters vor ihm, aber er hatte 
kein wirkliches Interesse an einem Gespräch. Also ergriff Sarimé selbst die Initiative und trat mit 
großen, zielstrebigen Schritten auf das Eingangsportal des steinernen Gebäudes zu. Wieder spürte 
sie die Tränen aufsteigen. Wieder eine Hoffnung zerfloss wie flüssiges Wachs. \\


Das heiße Wasser prickelte auf ihrer Haut, als sie sich vorsichtig hinein gleiten ließ. 
\textit{Immerhin spüre ich noch etwas.}\\
Seit Sarimé in der Festung Merandilas angekommen war, fühlte sie sich wie betäubt. Alles glitt an 
ihr vorbei, wie Treibgut in einem Fluss. Vielleicht war auch sie das Treibgut. Sie setzte sich auf 
den Boden der Wanne, umfasste ihr hüftlanges, rotes Haar und hob es über ihre Schulter nach vorn. 
Sie befand sich gerade in einem der zwei Nebenräume ihres Gemachs. Das Hauptzimmer selbst war so 
groß wie ein Erdgeschoss in der alten Stadtvilla der Sil'Veras. Aber in der Hauptstadt wurde in die 
Höhe gebaut, nicht in die Breite. Dieser Waschraum hier war deutlich kleiner. Lediglich die Wanne 
befand sich darin, ein Spiegel und ein schmales Fenster.\\
\textit{Der kommende Winter wird wohl der kälteste in meinem bisherigem Leben.}\\
Sie seufzte tief und schloss die Augen. Ihr Ziel war, sich fort zu träumen. Zu einem Paradies, einem 
zeitlosen Ort, in der es keine Angst gab, keinen alten Bräutigam und keine Pflichten. Im Hintergrund 
klapperte die Zofe und störte Sarimés Versuche. Als sich plötzlich heißes Wasser über ihren Kopf 
ergoss, entwich ihr ein hoher Schrei und sie riss die Hände schützend vor ihr Gesicht. Die Zofe, ein 
älteres, hageres Weib mit giftigem Blick, stand mit dem Eimer in der Hand über ihr und fragte 
süffisant: ``Zu heiß, Herrin?''\\
``Nein'', erwiderte Sarimé bemüht gelassen: ``Nur scheint der Dreck der Straße sich noch nicht fort 
waschen zu lassen. Es stinkt. Oder bin ich das etwas gar nicht?''\\
Man sah der Zofe an, dass sie sich kaum zusammen reißen konnte. Sarimé hielt ihrem Blick stand. 
Zuhause wäre es zu einem Streit gekommen und die Bedienstete hätte gekündigt. Aber sie war nicht 
mehr die Tochter eines verarmten Kaufmanns, sondern die Braut eines Grafen, der im Namen des Königs 
das Land verwaltete. \textit{Ich darf das nicht runter schlucken}, versuchte sie, sich 
selbst Mut zu machen: \textit{Sonst wird es schlimmer werden.}\\
``Du darfst gehen, ich komme jetzt alleine zurecht.'', erklärte das Mädchen und deutete auf die 
Türe, ohne den Blickkontakt zu unterbrechen: ``Vielen Dank für deine Dienste und das heiße 
Wasser!''\\
Die Zofe schloss die dünne Schiebetüre mit einem kraftvollen Ruck. Die Tür vibrierte noch lange 
nachdem Sarimé bereits alleine war. Sie seufzte schwer und alle Kraft schien ihr mit diesem Atemzug 
zu entweichen. Mit leeren Gedanken blickte sie an sich herab. Das heiße Wasser umschmeichelte ihren 
zierlichen Körper, entspannte die verkrampften Muskeln. Rosenblätter trudelten an ihren Knien 
vorbei. Ihr rotes Haar wirkte unter Wasser dunkel und bauschte sich auf wie eine Qualle. Ohne Luft 
zu holen, sank sie tiefer in die Wanne. Sie war klein genug, um darin komplett unterzutauchen. 
Langsam öffnete sie ihre Augen und starrte zur Oberfläche. Rosenblätter und Haarsträhnen trudelten 
durch ihr Sichtfeld. Auf der anderen Seite erkannte sie nur Licht. Es war, als würde man in eine 
andere Welt blicken, in der es keine Schatten und keine Dunkelheit gab. Nur ein gleißendes, 
vollkommenes Licht. \\
Sie wollte hier bleiben. An diesem stillen Ort. Aber ihr Körper entschied anders. Instinktiv schoss 
ihr Kopf empor, ihre Lungen sogen die plötzlich kalt gewordene Luft tief ein und ihre Hände 
krallten sich um den Rand der Wanne. Das Haar klebte ihr am Kopf und im Gesicht; Wasser ran 
strömend an ihr herunter. Ihr Oberkörper bebte und die Tränen verloren sich im nach Rosen duftenden 
Badewasser. Sarimé verbarg ihr Gesicht in den Händen und trotz des heißen Wassers zitterte sie 
schlimmer als eine Erfrierende.\\
Das Geräusch der Schiebetür riss sie aus ihrer Verzweiflung und sie blickte hoch. In der Tür stand 
der Bastard, seine Hand umschloss noch den Türrahmen. Einen Moment sahen sie einander stumm an. 
Sarimé rann eine weitere Träne über die Wange. Renec musste sie gehört haben, sonst wäre er niemals 
gekommen. Er hatte gelauscht und hatte ihre Schwäche nicht mehr ertragen.\\
``Dreh dich weg!'', brach sie schließlich die Stille.\\
Seine Augen weiteten sich kurz, dann färbten sich seine Wangen rot und er wandte sich ruckartig dem 
Hauptzimmer zu. ``Verzeiht, Herrin'', stammelte er: ``Ich hörte Euch weinen und... machte mir 
Sorgen.''\\
``Frauen weinen oft'', erwiderte Sarimé, erhob sich eilig aus dem Wasser und tastete nach dem 
wärmenden Mantel. Das Badewasser plätscherte, als sie aus der Wanne stieg.\\
``Manche'', sagte Renec: ``Und manche nicht.''\\
Sarimé, eingehüllt in das Gewand, betrachtete ihn daraufhin. Schließlich nickte sie, obwohl er 
weiterhin in die andere Richtung blickte. ``Du kannst dich wieder umdrehen. Und sage mir, was du 
wirklich wolltest.''\\
Er sah sie nicht an, als er antwortete. ``In diesen Zimmern wurden schon zu viele Tränen vergossen. 
Als ich es das letzte Mal ignoriert habe, ging es nicht gut aus. Ich wollte mich vergewissern, dass 
es Euch gut geht, Herrin.''\\
\textit{Es mir gut geht?} Sie würde ihm so gerne glauben. Sie würde so gerne glauben, dass es noch 
einen Menschen gab, dem sie wichtig war. Nur einen einzigen. Selbst wenn es ein Bastard wäre.\\
``Du hast nicht zufällig die Zofe gesehen?''\\
``Sie ist in Trauer, Herrin. Nehmt es ihr nicht übel'', verteidigte er sie.\\
``Kann ich mir selbst eine aussuchen? Ich will sie nicht. Eigentlich brauche ich überhaupt keine'', 
erklärte das Mädchen und trat in das Hauptzimmer. Es gab nur winzige Fenster, welche kaum Licht 
herein ließen. Ansonsten waren die Räumlichkeiten ansprechend. Spartanisch eingerichtete Zimmer 
waren ihr vertraut und die wenigen Möbel waren ansehnlich. Auf dem großen Bett lag eine bequeme 
Matratze und eine Vielzahl verschiedener Kissen und Decken. Die Vorhänge des Bettes ließen sich 
zuziehen und dann wirkte es wie ein geheimes Versteck. Das war das Erste, was Sarimé noch vor dem 
Bad ausprobiert hatte. Vor dem Bett lag ein weiches Fell, in das sie sich am Liebsten sofort 
eingehüllt hätte. Welches Tier es einst wohl gewärmt haben musste, konnte Sarimé nicht sagen. Sie 
kannte sich mit den Tieren des Nordens nicht aus.An der Wand stand eine Truhe, in deren Holz  sich 
Schnitzereien befanden. Man konnte ein tanzendes Paar, galoppierende Pferde, fliegende Falken und 
allerlei verschlungene Muster erkennen. An den Wänden hingen Gemälde von Landschaften. Das Meer, 
eine Schneewelt und Berge. Auf einem wuchtigen Schreibtisch war zahlreicher Schmuck ausgebreitet, 
den eine Bedienstete wohl noch zu putzen und polieren gedachte. \textit{Echter Schmuck.}\\
Aber Sarimé hatte diesen noch kaum eines Blickes gewürdigt. \textit{Stattdessen musste ich wie ein 
kleines Kind das Bett ausprobieren}, dachte sie und verzog das Gesicht über sich selbst.\\
``Eine Gräfin braucht eine Zofe'', meinte der Bastard: ``Jemand muss Euch helfen, ein passendes 
Kleid zu finden und Euch herzurichten.''\\
Mit einem leisen Lächeln sah sie ihn an. ``Es ist aber niemand da. Außer du natürlich, Bastard.''\\
``Ich werde sie suchen gehen.''\\
``Wozu? Ich richte mich für einen Mann her, dann kann ich auch den Rat eines Mannes einholen. Ich 
mag sie nicht. Und wenn du es so eilig hast, zeig mir die Kleider. Keine Sorge, ich kann mich 
selbst frisieren.''\\
``Herrin...''\\
``Das war ein Befehl'', erläuterte Sarimé beiläufig. \\
``Ihr scherzt?'' Der Bastard war irritiert und blickte sie ratlos an. Sarimé musste grinsen. Wieder 
spürte sie den Einklang und sie betete zu Osyma, dass er wirklich bestand und nicht nur Einbildung 
war. Sie brauchte einen Freund in diesem Feindesland. Jemanden, vor dem sie lachen konnte, plaudern 
und scherzen. Sie trug die Maske der Gräfin noch nicht einmal zwei Stunden und war bereits völlig 
erschöpft. Und das wusste er. Sarimé sah es in seinen Augen. \\
``Ich meine es ernst. Du bist ein Bastard und kein Höfling, kein Adeliger, kein hochrangiger 
Krieger. Die Rolle, die du spielst, passt nicht zu dir.''\\
``Die Herrin kennt so viele Adelige?'', entgegnete er und blickte sarkastisch drein.\\
\textit{Er weiß von der finanziellen Lage meiner Familie.}\\ 
``Der saß'', entgegnete sie: ``Können wir nun aufhören uns gegenseitig zu beleidigen?''\\
``Gerne. Das ist auf Dauer bestimmt anstrengend!''\\
Sie zögerte. ``Ich heiße Sarimé.''\\
Als hätte sie ihm erzählt, sie würde sich gleich in einen Drachen verwandeln, wurde Renec 
schlagartig wieder ernst. ``Ich weiß, Herrin.''\\
Tränen stiegen erneut in ihr auf. Sie hatte ihm Vertrautheit und Freundschaft angeboten, aber der 
Bastard hatte brüsk abgelehnt. \textit{Bilde ich es mir nur ein? Bin ich so 
verzweifelt?}\\
``Du kannst gehen. Ich werde mich ankleiden'', sagte sie leise und ihr Blick ruhte dabei auf 
dem Fell, welches am Boden lag. Sarimé wagte nicht, noch einmal den Blick zu heben, noch einmal in 
die Sturmaugen zu blicken und noch einmal enttäuscht zu werden. Dieses Mal war er derjenige, der 
zögerte. ``Darf ich Euch anschließend in der Festung herumführen?''\\
Sarimé nickte fahrig. ``Warte vor der Tür.''\\


Renec führte sie durch die Festung und zeigte ihr die wichtigsten Räumlichkeiten. Sie kamen an dem 
Speisezimmer vorüber, in dem ihr Verlobter sie am Abend erwarten würde. Sie warf einen Blick in das 
Arbeitszimmer des Grafen und auf die verschlossene Türe seines Gemachs. Sie besuchten die Küche, 
die Wäschefrauen in ihren Kammern und der Bastard stellte ihr einige Dienstmädchen vor. Schließlich 
verweilten sie auf der Terrasse, die von der Eingangshalle aus betreten wurde. Verwelkte Pflanzen 
begrüßten sie. Die Fließen waren schmutzig und rissig. Vertrocknete Blätter lagen vereinzelt, vom 
Wind hier her getragen, auf dem Boden. Sarimé setzte sich auf die steinerne Bank und legte die 
Hände in den Schoß. \\
``Um wen trauert die Zofe?'', griff Sarimé das Thema auf.\\
Renec lehnte sich an das Geländer und blickte hinab. Zwei Meter unter der Terrasse befand sich 
etwas, was wohl einst ein Garten war. Die Grünfläche war eben und es schien, als hätte man eigens 
dafür Erde abgetragen, um Platz zu schaffen. Unkraut und Gestrüpp überwucherten den kleinen Platz. 
In diese Dornen würde Sarimé niemanden schicken wollen.\\
``Um die Gräfin'', erklärte er versonnen, ohne sich ihr zuzuwenden.
\textit{Irgendwas ist da doch}, grübelte sie, \textit{wenn es ihm nur um seine Pflichten ginge, 
dann würde er respektvoller mit mir reden. Wenn er überhaupt kein Interesse hätte, dann würde er es 
gar nicht tun. Oder?}\\
``A'Riks verstorbene Frau?''\\
Er nickte. ``Sieva.''\\
``Bin ich also seine zweite Braut?''\\
``Die Dritte, um genau zu sein. Seine erste Frau schenkte ihm drei Erben und starb mit dem Vierten 
im Kindbett. Ein Sohn stürzte vom Pferd und erwachte nicht mehr, obwohl sein Herz noch einige Tage 
schlug. Der Älteste wurde krank. Der Mittlere starb zuletzt an einer Infektion, während er in einer 
anderen Grafschaft zu Gast war. Das ist erst sieben Jahre her.''\\
Sarime betrachtete ihn neugierig. ``Und dann kam Sieva.'' Sie konnte nur Renecs Profil sehen, aber 
erahnte ein verträumtes Lächeln.\\
``Drei Monate später, ja. Der Graf lernte sie in der Tempelstadt Na'Rash kennen. Ihre Mutter hat 
sechs gesunden Kindern das Leben geschenkt, ihre zwei Schwestern hatten auch schon jeweils drei. Er 
erhoffte sich, dass die Fruchtbarkeit an Sieva ebenfalls vererbt wurde. A'Rick braucht anerkannte 
Erben, die die Grafschaft weiter führen werden. Seine zweite Frau schenkte ihm keinen einzigen.''\\
\textit{Und jetzt bin ich an der Reihe.}\\
``Kennt man Sievas Familie?''\\
Renec schüttelte verneinend den Kopf. ``Reich, aber junger Adel. Sie haben sich den Titel 
vermutlich erkauft und so etwas bedeutet beim alten Adel nichts. Unter anderem darum wurde ihre 
Familie gemieden. Sie war weit unter seinem Stand, aber meinem Vater ist ein Erbe wichtiger.''\\
``Mein Name ist der Grund, warum ich hier bin, oder?'', fragte Sarimé leise.\\
Er zog die Schultern hoch und wählte seine Worte weise. ``Alter Adel hat sich seit Jahrhunderten 
bewiesen. Außerdem ist es für das Land besser. Falls er ohne Erben stirbt, wird der König einen 
seiner nahestehenden Vertrauten schicken, der die Witwe heiratet und die Grafschaft übernimmt. 
Königliche Vertraute heiraten nicht gerne namenlose Witwen.''\\
Sie biss sich auf die Zunge. Einerseits hatte sie wirklich gehofft, dass ihr Gatte bald zu Osyma 
kehren würde, wenn er wirklich so alt war, wie sie befürchtete. Andererseits hatte Sarimé gar nicht 
daran gedacht, dass dann ein anderer kommen würde. ``Wann ist Sieva verstorben? Und wie? Erzähl mir 
von ihr.''\\
Seine Miene verfinsterte sich. Sarimé konnte Renecs Mimik nicht deuten und biss sich vor Schreck 
auf die Zunge.\textit{Ich habe etwas Falsches gesagt.}\\
Der Bastard atmete tief ein, seine Hände zitterten und als er sprach, hatte er sein Gesicht 
vollkommen von ihr abgewandt. Schon das alleine wäre eine Beleidigung, aber Sarimé hatte keine 
Lust, die zickige Adelige zu spielen. Sie war es schlichtweg nicht. Ihre Kindheit war bestimmt von 
Sorgen um den Ruf und Streit um Geld. Also sagte sie dazu nichts und verfluchte sich für ihre 
Direktheit. Trotz ihrer Erwartungen setzte er schließlich zum Sprechen an.\\
``Sieva war wie ein warmer Sommerwind. Sie sprach wenig, interessierte sich für Poesie und Malerei. 
Sie schrieb selbst Gedichte. Wenn sie den Raum betrat, dann wurde es still und alle Aufmerksamkeit 
richtete sich auf sie. Wie eine zarte Blume wirkte sie. In allem was sie tat, war sie sanft und 
vorsichtig. Sieva schwebte, einem Sonnenstrahl gleich, durch die Gänge der Festung.''\\
\textit{Ein Sommerwind bringt nur den Schein einer Erfrischung. Zarte Blumen vergehen. 
Sonnenstrahlen verschwinden mit der Abenddämmerung.}\\
``War sie schön?''\\
``Sehr. Es gibt ein Gemälde von Sieva im Speisezimmer.'' Er sagte das mit einer verträumen Stimme. 
Sarimé fragte sich, ob je ein Mann ihren Namen so aussprechen würde. \\
``Du mochtest sie sehr'', mutmaßte Sarimé leise.\\
Renec versteifte sich augenblicklich und wandte sich ihr zu. Sein Blick war hart und distanziert. 
``Sieva hat A'Rik überzeugt, seinen Bastard an seinen Hof bringen zu lassen. Sie lehrte mir 
schreiben und lesen, setzte sich dafür ein, dass ich reiten lernte, fechten und mit der Politik und 
dem Adel vertraut wurde.''\\
*Worauf will er hinaus?*\\
Er verneigte sich. ``Wenn Ihr also Fragen habt, scheut Euch nicht, Euch an mich zu wenden. Soll ich 
Euch zum Speisezimmer führen?''\\
Sarimé schüttelte kaum merklich den Kopf. Der Bastard war ein Rätsel. Manchmal war er höflich und 
interessiert, dann wieder abweisend und wortkarg. ``Ich finde den Weg. Danke, dass du mir deine 
Zeit geopfert hast.'' Sie nickte ihm zum Abschied zu, raffte ihren Rock und trat über die einzelne 
Stufe zurück in das Gebäude.\\

Graf Evin A'Rik war kein Mann, der viel Wert auf Prunk oder Dekoration legte. Das Augenmerk des 
Speisezimmers richtete sich nahezu sofort, wenn man den Raum betrat, auf den Tisch. Er war breit, 
sodass man sein Gegenüber nicht berühren könnte, selbst wenn man sich über den Tisch ausstreckte. 
Wenn nötig, hätte man 20 Gäste daran unterbringen können und der Raum bot noch Platz für weitere 
Tische, die bei Bedarf aufgebaut werden könnten. Bei dieser Überlänge bot der gedeckte Tisch einen 
eher kläglichen Eindruck. Evin saß mit dem Blick zum Kamin. Vor ihm war in hölzernen Schüsseln das 
Essen aufgetragen. Zwei Porzellangedecke waren sorgsam gegenüber platziert. Vermutlich gab es noch 
mehr Porzellangeschirr, aber das wurde für wichtigere Anlässe aufgespart. \\
Als Sarimé in den Raum trat, erhob sich Graf A'Rik schwerfällig. Trotz seines Alters stand er 
aufrecht, den Kopf erhoben, die Miene grimmig. Sein graues Haar war zu einem Pferdeschwanz 
gebunden, auf seiner faltigen Wange zeigten sich graue Bartstoppel. Unter den zerzausten 
Augenbrauen starrten ihr zwei dunkle Augen entgegen. Vielleicht war er Ende fünfzig, vielleicht 
älter oder jünger. Sarimé konnte es nicht einschätzen. Auf jeden Fall wirkte er jünger, als sie 
befürchtet hatte, obwohl ihr das mittlerweile auch kein Trost mehr war.\\
Sarimé trat vor ihn, knickste tief und hielt den Blick gesenkt. Sie wollte die auswendig gelernten 
Begrüßungsformeln rezitieren, da winkte A'Rik ab und deutete auf den Platz ihm gegenüber. ``Setze 
dich.''\\
Sie musste um den Tisch herum gehen und als sie sich schließlich setzte, zitterten ihre Finger vor 
Nervosität. A'Rik griff nach einem Krug und goss den Inhalt in seinen Becher. Unsicher starrte sie 
das Essen an, was der Graf wohl falsch deutete.\\
``Bediene dich. Hier gibt es keine Bediensteten für solche Kleinigkeiten. Irgendwann fangen sie in 
der Hauptstadt auch noch an, sich von ihren Bediensteten füttern zu lassen, wenn diese Mode so 
weiter geht.''\\
Erst zögerte Sarimé, aber sie konnte nicht leugnen, dass sie riesigen Hunger hatte. Wochenlang 
Reiseproviant und Gasthausessen war nichts zu dem, was sie vor sich sah. Obwohl es vermutlich nur 
ein geringer Aufwand aus der Sicht eines reichen Adeligen war. An sich überraschte Evins Haltung 
sie. Er gehörte dem alten Adel an und bekam als Graf ordentliche Geldmittel zur Verfügung. Sie 
hätte erwartet, dass er mehr Wert auf diese Dinge legen würde, aber ihr war es recht. Weiches, 
frisch gebackenes Brot, ein dickflüssiger Gemüseeintopf, Spiegeleier, Käse und Butter. Und sogar 
ein zarter Schinken lag in Scheiben aufgeschnitten vor ihr auf einem Brett. Ihr kam es vor, als 
hätte derjenige, der das Essen bereitete, sich viel Mühe gegeben. Sarimé verstand das als 
wohlwollendes Zeichen der Köchin und Küchenhelfer. Vielleicht war sie doch nicht so unwillkommen, 
wie sie befürchtet hatte?\\
``Wie alt bist du eigentlich?'', fragte Evin, während er kaute: ``Man sagte mir, du wärst 19, 
siehst aber jünger aus.''\\
Sarimé richtete sich auf und ihre Wangen färbten sich rot vor Scham. ``16, Herr. Vergebt mir, dass 
meine Stiefmutter Euch belogen hat!''\\
Sie wäre fast vom Stuhl aufgesprungen und hätte sich entschuldigend und peinlich berührt verbeugt. 
Jedoch winkte Evin mit einer wegwerfenden Geste ab und schüttelte den Kopf.\\
``Solange du eine erwachsene Frau bist, ist mir das gleich.''\\
Sie biss sich auf die Zunge und starrte auf ihren Teller. Nun, er hatte es sogar ausgesprochen. 
Sarimé war nicht mehr, als Mutter seiner Erben und diese Forderung war deutlich in seiner Stimme 
mitgeschwungen. Nachdem das ausgesprochen war, schien sich Evin nicht mehr viel Mühe um seine junge 
Braut zu machen. Er beschäftigte sich ausgiebig mit dem Essen und las währenddessen Briefe.\\
Sarimé entdeckte das Gemälde, von dem Renec gesprochen hatte. Es zeigte eine junge Frau Mitte 
zwanzig. Sie saß auf einer steinernen Bank, welche Sarimé an die Terrasse erinnerte. Ihr Blick war 
in die Ferne gerichtet. Sieva trug auf dem Bild ein Kleid aus zarten rosanen Farben, das Mieder war 
mit hauchdünnen Goldstickereien verziert. Ihr Haar, in das bronzene Libellen verflochten waren, lag 
glatt schillernd über ihrer Schulter. Ihre Rehkitzaugen waren von dunklen Wimpern umrahmt, die 
geschwungenen Lippen trugen einen sehnsüchtigen Ausdruck. \textit{Renec hatte recht. Alles an ihr 
wirkt zart und sanft. Falls der Maler nicht übertrieben hat.}\\
Sarimé verglich die Frau mit dem Grafen, der vor ihr saß. Sie konnte sich nicht vorstellen, wie 
dieses ungleiche Paar nebeneinander stand. Es war so gegensätzlich. Und sie wirkte traurig. Sievas 
Blick erzählte eine eigene Geschichte voller Sorge, Angst und Sehnsucht nach etwas, welches der 
Betrachter des Bildes nie erfahren würde. Die Neugierde siegte schließlich über Sarimés Nervosität. 
``Sieva... die vorherige Gräfin... wie ist sie gestorben?''\\
Evin hob den Blick, aber diesmal schluckte er erst, bevor er sprach. Er strich sich über seinen 
Bartschatten und musterte Sarimé grimmig. ``Sie ist gesprungen. Sie war dem Leben schon immer fern, 
wie ein Windgeist. Sieva war zerbrechlich. Sie konnte keine Kinder bekommen. Und ja, ich verfluchte 
sie dafür. Gib die Schuld mir, wenn du es willst. Du wärst nicht die Einzige.''\\
Er sah Sarimé abwartend an, aber da sie nichts erwiderte, fuhr er fort: ``Sieva war krank. Im 
Kopf. Sie weinte, wenn eine Blume vertrocknete oder ihre Zofe ihr riet, eine andere Frisur zu 
wählen. Hübsch anzusehen, aber alles in allem keine Frau, die man gerne vorzeigt. Und nicht mal für 
Erben zu gebrauchen. War wohl mein Fehler. Ist ja bekannt, dass die Kasira alle verrückt sind.''\\
Sarimé horchte auf. ``Sieva kam aus Kasir?''\\
\textit{Das hat der Bastard nicht erwähnt.}\\
Evin nickte und biss in eine Brotscheibe. ``Händler, die sich einen Adelstitel kauften. Sie hatten 
in Na'Rash ein Sommerhaus, aber nach der Hochzeit ist ihre Familie zurück nach Kasir. Die Geschäfte 
liefen wohl nicht. Mir war es egal. Ich lasse das Bild abhängen, habe es ganz vergessen.'' Jetzt 
lächelte er und beugte sich etwas zu ihr. ``Ich betrachte lieber das stolz tanzende Feuer, als 
einen farblosen, zerbrechlichen Windgeist!''\\
Sarimé erwiderte sein Lächeln zögernd. Ihr war klar, dass er sie meinte. Aber sie kam sich selbst 
nicht sehr stolz vor. Im Gegenteil, sie wäre am Liebsten mit wehendem Kleid aus dem Zimmer 
geflohen. \\
``Du hast nicht viel mitgebracht, wenn ich mich nicht irre. Kannst dich an ihren alten Kleidern und 
Schmuckstücken bedienen, die waren teuer genug. Ich lasse nach einem Schneider schicken, der die 
Kleider kürzen wird. Du bist kleiner als Sieva. Aber erst mal nur einzelne Stoffe, vielleicht 
wächst ja doch noch.''\\
\textit{Die Kleider einer Toten tragen? Einer, die er Windgeist nennt?}\\
Aber darauf ging sie nicht ein. Es würde wohl nichts bringen, mit diesem Mann zu diskutieren. 
Sarimé verstand mittlerweile, warum der Adel der Hauptstadt einerseits über ihn fluchte, 
andererseits Respekt bekundete. Evin war ein harter Mann, das sah man ihm an. Vielleicht hat der 
Tod all seiner Kinder das verursacht, vielleicht auch seine Pflichten, die er bereits Jahre lang 
ausübte.\\
``Wann wird die Vermählung stattfinden?'', brach Sarimé schließlich die Stille. \\
``In einigen Tagen'', antwortete der Graf: ``Aber erwarte kein ausschweifendes Fest. Die Herren der 
benachbarten Grafschaften sind eingeladen, einige hochrangige Militärs und Adelige. Nach meinen 
Erfahrungen macht es nicht viel Sinn,  zu viel Geld und Zeit für eine Hochzeit auszugeben. Einen 
Abend. Das muss reichen. Ein Priester Osymas wird uns den Segen geben. Mach dir keine Sorgen, es 
wird ein bescheidenes, zweckgebundenes Fest.''\\
Ehrlich gesagt beruhigte Sarimé das wirklich. Natürlich wünschte sich fast jede junge Frau eine 
wundervolle Hochzeit. Alle Augen sollten auf die schöne Braut gerichtet sein. Aber in diesem Fall 
konnte Sarimé drauf verzichten. \\