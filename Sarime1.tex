\chapter{Ein anderes Leben}


Das heiße Wasser prickelte auf ihrer Haut, als sie sich vorsichtig hinein gleiten ließ. 
\textit{Immerhin spüre ich noch etwas.}\\
Seit Sarimé in der Festung Merandilas angekommen war, fühlte sie sich wie betäubt. Alles glitt an 
ihr vorbei, wie Treibgut in einem Fluss. Vielleicht war auch sie das Treibgut.  Sie setzte sich auf 
den Boden der Wanne, umfasste ihr hüftlanges, rotes Haar und hob es über ihre Schulter nach vorn. 
Sie befand sich gerade in einem der zwei Nebenräume ihres Gemachs. Das Hauptzimmer selbst war so 
groß wie ein Erdgeschoss in der alten Stadtvilla ihrer Familie. Aber die Hauptstadt war so 
weit entfernt. Ihre Familie fort. Sarimés Lippen wurden zu einem schmalen Strich.\\
\textit{Nicht weinen. Du bist nicht das einzige Mädchen, was an einen älteren Mann verheiratet 
wird}, sagte sie sich selbst und blinzelte die Tränen fort.\\
Sarimé Sil'Vera seufzte tief und schloss die Augen. Ihr Ziel war, sich fort zu träumen. Zu einem 
Paradies, einem zeitlosen Ort, in dem es keine Angst gab, keinen alten Bräutigam und keine 
Pflichten. Etwas, an dem sie willkommen war. Eine Vergangenheit, in der sie nicht als Tochter 
eines verarmten Kaufmanns aufgewachsen war. Es sollte einfach alles anders sein!\\
Im Hintergrund klapperte die Zofe und störte Sarimés Versuche. Als sich plötzlich heißes Wasser über 
ihren Kopf ergoss, entwich ihr ein hoher Schrei und sie riss die Hände schützend vor ihr Gesicht. 
Die Zofe, ein älteres Weib mit giftigem Blick, stand mit dem Eimer in der Hand über ihr und fragte 
süffisant: ``Zu heiß, Herrin?''\\
``Nein'', erwiderte Sarimé bemüht gelassen: ``Nur scheint der Dreck der Straße sich noch nicht fort 
waschen zu lassen. Es stinkt. Oder bin ich das etwas gar nicht?''\\
Man sah der Zofe an, dass sie sich kaum zusammen reißen konnte. Sarimé hielt ihrem Blick stand. 
Zuhause wäre es zu einem Streit gekommen und die Bedienstete hätte gekündigt. Aber sie war nicht 
mehr die Tochter eines verarmten Kaufmanns, sondern die Braut eines Grafen, der im Namen des Königs 
das Land verwaltete. \textit{Ich darf das nicht runter schlucken}, versuchte sie, sich selbst Mut zu 
machen: \textit{Sonst wird es schlimmer werden.}\\
``Du darfst gehen, ich komme jetzt alleine zurecht.'', erklärte das Mädchen und deutete auf die 
Tür, ohne den Blickkontakt zu unterbrechen: ``Vielen Dank für deine Dienste und das heiße 
Wasser!''\\
Die Zofe schloss die dünne Schiebetüre mit einem kraftvollen Ruck. Die Tür vibrierte noch lange 
nachdem Sarimé bereits alleine war. Sie seufzte schwer und alle Kraft schien ihr mit diesem Atemzug 
zu entweichen. Mit leeren Gedanken blickte sie an sich herab. Das heiße Wasser umschmeichelte ihren 
zierlichen Körper, entspannte die verkrampften Muskeln. Rosenblätter trudelten an ihren Knien 
vorbei. Ihr rotes Haar wirkte unter Wasser dunkel und bauschte sich auf wie eine Qualle. Ohne Luft 
zu holen, sank sie tiefer in die Wanne. Sie war klein genug, um darin komplett unterzutauchen. 
Langsam öffnete das Mädchen ihre Augen und starrte zur Oberfläche. Rosenblätter und Haarsträhnen 
trudelten durch ihr Sichtfeld. Auf der anderen Seite erkannte sie nur Licht. Es war, als würde man 
in eine andere Welt blicken. Eine Welt, in der es keine Schatten und keine Dunkelheit gab. Nur 
ein gleißendes, vollkommenes Licht. \\
Sie wollte hier bleiben. An diesem stillen Ort. Aber ihr Körper entschied anders. Instinktiv schoss 
ihr Kopf empor, ihre Lungen sogen die plötzlich kalt gewordene Luft tief ein und ihre Hände 
krallten sich um den Rand der Wanne. Das Haar klebte ihr am Kopf und im Gesicht. Wasser ran 
strömend an ihr herunter. Ihr Oberkörper bebte und die Tränen verloren sich im nach Rosen duftenden 
Badewasser. Sarimé verbarg ihr Gesicht in den Händen und trotz des heißen Wassers zitterte sie 
schlimmer als eine Erfrierende.\\
Das Geräusch der Schiebetür riss sie aus ihrer Verzweiflung und sie blickte hoch. Renec stand 
dort. Die einzige Seele in der gesamten Grafschaft, die ihr seit ihrer Ankunft ein ehrliches 
Lächeln schenkte. Und außerdem der Bastard ihres Verlobten. Sie hatten noch nicht viele Worte 
gewechselt, seit sie am Morgen in der Festung angekommen war. Daher konnte Sarimé für sich keine 
Erklärung finden, wieso der junge Mann nun in ihrem Badezimmer stand. Seine Hand umschloss 
immernoch den Türrahmen und einen Moment sahen sie einander nur stumm an. 
Sarimé rann eine weitere Träne über die Wange.\\
``Dreh dich weg!'', brach sie schließlich die Stille.\\
Seine grauen Augen weiteten sich kurz, dann färbten sich seine Wangen rot und er wandte sich 
ruckartig dem Hauptzimmer zu. ``Verzeiht, Herrin'', stammelte er: ``Ich hörte Euch weinen und... 
machte mir Sorgen.''\\
``Frauen weinen oft'', erwiderte Sarimé, erhob sich eilig aus dem Wasser und tastete nach dem 
wärmenden Mantel. Das Badewasser plätscherte, als sie aus der Wanne stieg.\\
``Manche'', sagte Renec: ``Und manche nicht.''\\
Sarimé, eingehüllt in das Gewand, betrachtete ihn daraufhin nachdenklich. Er war größer 
als sie. Wie fast jeder Mann und auch die meisten Frauen. Sein schwarzes Haar stand 
zerzaust ab und die Hand, die sie immernoch festklammerte, zeigte seine Nervosität. Schließlich 
nickte sie, obwohl er weiterhin in die andere Richtung blickte. ``Du kannst dich wieder umdrehen. 
Und sage mir, was du wirklich wolltest.''\\
Er sah sie nicht an, als er antwortete. ``In diesen Zimmern wurden schon zu viele Tränen vergossen. 
Als ich es das letzte Mal ignoriert habe, ging es nicht gut aus. Ich wollte mich vergewissern, dass 
es Euch gut geht, Herrin.''\\
\textit{Es mir gut geht?} Sie würde ihm so gerne glauben. Sie würde so gerne glauben, dass es noch 
einen Menschen gab, dem sie wichtig war. Nur einen einzigen. Selbst wenn es ein Bastard wäre.\\
``Du hast nicht zufällig die Zofe gesehen?''\\
``Sie ist in Trauer, Herrin. Nehmt es ihr nicht übel'', verteidigte er sie.\\
``Kann ich mir selbst eine aussuchen? Ich will sie nicht. Eigentlich brauche ich überhaupt keine'', 
erklärte das Mädchen und trat in das Hauptzimmer. Sie hatte schon immer selbst ihr Feuer geschürrt, 
die Kleider gewaschen und das Essen gekocht. Altes, adeliges Blut allein bezahlte keine 
Bedienstete. In ihrem Zimmer gab es nur wenige Fenster, welche jetzt zur Nachmittagsstunde kaum 
Licht herein ließen. Ansonsten 
waren die Räumlichkeiten ansprechend. Spartanisch eingerichtete Zimmer waren ihr vertraut und die 
wenigen Möbel waren ansehnlich. Auf dem großen Bett lag eine bequeme Matratze und eine Vielzahl 
verschiedener Kissen und Decken. Die Vorhänge des Bettes ließen sich zuziehen und dann wirkte es wie 
ein geheimes Versteck. Das war das Erste, was das Mädchen noch vor dem Bad ausprobiert hatte. Vor 
dem Bett lag ein weiches Fell, in das sie sich am Liebsten sofort eingehüllt hätte. Welches Tier es 
einst wohl gewärmt haben musste, konnte Sarimé nicht sagen. Sie kannte sich mit den Tieren des 
Nordens nicht aus. An der Wand stand eine Truhe, in deren Holz sich Schnitzereien befanden. Man 
konnte ein tanzendes Paar, galoppierende Pferde, fliegende Falken und allerlei verschlungene Muster 
erkennen. An den Wänden hingen Gemälde von Landschaften. Das Meer, eine Schneewelt und Berge. Auf 
einem wuchtigen Schreibtisch war zahlreicher Schmuck ausgebreitet, den eine Bedienstete wohl noch zu 
putzen und polieren gedachte. \textit{Echter Schmuck.}\\
Aber Sarimé hatte diesen noch kaum eines Blickes gewürdigt. \textit{Stattdessen musste ich wie ein 
kleines Kind das Bett ausprobieren}, dachte sie und verzog das Gesicht über sich selbst.\\
``Eine Gräfin braucht eine Zofe'', meinte der Bastard: ``Jemand muss Euch helfen, ein passendes 
Kleid zu finden und Euch herzurichten.''\\
Mit einem leisen Lächeln sah sie ihn an. ``Es ist aber niemand da. Außer du natürlich, Bastard.''\\
``Ich werde sie suchen gehen.''\\
``Wozu? Ich richte mich für einen Mann her, dann kann ich auch den Rat eines Mannes einholen. Ich 
mag sie nicht. Und wenn du es so eilig hast, zeig mir die Kleider. Keine Sorge, ich kann mich 
selbst frisieren.''\\
``Herrin...''\\
``Das war ein Befehl'', erläuterte Sarimé beiläufig. \\
``Ihr scherzt?'' Der Bastard war irritiert und blickte sie ratlos an. Sarimé musste grinsen. Sie 
mochte diesen jungen Mann und betete zu Osyma, dass er sie vielleicht auch etwas mögen könnte. Sie 
brauchte einen Freund in diesem Feindesland. Jemanden, vor dem sie lachen konnte, plaudern 
und scherzen. Sie trug die Maske der Gräfin noch nicht einmal zwei Stunden und war bereits völlig 
erschöpft. Und das wusste er. Sarimé sah es in seinen Augen.\\
``Ich meine es ernst. Du bist ein Bastard und kein Höfling, kein Adeliger, kein hochrangiger 
Krieger. Die Rolle, die du spielst, passt nicht zu dir.''\\
``Die Herrin kennt so viele Adelige?'', entgegnete er und blickte sarkastisch drein.\\
\textit{Er weiß von der finanziellen Lage meiner Familie.}\\ 
``Das traf'', entgegnete sie: ``Können wir nun aufhören uns gegenseitig zu beleidigen?''\\
``Gerne. Das ist auf Dauer bestimmt anstrengend!''\\
Sie zögerte. ``Ich heiße Sarimé.''\\
Als hätte sie ihm erzählt, sie würde sich gleich in einen Drachen verwandeln, wurde Renec 
schlagartig wieder ernst. ``Ich weiß, Herrin.''\\
Tränen stiegen erneut in ihr auf. Sie hatte ihm Vertrautheit und Freundschaft angeboten, aber der 
Bastard hatte brüsk abgelehnt. \textit{Bilde ich es mir nur ein? Bin ich so verzweifelt?}\\
``Du kannst gehen. Ich werde mich ankleiden'', sagte sie leise und ihr Augen ruhte dabei auf 
dem Fell, welches am Boden lag. Sarimé wagte nicht, noch einmal den Blick zu heben, noch einmal in 
die sturmgrauen Augen zu blicken und noch einmal enttäuscht zu werden. Dieses Mal war er derjenige, 
der zögerte. ``Darf ich Euch anschließend in der Festung herumführen?''\\
Sarimé nickte fahrig. ``Warte vor der Tür.''\\


Die Festung war so groß. Ein Bollwerk aus Stein und Holz. Er zeigte ihr die Ställe, die 
Falknerei, das Gehege der Luchse. Die Küche, die Säle, die Türen zu zahllosen Zimmern. 
Die Unterkunft der Wachen und Bediensteten. Sarimé war karge Räume gewohnt. Sie erkannte 
den Unterschied zwischen den leeren Stellen. Kanten und Flecke, an denen sich einst Gegenstände 
befanden, die mit dem Verlust eines Menschen ebenfalls verschwanden.\\
Auf der Terasse verweilte Sarimé. Sie trat über die brüchigen, schmutzigen Fließen zu einer 
steinernen Bank und legte die Hände in den Schoß. Vertrocknete Blätter lagen vereinzelt, vom Wind 
hier her getragen, auf dem Boden. Von ihrem Sitzplatz aus hatte sie die Aussicht auf die hügelige 
Landschaft Merandilas. Sie sah die Wolken, die sich am Horizont ballten. Sturmwolken, von der selben 
Farbe wie die Augen des Bastards. Sie sah die tiefgrünen Wälder und weiten Wiesen. Eine Straße, die 
ins nächste Tal verschwand und eingezäunte Weiden.\\
\textit{Meine Burg. Meine Ländereien. Sogar meine Sturmwolken.}\\
Nachdenklich schweiften ihre grünen Augen über die Landschaft. Sarimé wusste nicht viel über die 
Grafschaft. Nie hätte sie erwartet, hier hoch in den Norden Saleicas zu müssen. Hier den Rest 
ihres Lebens zu verbringen. Hier irgendwann zu sterben.\\
Seit fast hundert Jahren galt Merandila als eingebürgerte Grafschaft Saleicas. Gleich nach der 
Kapitulation der letzten Königin. Sie sei ins Meer gegangen, hieß es in den alten Märchen. Zurück 
in die Unterwasserstadt an der östlichen Küste Merandilas. Es hieß, die Bewohner Kasirs und 
Merandilas verließen ihre Unterwasserstadt vor tausenden von Jahren. Deshalb machte ihnen die 
eisigen Winter auch nichts aus.\\
``Um wen trauert die Zofe?'', griff Sarimé das Thema auf.\\
Renec lehnte sich an das Geländer und blickte hinab. Zwei Meter unter der Terrasse befand sich 
etwas, was wohl einst ein Garten war. Die Grünfläche war eben und es schien, als hätte man eigens 
dafür Erde abgetragen, um Platz zu schaffen. Unkraut und Gestrüpp überwucherten den kleinen Platz. 
In diese Dornen würde Sarimé niemanden schicken wollen.\\
``Um die Gräfin'', erklärte er versonnen, ohne sich ihr zuzuwenden.\\
``A'Riks verstorbene Frau?''\\
Er nickte. ``Sieva.''\\
``Dann bin ich also seine zweite Braut?''\\
``Die Dritte, um genau zu sein. Seine erste Frau schenkte ihm drei Erben und starb mit dem Vierten 
im Kindbett. Ein Sohn stürzte vom Pferd und erwachte nicht mehr, obwohl sein Herz noch einige Tage 
schlug. Der Älteste wurde krank. Der Mittlere starb zuletzt an einer Infektion, während er in einer 
anderen Grafschaft zu Gast war. Das ist erst sieben Jahre her.''\\
Sarime betrachtete ihn neugierig. ``Und dann kam Sieva.'' Sie konnte nur Renecs Profil sehen, aber 
erahnte ein verträumtes Lächeln.\\
``Drei Monate später, ja. Der Graf lernte sie in der Tempelstadt Na'Rash kennen. Ihre Mutter hat 
sechs gesunden Kindern das Leben geschenkt, ihre zwei Schwestern hatten auch schon jeweils drei. Er 
erhoffte sich, dass die Fruchtbarkeit an Sieva ebenfalls vererbt wurde. A'Rick braucht anerkannte 
Erben, die die Grafschaft weiter führen werden. Seine zweite Frau schenkte ihm keinen einzigen.''\\
\textit{Und jetzt bin ich an der Reihe.}\\
``Kennt man Sievas Familie?''\\
Renec schüttelte verneinend den Kopf. ``Reich, aber junger Adel. Sie haben sich den Titel 
vermutlich erkauft und so etwas bedeutet nichts. Unter anderem darum wurde ihre 
Familie gemieden. Sie war weit unter seinem Stand, aber meinem Vater ist ein Erbe wichtiger.''\\
``Mein Name ist der Grund, warum ich hier bin, oder?'', fragte Sarimé leise.\\
Er zog die Schultern hoch und wählte seine Worte bewusst. ``Alter Adel hat sich seit Jahrhunderten 
bewiesen. Außerdem ist es für das Land besser. Falls er ohne Erben stirbt, wird der König einen 
seiner nahestehenden Vertrauten schicken, der die Witwe heiratet und die Grafschaft übernimmt. 
Königliche Vertraute heiraten nicht gerne namenlose Witwen.''\\
Sie biss sich auf die Zunge. Einerseits hatte sie wirklich gehofft, dass ihr Gatte bald zu Osyma 
kehren würde, wenn er wirklich so alt war, wie sie befürchtete. Andererseits hatte Sarimé gar nicht 
daran gedacht, dass dann ein anderer kommen würde. ``Wann ist Sieva verstorben? Und wie? Erzähl mir 
von ihr.''\\
Seine Miene verfinsterte sich. Sarimé konnte Renecs Mimik nicht deuten und biss sich vor Schreck 
auf die Zunge.\textit{Ich habe etwas Falsches gesagt.}\\
Der Bastard atmete tief ein, seine Hände zitterten und als er sprach, hatte er sein Gesicht von ihr 
abgewandt.\\
``Sieva war wie ein warmer Sommerwind. Sie sprach wenig, interessierte sich für Poesie und Malerei. 
Sie schrieb selbst Gedichte. Wenn sie den Raum betrat, dann wurde es still und alle Aufmerksamkeit 
richtete sich auf sie. Wie eine zarte Blume wirkte sie. In allem was sie tat, war sie sanft und 
vorsichtig. Sieva schwebte, einem Sonnenstrahl gleich, durch die Gänge der Festung.''\\
Sarimé strich sich eine Haarsträhne hinter das Ohr, während sie ihn nachdenklich musterte. 
\textit{Er liebt sie}\\
Seltsamerweise tat es gut zu sehen, dass sie nicht die einzige unglückliche Seele hier war. 


Graf Evin A'Rik war kein Mann, der viel Wert auf Prunk oder Dekoration legte. Das Augenmerk des 
Speisezimmers richtete sich nahezu sofort, wenn man den Raum betrat, auf den Tisch. Er war breit, 
sodass man sein Gegenüber nicht berühren könnte, selbst wenn man sich über den Tisch ausstreckte. 
Wenn nötig, hätte man 20 Gäste daran unterbringen können und der Raum bot noch Platz für weitere 
Tische. Bei dieser Überlänge bot der gedeckte Tisch einen eher kläglichen Eindruck. Evin saß mit dem 
Blick zum Kamin. Vor ihm war in hölzernen Schüsseln das Essen aufgetragen. Zwei Porzellangedecke 
waren sorgsam gegenüber platziert.\\
Als Sarimé in den Raum trat, erhob sich Graf A'Rik schwerfällig. Trotz seines Alters stand er 
aufrecht, den Kopf erhoben, die Miene grimmig. Sein graues Haar war zu einem Pferdeschwanz 
gebunden, auf seiner faltigen Wange zeigten sich graue Bartstoppel. Unter den zerzausten 
Augenbrauen starrten ihr zwei dunkle Augen entgegen. Vielleicht war er Ende fünfzig, vielleicht 
älter oder jünger. Sarimé konnte es nicht einschätzen. Auf jeden Fall wirkte er jünger, als sie 
befürchtet hatte, obwohl ihr das mittlerweile auch kein Trost mehr war.\\
Sarimé trat vor ihn, knickste tief und hielt den Blick gesenkt. Sie wollte die auswendig gelernten 
Begrüßungsformeln rezitieren, da winkte A'Rik ab und deutete auf den Platz ihm gegenüber. ``Setz 
dich.''\\
Sie musste um den Tisch herum gehen und als sie sich schließlich setzte, zitterten ihre Finger vor 
Nervosität. A'Rik griff nach einem Krug und goss den Inhalt in seinen Becher. Unsicher starrte sie 
das Essen an, was der Graf wohl falsch deutete.\\
``Bediene dich. Hier gibt es keine Bediensteten für solche Kleinigkeiten. Irgendwann fangen sie in 
der Hauptstadt auch noch an, sich füttern zu lassen, wenn diese Mode so weiter geht.''\\
Erst zögerte Sarimé, aber sie konnte nicht leugnen, dass sie großen Hunger hatte. Wochenlang 
Reiseproviant und Gasthausessen war nichts zu dem, was sie vor sich sah. Obwohl es vermutlich nur 
ein geringer Aufwand aus der Sicht eines reichen Adeligen war. An sich überraschte Evins Haltung 
sie. Er gehörte dem alten Adel an und bekam als Graf ordentliche Geldmittel zur Verfügung. Sie 
hätte erwartet, dass er mehr Wert auf diese Dinge legen würde, aber ihr war es recht. Weiches, 
frisch gebackenes Brot, ein dickflüssiger Gemüseeintopf, Spiegeleier, Käse und Butter. Und sogar 
ein zarter Schinken lag in Scheiben aufgeschnitten vor ihr auf einem Brett. Ihr kam es vor, als 
hätte derjenige, der das Essen bereitete, sich viel Mühe gegeben. Sarimé verstand das als 
wohlwollendes Zeichen der Köchin und Küchenhelfer.\\
``Wie alt bist du eigentlich?'', fragte Evin, während er kaute: ``Man sagte mir, du wärst 19, 
siehst aber jünger aus.''\\
Sarimé richtete sich auf und ihre Wangen färbten sich rot vor Scham. ``16, Herr. Vergebt mir, dass 
meine Stiefmutter Euch belogen hat!''\\
Sie wäre fast vom Stuhl aufgesprungen und hätte sich entschuldigend und peinlich berührt verbeugt. 
Jedoch winkte Evin mit einer wegwerfenden Geste ab und schüttelte den Kopf.\\
``Solange du eine erwachsene Frau bist, ist mir das gleich.''\\
Sie biss sich auf die Zunge und starrte auf ihren Teller. Nun, er hatte es sogar ausgesprochen. 
Sarimé war nicht mehr, als Mutter seiner Erben und diese Forderung war deutlich in seiner Stimme 
mitgeschwungen. Nachdem das ausgesprochen war, schien sich Evin nicht mehr viel Mühe um seine junge 
Braut zu machen. Er beschäftigte sich ausgiebig mit dem Essen und las währenddessen Briefe.\\
Sarimé entdeckte ein Gemälde in einer dunklen Ecke des Zimmers. Es zeigte eine junge Frau Mitte 
zwanzig. Sie saß auf einer steinernen Bank, welche Sarimé an die Terrasse erinnerte. Ihr Blick war 
in die Ferne gerichtet. Sieva trug auf dem Bild ein Kleid aus zarten rosanen Farben, das Mieder war 
mit hauchdünnen Goldstickereien verziert. Ihr Haar, in das bronzene Libellen verflochten waren, lag 
glatt schillernd über ihrer Schulter. Ihre Rehkitzaugen waren von dunklen Wimpern umrahmt, die 
geschwungenen Lippen trugen einen sehnsüchtigen Ausdruck. \textit{Renec hatte recht. Alles an ihr 
wirkt zart und sanft. Falls der Maler nicht übertrieben hat.}\\
Sarimé verglich die Frau mit dem Grafen, der vor ihr saß. Sie konnte sich nicht vorstellen, wie 
dieses ungleiche Paar nebeneinander stand. Es war so gegensätzlich. Und sie wirkte traurig. Sievas 
Blick erzählte eine eigene Geschichte voller Sorge, Angst und Sehnsucht nach etwas, welches der 
Betrachter des Bildes nie erfahren würde. Die Neugierde siegte schließlich über Sarimés Nervosität. 
``Sieva... wie ist sie gestorben?''\\
Evin hob den Blick, aber diesmal schluckte er erst, bevor er sprach. Er strich sich über seinen 
Bartschatten und musterte Sarimé grimmig. ``Sie ist gesprungen. Sie war dem Leben schon immer fern, 
wie ein Windgeist. Sieva war zerbrechlich. Sie konnte keine Kinder bekommen. Und ja, ich verfluchte 
sie dafür. Gib die Schuld mir, wenn du es willst. Du wärst nicht die Einzige.''\\
Er sah Sarimé abwartend an, aber da sie nichts erwiderte, fuhr er fort: ``Sieva war krank. Im 
Kopf. Sie weinte, wenn eine Blume vertrocknete oder ihre Zofe ihr riet, eine andere Frisur zu 
wählen. Hübsch anzusehen, aber alles in allem keine Frau, die man gerne vorzeigt. Und nicht mal für 
Erben zu gebrauchen. War wohl mein Fehler. Ist ja bekannt, dass die Kasira alle verrückt sind.''\\
Sarimé horchte auf. ``Sieva kam aus Kasir?''\\
\textit{Das hat der Bastard nicht erwähnt.}\\
Evin nickte und biss in eine Brotscheibe. ``Händler, die sich einen Adelstitel kauften. Sie hatten 
in Na'Rash ein Sommerhaus, aber nach der Hochzeit ist ihre Familie zurück nach Kasir. Die Geschäfte 
liefen wohl nicht. Mir war es egal. Ich lasse das Bild abhängen, habe es ganz vergessen.'' Jetzt 
lächelte er und beugte sich etwas zu ihr. ``Ich betrachte lieber das stolz tanzende Feuer, als 
einen farblosen, zerbrechlichen Windgeist!''\\
Sarimé erwiderte sein Lächeln zögernd. Ihr war klar, dass er sie meinte. Aber sie kam sich selbst 
nicht sehr stolz vor. Im Gegenteil, sie wäre am Liebsten mit wehendem Kleid aus dem Zimmer 
geflohen.\\
``Du hast nicht viel mitgebracht, wenn ich mich nicht irre. Kannst dich an ihren alten Kleidern und 
Schmuckstücken bedienen, die waren teuer genug. Ich lasse nach einem Schneider schicken, der die 
Kleider kürzen wird. Du bist kleiner als Sieva. Aber erst mal nur einzelne Stoffe, vielleicht 
wächst du ja doch noch.''\\
\textit{Die Kleider einer Toten tragen? Einer, die er Windgeist nennt?}\\
Aber darauf ging sie nicht ein. Es würde wohl nichts bringen, mit diesem Mann zu diskutieren. 
Sarimé verstand mittlerweile, warum der Adel der Hauptstadt einerseits über ihn fluchte, 
andererseits Respekt bekundete. Evin war ein harter Mann, das sah man ihm an. Vielleicht hat der 
Tod all seiner Kinder das verursacht, vielleicht auch seine Pflichten, die er bereits Jahre lang 
ausübte. Er zog seine Festung den Städten vor. Verbrachte seine Tage damit, über die Grenze zu 
wachen und überließ den Priestern die Verwaltung Na'Rashs, Merandilas größter Stadt.\\
``Wann wird die Vermählung stattfinden?'', brach Sarimé schließlich die Stille. \\
``In einigen Tagen'', antwortete der Graf: ``Aber erwarte kein ausschweifendes Fest. Die Herren der 
benachbarten Grafschaften sind eingeladen, einige hochrangige Militärs und Adelige. Nach meinen 
Erfahrungen macht es nicht viel Sinn, zu viel Geld und Zeit für eine Hochzeit auszugeben. Einen 
Abend. Das muss reichen. Ein Priester Osymas wird uns den Segen geben. Mach dir keine Sorgen, es 
wird ein bescheidenes, zweckgebundenes Fest.''\\
Sarimé rang sich ein nicken ab und konzentrierte sich dann darauf, eine Brotscheibe mit Käse und 
Schinken zu belegen.\\

``Ist Merandila so verarmt, dass er kein neues Kleid für mich anfertigen lassen kann?'', rief 
Sarimé und ging unruhig in ihrem Zimmer umher.\\ 
Die Sonne war bereits untergegangen und ihr Gemach war von mehreren flackernden Kerzen erhellt. Sie 
war müde und gleichzeitig aufgewühlt. Und einsam. Sie sehnte sich nach Nähe, nach tröstenden Worten 
und Zuversicht. Nach jemandem, der sich für sie interessierte. ``Nicht einmal das Brautkleid?''\\
``Entschuldigt die Frage, Herrin. Aber habt Ihr mich allein aus diesem Grund gerufen?'', fragte 
Renec und lächelte amüsiert.\\
``Nein!'', erwiderte Sarimé und wandte sich ihm zu: ``Ich meine… ja… ach.''\\
Das junge Mädchen ließ sich auf der Bettkante nieder. \textit{Was habe ich schon zu verlieren?} \\
``Ich wollte mit jemandem sprechen. Und ich kenne niemanden außer dir, Bastard.''\\
Renec betrachtete sie lange aus seinen tiefgrauen Augen. Sarimé fürchtete, er ahnte wo genau ihre 
Sorgen herkamen. Sie hielt es ihm zugute, dass er nicht darauf einging und das unverfängliche 
Thema fortsetzte. ``Ihr fürchtet Sievas Kleider?''\\
``Nein, den Geist in ihnen.''\\
``Sorgt Euch nicht, Herrin. Ihr Geist wird Euch nichts zuleide tun. Warum sollte es auch so sein? 
Weder hat Sieva Grund Euch zu hassen, noch zu fürchten.''\\
``Sicher? Ich heirate schließlich ihren Gatten… Keine Frau teilt gerne ihren Liebsten. Auch nicht, 
wenn sie bereits im Grabe liegt.''\\
Renec lachte leise. ``Auch da sehe ich keinen Grund zur Sorge. Evin wäre wohl der letzte Mann auf 
Erden, den Sieva ihren Liebsten genannt hätte.''\\
Sarimé schüttelte den Kopf. ``Auch ein einsamer Geist kann gefährlich werden.''\\
``Einsam war sie auch nicht'', erwiderte Renec und sah sie einen Moment finster an. Aber er 
lächelte so plötzlich wieder, dass Sarimé sich nicht sicher war, ob sie seinen Blick richtig 
gedeutet hatte.
\textit{Der Vater nennt sie krank und einen Windgeist, der Sohn nennt sie sanft und lieb.}\\
Es heißt, Windgeister waren Wesen fern von Leben und Tod. Sie zogen durch das Land, sangen ihr 
wisperndes Klagelied und jeder der es vernahm, war dem Untergang geweiht. Ihm geschah das größte 
Elend, seine Albträume wurden wahr und er verlor alles, was ihm am Herzen lag. Das war der Fluch 
der Windgeister, sie predigten ihre Lieder und verfluchten ahnungslose Menschen, die ihren schönen, 
wehmütigen Klängen lauschten. Die Windgeister teilten ihren Schmerz, ohne dass er je kleiner wurde. 
Sie zogen nur noch mehr Wesen mit sich in den Abgrund.\\
``Hat sie hier in diesem Zimmer gelebt?'', fragte Sarimé.\\
Renec nickte nur und schwieg. Sie trat an ihm vorbei zum schmalen Fenster, öffnete es mit zwei 
schnellen Griffen und sah hinaus. \textit{Dann ist sie von hier gesprungen.}\\
Ihr entging sein schneller Schritt auf sie zu nicht, das kurze Flackern in seinen Augen, ehe er 
sich wieder unter Kontrolle hatte. Sarimé tat, als bemerkte sie nichts und blickte hinauf zu den 
Sternen. \textit{Sind das noch die selben Sterne? Ist das noch der selbe Mond?}\\
Vielleicht war sie in der Wanne gar nicht aufgetaucht. Vielleicht hatte die Welt aus Licht sie 
verschluckt.