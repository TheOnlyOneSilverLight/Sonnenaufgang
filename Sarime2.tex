\chapter{Die Gräfin Merandilas}


Sie schlenderte plaudernd neben Renec durch die Gänge der Burg. Als die beiden jungen Menschen das 
Eingangsportal durchschritten, war sie gerade dabei angegekommen zu beschreiben, welch Freude im 
Gesicht des Haushofmeisters Pedan gelegen war, als er feststellte, wie gut sie rechnen konnte. ``Was 
hat er denn erwartet?'', kicherte Sarimé: ``Ich bin eine Kaufmannstochter, ich konnte eher zählen 
als meinen Namen sprechen!''\\
``Ihr seid vermutlich, abgesehen vom Herrn, die erste Person seit Jahren, die sich mit Pedan über 
Zahlen unterhalten konnte'', erklärte Renec lachend: ``Selbst seine eigenen Kinder hat es eher in 
handwerkliche Berufe und in die Kaserne getrieben.''\\
Renecs Lachen verklang abrupt, als er zur Seite trat und sich vor dem Grafen verneigte. Evin, der 
gerade aus einen der Räume getreten war, ignorierte seinen Sohn und musterte Sarimé streng.\\
In den letzten Tagen hatte das Mädchen gespürt, dass er unzufrieden mit ihr war. Jedoch wusste sie 
auch nicht, was Evin von ihr erwartete. Sie fiel in einen tiefen Knicks. ``Mein Herr'', murmelte 
sie zur Begrüßung.\\
Zwei weitere Männer traten hinter ihrem Graf auf den Gang. Der Haushofmeister Pedan, der Sarimé 
freundlich zunickte, sowie ein alter Priester Osymas. Ein Büschel weißes Haar stand ihm vom Kopf ab 
und sein faltiges Gesicht verzerrte das blaue Tattoo auf seiner linken Wange. Er hustete betont 
auffällig und blickte Evin auffordernd an.\\
``Könnt Ihr reiten?'', fragte der Graf grimmig.\\
Sarimé nickte nur, da wandte Evin sich bereits ab und stampfte durch das Eingangsportal hinaus auf 
den Hof.\\
``Beeilt Euch, Herrin. Sonst reitet er ohne Euch los'', riet der Priester und schenkte ihr nun auch 
ein aufmunterndes Lächeln: ``Osyma erleuchte Eure Wege!''\\

Sarimé fand Evin neben einem untersetzten, schmächtigen Mann vor. Renec hatte ihn ihr als den 
Stallmeister Bale vorgestellt. Seine Zucht war eine der edelsten der nördlichen Grafschaften, hieß 
es. Selbst im Süden, in Kalesh und Mahag waren die Tiere begehrt. Es hieß das letzte Schlachtross 
König Kareens stammte auch aus Bales Zucht. Seine Falken wurden in die Kolonien verschifft. Seine 
Raubkatzen bereicherten den Zirkus in der Hauptstadt. Aber mittlerweile war er alt geworden und 
konnte sein Bein seit einem Sturz vom Pferd nicht mehr belasten.\\
Evin hielt einen aufgezäumten Rappen am Zügel. Das Tier war stämmig und muskulös, dichtes Fell und 
mächtige Hufe. Das Pferd eines Kriegers. Als Sarimé näher trat, betrachtete er flüchtig ihre 
Reitkleidung, bestehend aus einer Stoffweste, sowie Hose und Stiefel aus weichem Leder. Er nickte 
beiläufig und deutete auf eine Stute, die eben aus dem Stall geführt wurde. Sie wirkte wie ein 
Fohlen neben Evins Rappen. Zierlich tänzelte sie am Zügel über den Hof. Jedes Haar an ihr war 
schwarz wie Pech.\\
``Gutes Mädchen'', murmelte Bale, als er ihre Zügel ergriff: ``Zählt fünf Sommer, stammt aus der 
Zuchtlinie Arabors und Karima. Rabe ist ihr Name.''\\
Sarimé hob langsam die Hand und streichelte sacht den gebogenen Hals der Stute.\\
``Nicht gerade das Pferd einer Dame'', sagte sie.\\
``Aber das Pferd einer Gräfin'', knurrte Evin und schwang sich in den Sattel.\\
Sie erwiderte seinen Blick trotzig und ordnete die Zügel, ehe sie sich in den Sattel zog. Das 
Mädchen brauchte einen Moment um sich zurecht zu finden und rutschte in eine bequeme Position.\\
``Karima war einst das Pferd unserer ersten Gräfin'', erklärte Bale: ``Sie selbst hat die 
Zuchtlinie gegründet.''\\
Sarimé nickte um ihm zu zeigen, dass sie verstand. So viele Spuren Sieva auch in der Burg 
hinterlassen hatte, die Zeichen der ersten Gräfin waren längst verblasst. Die Zeit hatte sie mit 
sich genommen. \textit{Sie ist gestorben, bevor ich geboren wurde...}\\
Mit einem Druck der Fersen trieb sie die Stute in den Trab und durch das Tor, auf den Wald zu. Das 
Mädchen suchte den Rhythmus des Pferdes. Evin hatte schon längst den Hügel hinter sich gelassen und 
ritt auf den Waldrand zu. Daher ging Sarimé das Risiko ein und trieb das ihr fremde Pferd zum 
Galopp. In Brom-Dalar hatte sie kaum die Möglichkeit gehabt, so schnell zu reiten. Es galt als 
unschicklich. Außerdem waren die Straßen zu eng.\\
Nervös beugte sie sich über den gestreckten Hals des Pferdes und presste die Beine gegen den warmen 
Bauch. Der Wind peitschte ihr die schwarze Mähne ins Gesicht. Es war ein berauschendes Gefühl. Sie 
spürte die Finger des Windes durch ihr langes Haar fahren. Sie spürte den Boden erbeben unter Rabes 
Hufen.\\
Ihr Verlobter wartete, bis sie ihn eingeholt hatte und deutete dann stumm auf den Wald. 
Schweigend ritten sie langsam den Weg entlang. Sie betrachtete ihn von der Seite, versuchte seine 
harte Miene zu entziffern. Man sah die Ähnlichkeit zu seinem Sohn, aber seine Gesichtszüge 
waren härter und kantiger. \textit{Wenn er mich nicht will, warum bin ich dann hier?}\\
``Dieser Wald eignet sich für die Jagd?'', brach sie die Stille.\\
Evin ließ sich Zeit mit einer Antwort. Schließlich nickte er und fügte hinzu: ``Im Winter kommen 
Wölfe aus Kasir in diese Gegend und holen sich das Rotwild. Die Förster haben ordentlich zu tun mit 
den Biestern.''\\
``Gibt es hier wilde Raubkatzen?''\\
``Nur unsere weißen Luchse. Einige ließ ich vor Jahren auswildern. Aber sie zeigen sich selten.''\\
Sarimé sah sich gründlich um, während der Graf sie beobachtete. Er lachte trocken. ``Nein 
Stadtkind. Kein Tier wird sich uns hier zeigen. Für die Jäger steht die Sonne zu hoch und für die 
Beute sind wir zu laut.''\\
``Schade'', murmelte Sarimé.\\
Evin stoppte seinen Rappen an einer Weggabelung und deutete auf einen vierten, schmalen Pfad. 
``Aber alles hinterlässt seine Spuren.''\\
Prüfend blickte er das Mädchen an, ob sie auch wirklich Interesse hatten, dann fügte er hinzu: 
``Das ist ein Pfad der Tiere. Man sieht dort Hasenköttel.''\\
Sarimé deutete an ihm vorbei. ``Dort hängt Fell.''\\
Er nickte anerkennend. ``Das stammt vom Ochshorn-Wild. Die Farbe ist noch recht hell, also stammt 
es von einem Jungtier. Vermutlich letzten Frühling geboren...''\\
Evin verstummte, als er Schritte im Unterholz vernahm. Ein Mann mit einer Axt auf der Schulter kam 
summend auf sie zu. Als er die Reiter entdeckte, verneigte er sich eilig. ``Seid gegrüßt'', 
murmelte er: ``Habe ich Eure Jagd gestört, Herr?''\\
Evin erwiderte: ``Grüße auch an dich. Nein, ich bin auf dem Weg nach Talsmund.''\\
Der Holzfäller, welcher Sarimé einen neugierigen Blick zuwarf, nickte. ``Die Evilla hat's schon 
gesagt. Mögen Eure Wege hell sein, mein Herr.''\\
``Selbst in der dunkelsten Nacht'', erwiderte Evin den Gruß und bedeutete Sarimé, dass sie den 
Weg nehmen würden.\\

``Was wollt Ihr?'', fragte Sarimé direkt.\\
Sie betrachtete ihren Verlobten, musterte das wettergegerbte Gesicht, die rauen Hände und die 
breiten Schultern. Unrasiert, zerzaust, in einfacher Kleidung. In Brom-Dallar hätte niemand ihn für 
einen Mann von Adel gehalten. Aber hier ritt er. Der Graf Merandilas. Der Hüter der Grenze. Erbe 
der letzten freien Königin.\\
Er verstand die Tiefe ihrer Frage und ließ sich Zeit mit einer Antwort. Der Wald lag bereits hinter 
ihnen, als Evin antwortete. ``Ich will, dass Merandila besteht. Saleica ließ Ruinen zurück, aus 
denen meine Familie Neues erbaute. Felder, Wälder, Dörfer. Das ist Merandila. Wir sind nicht aus 
dem Süden, wo die Menschen sich in Städten versammeln. Wir sind Handwerker, Bauern, Jäger. Stirbt 
meine Linie aus, wird ein gepudertes Söhnchen aus dem Süden kommen und meinen Platz einnehmen. Mein 
Volk würde leiden. Unsere Traditionen sterben.''\\
``Ihr könntet Renec anerkennen'', schlug Sarimé vor.\\
Evin lachte höhnisch und Verachtung lag in seinem Blick. ``Menschen machen Fehler. Es war ein 
Fehler ihn zu zeugen. Es war ein Fehler ihn nicht gleich als Säugling zu ertränken und es war ein 
Fehler ihn an meinen Hof zu holen. Er ist ein Bastard und das wird er immer bleiben. Eine Ratte im 
Bau der weißen Luchse. Ein Feigling. Ein Schwätzer. Er kann reden, aber nicht handeln. Er kann 
fluchen, aber nicht kämpfen. Meine Söhne... alle drei...'' Er verstummte. Dann drehte er sich im 
Sattel und spuckte vor Verachtung aus. ``Ein jeder von ihnen war drei mal so viel Wert wie mein 
Bastard!''\\
\textit{Wie solltest du auch jemals den Bastard als Sohn betrachten, wenn deine Gedanken bei den 
Toten sind?}\\
``Und was erwartet Ihr von der Mutter eines Erben?'', fragte sie.\\
``Meint Ihr denn, Ihr hättet das Zeug dazu?'', konterte er.\\
``Ich kann besser rechnen als so mancher Mann. Lese viele Bücher und wenn ich auf einem Instrument 
spiele, klingt es, als würde ich gerade einen Hahn erwürgen. Und das einerlei auf welchem 
Instrument.''\\
``Eine Gräfin muss eine Frau des Volkes sein. Die Mutter aller Kinder. Die Schwester jedes Mannes 
und jeder Frau. Sie muss die Frau sein, für die die Soldaten in den Krieg ziehen. An deren Rock 
die Kinder ihre Tränen trocknen. In deren Augen die Menschen sich selbst sehen. Eine Gräfin muss 
vorausahnen können, klug sein und wachsam. Aber ich habe noch nie gehört, 
dass Krieg, Leid oder Ungerechtigkeit durch ein Lied beendet wurde.''\\
``Welch Trost. Alles andere kann ich lernen. Nur das musizieren ist hoffnungslos'', antwortete 
Sarimé, aber der Schalk war aus ihrer Stimme gewichen. Sie meinte es ernst.\\
``Das ist Rauch'', murmelte Evin leise und starrte an den Horizont.\\
Hinter den sich im Wind sanft wiegenden, goldgelben Gras der Wiese stieg dichter Qualm in den 
Himmel auf. Talsmund brannte.\\
Evin stieß seine Fersen in die Flanken seines Rappen und der Hengst preschte los. Er zeigte die 
ganze Kraft die in seinen Muskeln schlummerten und trug seinen Herrn den Qualm entgegen. Sarimé 
griff fest in die Zügel ihrer Stute um Rabe abzuhalten, dem Hengst zu folgen. Einen Moment sah sie 
Evin hinterher, dann wendete sie ihr Reittier und trieb es zum Galopp. Sie folgte dem breiten Wegen 
im Wald, suchte nach Lichtungen und Baumstümpfen. Die Zeichen führten sie richtig und Sarimé fand 
eine Gruppe von sieben Männern, die beieinander auf dem Waldboden hockten und überrascht zu ihr 
aufblickten. Ihr Haar hing ihr wild ins Gesicht und der Schweiß ließ das schwarze Fell ihrer 
Stute glänzen.\\
``Talsmund brennt!'', rief Sarimé den Holzfällern zu.\\
Diese Worte genügten. Die Männer ließen ihre Äxte liegen. Zwei zogen sich auf den breiten Rücken 
der Zugpferde, ein weiterer dachte noch daran, die Bänder, die die Pferde mit den schweren 
Baumstämmen verband zu lösen. Sie eilten voraus, während die übrigen zu Fuß folgten. Sarimé lenkte 
Rabe neben dem Mann, den sie bereits im Wald begegneten. ``Wo befinden sich die restlichen 
Arbeiter?''\\
Mitten am Tag waren die meisten Männer und Frauen vermutlich im Wald oder auf den Feldern.\\
``Wenn Ihr der Mittagssonne folgt, Herrin'', keuchte der Mann vom schnellen Lauf: ``Dann kommt Ihr 
zu den Weinbergen. Auf der anderen Seite des Bachlaufs.''\\
``Gut. Ich suche die Männer. Ihr kommt mit Eimern zum Bach!'', rief Sarimé und trieb Rabe wieder in 
eine schnellere Gangart.\\
Sie verschwendete keinen Blick auf den Qualm, der schwärzer und dichter geworden war, sondern hielt 
sich der Sonne entgegen und ließ keine Sorgen und Ängste zu. Die Arbeiter finden. Das war das 
Wichtigste und Einzige, was möglich war.\\
Rabe strebte mit gestrecktem Hals und weiten Sprüngen Hügel hinauf und hinunter, durch dichtes Gras 
und flachen Grund. Sie preschte durch den Bachlauf, dass das Wasser wie eine Fontäne spritzte und 
auf die Weinberge zu. Die Arbeiter sahen die wilde Reiterin auf sich zukommen und hielten in ihrer 
Arbeit inne.\\
Sarimé war dort noch nicht angekommen, als sie bereits rief: ``Talsmund brennt!''\\
Die Leute kippten die Reben aus ihren Eimern und rannten durcheinander um auch jedes Gefäß 
mitzunehmen, ehe sie die Steigung hinabeilten um ihr Dorf zu retten. Ein Mädchen von zehn Jahren 
ging dabei unter und fing laut das weinen an. Sarimé griff hinab und zog das Kind vor sich in den 
Sattel, ehe sie Rabe wendete und zurück ritt. Sie gönnte der Stute erst am Bach eine Pause, als sie 
aus dem Sattel glitt.\\
``Habt ihr einen Brunnen im Dorf?'', fragte sie eine Frau.\\
Sie nickte. ``Aber er ist im Spätsommer immer trocken. Wir müssen von hier beginnen.'' Sie deutete 
auf den Bachlauf. ``Eine lange Kette! Wir brauchen die Leute aus dem Dorf.''\\
``Die Förster sind auf dem Weg'', erklärte Sarimé und klopfte Rabe auf den verschwitzten Hals: 
``Und Evin ist im Dorf.''\\
Die Frau sah sie prüfend an, aber Sarimé hatte sich schon in den Sattel gezogen. ``Ich komme 
wieder'', sagte sie noch und ritt auf die Rauchwolken zu.\\

Sie fand Evin umgeben von drei schluchzenden Kindern und einer zitternden Frau, an deren Kleid 
Feuerzungen geleckt hatten. Die restlichen Dorfbewohner strebten schon dem Bach entgegen um eine 
lange Kette zu bilden und das Wasser zu dem brennendem Haus zu transportieren. Auch an Evins Händen 
entdeckte Sarimé rote Male.\\
``Die Holzfäller sind auf dem Weg'', erklärte Sarimé eilig: ``Und die Leute beim Weinberg beginnen 
schon mit der Kette.''\\
Evins Haar klebte vor Schweiß und Asche, als er einen Moment sprachlos zu ihr aufblickte. ``Gut'', 
sagte er dann nur.\\
``Seid Ihr etwa in die Flammen?'', fragte Sarimé.\\
Die schluchzende Frau hielt ihren Sohn fest umklammert. ``Er hat meinen Heo rausgeholt. Oh 
Helle! Eure Wege seien erleuchtet, Herr!''\\
Sarimé musterte Evin ernst, rutschte aus dem Sattel und griff nach seinen Handgelenken um die 
Wunden zu sehen. ``Ihr sagtet, vorausahnend sein. Und dann rennt Ihr in ein einstürzendes Haus für 
einen Knaben? Was braucht Merandila mehr? Einen Grafen oder einen Bauernknaben?''\\
``Ich darf also erst wieder mit Feuer spielen, wenn Ihr meinen Erben tragt?'', fragte Evin.\\
Sarimé hob eine Augenbraue. ``Sagen wir, wenn mindestens drei Erben neben Euch reiten.''\\
Evin verneigte sich mit einen flüchtigen Grinsen. ``Wie die Gräfin befiehlt!''\\
Sarimé befahl ebenso, dass Evin zurück zur Burg reiten solle um seine Wachen und Knechte zu rufen. 
Ein jeder der sich im Sattel halten könne, solle kommen um zu helfen. Aber es würde dauern bis die 
Unterstützung in Talsmund ankommen würde. Sarimé reihte sich in die Kette ein und einigte sich mit 
einem der Holzfäller, welcher auch einer der Dorfsprecher zu schienen sei, zuerst mit dem nur 
langsam kommenden Wassereimern die Umgebung des brennenden Bauernhauses zu befeuchten um eine 
Ausbreitung der Flammen zu verhindern. Das Vieh der Familie war aus den Ställen befreit und 
flüchtete sich über die Hügel. Die Kinder, die zu schwach oder zu langsam für die Kette waren, 
eilten hinterher um die Tiere zu hüten. Trotz der Verstärkung aus der Burg dauerte es bis in den 
Abend hinein, bis die letzte Glut zischend starb.\\
Erschöpft saßen mehr als Hundert Menschen unter dem klaren Sternenhimmel. Dankende Worte und Blicke 
fielen. Gebete und Flüche. Und immer wieder hörte Sarimé das eine Wort. \textit{Gräfin.}\\


Da stand sie. Sarimé Sil'Vera. Tochter eines Kaufmanns, der einst König Kareen beriet und die 
Häfen der Kolonien bereiste. Tochter einer Stadtherrin Brom-Dallars. Nicht mehr als ein Mädchen 
mit roten Locken und einem waldgrünen Kleid mit bronzenfarbenen Stickereien. Das Kleid einer 
anderen Frau. Der Spiegel einer anderen Frau. Eines Windgeists. Und heute würde sie deren Witwer 
heiraten. Ihren Titel erlangen. Ihre Ländereien und ihre Pflicht. Und sie betete zu Osyma, dass 
sie würdig sein würde.\\
Die Stickereien erinnerten an die Blätter einer Rose und zierten lediglich die rechte Seite des 
Kleides, vom Saum bis zur Schulter hinauf. Der grüne Stoff ließ ihr Haar noch flammender 
schimmern. Ihre Frisur richtete Sarimé selbst. Sie setzte sich an den Tisch und band das Haar mit 
einem Stoffband in der Farbe des Kleides hoch und fixierte den Knoten mit fünf hölzernen Stäbchen, 
die in einem Halbkreis aus dem Haarknoten hervorstachen. Die Stäbchen waren aus dunklem Holz und an 
den Spitzen vergoldet. Ihr Gesicht wurde mit feinem Puder bedeckt, bis keine einzelne 
Sommersprosse mehr zu sehen war. Zuletzt wählte Sarimé eine mehrgliedrige Goldkette und legte drei 
vergoldete Armreife um ihr rechtes Handgelenk. Klingende Geräusche erfüllten den Raum, wenn sie den 
Arm bewegte. Sie erhob sich und warf einen letzten Blick in den Spiegel.\\
``Sarimé Sil'Vera, Gattin des Grafen Evin A'Rik und Gräfin Merandilas'', sagte sie leise zu sich 
selbst.\\

Osyma sprach einst, dass Verbindungen zwischen Menschen unter freiem Himmel, bei Morgengrauen oder 
Abendrot stattfinden müssten. Flammen, die mit seinem Namen gesegnet waren, sollten über die 
Verbindung wachen. Denn durch das Feuer konnte er jedes Wort hören, jede Geste sehen und jede Lüge 
enttarnen. Dieser Brauch bezog sich auf Eheschließungen ebenso, wie auf Schwüre und Versprechen. Um 
die Regelung zu vereinfachen, hatten die Priester Osymas verkündet, dass jede Tempeldecke die 
Malerei eines Himmels zeigte und somit die Anforderungen erfüllte. Aber der einzige große Tempel in 
Merandila befand sich in der Stadt Na'Rash, drei Tagesritte von der Festung entfernt. Der Graf hatte 
beschlossen, dass sich diese Mühe kein weiteres Mal lohnte.\\
Sarimé war es recht. Wie jedes Mädchen hatte sie einst von einer wundervollen Hochzeit mit der 
Liebe ihres Lebens geträumt. Aber sie war kein kleines Mädchen mehr. Und sie glaubte nicht daran, 
dass es die Liebe gab.\\
Es gab nur Leidenschaft und Schmerz.\\
``Herrin...''\\
``Ich glaube nicht, dass dein Vater es gutheißen wird, wenn er erfährt, dass du im Gemach seiner 
Braut ein und ausgehst.''\\
``Er verachtet mich sowieso'', bemerkte Renec und fügte hinzu: ``Ihr seht wunderschön aus.''\\
``Ich weiß'', erwiderte Sarimé kühl.\\
\textit{Nur Leidenschaft und Schmerz.}\\
``Ich habe ein neues Buch für Euch'', erklärte er.\\
Sie wandte sich ihm zu. Das Abendlicht beschien sein Gesicht, wie er auf dem Bett saß und eine 
Seite des Buches umblätterte. ``Du weißt, dass ich heute deinen Vater heirate?''\\
``Die ganze Grafschaft weiß das'', antwortete er und sah sie fragend an: ``Wenn nicht das ganze 
Land. Habt Ihr die Geschenke gesehen? So viele Namen, die ich noch nie gehört habe. Alles Adel aus 
dem Süden. Der König hat meinem Vater ein Schwert schicken lassen. Ich habe bis jetzt nur den 
Juwel am Griff gesehen, aber es wird unglaubglich teuer sein.\\
Renec bemerkte ihr zögern und verstummte. Sarimé war sich selbst nicht klar, was sie ihm eigentlich 
sagen wollte. Also sagte sie schlicht: ``Danke.''\\
Seine Antwort war nur ein nicken. Einen Moment betrachtete er sie stumm, dann klappte er das Buch 
zu, verneigte sich und bot ihr die Hand, um sie zur Feier zu führen.


Der Abend war windig. Der Sonnenuntergang wurde von Wolken umlagert, die sich rötlich verfärbten. 
Vom Hügel aus konnte Sarimé sehen, wie das Gras sich sanft dem Drängen des Windes beugte. Zwei 
Fackeln tanzten im Mittelpunkt der Menschenmenge. \textit{Wenn ich jetzt noch wüsste, wer diese 
Menschen sind...}\\
Sarimé beugte sich im Sattel vor und strich über den Hals der Stute. Eine vertraute Wärme ging von 
dem Tier aus und sie verspürte das Verlangen, ihr Gesicht tief in der schwarzen Mähne zu vergraben.
Renec räusperte sich. ``Die Gäste warten, Herrin.''\\
Das Mädchen blickte über ihre Schulter und einen Moment lang trafen sich ihre Blicke. In seinen 
Augen meinte sie ebensolchen Kummer zu sehen, wie der, der gerade in ihrer Seele hauste. Aber 
vielleicht bildete sie es sich auch wieder nur ein. ``Na los'', murmelte sie der Stute zu und 
lenkte sie zur Menschenmenge.\\
Sie sah Evins große Gestalt bei den Fackeln stehen und die Gäste bildeten eine Gasse, als Sarimé 
näher kam. Sie konnte einen weiteren Blick über die Schulter nicht verhindern. Renec war ihr nicht 
gefolgt. \textit{Ich bin alleine}, dachte sie und spürte, wie etwas in ihr erzitterte.
Augenblicklich richtete sie sich höher auf. Ihre Finger umschlossen die Zügel fester, sie hob ihr 
Kinn und lächelte Evin an, wie es eine Braut und baldige Gräfin tun sollte. Ihre Angst, Sorge und 
Hilflosigkeit hatten sich in Resignation verwandelt. \textit{Es gibt keine Liebe. Osyma hält unser 
aller Leben in den Händen und ich kann mich nicht gegen seine flammenden Klauen wehren.}\\
Der Priester war ein großgewachsener, hagerer Mann. Für sein noch recht junges Alter machte er 
bereits ein viel zu ernstes Gesicht. Sarimé fragte sich, ob er etwas erlebt hatte, das eine solche 
Einstellung rechtfertigen würde oder ob er es sich durch Übung angeeignet hatte. Ebenfalls schien 
er noch nicht lange dem Priesterstand anzugehören, denn je treuer man zu Osyma war, desto mehr 
Tätowierungen zierten die Männer und Frauen. Dieser hier, der Sarimé und Evin das Versprechen der 
Ehe abnehmen sollte, besaß verschlungene Tätowierungen an Hals und Schlüsselbein und ebenso an 
seinem rechten Handgelenk. Da die Muster immer zusammenhingen, vermutete Sarimé, dass unter seinem 
Gewand die Zeichnungen von Handgelenk und Hals sich trafen. Neben ihm saß der Priester der 
Festung auf einem Stuhl. Er würde die Zeremonie nicht durchführen, dafür sah er sich als zu alt, 
hatte er erklärt. Seine Hand zitterte, als er sie zum Gruß hob und lächelnd nickte. Sarimé fasste 
Mut durch dieses Lächeln und erwiderte es flüchtig.\\ 
Das Mädchen glitt aus dem Sattel und sofort ergriff einer der Bediensteten Rabes Zügel. 
Sarimé stellte sich mit wenigen Schritten neben den Grafen. Der Priester breitete seine Arme aus und 
begann mit der Zeremonie. Seine Stimme war laut und hallend, als er zu sprechen begann: ``Zu dieser 
Stunde, zwischen Tag und Nacht, kehren wir hier zusammen. Zu dieser Stunde, in der der allmächtige 
auf unserer Sphäre wandelt und diesem Akt der Vereinigung zweier seiner Kinder seine Aufmerksamkeit 
schenken kann. Diese Vereinigung ist sein Wille und diese zwei dürfen von nichts getrennt werden, 
außer es ist der Wille des Mächtigen, in dem er sie durch Krankheit und Tode trennt. Graf Evin 
A’Rik, Osyma schenkte Euch das Leben in Merandila, Ihr wuchst unter seinem Schutz auf und wurdet 
dazu erwählt, diese Grafschaft für den gesegneten König Semric zu bewahren. Er schenkte Euch drei 
Erben und drei Erben nahm er euch, ebenso zwei Gattinnen. Betet, dass Euch ein solches Unglück kein 
weiteres Mal ereilt und bittet ergeben um seine Gnade und eine kinderreiche Ehe. Gelobt Ihr vor den 
allsehenden Augen des Mächtigen, die Treue gegenüber Osyma und Eurem Weib?''\\
Besagter Graf Evin A'Rik nickte nur gemächlich und lächelte amüsiert. \textit{Seine dritte Ehe. 
Vermutlich kann er die Predigt selbst fehlerfrei zitieren.}\\
Wieder wirkte der Priester alles andere als erfreut oder auch nur entspannt. Evins Reaktion war ihm 
nicht ernsthaft genug. Sarimé musste grinsen. Es war fast schon lächerlich. Da standen sie nun, 
umgeben von einer Hundertschaft an Gästen und versprachen sich Treue und Liebe.
\\textit{Mein Vater hat mich vermutlich schneller an Evin verkauft, als dieser Priester auch nur 
zum Luft holen Zeit braucht.}\\
``Sarimé Sil'Vera!'', sagte er schneidend und richtete seinen Blick auf das Mädchen.
Natürlich, er wagte es nicht, einen Grafen strafend an zu blicken. Die junge Braut war eine andere 
Angelegenheit. Sarimés Lächeln verblasste und sie sah ihm streng in die Augen. Hoffte darauf, dass 
sie ein ebensolches Funkeln besaß, wie der zornige Priester.\\
``Ihr wart geboren in unserer glorreichen Hauptstadt Brom-Dalar, direkt unter dem Schutz hunderter 
Priester des Mächtigen, ein gutes Omen. Eure Mutter gab Ihr Leben um Euch das Eure zu gewähren. 
Osyma schützte Eure zarte Seele. Dafür seid Ihr ihm auf ewig zu Dank und Treue verpflichtet. Gelobt 
Ihr, vor den allsehenden Augen des Mächtigen, die Treue gegenüber Osyma und Eurem Gatten?''\\
``Ja.''\\
Der Priester nickte knapp. Sein Lob dafür, das zumindest einer sich an das Protokoll hielt. Während 
er weitere Reden schwang, über Osymas Macht und Erhabenheit, hingen Sarimés Gedanken noch bei 
diesem kurzen Wort. \textit{Ja.}\\
Wie emotionslos dieses Wort geklungen hatte. Unpassend, für eine junge, aufgeregte Braut. Sie hätte 
strahlen sollen, ihrem Gatten einen verliebten Blick zuwerfen müssen und vor Vorfreude nicht 
stillhalten dürfen. Bitter blickte sie auf das niedergetrampelte Gras zu ihren Füßen. Zumindest 
wurden die Bräute in Romanen so beschrieben. Jetzt wagte sie auch tatsächlich einen Blick zu Evin. 
Er hatte die Augen zum Horizont gewandt und schien die Wolken zu studieren. Vielleicht rätselte er, 
ob es heute noch regnen würde.\\
Ruckartig hob der Priester seine Hände gen Himmel und legte den Kopf in den Nacken. Die weiten 
Ärmel seines groben Gewandes fielen zurück und offenbarten seine blasse Haut. Wie Sarimé vermutet 
hatte, war nur der rechte Arm tätowiert. Doch ihr Blick hing viel mehr an dem, was er in den Händen 
hielt. Die untergehende Sonne blendete ihre Sicht, daher konnte sie nicht viel erkennen. Außerdem 
erfassten die Sonnenstrahlen ebenfalls die beiden Gegenstände und ließ sie funkeln. \\
``Osyma, allmächtiger Gott, sei uns gnädig und segne mit deinem Licht und deiner Güte diese Bänder 
der Verbundenheit. Lass sie diese Eheleute stets an ihr Gelöbnis erinnern, an ihre Verbundenheit 
und an ihre Treue. Segne sie und schenke ihren Gebeten und Wünschen in der heutigen Nacht deine 
Aufmerksamkeit. Und, wenn sie deiner würdig sind, erfülle sie!''\\
Um seinen Worten Wirkung zu verleihen, stand der Priester noch einen Moment reglos da. Dann, als 
der letzte Strahl der Sonne hinter einem Hügel verblasste, senkte er seine Arme. Im dämmrigen 
Abendlicht war kaum noch etwas zu erkennen, doch die Fackeln spendeten zumindest dem Brautpaar 
etwas Licht. Sarimé sah die beiden Armbänder nun besser. Evins war grob und breit, mit 
eingravierten Mustern aus Wirbeln und Schlangen. Ihres dagegen feingliedrig und zart. Jedes Glied 
war wie eine kleine Blüte geformt. Sie erkannte Rosen, Malven, Sonnenblumen und Lilien, welche sich 
in regelmäßigen Abständen wiederholten. Das Band musste Unsummen gekostet haben. Kaum ein 
Goldschmied war in der Lage, so fein zu arbeiten. \textit{Vermutlich bin ich auch schon mindestens 
die Zweite, die es trägt}, dachte Sarimé. Aber sie musste sich trotzdem eingestehen, dass sie 
niemals mit etwas so schönem gerechnet hätte.\\
Sarime streckte ihren linken Arm aus. Evin wurde gerade sein Band angelegt, nun wandte sich der 
Priester ihr zu. Geschickt schlossen seine langen Finger den Verschluss und er nickte ihr 
auffordernd zu. Ehe Sarimé sich versah, ergriff Evin routiniert ihre linken Hand. Er hob die ihre 
mit in die Höhe, sodass die Gäste die beiden Bänder und somit die Verbundenheit des Paares sehen 
konnten.\\
``Und nun, lasst uns feiern!'', verkündete er laut.\\



Die Musik hallte durch die ganze Festung. Im Hof feierten die Bediensteten und die Bewohner 
Talsmunds. Sie tranken, sie speißten, lachten und tanzten auf das Wohl des neuen Grafenpaars. 
Renec hatte sich einige Stunden unter ihnen gemischt. Auch er hatte sich öfters einschenken 
lassen, aber seine Laune besserte sich dadurch nicht. Als der Holzfäller Arham mal wieder zu 
einer seiner Geschichten anstimmte - diemal handelte es von der Unterwasserstadt und deren 
Einwohnerin, darunter auch die letzte Königin - verließ der Bastard mit Kopfschmerzen den Hof. 
Nachdenklich schlenderte er, begleitet von leiser Musik, durch die von Fackeln erhellten Flure der 
Festung. Er hätte eigentilch schon längst weg sein sollen. Er wäre bereits in Kasir angekommen. 
Janka, hatte er sich als Ziel ausgesucht. Renec kannte die geschriebenen Worte Karkaros auswendig, 
hatte sie aber natürlich gleich verbrannt. Außer ihm konnte hier in dieser Burg vermutlich nur Evin 
die Sprache Kasirs, aber er wollte kein Risiko eingehen. Sein Vater würde den kleinsten Vorwand 
nutzen, ihn zu verstoßen oder hinzurichten. Renec hatte nichts gegen Saleica oder Merandila. Aber 
es war nicht seine Heimat. Er sah sich nirgends zuhause. Warum hätte er also bleiben sollen, 
nachdem Sieva gesprungen war? Renec hatte die Briefe an ihre Verwandten geschrieben um es ihnen 
mitzuteilen. Evin hatte überhaupt nicht daran gedacht, mit Kasira Kontakt aufzunehmen. Ein Brief 
kam als Antwort. Und als Einladung. Es wäre Verrat gewesen, das war Renec natürlich bewusst. Aber 
was sprach dagegen, sich mit Informationen ein neues Leben zu erkaufen?\\
Nun, es war ja jetzt auch egal. Er war nicht gegangen. Noch nicht.\\
Renec trat über den Seitengang hinaus auf die Terasse. Von hier aus war die Musik aus dem Festsaal 
laut. Er konnte die Stimmen der Damen und Herren hören. Der Bastard hielt sich in einer dunklen 
Ecke der Terasse und blickte zu den Sternen hoch. Ein jeder erzählte eine Geschichte, hatte seine 
Mutter ihm erklärt. Ein jeder stand für einen Profeten. Mara und wie sie alle hießen. Er wusste es 
nicht mehr. Die Geschichten seiner Kindheit - die Geschichten eines fremden Glaubens - waren zu 
lange her und auch Sieva hatte es nie gewagt, in diesen Mauern die Profeten Kasirs zu erwähnen.\\
Renec brauchte sich gar nichts vormachen. Er wusste, warum er diesen Mauern noch nicht den Rücken 
kehrte. Warum er nicht längst im Norden war. Er hatte mit Sieva dorthin gewollt. Er hätte sie 
geheiratet. Aber sie war gesprungen.\\
Er lauschte den näherkommenden Schritten und sah ihnen entgegen. Das Licht aus dem Festsaal umfloss 
die Gräfin wie einen Schein, während sie wenige Schritte hinter der Tür auf der Terrasse stand. Sie 
sah so verloren aus. Renec blinzelte gegen den Lichtkontrast an und kam vorsichtig näher. Vermutlich 
hatte sie ihn noch nicht gesehen.  Die Schultern bebten, die Hände krallten sich verzweifelt in den 
Stoff.\\
``Renec'', sagte Sarimé Sil'Vera mit einer festen Stimme.\\
Er erschrak. Seine Augenlider schlossen sich einen Moment und er ermahnte sich zur Einsicht. 
Zwang sich, zurück in die Realität zu kommen. Sieva war tot. Das Mädchen vor ihm war jemand völlig 
anderes. Ihre Hände griffen nicht nach Halt, sondern waren zur Faust geballt. Ihr Kopf stolz 
erhoben. Die Augen funkelten nicht vor Tränen, sondern Trotz. \\
``Ihr seht bezaubernd aus'', murmelte Renec gedämpft. Was nicht einmal eine Lüge war. So sehr man 
sie auch in ein Korsett zwingen würde, ihre Augen blieben die einer zornigen Raubkatze. Renec 
zögerte. \textit{Sie ist stark. Sie weiß es nur nicht.}\\
Beschämt senkte er den Blick. Stärke... das war es, was er Sieva stets vorgeworfen hatte. Sie war 
so schwach gewesen! Immer den Tränen nahe. Sie wollte immer hören, was sie tun sollte, anstatt auch 
nur einmal das Kinn zu eben und so zu blicken wie das Mädchen, was vor ihm stand. Und das, obwohl 
Sarimé so jung war. ``Geht es Euch gut?'', fragte er.\\
Sie trat näher und er konnte ihr Gesicht besser erkennen. Ihr Miene war verkniffen und angespannt.
``Nein.''\\
Die Ehrlichkeit, welche in diesen einem Wort steckte, überraschte Renec. Auch das entging der 
jungen Gräfin nicht. ``Und hör auf, mich so anzuschauen. Ich schaffe das. Alles.''\\
Er deutete auf die geöffnete Tür. ``Dann solltet Ihr wieder hinein gehen. Dort wartet Euer Graf.''\\
Ihr Blick folgte seinen Fingerzeig und sie zögerte.\\
\textit{Also doch nicht so mutig.}\\
``Dein Vater scheint alleine gut klar zu kommen.''\\
``Mein Vater ist es seit Jahren gewohnt, dass seine Frau keine Hilfe ist.''\\
Sie schwieg kurz und sprach das aus, was er vermieden hatte zu sagen. ``Aber das will er nicht von 
seiner neuen Gräfin.''\\
Der Bastard nickte nur.\\
``Ich werde es lernen'', wiederholte sie leise. Ihr Atem stockte. ``Aber nicht jetzt. Ich... ich 
kann da jetzt nicht rein...''\\
\textit{Sie ist einsam und hat Angst.} Renec seufzte leise. \textit{Wie war es wohl Sieva 
ergangen?}\\
Leicht neigte Renec seinen Oberkörper vor. ``Darf ich um einen Tanz bitten?''\\
Ihre Antwort war ein stummes nicken. Vorsichtig ergriff er ihre Hand und legte seine Zweite an ihre 
Hüfte. Stumm zählte er den Takt zur Musik, die aus dem Saal erklang und machte den ersten Schritt. 
Selbst im Tanz waren ihre grünen Augen unbeweglich auf sein Gesicht gerichtet. \textit{Sieva 
hätte mich nicht angeschaut. Sie wäre rot geworden und hätte beschämt gelächelt, obwohl wir uns 
nahe standen. Ich muss aufhören sie zu vergleichen!}\\
``Du grübelst auch'', bemerkte Sarimé nach einer Drehung.\\
``Es war ein langer Tag'', wich Renec aus und leitete den nächsten Tanzschritt ein. Er dachte an 
ihr Lachen, dass sie gerne bereit war zu verschenken. Jetzt lachte sie nicht. Die Gräfin Merandilas 
schien gefasst, das Mädchen dahinter eingesperrt hinter Korsett und Hochsteckfrisur. \\
``Du prüfst mich'', sagte sie leise.\\
``Was meint Ihr?'', flüsterte er zurück.\\
``Dieses Gesicht hast du immer, wenn du etwas vergleichst. Zu welchem Schluss bist du gekommen?''\\
Er schwieg lange, legte einen komplizierteren Tanzschritt dazwischen um sich Zeit zu verschaffen. 
Glücklicherweise endete das Lied und die Musiker schienen eine kurze Pause zu machen. Er nahm das 
als erfreulichen Zufall und ließ ihre Hüfte los. Der Griff ihrer blassen Hand verstärkte sich 
abrupt. ``Nein. Du antwortest mir. Jetzt'', beschloss sie: ``Hier ist niemand ehrlich zu mir. Also 
musst du es sein!''\\
Tief holte er Luft. Ihre Worte trafen ihn, erweckten sogleich die Stimmen seines Gewissens. ``Ihr 
könnt Herzen erobern'', sagte er schließlich: ``Wenn Ihr es richtig einsetzt. Eine wichtige Gabe für 
eine Gräfin. Es geht immer um Herzen. Gewinne das Herz deines Volks, deiner Soldaten, deiner 
Wachen, deines Königs.''
Ihre Augen senkten sich kurz und ihr Griff lockerte sich. Er wollte etwas sagen, als ihm klar 
wurde, wovon sie eigentlich sprach. Sie hatte keine Angst vor den Gästen oder davor im Mittelpunkt 
zu stehen. Seine Stimme wurde sanfter. ``Es wird vorbei gehen. Mach einfach die Augen zu. Evin will 
einen Erben und eine Frau, für die er sich nicht schämen muss. Alles andere ist ihm gleichgültig. 
Mach die Augen zu... und wenn es vorbei ist, verlässt du mit erhobenem Kopf das Zimmer, 
verstanden?'' Ohne es zu merken hatte er sie vertrauter angesprochen, als es ihm zustand. Das 
rothaarige Mädchen blickte dankbar zu ihm hoch und nickte. Ihre Augen glänzten vor Feuchtigkeit, 
aber keine Träne löste sich.\\
``Du bist stark, dass sah ich schon, als du aus der Kutsche gestiegen bist.''\\
``Danke.''\\
Sie wandte sich ruckartig um, raffte den Rock ihres Kleides um sich die einzelne Stufe zurück in 
den Saal zu erleichtern und Renec sah nur noch, wie sie auf einen der Gäste zu trat und mit einem 
aufgesetzten Lächeln ein Gespräch begann. Einen langen Moment blieb er dort stehen und sah sie an. 
Diesmal verglich er sie nicht mit Sieva. \\


Sie hatte vor ihren frisch angetrauten Gatten das Fest verlassen und war einen Umweg über den Hof 
gegangen. Die Hoffnung, ihn noch einmal zu sehen, zog Sarimé in das Mondlicht hinaus. Schließlich 
führten ihre Schritte sie aber doch zum Unvermeidbaren. Leise öffnete das Mädchen die Tür des 
Gemachs und schlüpfte durch einen Spalt in das dunkle Zimmer hinein. \textit{Ich mache die Augen 
zu. Nichts wahrnehmen. Nichts spüren. Nichts denken. Sich fort Träumen. Sich starke Schwingen 
wachsen lassen und fliegen. Überall hin, nur nicht hier sein in diesem Raum. Alleine. Mit meinem 
Mann. Evin A'Rik}, dachte sie, \textit[{Graf Merandilas. Der Hüter der Nordgrenze.}\\
Sarimé schüttelte kaum merklich den Kopf, seufzte leise in die Stille hinein. Was klagte sie? 
Es hätte schlimmer kommen können. Sie hatte immerhin Geld. Und Sicherheit. Und eine berechtigte 
Hoffnung, dass Evin zufrieden mit ihr sein würde. Sie tastete sich zu seinem Bett und setzte sich 
reglos auf die Kante. \textit{Es gibt keine Liebe. Einfach die Augen schließen. Nur Leidenschaft 
und Schmerz.}\\
Sarimé konzentrierte sich auf die Schwärze der Nacht. Alle anderen Eindrücke verbannte sie. Auch, 
wie sich die Türe öffnete und ihr Mann herein kam. Auch, wie er sie anfasste und die Geräusche, die 
er von sich gab. Seine Bewegungen. Ihr Schmerz. Die Enge und das ringen nach Atemluft und Raum. 
\textit{Es gibt keine Liebe.}\\
``Immerhin heulst du nicht wie der Windgeist'', sagte er und rollte sich von ihr herunter.\\
\textit{Es gibt keine Liebe!}\\
Sarimé erwiderte nichts. Sie stand stumm auf, kleidete sich im Dunkeln ein und verließ mit langen 
Schritten den Raum. Zwei Wachsoldaten vor der Tür des Grafen glotzten sie an, aber auch diesen 
schenkte das Mädchen keine Aufmerksamkeit. Die Tränen kamen erst, als sie die Kälte der leeren 
Gänge hinter sich gelassen hatte. Erst, als sie in ihrem Zimmer ankam und sah, wer in einem der 
Sessel eingeschlafen war.\\
Sarimé konnte nicht sagen, ob es an ihm lag, dass ihr nun die Tränen kamen oder lediglich an der 
Tatsache, dass es doch noch jemanden gab, der wohlwollend an sie dachte. Es hätte vielleicht auch 
die Köchin sein können, die in diesem Sessel saß und auf sie wartete. Die junge Gräfin blieb starr 
stehen und kniff die Augen zusammen. Das gelöste Haar fiel ihr in Locken in das Gesicht, die Hände 
zu Fäusten geballt, stand sie zitternd da und gab keinen Ton von sich. Vielleicht spürte der 
Schlafende ihre Anwesenheit, vielleicht ihren Herzschlag oder ihre leisen Atemzüge, auf jeden Fall 
schlug er seine Augen auf. Auch der Bastard regte sich nicht, sah nur stumm auf das zitternde 
Mädchen. \\
``Soll ich gehen?'', fragte er schließlich.\\
``Warum fragst du das?''\\
Er richtete sich auf und strich seinen Wams glatt. Er schien sich unwohl zu fühlen, das lockte ein 
flackerndes Lächeln auf Sarimés Lippen. ``Bisher schien es, als wäre es Euch lieber, alleine zu 
sein, wenn es euch nicht gut geht.''\\
Ihr Lächeln verblasste und sie hob das Kinn. ``Mir geht es gut. Danke der Sorge.''\\
Seine Stirn legte sich in Falten. ``Das heißt, ich soll gehen?''\\
Ihre Schultern sanken herab. ``Nein'', murmelte sie und senkte den Blick.\\
\textit{Wieso fragt er nicht? Wieso sagt er nichts? Wieso tut er nichts?!} Sie seufzte leise und 
musterte ihn direkt. Der Bastard wich ihr aus, blieb aber im Sessel sitzen. \textit{Er 
weiß, dass er nicht hier sein sollte. Im Gemach der Frau seines Vaters. Im Gemach der Gräfin.}\\
``Wie ist es, zu lieben?''\\
Seine Hand zuckten nervös. ``Was meint Ihr?''\\
Sarimé trat näher an ihn heran, drückte ihre Hand auf seinen Brustkorb und schubste ihn sanft 
zurück gegen die Lehne. Ihr Gesicht war so dicht vor dem seinen, dass sie seinen warmen Atem spüren 
konnte. ``Wie. Ist. Es'', sagte sie langsam: ``Zu lieben?''\\
Eine Antwort blieb aus. Renec sah sie nur an. Also tat Sarimé es. Tat, was sie so oft in blöden 
Romanen gelesen hatte. Sie ließ die letzte Distanz zwischen ihnen hinter sich und drückte ihre 
Lippen auf seine. Prüfend blickte sie in seine Augen. Er wehrte sich nicht. Aber er war auch alles 
andere als entspannt. Sarimé richtete sich wieder auf und trat ein paar Schritte in die Richtung 
ihres Bettes. Sie machte eine Handbewegung zur Tür. \\
``Warum hast du gewartet?''\\
Er erhob sich schnell, als fürchtete er einen weiteren Überfall. ``Ich wollte nur sicher gehen, 
dass Ihr...'' \\
``Das ich was? Nicht springe?'' Sie lachte. ``Keine Sorge. Ich werde erst aus einem Fenster 
steigen, wenn ich mir sicher bin, dass mir Flügel gewachsen sind. Flügel, die mich auch tragen 
können.'' Sie zwinkerte ihm zu und musste grinsen, als sie sah, wie unwohl er sich 
fühlte. Sanfter fügte sie hinzu: ``Nein. Ich bin dir wirklich dankbar.''\\
Er zögerte, bevor sein übliches schiefes Lächeln wieder erschien. ``Ihr meint, für den Kuss?''\\
``Das hätte auch eine Strohpuppe gekonnt. Du solltest dich mal rasieren. Aber... nun ja... ich 
wollte einfach wissen, wie es ist.''\\
``Was Ihr wollt, das bekommt Ihr auch, hm?''\\
``Was ich will, das hole ich mir. Das ist ein Unterschied. Aber ich bin bescheiden. Ich will nicht 
viel.''\\
``Nur wissen, wie sich Liebe anfühlt?''\\
Sie zuckte unbeteiligt mit den Schultern. ``Ich überlege noch, ob es so etwas überhaupt gibt. 
Meistens bin ich mir sicher, dass das nur eine Lüge ist. Ein blödes Wort. Und manchmal denke ich 
mir, vielleicht lohnt es sich ja doch, es herauszufinden. Aber nur selten. Sehr selten."\\
\textit{Und dann sehe ich dein Gesicht, wenn du an Sieva denkst. Und dann scheint es plötzlich wahr 
zu sein. Eine Wahrheit, die mir verborgen bleiben wird.}
