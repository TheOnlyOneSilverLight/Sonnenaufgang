\chapter{Hochzeit}

``Ist Merandila so verarmt, dass er kein neues Kleid für mich anfertigen lassen kann?'', rief 
Sarimé und ging unruhig in ihrem Zimmer umher. \\ 
Die Sonne war bereits untergegangen und ihr Gemach war von mehreren flackernden Kerzen erhellt. Sie 
war müde und gleichzeitig aufgewühlt. Und einsam. Sie sehnte sich nach Nähe, nach tröstenden Worten 
und Zuversicht. Nach jemandem, der sich für sie interessierte. ``Nicht einmal das Brautkleid?''\\
``Entschuldigt die Frage, Herrin. Aber habt Ihr mich allein aus diesem Grund gerufen?'', fragte 
Renec und lächelte amüsiert.\\
``Nein!'', erwiderte Sarimé und wandte sich ihm zu: ``Ich meine… ja… ach.''\\
Das junge Mädchen ließ sich auf der Bettkante nieder. \textit{Was habe ich schon zu verlieren?} \\
``Ich wollte mit jemandem sprechen. Und ich kenne niemanden außer dir, Bastard.''\\
Renec betrachtete sie lange aus seinen tiefgrauen Augen. Sarimé fürchtete, er ahnte wo genau ihre 
Sorge lag. Sie hielt es ihm zugute, dass er nicht darauf einging. ``Ihr fürchtet Sievas Kleider?''\\
``Nein, den Geist in ihnen.''\\
``Sorgt Euch nicht, Herrin. Sieva war eine der sanftesten und lieblichsten Frauen, die wohl je auf 
Saleicas Erde wandelten. Ihr Geist wird Euch nichts zuleide tun. Warum sollte es auch so sein? 
Weder hat Sieva Grund Euch zu hassen, noch zu fürchten.''\\
``Sicher? Ich heirate schließlich ihren Gatten… Keine Frau teilt gerne ihren Liebsten. Auch nicht, 
wenn sie bereits im Grabe liegt.''\\
Renec lachte leise. ``Auch da sehe ich keinen Grund zur Sorge. Evin wäre wohl der letzte Mann auf 
Erden, den Sieva ihren Liebsten genannt hätte.''\\
Sarimé schüttelte den Kopf. ``Auch ein einsamer Geist kann gefährlich werden.''\\
``Einsam war sie auch nicht'', erwiderte Renec und sah sie einen Moment finster an. Aber er 
lächelte so plötzlich wieder, dass Sarimé sich nicht sicher war, ob sie seinen Blick richtig 
gedeutet hatte.
\textit{Der Vater nennt sie krank und einen Windgeist, der Sohn nennt sie sanft und lieb.}\\
Es heißt, Windgeister waren Wesen fern von Leben und Tod. Sie zogen durch das Land, sangen ihr 
wisperndes Klagelied und jeder der es vernahm, war dem Untergang geweiht. Ihm geschah das größte 
Elend, seine Albträume wurden wahr und er verlor alles, was ihm am Herzen lag. Das war der Fluch 
der Windgeister, sie predigten ihre Lieder und verfluchten ahnungslose Menschen, die ihren schönen, 
wehmütigen Klängen lauschten. Die Windgeister teilten ihren Schmerz, ohne dass er je kleiner wurde. 
Sie zogen nur noch mehr Wesen mit sich in den Abgrund.\\
``Später könnt Ihr Euch die Kleider zeigen lassen. Ich verspreche Euch, sie werden Euch zusagen. 
Und über die Finanzen der Grafschaft gibt es auch keinen Grund zur Sorge. Merandila hat gutes 
Einkommen. Außerdem stellt uns König Semric zum Beginn jeder Jahreszeit großzügig Gold zur 
Verfügung, um die Grenze zu sichern. Der Graf gibt sein Geld nun einmal lieber für die Ausbildung 
und Finanzierung der Soldaten aus, als für hübsche Kleider seiner jungen Braut. Abgesehen davon, 
dass es bereits viele hübsche Kleider gibt. Erkundigt Euch bei dem Haushofmeister, er wird Euch mit 
Freude die Bücher zeigen und Ihr könnt Euch selbst ein Bild über die Einkünfte und Ausgaben machen. 
Ich wünsche Euch eine erholsame Nacht, Herrin.'' Er verneigte sich und zögerte kurz, dann fügte er 
leiser hinzu: ``Und sanfte Träume.''\\
Sarimé sah noch lange auf die geschlossene Tür. Sie lächelte. Vielleicht war doch eine Seele in 
Merandila freundlich zu ihr. \\

Renec hatte recht behalten, die Kleider waren sehr hübsch. Jedes einzelne. Und aus feinen Stoffen.
\textit{Es wäre wirklich eine Schande, wenn man sie nicht tragen würde.} \\
Sarimé ließ ihre Finger über einen goldgelben Stoff gleiten und seufzte, wie so oft in den letzten 
Tagen. Die Schneiderin hatte bereits damit begonnen, einige Kleider und das Hochzeitskleid zu 
kürzen. Bis dahin müsse Sarime mit ihren eigenen leben. Nun, wenigstens war es nicht so lange, dass 
sie ständig über den Saum stolperte oder dieser im Dreck landete und der Schmutz sich fest in das 
Gewebe sog. Jedes Kleid war handwerklich gut gearbeitet und eine Augenweide. Aber es viel Sarimé 
deutlich auf, in welche Richtung die Farbtöne tendierten. An einer großen, blonden Schönheit mit 
Rehaugen wären sie tatsächlich ein Traum. An ihr würden sie wohl eher lächerlich aussehen… die Töne 
waren zu sanft für ihren Geschmack, aber letzten Endes waren es mehr, als sie je besessen hatte und 
sie beschwerte sich nicht.\\
Die letzten Tage hatte sie damit verbracht, sich in ihren Zimmern wohnlich einzurichten. Sie legte 
selbst Hand an, staubte ab und klopfte die Teppiche aus. Die Zofe, welche am ersten Tag ihr Bad 
gerichtet hatte, war ein paar Mal aufgetaucht, aber Sarimé schickte sie fort. Sie war lieber 
alleine, als in der Gesellschaft von jemandem, der sie nicht leiden konnte. \textit{Zumindest 
wenn es sich vermeiden lässt.}\\
Sie verrückte einige kleinere, leichtere Möbel, entfernte Spinnenweben und dekorierte mit ihren 
wenigen persönlichen Gegenständen den Raum, um ein Gefühl der Vertrautheit zu schaffen. Es gelang 
ihr mehr schlecht als recht, aber es war das Beste, was sie erreichen konnte. Der Mittag war 
bereits verstrichen und Sarimé war auf dem Weg zu den niedrigeren Stockwerken. Sie wollte die 
Dienstmädchen und Knechte kennenlernen. Schließlich sollte sie irgendwann über sie bestimmen und 
haushalten. Und vielleicht fand sie noch jemand anderen als den Bastard, der ihr Gesellschaft 
leisten könnte.\\
In der Küche wurde sie von der freundlichen, leicht kleinwüchsigen Köchin begrüßt. Magda war ihr 
Name und sie erinnerte Sarimé an ihre verstorbene Mutter. Nicht äußerlich. Was das anging, waren 
die beiden Frauen komplett verschieden. Aber das Lächeln, welches direkt vom Herzen zu kommen 
schien, zauberte diese Ähnlichkeit herbei. Sarimé hatte nicht genügend Erinnerungen an ihre Mutter, 
als dass sie diese Begegnung mit Trauer erfüllen könnte. Im Gegenteil, sie fühlte sich geborgen und 
am Ende des langen Gesprächs über ihre Reise, Lieblingsgerichte und den Garten umarmte sie die 
Köchin zum Abschied.\\
Anschließend ging Sarimé zum Haushofmeister Pedan und erkundigte sich nach den Büchern und den 
Finanzen des Haushaltes. Die Unterlagen, welche die gesamte Grafschaft betrafen, behielt Evin in 
seinem Arbeitszimmer und holte lediglich zum Überprüfen der Rechnungen Pedan zu sich. Der 
Haushofmeister belehrte und erklärte eine ganze Stunde lang über die Unterlagen und Sarimé nickte 
brav. Sie tat ihm den Gefallen, denn scheinbar flohen die meisten seiner erwählten Zuhörer recht 
schnell. Deshalb vermied sie es auch, ihn daran zu erinnern, dass sie aus einer Kaufmannsfamilie 
stammte und schon eher zählen konnte, als ihren Namen sprechen. Stattdessen lauschte sie 
interessiert und brachte sich nach angemessener Zeit immer wieder in die Diskussionen mit ein. 
Pedan strotzte vor Stolz, da er glaubte, der baldigen Gräfin etwas beigebracht zu haben, und trabte 
nach dem Abschiedsgruß eilig davon.
\textit{Vermutlich prahlt er jetzt damit}, dachte Sarimé und musste lächeln. Vielleicht war sie 
doch nicht so alleine, wie sie befürchtete. \\

Drei Mal erklang das Klopfen an der Türe und Sarimé rief ohne den Blick zu heben: ``Miraja, ich 
habe dir doch gesagt, du musst nicht immer klopfen.''
Miraja war ein elfjähriges, strohblondes Dienstmädchen, welches ihr seit dem zweiten Tag 
hinterhergelaufen war. Letztendlich hatte sie das Mädchen zu ihrer Zofe ernannt. Sarimé mochte sie. 
Miraja erinnerte sie an ihre Schwestern. Bei dem Gedanken musste sie lachen. \textit{Jetzt habe 
ich eine Mutter in der Küche und eine Schwester als Zofe.}\\
Leises Lachen erklang auf der anderen Seite der Tür. Sarimé horchte auf. Das war nicht 
Mirajas übliches schüchternes Kichern. Sie trat zur Tür und öffnete. Überrascht blickte sie zu 
Renec auf, der sie immerhin um einen Kopf überragte. Verschämt wurde sie rot und stammelte: ``Ja 
bitte?''\\
``Ich hörte, Ihr beginnt Euch einzuleben?''\\
``Wie meinst du das?'', fragte sie zögernd.\\
``Nun, die Köchin ist ganz begeistert von Euren Kenntnissen über Braten… und wenn Ihr der kleine 
Miraja bereits erlaubt, in Euren Zimmern nach Belieben ein und aus zu gehen…''\\
``Ist das denn falsch?'', fragte Sarimé erschrocken.\\
Renec legte den Kopf schief und grinste. ``Nun… solange Ihr diese Gunst nicht allzu vielen Leuten 
schenkt… vor allen keinen männlichen.''\\
``Was willst du?'', fragte sie ihn eingeschnappt.\\
``Ich dachte, da Ihr Euch die Burg nun angeschaut habt, wollt Ihr vielleicht auch etwas mehr von 
der Grafschaft sehen. Aber wenn Ihr zu beschäftigt seid, kann ich natürlich wieder gehen. 
Langeweile ist das Letzte, was ich habe.'' Es klang fast schon vorwurfsvoll. Sarimé senkte den Blick 
und wurde noch röter.\\ 
``Verzeih mir, ich war unhöflich. Ich habe dich nicht einmal hereingebeten.''\\
``Unhöflichkeit bin ich gewöhnt, Herrin'', sagte er aufmunternd: ``Schließlich bin ich ein 
Bastard.''\\
Sarimé überlegte laut. ``Ich habe ein Gemälde von Evin gesehen, als er jünger war. Er sah fast 
genauso aus wie du jetzt.''\\
``Heißt das, ich werde genauso hässlich, wenn ich alt bin?'', rief er gespielt schockiert.\\
Sarimé lächelte. ``Es wäre mir eine Ehre, mir von Euch die Ländereien zeigen zu lassen'', sagte sie 
und machte einen Knicks.\\
``Oha… ich bin kein Prinz, Herrin'', lachte er und bot ihr seine Hand an: ``Und die Ländereien sind 
groß. Ich zeige Euch lediglich die Felder und das nächstliegende Dorf. Ich hoffe, das genügt?''\\
``Gewiss.''\\

 ``Wo steht die Kutsche?''\\
``Wir nehmen keine Kutsche, Herrin. Ich habe die Pferde bereits satteln lassen.''\\
Es war ein gutes Jahr her, dass sie auf einem Pferd saß. Und das dann meistens auch nur im Schritt. 
Auf den Straßen Brom-Dalars herrschte zu viel Verkehr für eine schnellere Gangart. Sarimé nickte 
nur und wartete auf Renec, der die Pferde aus dem Stall holte. Etwas ängstlich hoffte sie, ein 
ebenso ruhiges Tier wie den Wallach zu bekommen. Heraus kam Renec mit einer jungen Stute. Ihr Fell 
war tiefschwarz und zeigte keinen einzigen hellen Fleck. Die Mähne war fein gekämmt und glänzte im 
Licht der Nachmittagssonne. Die Stute war eher klein und zierlich, doch tänzelte sie unruhig und 
warf den Kopf hoch in die Luft, sodass die Mähne flog. Sattel und Zaumzeug waren ebenso schwarz wie 
das Fell und die Hufe hatte der Stallmeister polieren lassen. Dieser stand im Tor und beobachtete 
sie neugierig. Es war ihm deutlich anzusehen, wie stolz er auf dieses Tier war und er wollte jetzt 
sehen, wie die neue Herrin auf den Schatz des Stalles reagieren würde. Sarimé bekam eher Angst. Die 
Stute war völlig gegensätzlich zu dem ruhigen alten Wallach. Auch Renec blickte sie interessiert an 
und wartete auf irgendwelche Worte.\\
``Oh…'', stammelte Sarimé: ``Hübsch…''\\
``Ein Geschenk Eures baldigen Gemahls, Herrin'', erklärte Renec und streichelte der Stute den Hals: 
``Ruhig, Mädchen.'' Letzeres sagte er natürlich zu dem Pferd, Sarimé fühlte aber trotzdem, wie die 
leisen Worte auch auf sie eine beruhigende Wirkung hatten. Zumindest etwas. ``Du scheinst die Stute 
sehr gut zu kennen?''\\
``Ich war dabei, als sie geboren wurde. Und ich habe sie ausgebildet. Seid gewiss, Herrin, die 
Schwarze ist ein fabelhaftes Tier. Noch jung, aber sie ist gerade nur so unruhig, weil sie länger 
im Stall war und laufen möchte. Sie hört auf das kleinste Zupfen am Zügel und auch auf Euer Wort. 
Sie springt über Bachläufe und ist schnell wie der Wind.''\\
\textit{Für Sieva ausgebildet?}
``Vielleicht wäre sie besser als ein Jagdpferd geeignet'', sagte Sarimé: ``An mich ist sie doch 
verschwendet.''\\
Renec erwiderte dazu nichts. Ein grimmiger Zug legte sich über seine Lippen, aber nur flüchtig und 
statt noch etwas zu sagen, reichte er ihr die Zügel und ging zurück in den Stall. Sarimé strich der 
Stute über die Schläfe und die Nüstern. \textit{Er ist zu höflich, um mir zuzustimmen.}\\
Immerhin der Stallmeister kam humpelnd auf sie zu. In einer selbstverständlichen Geste trat er von 
der Seite auf die Stute zu, legte eine Hand an ihre Schulter und strich an ihr entlang, als er zu 
Sarimé trat. ``Herrin. Wenn Ihr Fragen habt, ich stehe stets zu Euren Diensten!''\\
Er deutete eine Verneigung an, aber die Bewegung fiel ihm sichtlich schwer. Der Mann war weit über 
seine Lebensmitte hinaus. Ein Flaum kurzgeschorenes Haares zierte nur doch die Mitte des sonst 
kahlen Kopfes. Er hatte große Hände, welche mit Schwielen und Narben übersät waren. Der 
Stallmeister war kaum größer als sie selbst und machte keine eindrucksvolle Erscheinung. Aber das 
Pferd beruhigte sich augenblicklich, sah ihn mit aufmerksamen Augen und gespitzten Ohren an.
\textit{Die Tiere respektieren ihn.}\\
``Ich... es tut mir Leid. Die Stute ist so ein wunderbares Tier, ich glaube nicht, dass ich gut 
genug für sie bin'', stieß Sarimé aus und senkte beschämt den Kopf. Nein, sie fühlte sich wirklich 
nicht gut genug. Und das nicht nur auf das Pferd bezogen. Sie war keine Herrin, sie war keine 
Gräfin! Sie war nicht mehr als eine verarmte Kaufmannstochter, deren Leben ein einziges Schauspiel 
gewesen war. Der Mann betrachtete sie eingehend. ``Man erkennt die Herzen der Menschen daran, wie 
die Tiere ihnen begegnen. Ein Pferd spürt jede Angst, ein Hund jede Sorge, eine Raubkatze jede 
Schwäche. Schaut Euch das Pferd eines Mannes an und ihr kennt sein Herz. Behandelt man die Tiere 
schlecht, erkennen sie jeden Hass, jeden Zorn und jede Grausamkeit in den Herzen der Menschen. Diese 
Dinge erlernen sie erst von den Begegnungen mit schlechten Menschen.''\\
``Ihr arbeitet schon sehr lange mit den Pferden?'', fragte sie, gebannt von seinen Worten.\\
Er nickte. ``Seit ich denken kann. Und mit den Hunden, den Falken und den Raubkatzen. Ich richte 
sie seit Jahrzehnten für den Herren ab. Die Stute ist freundlich, denn ihr ist noch nie ein 
schlechter Mensch zu nahe gekommen. Das schwöre ich. Die neuen Burschen lasse ich immer erst an die 
Katzen, die wehren sich eher als Pferde. Die zeigen den Burschen schon, wenn sie nicht hierher 
gehören. Dulden die Katzen deren Nähe, dann müssen sie sich den Falken beweisen, schließlich den 
Jagdhunden, dann den Hofhunden und zuletzt den Pferden. Aber ich schwöre Euch, ich würde niemals 
meiner Herrin ein Pferd geben, welches Hass erlebt hat.''\\
Wieder neigte er den Kopf, dieses Mal lächelte er aber breit. Er war glücklich darüber, eine neue 
Zuhörerin gefunden zu haben. Sarimé lauschte seinen Worten fasziniert und ergriff die Zügel der 
Stute. Das Tier blickte sie aufmerksam aus dunklen Augen an und schnaufte leise. \\
``Guten Tag. Ich heiße Sarimé Sil'Vera'', stellte sie sich der Stute vor. \\
Das Lächeln des alten Stallmeisters wurde zu einem leisen Lachen und mit diesem rauen Klang wirkte 
der Mann sogleich etliche Jahre jünger. ``Pferde haben sanfte Seelen. Sie flüchten lieber, als 
dass sie kämpfen. Sie brauchen einen Reiter, der sie stark und entschlossen führen kann, dies aber 
mit sanfter Hand macht. Entschuldigt, Herrin, aber ich muss wieder los. Die Falken müssen gefüttert 
werden. Die Burschen sind meist zu hektisch und regen die Vögel nur auf, daher muss ich dabei sein. 
Ich wünsche Euch einen angenehmen Ausritt. Und darf ich Euch noch einen weiteren Rat geben?''\\
Sarimé nickte auffordernd.\\
``Die Stute weiß, was sie tut. Vertraut ihr und sie wird euch vertrauen. Und wenn Ihr in Not seid, 
wendet Euch an Renec. Er ist ein exzellenter Reiter.''\\
Er verabschiedete sich mit einem weiteren Kopfneigen. Als er zurück in den Stall trat, fiel Sarimé 
auf, dass er das eine Bein nachzog und ihn humpeln ließ. Besorgt runzelte sie die Stirn. 
\textit{Vielleicht sollte man bald einen zweiten Stallmeister einstellen, der die Last der Arbeit 
mit ihm teilt.}\\

Der Bastard zeigte ihr die umliegenden Ländereien, sie ritten über Wiesen und an Feldern vorbei. 
Sie ritten an dem Dorf vorüber und tränkten an dem naheliegenden See die Pferde. Das Dorf hieß 
Talsmund und war angeblich so alt wie die Festung selbst. Sarimé sah es nur aus einigen hundert 
Metern Entfernung, aber es wirkte friedlich und harmonisch. Als Sarimé sich wieder in den Sattel 
zog – mit deutlich mehr Geschick als beim ersten Mal – fiel ihr der nahe Wald auf. ``Ich möchte 
dort entlang'', entschied das junge Mädchen.\\
Renec folgte ihrem Blick und schüttelte den Kopf. ``Es ist schon spät. Wir sollten zurückkehren.''\\
Sie warf ihm einen skeptischen Blick zu. Keinesfalls würde sie sich auf eine Diskussion mit ihm 
einlassen! Sanft stupste sie die Stute und nahm die Zügel auf. Kurz darauf trabte sie bereits aus 
dem Dorf, auf den Waldrand zu. Renecs frustriertes Brummen entging ihr nicht und sie musste 
grinsen.\\
``Mit Verlaub, Herrin, aber Ihr seid stur!'', rief der Bastard, als er sie eingeholt hatte.\\
``Ich weiß.''\\
Renec zeigte ihr einen Weg in den Wald. Wagenspuren, Huf- und Schuhabdrücke zeugten davon, dass er 
öfters benutzt wurde. Sarimé betrachtete den Bastard nachdenklich, während die Pferde nebeneinander 
hertrotteten.\\
``Wie heißt die Stute?'', brach sie das Schweigen.\\
``Wie immer Ihr sie nennen wollt, Herrin.''\\
Sarimé schüttelte den Kopf. ``Du hast sie aufgezogen, du hast ihr gewiss schon einen Namen gegeben. 
Sprich.''\\
Er zögerte. ``Hexe. Es ist kein schöner Name. Nicht für das Pferd einer Dame.''\\
Lachend erwiderte sie: ``Sei ehrlich, sehe ich für dich aus wie eine Dame?''\\
Er kniff die Augen zusammen und musterte sie eingehend. ``Mit genügend Haarklammern, einem schönen 
Kleid und einer Maske, die Euer Grinsen verbirgt, auf jeden Fall!''\\
``Ich möchte eine Dame sein, die von Herzen lachen kann, ohne sich schämen zu müssen. Ich scheue 
mich nicht davor, mein Feuer selbst zu entzünden und die Eimer voll heißem Wasser selbst zu meinem 
Bad zu schleppen. Ich kann besser rechnen als so mancher Mann, lese gerne Sachbücher und wenn ich 
auf einem Instrument spiele, klingt es, als würde ich gerade einen Hahn erwürgen. Und das einerlei 
auf welchem Instrument!''\\
Er wählte seine nächsten Worte mit Bedacht. ``Eine Gräfin muss ihr Volk kennen. Sie muss verwalten 
und richten. Sie sollte Wissen über die Tätigkeiten der Berufe haben und sich auch mit den Niederen 
verständigen können. Ihre Worte müssen wie der Wind sein, der die sterbende Glut ihrer Soldaten zu 
einem neuen Inferno entfacht. Sie muss vorausahnen können, klug sein und wachsam. Aber ich habe 
noch nie gehört, dass ein Krieg, Leid oder Ungerechtigkeit durch ein Lied beendet wurde.''\\
``Welch Trost. Alles andere kann ich lernen. Nur das musizieren ist hoffnungslos'', antwortete 
Sarimé, aber der Schalk war aus ihrer Stimme gewichen. Sie meinte es ernst. Renec warf ihr einen 
verblüfften Seitenblick zu. \textit{Hat er vorgehabt, mich mit diesen Worten einzuschüchtern?}\\
``Dein Vater sagte, er braucht keine Frau, die nur hübsch anzusehen ist... Was braucht er dann?''\\
``Eine Gräfin.''\\
Anscheinend war Renec gerade in Stimmung zu reden und die Gelegenheit nutzte Sarimé. ``Erzähl mir 
deine Geschichte, Bastard.''\\
Renec neigte leicht seinen Oberkörper, um einem Ast auszuweichen. ``Habe ich das nicht bereits?''\\
Sie schüttelte den Kopf. ``Du hast Sievas Geschichte erzählt, nicht deine.''\\
``Da gibt es nicht viel. Aufgewachsen bin ich im Niemandsland zwischen Kasir und Merandila. Wir 
hatten einen winzigen Hof, fünf Hühner und eine Kuh. Es ist amüsant, dort zu leben. Man hört 
gleichermaßen Gerüchte über Saleica, wie über Kasir. Saleicaner opfern ihre Kinder dem Gott, Kasira 
wärmen sich im Winter mit Menschenhaut. Solche Gerüchte, die belegen sollen, wie unzivilisiert doch 
das andere Land ist.'' Er schmunzelte. ``Sieva holte mich kurz nach ihrer Hochzeit mit meinem Vater 
an den Hof. Er gab ihr nach und ich wurde als sein Bastard anerkannt. Ich lernte lesen, schreiben, 
reiten und fechten. Ich spreche sowohl unsere Sprache, als auch Kasirisch.''\\
``Und was sind deine Aufgaben?'', hakte Sarimé nach.\\
Renec streckte die Hand aus und deutete auf einen schmalen Pfad, der sich vom Weg entfernte. ``Ich 
habe meinem Vater einige Anliegen der Burg abgenommen. Die Bediensteten kommen zum Beispiel mit 
ihren alltäglichen Sorgen zu mir. Und... Ihr wisst ja, dass Sieva krank war. Ich habe versucht, 
ihr, so viel es möglich war, zu helfen.'' Er verstummte und Trauer legte sich auf seine Züge.
``Es tut mir leid, Renec'', murmelte Sarimé.\\
\textit{Ihr Tod lastet noch schwer auf ihm.}\\
Renec lenkte sein Pferd den Pfad entlang und somit waren sie gezwungen, hintereinander zu reiten. 
Sarimé vermutete, dass tat er nun, um Distanz zu ihr zu gewinnen. Sie senkte den Blick kurz auf die 
Mähne ihrer Stute und schüttelte kaum merklich den Kopf. \textit{Ich sollte aufhören, ihn mit 
Fragen zu bedrängen.}\\


Die nächsten Tage verbrachte sie bei den Bediensteten, plauderte mit ihnen und legte auch selbst 
Hand an. Sie kochte mit der Köchin und ließ sich vom Stallmeister die Falken zeigen. Sie durfte 
sogar den Lederhandschuh anlegen und einen Vogel auf der Faust sitzend halten. Jeden Abend speiste 
sie mit dem Grafen und sie unterhielten sich über Förmlichkeiten. Sie fragte nach den Gästen die 
zur Hochzeit geladen waren, erkundigte sich nach den Beziehungen zu den Nachbargrafschaften und den 
Adeligen Merandilas. Auch Talsmund besuchte sie. Die Tage verstrichen allzu schnell und mit jedem, 
der verstrich, fühlte Sarimé sich wohler. Der Bastard jedoch war nachdenklicher geworden. Man 
konnte es nicht einmal als unhöflich bezeichnen, nur stiller. Oft spürte sie seine Blicke auf sich 
ruhen.\\
Als der Tag der Hochzeit gekommen war, lebte Sarimé bereits seit drei Wochen auf der Burg. Die Zeit 
war schnell verstrichen und sie hatte sich kaum noch Gedanken über ihre Eltern gemacht. Sie 
vermisste ihre Familie, aber ihr Vater gehörte nicht mehr dazu, seit er sie verstoßen hatte. Nein, 
eher dachte sie an ihren Bruder Mires und seine Frau Suja. \textit{Wie alt ihre Kinder nun wohl 
schon sind?}\\
Miraja öffnete, wie jeden Morgen, die Fensterläden und ließ frische Luft in das Zimmer. Sie stellte 
das Frühstückstablett ab und räumte Sarimés Gewand vom letzten Abend fort, welches sie aus 
Müdigkeit einfach auf den Boden hatte fallen lassen. Sarimé beobachtete die junge Zofe träge und 
blieb reglos im Bett sitzen. Nichts außerhalb der aufgewärmten Felle und Decken erschien ihr 
verlockend genug, um aufzustehen.\\
``Die Schneiderin bringt bald das Kleid'', erklärte Miraja: ``Die Köchin hat Euch extra Brot 
gebacken und süßen Brei mit geschickt.''
Sarimé mochte das Mädchen. Aber jetzt wollte sie keine Gesellschaft. ``Danke. Lass mich dann 
bitte allein.''\\
``Aber Ihr braucht doch Hilfe!''\\
``Die Zeremonie ist erst im Abendrot und ich komme heute alleine zurecht. Geh, ich entbinde dich 
von deinen heutigen Pflichten. Besuche deine Mutter und freue dich auf die Feier heute Abend.''\\

Nachdem das Mädchen das Zimmer verlassen hatte, erhob sich Sarimé und trat an das Fenster. Sie 
lehnte sich hinaus und sah hinunter auf die grünen Hügel Merandilas. In wenigen Stunden würde sie 
die Gräfin dieses Landes sein. In wenigen Stunden würde sie Evin Treue und Liebe schwören und mit 
ihm das Bett teilen. Entgegen der Behauptung ihrer Stiefmutter, dass man jeden Mann lieben lernen 
kann, glaubte sie nicht daran. Sie bezweifelte, dass sie Evin je liebevoll ansehen könnte. Ihn je 
\emph{Liebster} nennen könnte.\\
Renec kam ihr in den Sinn. Der charmante junge Mann mit dem zerzausten schwarzen Haar und Augen wie 
Sturmwolken. Auch wenn er sich die letzten Tage zurückgezogen hatte, waren es immer wieder kleine 
Aufmerksamkeiten, die dafür sorgten, dass er in ihren Gedanken blieb. Nette, aber kurze Worte am 
Morgen. Ein Blick, der ihr über den Hof folgte. Und vorgestern, als sie den Pferden auf der Weide 
zusah, schenkte er ihr eine einzelne Wildblume.\\
Evin war kein grausamer Mensch, aber auch niemand, in den sie sich verlieben könnte. Und Renec? 
Sein Herz hing an Sieva. Egal wie lange sie bereits verstorben war, sie sah es in seinem Blick. 
Jedes Mal wenn man ihren Namen aussprach oder etwas erwähnte, was mit ihr in Verbindung stand. 
Sarimé fühlte sich schlecht dabei, weil sie es regelrecht ausprobierte, ab wann sich seine Miene 
verfinsterte und er sich unter irgendeinem Vorwand entschuldigte.\\
Sie mochte ihn und deshalb versetzte es ihr einen Stich, wenn sie ihn wieder mit einer Andeutung 
verletzte. Aber sie musste auch wissen, was die vergangenen Jahre in der Burg geschehen war. Der 
Bastard hatte die Gräfin geliebt. Sie hatte ihn auf die Burg geholt und ihn lernen lassen. Sie gab 
ihm eine Anstellung und die Anerkennung seines Vaters. Bestimmt hatte die Liebe auf Gegenseitigkeit 
beruht und vermutlich hatten sie auch das Bett geteilt. \textit{Es schadet nie zu wissen, wer in 
dieser Burg was weiß und denkt.}\\
``Der Bastard des Grafen mit dessen Gattin'', murmelte Sarimé und richtete sich wieder auf. Das 
wäre ein regelrechter Skandal und Sarimé konnte sich nicht vorstellen, dass Evin das wusste. Er 
hätte es schon alleine auf Grund seiner Ehre nicht geduldet. Vermutlich wusste es jeder Bedienstete 
hier. Aber Evin schien sich auch nie wirklich für Sieva oder seinen Bastard interessiert zu haben. 
Vielleicht hatte er gerne die Augen vor diesem Geheimnis verschlossen.\\
Sarimé begann den Tag damit, ihr Frühstück zu sich zu nehmen und anschließend sofort wieder in das 
Bett zu verschwinden. Sie las die nächsten Stunden eines von Renecs Büchern durch und stand erst 
auf, als es an der Türe klopfte und eine Magd Eimer voll heißem Wasser für ein Bad herein brachte. 
Sarimé schickte auch sie nach getaner Arbeit wieder fort und streckte sich in der Wanne aus.\\

Als sie wieder in ihr Hauptzimmer trat, fand sie das Hochzeitskleid vor. Es war bereits später 
Nachmittag, also schlüpfte sie hinein und strich es an sich glatt. Im Spiegel musterte sie sich. 
Sie war bald eine Gräfin und wollte auch so aussehen. Das Kleid war wahrlich zauberhaft und stand 
ihr gut, obwohl auch dieses aus Sievas Bestand war. Aber Sarimé konnte sich nicht vorstellen, dass 
Sieva es je getragen hatte. Es war aus einem dunklen grünen Stoff mit bronzefarbenen Stickereien 
verziert. Sievas übrige Kleider waren stets von zarten Farben, voller Verzierungen und ihr Schmuck 
aus Silber. Der Farbton ließ bei der Rothaarigen dagegen ihr Haar noch feuriger erscheinen und der 
Stoff selbst schien beinahe den selben Farbton wie ihre Augen zu besitzen. Die Stickereien 
erinnerten an die Blätter einer Rose und zierten lediglich die rechte Seite des Kleides, vom Saum 
bis zur Schulter hinauf. \\
Ihre Frisur richtete Sarimé ebenfalls selbst. Sie setzte sich an den Tisch und band das Haar mit 
einem Stoffband in der Farbe des Kleides hoch und fixierte den Knoten mit fünf hölzernen Stäbchen, 
die in einem Halbkreis aus dem Haarknoten hervorstachen. Die Stäbchen waren aus dunklem Holz und an 
den Spitzen vergoldet. Einzelne Haarsträhnen zupfte Sarimé aus dem Knoten, um die Frisur nicht 
allzu streng wirken zu lassen. Ihr Gesicht wurde mit feinem Puder bedeckt, bis keine einzelne 
Sommersprosse mehr zu sehen war. Zuletzt wählte Sarimé eine mehrgliedrige Goldkette und legte drei 
vergoldete Armreife um ihr rechtes Handgelenk. Klingende Geräusche erfüllten den Raum, wenn sie den 
Arm bewegte. Sie erhob sich und warf einen letzten Blick in den Spiegel.\\
``Sarimé Sil'Vera, Gattin des Grafen Evin A'Rik und Gräfin Merandilas'', sagte sie leise zu sich 
selbst.\\

Osyma sprach einst, dass Verbindungen zwischen Menschen unter freiem Himmel, bei Morgengrauen oder 
Abendrot stattfinden müssten. Flammen, die mit seinem Namen gesegnet waren, sollten über die 
Verbindung wachen. Denn durch das Feuer konnte er jedes Wort hören, jede Geste sehen und jede Lüge 
enttarnen. Dieser Brauch bezog sich auf Eheschließungen ebenso, wie auf Schwüre und Versprechen. Um 
die Regelung zu vereinfachen, hatten die Priester Osymas verkündet, dass jede Tempeldecke die 
Malerei eines Himmels zeigte und somit die Anforderungen erfüllte. Aber der einzige große Tempel in 
Merandila befand sich in der Stadt Na'Rash, drei Tagesritte von der Festung entfernt. Der Graf hatte 
beschlossen, dass sich diese Mühe kein weiteres Mal lohnte. \\
Sarimé war es recht. Wie jedes Mädchen hatte sie einst von einer wundervollen Hochzeit mit der 
Liebe ihres Lebens geträumt. Mittlerweile beschränkten sich ihre Träume jedoch auf einfachere 
Dinge. Ein gutes Buch, ein Ausritt mit Hexe, ein interessantes Gespräch mit den Bediensteten. 
\textit{Und Renec}, dachte sie unwillkürlich und ihre grünen Augen weiteten sich. \textit{Habe ich 
das wirklich gerade gedacht?}\\
Sarimé warf einen kurzen Blick zum Stalleingang, aus dem Renec gerade ihre Rappstute führte. 
\textit{Nein. Es gibt keine Liebe. Das sind bloße Worte in Romanen - geschrieben von Männern mit zu 
viel Zeit und für Frauen, die sich lieber in erfundene Welten flüchten, als sich der Realität zu 
stellen.}\\
``Ihr seht hübsch aus, Herrin'', erklärte Renec und drapierte die Zügel des Pferdes um den 
Sattelknauf. ``Aber vielleicht ein bisschen zu viel rote Farbe auf den Wangen.''\\
Sarimé hatte nur weißes Puder benutzt, aber auch das reichte anscheinend nicht, um ihre geröteten 
Wangen zu verbergen. Sie beschloss, auf das Kommentar nicht einzugehen und trat mit unsicheren 
Beinen auf die Trittleiter. Schwer gerüstete Soldaten nutzten solche Hocker, um leichter in den 
Sattel zu kommen. Ebenso die Damen, die es nicht lassen konnten oder durften, in einem Kleid zu 
reiten. Auf dem Pferderücken versuchte Sarimé eine bequeme Position zu erlangen und Hexe unter ihr 
schnaufte gereizt. \textit{Sie mag den Damensattel nicht.}\\
Sarimé seufzte und tätschelte den Hals der Stute. ``Glaube mir, ich will es genauso wenig wie du. 
Aber wir haben keine Wahl.''\\
Der Abend war windig. Der Sonnenuntergang wurde von Wolken umlagert, die sich rötlich verfärbten. 
Vom Hügel aus konnte Sarimé sehen, wie das Gras sich sanft dem Drängen des Windes beugte. Zwei 
Fackeln tanzten im Mittelpunkt der Menschenmenge. \textit{Wenn ich jetzt noch wüsste, wer diese 
Menschen sind...}\\
Sarimé beugte sich im Sattel vor und strich über den Hals der Stute. Eine vertraute Wärme ging von 
dem Tier aus und sie verspürte das Verlangen, ihr Gesicht tief in der schwarzen Mähne zu vergraben.
Renec räusperte sich. ``Die Gäste warten, Herrin.''\\
Das Mädchen blickte über ihre Schulter und einen Moment lang trafen sich ihre Blicke. In seinen 
Augen meinte sie ebensolchen Kummer zu sehen, wie der, der gerade in ihrer Seele hauste. Aber 
vielleicht bildete sie es sich auch wieder nur ein. ``Na los'', murmelte sie der Stute zu und 
lenkte sie zur Menschenmenge.\\
Sie sah Evins große Gestalt bei den Fackeln stehen und die Gäste bildeten eine Gasse, als Sarimé 
näher kam. Sie konnte einen weiteren Blick über die Schulter nicht verhindern. Renec war ihr nicht 
gefolgt. \textit{Ich bin alleine}, dachte sie und spürte, wie etwas in ihr erzitterte. 
Augenblicklich richtete sie sich höher auf. Ihre Finger umschlossen die Zügel fester, sie hob ihr 
Kinn und lächelte Evin an, wie es eine Braut und baldige Gräfin tun sollte. Ihre Angst, Sorge und 
Hilflosigkeit hatten sich in Resignation verwandelt. \textit{Es gibt keine Liebe. Osyma hält unser 
aller Leben in den Händen und ich kann mich nicht gegen seine flammenden Klauen wehren.}\\
Der Priester war ein großgewachsener, hagerer Mann. Für sein noch recht junges Alter machte er 
bereits ein viel zu ernstes Gesicht. Sarimé fragte sich, ob er etwas erlebt hatte, das eine solche 
Einstellung rechtfertigen würde oder ob er es sich durch Übung angeeignet hatte. Ebenfalls schien 
er noch nicht lange dem Priesterstand anzugehören, denn je treuer man zu Osyma war, desto mehr 
Tätowierungen zierten die Männer und Frauen. Dieser hier, der Sarimé und Evin das Versprechen der 
Ehe abnehmen sollte, besaß verschlungene Tätowierungen an Hals und Schlüsselbein und ebenso an 
seinem rechten Handgelenk. Da die Muster immer zusammenhingen, vermutete Sarimé, dass unter seinem 
Gewand die Zeichnungen von Handgelenk und Hals sich trafen.\\ 
Das Mädchen glitt aus dem Sattel und sofort ergriff einer der Bediensteten die ledernen Zügel. 
Sarimé stellte sich mit wenigen Schritten neben den Grafen, ließ aber mehr Abstand als es ziemlich 
war. Das entging nicht einmal dem Priester, denn er kniff die Augen zusammen. Statt einem Tadel 
breitete er jedoch seine Arme aus und begann mit der Zeremonie. Seine Stimme war laut und hallend, 
als er zu sprechen begann: ``Zu dieser Stunde, zwischen Tag und Nacht, kehren wir hier zusammen. Zu 
dieser Stunde, in der der allmächtige Osyma auf unserer Sphäre wandeln und diesem Akt der 
Vereinigung zweier seiner Kinder seine Aufmerksamkeit schenken kann. Diese Vereinigung ist sein 
Wille und diese zwei dürfen von nichts getrennt werden, außer es ist der Wille des Mächtigen, in 
dem er sie durch Krankheit und Tode trennt. Graf Evin A’Rik, Osyma schenkte Euch das Leben in 
Merandila, Ihr wuchst unter seinem Schutz auf und wurdet dazu erwählt, diese Grafschaft für den 
gesegneten König Semric zu bewahren. Er schenkte Euch drei Erben und drei Erben nahm er euch, 
ebenso zwei Gattinnen. Betet, dass Euch ein solches Unglück kein weiteres Mal ereilt und bittet 
ergeben um seine Gnade und eine kinderreiche Ehe. Gelobt Ihr vor den allsehenden Augen des 
Mächtigen, die Treue gegenüber Osyma und Eurem Weib?''\\
Besagter Graf Evin A'Rik nickte nur gemächlich und lächelte amüsiert. \textit{Seine dritte Ehe. 
Vermutlich kann er die Predigt selbst fehlerfrei zitieren.}\\
Wieder wirkte der Priester alles andere als erfreut oder auch nur entspannt. Evins Reaktion war ihm 
nicht ernsthaft genug. Sarimé musste grinsen. Es war fast schon lächerlich. Da standen sie nun, 
umgeben von einer Hundertschaft an Gästen und versprachen sich Treue und Liebe.
\\textit{Mein Vater hat mich vermutlich schneller an Evin verkauft, als dieser Priester auch nur 
zum Luft holen Zeit braucht.}\\
``Sarimé Sil'Vera!'', sagte er schneidend und richtete seinen Blick auf das Mädchen.
Natürlich, er wagte es nicht, einen Grafen strafend an zu blicken. Die junge Braut war eine andere 
Angelegenheit. Sarimés Lächeln verblasste und sie sah ihm streng in die Augen. Hoffte darauf, dass 
sie ein ebensolches Funkeln besaß, wie der zornige Priester.
``Ihr wart geboren in unserer glorreichen Hauptstadt Brom-Dalar, direkt unter dem Schutz hunderter 
Priester des Mächtigen, ein gutes Omen. Eure Mutter gab Ihr Leben um Euch das Eure zu gewähren. 
Osyma schützte Eure zarte Seele. Dafür seid Ihr ihm auf ewig zu Dank und Treue verpflichtet. Gelobt 
Ih, vor den allsehenden Augen des Mächtigen, die Treue gegenüber Osyma und Eurem Gatten?''\\
``Ja.''\\
Der Priester nickte knapp. Sein Lob dafür, das zumindest einer sich an das Protokoll hielt. Während 
er weitere Reden schwang, über Osymas Macht und Erhabenheit, hingen Sarimés Gedanken noch bei 
diesem kurzen Wort. \textit{Ja.}\\
Wie emotionslos dieses Wort geklungen hatte. Unpassend, für eine junge, aufgeregte Braut. Sie hätte 
strahlen sollen, ihrem Gatten einen verliebten Blick zuwerfen müssen und vor Vorfreude nicht 
stillhalten dürfen. Bitter blickte sie auf das niedergetrampelte Gras zu ihren Füßen. Zumindest 
wurden die Bräute in Romanen so beschrieben. Jetzt wagte sie auch tatsächlich einen Blick zu Evin. 
Er hatte die Augen zum Horizont gewandt und schien die Wolken zu studieren. Vielleicht rätselte er, 
ob es heute noch regnen würde.\\
Ruckartig hob der Priester seine Hände gen Himmel und legte den Kopf in den Nacken. Die weiten 
Ärmel seines groben Gewandes fielen zurück und offenbarten seine blasse Haut. Wie Sarimé vermutet 
hatte, war nur der rechte Arm tätowiert. Doch ihr Blick hing viel mehr an dem, was er in den Händen 
hielt. Die untergehende Sonne blendete ihre Sicht, daher konnte sie nicht viel erkennen. Außerdem 
erfassten die Sonnenstrahlen ebenfalls die beiden Gegenstände und ließ sie funkeln. \\
``Osyma, allmächtiger Gott, sei uns gnädig und segne mit deinem Licht und deiner Güte diese Bänder 
der Verbundenheit. Lass sie diese Eheleute stets an ihr Gelöbnis erinnern, an ihre Verbundenheit 
und an ihre Treue. Segne sie und schenke ihren Gebeten und Wünschen in der heutigen Nacht deine 
Aufmerksamkeit. Und, wenn sie deiner würdig sind, erfülle sie!''\\
Um seinen Worten Wirkung zu verleihen, stand der Priester noch einen Moment reglos da. Dann, als 
der letzte Strahl der Sonne hinter einem Hügel verblasste, senkte er seine Arme. Im dämmrigen 
Abendlicht war kaum noch etwas zu erkennen, doch die Fackeln spendeten zumindest dem Brautpaar 
etwas Licht. Sarimé sah die beiden Armbänder nun besser. Evins war grob und breit, mit 
eingravierten Mustern aus Wirbeln und Schlangen. Ihres dagegen feingliedrig und zart. Jedes Glied 
war wie eine kleine Blüte geformt. Sie erkannte Rosen, Malven, Sonnenblumen und Lilien, welche sich 
in regelmäßigen Abständen wiederholten. Das Band musste Unsummen gekostet haben. Kaum ein 
Goldschmied war in der Lage, so fein zu arbeiten. \textit{Vermutlich bin ich auch schon mindestens 
die Zweite, die es trägt}, dachte Sarimé. Aber sie musste sich trotzdem eingestehen, dass sie 
niemals mit etwas so schönem gerechnet hätte.\\
Sarime streckte ihren linken Arm aus. Evin wurde gerade sein Band angelegt, nun wandte sich der 
Priester ihr zu. Geschickt schlossen seine langen Finger den Verschluss und er nickte ihr 
auffordernd zu. Ehe Sarimé sich versah, ergriff Evin routiniert ihre linken Hand. Er hob die ihre 
mit in die Höhe, sodass die Gäste die beiden Bänder und somit die Verbundenheit des Paares sehen 
konnten. Sarimé musste sich auf die Zehenspitzen stellen und trotzdem hielt er ihren Arm so hoch, 
dass es schmerzte. So plötzlich wie er sie ergriffen hatte, ließ Evin die Hand auch wieder fallen.\\
``Und nun, lasst uns feiern!'', verkündete er laut.\\
Lautstark stimmten seine Gäste zu und machten sich eilig auf den Weg zur Festung. Ein leises 
Donnern kündigte den nahenden Sturm an und auch wenn sich die Wolken nur am Horizont türmten, 
wollte niemand das Risiko eingehen, seine feine Garderobe zu ruinieren.\\

In ihrem Leben hatte sie kaum Feste besucht. Als Gastgeber hatte die Familie Sil'Vera nie fungiert, 
dafür fehlten die finanziellen Mittel und die Beziehungen. Mit Sarimés fortschreitendem Alter hatte 
sich ihre Stiefmutter jedoch bemüht, sie immer mal wieder auf kleinere Feiern zu bringen, um 
potentielle Gatten kennenzulernen. Sarimé hatte meist nur brav gelächelt und über Nichtigkeiten 
geplaudert. Spätestens wenn sie sich in die Schlange der Mädchen einreihen sollte, die elegant den 
Bogen der Geige schwangen oder deren Finger zart über die Tasten des Klaviers tanzten, war Sarimé 
schlecht genug gelaunt, um jeden Mann schon mit ihrem Blick fortzujagen. \textit{Diese Feiern sind 
immer gleich abgelaufen. Fast schon wie auf dem Viehmarkt.}\\
``Ihr seid ein bezaubernder Anblick, Gräfin'', richtete plötzlich jemand das Wort an sie: ``Es ist 
so erfrischend das Lächeln einer verliebten jungen Braut zu sehen!''\\
Sarimés Lächeln hielt an, als sie sich zu dem Sprecher um wandte. Eine ältere Frau. Ihr Haar war 
streng hochgesteckt und von silbernen Strähnen durchzogen. Falten zeigten sich bereits in ihrem 
Gesicht, aber der Glanz der Jugend war noch deutlich sichtbar. Schwerer Schmuck lag um ihren 
Hals, ihr Kleid war hoch geschlossen, aber aus teurem Stoff. Sarimés Lächeln breitete sich aus, sie 
ergriff den Stoff ihres Hochzeitkleides und machte einen tiefen Knicks. ``Vielen Dank für diese 
herzlichen Worte, Gräfin Rièla.''\\
Die Herrin über die Grafschaft Ringen hatte einen strengen Blick. Kälte funkelte in ihren Augen, 
trotz des Lächelns, das zum Verwechseln echt aussah. Aber Sarimé kannte dieses Lächeln. Ihre 
Stiefmutter hatte diese Kunst selbst zur Perfektion beherrscht. Und Sarimé darin unwillkürlich 
eingeweiht.\\
``Wenn ich Euch irgendeinen Wunsch erfüllen darf, sagt es mir'', forderte das Mädchen ergeben auf.\\
``Oh... wie großzügig. Eine vollendete Gastgeberin, ich sehe schon. Nur bräuchtet Ihr vielleicht 
mehr Übung, aber der Wille ist anscheinend da.'' \\
\textit{Sie spielt auf mein Alter an. Alte Krähe.}\\
Die Dame warf einen Blick durch den Raum voller Gäste. In einer Ecke stand das Streichquartett, der 
Großteil des Platzes wurde von der reichlich gedeckten Festtafel eingenommen. Sarime selbst stand 
in der Nähe der Musik, lauschte deren Klängen und versuchte, einen kühlen Lufthauch von der 
geöffneten Terassentüre zu erhaschen.\\
``Vielleicht hätte man einiges ändern sollen. Dies hier wirkt etwas... nachlässig. Wenig 
durchdacht."\\
Sarimés Lächeln wurde angestrengter. Wieder senkte sie ergeben den Kopf. ``Vielleicht. Ich bin erst 
seit drei Wochen in Merandila und mein werter Gatte wollte mir Zeit zum Ankommen lassen. Deshalb 
plante er die Veranstaltung selbst.''\\
``Man sieht es. Männer haben von so etwas keine Ahnung.'' Sie lachte trocken.\\
Die junge Gräfin zwang sich zu einem Lächeln und senkte wieder ihr Haupt. ``Vielleicht. 
Entschuldigt mich bitte einen Augenblick, ich brauche etwas frische Luft.\\


textbf{Renecs Sicht} \

Er sah nur ihre Umrisse. Das Licht aus dem Festsaal umfloss Sieva wie einen Schein, während sie 
wenige Schritte hinter der Tür auf der Terrasse stand. Sie sah so verloren aus. So ängstlich und 
einsam, er konnte nicht daran vorbei gehen. Nie hatte er den Blick abwenden können, wenn sie traurig 
war. Renec blinzelte gegen den Lichtkontrast an und kam vorsichtig näher. Vermutlich hatte sie ihn 
noch nicht gesehen.  Die Schultern bebten, die Hände krallten sich verzweifelt in den Stoff. \\
``Renec'', sagte Sarimé Sil'Vera mit einer festen Stimme.\\
Er erschrak. Seine Augenlider schlossen sich einen Moment und er ermahnte sich zur Einsicht. 
Zwang sich, zurück in die Realität zu kommen. Sieva war tot. Das Mädchen vor ihm war jemand völlig 
anderes. Ihre Hände griffen nicht nach Halt, sondern waren zur Faust geballt. Ihr Kopf stolz 
erhoben. Die Augen funkelten nicht vor Tränen, sondern Trotz. \\
``Ihr seht bezaubernd aus'', murmelte Renec gedämpft. Was nicht einmal eine Lüge war. So sehr man 
sie auch in ein Korsett zwingen würde, ihre Augen blieben die einer zornigen Raubkatze. Renec 
zögerte. \textit{Sie ist stark. Sie weiß es nur nicht.}\\
Beschämt senkte er den Blick. Stärke... das war es, was er Sieva stets vorgeworfen hatte. Sie war 
so schwach gewesen! Immer den Tränen nahe. Sie wollte immer hören, was sie tun sollte, anstatt auch 
nur einmal das Kinn zu eben und so zu blicken wie das Mädchen, was vor ihm stand. Und das, obwohl 
Sarimé so jung war. Sie hatte keine Ahnung, was dort im Saal vor sich ging. Welche Schlacht gerade 
in diesem Moment geschlagen wird. Wie sollte er es ihr beibringen? ``Geht es Euch gut?'', fragte 
er.\\
Sie trat näher und er konnte ihr Gesicht besser erkennen. Ihr Miene war verkniffen und angespannt.
``Nein.''\\
Die Ehrlichkeit, welche in diesen einem Wort steckte, überraschte Renec. Auch das entging der 
jungen Gräfin nicht. ``Und höre auf, mich so anzuschauen. Ich sagte dir auf dem Ausritt... ich kann 
das alles lernen. Ich werde es lernen!''\\
Er deutete auf die geöffnete Türe. ``Dann solltet Ihr wieder hinein gehen. Bei solchen Festen geht 
es immer um Politik. Verbündete, Verhandlungen, Feinde.''\\
Ihr Blick folgte seinen Fingerzeig und sie zögerte.
\textit{Also doch nicht so mutig.}\\
``Dein Vater scheint alleine gut klar zu kommen.''\\
``Mein Vater ist es seit Jahren gewohnt, dass seine Frau keine Hilfe ist.''\\
Sie schwieg kurz und sprach das aus, was er vermieden hatte zu sagen. ``Aber das will er nicht von 
seiner neuen Frau.''\\
Der Bastard nickte nur.\\
``Ich werde es lernen'', wiederholte sie leise. Ihr Atem stockte. ``Aber nicht jetzt. Ich... ich 
kann da jetzt nicht rein...''\\
\textit{Sie ist einsam und hat Angst.} Renec seufzte leise, obwohl es seinem Vorhaben doch 
eigentlich entgegen kam. \textit{Wie war es wohl Sieva ergangen?} Sarimé hatte furchtbare Angst, 
auch wenn sie es einigermaßen verbergen konnte. Leicht neigte Renec seinen Oberkörper. ``Darf ich um 
einen Tanz bitten?''\\
Ihre Antwort war ein stummes nicken. Vorsichtig ergriff er ihre Hand und legte seine zweite an ihre 
Hüfte. Stumm zählte er den Takt zur Musik, die aus dem Saal erklang und machte den ersten Schritt. 
Selbst im Tanz waren ihre grünen Augen unbeweglich auf sein Gesicht gerichtet. \textit{Sieva 
hätte mich nicht angeschaut. Sie wäre rot geworden und hätte beschämt gelächelt, obwohl wir uns 
nahe standen. Ich muss aufhören sie zu vergleichen!}\\
``Du grübelst auch'', bemerkte Sarimé nach einer Drehung.\\
``Es war ein langer Tag'', wich Renec aus und leitete den nächsten Tanzschritt ein. Er dachte an 
ihr Lachen, dass sie gerne bereit war zu verschenken. Jetzt lachte sie nicht. Die Gräfin Merandilas 
schien gefasst, das Mädchen dahinter eingesperrt hinter Korsett und Hochsteckfrisur. \\
``Du prüfst mich'', sagte sie leise.\\
``Was meint Ihr?'', flüsterte er zurück.\\
``Dieses Gesicht hast du immer, wenn du etwas vergleichst. Zu welchem Schluss bist du gekommen?''\\
Er schwieg lange, legte einen komplizierteren Tanzschritt dazwischen um sich Zeit zu verschaffen. 
Glücklicherweise endete das Lied und die Musiker schienen eine kurze Pause zu machen. Er nahm das 
als erfreulichen Zufall und ließ ihre Hüfte los. Der Griff ihrer blassen Hand verstärkte sich 
abrupt. ``Nein. Du antwortest mir. Jetzt'', beschloss sie: ``Hier ist niemand ehrlich zu mir. Also 
musst du es sein!''\\
Tief holte er Luft. Ihre Worte trafen ihn, erweckten sogleich die Stimmen seines Gewissens. ``Ihr 
könnt Herzen erobern'', sagte er schließlich: ``Wenn Ihr es richtig einsetzt. Eine wichtige Gabe für 
eine Gräfin. Es geht immer um Politik.''
Ihre Augen senkten sich kurz und ihr Griff lockerte sich. Er wollte etwas sagen, als ihm klar 
wurde, wovon sie eigentlich sprach. Sie hatte keine Angst vor den Gästen oder davor im Mittelpunkt 
zu stehen. Seine Stimme wurde sanfter. ``Es wird vorbei gehen. Mach einfach die Augen zu. Evin will 
einen Erben und eine Frau, für die er sich nicht schämen muss. Alles andere ist ihm gleichgültig. 
Mach die Augen zu... und wenn es vorbei ist, verlässt du mit erhobenem Kopf das Zimmer, 
verstanden?'' Ohne es zu merken hatte er sie vertrauter angesprochen, als es ihm zustand. Das 
rothaarige Mädchen blickte dankbar zu ihm hoch und nickte. Ihre Augen glänzten vor Feuchtigkeit, 
aber keine Träne löste sich.\\
``Du bist stark, dass sah ich schon, als du aus der Kutsche gestiegen bist.''\\
``Danke.''\\
Sie wandte sich ruckartig um, raffte den Rock ihres Kleides um sich die einzelne Stufe zurück in 
den Saal zu erleichtern und Renec sah nur noch, wie sie auf einen der Gäste zu trat und mit einem 
aufgesetzten Lächeln ein Gespräch begann. Einen langen Moment blieb er dort stehen und sah sie an. 
Diesmal verglich er sie nicht mit Sieva. \\


\textbf{Sarimés Sicht}

Sie hatte vor ihren frisch angetrauten Gatten das Fest verlassen und war einen Umweg über den Hof 
gegangen. Die Hoffnung, ihn noch einmal zu sehen, zog sie in das Mondlicht hinaus. Schließlich 
führten ihre Schritte sie aber doch zum Unvermeidbaren. Leise öffnete sie die Türe des Gemachs und 
schlüpfte durch einen Spalt in das dunkle Zimmer hinein. Sie wollte gar nichts von ihrer Umgebung 
sehen. \textit{Ich mache die Augen zu. Nichts wahrnehmen. Nichts spüren. Nichts denken. Sich fort 
Träumen. Sich starke Schwingen wachsen lassen und fliegen. Überall hin, nur nicht hier sein in 
diesem Raum. Alleine. Mit meinem Mann. Evin A'Rik}, dachte sie, \textit[{Graf Merandilas. Der 
Wächter der Nordgrenze.}\\
Sie schüttelte kaum merklich den Kopf. Sein Titel war so... elegant. Man hätte sich einen jüngeren, 
netteren Mann darunter vorstellen können. Sie seufzte leise in die Stille hinein. Was klagte sie? 
Es hätte schlimmer kommen können. Sie hatte immerhin Geld. Und Sicherheit. Und eine berechtigte 
Hoffnung, dass Evin sie höchstens mit Nichtachtung anstatt Schläge strafen würde. Sie tastete sich 
zu seinem Bett und setzte sich reglos auf die Kante. \textit{Es gibt keine Liebe. Einfach die Augen 
schließen.}\\
Sarimé konzentrierte sich auf die Schwärze der Nacht. Alle anderen Eindrücke verbannte sie. Auch, 
wie sich die Türe öffnete und ihr Mann herein kam. Auch, wie er sie anfasste und die Geräusche, die 
er von sich gab. Und erst recht seine Bewegungen. \textit{Es gibt keine Liebe.}\\
``Immerhin heulst du nicht wie der Windgeist'', sagte er und rollte sich von ihr herunter.\\
\textit{Es gibt keine Liebe!}\\
Sarimé erwiderte nichts. Sie stand stumm auf, kleidete sich im Dunkeln ein und verließ mit langen 
Schritten den Raum. Zwei Wachsoldaten vor der Tür des Grafen glotzten sie an, aber auch diesen 
schenkte das Mädchen keine Aufmerksamkeit. Die Tränen kamen erst, als sie die Kälte der leeren 
Gänge hinter sich gelassen hatte. Erst, als sie in ihrem Zimmer ankam und sah, wer in einem der 
Sessel eingeschlafen war. \textit{Natürlich er}, dachte sie voller Dankbarkeit.\\
Sarimé konnte nicht sagen, ob es an ihm lag, dass ihr nun die Tränen kamen oder lediglich an der 
Tatsache, dass es doch noch jemanden gab, der wohlwollend an sie dachte. Es hätte vielleicht auch 
die Köchin sein können, die in diesem Sessel saß und auf sie wartete. Die junge Gräfin blieb starr 
stehen und kniff die Augen zusammen. Das gelöste Haar fiel ihr in Locken in das Gesicht, die Hände 
zu Fäusten geballt, stand sie zitternd da und gab keinen Ton von sich. Vielleicht spürte der 
Schlafende ihre Anwesenheit, vielleicht ihren Herzschlag oder ihre leisen Atemzüge, auf jeden Fall 
schlug er seine Augen auf. Auch der Bastard regte sich nicht, sah nur stumm auf das zitternde 
Mädchen. \\
``Soll ich gehen?'', fragte er schließlich.\\
``Warum fragst du das?''\\
Er richtete sich auf und strich seinen Wams glatt. Er schien sich unwohl zu fühlen, das lockte ein 
flackerndes Lächeln auf Sarimés Lippen. ``Bisher schien es, als wäre es Euch lieber, alleine zu 
sein, wenn es euch nicht gut geht.''\\
Ihr Lächeln verblasste und sie hob das Kinn. ``Mir geht es gut. Danke der Sorge.''\\
Seine Stirn legte sich in Falten. ``Das heißt, ich soll gehen?''\\
Ihre Schultern sanken herab. ``Nein'', murmelte sie und senkte den Blick.\\
\textit{Wieso fragt er nicht? Wieso sagt er nichts? Wieso tut er nichts?!} Sie seufzte leise und 
musterte ihn direkt. Der Bastard wich ihrem Blick aus, blieb aber im Sessel sitzen. \textit{Er 
weiß, dass er nicht hier sein sollte. Im Gemach der Frau seines Vaters. Im Gemach der Gräfin.}\\
``Wie ist es, zu lieben?''\\
Seine Hand zuckten nervös. ``Was meint Ihr?''\\
Sarimé trat näher an ihn heran, drückte ihre Hand auf seinen Brustkorb und schubste ihn sanft 
zurück gegen die Lehne. Ihr Gesicht war so dicht vor dem Seinen, dass sie seinen warmen Atem spüren 
konnte. ``Wie. Ist. Es'', sagte sie langsam: ``Zu lieben?''\\
Eine Antwort blieb aus. Renec sah sie nur an. Also tat Sarimé es. Tat, was sie so oft in blöden 
Romanen gelesen hatte. Sie ließ die letzte Distanz zwischen ihnen hinter sich und drückte ihre 
Lippen auf seine. Prüfend blickte sie in seine Augen. Er wehrte sich nicht. Aber er war auch alles 
andere als entspannt. Sarimé richtete sich wieder auf und trat ein paar Schritte in die Richtung 
ihres Bettes. Sie machte eine Handbewegung zur Tür. \\
``Warum hast du gewartet?''\\
Er erhob sich schnell, als fürchtete er einen weiteren Überfall. ``Ich wollte nur sicher gehen, 
dass Ihr...'' \\
``Das ich was? Nicht springe?'' Sie lachte. ``Keine Sorge. Ich werde erst aus einem Fenster 
steigen, wenn ich mir sicher bin, dass mir Flügel gewachsen sind. Flügel, die mich auch tragen 
können. Vermutlich würde ich dich vorher nach deinen Rat fragen. Also sei dir sicher, Renec, du 
wirst es erfahren.'' Sie zwinkerte ihm zu und musste grinsen, als sie sah, wie unwohl er sich 
fühlte. Sanfter fügte sie hinzu: ``Nein. Ich bin dir wirklich dankbar.''\\
Er zögerte, bevor sein übliches schiefes Lächeln wieder erschien. ``Ihr meint, für den Kuss?''\\
``Das hätte auch eine Strohpuppe gekonnt. Du solltest dich mal rasieren. Aber... nun ja... ich 
wollte einfach wissen, wie es ist.''\\
``Was Ihr wollt, das bekommt Ihr auch, hm?''\\
``Was ich will, das hole ich mir. Das ist ein Unterschied. Aber ich bin bescheiden. Ich will nicht 
viel.''\\
``Nur wissen, wie sich Liebe anfühlt?''\\
Sie zuckte unbeteiligt mit den Schultern. ``Ich überlege noch, ob es so etwas überhaupt gibt. 
Meistens bin ich mir sicher, dass das nur eine Lüge ist. Ein blödes Wort. Und manchmal denke ich 
mir, vielleicht lohnt es sich ja doch, es herauszufinden. Aber nur selten. Sehr selten."\\
\textit{Und dann sehe ich dein Gesicht, wenn du an Sieva denkst. Und scheint es plötzlich wahr 
zu sein. Eine Wahrheit, die mir verborgen bleiben wird.}
