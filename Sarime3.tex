
\chapter{Die Pflicht}

\textbf{Zeitsprung von drei Wochen. Evin war in Na'Rash beim Priester, Kriegserklärung an Kasir 
wurde verkündet}\\

Schon wenige Tage nach ihrer Eheschließung musste Evin in die Hauptstadt der Grafschaft reisen. 
Sarimé hatte von einer Nachricht eines Priesters an Evin erfahren und darauf bestanden, mit ihm zu 
reisen. Aber ihr Gatte hatte abgelehnt und sie dazu verdonnert, in der Festung alleine zu bleiben. 
Sarimé bereute es kaum. Ohne Evin war es ruhiger in der Burg. Die Diener entspannter, die Regeln 
unwichtiger. Die junge Gräfin nutzte die Zeit, sich weiter mit den Bediensteten bekannt zu machen. 
Sie ließ sich die Buchhaltung der letzten drei Jahre zeigen und rechnete alles nach. Es machte ihr 
Spaß mit den Zahlen umzugehen. Es erinnerte sie an die unbeschwerten Tage, als ihr Vater ihr das 
Rechnen beibrachte. Da ihr die Ausgaben unnötig hoch erschienen, koordinierte sie Einsparungen. 
Letzten Endes wäre es egal, meinten der Buchhalter und die Köchin, denn die Ausgaben lagen noch 
deutlich im Budget, welches jedes Quartal vom König gestellt wurde. Aber Sarimé war nie gerne 
abhängig und notfalls würden die kleinen Einsparungen wichtigere Dinge finanzieren lassen. \\
Sie ritt öfters mit ihrer Stute hinunter nach Talsmund, beobachtete die Handwerker und 
Bauern und wechselte ein paar höfliche Worte mit ihnen. Es war... schön. Sarimé konnte nicht 
bestreiten, dass ihr die faszinierten Blicke gut taten. Es gab ihr das Gefühl, etwas richtig zu 
machen. Auch mit Renec vertrieb sie ihre Zeit. Sie ließ sich von ihm die Bibliothek zeigen, 
verlangte, dass er ihr bei Inventur über die Bücher half und mittlerweile schickte sie ihn immer 
wieder fort, um in der umliegenden Gegend nach weiteren Büchern zu suchen. Und oft genug redeten sie 
auch nur. Ihr war gar nicht bewusst, wie viel sie ihm sagte. Sie erzählte ihm von all ihren Ideen. 
Von den Einsparungen bis hin zu der Überlegung, für die Bewohner Talsmund eine kleine Ansammlung 
nützlicher Sachbücher bereitzustellen. Prinzipiell gab es eine kostenlose Schule in Saleica, in der 
jeder Mensch unterrichtet werden konnte. Aber dieses Angebot befand sich meist in den größeren 
Städten und dort eine Unterkunft zu finden konnten sich die Dorfbewohner selten leisten. Trotzdem 
konnten überraschend viele Lesen. Sarimé fragte nach und erfuhr, dass der Holzfäller und 
Geschichtenerzähler Arham diese Fertigkeit an Interessierte weitergab. Als Anerkennung gab sie in 
der Schmiede auf der Burg eine neue Klinge für sein Beil in Auftrag. Die Tage vergingen und der 
Herbst zog auf. Sogar ein Brief aus Evins Feder fand sich ein. Sarimé überraschte das sehr, damit 
hatte sie nicht gerechnet. Aber es waren lediglich sachliche Informationen und Anweisungen. 
\textit{Krieg.}\\
Sie hielt das Pergament noch in ihren Händen und überflog den Absatz zum dritten Mal. Evin beschrieb 
über die Verkündung der Kriegserklärung gegen Kasir, ihrem Nachbarland im Norden. Es grenzte direkt 
an Merandila. Sarimé starrte lange auf die verschlungene Schrift und überlegte, was dies bedeuten 
würde. \textit{Ich werde mit Evin reden müssen. Ich muss erfahren, was auf uns zu kommt.}\\
Die Nachricht zog schnell ihre Bahnen. Immer öfter sah man trainierende Soldaten aus dem nahen 
stehenden Lager. Renec bat Sarimé, nicht mehr alleine auszureiten. Darüber konnte sie jedoch nur den 
Kopf schütteln. Sie waren drei Tagesritte von der Grenze entfernt. So schnell würden keine Kasira 
nach Merandila kommen. Trotzdem, die Unsicherheit blieb, bis Renec ihr eines Abends beim Essen – 
welches Sarimé mittlerweile gerne in Gesellschaft vom Bastard einnahm – versicherte: ``Der Herbst 
kommt. Es hört kaum auf zu regnen und da oben in Kasir wird der Winter noch schneller kommen. Vor 
dem ersten Frühlingstag wird es keine Schlacht geben.''\\
Sie nickte nur und als Antwort auf seine Vermutung klammerte sich an diesen Gedanken. Und, sie 
konnte es kaum selbst glauben, sie hoffte, Evin würde bald wieder zurück kommen. Ihr entging nicht, 
wie Leute sie beobachten, auf eine Reaktion warteten. Sarimé bemühte sich so weiter zu machen wie 
vorher. \\

Drei Tage hatte es ununterbrochen geregnet. Die junge Gräfin blieb meist innerhalb der Burg, 
verbrachte die Abende vor dem Kamin mit einem Buch. Aber mit jeden Abend wurde sie nervöser. Evin 
hatte mitgeteilt, dass er bald wieder hier sein würde. Er hätte schon längst aus Na'Rash 
zurückgekehrt sein müssen. Nun war sie schon fast über ihrem Buch eingeschlafen, als eine Magd 
heftig gegen ihre Zimmertüre klopfte und ohne auf eine Antwort zu warten eintrat. \\
``Herrin'', rief sie atemlos: ``Man hat Lichter gesehen. Eine Gruppe Reiter kommt.''\\
Sarimé erhob sich in einer fließenden Bewegung und schlug das Buch zu. ``Ich komme. Sorge dafür, 
dass 
ein heißes Bad für den Grafen vorbereitet wird. Und schau nach, ob die Köchin noch etwas von den 
Resten des heutigen Abends aufwärmen kann.''\\
Die junge Gräfin schlüpfte in einen eher praktischen als modischen Pelzmantel und schritt eilig 
durch die Gänge der Burg. Die Türen des Portals waren bereits geöffnet, sie war nicht die Einzige, 
die die Heimkehrer erwartete. Sie trat die Stufen hinunter in den Regen und sah, wie eine Gruppe 
Reiter durch das Tor trabten. Stallburschen eilten herbei um den Männern die Zügel abzunehmen. 
Sarimé kniff die Augen zusammen und starrte auf das stämmige Schlachtross. So stolz und stark es 
auch da stand, Evin fiel regelrecht aus dem Sattel. Vielleicht lag es an der Dunkelheit der 
Sturmnacht, aber der Graf war nur ein Schatten seiner selbst.\\
``Helft ihm!'', befahl Sarimé zwei Wachsoldaten und deutete auf ihren Gatten. Sie trat einige 
Schritte 
näher, beobachtete, wie die Männer ihren Mann unter die Arme griffen und aufrichteten. ``Ihr 
hättet eine Kutsche nehmen sollen'', rief sie.\\
``Die ist was für Weiber'', zischte er und lachte glucksend. Schnell wurde das Geräusch zu einem 
rauen 
Husten.\\
``Eine Erkältung auch?'', murmelte Sarimé und schüttelte den Kopf.\\
Der Graf wurde in die Burg gebracht. Sie klatschte in die Hände und rief den Burschen zu: ``Beeilt 
euch, kommt ins trockene. Wärmt euch nach der Arbeit in der Küche auf, dort brennt der Ofen 
immer.''\\

Am nächsten Morgen kam Evin nicht zum Frühstück. Sarimé aß nicht viel, sondern beeilte sich ihn zu 
suchen. So viele Fragen lagen ihr auf der Zunge. Nach Na'Rash, der Kriegserklärung, dem Brief des 
Königs und den Priestern Oysams im dortigen Tempel. Was er nun zu tun gedenke, wie es ablaufen 
würde, ob sie noch mehr Soldaten brauchten, ob es vielleicht noch einen Weg zum Verhandeln mit 
Kasir 
gab. Auf dem Flur begegnete sie seinem Kammerdiener. Ein hagerer, gebückter Mann mit ergrautem 
Haar. Sarimé schätzte ihn auf das selbe Alter wie ihren Gatten. ``Morris, schläft der Graf noch?''\\
Er verneigte sich schwerfällig und nickte. ``Ja, meine Herrin. Die Reise war sehr anstrengend für 
ihn.''\\
Sarimé runzelte die Stirn. ``Ich werde nach ihm sehen. Seid Ihr auf dem Weg zur Küche?''\\
``Ich wollte Frühstücken, Herrin. Kann ich Euch dienen?''\\
``Lass eine Küchenhilfe Suppe zum Grafen bringen, das wäre alles, danke Morris.''\\
Evins Zimmer war abgedunkelt und stickig. Sarimé schob die Vorhänge zur Seite und öffnete die 
Fensterläden. Die frische Morgenluft, nach Regen duftend, strömte in den Raum. Durch die Wolken 
brachen einzelne Sonnenstrahlen. Sarimé wandte sich ihrem Gatten zu. Er schlief keineswegs. Aber 
richtig wach wirkte er auch nicht. Evin lag begraben unter drei Decken. Schweißperlen glitzerten 
auf 
seiner wettergegerbten Stirn. Sein glasiger Blick hing weit in der Ferne. Schnell war sie neben 
ihm. ``Evin?''\\
Seine Haut fühlte sich an wie glühende Kohle, als sie ihre Hand auf seine Stirn legte. Erschrocken 
zuckte sie zurück. Sie kannte sich mit Krankheiten oder besser gesagt deren Heilung nicht aus. Sie 
selbst war bisher noch relativ verschont geblieben. Einer ihrer Neffen war ein kränkliches Kind und 
hatte oft mit Fieber mit Bett gelegen. Nach einigen Tagen war bei dem Jungen alles wieder 
überstanden gewesen und er rannte wieder über den Hof. Sarimé zweifelte daran, dass Evin ebenso 
schnell genesen würde. Es klopfte an der Türe. Die junge Gräfin nahm es kaum war. Reglos saß sie 
auf der Bettkante und starrte das fahle Gesicht und die glasigen Augen ihres Gatten an.\\
``Herrin?''\\
Es war die Magd, die sie am gestrigen Tag in die Burg geleitet hatte.\\
``Bring mir die Suppe… und gibt es einen Arzt oder Heilkundigen in der Burg?''\\
Sie trat zögernd näher und stellte das Tablett auf den Beistelltisch. ``Verzeiht… der nächste Arzt 
befindet sich in einem Dorf, eine Stunde entfernt. Ich kann einen Boten aus schicken.''\\
``Mach das.''\\
``Kann ich noch für Euch dienen?''\\
Sarimé zögerte. ``Hat Evin gestern gebadet?''\\
``Ich weiß es nicht, Herrin. Ich richtete das Bad… der Kammerdiener des Herrn wird Eure Frage 
bestimmt beantworten können.''\\
Sarimé nickte. ``Danke… du darfst gehen.''\\
Die Magd verließ den Raum, leise fiel die Tür ins Schloss. Sarimé riss ihren Blick von Evin, stand 
auf und schloss die Fenster. Sie fröstelte und kniete sich vor den Kamin. Sie befreite die 
glimmenden Kohlestücken von der dunklen Asche und entfachte das Feuer neu. \\
Mit mühevollen Überzeugungsversuchen gelang es ihr, Evin ein Stück in die Gegenwart zu holen. Er 
richtete sich auf und lehnte sich schwer gegen ein Kissen, während sie ihm die Suppe einflößte. 
Zwischendurch kam der Kammerdiener, erzählte, dass er einen Reiter zum Arzt geschickt habe und 
verließ den Raum sehr schnell. Generell, bemerkte sie, schienen die Dienstleute es eilig zu haben, 
den Raum zu verlassen. \\
``Sie fürchten sich, dass Ihr sie ansteckt'', sagte Sarimé zum schlafenden Grafen. Er schlief viel 
in 
den Tagen der Krankheit. Und Sarimé sprach viel. \\
Der Arzt kam und ging wieder, nachdem er keine andere Behandlungsmöglichkeit als Schlaf und warme 
Suppe sah. Einen Moment lang wollte die junge Gräfin aufbegehren, als der Arzt einen Lohn 
verlangte, 
doch dann gab sie sich geschlagen und trug dem Kammerdiener auf, die gewünschte Anzahl an Münzen 
für 
ihn zu holen. \\

Sarimé war müde. Sie schlief im Sessel neben Evins Bett, schreckte bei jedem stockenden Atemzug 
oder 
Hustenanfall auf. Immer wieder musste sie nach ihm sehen. Das Fieber sank nicht, egal wie viele 
feuchte Tücher sie auf seine Stirn legte oder sie das Zimmer lüftete. Die Untätigkeit war das 
Schlimmste. Sie wollte das Zimmer nicht verlassen. Er war ihr Gatte, auch wenn sie ihn nicht 
liebte. Er hatte ihr eine Heimat geboten, auch wenn nur aus Eigennutz. Sie begann zu sticken und zu 
lesen, zwischendurch fütterte sie ihn oder las ihm Briefe vor. Briefe über die 
Kriegsvorbereitungen. Briefe über wirtschaftliche Belange. Besserungswünsche vom Adel. Innerhalb 
dieser zwei Woche verließ sie nur zwei Mal den Raum um sich Bücher zu holen und frisch 
anzukleiden.\\
Die Tage waren eintönig und Evins Zustand schwankte bedenklich. Nach wenigen Tagen seiner 
Erkrankung 
sah es so aus, als würde er wieder genesen. Er stand sogar am Fenster und sah auf das Treiben im 
Hof. Doch dann ging es rapide abwärts. Der Graf öffnete kaum noch die Augen. Er aß nichts und 
Sarimé 
verlor ebenso den Appetit. Die junge Gräfin steckte gerade zum wiederholten Mal die Nadel in das 
Tuch und formte ein Muster, als die Magd sie ansprach. Sie räumte gerade das Geschirr fort. \\
``Herrin…'', sagte sie vorsichtig.\\
``Hm?'', entgegnete Sarimé, warf einen schnellen Blick auf den schlafenden Grafen und blickte dann 
wieder zu ihrer Stickarbeit.\\
``Verzeiht wenn ich Euch zu nahe trete… aber ich habe mit eurer Zofe gesprochen und… es sind 
mehrere 
denen es aufgefallen ist…''\\
``Wovon sprichst du?'', fragte Sarimé ungeduldig.\\
``Ihr hab schon lange nicht mehr…'' Sie sah zum Grafen und flüsterte: ``geblutet....''\\
``Du meinst…?'' Sarimé verstummte.\\
Die Magd hatte recht. Sie hatte schon seit etlichen Wochen ihre Regel nicht bekommen. Sie stand 
auf. 
``Sorge dafür, dass der Graf nicht alleine ist. Ich gehe spazieren. Ich war viel zu lange nicht 
mehr 
draußen.''\\

Die Burg besaß einen Garten, der zum größten Teil von Unkraut überwuchert war. Bisher hatte Sarimé 
ihm keine Beachtung geschenkt, doch jetzt trugen ihre Füße sie an diesen Ort. Sogar ein steinerner 
Brunnen stand in der Mitte der kleinen Anlage. Die Pumpe funktionierte jedoch nicht und selbst das 
Regenwasser, welches das Becken aufgefangen hatte, war mit Pflanzen überwuchert.Die wilden Blüten 
und das Gras waren ausgetrocknet und verblüht. Toter Efeu klammerte sich an das Gestein einer 
Statue, als hätte es den Kampf ums Überleben nicht bereits verloren. Alles hier in diesem Garten 
zeugte davon, dass der Herbst mit großen Schritten näher kam. Sarimé trat zu einem älteren Mann, 
der auf einer Bank am Brunnen saß. Er trug eine ausgeblichene Mütze und sein Blick schweifte 
unbeirrt über den Garten. Als würde er hier noch die alten Zeiten sehen, in denen der Garten noch 
gepflegt wurde.\\
``Ihr seid der Gärtner, oder?''\\
Er sah auf, starrte sie einen Moment an und neigte dann langsam den Oberkörper. ``Meine Herrin.''\\
``Warum tust du nicht die Arbeit, für die du bezahlt wirst?''\\
``Tue ich. Ich hege den Gemüsegarten.''\\
Sarimé deutete auf den Ziergarten. ``Und das hier?''\\
Als Antwort zuckte der Gärtner nur mit den Schultern. ``Der Herr hatte nie großes Interesse an dem 
Garten. Seit dem Tod seiner ersten Frau liegt er brach. Man trug mir auf, etwas sinnvolleres zu 
tun.''\\
Die junge Gräfin biss sich nachdenklich auf die Zunge, dann setzte sie sich neben den Mann und 
teilte sein Schweigen. "Wie sah er früher aus?", fragte sie dann leise.\\
Der Gärtner legte den Kopf schief und lächelte verträumt. "Genauso klein. Aber die Herrin wählte so 
viele verschiedene Blumen aus... die Farben kann ich gar nicht beschreiben. Sie war oft hier. Jede 
freie Minute. Und die Kinder auch. Jedes ist mindestens einmal in den Brunnen geklettert, wenn ihre 
Mutter nicht hinsah."\\
Die junge Gräfin lauschte seinen Worten nachdenklich und versuchte dieses Stück verwilderte Erde 
mit 
seinen Augen zu sehen. \textit{Kinder... Mein Kind könnte hier spielen.}\\
Sie legte ihre Hand auf ihren Bauch und horchte in sich. Sie suchte nach dem neuen Leben in 
sich. \textit{Mein Kind wird hier spielen.}\\
Es war wie ein feierlicher Beschluss und ein Schwur, dass sie endlich dieses Land als ihre Heimat 
akzeptieren würde. Und als die Heimat ihrer Kinder.\\
``Bereite den Garten für den Winter vor. Schneide die Hecken zurück. Und im Frühling besprechen 
wir, 
welche Pflanzen im Garten angepflanzt werden sollen.''\\
``Wie Ihr wünscht, Herrin!'', sagte er und ein Lächeln huschte über seine schmalen Lippen.\\
Die Gräfin verabschiedete sich höflich und als sie die Stufen zum Eingangsportal hinaufstieg, 
lächelte sie. Ihre Hand ruhte wieder beiläufig auf ihrem Bauch. \\

Auf ihrem Rückweg eilte sie durch die Gänge. Sie hasste Evin dafür, wie er sie in der 
Hochzeitsnacht und die Male danach genommen hatte. Sie hasste ihn dafür, dass er sie gekauft hatte 
und sie wegen ihm in die Fremde musste. Aber sie wusste auch, wie sehr er sich einen Erben wünschte 
und sie konnte ihn nicht genug hassen, um ihm stillschweigend beim Sterben zuzusehen. Es war ihre 
Pflicht und sein Recht zu erfahren, dass sein Wunsch in Erfüllung gehen könnte, falls das Kind 
überleben würde. Sarimé kniete vor dem Bett, hielt Evins Hand. Sie sprach mit ihm, wollte, dass er 
einige Minuten wach blieb, bei ihr war. Als sie in das Zimmer trat, sah sie den Arzt an Evins Bett 
stehen. Der Arzt war erneut gerufen worden. Ungeduldig, sie wollte mit Evin alleine sein, fragte 
sie ihm nach dem Befinden des Grafen. \textit{Wie oft habe ich diese Frage in den letzten Wochen 
gestellt? Wie oft hat er geantwortet, dass wir beten müssen?}\\
Dies Mal sagte er nichts der gleichen. Stattdessen wagte er es kaum ihr in die Augen zu blicken und 
erwiderte: ``Der Graf ist schon bei Osyma. Bald wird ihm der letzte Lebensfunke folgen.''\\
Als er dieses Mal ging, erhielt er keinen Lohn.\\
``Ich entlohne keine Männer, deren Dienst mir nicht hilft'', waren ihre Abschiedsworte und mit 
energischem Blick schickte sie ihn fort. Danach kniete sie sich neben das Bett und ergriff seine 
Hand. In den Tagen der Krankheit schien er um Jahre gealtert zu sein. \\
"Du bist wirklich so anders als der Windgeist", keuchte er und sein Mund verzog sich zu einem 
Grinsen.\\
"Wir hätten uns früher kennenlernen müssen. Wer weiß, was dann heute wäre."\\
"Früher? Mein Graf, ich bin sechzehn. Ich glaube nicht, dass Ihr ein Kind als Braut hättet 
gebrauchen können."\\
Er lachte, bis das Geräusch in rasselnden Atem überging. Als sein Brustkorb wieder regelmäßiger hob 
und sank sprach er: "Als ich Sieva das erste Mal sah... ach.. sie sah aus wie der stolze Löwe auf 
Saleicas Wappen. Deshalb nahm ich sie mit, weißt du. Ich Narr. Ich verwechselte Stolz mit 
Unnahbarkeit. Erhabenheit mit Abwesenheit. Anfangs... ich habe mich wirklich bemüht. Ich 
vernachlässigte meine Pflichten, um ihre Wünsche zu erfüllen. Um sie zum lächeln zu bringen. So 
viel vergebene Mühe! Und was hatte ich dann davon? Sie zwang mich, Renec an den Hof zu holen. Ich 
tat alles, was sie sich wünschte, weißt du... Aber mein Wunsch wurde nicht erfüllt. Stattdessen 
vögelte sie meinen Bastard! Immerhin hat er es auch nicht geschafft, sie zu schwängern."\\
Seine Rede hatte mit Worten voller Reue begonnen und endete mit Verbitterung.\\
``Mein Graf'', sagte sie eindringlich und drückte fest seine hagere Hand: ``Osyma ist gnädig mit 
Euch. 
Ich bin schwanger.''\\
In seinem Gesicht zuckte etwas. Vielleicht ein flüchtiges Lächeln? Er schloss die Augen und seine 
Hand wurde schlaff. "Erst so kurz. So viele Leben sterben bereits, bevor sie geboren sind. Ich 
werde nicht da sein um aufzupassen, dass es wirklich geboren wird."\\
Sarimé schluckte und suchte fieberhaft nach Worten. "Ihr könnt Osyma direkt sagen, dass er 
gefälligst dafür zu sorgen hat."\\
Seine Hand hob sich und strich ungeschickt über ihr Haar. "Du willst nicht hier sein. Uns verbindet 
nichts. Wer weiß, wie es nun wäre, wenn wir uns früher begegnet wären", murmelte er: "Du bist zu 
jung zum herrschen. Zu jung um alleine Mutter zu sein."\\
"Unterschätzt mich nicht."\\
"Schwöre mir, dass du alles tun wirst, um das Kind zur Welt zu bringen. Schwöre mir, dass du dafür 
sorgen wirst, dass es lebt und überlebt. Dass mein Blut nicht ausstirbt."\\
"Ich schwöre es bei Osymas Flammen und meinem Leben."\\
Es war die letzte Nacht, die Sarimé an seiner Seite verbrachte. Am nächsten Vormittag verstarb Graf 
Evin A’Rick, Wächter der Nordgrenze und letzter Lebender Nachfahre einer Familie, die im Aufbau des 
gesamten Reiches eine große Rolle gespielt hatte.\\


Sarimé fand sich als alleinige Herrin der Burg, der ganzen Grafschaft wieder. Zuallererst schickte 
sie den Kammerdiener los, einen Priester zu holen, der sich um Evin kümmern sollte. Anschließend 
zog sie sich in ihr Gemach zurück, um Briefe an die umliegenden Grafen und den hohe Adelige zu 
verfassen. Mägde wies sie an, dass Zimmer zu reinigen, denn während der gesamten Krankheitsdauer 
über war dort, bis auf Sarimés Lüften und Auskehren der Asche im Kamin, nicht geputzt worden. Als 
Witwe überwachte sie selbst, dass Evins Körper mit dem angemessenen Respekt behandelt und die Riten 
eingehalten wurden. Am Abend schlich sie in ihre Zimmer und fand keine Ruhe. Unschlüssig lief sie 
immer wieder durch den Raum und grübelte über ihre Zukunft nach. Angst überkam sie.
\textit{Ich bin dem allen nicht gewachsen. Viel zu jung und unerfahren! Und nun auch noch das 
Kind.. Wer würde mich jetzt noch heiraten wollen, selbst wenn er Graf wird? Wer würde jetzt 
freiwillig Graf Merandilas werden wollen, wenn der Krieg hier toben wird?}\\
Den Tränen nahe kauerte sie auf dem Bett. Die roten Haare verbargen ihr Gesicht, wie ein 
geschlossener Vorhang eine Theaterbühne. Sarimé nahm das Klopfen nicht wahr, aber auch ohne ihre 
Aufforderung betrat der Bastard ihr Zimmer. Vorsichtig öffnete er die Türe und steckte den Kopf 
durch den Spalt. Als Renec die junge Gräfin entdeckte, war er schnell an ihrer Seite und nahm sie 
in den Arm. Die Berührung tat gut, aber sie wich ihr aus. Es ziemte sich nicht für eine Witwe und 
erst recht nicht für eine werdende Mutter, einen anderen Mann zu umarmen. Renec machte keine 
Anstalten die Geste zu wiederholen, sondern setzte sich aufrecht hin und schwieg, bis sie das Wort 
an ihn richtete.\\
``Was passiert nun?'', fragte sie unsicher.\\
Renec erklärte: ``Die Priester werden den Grafen für die Reise zum Meer bereit machen. Er wird dem 
Wind und dem Meer übergeben. Seine letzte Reise, an der Küste nimmt er von unserer Welt 
abschied.''\\
Sarimé sah auf. ``Wird er nicht verbrannt? In Brom-Dalar werden die Toten verbrannt... In Osymas 
Flammen.''\\
Renec zuckte mit den Schultern. ``Ihr wisst doch, Merandila hat sich viele alte Bräuche behalten. 
Osyma ist der Gott des Feuers und der Ehre. Bevor Merandila eine Grafschaft wurde, verehrten die 
Menschen die Helle.''\\
``Die Helle'', murmelte Sarimé. Sie hatte schon einige Leute diesen Namen aussprechen hören, sich 
aber 
nie weiter Gedanken darüber gemacht.\\
Renec nickte. ``Die Helle, sagte man, hat viele Gestalten. Eine weißhaarige Frau, ein weißer Wolf, 
Fuchs, Kaninchen oder Vogel. Ein klarer Bach, ein Schneeglöckchen, weiße Kirschblüten. Die Gischt 
des Meeres. Sie kann auch ein Feuer sein, aber eher eine kleine Kerzenflamme als das gleißende 
Inferno Osymas. Sie ist Wind und Meer, hell und zart. Das Licht.''\\
``Eine schwache Göttin'', bemerkte Sarimé, als sie die beiden Götter verglich. \\
Renec kniff die Augen zusammen. ``Auf den ersten Blick vielleicht. Nun, Ihr könnt Evin natürlich 
auch 
in Osymas Flammen geben. Aber das wäre ein weiterer Beweis für das Volk, dass ihr keine Merandil 
seid.''\\
``Es gibt das Land Merandila nicht mehr'', sagte Sarimé.\\
``In den Herzen der Leute wird es immer ihr Land bleiben. Da kann der König so viele Priester und 
Krieger schicken, wie er will.''\\
``Das Volk verehrt Osyma. Sie beten in seinen Tempeln, sie melden sich bei der Armee, ziehen für 
ihn 
in den Krieg, ehren seine Lehren und lieben den König.''\\
``Die Städter, ja, aber schaut Euch die Dörfer an, Herrin. Reitet durch die Landschaft, unterhaltet 
Euch mit den Bauern und Handwerkern. Ich will nicht behaupten, dass sie Osyma nicht ehren, aber 
trotzdem... mit jedem Rind, das sie dem Feuergott opfern, halten sie einen Moment vor einer weißen 
Blume oder einer Kerzenflamme inne. Es steckt zu Tief in ihrem Blut, sie können nicht ändern, was 
sie sind. Sie haben die Helle nicht vergessen.''\\
``Ein Gott oder der Andere, wer schert sich darum?'', schnaufte Sarimé und erhob sich: ``Das 
Einzige 
was die Leute wollen ist ein Heim, genügend zu Essen, Sicherheit und Ehre. Osyma ist nur ihre 
Rechtfertigung, sich das alles von anderen zu nehmen.''\\
``Einen Unterschied gibt es'', erwiderte Renec und lächelte: ``Die Helle existiert.''\\
Sarimé sah ihn einen langen Moment stumm an. Dann schüttelte sie den Kopf. ``Alles existiert, wenn 
man nur fest genug daran glaubt. Fragt die Priester Osymas, ob ihr Allmächtiger existiert.''\\
Ihre Zurechtweisung traf ihn sichtlich. Renec richtete sich gerade auf und räusperte sich. ``Was 
gedenkt Ihr zu tun, Herrin?'', fragte er trocken.\\
``Zur Küste reisen und Evin deiner Hellen überbringen. Ich bin keine Merandil, du hast Recht. Und 
ich 
werde nie eine sein. Ich werde bald einen Anderen heiraten, der die Grafschaft führen wird. Er kann 
sich anschließend mit dem Zorn der Priester oder dem Volk auseinandersetzten.''\\
``Und Ihr?''\\
Sarimé legte die Hand auf ihren Bauch. ``Ich will nur ein Heim, genügend zu essen und 
Sicherheit.''\\
Renec folgte ihrer Hand und seine grauen Augen weiteten sich. ``Ihr tragt seinen Erben in Euch?''\\
Sarimé nickte und eine mühsam unterdrückte Träne rann über ihre Wange. Renec sprang auf. ``Warum 
sagtet Ihr nichts? Das ändert alles!''\\
``Nämlich? Dass ich doch nicht ganz unnütz bin, sondern immerhin Kinder gebären kann?''\\
\textit{Falls es überhaupt leben wird...}\\
``Nein! Versteht Ihr nicht? Durch das Kind seid Ihr ein Teil Merandilas. Kein Merandil wird sich 
gegen Euch erheben, wenn dieses Kind leben wird. Durch dieses Kind seid Ihr nicht gezwungen, einen 
Mann, den der König Euch aussucht, zu ehelichen, Herrin. Ihr wärt die alleinige Herrin der 
Grafschaft. A'Rik war ein Mann des Königs. Aber in seinen Adern floss das Blut der letzten freien 
Königin Merandilas. Daher hat sich nie jemand gegen ihn erhoben. Er war eine der letzten, wenigen 
Verbindungen zur Vergangenheit. Zur Freiheit. Auch wenn er selbst nicht mehr als ein Sklave 
Saleicas war. Wie... wie das Gemälde einer Landschaft, die längst zerstört wurde.''\\
Sarimé lauschte seinen Worten und beobachtete, wie Renec aufgeregt durch den Raum schritt. 
\textit{So... triumphierend habe ich ihn noch nie erlebt.}\\
Als er endete schüttelte sie den Kopf. ``Das schaffe ich nicht.''\\
``Ich helfe Euch! Ihr werdet es schaffen. Ihr seid schön, beliebt und klug. Zeigt Euch dem Volk. Es 
wird Euch so sehr lieben wie die Bewohner der Burg und Talsmund. Deren Herz habt Ihr erobert, den 
Rest Merandils werdet Ihr auch erlangen.''\\
``Habe ich das?'', fragte sie zweifelnd.\\
``Hört Euch doch um! Schon als ich vorhin mein Pferd in den Stall brachte, erzählten die 
Stallburschen, dass Ihr Evin nicht von der Seite gewichen seid! Wie aufopferungsvoll Ihr Euch um 
ihn gekümmert habt, obwohl die Liebe zwischen euch als Ehepaar nicht sehr offensichtlich gewesen 
war, wenn Ihr versteht, was ich meine. Sie bewundern Euer Pflichtgefühl und dass Ihr Dinge 
erledigt, ohne zu klagen.''\\
Sarimé holte tief Luft. ``Er war mein Mann. Vor Osyma und Zeugen angetraut. Meine Vorstellung von 
Liebe hat sich nicht geändert. Es war... niemand verdient es, einsam zu sterben. Deshalb habe ich 
es ihm auch gesagt. Er ist mit der Gewissheit gestorben, dass er einen rechtmäßigen Erben haben 
wird. Pflichtgefühl... Es gibt eben Dinge, die man tun muss. Hätten seine anderen Frauen das nicht 
für Evin getan?''\\
``Seine Erste vermutlich schon. Das soll wohl wirklich eine Heirat aus Liebe gewesen sein.''\\
``Und Sieva?''\\
Renecs Stirn legte sich in Falten und der Name der verstorbenen Gräfin schien ihn zu verärgern. 
Sarimé konnte in diesem Moment darüber nur den Kopf schütteln. Sein Vater war eben erst gestorben 
und Renec zeigte keinerlei Trauer. Aber auch sein Verhalten auf die Reaktion ihres Namens hatte 
sich geändert.\textit{Was es wohl bedeutet, dass er wütend statt traurig wird, wenn man den Namen 
seiner Geliebten ausspricht?}\\
``Nein, auch Sieva nicht. Sie war, im Gegensatz zu Euch, nicht für das Herrschen bestimmt.''\\
\textit{Herrschen?}, wiederholte Sarimé gedanklich: \textit{Ich herrsche nicht. Ich verwalte das 
Land des Königs. Wenn überhaupt.}\\
Sarimé schüttelte weiter den Kopf und trat auf das Fenster zu. Die Blätter der Bäume glichen Osymas 
Flammen, die sich zornig und aufgebracht gen Himmel reckten. \textit{Weiß er, wovon Renec spricht? 
Sagt der rachsüchtige Gott es seinen Priestern, seinem König? Erzählt er ihnen von den Worten des 
Verrats, die in diesem Raum, in diesem Moment ausgesprochen wurden?}\\

\textit{Es ist vorbei}, dachte Sarimé und es war, als floss alle Kraft aus ihr heraus. Die Reise 
zum Meer war endlich überstanden. Scharen von Menschen waren ihnen für einzelne Wegstunden gefolgt. 
Viele Kilometer wurden schweigend zurück gelegt, auf anderen war sie umringt von Menschen, die ihre 
Anteilnahme ausdrücken wollten. Völlig fremde Menschen hielten ihre Hand, küssten sie auf die Wange, 
streichelten über ihr Haar. Sarimé hatte Frauen und Kinder umarmt, hatte ein Kleinkind, welches im 
Gedränge die Mutter verloren hatte, vor sich in den Sattel gesessen. Und als die Mutter sie dann 
wieder fand, waren sie Seite an Seite, das Pferd führend mit dem Kind im Sattel, den Weg weiter 
geschritten. Es war... fremd gewesen. Und doch war das Gefühl so richtig!\\
\textit{Ich bin keine Fremde}, dachte sie und legte ihre Hand auf ihren Bauch: \textit{Sie wissen 
es.} Woher, konnte Sarimé nicht erahnen. Aber sie hatte Renecs Blick aufgefangen. Er schien daran 
nicht ganz unschuldig zu sein. \\
Als die Reisegesellschaft das Zeltlager aufschlug, war Sarimé wie in Trance. Falls jemand sie 
ansprach, bekam sie es kaum mit. Renec sorgte dafür, dass ihr Zelt gerichtet wurde und sie sich 
einige Stunden ausruhen konnte. Die vielen Eindrücke der Reise und die Sorgen über die Zukunft 
schwirrten durch ihre Gedanken. Auch dann noch, als sie sich für den Abend kleidete und schließlich 
das Zelt verließ.\\

``... unsere gnädige Gräfin...''\\
Sarimé blickte auf. Der Priester sah sie fragend an. Sie benötigte einen Moment, um wieder in die 
Gegenwart zurück zu kommen. Die versammelte Gruppe war sehr klein, gerade mal ein Dutzend Adelige 
und deren Bedienstete. Einige Priester und Menschen, die sich als enge Angehörige von Evin 
auszeichneten, waren anwesend. Die Sonne war längst hinter dem Meer verschwunden und ein kühler 
Wind zerrte an Sarimés Gewand. Der Priester, dessen Gesicht und rechter Arm mit ineinander 
verschlungenen tätowierten Mustern überzogen war, hielt eine flackernde Fackel in der Hand. Sarimé 
richtete sich auf und nahm die Fackel entgegen. Achtsam schritt sie zum errichteten Scheiterhaufen. 
Das Feuer knisterte laut, kämpfte gegen den erstickenden Wind an. Desto freudiger sprang es auf das 
trockene Stroh und Geäst über und breitete sich rasch aus. Die Hitze stieg ihr wie ein Faustschlag 
entgegen. Im selben Moment schloss sich eine Hand um ihre Hüfte und zog sie zurück. \\
``Vorsicht Herrin'', mahnte Renec. \\
Während das Feuer weiter tobte, blieb er dicht hinter ihr. Seine Nähe fühlte sich vertraut an und 
sie verspürte noch die Selbe Sehnsucht nach einer Verbindung zwischen ihnen, wie am ersten Tag in 
der Burg. Wie betäubt betrachtete Sarimé die Flammen. Feuer hatte sie stets faszinierend gefunden. 
Der wilde, unbezähmbare Tanz der Flammen, zuckend und gierig um sich greifend. Das Feuer war sich 
seiner Macht und Schönheit bewusst und geizte nicht damit, sich stolz zu präsentieren. Ihm war  
alles gleichgültig, er wusste von der Angst ihm gegenüber, dem Respekt und seiner Notwendigkeit. 
Seit Kindheit an hatte man Sarimé gelehrt, dass das Feuer Osymas liebstes Spielzeug war. Es war die 
Verbundenheit zwischen den Sterblichen und dem tobenden Gott. Aber nein, die Lehren der Priester 
waren ihr stets fremd vorgekommen. Sie konnte sich nicht vorstellen, dass sich das Feuer einem Gott 
beugen würde. Außerdem gierte es nicht stets nach Tod und Schrecken, Ehre und Blut wie der 
Allmächtige. Nein, das Feuer war zu frei und stolz um ein Spielzeug zu sein. Manchmal wirkte es, 
als würde es vor Freude tanzen, ein andermal war die Flamme ruhig und beständig. Ja, Feuer tötete, 
aber es schenkte auch Leben, Wärme, Hoffnung. Das Feuer holte sich alles, was es in die Finger 
bekam, aber nicht aus Bosheit, nicht aus Stolz oder Machtbesessenheit. Das war ihm alles unwürdig. 
Alles gleichgültig.\\
Die Helle, eine stille Göttin. Sie lebt im Kerzenschein, im Mond und im Meer.  Osyma, der 
Allmächtige, der Tobende. Der, der nach Ehre giert. Zwei Götter, die sich angeblich in ein und dem 
Selben Feuer verkörpern können.\\
Am Rande bemerkte sie, wie die ersten Anwesenden unruhig wurden. Es war spät, die Reise lange 
gewesen. Der Scheiterhaufen würde noch stundenlang brennen und die Abschiedsgäste nahmen an, sie 
hätten ihre Anwesenheitspflicht erfüllt. \\
``Herrin?'', ergriff Renec leise das Wort: ``Wir sollten zurück in das Zeltlager.''\\
``Wenn du frierst, dann geh'', erwiderte Sarimé laut. \\
Die Leute hören ihre klare Stimme aus der Stille deutlich herausstechen und wandten sich neugierig 
noch einmal um. Vielsagende Blicke wurden ausgetauscht. Sarimé konnte sich denken, was diese zu 
bedeuten hatten. Renec hatte es gut angestellt. Die Leute vermuteten bereits, dass er ihr Geliebter 
war. Eventuell sogar bald der neue Graf? \\
``Ich bleibe bei meinem Gemahl'', sagte sie ruhig und hielt ihren Blick auf den Scheiterhaufen 
gerichtet. Aber ihr entging die Unsicherheit auf seinen Zügen nicht. Nur ein winziger Moment. Hätte 
sie Renec nicht so gut gekannt, dann hätte sie die Regung einer Täuschung durch das flackernde 
Licht der Flammen zugeschrieben. Vielleicht nicht nur Unsicherheit sondern sogar Furcht? Vor ihr? 
Oder vor dem, was sie als Gräfin tun konnte? Diese Gedanken schenkten Sarimé Genugtuung. \\
\textit{Ich bin die Gräfin Merandilas. Witwe und bald Mutter. Das Volk hat mir auf der Reise hier 
her gegrüßt und willkommen geheißen. Ich brauche keinen Bastard, der mich behandelt wie ein kleines 
Kind.}\\
``Schlaf gut!'', wünschte sie und wandte sich wieder den aufstrebenden Flammen zu. Mit diesen 
letzten Worten war jede Diskussion ausgeschlossen. Würde Renec jetzt widersprechen, dann würden 
Sarimés Wachen eingreifen. Es war offensichtlich, dass sie seine Gesellschaft nicht wünschte. 
Widerwillig zog er sich zurück. Diese aber immerhin ohne Widerspruch. Die anderen Würdenträger 
nahmen es wohl als ein Zeichen, dass sie ebenfalls gehen konnten. Schlafen werden sie noch nicht. 
Nein, sie werden auf meine Kosten saufen und speisen, vielleicht sogar tanzen, während hier ihr 
Graf verbrennt und seine Witwe Andacht hält. Das war nun mal die Religion Osymas. Helden und Taten 
voller Ehre waren wichtig, Ausgelassenheit und Leidenschaft. Eine Beerdigung wurde gefeiert wie 
eine Hochzeit, solange der Verstorbene würdevoll gegangen ist. Nun, würdevoll würde sie sein 
Abtreten nicht bezeichnen, aber sein Leben mag es vor vielen Jahren einmal gewesen sein. Die Helle 
jedoch soll einst, als sie noch unter den Menschen lebte, gesprochen haben: Tod ehrt man mit Stille.
Mein Volk will die Stille, mein König und meine Adeligen die Leidenschaft.\\
Sarime seufzte. Seit sie in diese Grafschaft gekommen war, war alles kompliziert geworden. Selbst 
die Beerdigung ihres Gatten war eine Schlacht zwischen zwei Göttern. Sarimé blieb mit ihren fünf 
Wachen alleine beim Feuer zurück. Bald war nichts mehr zu hören als das beständige Knistern und die 
rhythmischen Wellen des Meeres. Der salzige Wind trug den Geruch des Feuers mit sich. Funken 
stiegen in den Himmel und verglommen wie verlorene Träume - ungehört, unwiederbringlich. \\
``Herrin'', sprach einer der Wachen sie an.\\
Sarimé hatte seinen Namen vergessen. Die letzten Tage waren zu ereignisreich gewesen, da war diese 
Information verschwunden, während sie sich über ihre Zukunft, die Grafschaft, ihre Pflichten und 
das ungeborene Leben in ihr sorgte. Aber sie schenkte ihm ein Lächeln, denn auch wenn sie seinen 
Namen nicht wusste, wusste sie, dass er die letzten Tage fast ununterbrochen in ihrer Nähe gewesen 
war. \\
\textit{ Aber vielleicht sollte ich vorsichtiger werden.}\\
Sie dachte an Renec, seine freundlichen Worte. Er war der Einzige gewesen, der sie hier als neue 
Braut des Grafen und völlig fremdes Mädchen willkommen geheißen hatte. Er war eine Stütze gewesen, 
die sie so dringend benötigt hatte. Ihr lief ein Schauer über den Rücken. Sie musste prüfen, ob sie 
ihm wirklich ihr Vertrauen schenken sollte.\\
``Braucht Ihr eine Decke?''\\
``Nein. Später vielleicht'', erwiderte sie: ``Aber ihr könnt euch gerne ebenfalls zurückziehen. Mir 
wird nichts passieren.'' \\ 
Sarimé lachte leise. Es war so albern. Fünf bewaffnete Männer schützten eine schwangere Witwe 
mitten im Nirgendwo, während die Leiche ihres Gatten gerade verbrannte.\\
``Gräfin'', sagte der Mann.\\
\textit{Samos}, erinnerte sich Sarimé wieder.\\
``Ihr wisst genau, dass wir das nicht tun werden'', entgegnete Samos entschieden.\\
``Weil ich euch bezahle'', stellte Sarimé fest.\\
``Unter anderem.''\\


Der Mond zog unbeirrbar seine Bahnen über das Sternenzelt, während die Geräusche des Zeltlagers zu 
Sarimé und den Wachen drang. Das beständige Rauschen der Wellen und das Knistern der Flammen 
versuchten die Musik und die Stimmen zu ersticken, aber es gelang ihnen kaum. Sarimé versuchte es 
zu ignorieren, aber je mehr Zeit verstrich, desto schwieriger wurde es. \\
``Was macht eine Witwe in diesen Stunden normalerweise?'', fragte sie irgendwann – weit nach 
Mitternacht – Samos. Er war der Einzige der fünf Wachen, den man noch als wach bezeichnen konnte. 
Als die ersten Beiden im kalten Sand eingenickt waren, hatte Sarmos sie wütend strafen wollen, doch 
Sarimé hatte ihn mit einer strengen Geste abgehalten.\\
\textit{Morgen, wenn ich wieder unter die Adeligen muss, brauche ich Wachen dringender als jetzt.}\\
``Was meint Ihr, Herrin?''\\
Sarimé zog die Schultern hoch und lehnte sich auf der Suche nach einer bequemeren Position auf den 
Stuhl zurück. Sie hatte Samos nicht widersprochen, als er sie vor wenigen Stunden dazu gedrängt 
hat, sich wenigstens zu setzen. \\
``Im Sinne der Hellen'', fügte sie hinzu und ihr Blick glitt flüchtig hoch zu den Sternen.\\
Samos war ein Merandil, hier geboren und hier würde er sterben, hatte er vor einigen Minuten 
erklärt. Die Merandil hatten – wenn man es genau nahm – einen ähnlich verbohrten Stolz wie die 
Saleicaner, wenn es um ihr Pflichtgefühl ging. Samos erklärte mit einer verschwörerischen Stimme, 
dass ihr Verhalten an Evins Totenbett ihr daher so viel Ansehen in der Burg eingebracht hatte. \\
``An gemeinsame, besinnliche Momente erinnern. Denn Erinnerungen sind das, was nicht einmal Dämonen 
uns nehmen können'', beantwortete er ihre letzte Frage.\\
Sarimé runzelte die Stirn und ihr Blick vertiefte sich wieder in die Flammen. Als suche sie nach 
einem Geheimnis, welches nur das Feuer kannte. Sarimé hatte keine besinnlichen Stunden mit ihrem 
Mann verbracht. Weder vor noch nach der Hochzeit. Das einzig intime war wohl, als sie ihm von dem 
Erben in ihr erzählt hatte. Aber zu diesem Zeitpunkt hatte der Tod sein Herz bereits umklammert.
Aber andere Momente kamen ihr in den Sinn. Der Tanz auf der Terrasse an ihrem Hochzeitstag. Der 
Kuss, wenige Stunden später in der Nacht, auf ihrem Zimmer. Gespräche und Ausritte. Objektiv 
zerlegte sie jeden dieser Momente in seine winzigsten Einzelteile und versuchte sie zu analysieren, 
während der Mond seinen Weg fortsetzte. Es waren schöne Momente. Bis auf den Kuss vielleicht. Der 
hatte nichts zu bedeuten. Sie grübelte lange und versuchte wieder einmal herauszufinden, was Liebe 
war und was sie wohl falsch machte. Unschlüssig schüttelte sie den Kopf. \textit{Wieso denke ich 
immer wieder darüber nach? Ich habe doch schon entschieden, dass es sie nicht gibt!}\\
Aber warum tat es dann so weh, dass sie Renec nicht vertrauen konnte? Sie hatte nie damit 
gerechnet, einen der Adeligen hier trauen zu können. Aber der Bastard war vom ersten Tag an anders 
gewesen. Sie hatte ihm viel zu schnell vertraut, obwohl sie doch wusste, dass ihr Leben teil eines 
gefährlichen Spiels war und sie vorsichtig sein musste.\\
``Die Sonne geht bald auf, Herrin'', sagte Samos. \\
Sarimé erwachte aus ihrer Erstarrung und blickte erstaunt zu dem heller werdenden Himmel hinter 
ihr. Die Flammen waren mittlerweile erloschen. Die Glut verbarg sich glimmend in den letzten tiefen 
Furchen der Stämme, welches das Fundament des Scheiterhaufens dargestellt hatten. Sarimé erhob sich 
und strich ihr Kleid glatt, während Samos mit einem kräftigen Tritt seine Kameraden weckte. \\
``Nun'', sagte sie feierlich zur Glut: ``Dann beginnt es also'' \\

