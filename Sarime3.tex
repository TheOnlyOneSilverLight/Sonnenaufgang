
\chapter{Die Pflicht}

Der Herbst zog ein in die nördlichste Grafschaft Saleicas. Und mit ihm kamen die Stürme. Es wurde 
stiller in der Burg. Eingehüllt in Pelzen und Mänteln huschten die Bediensteten und Wachen über den 
Hof, ohne den Kopf zu heben. Die Tiere wurden unruhig in ihren Ställen und egal welchen Raum Sarimé 
betrat, immer entdeckte sie eine Spur aus Matsch.\\
``Hattet Ihr nie den Wunsch, den kalten Jahreszeiten zu entfliehen?'', fragte Sarimé ihren Gatten 
und spießte eine Kartoffel auf.\\
Sie war nun seit Monaten hier in dieser Burg und sehnte sich nach Gesellschaft. Ab und an 
verbrachten sie die Abende gemeinsam in Salon, vor dem flackernden Kamin. Evin studierte Briefe, sie 
las Bücher. Immerhin hatte sie ihn zu einem Schachspiel überreden können, bis er verlor. Tagsüber 
verbrachte er viele Stunden in seinem Arbeitszimmer und nur zum Abendessen trafen sie sich. Und die 
Nächte natürlich.\\
Sarimé hatte keine Anstalten gemacht, zu ihm in ein gemeinsames Gemach zu ziehen. Sie gehorchte, 
wenn der Graf sie rufen ließ und stahl sich im dunkel davon. Noch immer schaffte sie es nicht ohne 
Tränen in diesen Nächten durch die Gänge der Burg zu gehen. Und jedes Mal wartete der Bastard aus 
sie. Und jedes Mal fürchtete sie, er würde nicht dort im Sessel sitzen, mit einem Buch in der Hand, 
in dem er ihr vorlas bis sie einschlief.\\
``Im Winter lassen die Stürme nach'', brummte Evin.``\\
''Aber warum reißen wir nicht nach Na'Rash?``, fragte Sarimé: ''Ihr seid der Graf. Ihr solltet 
regieren. Mehr als nur über verstreute Dörfer.``\\
''Die Burg liegt nähre an der Grenze. Und ich bin Hüter der Grenze!``, knurrte Evin: ''Außerdem 
verwalte ich die Grafschaft nur. Die Stadthalter Na'Rashs melden sich, wenn sie mich brauchen.``\\
''Ach ja? Über Briefe? Sie brauchen lang hier her. Ich sah, es kam heute einer mit dem Stadtsiegel 
an. Habt Ihr ihn schon gelesen?``\\
Er starrte sie böse an. ''Ihr nervt. Es war das Siegel der Priester und die haben selten was 
brauchbares zu sagen.``\\
''Nun, dann werde ich Euch diese Bürde abnehmen.``\\
Sie winkte Evins Kammerdiener und ließ nach dem Pergament schicken. Ihr Gatte ließ sich nicht aus 
den Augen ohne mit dem essen inne zu halten. Nur sein Husten, welcher ihn schon seit Tagen plagte, 
störte die Stille.\\
Sarimé betrachtete den gefalteten Brief und das Siegel aus goldenem Wachs. Das Symbol der 
gesegneten Flamme. Goldenes Wachs war bestimmt für den Glauben und den König. Wobei letzteres 
einen brüllenden Löwenkopf zeigte.\\
Die junge Gräfin brach das Siegel, faltete das raue Pergament auf und betrachtete die geschwungenen 
Linien. Zum dritten Mal nun überfolg sie die Zeilen, ehe sie den Brief auf den Tisch sinken ließ 
und ihren Gatten stumm ansah. Er biss gerade ein großes Stück Brot ab. ''Hm?``, fragte er mit 
vollem Mund: ''Einladung in den Tempel? Gemecker, dass ich bei den letzten drei nicht dabei war? 
Verspätete Glückwünsche zur Hochzeit? Nervige Fragen?``\\
''Nein``, sagte Sarimé langsam: ''Der Brief stammt nicht aus Na'Rash, sondern aus der Hauptstadt. 
Hohepriester Hisio-Mahar und König Semric verkünden den Krieg im Namen des Glaubens und des 
Allmächtigen.``\\
Evin kaute und schluckte, ehe er fragte: ''Mit einer Kolonie?``\\
Sarimé schwieg einen Moment. Dann sah sie auf. ''Wie viele Tagesritte von hier liegt Kasir?``\\
Der Graf erhob sich abrupt, griff nach dem Pergament und las selbst die Worte des Hohepriesters. 
Unterschrieben vom König persönlich.\\
''Schwachsinn``, fluchte Evin: ''Kasir ist unser Handelspartner. Wir haben einen Waffenstillstand, 
seit Generationen.``\\
''Seit mehr als Hundert Jahren. König Kareen ehelichte eine Prinzessin Kasirs, um den 
Waffenstillstand zu erhalten.``\\
Ein weiterer Hustenanfall ließ Evin schwanken. Er ballte seine Hände zu Fäusten und zerknüllte das 
Pergament. ''Krieg des Glaubens``, spuckte er aus: ''Ich zeige euch schon, was ich von eurem 
verfluchten Glauben halte! Lasst mein Pferd satteln!``\\
Der Kammerdiener eilte pflichtbewusst hinaus um den Befehl weiter zu geben. Überrascht schüttelte 
Sarimé den Kopf. ''Mein Herr``, rief sie: ''Was habt Ihr vor? Ihr könnt jetzt nicht reiten! Der 
Sturm tobt. Ihr werdet von einem Blitz erschlagen!``\\
''Ich stopf dem nächsten besten Hohepriester seine Kriegserklärung ins Maul``, knurrte Evin: ''Und 
der hockt im Tempel in Na'Rash! Krieg, wie stellt der König sich das vor? Im Schneesturm oder was? 
Das Bübchen, was noch nie Monate des Eises gesehen hat, schickt uns gegen ein Volk, welches auf 
Mammuts reitet und der Schnee auf ihren Bergen niemals schmilzt! Es gibt seine Gründe, wieso wir
immer gegen Kasir verloren. Selbst im Sommer!``\\
''Lasst mich mitkommen!``, bat Sarimé.\\
Er ignorierte sie und stampfte an ihr vorbei aus dem Speisesaal. Sarimé blieb alleine zurück, 
lauschte dem Heulen des Windes und dem Donnern, nachdem ein Blitz den Nachthimmel erhellte.\\


Drei weitere Tage tobte der Sturm. Dachziegeln der Burgen flogen durch die Luft und zerschellten an 
den steinernen Mauern. Ein Blitz schlug in die alte Eiche ein. Einer ihrer langen Äste löste sich 
und krachte auf das Dach der Stallungen. Zwei Luchse wurden erschlagen, fünf Pferde verschwanden 
hinaus in den Sturm und vier Falken fand man schutzssuchend in den Räumlichkeiten der Burg wieder. 
Am vierten Tag klarrte der Himmel auf. Mit der Unschuld eines Kindes beschien die Morgensonne die 
Hügeln und Täler, als wolle sie verdrängen, was die Winde und Blitze hinterlassen hatten. Am 
darauffolgenden Tag kehrte Evin A'Rik heim. Nicht mehr reitend auf seinem schwarzen Hengst, sondern 
liegend in einem klapprigen Karren, gelenkt von einem Müller und gezogen von einem stämmigem 
Ackergaul. Seine beiden Begleiter ritten mit gesenkten Köpfen auf den Hof. Sarimé erwartete sie 
bereits. Neben ihr der alte Priester der Gegend, der in der ersten Nacht des Sturms an die Tore der 
Burg klopfte und um Einlass bat.\\
Es wurden keine Worte des Grußes getauscht.\\
''Seit wann hat er das Fieber?``, fragte der alte Priester nach einem Blick auf den zitternden 
Grafen.\\
''Vielleicht schon, als er aufbrach``, murmelte Sarimé und lauschte der Erzählung der Wachen, wie 
ihr Graf schon am ersten Abend nach einem wilden Galopp aus dem Sattel fiel und im Matsch landete. 
Die Zeit des Sturms hatten sie in der Mühle ausgeharrt. Evin ging es dabei immer schlechter. 
Seine Flüche verstummten bald, wurden zu Husten und nach Luft ringen, sich in Albträume windend und 
vor Kälte zitternd.\\
''Sobald der Sturm vorbei gezogen war, machten wir uns auf den Weg zurück``, erklärte die Wache 
namens Samos.\\
''Bringt ihn in sein Gemach``, ordnete Sarimé an: ''Ihr beide erholt euch. Und Müller, sag, was 
kann ich dir geben um meine Dankbarkeit auszudrücken?``\\
Der Müller zuckte nur mit den Achseln und murmelte kleinlaut: ''Ich diene gern, Herrin.``\\
''Sprich``, forderte Sarimé ungeduldig und beobachtete, wie Evins Wachleute ihn mühsam auf eine 
Trage hoben.\\
''Der Sturm hat einigen Schaden angerichtet``, erklärte der Mann.\\
''Willst du Geld oder Helfer?``, fragte Sarimé.\\
''Zwei Burschen würden reichen.``\\
''Dann geh in die Küche, iss und wärm dich auf. Dort findest du auch die Knechte zur Mittagsstunde. 
Such dir zwei aus. Und gute Reise.``\\

Evin lag begraben unter drei Decken. Schweißperlen glitzerten auf seiner wettergegerbten Stirn. 
Sein glasiger Blick hing weit in der Ferne. Seine Haut fühlte sich an wie glühende Kohle, als sie 
ihre Hand auf seine Stirn legte.  Reglos saß sie auf der Bettkante und starrte das fahle Gesicht und 
die glasigen Augen ihres Gatten an. Der Priester Hochna zerstampfte Kräuter zu seinem Sud und 
murmelte vor sich hin: ''Auf die Welt hab ich ihn geholt. Ihn und alle seine Kinder. Und allen 
habe ich den letzten Segen geweiht. Zwei mal gab ich ihm unter Osymas Himmel eine Frau in die Hand. 
Und dann fällt er mir einfach vom Pferd!``\\
Sarimé sah auf. ''Wieso habt Ihr nicht bei unserer Hochzeit die Zeremonie geleitet?``\\
Er schnaufte nur. ''Ich sagte ihm, nach allem was geschehen ist, dass der Allmächtige andere Pläne 
für ihn hat. Es ist nicht Osymas Wille, dass er ein drittes mal heiratete. Er war schon immer so 
stur.``\\
''Dann danke ich Osyma, dass der Sturm Euch zurück geführt hat``, murmelte Sarimé und beobachtete, 
wie der Priester einen kühlenden Umschlag anbrachte. ''Flößt ihm den Tee ein. Er muss trinken``, 
befahl er, nachdem er mit seinen vor Alter zitternden Händen nicht in der Lage war, die Tasse an 
Evins trockenen Lippen zu halten.\\
''Was können wir noch tun?``, fragte sie.\\
''Beten.``\\

Sie schlief im Sessel neben Evins Bett, schreckte bei jedem stockenden Atemzug oder Hustenanfall 
auf. Das Fieber sank nicht, egal wie viele feuchte Tücher sie auf seine Stirn legte oder sie das 
Zimmer lüftete. Die Untätigkeit war das Schlimmste. Sie wollte das Zimmer nicht verlassen. Er war 
ihr Gatte, auch wenn sie ihn nicht liebte. Er hatte ihr eine Heimat geboten, auch wenn nur aus 
Eigennutz. Sarimé begann zu sticken und zu lesen, zwischendurch fütterte sie ihn oder las ihm 
Briefe vor. Briefe über die Kriegsvorbereitungen. Briefe über wirtschaftliche Belange. 
Besserungswünsche vom Adel. Innerhalb dieser zwei Wochen verließ sie nur zwei Mal den Raum um sich 
Bücher zu holen und frisch anzukleiden.\\
Die Tage waren eintönig und Evins Zustand schwankte bedenklich. Nach wenigen Tagen seiner 
Erkrankung sah es so aus, als würde er wieder genesen. Er stand sogar am Fenster und sah auf das 
Treiben im Hof. Doch dann ging es rapide abwärts. Der Graf öffnete kaum noch die Augen. Er aß nichts 
und Sarimé verlor ebenso den Appetit. Übelkeit übermannte sie, auch wenn sie kaum erbrach, da sie 
nicht viel zu sich nahm. Sie fürchtete schon, ebenfalls bald vom Fieber ans Bett gefesselt zu 
werden. Die junge Gräfin steckte gerade zum wiederholten Mal die Nadel in das Tuch und formte ein 
Muster, als die Magd sie ansprach. Sie räumte gerade das Geschirr fort. \\
``Herrin…'', sagte sie vorsichtig.\\
``Hm?'', entgegnete Sarimé, warf einen schnellen Blick auf den schlafenden Grafen und blickte dann 
wieder zu ihrer Stickarbeit.\\
``Verzeiht wenn ich Euch zu nahe trete… aber ich habe mit eurer Zofe gesprochen und… es sind 
mehrere denen es aufgefallen ist…''\\
``Wovon sprichst du?'', fragte Sarimé ungeduldig.\\
``Ihr hab schon lange nicht mehr…'' Sie sah zum Grafen und flüsterte: ``geblutet....''\\
``Du meinst…?'' Sarimé verstummte.\\
Die Luft fühlte sich plötzlich viel zu drückend an. Ihre Beine taub und die Stichwunde der Nadel 
pochte. Sie erhob sich, legte vorsichtig ihre Stickarbeit nieder und sah den Schlafenden einen 
langen Moment an. ``Sorge dafür, dass der Graf nicht alleine ist.''\\


Die Burg besaß einen Garten, der zum größten Teil von Unkraut überwuchert war. Bisher hatte Sarimé 
ihm keine Beachtung geschenkt, doch jetzt trugen ihre Füße sie an diesen Ort. Sogar ein steinerner 
Brunnen stand in der Mitte der kleinen Anlage. Die Pumpe funktionierte jedoch nicht und selbst das 
Regenwasser, welches das Becken aufgefangen hatte, war mit Pflanzen überwuchert.Die wilden Blüten 
und das Gras waren ausgetrocknet und verblüht. Toter Efeu klammerte sich an das Gestein einer 
Statue, als hätte es den Kampf ums Überleben nicht bereits verloren. Alles hier in diesem Garten 
zeugte davon, dass der Herbst mit seinen kalten Winden das Land erobert hatte. Sarimé trat zu 
einem älteren Mann, der auf einer Bank am Brunnen saß. Er trug eine ausgeblichene Mütze und sein 
Blick schweifte unbeirrt über den Garten. Als würde er hier noch die alten Zeiten sehen, in denen 
der Garten in voller Blüte stand.\\
``Du bist der Gärtner, oder?''\\
Er sah auf, starrte sie einen Moment an und neigte dann langsam den Oberkörper. ``Meine Herrin.''\\
``Warum tust du nicht die Arbeit, für die du bezahlt wirst?''\\
``Tue ich. Ich hege den Gemüsegarten.''\\
Sarimé deutete auf den Ziergarten. ``Und das hier?''\\
Als Antwort zuckte der Gärtner nur mit den Schultern. ``Der Herr hatte nie großes Interesse daran. 
Seit dem Tod seiner ersten Frau liegt er brach. Man trug mir auf, etwas sinnvolleres zu tun.''\\
Die junge Gräfin biss sich nachdenklich auf die Zunge, dann setzte sie sich neben den Mann und 
teilte sein Schweigen. \\
"Wie sah er früher aus?", fragte sie dann leise.\\
Der Gärtner legte den Kopf schief und lächelte verträumt. "Genauso klein. Aber die Herrin wählte so 
viele verschiedene Blumen aus... die Farben kann ich gar nicht beschreiben. Sie war oft hier. Jede 
freie Minute. Und die Kinder auch. Jedes ist mindestens einmal in den Brunnen geklettert, wenn ihre 
Mutter nicht hinsah."\\
Die junge Gräfin lauschte seinen Worten nachdenklich und versuchte dieses Stück verwilderte Erde 
mit seinen Augen zu sehen. \textit{Kinder... Mein Kind könnte hier spielen.}\\
Sie legte ihre Hand auf ihren Bauch und horchte in sich. Sie suchte nach dem neuen Leben in 
sich. ``Wie war die Familie des Grafen?''\\
``Laut'', sagte er nachdenklich und lächelte melancholisch. ``Gräfin Linessa hörte man durch die 
halbe Burg. Sie redete oder sang fast immer. Oder tadelte die Kinder. Sie lud oft zum Tanz und 
geselligen Runden.''\\
``Wen?''\\
``Ihre Freunde. Und die unseres Grafen. Einige Adelige, Kaufleute, Verwandte.''\\
Sarimé versuchte sich Evin bei solchen Gesellschaften vorzustellen. Seine erste Frau, die laut 
lachend Gespräche führte. Wie er selbst auf seinem Stuhl saß, am Bier nippte und sie dabei 
beobachtete, während ein Freund einen Witz erzählte. Zu dieser Zeit muss Evin ein anderer Mensch 
gewesen sein. Ohne die Leere, die der Tod in seinem Herzen hinterlassen hatte. Sarimé fiel es nicht 
schwer, ihren Gatten als so einen Mann zu sehen. ``Und die Söhne?''\\
Der Gärtner lachte leise. ``Oh... der Älteste wäre jetzt ende Zwanzig. Er hätte Euch gefallen, 
Gräfin. Ein kluger junger Mann war er, zeichnete mit Leidenschaft Karten, nachdem er die Strecken 
abgeritten war. Dabei verstellte er sich oft, weil er keine Lust hatte, bei der Bevölkerung 
Aufsehen zu erregen. Er hieß auch Evin. Der Mittlere wollte immer nur Soldat werden. Einmal hat er 
einen Hungerstreik angefangen, weil seine Mutter ihm es verbieten wollte. Da war er zehn. Er hats 
nicht lange ausgehalten! Man fand ihn an einem Morgen eingeschlafen in der Speisekammer, umgeben 
von Brotstücken und weicher Butter. Der Jüngste kam ganz nach Gräfin Linessa. Er folgte seinem 
Herzen, sprach immer aus, was er dachte. Und das war selten höflich. Er starb in der Grafschaft 
Kantor, wollte da die Bastardtochter der Gräfin den Hof machen.''\\
Er seufzte tief und verfiel in Schweigen. Auch Sarimé träumte sich einen Moment weg von der 
Realität und stellte sich die Welt vor, wie sie sein würde, wenn diese Tode nicht über Evins 
Familie gekommen wären. Vielleicht wäre sein Sohn General geworden, der Jüngste ein Künstler, 
welche weit fort vom Leben des Adels geblieben wäre. Und vielleicht hätte Sarimé einem jungen, 
Karten zeichnendem Evin gegenüber gestanden, als sie die Kutsche vor dieser Burg verließ. Sie 
dachte an die verstorbene Gräfin, die hier im Garten stand, ihr Lied sang und dabei das Unkraut aus 
der Erde zupfte. Hinter ihr tollten die Söhne über die winzige Grasfläche. Der Älteste saß 
vielleicht auf dieser Bank und las ein Buch?\\
\textit{Mein Kind wird hier spielen.}\\
Es war wie ein feierlicher Beschluss und ein Schwur, dass sie endlich dieses Land als ihre Heimat 
akzeptieren würde. Und als die Heimat ihrer Kinder.\\
``Bereite den Garten für den Winter vor. Schneide die Hecken zurück. Und im Frühling besprechen 
wir, welche Pflanzen im Garten angepflanzt werden sollen.''\\
``Wie Ihr wünscht, Herrin!'', sagte er und ein Lächeln huschte über seine schmalen Lippen.\\
Die Gräfin verabschiedete sich höflich und als sie die Stufen zum Eingangsportal hinaufstieg, 
lächelte sie. Ihre Hand ruhte wieder beiläufig auf ihrem Bauch. \\

Auf ihrem Rückweg eilte sie durch die Gänge. Sie hasste Evin dafür, wie er sie in der 
Hochzeitsnacht und die Male danach genommen hatte. Sie hasste ihn dafür, dass er sie gekauft hatte 
und sie wegen ihm in die Fremde musste. Aber sie wusste auch, wie sehr er sich einen Erben wünschte 
und sie konnte ihn nicht genug hassen, um ihm stillschweigend beim Sterben zuzusehen. Es war ihre 
Pflicht und sein Recht zu erfahren, dass sein Wunsch in Erfüllung gehen könnte, falls das Kind 
überleben würde. Als sie in das Zimmer trat, sah sie den Priester Hochna an Evins Bett stehen. 
Ungeduldig, sie wollte mit Evin alleine sein, fragte sie ihm nach dem Befinden des Grafen. 
\textit{Wie oft habe ich diese Frage in den letzten Wochen gestellt? Wie oft hat er geantwortet, 
dass wir beten müssen?}\\
Dies Mal sagte er nichts der gleichen. Stattdessen wagte er es kaum ihr in die Augen zu blicken und 
erwiderte: ``Der Graf ist schon bei Osyma. Bald wird ihm der letzte Lebensfunke folgen.''\\
Sarimé nickte nur und blickte zu Boden, als sie den Priester bedeutete, zu gehen. Dann trat sie 
langsam an die Seite des mit Fellen belegten Bettes und kniete nieder. Rau und hart drückte das 
Gestein des Bodens in ihre Knie. Tage hatte sie widerstanden. Hatte sie sich geweigert und es allen 
anderen überlassen. Sie dachte, solange sie es nicht für nötig hielt, wäre es nicht nötig. Aber nun 
suchte sie stumm die richtigen Worte für ein Gebet.\\
Evin kam ihr zuvor und flüsterte in die Stille: ``Du bist anders als der Windgeist.''\\
``Ich wäre nie ein guter Windgeist. Jeder Mensch würde fliehen, sobald ich anfange zu singen.''\\
Er lachte, bis das Geräusch in rasselnden Atem überging. Als sein Brustkorb wieder regelmäßiger hob 
und sank sprach er: "Als ich Sieva das erste Mal sah... ach.. sie sah aus wie der stolze Löwe auf 
Saleicas Wappen. Deshalb nahm ich sie mit, weißt du. Ich Narr. Ich verwechselte Stolz mit 
Unnahbarkeit. Erhabenheit mit Abwesenheit. Anfangs... ich habe mich wirklich bemüht. Ich 
vernachlässigte meine Pflichten, um ihre Wünsche zu erfüllen. Um sie zum lächeln zu bringen. Ich 
ritt tagelang um Blumen zu finden, die die Farbe ihres Haars hatten. Ich warf das Geld des Königs 
kasirsichen Kaufleuten in den Wolfsrachen um ihr die reinsten Perlen zu kaufen. 
Und was hatte ich dann davon? Sie zwang mich, Renec an den Hof zu holen. Ich tat alles, was sie sich 
wünschte, weißt du... Aber mein Wunsch wurde nicht erfüllt. Stattdessen vögelte sie meinen Bastard! 
Immerhin hat er es auch nicht geschafft, sie zu schwängern."\\
Seine Rede hatte mit Worten voller Reue begonnen und endete mit Verbitterung, immer wieder von 
Keuchen und trockenem Husten begleitet.\\
``Mein Graf'', sagte sie eindringlich und drückte fest seine hagere Hand: ``Osyma ist gnädig mit 
Euch. Ich bin schwanger.''\\
In seinem Gesicht zuckte etwas. Vielleicht ein flüchtiges Lächeln? Er schloss die Augen und seine 
Hand wurde schlaff. "Erst so kurz. So viele Leben sterben bereits, bevor sie geboren sind. Ich 
werde nicht da sein um aufzupassen, dass es wirklich geboren wird."\\
Sarimé schluckte und suchte fieberhaft nach Worten. "Ihr könnt Osyma direkt sagen, dass er 
gefälligst dafür zu sorgen hat."\\
Seine Hand hob sich und strich ungeschickt über ihr Haar. "Du willst nicht hier sein. Uns verbindet 
nichts", murmelte er: "Du bist zu jung zum herrschen. Zu jung um alleine Mutter zu sein."\\
"Unterschätzt mich nicht."\\
"Schwöre mir, dass du alles tun wirst, um das Kind zur Welt zu bringen. Schwöre mir, dass du dafür 
sorgen wirst, dass es lebt und überlebt. Dass mein Blut nicht ausstirbt."\\
"Ich schwöre es bei Osymas Flammen und dem Licht, das er uns schenkt."\\
Es war die letzte Nacht, die Sarimé an seiner Seite verbrachte. Am nächsten Vormittag verstarb Graf 
Evin A’Rick, Wächter der Nordgrenze, der weiße Luchs und letzter Lebender Nachfahre der besiegten 
Königin Merandilas.\\


Sarimé fand sich als alleinige Herrin der Burg, der ganzen Grafschaft wieder. Zuallererst schickte 
sie den Kammerdiener los, einen Priester zu holen, der sich um Evin kümmern sollte. Anschließend 
zog sie sich in ihr Gemach zurück, um Briefe an die umliegenden Grafen und den hohe Adel zu 
verfassen. Mägde wies sie an, dass Zimmer zu reinigen, denn während der gesamten Krankheitsdauer 
über war dort, bis auf Sarimés Lüften und Auskehren der Asche im Kamin, nicht geputzt worden. Als 
Witwe überwachte sie selbst, dass Evins Körper mit dem angemessenen Respekt behandelt und die Riten 
eingehalten wurden. Am Abend schlich sie in ihre Zimmer und fand keine Ruhe. Unschlüssig lief sie 
immer wieder durch den Raum und grübelte über ihre Zukunft nach. Angst überkam sie.
\textit{Ich bin dem allen nicht gewachsen. Viel zu jung und unerfahren! Und nun auch noch das 
Kind.. Wer würde mich jetzt noch heiraten wollen, selbst wenn er Graf wird? Wer würde jetzt 
freiwillig Graf Merandilas werden wollen, wenn der Krieg hier toben wird?}\\
Den Tränen nahe kauerte sie auf dem Bett. Die roten Haare verbargen ihr Gesicht, wie ein 
geschlossener Vorhang eine Theaterbühne. Sarimé nahm das Klopfen nicht wahr, aber auch ohne ihre 
Aufforderung betrat der Bastard ihr Zimmer. Vorsichtig öffnete er die Türe und steckte den Kopf 
durch den Spalt. Als Renec die junge Gräfin entdeckte, war er schnell an ihrer Seite und nahm sie 
in den Arm. Die Berührung tat gut, aber sie wich ihr aus. Es ziemte sich nicht für eine Witwe und 
erst recht nicht für eine werdende Mutter, einen anderen Mann zu umarmen. Renec machte keine 
Anstalten die Geste zu wiederholen, sondern setzte sich aufrecht hin und schwieg, bis sie das Wort 
an ihn richtete.\\
``Warum bist du nicht bei ihm?'', fragte sie hart.\\
Es galt die Totenwache zu halten um den Körper des Verstorbenen nicht hungrigen Dämonen zu 
überlassen. Mindestens fünf gesegnete Flammen mussten in einem Kreis um den Toten scheinen und kein 
anderes Licht darf ihn treffen. Osyma selbst urteilte in diesen Stunden über die Seele und dessen 
vergangenem Leben.\\
``Hochna gab mir zu verstehen, dass er jeden Erben des Grafen auf die Welt half. Ich gehöre nicht 
dazu.''\\
``Wie wird es nun weiter gehen?'', fragte Sarimé und blickte auf.\\
\textit{Er hat wirklich Evins Augen.}
Renec erklärte: ``Er sollte dem Wind und dem Meer übergeben. Seine letzte Reise, an der Küste nimmt 
er von unserer Welt abschied.''\\
Sarimé sah auf. ``Wird er nicht verbrannt? In Brom-Dalar werden die Toten verbrannt... In 
gesegneten Flammen.''\\
Renec zuckte mit den Schultern. ``Ihr wisst doch, Merandila hat sich viele alte Bräuche behalten. 
Osyma ist der Gott des Feuers und der Ehre. Bevor Merandila eine Grafschaft wurde, verehrten die 
Menschen die Helle.''\\
``Die Helle'', murmelte Sarimé. Sie hatte schon einige Leute diesen Namen aussprechen hören, sich 
aber nie weiter Gedanken darüber gemacht.\\
Renec nickte. ``Die Helle, sagte man, hat viele Gestalten. Eine weißhaarige Frau, ein weißer Wolf, 
Fuchs, Kaninchen oder Vogel. Es heißt, sie erschien den ersten Königen als weißer Luchs. Sie ist 
aber so viel mehr. Sie ist Wind und Meer, hell und zart. Das Licht.''\\
``Eine schwache Göttin'', bemerkte Sarimé, als sie die beiden Götter verglich. \\
Renec kniff die Augen zusammen. ``Auf den ersten Blick vielleicht. Nun, Ihr könnt Evin natürlich 
auch in Osymas Flammen geben. Aber das wäre ein weiterer Beweis für das Volk, dass ihr keine 
Merandil seid.''\\
``Es gibt das Land Merandila nicht mehr'', sagte Sarimé.\\
``In den Herzen der Leute wird es immer ihr Land bleiben. Da kann der König so viele Priester und 
Krieger schicken, wie er will.''\\
``Das Volk verehrt Osyma. Sie beten in seinen Tempeln, sie melden sich bei der Armee, ziehen für 
ihn in den Krieg, ehren seine Lehren und lieben den König. Sie tragen das Banner des Löwen.''\\
``Die Städter, ja, aber schaut Euch die Dörfer an, Herrin. Reitet durch die Landschaft, unterhaltet 
Euch mit den Bauern und Handwerkern. Ich will nicht behaupten, dass sie Osyma nicht ehren, aber 
trotzdem... mit jedem Rind, das sie dem Feuergott opfern, halten sie einen Moment vor einer weißen 
Blume oder einer Kerzenflamme inne. Es steckt zu Tief in ihrem Blut, sie können nicht ändern, was 
sie sind. Sie haben die Helle nicht vergessen. Auch mein Vater vergaß sie nie.''\\
``Ein Gott oder der Andere, wer schert sich darum?'', schnaufte Sarimé und erhob sich: ``Das 
Einzige was die Leute wollen ist ein Heim, genügend zu Essen, Sicherheit und Ehre. Osyma ist nur 
ihre Rechtfertigung, sich das alles von anderen zu nehmen.''\\
``Einen Unterschied gibt es'', erwiderte Renec und lächelte: ``Die Helle lebt.''\\
Sarimé sah ihn einen langen Moment stumm an. Dann schüttelte sie den Kopf. ``Alles existiert, wenn 
man nur fest genug daran glaubt. Fragt die Priester Osymas, ob ihr Allmächtiger existiert. Frag 
Hochna.''\\
Ihre Zurechtweisung traf ihn sichtlich. Renec richtete sich gerade auf und räusperte sich. ``Was 
gedenkt Ihr zu tun, Herrin?'', fragte er trocken.\\
``Zur Küste reisen und Evin deiner Hellen überbringen. Ich bin keine Merandil, du hast Recht. Und 
ich werde nie eine sein. Ich werde bald einen Anderen heiraten, der die Grafschaft führen wird. Er 
kann sich anschließend mit dem Zorn der Priester oder dem Volk auseinandersetzten.''\\
``Und Ihr?''\\
Sarimé legte die Hand auf ihren Bauch. ``Ich will nur ein Heim, genügend zu essen und 
Sicherheit.''\\
Renec folgte ihrer Hand und seine grauen Augen weiteten sich. ``Ihr tragt seinen Erben in Euch?''\\
Sarimé nickte und eine mühsam unterdrückte Träne rann über ihre Wange. Renec sprang auf. ``Warum 
sagtet Ihr nichts? Das ändert alles!''\\
``Nämlich? Dass ich doch nicht ganz unnütz bin, sondern immerhin Kinder gebären kann?''\\
\textit{Falls es überhaupt leben wird...}\\
``Nein! Versteht Ihr nicht? Durch das Kind seid Ihr ein Teil Merandilas. Kein Merandil wird sich 
gegen Euch erheben, wenn dieses Kind leben wird. Durch dieses Kind seid Ihr nicht gezwungen, einen 
Mann, den der König Euch aussucht, zu ehelichen, Herrin. Ihr wärt die alleinige Herrin der 
Grafschaft. A'Rik war ein Mann des Königs. Aber in seinen Adern floss das Blut der letzten freien 
Königin Merandilas. Daher hat sich nie jemand gegen ihn erhoben. Er war eine der letzten, wenigen 
Verbindungen zur Vergangenheit. Zur Freiheit. Auch wenn er selbst nicht mehr als ein Sklave 
Saleicas war.''\\
Sarimé lauschte seinen Worten und beobachtete, wie Renec aufgeregt durch den Raum schritt. 
\textit{So... triumphierend habe ich ihn noch nie erlebt.}\\
Als er endete schüttelte sie den Kopf. ``Das schaffe ich nicht.''\\
``Ich helfe Euch! Ihr werdet es schaffen. Ihr seid schön, beliebt und klug. Zeigt Euch dem Volk. Es 
wird Euch so sehr lieben wie die Bewohner der Burg und Talsmund. Deren Herz habt Ihr erobert, den 
Rest Merandils werdet Ihr auch erlangen.''\\
``Habe ich das?'', fragte sie zweifelnd.\\
``Hört Euch doch um! Schon als ich vorhin mein Pferd in den Stall brachte, erzählten die 
Stallburschen, dass Ihr Evin nicht von der Seite gewichen seid! Wie aufopferungsvoll Ihr Euch um 
ihn gekümmert habt, obwohl die Liebe zwischen euch als Ehepaar nicht sehr offensichtlich gewesen 
war, wenn Ihr versteht, was ich meine. Sie bewundern Euer Pflichtgefühl und dass Ihr Dinge 
erledigt, ohne zu klagen.''\\
Sarimé holte tief Luft. ``Er war mein Mann. Vor Osyma und Zeugen angetraut. Meine Vorstellung von 
Liebe hat sich nicht geändert. Es war... niemand verdient es, einsam zu sterben. Deshalb habe ich 
es ihm auch gesagt. Er ist mit der Gewissheit gestorben, dass er einen rechtmäßigen Erben haben 
wird. Nenne es Pflicht, wenn du willst.''\\ 
Sarimé konnte nur den Kopf schütteln. Sein Vater war eben erst gestorben und Renec zeigte keinerlei 
Trauer.\\
``Mein Vater ist mit dem Wissen gestorben, dass seine Witwe bestimmt ist zu herrschen!''\\
\textit{Herrschen?}, wiederholte Sarimé gedanklich: \textit{Ich herrsche nicht. Ich verwalte das 
Land des Königs. Wenn überhaupt.}\\
Sarimé schüttelte weiter den Kopf und trat auf das Fenster zu. Die Blätter der Bäume glichen Osymas 
Flammen, die sich zornig und aufgebracht gen Himmel reckten. \textit{Weiß er, wovon Renec spricht? 
Sagt der rachsüchtige Gott es seinen Priestern, seinem König? Erzählt er ihnen von den Worten des 
Verrats, die in diesem Raum, in diesem Moment ausgesprochen wurden?}\\
