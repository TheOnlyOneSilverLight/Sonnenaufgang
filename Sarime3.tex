
\chapter{Die Pflicht}

\textbf{Zeitsprung von drei Wochen. Evin war in Na'Rash beim Priester, Kriegserklärung an Kasir 
wurde verkündet}\\

Schon wenige Wochen nach ihrer Eheschließung musste Evin in die Hauptstadt der Grafschaft reisen. 
Sarimé hatte von einer Nachricht eines Priesters an Evin erfahren und darauf bestanden, mit ihm zu 
reisen. Aber ihr Gatte hatte abgelehnt und sie dazu verdonnert, in der Festung alleine zu bleiben. 
Sarimé bereute es kaum. Ohne Evin war es ruhiger in der Burg. Die Diener entspannter, die Regeln 
unwichtiger. Die junge Gräfin nutzte die Zeit, sich weiter mit den Bediensteten bekannt zu machen. 
Sie ließ sich die Buchhaltung der letzten drei Jahre zeigen und rechnete alles nach. Es machte ihr 
Spaß mit den Zahlen umzugehen. Es erinnerte sie an die unbeschwerten Tage, als ihr Vater ihr das 
Rechnen beibrachte. Da ihr die Ausgaben unnötig hoch erschienen, koordinierte sie Einsparungen. 
Letzten Endes wäre es egal, meinten der Buchhalter und die Köchin, denn die Ausgaben lagen noch 
deutlich im Budget, welches jedes Quartal vom König gestellt wurde. Aber Sarimé war nie gerne 
abhängig und notfalls würden die kleinen Einsparungen wichtigere Dinge finanzieren lassen. \\
Sie ritt öfters mit ihrer Stute hinunter nach Talsmund, beobachtete die Handwerker und 
Bauern und wechselte ein paar höfliche Worte mit ihnen. Es war... schön. Sarimé konnte nicht 
bestreiten, dass ihr die faszinierten Blicke gut taten. Es gab ihr das Gefühl, etwas richtig zu 
machen. Auch mit Renec vertrieb sie ihre Zeit. Sie ließ sich von ihm die Bibliothek zeigen, 
verlangte, dass er ihr bei Inventur über die Bücher half und mittlerweile schickte sie ihn immer 
wieder fort, um in der umliegenden Gegend nach weiteren Büchern zu suchen. Und oft genug redeten sie 
auch nur. Ihr war gar nicht bewusst, wie viel sie ihm sagte. Sie erzählte ihm von all ihren Ideen. 
Von den Einsparungen bis hin zu der Überlegung, für die Bewohner Talsmund eine kleine Ansammlung 
nützlicher Sachbücher bereitzustellen. Prinzipiell gab es eine kostenlose Schule in Saleica, in der 
jeder Mensch unterrichtet werden konnte. Aber dieses Angebot befand sich meist in den größeren 
Städten und dort eine Unterkunft zu finden konnten sich die Dorfbewohner selten leisten. Trotzdem 
konnten überraschend viele Lesen. Sarimé fragte nach und erfuhr, dass der Holzfäller und 
Geschichtenerzähler Arham diese Fertigkeit an Interessierte weitergab. Als Anerkennung gab sie in 
der Schmiede auf der Burg eine neue Klinge für sein Beil in Auftrag. Die Tage vergingen und der 
Herbst zog auf. Sogar ein Brief aus Evins Feder fand sich ein. Sarimé überraschte das sehr, damit 
hatte sie nicht gerechnet. Aber es waren lediglich sachliche Informationen und Anweisungen. 
\textit{Krieg.}\\
Sie hielt das Pergament noch in ihren Händen und überflog den Absatz zum dritten Mal. Evin beschrieb 
über die Verkündung der Kriegserklärung gegen Kasir, ihrem Nachbarland im Norden. Es grenzte direkt 
an Merandila. Sarimé starrte lange auf die verschlungene Schrift und überlegte, was dies bedeuten 
würde. \textit{Ich werde mit Evin reden müssen. Ich muss erfahren, was auf uns zu kommt.}\\
Die Nachricht zog schnell ihre Bahnen. Immer öfter sah man trainierende Soldaten aus dem nahen 
stehenden Lager. Renec bat Sarimé, nicht mehr alleine auszureiten. Darüber konnte sie jedoch nur den 
Kopf schütteln. Sie waren drei Tagesritte von der Grenze entfernt. So schnell würden keine Kasira 
nach Merandila kommen. Trotzdem, die Unsicherheit blieb, bis Renec ihr eines Abends beim Essen – 
welches Sarimé mittlerweile gerne in Gesellschaft vom Bastard einnahm – versicherte: ``Der Herbst 
kommt. Es hört kaum auf zu regnen und da oben in Kasir wird der Winter noch schneller kommen. Vor 
dem ersten Frühlingstag wird es keine Schlacht geben.''\\
Sie nickte nur und als Antwort auf seine Vermutung klammerte sich an diesen Gedanken. Und, sie 
konnte es kaum selbst glauben, sie hoffte, Evin würde bald wieder zurück kommen. Ihr entging nicht, 
wie Leute sie beobachten, auf eine Reaktion warteten. Sarimé bemühte sich so weiter zu machen wie 
vorher. \\

Drei Tage hatte es ununterbrochen geregnet. Die junge Gräfin blieb meist innerhalb der Burg, 
verbrachte die Abende vor dem Kamin mit einem Buch. Aber mit jeden Abend wurde sie nervöser. Evin 
hatte mitgeteilt, dass er bald wieder hier sein würde. Er hätte schon längst aus Na'Rash 
zurückgekehrt sein müssen. Nun war sie schon fast über ihrem Buch eingeschlafen, als eine Magd 
heftig gegen ihre Zimmertüre klopfte und ohne auf eine Antwort zu warten eintrat. \\
``Herrin'', rief sie atemlos: ``Man hat Lichter gesehen. Eine Gruppe Reiter kommt.''\\
Sarimé erhob sich in einer fließenden Bewegung und schlug das Buch zu. ``Ich komme. Sorge dafür, 
dass 
ein heißes Bad für den Grafen vorbereitet wird. Und schau nach, ob die Köchin noch etwas von den 
Resten des heutigen Abends aufwärmen kann.''\\
Die junge Gräfin schlüpfte in einen eher praktischen als modischen Pelzmantel und schritt eilig 
durch die Gänge der Burg. Die Türen des Portals waren bereits geöffnet, sie war nicht die Einzige, 
die die Heimkehrer erwartete. Sie trat die Stufen hinunter in den Regen und sah, wie eine Gruppe 
Reiter durch das Tor trabten. Stallburschen eilten herbei um den Männern die Zügel abzunehmen. 
Sarimé kniff die Augen zusammen und starrte auf das stämmige Schlachtross. So stolz und stark es 
auch da stand, Evin fiel regelrecht aus dem Sattel. Vielleicht lag es an der Dunkelheit der 
Sturmnacht, aber der Graf war nur ein Schatten seiner selbst.\\
``Helft ihm!'', befahl Sarimé zwei Wachsoldaten und deutete auf ihren Gatten. Sie trat einige 
Schritte 
näher, beobachtete, wie die Männer ihren Mann unter die Arme griffen und aufrichteten. ``Ihr 
hättet eine Kutsche nehmen sollen'', rief sie.\\
``Die ist was für Weiber'', zischte er und lachte glucksend. Schnell wurde das Geräusch zu einem 
rauen 
Husten.\\
``Eine Erkältung auch?'', murmelte Sarimé und schüttelte den Kopf.\\
Der Graf wurde in die Burg gebracht. Sie klatschte in die Hände und rief den Burschen zu: ``Beeilt 
euch, kommt ins trockene. Wärmt euch nach der Arbeit in der Küche auf, dort brennt der Ofen 
immer.''\\

Am nächsten Morgen kam Evin nicht zum Frühstück. Sarimé aß nicht viel, sondern beeilte sich ihn zu 
suchen. So viele Fragen lagen ihr auf der Zunge. Nach Na'Rash, der Kriegserklärung, dem Brief des 
Königs und den Priestern Oysams im dortigen Tempel. Was er nun zu tun gedenke, wie es ablaufen 
würde, ob sie noch mehr Soldaten brauchten, ob es vielleicht noch einen Weg zum Verhandeln mit 
Kasir 
gab. Auf dem Flur begegnete sie seinem Kammerdiener. Ein hagerer, gebückter Mann mit ergrautem 
Haar. Sarimé schätzte ihn auf das selbe Alter wie ihren Gatten. ``Morris, schläft der Graf noch?''\\
Er verneigte sich schwerfällig und nickte. ``Ja, meine Herrin. Die Reise war sehr anstrengend für 
ihn.''\\
Sarimé runzelte die Stirn. ``Ich werde nach ihm sehen. Seid Ihr auf dem Weg zur Küche?''\\
``Ich wollte Frühstücken, Herrin. Kann ich Euch dienen?''\\
``Lass eine Küchenhilfe Suppe zum Grafen bringen, das wäre alles, danke Morris.''\\
Evins Zimmer war abgedunkelt und stickig. Sarimé schob die Vorhänge zur Seite und öffnete die 
Fensterläden. Die frische Morgenluft, nach Regen duftend, strömte in den Raum. Durch die Wolken 
brachen einzelne Sonnenstrahlen. Sarimé wandte sich ihrem Gatten zu. Er schlief keineswegs. Aber 
richtig wach wirkte er auch nicht. Evin lag begraben unter drei Decken. Schweißperlen glitzerten 
auf seiner wettergegerbten Stirn. Sein glasiger Blick hing weit in der Ferne. Schnell war sie neben 
ihm. ``Evin?''\\
Seine Haut fühlte sich an wie glühende Kohle, als sie ihre Hand auf seine Stirn legte. Erschrocken 
zuckte sie zurück. Sie kannte sich mit Krankheiten oder besser gesagt deren Heilung nicht aus. Sie 
selbst war bisher noch relativ verschont geblieben. Einer ihrer Neffen war ein kränkliches Kind und 
hatte oft mit Fieber mit Bett gelegen. Nach einigen Tagen war bei dem Jungen alles wieder 
überstanden gewesen und er rannte wieder über den Hof. Sarimé zweifelte daran, dass Evin ebenso 
schnell genesen würde. Es klopfte an der Türe. Die junge Gräfin nahm es kaum war. Reglos saß sie 
auf der Bettkante und starrte das fahle Gesicht und die glasigen Augen ihres Gatten an.\\
``Herrin?''\\
Es war die Magd, die sie am gestrigen Tag in die Burg geleitet hatte.\\
``Bring mir die Suppe… und gibt es einen Arzt oder Heilkundigen in der Burg?''\\
Sie trat zögernd näher und stellte das Tablett auf den Beistelltisch. ``Verzeiht… der nächste Arzt 
befindet sich in einem Dorf, eine Stunde entfernt. Ich kann einen Boten aus schicken.''\\
``Mach das.''\\
``Kann ich noch für Euch dienen?''\\
Sarimé zögerte. ``Hat Evin gestern gebadet?''\\
``Ich weiß es nicht, Herrin. Ich richtete das Bad… der Kammerdiener des Herrn wird Eure Frage 
bestimmt beantworten können.''\\
Sarimé nickte. ``Danke… du darfst gehen.''\\
Die Magd verließ den Raum, leise fiel die Tür ins Schloss. Sarimé riss ihren Blick von Evin, stand 
auf und schloss die Fenster. Sie fröstelte und kniete sich vor den Kamin. Sie befreite die 
glimmenden Kohlestücken von der dunklen Asche und entfachte das Feuer neu. \\
Mit mühevollen Überzeugungsversuchen gelang es ihr, Evin ein Stück in die Gegenwart zu holen. Er 
richtete sich auf und lehnte sich schwer gegen ein Kissen, während sie ihm die Suppe einflößte. 
Zwischendurch kam der Kammerdiener, erzählte, dass er einen Reiter zum Arzt geschickt habe und 
verließ den Raum sehr schnell. Generell, bemerkte sie, schienen die Dienstleute es eilig zu haben, 
den Raum zu verlassen. \\
``Sie fürchten sich, dass Ihr sie ansteckt'', sagte Sarimé zum schlafenden Grafen. Er schlief viel 
in den Tagen der Krankheit. Und Sarimé sprach viel. \\
Der Arzt kam und ging wieder, nachdem er keine andere Behandlungsmöglichkeit als Schlaf und warme 
Suppe sah. Einen Moment lang wollte die junge Gräfin aufbegehren, als der Arzt einen Lohn 
verlangte, doch dann gab sie sich geschlagen und trug dem Kammerdiener auf, die gewünschte Anzahl an 
Münzen für ihn zu holen. \\

Sarimé war müde. Sie schlief im Sessel neben Evins Bett, schreckte bei jedem stockenden Atemzug 
oder Hustenanfall auf. Immer wieder musste sie nach ihm sehen. Das Fieber sank nicht, egal wie 
viele feuchte Tücher sie auf seine Stirn legte oder sie das Zimmer lüftete. Die Untätigkeit war das 
Schlimmste. Sie wollte das Zimmer nicht verlassen. Er war ihr Gatte, auch wenn sie ihn nicht 
liebte. Er hatte ihr eine Heimat geboten, auch wenn nur aus Eigennutz. Sie begann zu sticken und zu 
lesen, zwischendurch fütterte sie ihn oder las ihm Briefe vor. Briefe über die Kriegsvorbereitungen. 
Briefe über wirtschaftliche Belange. Besserungswünsche vom Adel. Innerhalb dieser zwei Woche verließ 
sie nur zwei Mal den Raum um sich Bücher zu holen und frisch anzukleiden.\\
Die Tage waren eintönig und Evins Zustand schwankte bedenklich. Nach wenigen Tagen seiner 
Erkrankung sah es so aus, als würde er wieder genesen. Er stand sogar am Fenster und sah auf das 
Treiben im Hof. Doch dann ging es rapide abwärts. Der Graf öffnete kaum noch die Augen. Er aß nichts 
und Sarimé verlor ebenso den Appetit. Übelkeit übermannte sie, auch wenn sie kaum erbrach, da sie 
nicht viel zu sich nahm. Sie fürchtete schon, ebenfalls bald vom Fieber ans Bett gefesselt zu 
werden. Die junge Gräfin steckte gerade zum wiederholten Mal die Nadel in das Tuch und formte ein 
Muster, als die Magd sie ansprach. Sie räumte gerade das Geschirr fort. \\
``Herrin…'', sagte sie vorsichtig.\\
``Hm?'', entgegnete Sarimé, warf einen schnellen Blick auf den schlafenden Grafen und blickte dann 
wieder zu ihrer Stickarbeit.\\
``Verzeiht wenn ich Euch zu nahe trete… aber ich habe mit eurer Zofe gesprochen und… es sind 
mehrere denen es aufgefallen ist…''\\
``Wovon sprichst du?'', fragte Sarimé ungeduldig.\\
``Ihr hab schon lange nicht mehr…'' Sie sah zum Grafen und flüsterte: ``geblutet....''\\
``Du meinst…?'' Sarimé verstummte.\\
Die Magd hatte recht. Sie hatte schon seit etlichen Wochen ihre Regel nicht bekommen. Sie stand 
auf. ``Sorge dafür, dass der Graf nicht alleine ist. Ich gehe spazieren. Ich war viel zu lange 
nicht mehr draußen.''\\

Die Burg besaß einen Garten, der zum größten Teil von Unkraut überwuchert war. Bisher hatte Sarimé 
ihm keine Beachtung geschenkt, doch jetzt trugen ihre Füße sie an diesen Ort. Sogar ein steinerner 
Brunnen stand in der Mitte der kleinen Anlage. Die Pumpe funktionierte jedoch nicht und selbst das 
Regenwasser, welches das Becken aufgefangen hatte, war mit Pflanzen überwuchert.Die wilden Blüten 
und das Gras waren ausgetrocknet und verblüht. Toter Efeu klammerte sich an das Gestein einer 
Statue, als hätte es den Kampf ums Überleben nicht bereits verloren. Alles hier in diesem Garten 
zeugte davon, dass der Herbst mit großen Schritten näher kam. Sarimé trat zu einem älteren Mann, 
der auf einer Bank am Brunnen saß. Er trug eine ausgeblichene Mütze und sein Blick schweifte 
unbeirrt über den Garten. Als würde er hier noch die alten Zeiten sehen, in denen der Garten noch 
gepflegt wurde.\\
``Ihr seid der Gärtner, oder?''\\
Er sah auf, starrte sie einen Moment an und neigte dann langsam den Oberkörper. ``Meine Herrin.''\\
``Warum tust du nicht die Arbeit, für die du bezahlt wirst?''\\
``Tue ich. Ich hege den Gemüsegarten.''\\
Sarimé deutete auf den Ziergarten. ``Und das hier?''\\
Als Antwort zuckte der Gärtner nur mit den Schultern. ``Der Herr hatte nie großes Interesse an dem 
Garten. Seit dem Tod seiner ersten Frau liegt er brach. Man trug mir auf, etwas sinnvolleres zu 
tun.''\\
Die junge Gräfin biss sich nachdenklich auf die Zunge, dann setzte sie sich neben den Mann und 
teilte sein Schweigen. "Wie sah er früher aus?", fragte sie dann leise.\\
Der Gärtner legte den Kopf schief und lächelte verträumt. "Genauso klein. Aber die Herrin wählte so 
viele verschiedene Blumen aus... die Farben kann ich gar nicht beschreiben. Sie war oft hier. Jede 
freie Minute. Und die Kinder auch. Jedes ist mindestens einmal in den Brunnen geklettert, wenn ihre 
Mutter nicht hinsah."\\
Die junge Gräfin lauschte seinen Worten nachdenklich und versuchte dieses Stück verwilderte Erde 
mit seinen Augen zu sehen. \textit{Kinder... Mein Kind könnte hier spielen.}\\
Sie legte ihre Hand auf ihren Bauch und horchte in sich. Sie suchte nach dem neuen Leben in 
sich. ``Wie war die Familie des Grafen?''\\
``Laut'', sagte er nachdenklich und lächelte melancholisch. ``Gräfin Linessa hörte man durch die 
halbe Burg. Sie redete oder sang fast immer. Oder tadelte die Kinder. Sie lud oft zum Tanz und 
geselligen Runden.''\\
``Wen?''\\
``Ihre Freunde. Und die unseres Grafen. Einige Adelige, Kaufleute, Verwandte.''\\
Sarimé versuchte sich Evin bei solchen Gesellschaften vorzustellen. Seine erste Frau, die laut 
lachend Gespräche führte. Wie er selbst auf seinem Stuhl saß, am Bier nippte und sie dabei 
beobachtete, während ein Freund einen Witz erzählte. Zu dieser Zeit muss Evin ein anderer Mensch 
gewesen sein. Ohne die Leere, die der Tod in seinem Herzen hinterlassen hatte. Sarimé fiel es nicht 
schwer, ihren Gatten als so einen Mann sich vorzustellen. ``Und die Söhne?''\\
Der Gärtner lachte leise. ``Oh... der Älteste wäre jetzt ende Zwanzig. Er hätte Euch gefallen, 
Gräfin. Ein kluger junger Mann war er, zeichnete mit Leidenschaft Karten, nachdem er die Strecken 
abgeritten war. Dabei verstellte er sich oft, weil er keine Lust hatte, bei der Bevölkerung 
Aufsehen zu erregen. Er hieß auch Evin. Der Mittlere wollte immer nur Soldat werden. Einmal hat er 
einen Hungerstreik angefangen, weil seine Mutter ihm es verbieten wollte. Da war er zehn. Er hats 
nicht lange ausgehalten! Man fand ihn an einem Morgen eingeschlafen in der Speisekammer, umgeben 
von Brotstücken und weicher Butter. Der Jüngste kam ganz nach Gräfin Linessa. Er folgte seinem 
Herzen, sprach immer aus, was er dachte. Und das war selten höflich. Er starb in der Grafschaft 
Kantor, wollte da die Bastardtochter der Gräfin den Hof machen.''\\
Er seufzte tief und verfiel in Schweigen. Auch Sarimé träumte sich einen Moment weg von der 
Realität und stellte sich die Welt vor, wie sie sein würde, wenn diese Tode nicht über diese 
Familie gekommen wären. Vielleicht wären Evins Sohn General geworden, der Jüngste ein Künstler, 
welche weit fort vom Leben des Adels geblieben wäre. Und vielleicht hätte Sarimé einem jungen, 
Karten zeichnendem Evin gegenüber gestanden, als sie die Kutsche vor dieser Burg verließ. Sie 
dachte an die verstorbene Gräfin, die hier im Garten stand, ihr Lied sang und dabei das Unkraut aus 
der Erde zupfte. Hinter ihr tollten die Söhne über die winzige Grasfläche. Der Älteste saß 
vielleicht auf dieser Bank und las ein Buch?\\
\textit{Mein Kind wird hier spielen.}\\
Es war wie ein feierlicher Beschluss und ein Schwur, dass sie endlich dieses Land als ihre Heimat 
akzeptieren würde. Und als die Heimat ihrer Kinder.\\
``Bereite den Garten für den Winter vor. Schneide die Hecken zurück. Und im Frühling besprechen 
wir, welche Pflanzen im Garten angepflanzt werden sollen.''\\
``Wie Ihr wünscht, Herrin!'', sagte er und ein Lächeln huschte über seine schmalen Lippen.\\
Die Gräfin verabschiedete sich höflich und als sie die Stufen zum Eingangsportal hinaufstieg, 
lächelte sie. Ihre Hand ruhte wieder beiläufig auf ihrem Bauch. \\

Auf ihrem Rückweg eilte sie durch die Gänge. Sie hasste Evin dafür, wie er sie in der 
Hochzeitsnacht und die Male danach genommen hatte. Sie hasste ihn dafür, dass er sie gekauft hatte 
und sie wegen ihm in die Fremde musste. Aber sie wusste auch, wie sehr er sich einen Erben wünschte 
und sie konnte ihn nicht genug hassen, um ihm stillschweigend beim Sterben zuzusehen. Es war ihre 
Pflicht und sein Recht zu erfahren, dass sein Wunsch in Erfüllung gehen könnte, falls das Kind 
überleben würde. Als sie in das Zimmer trat, sah sie den Arzt an Evins Bett stehen. Der Arzt war 
erneut gerufen worden. Ungeduldig, sie wollte mit Evin alleine sein, fragte sie ihm nach dem 
Befinden des Grafen. \textit{Wie oft habe ich diese Frage in den letzten Wochen gestellt? Wie oft 
hat er geantwortet, dass wir beten müssen?}\\
Dies Mal sagte er nichts der gleichen. Stattdessen wagte er es kaum ihr in die Augen zu blicken und 
erwiderte: ``Der Graf ist schon bei Osyma. Bald wird ihm der letzte Lebensfunke folgen.''\\
Als er dieses Mal ging, erhielt er keinen Lohn.\\
``Ich entlohne keine Männer, deren Dienst mir nicht hilft'', waren ihre Abschiedsworte und mit 
energischem Blick schickte sie ihn fort. Danach kniete sie sich neben das Bett und ergriff Evins 
Hand. In den Tagen der Krankheit schien er um Jahre gealtert zu sein. \\
"Du bist wirklich so anders als der Windgeist", keuchte er und sein Mund verzog sich zu einem 
Grinsen.\\
``Ich wäre auch kein guter Windgeist. Jeder Mensch würde fliehen, sobald ich anfange zu singen.''\\
Er lachte, bis das Geräusch in rasselnden Atem überging. Als sein Brustkorb wieder regelmäßiger hob 
und sank sprach er: "Als ich Sieva das erste Mal sah... ach.. sie sah aus wie der stolze Löwe auf 
Saleicas Wappen. Deshalb nahm ich sie mit, weißt du. Ich Narr. Ich verwechselte Stolz mit 
Unnahbarkeit. Erhabenheit mit Abwesenheit. Anfangs... ich habe mich wirklich bemüht. Ich 
vernachlässigte meine Pflichten, um ihre Wünsche zu erfüllen. Um sie zum lächeln zu bringen. So 
viel vergebene Mühe! Und was hatte ich dann davon? Sie zwang mich, Renec an den Hof zu holen. Ich 
tat alles, was sie sich wünschte, weißt du... Aber mein Wunsch wurde nicht erfüllt. Stattdessen 
vögelte sie meinen Bastard! Immerhin hat er es auch nicht geschafft, sie zu schwängern."\\
Seine Rede hatte mit Worten voller Reue begonnen und endete mit Verbitterung.\\
``Mein Graf'', sagte sie eindringlich und drückte fest seine hagere Hand: ``Osyma ist gnädig mit 
Euch. Ich bin schwanger.''\\
In seinem Gesicht zuckte etwas. Vielleicht ein flüchtiges Lächeln? Er schloss die Augen und seine 
Hand wurde schlaff. "Erst so kurz. So viele Leben sterben bereits, bevor sie geboren sind. Ich 
werde nicht da sein um aufzupassen, dass es wirklich geboren wird."\\
Sarimé schluckte und suchte fieberhaft nach Worten. "Ihr könnt Osyma direkt sagen, dass er 
gefälligst dafür zu sorgen hat."\\
Seine Hand hob sich und strich ungeschickt über ihr Haar. "Du willst nicht hier sein. Uns verbindet 
nichts", murmelte er: "Du bist zu jung zum herrschen. Zu jung um alleine Mutter zu sein."\\
"Unterschätzt mich nicht."\\
"Schwöre mir, dass du alles tun wirst, um das Kind zur Welt zu bringen. Schwöre mir, dass du dafür 
sorgen wirst, dass es lebt und überlebt. Dass mein Blut nicht ausstirbt."\\
"Ich schwöre es bei Osymas Flammen und dem Licht, was er uns schenkt."\\
Es war die letzte Nacht, die Sarimé an seiner Seite verbrachte. Am nächsten Vormittag verstarb Graf 
Evin A’Rick, Wächter der Nordgrenze und letzter Lebender Nachfahre einer Familie, die im Aufbau des 
gesamten Reiches eine große Rolle gespielt hatte.\\


Sarimé fand sich als alleinige Herrin der Burg, der ganzen Grafschaft wieder. Zuallererst schickte 
sie den Kammerdiener los, einen Priester zu holen, der sich um Evin kümmern sollte. Anschließend 
zog sie sich in ihr Gemach zurück, um Briefe an die umliegenden Grafen und den hohe Adelige zu 
verfassen. Mägde wies sie an, dass Zimmer zu reinigen, denn während der gesamten Krankheitsdauer 
über war dort, bis auf Sarimés Lüften und Auskehren der Asche im Kamin, nicht geputzt worden. Als 
Witwe überwachte sie selbst, dass Evins Körper mit dem angemessenen Respekt behandelt und die Riten 
eingehalten wurden. Am Abend schlich sie in ihre Zimmer und fand keine Ruhe. Unschlüssig lief sie 
immer wieder durch den Raum und grübelte über ihre Zukunft nach. Angst überkam sie.
\textit{Ich bin dem allen nicht gewachsen. Viel zu jung und unerfahren! Und nun auch noch das 
Kind.. Wer würde mich jetzt noch heiraten wollen, selbst wenn er Graf wird? Wer würde jetzt 
freiwillig Graf Merandilas werden wollen, wenn der Krieg hier toben wird?}\\
Den Tränen nahe kauerte sie auf dem Bett. Die roten Haare verbargen ihr Gesicht, wie ein 
geschlossener Vorhang eine Theaterbühne. Sarimé nahm das Klopfen nicht wahr, aber auch ohne ihre 
Aufforderung betrat der Bastard ihr Zimmer. Vorsichtig öffnete er die Türe und steckte den Kopf 
durch den Spalt. Als Renec die junge Gräfin entdeckte, war er schnell an ihrer Seite und nahm sie 
in den Arm. Die Berührung tat gut, aber sie wich ihr aus. Es ziemte sich nicht für eine Witwe und 
erst recht nicht für eine werdende Mutter, einen anderen Mann zu umarmen. Renec machte keine 
Anstalten die Geste zu wiederholen, sondern setzte sich aufrecht hin und schwieg, bis sie das Wort 
an ihn richtete.\\
``Was passiert nun?'', fragte sie unsicher.\\
Renec erklärte: ``Die Priester werden den Grafen für die Reise zum Meer bereit machen. Er wird dem 
Wind und dem Meer übergeben. Seine letzte Reise, an der Küste nimmt er von unserer Welt 
abschied.''\\
Sarimé sah auf. ``Wird er nicht verbrannt? In Brom-Dalar werden die Toten verbrannt... In 
gesegneten Flammen.''\\
Renec zuckte mit den Schultern. ``Ihr wisst doch, Merandila hat sich viele alte Bräuche behalten. 
Osyma ist der Gott des Feuers und der Ehre. Bevor Merandila eine Grafschaft wurde, verehrten die 
Menschen die Helle.''\\
``Die Helle'', murmelte Sarimé. Sie hatte schon einige Leute diesen Namen aussprechen hören, sich 
aber nie weiter Gedanken darüber gemacht.\\
Renec nickte. ``Die Helle, sagte man, hat viele Gestalten. Eine weißhaarige Frau, ein weißer Wolf, 
Fuchs, Kaninchen oder Vogel. Ein klarer Bach, ein Schneeglöckchen, weiße Kirschblüten. Die Gischt 
des Meeres. Sie kann auch ein Feuer sein, aber eher eine kleine Kerzenflamme als das gleißende 
Inferno Osymas. Sie ist Wind und Meer, hell und zart. Das Licht.''\\
``Eine schwache Göttin'', bemerkte Sarimé, als sie die beiden Götter verglich. \\
Renec kniff die Augen zusammen. ``Auf den ersten Blick vielleicht. Nun, Ihr könnt Evin natürlich 
auch in Osymas Flammen geben. Aber das wäre ein weiterer Beweis für das Volk, dass ihr keine 
Merandil seid.''\\
``Es gibt das Land Merandila nicht mehr'', sagte Sarimé.\\
``In den Herzen der Leute wird es immer ihr Land bleiben. Da kann der König so viele Priester und 
Krieger schicken, wie er will.''\\
``Das Volk verehrt Osyma. Sie beten in seinen Tempeln, sie melden sich bei der Armee, ziehen für 
ihn in den Krieg, ehren seine Lehren und lieben den König. Sie tragen das Banner des Löwen.''\\
``Die Städter, ja, aber schaut Euch die Dörfer an, Herrin. Reitet durch die Landschaft, unterhaltet 
Euch mit den Bauern und Handwerkern. Ich will nicht behaupten, dass sie Osyma nicht ehren, aber 
trotzdem... mit jedem Rind, das sie dem Feuergott opfern, halten sie einen Moment vor einer weißen 
Blume oder einer Kerzenflamme inne. Es steckt zu Tief in ihrem Blut, sie können nicht ändern, was 
sie sind. Sie haben die Helle nicht vergessen.''\\
``Ein Gott oder der Andere, wer schert sich darum?'', schnaufte Sarimé und erhob sich: ``Das 
Einzige was die Leute wollen ist ein Heim, genügend zu Essen, Sicherheit und Ehre. Osyma ist nur 
ihre Rechtfertigung, sich das alles von anderen zu nehmen.''\\
``Einen Unterschied gibt es'', erwiderte Renec und lächelte: ``Die Helle existiert.''\\
Sarimé sah ihn einen langen Moment stumm an. Dann schüttelte sie den Kopf. ``Alles existiert, wenn 
man nur fest genug daran glaubt. Fragt die Priester Osymas, ob ihr Allmächtiger existiert.''\\
Ihre Zurechtweisung traf ihn sichtlich. Renec richtete sich gerade auf und räusperte sich. ``Was 
gedenkt Ihr zu tun, Herrin?'', fragte er trocken.\\
``Zur Küste reisen und Evin deiner Hellen überbringen. Ich bin keine Merandil, du hast Recht. Und 
ich werde nie eine sein. Ich werde bald einen Anderen heiraten, der die Grafschaft führen wird. Er 
kann sich anschließend mit dem Zorn der Priester oder dem Volk auseinandersetzten.''\\
``Und Ihr?''\\
Sarimé legte die Hand auf ihren Bauch. ``Ich will nur ein Heim, genügend zu essen und 
Sicherheit.''\\
Renec folgte ihrer Hand und seine grauen Augen weiteten sich. ``Ihr tragt seinen Erben in Euch?''\\
Sarimé nickte und eine mühsam unterdrückte Träne rann über ihre Wange. Renec sprang auf. ``Warum 
sagtet Ihr nichts? Das ändert alles!''\\
``Nämlich? Dass ich doch nicht ganz unnütz bin, sondern immerhin Kinder gebären kann?''\\
\textit{Falls es überhaupt leben wird...}\\
``Nein! Versteht Ihr nicht? Durch das Kind seid Ihr ein Teil Merandilas. Kein Merandil wird sich 
gegen Euch erheben, wenn dieses Kind leben wird. Durch dieses Kind seid Ihr nicht gezwungen, einen 
Mann, den der König Euch aussucht, zu ehelichen, Herrin. Ihr wärt die alleinige Herrin der 
Grafschaft. A'Rik war ein Mann des Königs. Aber in seinen Adern floss das Blut der letzten freien 
Königin Merandilas. Daher hat sich nie jemand gegen ihn erhoben. Er war eine der letzten, wenigen 
Verbindungen zur Vergangenheit. Zur Freiheit. Auch wenn er selbst nicht mehr als ein Sklave 
Saleicas war.''\\
Sarimé lauschte seinen Worten und beobachtete, wie Renec aufgeregt durch den Raum schritt. 
\textit{So... triumphierend habe ich ihn noch nie erlebt.}\\
Als er endete schüttelte sie den Kopf. ``Das schaffe ich nicht.''\\
``Ich helfe Euch! Ihr werdet es schaffen. Ihr seid schön, beliebt und klug. Zeigt Euch dem Volk. Es 
wird Euch so sehr lieben wie die Bewohner der Burg und Talsmund. Deren Herz habt Ihr erobert, den 
Rest Merandils werdet Ihr auch erlangen.''\\
``Habe ich das?'', fragte sie zweifelnd.\\
``Hört Euch doch um! Schon als ich vorhin mein Pferd in den Stall brachte, erzählten die 
Stallburschen, dass Ihr Evin nicht von der Seite gewichen seid! Wie aufopferungsvoll Ihr Euch um 
ihn gekümmert habt, obwohl die Liebe zwischen euch als Ehepaar nicht sehr offensichtlich gewesen 
war, wenn Ihr versteht, was ich meine. Sie bewundern Euer Pflichtgefühl und dass Ihr Dinge 
erledigt, ohne zu klagen.''\\
Sarimé holte tief Luft. ``Er war mein Mann. Vor Osyma und Zeugen angetraut. Meine Vorstellung von 
Liebe hat sich nicht geändert. Es war... niemand verdient es, einsam zu sterben. Deshalb habe ich 
es ihm auch gesagt. Er ist mit der Gewissheit gestorben, dass er einen rechtmäßigen Erben haben 
wird. Pflichtgefühl... Es gibt eben Dinge, die man tun muss. Hätten seine anderen Frauen das nicht 
für Evin getan?''\\
``Seine Erste vermutlich schon. Das soll wohl wirklich eine Heirat aus Liebe gewesen sein.''\\
``Und Sieva?''\\
Renecs Stirn legte sich in Falten und der Name der verstorbenen Gräfin schien ihn zu verärgern. 
Sarimé konnte in diesem Moment darüber nur den Kopf schütteln. Sein Vater war eben erst gestorben 
und Renec zeigte keinerlei Trauer. Aber auch sein Verhalten auf die Reaktion ihres Namens hatte 
sich geändert.\textit{Was es wohl bedeutet, dass er wütend statt traurig wird, wenn man den Namen 
seiner Geliebten ausspricht?}\\
``Nein, auch Sieva nicht. Sie war, im Gegensatz zu Euch, nicht für das Herrschen bestimmt.''\\
\textit{Herrschen?}, wiederholte Sarimé gedanklich: \textit{Ich herrsche nicht. Ich verwalte das 
Land des Königs. Wenn überhaupt.}\\
Sarimé schüttelte weiter den Kopf und trat auf das Fenster zu. Die Blätter der Bäume glichen Osymas 
Flammen, die sich zornig und aufgebracht gen Himmel reckten. \textit{Weiß er, wovon Renec spricht? 
Sagt der rachsüchtige Gott es seinen Priestern, seinem König? Erzählt er ihnen von den Worten des 
Verrats, die in diesem Raum, in diesem Moment ausgesprochen wurden?}\\
