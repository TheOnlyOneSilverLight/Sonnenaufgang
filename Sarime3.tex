
\chapter{Die Pflicht}

Der Herbst zog ein in die nördlichste Grafschaft Saleicas. Und mit ihm kamen die Stürme. Es wurde 
stiller in der Burg. Eingehüllt in Pelzen und Mänteln huschten die Bediensteten und Wachen über den 
Hof, ohne den Kopf zu heben. Die Tiere wurden unruhig in ihren Ställen und egal welchen Raum Sarimé 
betrat, immer entdeckte sie eine Spur aus Matsch.\\
``Hattet Ihr nie den Wunsch, den kalten Jahreszeiten zu entfliehen?'', fragte Sarimé ihren Gatten 
und spießte eine Kartoffel auf.\\
Sie war nun seit Monaten hier in dieser Burg und sehnte sich nach Gesellschaft. Ab und an 
verbrachten sie die Abende gemeinsam in Salon, vor dem flackernden Kamin. Evin studierte Briefe, sie 
las Bücher. Immerhin hatte sie ihn zu einem Schachspiel überreden können, bis er verlor. Tagsüber 
verbrachte er viele Stunden in seinem Arbeitszimmer und nur zum Abendessen trafen sie sich. Und die 
Nächte.\\
Sarimé hatte keine Anstalten gemacht, in ein gemeinsames Gemach zu ziehen. Sie gehorchte, 
wenn der Graf sie rufen ließ und stahl sich im Dunkeln davon. Noch immer schaffte sie es nicht ohne 
Tränen in diesen Nächten durch die Gänge der Burg zu gehen. Und jedes Mal wartete der Bastard auf 
sie. Und jedes Mal fürchtete sie, er würde nicht dort im Sessel sitzen, mit einem Buch in der Hand, 
in dem er ihr vorlas bis sie einschlief.\\
``Im Winter lassen die Stürme nach'', brummte Evin.``\\
Sarimé blieb eine Weile stumm. Dann fragte sie direkt: ''Warum fällt plötzlich ein halbes Dorf den 
Flammen zum Opfer?``\\
Evin kniff die Augen zusammen. ''Ist das nicht lange genug her um es zu vergessen?``\\
''Ich war damals mit kindischen Gedanken an meine Hochzeit beschäftigt. Aber ich besuchte gestern 
Talsmund und sah den kleinen Heo mit seinen Brandwunden... Habt Ihr den Fall untersuchen 
lassen?``\\
Der Graf atmete rasselnd ein.\\
''Man erzählte mir, dass so etwas gar nicht so selten in Merandila ist``, fügte Sarimé hinzu. Sie 
wollte Evin zu einer Antwort zwingen, aber er erwiederte nur spöttisch: ''Kaum ein paar Monate 
hier, schon habt Ihr Informanten?``\\
''Woher kam das Feuer?``, hackte sie nach.\\
Evin zuckte mit den Schultern. ''In der Regel von einem Funken.``\\
''Warum habt Ihr keine Untersuchungen veranlasst? Weil Ihr den Grund genau kennt? Ich bin Eure 
Gräfin, Evin. Mir stehen Antworten zu!``\\
Der Graf streckte die Hand aus und hielt sie über die Kerzenflamme. ''Euch ist bestimmt nicht 
entgangen, dass die Priester Osymas hier mit den Merandil noch einiges an Arbeit haben. Der 
Hohepriester Na'Rashs, Em'Hir, hat die offizielle Erlaubnis der Krone, Ketzer zu verurteilen und 
zu richten. Einige Priester sind wohl der Meinung, dass Abschreckung mehr bewirkt, als auf Beweise 
für Ketzerei zu warten. Talsmund ist selbst schuld, sie haben ihren Glauben zu offen gelebt.``\\
Sarimé fragte entsetzt: ''Aber das ist Euer Volk! Ihr müsst etwas tun!``\\
''Ich habe keine Beweise``, knurrte Evin und seine Faust schloss sich um die Kerzenflamme: 
''Außerdem würde ich mich dann auf eine Seite stellen. Entweder für Saleica. Oder für mein Volk. 
Kannst du junges Ding dir vorstellen, was das heißt?``\\
''Ihr seid ein Feigling``, flüsterte Sarimé: ''Schreibt dem König von diesen Verbrechen! Klagt die 
Priester an!``\\
''Und dann werden die Bauern zu ihren Mistgabeln greifen, als wütender Mob auf die Priester 
losgehen und sterben. Die Garnisionen sind nicht nur wegen der Grenze hier. Ob König oder 
Hohepriester, die Saleicaner wollen einen Aufstand. Sie wollen einen Grund uns zu bestrafen. Dann 
habe ich kein Volk mehr, was ich beschützen könnte.``\\
''Es muss doch eine Lösung geben``, murmelte Sarimé und starrte auf die Kartoffeln vor ihr.\\
Evins Stimme blieb leise und grimmig: ''Die gibt es. Ich arbeite daran.``\\
Das Mädchen setzte zur Frage an, was genau sein Plan war, als Evins Kammerdiener in das 
Speißezimmer stürzte. Er verneigte sich hastig und sprach: ''Ein Bote, meine Herrschaften. Aus 
der Hauptstadt!``\\
Evin wirkte ehrlich überrascht und musterte den eintretenden, uniformierten Boten skeptisch. Der 
letzte Brief aus der Hauptstadt hatte lediglich Glückwünsche zur Hochzeit enthalten. Sarimé sprang 
auf. Sie war zu neugierig, um die ausschweifende Begrüßung des Boten abzuwarten. Sie nahm dem nach 
alten Schweiß riechenden Mann das Pergament aus den Händen und betrachtete den 
gefalteten Brief und das Siegel aus goldenem Wachs. Das Symbol der gesegneten Flamme. Goldenes Wachs 
war bestimmt für den Glauben und den König. Wobei letzteres einen brüllenden Löwenkopf zeigte.\\
Die junge Gräfin brach das Siegel, faltete das raue Pergament auf und betrachtete die geschwungenen 
Linien. Zum dritten Mal nun überfolg sie die Zeilen, ehe sie den Brief auf den Tisch sinken ließ 
und ihren Gatten stumm ansah. Er biss gerade ein großes Stück Brot ab. ''Hm?``, fragte er mit 
vollem Mund: ''Einladungen? Beschwerden über die letzten Abrechnungen? Oder bekommen wir gar 
mehr Geld?``\\
''Nein``, sagte Sarimé langsam: ''Hohepriester Hisio-Mahar und König Semric verkünden den Krieg im 
Namen des Glaubens und des Allmächtigen.``\\
Evin kaute und schluckte, ehe er fragte: ''Mit einer Kolonie?``\\
Sarimé schwieg einen Moment. Dann sah sie auf. ''Wie viele Tagesritte von hier liegt Kasir?``\\
Der Graf erhob sich abrupt, griff nach dem Pergament und las selbst die Worte des Hohepriesters. 
Unterschrieben vom König persönlich.\\
''Schwachsinn``, fluchte Evin: ''Kasir ist unser Handelspartner. Wir haben einen Waffenstillstand, 
seit Generationen. König Semric ist selbst ein halber Kasira!``\\
Ein weiterer Hustenanfall ließ Evin schwanken. Er ballte seine Hände zu Fäusten und zerknüllte das 
Pergament. ''Krieg des Glaubens``, spuckte er aus: ''Ich zeige euch schon, was ich von eurem 
verfluchten Glauben halte! Lasst mein Pferd satteln!``\\
Der Kammerdiener eilte pflichtbewusst hinaus um den Befehl weiter zu geben. Überrascht schüttelte 
Sarimé den Kopf. ''Evin``, rief sie: ''Was habt Ihr vor? Ihr könnt jetzt nicht reiten! Der 
Sturm tobt. Ihr werdet von einem Blitz erschlagen!``\\
''Ich stopf dem nächsten Hohepriester seine Kriegserklärung ins Maul``, knurrte Evin: ''Und 
der hockt im Tempel in Na'Rash! Krieg, wie stellt der König sich das vor? Im Schneesturm oder was? 
Das Bübchen, was noch nie Monate des Eises gesehen hat, schickt uns gegen ein Volk, welches auf 
Mammuts reitet und der Schnee auf ihren Bergen niemals schmilzt! Es gibt Gründe, wieso 
unsere Könige zuvor einen Frieden anstrebten!``\\
''Lasst mich mitkommen!``, bat Sarimé.\\
Er ignorierte sie und stampfte an ihr vorbei aus dem Speisesaal. Sarimé blieb alleine zurück, 
lauschte dem Heulen des Windes und dem Donnern, nachdem ein Blitz den Nachthimmel erhellte.\\


Drei weitere Tage tobte der Sturm. Dachziegeln der Burgen flogen durch die Luft und zerschellten an 
den steinernen Mauern. Ein Blitz schlug in die alte Eiche ein. Einer ihrer langen Äste löste sich 
und krachte auf das Dach der Stallungen. Zwei Luchse wurden erschlagen, fünf Pferde verschwanden 
hinaus in den Sturm und vier Falken fand man schutzssuchend in den Räumlichkeiten der Burg wieder. 
Am vierten Tag klarrte der Himmel auf. Mit der Unschuld eines Kindes beschien die Morgensonne die 
Hügeln und Täler, als wolle sie verdrängen, was die Winde und Blitze hinterlassen hatten. Am 
darauffolgenden Tag kehrte Evin A'Rik heim. Nicht mehr reitend auf seinem schwarzen Hengst, sondern 
liegend in einem klapprigen Karren, gelenkt von einem Müller und gezogen von einem stämmigem 
Ackergaul. Seine beiden Begleiter ritten mit gesenkten Köpfen auf den Hof. Sarimé erwartete sie 
bereits. Neben ihr der alte Priester der Gegend, der in der ersten Nacht des Sturms an die Tore der 
Burg klopfte und um Einlass bat.\\
Es wurden keine Worte des Grußes getauscht.\\
''Seit wann hat er das Fieber?``, fragte der alte Priester nach einem Blick auf den zitternden 
Grafen.\\
''Vielleicht schon, als er aufbrach``, murmelte Sarimé und lauschte der Erzählung der Wachen, wie 
ihr Graf schon am ersten Abend nach einem wilden Galopp aus dem Sattel fiel und im Matsch landete. 
Die Zeit des Sturms hatten sie in der Mühle ausgeharrt. Evin ging es dabei immer schlechter. 
Seine Flüche verstummten bald, wurden zu Husten und nach Luft ringen, sich in Albträume windend und 
vor Kälte zitternd.\\
''Sobald der Sturm vorbei sich legte, machten wir uns auf den Weg zurück``, erklärte die Wache 
namens Samos.\\
''Bringt ihn in sein Gemach``, ordnete Sarimé an: ''Ihr beide erholt euch. Und Müller, sag, was 
kann ich dir geben um meine Dankbarkeit auszudrücken?``\\
Der Müller zuckte nur mit den Achseln und murmelte kleinlaut: ''Ich diene gern, Herrin.``\\
''Sprich``, forderte Sarimé ungeduldig und beobachtete, wie Evins Wachleute ihn mühsam auf eine 
Trage hoben.\\
''Der Sturm hat einigen Schaden angerichtet``, erklärte der Mann.\\
''Willst du Geld oder Helfer?``, fragte Sarimé.\\
''Zwei Burschen würden reichen.``\\
''Dann geh in die Küche, iss und wärm dich auf. Dort findest du auch die Knechte zur Mittagsstunde. 
Such dir zwei aus. Und gute Reise.``\\

Evin lag begraben unter drei Decken. Schweißperlen glitzerten auf seiner wettergegerbten Stirn. 
Sein glasiger Blick hing weit in der Ferne. Seine Haut fühlte sich an wie glühende Kohle, als sie 
ihre Hand auf seine Stirn legte.  Reglos saß sie auf der Bettkante und starrte das fahle Gesicht und 
die glasigen Augen ihres Gatten an. Der Priester Hochna zerstampfte Kräuter zu seinem Sud und 
murmelte vor sich hin: ''Auf die Welt hab ich ihn geholt. Ihn und alle seine Kinder. Und allen 
habe ich den letzten Segen geweiht. Zwei mal gab ich ihm unter Osymas Himmel eine Frau in die Hand. 
Und dann fällt er mir einfach vom Pferd!``\\
Sarimé sah auf. ''Wieso habt Ihr nicht bei unserer Hochzeit die Zeremonie geleitet?``\\
Er schnaufte nur. ''Ich sagte ihm, nach allem was geschehen ist, dass der Allmächtige andere Pläne 
für ihn hat. Es ist nicht Osymas Wille, dass er ein drittes Weib heiratete. Er war schon immer so 
stur.``\\
''Dann danke ich Osyma, dass der Sturm Euch zurück geführt hat``, murmelte Sarimé und beobachtete, 
wie der Priester einen kühlenden Umschlag anbrachte. ''Flößt ihm den Tee ein. Er muss trinken``, 
befahl er, nachdem er mit seinen vor Alter zitternden Händen nicht in der Lage war, die Tasse an 
Evins trockenen Lippen zu halten.\\
''Was können wir noch tun?``, fragte sie.\\
''Beten.``\\

Sie schlief im Sessel neben Evins Bett, schreckte bei jedem stockenden Atemzug oder Hustenanfall 
auf. Das Fieber sank nicht, egal wie viele feuchte Tücher sie auf seine Stirn legte oder sie das 
Zimmer lüftete. Die Untätigkeit war das Schlimmste. Sie wollte das Zimmer nicht verlassen. Er war 
ihr Gatte, auch wenn sie ihn nicht liebte. Er hatte ihr eine Heimat geboten, auch wenn nur aus 
Eigennutz. Sarimé begann zu sticken und zu lesen, zwischendurch fütterte sie ihn oder las ihm 
Briefe vor. Briefe über die Kriegsvorbereitungen. Briefe über wirtschaftliche Belange. 
Besserungswünsche vom Adel. Innerhalb dieser zwei Wochen verließ sie nur zwei Mal den Raum um sich 
Bücher zu holen und frisch anzukleiden.\\
Die Tage waren eintönig und Evins Zustand schwankte bedenklich. Nach wenigen Tagen seiner 
Erkrankung sah es so aus, als würde er wieder genesen. Er stand sogar am Fenster und sah auf das 
Treiben im Hof. Doch dann ging es rapide abwärts. Der Graf öffnete kaum noch die Augen. Er aß nichts 
und Sarimé verlor ebenso den Appetit. Übelkeit übermannte sie, auch wenn sie kaum erbrach, da sie 
nicht viel zu sich nahm. Sie fürchtete schon, ebenfalls bald vom Fieber ans Bett gefesselt zu 
werden. Die junge Gräfin steckte gerade zum wiederholten Mal die Nadel in das Tuch und formte ein 
Muster, als die Magd sie ansprach. Sie räumte gerade das Geschirr fort. \\
``Herrin…'', sagte sie vorsichtig.\\
``Hm?'', entgegnete Sarimé, warf einen schnellen Blick auf den schlafenden Grafen und blickte dann 
wieder zu ihrer Stickarbeit.\\
``Verzeiht wenn ich Euch zu nahe trete… aber ich habe mit eurer Zofe gesprochen und… es sind 
mehrere denen es aufgefallen ist…''\\
``Wovon sprichst du?'', fragte Sarimé ungeduldig.\\
``Ihr hab schon lange nicht mehr…'' Sie sah zum Grafen und flüsterte: ``geblutet....''\\
``Du meinst…?'' Sarimé verstummte.\\
Die Luft fühlte sich plötzlich viel zu drückend an. Ihre Beine taub und die Stichwunde der Nadel 
pochte. Sie erhob sich, legte vorsichtig ihre Stickarbeit nieder und sah den Schlafenden einen 
langen Moment an. ``Sorge dafür, dass der Graf nicht alleine ist.''\\


Die Burg besaß einen Garten, der zum größten Teil von Unkraut überwuchert war. Bisher hatte Sarimé 
ihm keine Beachtung geschenkt, doch jetzt trugen ihre Füße sie an diesen Ort. Sogar ein steinerner 
Brunnen stand in der Mitte der kleinen Anlage. Die Pumpe funktionierte jedoch nicht und selbst das 
Regenwasser, welches das Becken aufgefangen hatte, war mit Pflanzen überwuchert.Die wilden Blüten 
und das Gras waren ausgetrocknet und verblüht. Toter Efeu klammerte sich an das Gestein einer 
Statue, als hätte es den Kampf ums Überleben nicht bereits verloren. Alles hier in diesem Garten 
zeugte davon, dass der Herbst mit seinen kalten Winden das Land erobert hatte. Sarimé trat zu 
einem älteren Mann, der auf einer Bank am Brunnen saß. Er trug eine ausgeblichene Mütze und sein 
Blick schweifte unbeirrt über den Garten. Als würde er hier noch die alten Zeiten sehen, in denen 
der Garten in voller Blüte stand.\\
``Du bist der Gärtner, oder?''\\
Er sah auf, starrte sie einen Moment an und neigte dann langsam den Oberkörper. ``Meine Herrin.''\\
``Warum tust du nicht die Arbeit, für die du bezahlt wirst?''\\
``Tue ich. Ich hege den Gemüsegarten.''\\
Sarimé deutete auf den Ziergarten. ``Und das hier?''\\
Als Antwort zuckte der Gärtner nur mit den Schultern. ``Der Herr hatte nie großes Interesse daran. 
Seit dem Tod seiner ersten Frau liegt er brach. Man trug mir auf, etwas sinnvolleres zu tun.''\\
Die junge Gräfin biss sich nachdenklich auf die Zunge, dann setzte sie sich neben den Mann und 
teilte sein Schweigen. \\
"Wie sah er früher aus?", fragte sie dann leise.\\
Der Gärtner legte den Kopf schief und lächelte verträumt. "Genauso klein. Aber die Herrin wählte so 
viele verschiedene Blumen aus... die Farben kann ich gar nicht beschreiben. Sie war oft hier. Jede 
freie Minute. Und die Kinder auch. Jedes ist mindestens einmal in den Brunnen geklettert, wenn ihre 
Mutter nicht hinsah."\\
Die junge Gräfin lauschte seinen Worten nachdenklich und versuchte dieses Stück verwilderte Erde 
mit seinen Augen zu sehen. \textit{Kinder... Mein Kind könnte hier spielen.}\\
Sie legte ihre Hand auf ihren Bauch und horchte in sich. Sie suchte nach dem neuen Leben in 
sich. ``Wie war die Familie des Grafen?''\\
``Laut'', sagte er nachdenklich und lächelte melancholisch. ``Gräfin Linessa hörte man durch die 
halbe Burg. Sie redete oder sang fast immer. Oder tadelte die Kinder. Sie lud oft zum Tanz und 
geselligen Runden.''\\
``Wen?''\\
``Ihre Freunde. Und die unseres Grafen. Einige Adelige, Kaufleute, Verwandte.''\\
Sarimé versuchte sich Evin bei solchen Gesellschaften vorzustellen. Seine erste Frau, die laut 
lachend Gespräche führte. Wie er selbst auf seinem Stuhl saß, am Bier nippte und sie dabei 
beobachtete, während ein Freund einen Witz erzählte. Zu dieser Zeit muss Evin ein anderer Mensch 
gewesen sein. Ohne die Leere, die der Tod in seinem Herzen hinterlassen hatte. Sarimé fiel es nicht 
schwer, ihren Gatten als so einen Mann zu sehen. ``Und die Söhne?''\\
Der Gärtner lachte leise. ``Oh... der Älteste wäre jetzt ende Zwanzig. Er hätte Euch gefallen, 
Gräfin. Ein kluger junger Mann war er, zeichnete mit Leidenschaft Karten, nachdem er die Strecken 
abgeritten war. Dabei verstellte er sich oft, weil er keine Lust hatte, bei der Bevölkerung 
Aufsehen zu erregen. Er hieß auch Evin. Der Mittlere wollte immer nur Soldat werden. Einmal hat er 
einen Hungerstreik angefangen, weil seine Mutter ihm es verbieten wollte. Da war er zehn. Er hats 
nicht lange ausgehalten! Man fand ihn an einem Morgen eingeschlafen in der Speisekammer, umgeben 
von Brotstücken und weicher Butter. Der Jüngste kam ganz nach Gräfin Linessa. Er folgte seinem 
Herzen, sprach immer aus, was er dachte. Und das war selten höflich. Er starb in der Grafschaft 
Kantor, wollte da die Bastardtochter der Gräfin den Hof machen.''\\
Er seufzte tief und verfiel in Schweigen. Auch Sarimé träumte sich einen Moment weg von der 
Realität und stellte sich die Welt vor, wie sie sein würde, wenn diese Tode nicht über Evins 
Familie gekommen wären. Vielleicht wäre sein Sohn General geworden, der Jüngste ein Künstler, 
welche weit fort vom Leben des Adels geblieben wäre. Und vielleicht hätte Sarimé einem jungen, 
Karten zeichnendem Evin gegenüber gestanden, als sie die Kutsche vor dieser Burg verließ. Sie 
dachte an die verstorbene Gräfin, die hier im Garten stand, ihr Lied sang und dabei das Unkraut aus 
der Erde zupfte. Hinter ihr tollten die Söhne über die winzige Grasfläche. Der Älteste saß 
vielleicht auf dieser Bank und las ein Buch?\\
\textit{Mein Kind wird hier spielen.}\\
Es war wie ein feierlicher Beschluss und ein Schwur, dass sie endlich dieses Land als ihre Heimat 
akzeptieren würde. Und als die Heimat ihrer Kinder.\\
``Bereite den Garten für den Winter vor. Schneide die Hecken zurück. Und im Frühling besprechen 
wir, welche Pflanzen im Garten angepflanzt werden sollen.''\\
``Wie Ihr wünscht, Herrin!'', sagte er und ein Lächeln huschte über seine schmalen Lippen.\\
Die Gräfin verabschiedete sich höflich und als sie die Stufen zum Eingangsportal hinaufstieg, 
lächelte sie. Ihre Hand ruhte wieder beiläufig auf ihrem Bauch. \\

Auf ihrem Rückweg eilte sie durch die Gänge. Sie hasste Evin dafür, wie er sie in der 
Hochzeitsnacht und die Male danach genommen hatte. Sie hasste ihn dafür, dass er sie gekauft hatte 
und sie wegen ihm in die Fremde musste. Aber sie wusste auch, wie sehr er sich einen Erben wünschte 
und sie konnte ihn nicht genug hassen, um ihm stillschweigend beim Sterben zuzusehen. Es war ihre 
Pflicht und sein Recht zu erfahren, dass sein Wunsch in Erfüllung gehen könnte, falls das Kind 
überleben würde. Als sie in das Zimmer trat, sah sie den Priester Hochna Und Renec an Evins Bett 
stehen. Die beiden Männer verstummten, als sie näher kam. Misstrauisch beobachtete Sarimé die 
Beiden. Priester Hochna verabscheute den Bastard. Und eben dieser hatte sich seit Evins Erkrankung 
nicht mehr in seine Nähe gewagt. Nun lag Renecs Hand auf den Fellen, umschlungen von Evins Griff.
Ungeduldig fragte das Mädchen Hochna nach dem Befinden des Grafen. 
\textit{Wie oft habe ich diese Frage in den letzten Wochen gestellt? Wie oft hat er geantwortet, 
dass wir beten müssen?}\\
Dies Mal sagte er nichts der gleichen. Stattdessen wagte er es kaum ihr in die Augen zu blicken und 
erwiderte: ``Der Graf ist schon bei Osyma. Bald wird ihm der letzte Lebensfunke folgen.''\\
Sarimé nickte nur und blickte zu Boden, als sie den Priester bedeutete, zu gehen. Dann trat sie 
langsam an die Seite des mit Fellen belegten Bettes und kniete nieder. Renec suchte ihrem Blick und 
sah sie fragend an, aber Sarimé schüttelte den Kopf. Sie würde ihn nicht vom Sterbebett seines 
Vaters jagen. Rau und hart drückte das Gestein des Bodens in ihre Knie. Tage hatte sie widerstanden. 
Hatte sie sich geweigert und es allen anderen überlassen. Aber nun suchte sie stumm die richtigen 
Worte für ein Gebet.\\
Evin kam ihr zuvor und flüsterte in die Stille: ``Du bist anders als der Windgeist.''\\
``Ich wäre nie ein guter Windgeist. Jeder Mensch würde fliehen, sobald ich anfange zu singen.''\\
Er lachte, bis das Geräusch in rasselnden Atem überging. Als sein Brustkorb sich wieder 
regelmäßiger hob und sank sprach er: "Als ich Sieva das erste Mal sah... ach.. sie sah aus wie der 
stolze Löwe auf Saleicas Wappen. Deshalb nahm ich sie mit, weißt du. Ich Narr. Ich verwechselte 
Stolz mit Unnahbarkeit. Erhabenheit mit Abwesenheit. Anfangs... ich habe mich wirklich bemüht. Ich 
vernachlässigte meine Pflichten, um ihre Wünsche zu erfüllen. Um sie zum lächeln zu bringen. Ich 
ritt tagelang um Blumen zu finden, die die Farbe ihres Haars hatten. Ich warf das Geld des Königs 
kasirsichen Kaufleuten in den Wolfsrachen um ihr die reinsten Perlen zu kaufen. 
Und was hatte ich dann davon? Sie zwang mich, Renec an den Hof zu holen. Ich tat alles, was sie sich 
wünschte, weißt du... Aber mein Wunsch wurde nicht erfüllt. Stattdessen vögelte sie meinen Bastard! 
Immerhin hast du es auch nicht geschafft, sie zu schwängern!"\\
Seine Rede hatte mit Worten voller Reue begonnen und endete mit Verbitterung, immer wieder von 
Keuchen und trockenem Husten begleitet. Immer noch hielt er Renecs Hand umklammert. Sein Sohn sah 
schweigend zur Seite. Flüchtig sah Sarimé ihn an. Versuchte zu ergründen, ob er es Evin eben selbst 
gebeichtet hatte.\\
``Mein Graf'', sagte sie eindringlich und drückte fest seine hagere Hand: ``Osyma ist gnädig mit 
Euch. Ich bin schwanger.''\\
In seinem Gesicht zuckte etwas. Vielleicht ein flüchtiges Lächeln? Er schloss die Augen und seine 
Hand wurde schlaff. "Erst so kurz. So viele Leben sterben bereits, bevor sie geboren sind. Ich 
werde nicht da sein um aufzupassen, dass es wirklich geboren wird."\\
Sarimé schluckte und suchte fieberhaft nach Worten. "Ihr könnt Osyma direkt sagen, dass er 
gefälligst dafür zu sorgen hat."\\
Seine Hand hob sich und strich ungeschickt über ihr Haar. "Du willst nicht hier sein. Uns verbindet 
nichts", murmelte er: "Du bist zu jung zum herrschen. Zu jung um alleine Mutter zu sein."\\
"Unterschätzt mich nicht."\\
"Schwöre mir, dass du alles tun wirst, um das Kind zur Welt zu bringen. Schwöre mir, dass du dafür 
sorgen wirst, dass es lebt und überlebt. Dass mein Blut nicht ausstirbt."\\
"Ich schwöre es bei Osymas Flammen und dem Licht, das er uns schenkt."\\
Evins Blick zuckte zu seinem Sohn. ``Nein'', kam Sarimé ihm eilig zuvor: ``Es ist Euer Erbe, 
Evin.''\\
``Schwörst du es?'', krächzte er zu Renec.\\
Der Bastard nickte. ``Beim Leben der Hellen.''\\
Es war die letzte Nacht, die Sarimé an seiner Seite verbrachte. Am nächsten Vormittag verstarb Graf 
Evin A’Rick, Wächter der Nordgrenze, der weiße Luchs und letzter Lebender Nachfahre der besiegten 
Königin Merandilas.\\


Sarimé fand sich als alleinige Herrin der Burg, der ganzen Grafschaft wieder. Zuallererst schickte 
sie den Kammerdiener los, einen Priester zu holen, der sich um Evin kümmern sollte. Anschließend 
zog sie sich in ihr Gemach zurück, um Briefe an die umliegenden Grafen und den hohe Adel zu 
verfassen. Und als Witwe überwachte sie selbst, dass Evins Körper mit dem angemessenen Respekt 
behandelt und die Riten eingehalten wurden. Am Abend schlich sie durch ihre Zimmer und fand keine 
Ruhe. Unschlüssig lief sie immer wieder durch den Raum und grübelte über ihre Zukunft nach. Angst 
überkam sie. Den Tränen nahe kauerte sie auf dem Bett. Die roten Haare verbargen ihr Gesicht, wie 
ein geschlossener Vorhang eine Theaterbühne. Sarimé nahm das Klopfen nicht wahr, aber auch ohne 
ihre Aufforderung betrat der Bastard ihr Zimmer. Vorsichtig öffnete er die Tür und steckte den 
Kopf durch den Spalt. Als Renec die junge Gräfin entdeckte, war er schnell an ihrer Seite und nahm 
sie in den Arm. Die Berührung tat gut, aber sie wich ihr aus. Es ziemte sich nicht für eine Witwe 
und erst recht nicht für eine werdende Mutter, einen anderen Mann zu umarmen. Renec machte keine 
Anstalten die Geste zu wiederholen, sondern setzte sich aufrecht hin und schwieg, bis sie das Wort 
an ihn richtete.\\
``Warum bist du nicht bei ihm?'', fragte sie hart.\\
Es galt die Totenwache zu halten um den Körper des Verstorbenen nicht hungrigen Dämonen zu 
überlassen. Mindestens fünf gesegnete Flammen mussten in einem Kreis um den Toten scheinen und kein 
anderes Licht durfte ihn treffen. Osyma selbst urteilte in diesen Stunden über die Seele und dessen 
vergangenem Leben.\\
``Hochna hat mich verjagd.''\\
Einen Moment saßen sie schweigend nebeneinander, bis Renec hinzufügte: ``Ich hätte nicht gedacht, 
dass mein Vater so bald sterben könnte.''\\
``Ich auch nicht... Oh Renec... als ich hier her kam, habe ich ihm den Tod gewünscht!'', schluchzte 
Sarimé auf.\\
Tröstend ergriff er ihre zitternden Hände. ``Glaub mir, da warst du nicht die Einzige.''\\
Sarimé sah ihn vorwurfsvoll an.\\
``Nein'', erwiderte er eilig: ``Ich spreche nicht von mir. Also, es stimmt? Du bist schwanger?''\\
Das Mädchen nickte und eine weitere Träne rann ihr über die Wange. \textit{Wie soll ich das nur 
schaffen?}\\
``Wer weiß noch davon?'', fragte Renec ernst.\\
``Niemand. Ein paar Mägde haben einen Verdacht...'', murmelte sie.\\
Er seufzte. ``Also wissen es bald alle.''\\
Der Bastard grübelte einige Sekunden und starrte an ihr vorbei, ehe er fragte: ``Du stehst noch zu 
deinem Wort? Du weißt schon... wegen den Bränden und Bauern, die als Ketzer verurteilt werden?''\\
Sarimé erinnerte sich noch an jedes Wort, dass sie mit Renec gewechselt hatte, ehe sie Evin am Tag 
seiner Abreise darauf angesprochen hatte. Sie nickte und schluckte schwer.\\
Sein Griff verstärkte sich. ``Du wirst bleiben? Du wirst Gräfin sein und dich für dein Volk 
einsetzen?''\\
Zweifelnd begegnete sie seinem forschenden Blick. Sarimé zögerte. \textit{Welche Wahl habe ich?}\\
``Der König wird jemanden schicken um Evins Platz einzunehmen'', sagte sie leise.\\
``Nein. Nicht, wenn du dich bekennst. Du bist jung, aber es ist dein Recht. Und das deines Kindes. 
Bestehe darauf!''\\
``Ich schaff das nicht'', schluchzte Sarimé auf und hätte sich am liebsten unter den Fellen 
versteckt. Renec hielt sie eisern fest. ``Doch. Du bist nicht allein. Und du hast Evin etwas 
geschworen!''\\
Sie nickte langsam und blinzelte weitere Tränen fort. Ihre Schwäche beschämte sie. ``Wie wird es nun 
weiter gehen? Mit Evin?''\\
Renec ließ sie los und strich sich über das Kinn. ``Du hast die Wahl. Krieche vor Osyma und 
übergebe seinen Körpern den Flammen. Oder Bring ihn zum Meer. Das würde ein Zeichen für die 
Merandil setzen. Ihr Graf würde ins Meer gehen, wie ihre letzte freie Königin.''
\textit{Das klingt romantisch...}\\
``Glaubst du wirklich an diese Helle?'', fragte sie leise.\\
``Da gibt es nichts zu glauben. Die Helle lebt'', antwortete er entschieden: ``Sie zeigte sich den 
Königen Merandils und ritt an deren Seite. Sie hat uns nicht verlassen. Sie ist Wind und Meer, 
Sturm und Gischt. Sie ist Licht und Schatten. Du hörst sie in der Stille und siehst sie in der 
Finsternis.''\\
Sarimé biss sich zweifelnd auf die Unterlippe.\\
``Die Helle lebt, Sarimé. Und sie ist mit uns. Sie will den Merandil ihre Freiheit schenken.''
Sarimé legte die Hand auf ihren Bauch und schwieg.\\
``Du musst die Gelegenheit jetzt ergreifen, Sarimé. Glaub an dich. Die Menschen werden es auch tun. 
Hör dich um, schau sie dir an. Schon jetzt verbreiten sich die Erzählungen, wie du an seiner Seite 
Wache hieltest. Wie pflichterfüllt und aufopferungsvoll du zu deinem Eheschwur standest. Und wie 
sehr werden sie dich erst lieben, wenn sie erfahren, dass sein letzter Wunsch sich erfüllt? Dass er 
einen Erben haben wird! Dass du und sein Kind über Merandila herrschen werdet!''\\
\textit{Herrschen?}, wiederholte Sarimé stumm: \textit{Ich herrsche nicht. Ich verwalte das 
Land des Königs. Wenn überhaupt.}\\
Sarimé schüttelte weiter den Kopf und trat auf das Fenster zu. Die Blätter der Bäume glichen Osymas 
Flammen, die sich zornig und aufgebracht gen Himmel reckten. \textit{Weiß er, wovon Renec spricht? 
Sagt der rachsüchtige Gott es seinen Priestern, seinem König? Erzählt er ihnen von den Worten des 
Verrats, die in diesem Raum, in diesem Moment ausgesprochen wurden?}\\
