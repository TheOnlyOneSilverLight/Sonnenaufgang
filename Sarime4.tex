
\chapter{Neue Bande}

\textit{Es ist vorbei}, dachte Sarimé und es war, als floss alle Kraft aus ihr heraus. Die Reise 
zum Meer war endlich überstanden. Scharen von Menschen waren ihnen für einzelne Wegstunden gefolgt. 
Viele Kilometer wurden schweigend zurück gelegt, auf anderen war sie umringt von Menschen, die ihre 
Anteilnahme ausdrücken wollten. Völlig fremde Menschen hielten ihre Hand, küssten sie auf die 
Wange, streichelten über ihr Haar. Sarimé hatte Frauen und Kinder umarmt, hatte ein Kleinkind, 
welches im Gedränge die Mutter verloren hatte, vor sich in den Sattel gesessen. Und als die Mutter 
sie dann wieder fand, waren sie Seite an Seite, das Pferd führend mit dem Kind im Sattel, den Weg 
weiter geschritten. Es war... fremd gewesen. Und doch war das Gefühl so richtig!\\
\textit{Ich bin keine Fremde}, dachte sie und legte ihre Hand auf ihren Bauch: \textit{Sie wissen 
es.} Woher, konnte Sarimé nicht erahnen. Aber sie hatte Renecs Blick aufgefangen. Er schien daran 
nicht ganz unschuldig zu sein. \\
Als die Reisegesellschaft das Zeltlager aufschlug, war Sarimé wie in Trance. Falls jemand sie 
ansprach, bekam sie es kaum mit. Renec sorgte dafür, dass ihr Zelt gerichtet wurde und sie sich 
einige Stunden ausruhen konnte. Die vielen Eindrücke der Reise und die Sorgen über die Zukunft 
schwirrten durch ihre Gedanken. Auch dann noch, als sie sich für den Abend kleidete und schließlich 
das Zelt verließ.\\

``... unsere gnädige Gräfin...''\\
Sarimé blickte auf. Priester Hochna sah sie fragend an. Sie benötigte einen Moment, um wieder in 
die Gegenwart zurück zu kommen. Die versammelte Gruppe war sehr klein, gerade mal ein Dutzend 
Adelige und deren Bedienstete. Einige Priester und Menschen, die sich als enge Angehörige von Evin 
auszeichneten, waren anwesend. Die Sonne war längst hinter dem Meer verschwunden und ein kühler 
Wind zerrte an Sarimés Gewand. Der Priester, dessen Gesicht und rechter Arm mit ineinander 
verschlungenen tätowierten Mustern überzogen war, hielt eine flackernde Fackel in der Hand. Sarimé 
richtete sich auf und nahm sie entgegen. Achtsam schritt sie zum errichteten Scheiterhaufen. 
Das Feuer knisterte laut, kämpfte gegen den erstickenden Wind an. Desto freudiger sprang es auf das 
trockene Stroh und Geäst über und breitete sich rasch aus. Die Hitze stieg ihr wie ein Faustschlag 
entgegen. Im selben Moment schloss sich eine Hand um ihre Hüfte und zog sie zurück. \\
``Vorsicht Herrin'', mahnte Renec. \\
Während das Feuer weiter tobte, blieb er dicht hinter ihr. Seine Nähe fühlte sich vertraut an und 
sie verspürte noch die Selbe Sehnsucht nach einer Verbindung zwischen ihnen, wie am ersten Tag in 
der Burg. Wie betäubt betrachtete Sarimé die Flammen. Feuer hatte sie stets faszinierend gefunden. 
Der wilde, unbezähmbare Tanz der Flammen, zuckend und gierig um sich greifend. Das Feuer war sich 
seiner Macht und Schönheit bewusst und geizte nicht damit, sich stolz zu präsentieren. Ihm war  
alles gleichgültig, es wusste von der Angst ihm gegenüber, dem Respekt und seiner Notwendigkeit. 
Seit Kindheit an hatte man Sarimé gelehrt, dass das Feuer Osymas liebstes Spielzeug war. Es war die 
Verbundenheit zwischen den Sterblichen und dem tobenden Gott. Aber nein, die Lehren der Priester 
waren ihr stets fremd vorgekommen. Sie konnte sich nicht vorstellen, dass sich das Feuer einem Gott 
beugen würde. Außerdem gierte es nicht stets nach Tod und Schrecken, Ehre und Blut wie der 
Allmächtige. Nein, das Feuer war zu frei und stolz um ein Spielzeug zu sein. Manchmal wirkte es, 
als würde es vor Freude tanzen, ein andermal war die Flamme ruhig und beständig. Ja, Feuer tötete, 
aber es schenkte auch Leben, Wärme, Hoffnung. Das Feuer holte sich alles, was es in die Finger 
bekam, aber nicht aus Bosheit, nicht aus Stolz oder Machtbesessenheit. Das war ihm alles unwürdig. 
Alles gleichgültig.\\
Die Helle, eine stille Göttin. Sie lebt im Kerzenschein, im Mond und im Meer.  Osyma, der 
Allmächtige, der Tobende. Der, der nach Ehre giert. Zwei Götter, die sich angeblich in ein und dem 
Selben Feuer verkörpern können.\\
Am Rande bemerkte sie, wie die ersten Anwesenden unruhig wurden. Es war spät, die Reise lange 
gewesen. Der Scheiterhaufen würde noch stundenlang brennen und die Abschiedsgäste nahmen an, sie 
hätten ihre Anwesenheitspflicht erfüllt. \\
``Herrin?'', ergriff Renec leise das Wort: ``Wir sollten zurück in das Zeltlager.''\\
``Wenn du frierst, dann geh'', erwiderte Sarimé leise: ``Ich bleibe bei meinem Gemahl. Deinem 
Vater!''\\
``Wir sind die letzten Verwandten der Familie A'Rik'', antwortete er nach einem Moment der Stille: 
``Einer von uns sollte sich um die Gäste kümmern.''\\
Das Mädchen spürte seinen warmen Atem im Nacken. ``Dann mach das. Ich werde bei Evin wachen.''\\
\textit{Wie es die Tradition und Pflicht erwartet.}\\
Renec löste die Umarmung und machte den Gästen ein Zeichen, dass man nun zum Lager zurückkehren 
würde. \textit{sie werden auf meine Kosten saufen und speisen, vielleicht sogar tanzen, während hier 
ihr Graf verbrennt und seine Witwe Andacht hält. Das war nun mal die Religion Osymas. Helden und 
Taten voller Ehre waren wichtig, Ausgelassenheit und Leidenschaft. Eine Beerdigung wurde gefeiert 
wie eine Hochzeit, solange der Verstorbene würdevoll gegangen ist.}\\
Nun, würdevoll würde sie sein Abtreten nicht bezeichnen, aber sein Leben mag es vor vielen Jahren 
einmal gewesen sein. Die Helle jedoch soll einst, als sie noch unter den Menschen lebte, gesprochen 
haben: Tod ehrt man mit Stille.\\
\\textit{Mein Volk will die Stille, mein König und meine Adeligen die Leidenschaft.}\\
Sarime seufzte. Seit sie in diese Grafschaft gekommen war, war alles kompliziert geworden. Selbst 
die Beerdigung ihres Gatten war eine Schlacht zwischen zwei Göttern. Sie blieb mit ihren fünf 
Wachen alleine beim Feuer zurück. Bald war nichts mehr zu hören als das beständige Knistern und die 
rhythmischen Wellen des Meeres. Der salzige Wind trug den Geruch des Feuers mit sich. Funken 
stiegen in den Himmel und verglommen wie verlorene Träume - ungehört, unwiederbringlich. \\
``Herrin'', sprach einer der Wachen sie an.\\
Sarimé hatte seinen Namen vergessen. Die letzten Tage waren zu ereignisreich gewesen, da war diese 
Information verschwunden, während sie sich über ihre Zukunft, die Grafschaft, ihre Pflichten und 
das ungeborene Leben in ihr sorgte. Aber sie schenkte ihm ein Lächeln, denn auch wenn sie seinen 
Namen nicht wusste, wusste sie, dass er die letzten Tage fast ununterbrochen in ihrer Nähe gewesen 
war. \\
\textit{ Aber vielleicht sollte ich vorsichtiger werden.}\\
Sie dachte an Renec, seine freundlichen Worte. Er war der Einzige gewesen, der sie hier als neue 
Braut des Grafen und völlig fremdes Mädchen willkommen geheißen hatte. Er war eine Stütze gewesen, 
die sie so dringend benötigt hatte. Ihr lief ein Schauer über den Rücken. \textit{Ist das Liebe?}\\
``Braucht Ihr eine Decke?''\\
``Nein. Später vielleicht'', erwiderte sie: ``Aber ihr könnt euch gerne ebenfalls zurückziehen. Mir 
wird nichts passieren.'' \\ 
Sarimé lachte leise. Es war so albern. Fünf bewaffnete Männer schützten eine schwangere Witwe 
mitten im Nirgendwo, während die Leiche ihres Gatten gerade verbrannte.\\
``Gräfin'', sagte der Mann.\\
\textit{Samos}, erinnerte sich Sarimé wieder.\\
``Ihr wisst genau, dass wir das nicht tun werden'', entgegnete Samos entschieden.\\
``Weil ich euch bezahle'', stellte Sarimé fest.\\
``Unter anderem.''\\


Der Mond zog unbeirrbar seine Bahnen über das Sternenzelt, während die Geräusche des Zeltlagers zu 
Sarimé und den Wachen drang. Das beständige Rauschen der Wellen und das Knistern der Flammen 
versuchten die Musik und die Stimmen zu ersticken, aber es gelang ihnen kaum. Sarimé versuchte es 
zu ignorieren, aber je mehr Zeit verstrich, desto schwieriger wurde es. \\
``Was macht eine Witwe in diesen Stunden normalerweise?'', fragte sie irgendwann – weit nach 
Mitternacht – Samos. Er war der Einzige der fünf Wachen, den man noch als wach bezeichnen konnte. 
Als die ersten Beiden im kalten Sand eingenickt waren, hatte Sarmos sie wütend strafen wollen, doch 
Sarimé hatte ihn mit einer strengen Geste abgehalten.\\
\textit{Morgen, wenn ich wieder unter die Adeligen muss, brauche ich Wachen dringender als jetzt.}\\
``Was meint Ihr, Herrin?''\\
Sarimé zog die Schultern hoch und lehnte sich auf der Suche nach einer bequemeren Position auf den 
Stuhl zurück. Sie hatte Samos nicht widersprochen, als er sie vor wenigen Stunden dazu gedrängt 
hat, sich wenigstens zu setzen. \\
``Im Sinne der Hellen'', fügte sie hinzu und ihr Blick glitt flüchtig hoch zu den Sternen.\\
Samos war ein Merandil, hier geboren und hier würde er sterben, hatte er vor einigen Minuten 
erklärt. Die Merandil hatten – wenn man es genau nahm – einen ähnlich verbohrten Stolz wie die 
Saleicaner, wenn es um ihr Pflichtgefühl ging. Samos erklärte mit einer verschwörerischen Stimme, 
dass ihr Verhalten an Evins Totenbett ihr daher so viel Ansehen in der Burg eingebracht hatte. \\
``An gemeinsame, besinnliche Momente erinnern. Denn Erinnerungen sind das, was nicht einmal Dämonen 
uns nehmen können'', beantwortete er ihre letzte Frage.\\
Sarimé runzelte die Stirn und ihr Blick vertiefte sich wieder in die Flammen. Als suche sie nach 
einem Geheimnis, welches nur das Feuer kannte. Sie hatte keine besinnlichen Stunden mit ihrem 
Mann verbracht. Weder vor noch nach der Hochzeit. Das einzig intime war wohl, als sie ihm von dem 
Erben in ihr erzählt hatte. Aber zu diesem Zeitpunkt hatte der Tod sein Herz bereits umklammert.
Aber andere Momente kamen ihr in den Sinn. Der Tanz auf der Terrasse an ihrem Hochzeitstag. Der 
Kuss, wenige Stunden später in der Nacht, auf ihrem Zimmer. Gespräche und Ausritte. Objektiv 
zerlegte sie jeden dieser Momente in seine winzigsten Einzelteile und versuchte sie zu analysieren, 
während der Mond seinen Weg fortsetzte. Es waren schöne Momente. Bis auf den Kuss vielleicht. Der 
hatte nichts zu bedeuten. Sie grübelte lange und versuchte wieder einmal herauszufinden, was Liebe 
war und was sie wohl falsch machte. Unschlüssig schüttelte sie den Kopf. \textit{Wieso denke ich 
immer wieder darüber nach? Ich habe doch schon entschieden, dass es sie nicht gibt!}\\
Aber warum tat es dann so weh?\\
Samos wandte sich abrupt dem Geräusch der näherkommeden Schritte zu. Wartend lauschte die Gräfin, 
bis der Bastard leise ihren Namen nannte.\\
``Du wirkst überraschend nüchtern'', bemerkte sie und erhob sich um ihre müden Glieder zu 
strecken.\\
Renec stellte sich neben sie und nickte. ``Ich habe das Gefühl, dass ich mich rechtfertigen 
muss.''\\
``Wofür?''\\
Er warf einen Blick zu Samos und schwieg. Der Gräfin war bewusst, was er wollte, aber sie wandte 
den Blick nicht ab. Sie würde den Wachmann nicht fort schicken. Schon alleine aus dem Grund, weil 
Renec es wollte. Es tat gut, zu sehen, wie sein Gesicht vor Unzufriedenheit zuckte.\\
\textit{Meine kleine, bescheidene Macht}, dachte sie und verkniff sich ein Lächeln.\\
``Evin war mir ein Fremder. Er ritt wie so oft an der Grenze entlang, entdeckte meine Mutter und 
schwängerte sie. Wir wissen ja, wie schnell das manchmal passieren kann'', erklärte er sachlich: 
``Das fanden meine Großeltern nicht wirklich toll... Meine Mutter blieb Alleinstehend, denn kein 
Mann im Niemandsland wollte einen Bastard, dessen Herkunft nicht belegt werden konnte. Meine Mutter 
erzählte zwar eifrig, dass der Graf selbst dafür verantwortlich war, aber das änderte nichts. Erst 
als ich zwölf war, fand sie einen Mann. Er akzeptierte mich nur, weil er eine helfende Hand auf 
seinem Hof brauchte. Ich war mir bis zu dem Tag an dem Sieva kam nicht sicher, ob ich ein Kasira 
oder ein Saleicaner war.''\\
``Du meinst Merandil'', verbesserte Sarimé.\\
Renec versteifte sich und warf einen Blick zu Samos. Dieser starrte demonstrativ in die 
Dunkelheit.\\
``Du hast mit dieser Unterteilung angefangen, mach mir nun keinen Vorwurf'', spottete Sarimé: 
``Geht deine Rechtfertigung noch weiter? Nein warte... ich kann es mir denken. Du warst bereits 
fast ein junger Mann, als du ihn kennengelernt hast. Deinen Vater, der sich vorher einen Dreck um 
dich geschert hatte. Immerhin hatte er drei anerkannte Söhne, die er sehr geliebt hatte und nun 
betrauerte. Dabei seid ihr euch so ähnlich gewesen. Du und dein Vater. Es könnte auch an Sieva 
gelegen haben, dass ihr einander nie nahe kamt. Sie stand zwischen euch.''\\
``Was willst du von mir hören?'', fragte Renec leise: ``Dass meine Liebe zu ihr ein Fehler war? Ja, 
das war es. Es war das dümmste, was ich je getan habe.''\\
``Nein. Ich will, dass du dich zusammen reist. Bastarde sind nicht verdammt. Sie haben es in 
Saleica etwas schwieriger als anerkannte Kinder und erhalten kein Erbe, aber auch sie können sich 
Respekt verdienen. Und so weit ich weiß, machen die Merandil überhaupt keinen Unterschied, in 
welcher Beziehung ein Kind gezeugt wurde. Ich will, dass du deine Pflichten Evin gegenüber 
erfüllst. Er war dein Vater und hat dich aufgenommen. Also trauere. Zeige Bedauern und Reue. Bete 
für ihn! Ich bin mir sicher, dass du mehr für ihn empfunden hast als ich. Wenn ich es schaffe, dann 
du auch. Niemand hat es verdient, einsam zu sterben.''\\
``Ich bin wegen dir hier, nicht wegen ihm'', murmelte Renec und griff nach ihrer Hand.\\
Sie erwiderte den Druck. ``Dann mach es für mich.''\\
Er führte Sarimés Hand an seine Lippen und hauchte einen Kuss auf ihre kalte Haut. ``Ich werde mich 
bemühen.''\\


Den fehlenden Schlaf holte Sarimé auf der Reise nach. Sie hatte die Kutsche gewählt, aber über den 
Tag verteilt stieg sich auch einige Stunden in den Sattel, wenn das Wetter nicht zu unangenehm war 
und ihr Körper sich nach Bewegung sehnte. Es verlief alles unscheinbarer und ohne dass die Menschen 
groß auf den Zug aufmerksam wurden. Viele Adelige hatten sich bereits nach der Feier am Meer 
verabschiedet und nun zogen lediglich Sarimé, ihre Wachen und wenigen Bediensteten sowie der Bastard 
mit ihnen. Saß die Gräfin auf ihrer Stute, hielt Samos sich auf seinem Fuchs stets neben ihr. Auch 
wenn sie sich in die Kutsche zurück zog, hörte sie oft seine Stimme, wenn er Befehle erteilte. Am 
morgen des letzten Reisetages lenkte er sein Reittier wieder neben ihres und sprach sie an.
Sarimé hörte ihm schweigend zu und wiederholte dann den Kern seiner Aussage. \\
``Sie wollen mir eine Wacheinheit zusammenstellen? Irre ich mich oder gibt es nicht schon eine, der 
Sie bereits angehören?''\\
``Wir waren die Wachen des Grafen, Herrin.''\\
Sie schmunzelte. ``Und ich bin die Gräfin. Die alleine über die Grafschaft verwaltende Herrin.''\\
\textit{Noch.}\\
Doch Samos Blick blieb ernst. ``Es wird Krieg geben. Viele der Männer der Wacheinheit wollen sich 
zu den Garnisonen aufmachen.''\\
``Sie haben einen Vertrag unterschrieben.''\\
``Den kann man auflösen. Das ist eine reine geschäftliche Angelegenheit.''\\
Sarimé fragte wachsam: ``Und Sie wollen eine Wacheinheit für mich, die keine reine geschäftliche 
Angelegenheit ist? Wie kann ich mir dann sicher sein, dass die Leute die Sie auswählen auch 
wirklich an meiner Seite stehen?''\\
Seine nächsten Worte waren kaum mehr als ein Flüstern. ``Durch einen Eid.''\\
Sie sagte erst nichts und sah ihn nur prüfend von der Seite an. \textit{Nur die Königsfamilie darf 
Männer und Frauen durch einen Eid an sich binden. Das wäre Verrat an der Krone und der Tradition.}\\
Das konnte sie nicht wagen. Aber Samos schien es ernst zu meinen und Sarimé erinnerte sich auch an 
seine Worte in der Nacht, als sie am Scheiterhaufen Wache hielt. ``Sorgen Sie dafür, dass die 
Männer die sich den Garnisonen anschließen wollen sich beim Haushofmeister melden und eine 
sachgerechte Auflösung ihres Beschäftigungsvertrages erhalten'', sagte sie stattdessen: ``Und dann 
dürfen Sie sich nach Nachfolgern umsehen und sie mir vorstellen. Ich kann keinem tapferen 
Saleicanern verbieten für Osyma und ihr Vaterland in den Krieg zu ziehen.''\\
Sie steuerte Rabe an den Wegesrand und brachte sie zum stehen. Während sie aus dem Sattel glitt 
fügte sie noch hinzu: ``Und schicken Sie Renec bitte zu mir.''\\
Sarimé stieg in die Kutsche und ließ die Reisegesellschaft erst weiter ziehen, als Renec seinen 
Wallach einer Wache übergeben hatte und sich zu ihr gesellte. Sarimé hob den schweren Stoff am 
Fenster und spähte hinaus. Langsam zogen Bäume und Sträucher an ihnen vorbei. Ein Pferdeschweif 
fächerte durch ihr Blickfeld. \\
``Vermisst du mich so sehr?'', fragte Renec grinsend.\\
``Was kommt jetzt auf uns zu?'', antwortete Sarimé und ignorierte seine Anspielung.
Doch Renec legte auch nur den Kopf schief und zuckte mit den Schultern. ``Keine Ahnung. Ich war 
bisher auch noch nicht dabei, als eine sechzehnjährige alleinige Gräfin wurde. Das Volk mag dich, 
wie wir bei der Prozession gesehen haben.''\\
``Was habe ich getan, dass es so ist?'', überlegte Sarimé laut.\\
Der Bastard lümmelte sich zurück und versuchte ein bequeme Position auf der schmalen Bank zu 
finden. Schließlich legte er seine Füße rechts von Sarimé auf das Polster um sich zu Strecken. ``Du 
bringst frischen Wind nach Merandila. Hoffen sie. Viele Herrscher werden erst gefeiert, bevor die 
Leute erkennen, dass sie genauso schlimm sind wie die Vorherigen.''\\
Sarimé verdrehte die Augen. ``Tolle Aussicht.''\\
``Den Adel kannst du versuchen zu kaufen'', fuhr Renec fort: ``Und die Priester... spiel ihnen das 
scheue Mädchen vor, dass sie unter ihre Fittiche nehmen. Schlimmstenfalls endest du wie unser 
geliebter König in der Hauptstadt.''\\
``Ich halte nichts davon, Geld zu verschwenden.''\\
Er lachte trocken. ``Das ist keine Verschwendung, sondern Investition in deine Zukunft.''\\
Erneut hob sie den Vorhang um einen Blick auf die Landschaft zu erhaschen - nur um Renecs bohrenden 
Blick auszuweichen. Der Himmel war Wolken verhangen, wie durchgehend in den letzten Tagen, aber 
immerhin regnete es noch nicht. ``Als ich deinen Vater das erste Mal sah, dachte ich nur, dass er 
wohl doch nicht so schnell sterben würde. Er sah noch so gesund aus. Stark.''\\
Renec seufzte tief. ``Die Dinge können sich schnell ändern.''\\
``Aber es ändert sich nicht, dass man ein stattlicher, gesunder Mann war'', entgegnete Sarimé: 
``Was, wenn jemand nachgeholfen hat?''\\
``Du warst die ganze Zeit bei ihm.''\\
Sarimé biss sich grübelnd auf die Zunge. Ihr Blick streifte die Reiter, die ihre Kutsche umgaben. 
``Und wer hätte schon etwas davon?'', sagte sie: ``Du vielleicht?''\\
Renec blieb gelassen und zupfte einen Fusel von seiner ledernen Weste. ``Dann hätte ich es doch 
schon getan, als Sieva noch lebte.''\\
``Um mit ihr gemeinsam die Grafschaft zu verwalten? Was sollte sich an dem Ziel geändert haben? 
Abgesehen von der Frau'', fragte das Mädchen.\\
Renec setzte die Füße wieder auf dem Boden auf und beugte sich zu ihr vor. ``Ich mag dich wirklich 
sehr, Sarimé. Ich bewundere dich, du machst das gut. Wenn man die Umstände berücksichtigt.''\\
``Sieva hast du geliebt'', konterte Sarimé.\\
Der Bastard ließ sich Zeit mit seiner Antwort, betrachtete eingehend ihre Gesichtszüge. Das feuchte 
Haar, der ausweichende Blick. ``Mittlerweile bin ich mir, was das angeht, nicht mehr sicher. Kann 
es wirklich Liebe sein, wenn es auf Schuld, Sorge und Mitgefühl basiert?''\\
Sie presste ihren Kiefer fest zusammen und versuchte sich nicht anmerken zu lassen, wie sehr seine 
Antwort sie verunsicherte. Nervös suchte Sarimé noch nach den richtigen Worten, die sie erwidern 
könnte. Aber da setzte schon Renec zu einer weiteren Frage an. ``Interessant ist doch eher, was dein 
Ziel ist.''\\
Da brauchte sie nun nicht zu überlegen. ``Das hier überstehen.''\\
``Das hier hat nichts mit überstehen zu tun. Das hier ist dein Leben. Ich kann mir kein Szenario 
vorstellen, in dem du ohne einen Kratzer aus der Sache wieder herauskommst. Du könntest im Exil 
enden, mit einem Namen, der alle Ehre verloren hat. Oder als Schatten hinter einem von Priestern 
ausgewählten neuen Gatten. Dein Kind könnte sterben. Du könntest sterben.''\\
``Und Bestenfalls?'', unterbrach Sarimé ihn grob. Ihre Stimme klang schriller, als sie wollte.\\
Renec straffte sich, hob seine Schultern an. Anscheinend hatte er jetzt nach den wenigen Minuten in 
der Kutsche schon Verspannungen im Rückenbereich. ``Führst du dieses Land durch den Krieg. Mit 
Hilfe natürlich. Aber wenn wieder Frieden herrscht, ohne dass Merandila nur noch aus Ruinen 
besteht, können weder die Priester noch der König selbst dir den Titel wieder nehmen. Du weißt 
doch, Saleica ist gut zu seinen Helden. Osyma liebt seine Löwen, die ihm Ehre bringen. Und die 
Bauern lieben es, wenn ihre Felder nicht aus Asche und Salz bestehen.''\\
``Und falls mein Kind nicht lebt?'', flüsterte sie zurück.\\
``Dann könnte das Volk dich immer noch an die Spitze ihres Bürgerkriegs setzen. Falls du ihnen 
weiterhin sympathisch bleibst.''\\
Wieder fehlten Sarimé die Worte, während sie versuchte, die Bedeutung des Gesagten zu 
verinnerlichen. Da erregten Stimmen ihre Aufmerksamkeit. Wachsam hob sie den Stoff des Vorhangs zur 
Seite und versuchte etwas zu erkennen, aber das Geschehen fand außerhalb ihrer Sichtweite statt. 
Die Gräfin erkannte Samos Stimme der laut etwas erwiderte. Und dann fast schon ein Schrei.\\
``Beim Allmächtigen, na lasst ihn doch erst mal ausreden! Eine Frechheit!''\\
Sarimé lachte los. Sie kannte die Stimme. Und sie sollte sich beeilen, sonst würde Samos vielleicht 
bald eine Ohrfeige abbekommen. Das Mädchen riss die Türe auf und rief ``Halt'', aus der noch 
langsam vor sich hin rollenden Kutsche. Die Pferde wieherten, als der Kutscher abrupt die Zügel 
anzog und da war Sarimé auch schon an einer Pfütze vorbei auf die Straße gesprungen. Sie musste sich 
umwenden, denn der Reisezug war für die Gestalten nicht stehen geblieben. Samos und zwei weitere 
Wachen waren einige Meter zurückgefallen und versperrten mit ihren Pferden die Straße. Trotzdem 
erkannte Sarimé die großgewachsene blonde Gestalt. Suja hatte sich nicht verändert. Das dünne Haar 
reichte bis zur Hüfte und wie stets trug sie einen lockeren Zopf damit es kräftiger aussah. Die Frau 
war blass und ihre Augen funkelten. Trotz ihrer durchschnittlichen Erscheinung war Suja für Sarimé 
immer schon wie ein Licht in der Dunkelheit erschienen. Betrat sie den Raum wurde die Atmosphäre 
wärmer. Neben ihr verschwand ihr Mann regelrecht, obwohl er oberflächlich betrachtet attraktiver 
aussah. Sein eingeschüchterter Blick fiel Sarimé gleich auf. Auch seine Haltung, die ausdrückte, 
dass er Samos gehorchen und fort wollte. Die Gräfin freute sich über Sujas Anblick mehr, trotzdem 
begrüßte sie deren Gatten der höflichkeitshalber zuerst. \\
Sie schlüpfte an Samos Pferd vorbei und trat mit erhobenen Kopf auf den Mann – welcher vermutlich 
schon bald sein dreißigstes Lebensjahr erreichen würde – zu. ``Mires!'', sagte sie nur und schloss 
ihn in die Arme. \\
Verhalten erwiderte er die Umarmung. Ihrem Bruder war anzusehen, dass er nicht wusste, wie er 
reagieren sollte. ``Ähm... Werte Gräfin...'' Er hustete.\\
Suja schob ihn zur Seite und ehe Sarimé sich versah, lag sie in deren starken Griff. 
``Ach Mädchen! Lang, lang ist's her. Wie alt bist du jetzt?''\\
Samos räusperte sich. Anscheinend gefiel es ihm nicht, dass seine Herrin direkt angesprochen wurde.
``16'', erwiderte Sarimé und ihr traten Tränen in die Augen. Wer hatte sie das letzte mal besorgt 
angeschaut und dabei nicht an sein eigenes Ziel gedacht? \\
``Wir haben deinen Brief bekommen'', meldete Mires sich wieder zu Wort: ``Und wollten die Einladung 
natürlich nicht ausschlagen.''\\
``Ja, weil mir pleite sind. Weil wir immer schon pleite waren. Ich schätze, das hast du nun hinter 
dir? Der Schmuck ist bestimmt echt.'' Suja zwinkerte ihr zu. \\
Sarimé musste lachen und ließ die Hand ihrer Schwägerin nicht los. Suja stammte aus einer Familie 
von Handwerkern und auch wenn es schien, als würde Mires viel zu oft peinlich berührt sein, wusste 
Sarimé, dass die beiden glücklich miteinander waren. Suja liebte Streit und an wen konnte man das 
besser auslassen als an Mires, dem nie Erwiderungen einfielen?
``Oh, du bist groß geworden!'' Sarimé kniete sich nieder und strahlte ihren ältesten Neffen an. 
Jako war nun sechs und zeigte eine Zahnlücke als er grinste. Er hielt seine zwei Jahre alte 
Schwester an der Hand, die Sarimé nicht kannte und sich schüchtern versteckte. Sarimé drückte ein 
letztes Mal Sujas Hand. ``Wir sprechen uns beim Abendessen. Steigt in die Kuschte, ihr müsst einen 
langen Weg hinter euch haben!''\\
``Wir wollen dir nicht im Weg sein'', murmelte Mires.\\
``Nein, ruht euch aus. Ich wollte die letzten Kilometer reiten.'' An Jako gerichtet fügte sie 
hinzu: ``Na? Willst du mit mir reiten? Siehst du das schwarze Pferd, das ist meines. Sie heißt 
Rabe.''\\
Jako's Schüchternheit verflog und er ließ sofort die Hand seiner Schwester los um Sarimés' zu 
ergreifen. \\


Seit langer Zeit wurde der Speisesaal wieder gedeckt. Die Nachricht das Bruder, Schwägerin, Neffe 
und Nichte der Gräfin angereist waren verbreitete sich wie ein Lauffeuer in der Burg. Die Freude 
war groß und überdeckte die lähmende Stille die seit Evins Erkrankung innerhalb der Festung hing. 
Jako und seine Schwester Milli tobten und rannten durch Gänge. Sie spielten verstecken, aber keiner 
von den beiden schaffte es nicht in Kichern auszubrechen, wenn sie Gemälde sahen die in ihren Augen 
komisch waren. Suja versuchte anfangs ihnen nachzujagen, ließ es dann aber bleiben und plapperte 
strahlend mit einer Zofe die sie zu einem Bad geleiten sollte. Auch Sarimés' Bruder zeigte 
schließlich ein Lächeln. Auch ihn nahm sie ein weiteres Mal in den Arm und hieß ihn in ihrem neuen 
Leben willkommen. Einen Leben, dass sie doch eigentlich nie haben wollte, aber so viel leichter 
schien mit diesen vier Personen. \textit{Meine Familie.}\\
Suja hatte schon recht, sie waren pleite und böse Zungen könnten behaupten, dass sie deshalb hier 
waren. Aber war das wichtig? Sie waren hier und auch wenn Geld eine Rolle spielte, dann nicht nur. 
Sarimé hatte einen Platz in deren Herzen, auch wenn sie das jetzt erst zu erkennen schien. Drei 
Jahre war es her, dass sie ihren Bruder und seine Familie nicht mehr gesehen hatte. Es fühlte sich 
nun richtig an. Abgesehen davon war Mires ein fähiger Kaufmann. Während Sarimé mit beschwingten 
Schritten durch ihre Burg lief überlegte sie schon, wie sie alles einrichten würde.\\
Auf ihr Klopfen hin öffnete sich sogleich die Türe. Renec hatte sich ebenfalls frische Kleidung 
angezogen und sein Haar war noch nass vom waschen. Der Bastard sah sich kurz um ehe er es wagte sie 
wieder direkt anzusprechen. ``Was willst du?''\\
``Wieso bist du sauer?'', entgegnete sie und verschränkte die Arme vor dem Brustkorb. \\
Er schnaufte nur.\\
``Und vielleicht solltest du dich endlich mal entscheiden, ob du mich nun Herrin oder blöde Kuh 
nennst'', feixte sie. Sarimé war zu gut gelaunt um sich reizen zu lassen. \\
``Dann dumme Kuh'', murrte er, aber leise so dass es niemand der zufällig in der Nähe war hören 
konnte.\\
``Das du reicht auch. Ich wollte dich zum Abendessen abholen. Meine Schwägerin braucht vielleicht 
noch einen Moment, weil sie erst noch die Kinder einfangen und baden lassen muss, aber dann bleibt 
noch Zeit etwas zu besprechen.''\\
``Ich soll mit euch essen?'', fragte er überrascht.\\
``Natürlich.'' Sarimé nickte bekräftigend. ``Du gehörst zur Familie. Ich bin deine Stiefmutter!''
Seine Miene blieb grimmig. ``Nein bist du nicht. Ich bin kein A'Rik.''\\
Aber er folgte ihr.\\
``Wie war Sieva's Nachname?''\\
``Heral.''\\
Das Mädchen nickte nur. Der Rest des Weges verlief schweigend und als sie sich an der gedeckten 
Tafel nieder ließen fragte sie: ``Was meinst du dazu, dass ich meinen Bruder die Finanzen über die 
Festung übergebe und Pedan stattdessen woanders einsetze?''\\
``Wo?''\\
``Das ist die Frage. Wo bräuchte die Gräfin von Merandila am dringendsten einen treuen 
Verwalter?''\\
Renec dachte einen Moment darüber nach. ``Na'Rash. Ist die größte Stadt und relativ in der Nähe, 
aber in der Hand der Priester.''\\
``Sie werden widersprechen?'' \\
``Bestimmt. Aber Pedan hat Erfahrung und kann sich durchsetzen. Es ist in den Städten üblich das 
die Verwaltung drei Personen obliegt. Ich müsste mich informieren, welche drei das momentan sind, 
aber ein Priester ist bestimmt dabei. Es gibt das Gesetz, dass mindestens zwei der Vertreter 
jeweils von der Bevölkerung des Stadt und vom Grafen festgelegt werden.''\\
``Das heißt, ich kann einen Posten entscheiden. Bitte, informiere dich wen Evin bestimmt hat. 
Dieser jemand wird vermutlich alles andere als freudig sein, wenn ich ihn ersetze...''\\
``Warten wir ab, Pedan sollte einige Wochen haben um deinen Bruder einzuarbeiten und seine 
Rechnungen zu überprüfen.''\\
``Das kann ich auch'', murmelte Sarimé nachdenklich und drehte eine Haarsträhne auf.\\
Renec seufzte. ``Aber das solltest du nicht. Das ist nicht deine Aufgabe.''\\
``Aber meine Verantwortung.''\\
``Nicht nur für diese Burg. Für das ganze Land!''\\
Das Mädchen hob den Blick. ``Du meinst die Grafschaft.''\\
Renec schwieg kurz, dann nickte er resigniert. ``Ja.''\\
Die Gräfin stützte ihre Ellbogen auf den Tisch und legte ihr Kinn in die Handflächen. ``Und ich 
brauche einen neuen Stallmeister.''\\
Renec schüttelte sofort den Kopf. ``Er ist sein ganzes Leben lang hier und kümmert sich gut um 
den Stall!''\\
``Ich mag den Mann'', erklärte sie versöhnlich: ``Aber ihm geht es nicht gut. Er ist alt.''\\
``Er ist nicht zu alt!''\\
``Es reicht, Bastard!'', rief sie, als Renec ihr erneut ins Wort fiel: ``Er wird sich weiterhin um 
die Raubkatzen und die Falken kümmern. Die Verantwortung über die Pferde wird er abgeben. Er kann 
sie nicht einreiten. Und jetzt stelle ich dir die Frage, ob du mir jemanden empfehlen 
kannst oder ich mich selbst darum kümmern muss!''\\
Verbissen starrte er auf das Silberbesteck. Sarimé strich das Tischtuch glatt und betrachtete die 
Dekoration. Welche Zofe auch immer für das Gedeck zuständig war, sie hatte sie an diesen Abend 
deutlich mehr Mühe gegeben als zu den Zeiten, in denen Sarimé mit ihrem Gatten hier saß. 
``Einige der Pferde in unserem Stall stammen aus einer Zucht hier in Merandila. Es gibt nur eine 
handvoll guter Züchter für Reitpferde und Schlachtrösser.''\\
``Hast du einen Namen im Sinn?''\\
Renec nickte. ``Der jetzige Stallmeister hat ihn öfters erwähnt, aber ich bin dem Mann noch nicht 
begegnet.''\\
``Dann ändere das und stell ihn mir vor.''\\
Es klopfte an der Türe und kurz darauf trat ihre Schwägerin gefolgt von ihrer kleinen Familie ein. 
Sarimé lächelte ihnen zu und deutete einladend auf die Stühle. \\


Die Tage verstrichen viel zu schnell. Sie konnte sich kaum noch vorstellen, dass sie sich vor 
wenigen Wochen innerhalb dieser Mauern noch gelangweilt hatte. Nun fand sie kaum einen Moment der 
Ruhe. Sie las Briefe über die momentane wirtschaftliche und militärische Lage indem sie  
Austausch mit einigen Adeligen und Generälen hielt. Ständige Besprechungen mit Pedan und 
Priester Hochna. Müde nickte Sarimé einige Dinge ab und verbat die anderen. Zwischen diesen Stunden 
kam Samos und stellte ihr einen Soldaten für ihre Leibwache vor. Mit ihm besprach sie auch den 
letzten Brief des Königs, dass ein aufstrebender Kommandant mit seiner Truppe auf den Weg nach 
Merandila war. Vermutlich nur der erste von vielen, aber dieser sollte ihr in militärischen Belangen 
beistehen. \\
``Jozah Mi´ḱae'', murmelte sie und faltete den Brief wieder zusammen.\\
``Noch nie gehört'', schnaupte Samos trotzig.\\
Sarimé musterte ihn mahnend. \textit{Ihm gefällt das nicht.}\\
``Unser König schreibt, dass er in den Kolonien kämpfte und Erfahrungen mit Schlachten und 
Belagerungen hat. Freue dich, dann musst du mir nicht zum fünften Mal beantworten, wie dieser oder 
jener General noch hieß. Dann kannst du dich wieder mehr um die Wachen kümmern.''\\
Er lief rot an und Sarimé wusste nur zu gut, dass gerade ein großer Teil seines Plans durchkreuzt 
wurde. Sollte er doch seine Andeutungen über Verschwörungen machen, aber gerade hatte er seinen 
Unwillen zu offensichtlich gezeigt. Noch duldete sie diese gemurmelten Worte, aber wenn ein 
saleicanischer Kommandant hier unter ihrem Dach war, könnte Samos ihr zum Verhängnis werden. 
Außerdem brauchte sie wirklich Hilfe mit dem Militär. Die Generäle der stehenden Heere respektierten 
ihre junge Gräfin nicht und versuchte nicht einmal diese Tatsache zu verbergen. Samos war eine 
Wache, kein eingeschriebener Soldat der Armee und erst recht kein Kommandant der in den Kolonien 
gedient hatte. Und eine persönliche Empfehlung vom König konnte er auch nicht vorweisen. Sarimé warf 
einen Blick auf die Wanduhr und erhob sich. ``Ich gehe reiten.''\\
Auf dem Weg zur Tür wandte sie sich ihm nocheinmal zu. ``Ohne eine Begleitung!''\\
Sie brauchte Ruhe und Abstand von diesen Männern, die an ihr zerrten. \\

Ihr Wunsch ging jedoch nicht in Erfüllung. Bereits ehe sie das Eingangsportal erreicht hatte, eilte 
Renec auf sie zu und hielt sie auf, in dem er sie am Arm festhielt. Erst wollte sie die Berührung 
mit wirschen Gesten abwenden, doch dann blickte das Mädchen auf in sein Gesicht und hielt inne. Sie 
hatte Renec bisher noch nie so blass und nervös erlebt. \\
``Was ist los?'', flüsterte sie leise.\\
Er warf einen flüchtigen Blick durch den Eingangsraum und murmelte: ``Der Hohepriester aus Na'Rash 
ist hier. Hat er sich zufällig angekündigt und du hast es nicht wichtig genug gefunden, es mir zu 
sagen?''\\
Sarimés Augen weiteten sich überrascht. In jeder anderen Situation hätte sie etwas trotziges 
erwidert, aber dafür schien ihr die Lage zu ernst. Sie schüttelte den Kopf. ``Ich wusste nichts.''\\
Renec sah hektisch über seine Schulter hinaus auf den Hof. ``Man nimmt ihnen gerade ihre Pferde ab. 
Wir gehen hoch.''\\
``Aber werden sie sich nicht beschweren?'', wehrte sie seine drängende Bewegung ab.\\
Verärgert runzelte er die Stirn. ``Vor 20 Jahren noch hätten sie sich vor Grafen auf die Knie 
geworfen!''\\
Sarimé sah den Bastard verwirrt an, ließ sich nun aber widerstandslos die Treppe hinauf schieben. 
Der Wachmann vor ihrem Gemach öffnete schon den Mund um höflich zu grüßen, als Renec sie bereits 
in den Raum stieß. ``Die Gräfin nimmt ein Bad'', erklärte der Bastard: 
``Sie will nicht gestört werden.''\\
Da Renec keine Anstalten machte das Gemach seiner Gräfin wieder zu verlassen, grinste die Wache 
ihnen anzüglich zu. Renec begegnete ihm mit einem eisigem Blick, ehe er die schwere Holztüre zustieß 
und den Schlüssel im Schloss drehte. \\
Sarimé stand irritiert mitten im Raum. ``Was hast du vor?''\\
``Dich aufklären, du dumme Nuss'', seufzte Renec und strich sich über die Stirn. Nachdenklich lief 
er zwei Mal durch den Raum, um dann wieder vor der Tür stehen zu bleiben. ``Du hast sie also nicht 
eingeladen und sie haben dir nicht Bescheid gesagt, dass sie kommen.''\\
Sarimé nickte. ``Ich hatte Briefkontakt mit dem Hohepriester. Gestern erst habe ich eine Antwort 
zurück geschickt. Es ging nur um Evins Tod. Ein paar höfliche Floskeln... Osyma schütze dich, mein 
Kind... und so weiter. Was ist schlimm daran, dass er hier ist?''\\
``Naja...'', rief Renec sarkastisch: ``Das letzte Mal als ein Kind in eine hohe Position erhoben 
wurde, hat ein Hohepriester ihm diese Bürde abgenommen und denkt bis heute nicht daran, dass wieder 
zu ändern!''\\
Sein Tonfall und sein Gehabe schüchterten das Mädchen ein und sie bemühte sich, die Botschaft 
seiner Worte zu entziffern. ``Du sprichst von unserem König?'', wagte sie zu fragen.\\
Renec schnaufte nur. ``Saleica ist nicht mehr das, was es vor 20 Jahren war! Du bist eine Adelige 
aus der Hauptstadt. Sag mir also, was macht der Adel den ganzen Tag lang?''\\
Sarimé zögerte. Man sah den Adel oft auf der Straße, zu jeder Tageszeit. Beim einkaufen, beim 
flanieren, beim feiern. Darauf wollte der Bastard wohl hinaus. ``Mein Vater saß im Rat'', 
antwortete sie mit aufkeimenden Trotz. Sie war anders als diese verwöhnten Mädchen. Ihre Familie 
war schon immer anders gewesen!\\
``Ja'', murrte er: ``Vor 20 Jahren. Der Adel ist heute nichts mehr als ein paar Leute mit viel 
Freizeit, viel Geld und vielen Ahnen. Irgendwelche Vorfahren haben vor vielen, vielen Jahren etwas 
ganz tolles gemacht und darauf berufen sie sich immer noch.''\\
``Mein Vater hat sich hochgearbeitet! König Kareen hat seine Worte sehr geschätzt. Er war einer der 
Kaufleute, die dem König Handelsverbindungen in den Kolonien ermöglichten'', verteidigte sie.\\
Renec hielt in seinem nervösem hin und her laufen inne und blickte sie ernst an. Schüchtern senkte 
Sarimé den Kopf. ``Ja... vor 20 Jahren...''\\
``Saleica wird von den Priestern regiert, Sarimé'', sagte Renec: ``Sie haben das Sagen. Und der 
Adel hat nichts mehr. Er tut nichts mehr. Deshalb rennen sie jetzt auch alle weg. Keiner der 
genügend Geld hat, sich in einer anderen Grafschaft eine neue Existenz aufzubauen, bleibt hier in 
Merandila. Der Krieg naht. Ein Krieg, den die Priester wollen.''\\
``Woher weißt du das alles?'', unterbrach sie.\\
Renec atmete tief aus. ``Von meinem Vater.''\\
Nachdenklich spielte die junge Gräfin mit ihren Hochzeitsband aus zarten Blüten. Trotz ihren neuen 
Status als Witwe hatte sie nicht daran gedacht, es abzulegen. Immerhin war es bisher das einzige 
Sichtbare, was sie mit diesem Land, seinem Volk und seinen Traditionen verband. ``Und nun? Was 
soll ich tun?''\\
``Ich weiß es nicht. Ich dachte nicht, dass sie gleich hier her kommen.''\\
\textit{Ich muss mich zusammen reisen}, dachte sie und sah in den Spiegel um sich zu 
vergewissern, dass sie eine annehmbare Erscheinung abgab. ``Dann höre ich mir mal an, was sie zu 
sagen haben.''\\
``Ich denke, gleich zu ihnen zu kommen wäre zu viel der Ehre.''\\
``Du hast mir gerade versucht zu sagen, wie viel Macht die Priester haben. Vielleicht sollten sie 
erst einmal nicht befürchten, dass mir das egal ist. Bring sie in mein Arbeitszimmer.''\\


Na'Rash war die größte Stadt in Merandila. Sie besaß die größte Schule, die größte Bibliothek, den 
größten Hafen und den größten Tempel. Und hier vor hier saß der Mann, der zumindest inoffiziell 
über all das herrschte. In Vertretung eines Gottes. \\
Der Hohepriester Em'Hir war klein und mager. Sarimé verspürte unweigerlich den Drang, sich gerade 
aufzurichten, nur um zu sehen, ob sie ihn dann vielleicht doch wenige Millimeter überragen würde. 
Es war immerhin ungewohnt, dass sie einem Mann hohen Ranges gegenüberstand, der nicht größer
als sie war. Und es war beängstigend in seine kalten Augen zu blicken. Erst jetzt viel dem 
Mädchen auf, welche Erleichterung es auch sein konnte, wenn die Männer größer waren. Es war 
einfacher ihren Blicken auszuweichen. Vor den Augen des Priesters schien es jedoch kein Entkommen zu 
geben.\\
``Mein Herr'', sprach sie und ihre Hände umschlossen die ausgestreckte Rechte des Priesters. Sie 
neigte leicht ihr Haupt. ``Hättet Ihr mich doch informiert! Es beschämt mich, dass ihr die Burg in 
diesem Zustand seht. Ich hätte Euch Begleiter entgegen geschickt, Euch im Hof empfangen, ein 
Festmahl und einen Gottesdienst vorbereiten lassen!''\\
``Ist gut, Kind'', säuselte der Mann und tätschelte ihre Hände. Die Tätowierungen in seinem Gesicht 
machten es fast unmöglich, seine Mimik zu entziffern. ``Ich weiß doch, welche Sorgen Euch plagen 
und wie überfordert Ihr Euch fühlen müsst, angesichts dieser Fülle an Pflichten und 
Verantwortung.''\\
Sarimé entging nicht, wie Renec - der neben Samos stand - den Mund zusammenkniff. \\
\textit{So ein stolzer Bastard, der die Ehre seiner Gräfin verteidigen will}, dachte sie bissig, 
konzentrierte sich jedoch wieder auf den Hohepriester vor ihr. ``Nein, nein'', rief sie aus: 
``Nicht einmal Zimmer sind hergerichtet! Das darf keine gute Gastgeberin zulassen. Bitte, erlaubt 
mir, Euch und Euren Begleitern im Speisezimmer eine kleine Stärkung anzubieten, bis die Gemächer 
bereit sind. Ihr müsst müde sein von der Reise.''\\
``Übernehmt Euch nicht, werte Dame'', unterbricht Em'Hir sie lächelnd: ``Aufregung ist für Frauen 
nie gut, besonders nicht in gesegneten Umständen. Das ist auch der Grund, wieso ich so schnell 
hier her geeilt bin. Zum einen fühle ich mich leider etwas gekränkt, dass ich nicht bedacht wurde 
bei der Verkündung dieser Neuigkeit.''\\
Sarimé blieb einen Augenblick stumm und sah in die dunklen Augen. Dann fasst sie sich wieder, 
setzte ein schüchternes Lächeln auf und erwiderte: ``Verzeiht... ich habe nicht daran gedacht... 
ich beeilte mich so sehr, unserem König die Nachricht zu übermitteln und dann meinen nächsten 
Familienangehörigen...''\\
Wieder tätschelte er ihr die Hand. ``Ich verstehe schon. Für jede Frau muss die erste 
Schwangerschaft überwältigend sein und zu unüberlegten Handeln führen. Und erst recht bei einem so 
jungen - wenn auch sehr bezaubernden - Mädchen wir Euch! Bevor ich Euer Angebot annehme, will ich 
aber noch schnell den zweiten Grund meiner Anwesenheit offenbaren. Ich halte nicht viel davon, 
solch eine Strecke zurück zu legen und dann noch Tage zu warten, bis man über das eigentliche Thema 
spricht. Ich möchte persönlich den Segen für den Erben der Grafschaft erbitten.''\\
Die letzten Worte sprach er feierlich und hob dabei den Blick zum Himmel, als würde direkt über 
ihnen Osyma persönlich schweben. Sarimé folgte zögernd seinen Augen und sah dann unsicher zur 
Seite, an der Samos und Renec steif standen.\\
``Den Segen'', wiederholte sie stockend: ``Dass... diese Ehre kann ich doch nicht annehmen, Herr. 
Ich plante, dass der Priester unserer Burg den Segen erbitten würde... er diente schon meinem 
Gatten und seiner Familie seit vielen Jahren und ihm würde es viel bedeuten. Er ist schon ein Mann 
hohen Alters und es wird vielleicht der letzte Segen eines A'Riks sein.''\\
Sie wusste, dass ihre Worte nichts ändern würden, aber irgendetwas hatte sie sagen müssen. Dass 
Em'Hir ihr und dem ungeborenen Leben in ihr, den Segen erbitten würde, bedeutete, er würde 
mindestens einige Wochen hier verbringen. Schlimmer konnte sich Sarimé die Situation nur 
vorstellen, wenn der König persönlich gekommen wäre.


