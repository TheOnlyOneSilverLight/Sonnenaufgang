
\chapter{Neue Bande}

Den fehlenden Schlaf holte Sarimé auf der Reise nach. Sie hatte die Kutsche gewählt, aber über den 
Tag verteilt stieg sich auch einige Stunden in den Sattel, wenn das Wetter nicht zu unangenehm war 
und ihr Körper sich nach Bewegung sehnte. Es verlief alles unscheinbarer und ohne dass die Menschen 
groß auf den Zug aufmerksam wurden. Viele Adelige hatten sich bereits nach der Feier am Meer 
verabschiedet und nun zogen lediglich Sarimé, ihre Wachen und wenigen Bediensteten sowie der Bastard 
mit ihnen. Saß die Gräfin auf ihrer Stute, hielt Samos sich auf seinem Fuchs stets neben ihr. Auch 
wenn sie sich in die Kutsche zurück zog, hörte sie oft seine Stimme, wenn er Befehle erteilte. Am 
morgen des letzten Reisetages lenkte er sein Reittier wieder neben ihres und sprach sie an.
Sarimé hörte ihm schweigend zu und wiederholte dann den Kern seiner Aussage. \\
``Sie wollen mir eine Wacheinheit zusammenstellen? Irre ich mich oder gibt es nicht schon eine, der 
Sie bereits angehören?''\\
``Wir waren die Wachen des Grafen, Herrin.''\\
Sie schmunzelte. ``Und ich bin die Gräfin. Die alleine über die Grafschaft verwaltende Herrin.''\\
\textit{Noch. Bis der König mich zu einer Heirat zwingt.}\\
Doch Samos Blick blieb ernst. ``Es wird Krieg geben. Viele der Männer der Wacheinheit wollen sich 
zu 
den Garnisonen aufmachen.''\\
``Sie haben einen Vertrag unterschrieben.''\\
``Den kann man auflösen. Das ist eine reine geschäftliche Angelegenheit.''\\
Sarimé fragte wachsam: ``Und Sie wollen eine Wacheinheit für mich, die keine reine geschäftliche 
Angelegenheit ist? Wie kann ich mir dann sicher sein, dass die Leute die Sie auswählen auch 
wirklich 
an meiner Seite stehen?''\\
Seine nächsten Worte waren kaum mehr als ein Flüstern. ``Durch einen Eid.''\\
Sie sagte erst nichts und sah ihn nur prüfend von der Seite an. \textit{Nur die Königsfamilie darf 
Männer und Frauen durch einen Eid an sich binden. Das wäre Verrat an der Krone und der Tradition.}\\
Das konnte sie nicht wagen. Aber Samos schien es ernst zu meinen und Sarimé erinnerte sich auch an 
seine Worte in der Nacht, als sie am Scheiterhaufen Wache hielt. ``Sorgen Sie dafür, dass die 
Männer die sich den Garnisonen anschließen wollen sich beim Haushofmeister melden und eine 
sachgerechte Auflösung ihres Beschäftigungsvertrages erhalten'', sagte sie stattdessen: ``Und dann 
dürfen Sie sich nach Nachfolgern umsehen und sie mir vorstellen. Ich kann keinem tapferen 
Saleicanern verbieten für Osyma und ihr Vaterland in den Krieg zu ziehen.''\\
Sie steuerte Hexe an den Wegesrand und brachte sie zum stehen. Während sie aus dem Sattel glitt 
fügte sie noch hinzu: ``Und schicken Sie Renec bitte zu mir.''\\
Sarimé stieg in die Kutsche und ließ die Reisegesellschaft erst weiter ziehen, als Renec seinen 
Wallach einer Wache übergeben hatte und sich zu ihr gesellte. Seine Mimik zeigte Distanziertheit 
und Wachsamkeit. Sarimé hob den schweren Stoff am Fenster und spähte hinaus. Langsam zogen Bäume 
und Sträucher an ihnen vorbei. Ein Pferdeschweif fächerte durch ihr Blickfeld. \
``Herrin'', Renec räusperte sich: ``Wie kann ich behilflich sein?''\\
``Ich kann dir nicht vertrauen'', antwortete sie ohne ihn anzublicken.\\
Der Bastard erwiderte darauf nichts.\\
Sie ließ den Vorhang fallen und sah ihm im Halbdunkel an. ``Du bist der Bastard meines verstorbenen 
Gatten. Du bist im Grenzgebiet zu Kasir geboren, sprichst also vermutlich beide Sprachen. Wer sagt 
mir, dass du nicht mit Kasir sympathisierst? Die Frau deines Vaters holte dich an den Hof. Sie 
sorgte dafür, dass Evin dich anerkannte. Sie war eine Kasira. Du warst ihr Geliebter. Sie ist 
gesprungen.'' Ihr Blick war hart, ihre Stimme emotionslos, während sie mit der Aufzählung fortfuhr. 
``Die neue Braut deines Vaters kam an den Hof, du schmeichelst dich bei ihr ein. Dein Vater starb. 
Und nun sitzt du hier.''\\
``Worauf wollt Ihr hinaus?'', fragte er schneidend.\\
``Was ist dein Ziel?''\\
Renecs Augen weiteten sich und ihm schienen die Worte zu fehlen. \\
``Sprich!''\\
Der Bastard senkte den Blick. ``Ich weiß nicht, was Ihr hören wollt. Wie Ihr sagtet, meine Geliebte 
starb.''\\
``Die Gattin deines Vaters.''\\
Als er sie nun wieder ansah, funkelte Zorn in seinen grauen Augen. ``Das hat nie etwas bedeutet.''\\
\textit{Ist das Liebe? Wenn solche Dinge bedeutungslos sind? Gehört der Zorn zur Liebe?
Fasst schon beneidete Sarimé ihn um die Erfahrung geliebt zu haben und geliebt zu werden.}\\
``Was ist Euer Ziel, Gräfin von Merandila?'', erwiderte er kalt.\\
Sarimé brauchte nicht zu überlegen. ``Das hier überstehen.''\\
``Das hier hat nichts mit überstehen zu tun. Das hier ist Euer Leben. Wenn Ihr die Verantwortung 
für 
eine Grafschaft nicht tragen wollt, schreibt an den König er schickt Euch sofort einen neuen 
Verlobten. Oder... oder du lernst endlich zu vertrauen.''\\
``Ich will, dass es meinem Kind gut gehen wird. Ich will, dass es mir gut gehen wird. Und ich will 
keinen fremden Adeligen heiraten, aber welche Wahl habe ich?''\\
Er beugte sich zu ihr vor. Seine Stimme wurde leiser und erfüllte doch den Raum der Kutsche. ``Du 
bist die anerkannte Gräfin. Evin A'Rik, Wächter über die Nordgrenze und Graf, in dessen Adern das 
Blut der letzten Königin Merandilas floss hat dich vor den Augen des Volkes und der Ahnen 
anerkannt. 
Die Priester haben dir ihren Segen gegeben und somit auch auch Osyma der Allmächtige Gott und Vater 
Saleicas dich als Gräfin anerkannt. Der König unter Osymas Gnaden hat der Ehe zugestimmt – auch 
wenn 
es vermutlich nur eine Unterschrift von Dutzenden war, die er diesen Tag geschrieben hatte.''\\
``Nach dem saleicanischem Gesetz kann er mir jeder Zeit einen neuen Verlobten vorstellen.''\\
``Nach den Traditionen der Merandil nicht. Das Kind in dir macht dich zur Merandil und das Volk 
will jemanden ihresgleichen. Solange das Kind lebt kann der König dich nicht zu einer Heirat 
zwingen. Das Volk würde das nicht dulden.''\\
``Und falls es nicht lebt?'', flüsterte sie zurück.\\
``Auch dann wird Merandila keine zwei Saleicaner als Grafenpaar dulden.''\\
Sie schüttelte den Kopf. ``Wieso verdrängst du die Tatsache, dass mein Verlobter auch jemand aus 
Merandila sein könnte? Es gibt keine offizielle Unterteilung mehr. Merandila ist eingegliedert in 
das Land.''\\
``Das denken der König und Saleica.''\\
Ihre grünen Augen betrachteten jede Einzelheit seines Gesichts. ``Geh.'' \\
Renec zögerte einen Moment als würde er auf weitere Worte warten. Aber Sarimé dachte nicht einmal 
daran das gesprochene Wort auch nur zu wiederholen. Sie musste nachdenken.\\
\textit{Lerne zu vertrauen.}\\
Über diesen Vorwurf dachte sie die nächsten Stunden alleine in der Kutsche nach. Ihre Gedanken 
drehten sich im Kreis, während sie herauszufinden versuchte, was Vertrauen bedeutete. Als sie am 
späten Nachmittag zu dem Beschluss kam, dass es anscheinend genauso schwierig war wie Liebe zu 
definieren, erregten Stimmen ihre Aufmerksamkeit. Wachsam hob sie den Stoff des Vorhangs zur Seite 
und versuchte etwas zu erkennen, aber das Geschehen fand außerhalb ihrer Sichtweite statt. Die 
Gräfin erkannte Samos Stimme der laut etwas erwiderte. Und dann fast schon ein Schrei.\\
``Beim Allmächtigen, na lasst ihn doch erst mal ausreden! Eine Frechheit!''\\
Sarimé lachte los. Sie kannte die Stimme. Und sie sollte sich beeilen, sonst würde Samos vielleicht 
bald eine Ohrfeige abbekommen. Das Mädchen riss die Türe auf und rief ``Halt'', aus der noch 
langsam vor sich hin rollenden Kutsche. Die Pferde wieherten, als der Kutscher abrupt die Zügel 
anzog und da war Sarimé auch schon an einer Pfütze vorbei auf die Straße gesprungen. Sie musste sich 
umwenden, denn der Reisezug war für die Gestalten nicht stehen geblieben. Samos und zwei weitere 
Wachen waren einige Meter zurückgefallen und versperrten mit ihren Pferden die Straße. Trotzdem 
erkannte Sarimé die großgewachsene blonde Gestalt. Suja hatte sich nicht verändert. Das dünne Haar 
reichte bis zur Hüfte und wie stets trug sie einen lockeren Zopf damit es kräftiger aussah. Die Frau 
war blass und ihre Augen funkelten. Trotz ihrer durchschnittlichen Erscheinung war Suja für Sarimé 
immer schon wie ein Licht in der Dunkelheit erschienen. Betrat sie den Raum wurde die Atmosphäre 
wärmer. Neben ihr verschwand ihr Mann regelrecht, obwohl er oberflächlich betrachtet attraktiver 
aussah. Sein eingeschüchterter Blick fiel Sarimé gleich auf. Auch seine Haltung, die ausdrückte, 
dass er Samos gehorchen und fort wollte. Die Gräfin freute sich über Sujas Anblick mehr, trotzdem 
begrüßte sie deren Gatten der höflichkeitshalber zuerst. \\
Sie schlüpfte an Samos Pferd vorbei und trat mit erhobenen Kopf auf den Mann – welcher vermutlich 
schon bald sein dreißigstes Lebensjahr erreichen würde – zu. ``Mires!'', sagte sie nur und schloss 
ihn in die Arme. \\
Verhalten erwiderte er die Umarmung. Ihrem Bruder war anzusehen, dass er nicht wusste, wie er 
reagieren sollte. ``Ähm... Werte Gräfin...'' Er hustete.\\
Suja schob ihn zur Seite und ehe Sarimé sich versah, lag sie in deren starken Griff. 
``Ach Mädchen! Lang, lang ist's her. Wie alt bist du jetzt?''\\
Samos räusperte sich. Anscheinend gefiel es ihm nicht, dass seine Herrin direkt angesprochen wurde.
``16'', erwiderte Sarimé und ihr traten Tränen in die Augen. Wer hatte sie das letzte mal besorgt 
angeschaut und dabei nicht an sein eigenes Ziel gedacht? \\
``Wir haben deinen Brief bekommen'', meldete Mires sich wieder zu Wort: ``Und wollten die Einladung 
natürlich nicht ausschlagen.''\\
``Ja, weil mir pleite sind. Weil wir immer schon pleite waren. Ich schätze, das hast du nun hinter 
dir? Der Schmuck ist bestimmt echt.'' Suja zwinkerte ihr zu. \\
Sarimé musste lachen und ließ die Hand ihrer Schwägerin nicht los. Suja stammte aus einer Familie 
von Handwerkern und auch wenn es schien, als würde Mires viel zu oft peinlich berührt sein, wusste 
Sarimé, dass die beiden glücklich miteinander waren. Suja liebte Streit und an wen konnte man das 
besser auslassen als an Mires, dem nie Erwiderungen einfielen?
``Oh, du bist groß geworden!'' Sarimé kniete sich nieder und strahlte ihren ältesten Neffen an. 
Jako war nun sechs und zeigte eine Zahnlücke als er grinste. Er hielt seine zwei Jahre alte 
Schwester an der Hand, die Sarimé nicht kannte und sich schüchtern versteckte. Sarimé drückte ein 
letztes Mal Sujas Hand. ``Wir sprechen uns beim Abendessen. Steigt in die Kuschte, ihr müsst einen 
langen Weg hinter euch haben!''\\
``Wir wollen dir nicht im Weg sein'', murmelte Mires.\\
``Nein, ruht euch aus. Ich wollte die letzten Kilometer reiten.'' An Jako gerichtet fügte sie 
hinzu: ``Na? Willst du mit mir reiten? Siehst du das schwarze Pferd, das ist meines. Sie heißt Hexe, 
weil sie oft so frech ist.''\\
Jako's Schüchternheit verflog und er ließ sofort die Hand seiner Schwester los um Sarimés' zu 
ergreifen. \\


Seit langer Zeit wurde der Speisesaal wieder gedeckt. Die Nachricht das Bruder, Schwägerin, Neffe 
und Nichte der Gräfin angereist waren verbreitete sich wie ein Lauffeuer in der Burg. Die Freude 
war groß und überdeckte die lähmende Stille die seit Evins Erkrankung innerhalb der Festung hing. 
Jako und seine Schwester Milli tobten und rannten durch Gänge. Sie spielten verstecken, aber keiner 
von den beiden schaffte es nicht in Kichern auszubrechen, wenn sie Gemälde sahen die in ihren Augen 
komisch waren. Suja versuchte anfangs ihnen nachzujagen, ließ es dann aber bleiben und plapperte 
strahlend mit einer Zofe die sie zu einem Bad geleiten sollte. Auch Sarimés' Bruder zeigte 
schließlich ein Lächeln. Auch ihn nahm sie ein weiteres Mal in den Arm und hieß ihn in ihrem neuen 
Leben willkommen. Einen Leben, dass sie doch eigentlich nie haben wollte, aber so viel leichter 
schien mit diesen vier Personen. \textit{Meine Familie.}\\
Suja hatte schon recht, sie waren pleite und böse Zungen könnten behaupten, dass sie deshalb hier 
waren. Aber war das wichtig? Sie waren hier und auch wenn Geld eine Rolle spielte, dann nicht nur. 
Sarimé hatte einen Platz in deren Herzen, auch wenn sie das jetzt erst zu erkennen schien. Drei 
Jahre war es her, dass sie ihren Bruder und seine Familie nicht mehr gesehen hatte. Es fühlte sich 
nun richtig an. Abgesehen davon war Mires ein fähiger Kaufmann. Während Sarimé mit beschwingten 
Schritten durch ihre Burg lief überlegte sie schon, wie sie alles einrichten würde. Der jetzige 
Haushofmeister mochte Sarimé und Mires würde sie erst gerne noch eine Weile in ihrer Nähe behalten.
\textit{Lerne zu vertrauen.}\\
Sie schüttelte den Kopf. \textit{Misstraue ich meinem Bruder? Oder will ich ihn einfach bei mir 
haben?}\\
Letzteres war es nicht vollkommen und das wusste Sarimé auch. Auf ihr Klopfen hin öffnete sich 
sogleich die Türe. Renec hatte sich ebenfalls frische Kleidung angezogen und sein Haar war noch 
nass vom waschen. Der Bastard sah sich kurz um ehe er es wagte sie wieder direkt anzusprechen. 
``Was willst du?''\\
``Wieso bist du sauer?'', entgegnete sie und verschränkte die Arme vor dem Brustkorb. \\
Er schnaufte nur.\\
``Und vielleicht solltest du dich endlich mal entscheiden, ob du mich nun Herrin oder blöde Kuh 
nennst'', feixte sie. Sarimé war zu gut gelaunt um sich reizen zu lassen. \\
``Dann dumme Kuh'', murrte er, aber leise so dass es niemand der zufällig in der Nähe war hören 
konnte.\\
``Das du reicht auch. Ich wollte dich zum Abendessen abholen. Meine Schwägerin braucht vielleicht 
noch einen Moment, weil sie erst noch die Kinder einfangen und baden lassen muss, aber dann bleibt 
noch Zeit etwas zu besprechen.''\\
``Ich soll mit euch essen?'', fragte er überrascht.\\
``Natürlich.'' Sarimé nickte bekräftigend. ``Du gehörst zur Familie. Ich bin deine Stiefmutter!''
Seine Miene blieb grimmig. ``Nein bist du nicht. Ich bin kein A'Rik.''\\
Aber er folgte ihr.\\
``Wie war Sieva's Nachname?''\\
``Heral.''\\
Das Mädchen nickte nur. Der Rest des Weges verlief schweigend und als sie sich an der gedeckten 
Tafel nieder ließen fragte sie: ``Was meinst du dazu, dass ich meinen Bruder die Finanzen über die 
Festung übergebe und Penal stattdessen woanders einsetze?''\\
``Wo?''\\
``Das ist die Frage. Wo bräuchte die Gräfin von Merandila am dringendsten einen treuen 
Verwalter?''\\
Renec dachte einen Moment darüber nach. ``Na'Rash. Ist die größte Stadt und relativ in der Nähe, 
aber in der Hand der Priester.''\\
``Sie werden widersprechen?'' \\
``Bestimmt. Aber Penal hat Erfahrung und kann sich durchsetzen. Es ist in den Städten üblich das 
die Verwaltung drei Personen obliegt. Ich müsste mich informieren, welche drei das momentan sind, 
aber ein Priester ist bestimmt dabei. Es gibt das Gesetz, dass mindestens zwei der Vertreter 
jeweils von der Bevölkerung des Stadt und vom Grafen festgelegt werden.''\\
``Das heißt, ich kann einen Posten entscheiden. Bitte, informiere dich wen Evin bestimmt hat. 
Dieser jemand wird vermutlich alles andere als freudig sein, wenn ich ihn ersetze...''\\
``Warten wir ab, Penal sollte einige Wochen haben um deinen Bruder einzuarbeiten und seine 
Rechnungen zu überprüfen.''\\
``Das kann ich auch'', murmelte Sarimé nachdenklich und drehte eine Haarsträhne auf.\\
Renec seufzte. ``Aber das solltest du nicht. Das ist nicht deine Aufgabe.''\\
``Aber meine Verantwortung.''\\
``Nicht nur für diese Burg. Für das ganze Land!''\\
Das Mädchen hob den Blick. ``Du meinst die Grafschaft.''\\
Renec schwieg kurz, dann nickte er resigniert. ``Ja.''\\
Die Gräfin stützte ihre Ellbogen auf den Tisch und legte ihr Kinn in die Handflächen. ``Das Tor ist 
zu klein'', wechselte sie das Thema.\\
``Welches Tor?'', fragte der Bastard verwirrt.\\
Auch seine Körperhaltung war nun ihr zugeneigt und sie erkannte deutlich, dass er nicht wusste, ob 
sie schon wieder scherzte. ``Es passt nicht einmal eine Kutsche hindurch. Das muss geändert 
werden.''\\
``Wofür?''\\
``Für Kutschen. Es ist so lächerlich, dass sie ihre Insassen erst vor das Tor fahren, dann fährt 
die Kutsche nach Talsmund, wird dort untergestellt und der Kutscher muss die Pferde wieder zur Burg 
bringen.''\\
``Der Kutscher lebt dort unten und besagte Pferde auch.''\\
Sarimé hob eine Augenbraue. ``Das Tor wird vergrößert werden. Schickst du mir einen Baumeister oder 
soll ich jemand anderen fragen?''\\
Er hielt ihnen Blick stand und erwiderte: ``Ich kümmere mich darum.''\\
Das Mädchen nickte bekräftigend. ``Gut. Und gleich einen Stallmeister.''\\
Renec schüttelte sofort den Kopf. ``Nein, wieso? Er ist sein ganzes Leben lang hier und kümmert 
sich exzellent um den Stall!''\\
\textit{Und du hast anscheinend eine enge Beziehung zu ihm.}\\
``Ich mag den Mann'', erklärte sie versöhnlich: ``Aber ihm geht es nicht gut. Er ist alt.''\\
``Er ist nicht zu alt!''\\
``Es reicht, Bastard!'', rief sie, als Renec ihr erneut ins Wort fiel: ``Er wird sich weiterhin um 
die Raubkatzen und die Falken kümmern. Die Verantwortung über die Pferde wird er abgeben. Er kann 
sie nicht einreiten. Und auch jetzt stelle ich dir wieder die Frage, ob du mir jemanden empfehlen 
kannst oder ich mich selbst darum kümmern muss!''\\
Verbissen starrte er auf das Silberbesteck. Sarimé strich das Tischtuch glatt und betrachtete die 
Dekoration. Welche Zofe auch immer für das Gedeck zuständig war, sie hatte sie an diesen Abend 
deutlich mehr Mühe gegeben als zu den Zeiten, in denen Sarimé mit ihrem Gatten hier saß. 
``Einige der Pferde in unserem Stall stammen aus einer Zucht hier in Merandila. Es gibt nur eine 
handvoll guter Züchter für Reitpferde und Schlachtrösser.''\\
``Hast du einen Namen im Sinn?''\\
Renec nickte. ``Der jetzige Stallmeister hat ihn öfters erwähnt, aber ich bin dem Mann noch nicht 
begegnet.''\\
``Dann ändere das und stell ihn mir vor. Am Besten soll er gleich ein paar seiner Tiere vorführen, 
damit ich mir ein Bild machen kann.''\\
Es klopfte an der Türe und kurz darauf trat ihre Schwägerin gefolgt von ihrer kleinen Familie ein. 
Sarimé lächelte ihnen zu und deutete einladend auf die Stühle. \\


Die Tage verstrichen viel zu schnell. Sie konnte sich kaum noch vorstellen, dass sie sich vor 
wenigen Wochen innerhalb dieser Mauern noch gelangweilt hatte. Nun fand sie kaum einen Moment der 
Ruhe. Sie las Briefe über die momentane wirtschaftliche und militärische Lage indem sie  
Austausch mit einigen Adeligen und Generälen hielt. Die Besprechungen mit dem Baumeister über 
das geplante Tor war ermüdend, denn er drängte zu immer weiteren Baumaßnahmen an der Burg. Immer 
wieder sprach er an, dass in einem Jahr ein kasirisches Heer hier vor der Mauer stehen könnte und 
alles was nur möglich war verstärkt werden müsste. Müde nickte Sarimé einige Dinge ab und verbat die 
anderen, die ihr dann doch zu unsinnig vorkamen. Immerhin musste sie mit dem Geld das sie vom König 
bekam haushalten. Zwischen diesen Stunden kam Samos und stellte ihr einen Soldaten für ihre 
Leibwache vor. Mit ihm besprach sie auch den letzten Brief des Königs, dass ein aufstrebender 
Kommandant mit seiner Truppe auf den Weg nach Merandila war. Vermutlich nur der erste von vielen, 
aber dieser sollte ihr in militärischen Belangen beistehen. \\
``Jozah Mi´ḱae'', murmelte sie und faltete den Brief wieder zusammen.\\
``Noch nie gehört'', schnaupte Samos trotzig.\\
Sarimé musterte ihn mahnend. \textit{Ihm gefällt das nicht.}\\
``Unser König schreibt, dass er in den Kolonien kämpfte und Erfahrungen mit Schlachten und 
Belagerungen hat. Freue dich, dann musst du mir nicht zum fünften Mal beantworten, wie dieser oder 
jener General noch hieß. Dann kannst du dich wieder mehr um die Wachen kümmern.''\\
Er lief rot an und Sarimé wusste nur zu gut, dass gerade ein großer Teil seines Plans durchkreuzt 
wurde. Sollte er doch seine Andeutungen über Verschwörungen machen, aber gerade hatte er seinen 
Unwillen zu offensichtlich gezeigt. Noch duldete sie diese gemurmelten Worte, aber wenn ein 
saleicanischer Kommandant hier unter ihrem Dach war, könnte Samos ihr zum Verhängnis werden. 
Außerdem brauchte sie wirklich Hilfe mit dem Militär. Die Generäle der stehenden Heere respektierten 
ihre junge Gräfin nicht und versuchte nicht einmal diese Tatsache zu verbergen. Samos war eine 
Wache, kein eingeschriebener Soldat der Armee und erst recht kein Kommandant der in den Kolonien 
gedient hatte. Und eine persönliche Empfehlung vom König konnte er auch nicht vorweisen. Sarimé warf 
einen Blick auf die Wanduhr und erhob sich. ``Ich gehe reiten.''\\
Auf dem Weg zur Tür wandte sie sich ihm nocheinmal zu. ``Ohne eine Begleitung!''\\
Sie brauchte Ruhe und Abstand von diesen Männern, die an ihr zerrten. \\

Ihr Wunsch ging jedoch nicht in Erfüllung. Bereits ehe sie das Eingangsportal erreicht hatte, eilte 
Renec auf sie zu und hielt sie auf, in dem er sie am Arm festhielt. Erst wollte sie die Berührung 
mit wirschen Gesten abwenden, doch dann blickte das Mädchen auf in sein Gesicht und hielt inne. Sie 
hatte Renec bisher noch nie so blass und nervös erlebt. \\
``Was ist los?'', flüsterte sie leise.\\
Er warf einen flüchtigen Blick durch den Eingangsraum und murmelte: ``Der Hohepriester aus Na'Rash 
ist hier. Hat er sich zufällig angekündigt und du hast es nicht wichtig genug gefunden, es mir zu 
sagen?''\\
Sarimés Augen weiteten sich überrascht. In jeder anderen Situation hätte sie etwas trotziges 
erwiedert, aber dafür schien ihr die Lage zu ernst. Sie schüttelte den Kopf. ``Ich wusste 
nichts.''\\
Renec sah hektisch über seine Schulter hinaus auf den Hof. ``Man nimmt ihnen gerade ihre Pferde ab. 
Wir gehen hoch.''\\
``Aber werden sie sich nicht beschweren?'', wehrte sie seine drängende Bewegung ab.\\
Verärgert runzelte er die Stirn. ``Vor 20 Jahren noch hätten sie sich vor Grafen auf die Knie 
geworfen!''\\
Sarimé sah den Bastard verwirrt an, ließ sich nun aber widerstandslos die Treppe hinauf schieben. 
Der Wachmann vor ihrem Gemach öffnete schon den Mund um höflich zu grüßen, als Renec sie bereits 
in den Raum stieß. ``Die Gräfin nimmt ein Bad'', erklärte der Bastard: 
``Sie will nicht gestört werden.''\\
Da Renec keine Anstalten machte das Gemach seiner Gräfin wieder zu verlassen, grinste die Wache 
ihnen anzüglich zu. Renec begegnete ihm mit einem eisigem Blick, ehe er die schwere Holztüre zustieß 
und den Schlüssel im Schloss drehte. \\
Sarimé stand irritiert mitten im Raum. ``Was hast du vor?''\\
``Dich aufklären, du dumme Nuss'', seufzte Renec und strich sich über die Stirn. Nachdenklich lief 
er zwei Mal durch den Raum, um dann wieder vor der Tür stehen zu bleiben. ``Du hast sie also nicht 
eingeladen und sie haben dir nicht bescheid gesagt, dass sie kommen.''\\
Sarimé nickte. ``Ich hatte Briefkontakt mit dem Hohepriester. Gestern erst habe ich eine Antwort 
zurück geschickt. Es ging nur um Evins Tod. Ein paar höfliche Flokseln... Osyma schütze dich, mein 
Kind... und so weiter. Was ist schlimm daran, dass er hier ist?''\\
``Naja...'', rief Renec sarkastisch: ``Das letzte Mal als ein Kind in eine hohe Position erhoben 
wurde, hat ein Hohepriester ihm diese Bürde abgenommen und denkt bis heute nicht daran, dass wieder 
zu ändern!''\\
Sein Tonfall und sein Gehabe schüchterten das Mädchen ein und sie bemühte sich, die Botschaft 
seiner Worte zu entziffern. ``Du sprichst von unserem König?'', wagte sie zu fragen.\\
Renec schnaufte nur. ``Saleica ist nicht mehr das, was es vor 20 Jahren war! Du bist eine Adelige 
aus der Hauptstadt. Sag mir also, was macht der Adel den ganzen Tag lang?''\\
Sarimé zögerte. Man sah den Adel oft auf der Straße, zu jeder Tageszeit. Beim einkaufen, beim 
flanieren, beim feiern. Darauf wollte der Bastard wohl hinaus. ``Mein Vater saß im Rat'', 
antwortete sie mit aufkeimenden Trotz. Sie war anders als diese verwöhnten Mädchen. Ihre Familie 
war schon immer anders gewesen!\\
``Ja'', murrte er: ``Vor 20 Jahren. Der Adel ist heute nichts mehr als ein paar Leute mit viel 
Freizeit, viel Geld und vielen Ahnen. Irgendwelche Vorfahren haben vor vielen, vielen Jahren etwas 
ganz tolles gemacht und darauf berufen sie sich immer noch.''\\
``Mein Vater hat sich hochgearbeitet! König Kareen hat seine Worte sehr geschätzt. Er war einer der 
Kaufleute, die dem König Handelsverbindungen in den Kolonien ermöglichten'', verteidigte sie.\\
Renec hielt in seinem nervösem hin und her laufen inne und blickte sie ernst an. Schüchtern senkte 
Sarimé den Kopf. ``Ja... vor 20 Jahren...''\\
``Saleica wird von den Priestern regiert, Sarimé'', sagte Renec: ``Sie haben das Sagen. Und der 
Adel hat nichts mehr. Er tut nichts mehr. Deshalb rennen sie jetzt auch alle weg. Keiner der 
genügend Geld hat, sich in einer anderen Grafschaft eine neue Existenz aufzubauen, bleibt hier in 
Merandila. Der Krieg naht. Ein Krieg, den die Priester wollen.''\\
``Woher weißt du das alles?'', unterbrach sie.\\
Renec atmete tief aus. ``Von meinem Vater.''\\
Nachdenklich spielte die junge Gräfin mit ihren Hochzeitsband aus zarten Blüten. Trotz ihren neuen 
Status als Witwe hatte sie nicht daran gedacht, es abzulegen. Immerhin war es bisher das einzige 
Sichtbare, was sie mit diesem Land, seinem Volk und seinen Traditionen verband. ``Und nun? Was 
soll ich tun?''\\
``Ich weiß es nicht. Ich dachte nicht, dass sie gleich hier her kommen.''\\
\textit{Ich muss mich zusammen reisen}, dachte sie und sah in den Spiegel um sich zu 
vergewissern, dass sie eine annehmbare Erscheinung abgab. ``Dann höre ich mir mal an, was sie zu 
sagen haben.''\\
``Ich denke, gleich zu ihnen zu kommen wäre zu viel der Ehre.''\\
``Du hast mir gerade versucht zu sagen, wie viel Macht die Priester haben. Vielleicht sollten sie 
erst einmal nicht befürchten, dass mir das egal ist. Bring sie in mein Arbeitszimmer.''\\


Na'Rash war die größte Stadt in Merandila. Sie besaß die größte Schule, die größte Bibliothek, den 
größten Hafen und den größten Tempel. Und hier vor hier saß der Mann, der zumindest inoffiziell 
über all das herrschte. In Vertretung eines Gottes. \\
Der Hohepriester Em'Hir war klein und mager. Sarimé verspürte unweigerlich den Drang, sich gerade 
aufzurichten, nur um zu sehen, ob sie ihn dann vielleicht doch wenige Millimeter überragen würde. 
Es war immerhin ungewohnt, dass sie einem Mann hohen Ranges gegenüberstand, der nicht größer
als sie war. Und es war beängstigend in seine kalten Augen zu blicken. Erst jetzt viel dem 
Mädchen auf, welche Erleichterung es auch sein konnte, wenn die Männer größer waren. Es war 
einfacher ihren Blicken auszuweichen. Vor den Augen des Priesters schien es jedoch kein Entkommen zu 
geben.\\
``Mein Herr'', sprach sie und ihre Hände umschlossen die ausgestreckte Rechte des Priesters. Sie 
neigte leicht ihr Haupt. ``Hättet Ihr mich doch informiert! Es beschämt mich, dass ihr die Burg in 
diesem Zustand seht. Ich hätte Euch Begleiter entgegen geschickt, Euch im Hof empfangen, ein 
Festmahl und einen Gottesdienst vorbereiten lassen!''\\
``Ist gut, Kind'', säuselte der Mann und tätschelte ihre Hände. Die Tätowierungen in seinem Gesicht 
machten es fast unmöglich, seine Mimik zu entziffern. ``Ich weiß doch, welche Sorgen Euch plagen 
und wie überfordert Ihr Euch fühlen müsst, angesichts dieser Fülle an Pflichten und 
Verantwortung.''\\
Sarimé entging nicht, wie Renec - der neben Samos stand - den Mund zusammenkniff. \\
\textit{So ein stolzer Bastard, der die Ehre seiner Gräfin verteidigen will}, dachte sie bissig, 
konzentrierte sich jedoch wieder auf den Hohepriester vor ihr. ``Nein, nein'', rief sie aus: 
``Nicht einmal Zimmer sind hergerichtet! Das darf keine gute Gastgeberin zulassen. Bitte, erlaubt 
mir, Euch und Euren Begleitern im Speisezimmer eine kleine Stärkung anzubieten, bis die Gemächer 
bereit sind. Ihr müsst müde sein von der Reise.''\\
``Übernehmt Euch nicht, werte Dame'', unterbricht Em'Hir sie lächelnd: ``Aufregung ist für Frauen 
nie gut, besonders nicht in gesegneten Umständen. Das ist auch der Grund, wieso ich so schnell 
hier her geeilt bin. Zum einen fühle ich mich leider etwas gekränkt, dass ich nicht bedacht wurde 
bei der Verkündung dieser Neuigkeit.''\\
Sarimé blieb einen Augenblick stumm und sah in die dunklen Augen. Dann fasst sie sich wieder, 
setzte ein schüchternes Lächeln auf und erwiderte: ``Verzeiht... ich habe nicht daran gedacht... 
ich beeilte mich so sehr, unserem König die Nachricht zu übermitteln und dann meinen nächsten 
Familienangehörigen...''\\
Wieder tätschelte er ihr die Hand. ``Ich verstehe schon. Für jede Frau muss die erste 
Schwangerschaft überwältigend sein und zu unüberlegten Handeln führen. Und erst recht bei einem so 
jungen - wenn auch sehr bezaubernden - Mädchen wir Euch! Bevor ich Euer Angebot annehme, will ich 
aber noch schnell den zweiten Grund meiner Anwesenheit offenbaren. Ich halte nicht viel davon, 
solch eine Strecke zurück zu legen und dann noch Tage zu warten, bis man über das eigentliche Thema 
spricht. Ich möchte persönlich den Segen für den Erben der Grafschaft erbitten.''\\
Die letzten Worte sprach er feierlich und hob dabei den Blick zum Himmel, als würde direkt über 
ihnen Osyma persönlich schweben. Sarimé folgte zögernd seinen Augen und sah dann unsicher zur 
Seite, an der Samos und Renec steif standen.\\
``Den Segen'', wiederholte sie stockend: ``Dass... diese Ehre kann ich doch nicht annehmen, Herr. 
Ich plante, dass der Priester unserer Burg den Segen erbitten würde... er diente schon meinem 
Gatten und seiner Familie seit vielen Jahren und ihm würde es viel bedeuten. Er ist schon ein Mann 
hohen Alters und es wird vielleicht der letzte Segen eines A'Riks sein.''\\
Sie wusste, dass ihre Worte nichts ändern würden, aber irgendetwas hatte sie sagen müssen. Dass 
Em'Hir ihr und dem ungeborenen Leben in ihr, den Segen erbitten würde, bedeutete, er würde 
mindestens einige Wochen hier verbringen. Schlimmer konnte sich Sarimé die Situation nur 
vorstellen, wenn der König persönlich gekommen wäre.


