\chapter{Inhalt: Diskussion der Verschwörer}

Leises Gemurmel erfüllte die kreisförmige Hütte. Zwischen den Spalten der Bretter drang das Licht 
des Morgens ebenso ein wie der Geruch nach feuchtem Laub und Regen. Der Morgen war klar - und dafür 
dankte Renec der Hellen aus tiefstem Herzen. Es war schrecklich bei Regen in dieser Hütte zu sein. 
Die Tropfen trommelten nervenzerreißend laut in ihrem unbeeinflussbaren Rhythmus auf das Dach. Der 
Wind peitschte durch die Wände und der Boden wurde zu einem Sumpf. Die Männer und Frauen in diesem 
Raum saßen auf mitgebrachten Sitzgelegenheiten, manch einer sogar auf dem feuchten Boden, und 
unterhielten sich innerhalb kleiner Gruppen. Renec stand jedoch am Rand, hatte die Arme verschränkt 
und blickte nachdenklich in die Mitte des Kreises. Seit seinem Eintreffen hatte er etliche 
freundliche Blicke erhalten, die ihn aufforderten, sich doch zu ihnen zu gesellen, aber er 
verharrte regungslos. Früher hatte er den Großteil der Leute hier als Freunde bezeichnet. Durch die 
Ereignisse der letzten Monate hatte sich seine Sichtweise etwas geändert. Die Dinge, die hier 
besprochen wurden, klangen zwar von solcher Wichtigkeit, aber letzten Endes hatten sie nichts mit 
der Realität zu tun. Der Krieg kam. Soldaten kamen. Und das von beiden Seiten Merandilas.\\
Samos, der Hauptmann der Wache, trat ein. Sein Blick ging über den Bastard hinweg und dann gesellte 
der Mann sich zu einer in den letzten Tagen immer größer werdenden Gruppe.\\
Diese Treffen hatten schon immer in leisen Kreisen stattgefunden. Geheim und mit Sorge. Samos 
Gruppe jedoch waren die Lauten. Sie flüsterten nicht mehr, wenn sie von Verrat sprachen. Sie 
verstummten nicht mehr, wenn die Andeutungen zu ernst wurden. Sie hatten keine Angst mehr in der 
Stimme, sondern Zorn und Ehrgeiz.\\
Plötzlich wurde es still, als die nächste Person die Hütte und den Kreis aus Menschen betrat. Es war 
ein unscheinbarer Mann und doch bewirkte seine Stimme so viel in den Menschen. Arham war einer der 
Holzfäller Talsmund und verbrachte seinen Tag damit durch den Wald zu streifen. Er lief von Dorf zu 
Dorf und erzählte seine Geschichten. Er hatte auch diese Hütte errichtet. Es war ungewohnt ihn ohne 
der schartigen Axt über seiner Schulter zu sehen. Seine Kluft war aber die übliche, geziert mit 
Schmutz des Waldes. Der Mann war vermutlich nahe seines vierzigsten Lebensjahrs und erste graue 
Haare schimmerten in dem unordentlichen dunklen Mopp auf seinem Kopf. Sein Bart war kurz und 
zerzaust, ein seltener Anblick in der von Saleicanern bestimmten Kultur.\\
Er neigte freundlich lächelnd den Kopf und sah sich um. Die Menschen lächelten schüchtern, wenn sich 
ihr Blick mit seinem traf. Auch Renec, der weit außerhalb des Kreises stand, fanden seine Augen und 
über Renecs Lippen huschte ein flüchtiges Zucken. Er nickte Arham zu und trat etwas näher, weil er 
keines seiner Worte verpassen wollte.\\
Während die meisten Leute saßen, stand der Holzfäller in der Mitte. Schweigend bückte er sich und 
holte einige wenige Gegenstände aus seinem Beutel hervor. Eine schmale weiße Kerze folgte einem 
scharfkantigen Stück Eis. Die zarten Blätter einer frühen Winterblume legte er neben einem 
unförmigen Stein und als letztes kam ein weißes Raubkatzenfell dazu.\\
Renec verzog unwillkürlich das Gesicht. Die Raubkatzen lebten hier in der Gegend nicht mehr in der 
Wildnis, sondern ließen sich nur in der Zucht der Grafschaft finden. Wie auch immer er dieses Fell 
erhalten hatte, das Tier welches es einst zierte gehörte den Grafen Merandilas.\\
Arham begann gestikulierend zu erzählen. Seine ruhigen Worte erfüllten die Hütte und er schlenderte 
im Kreis umher um jeden der Anwesenden seine Aufmerksamkeit zu schenken. ``Der Winter wartet darauf, 
dass er sich über unser Land legen kann. Er will die Erde von der Vergangenheit reinigen und Tier 
und Mensch Frieden und Ruhe schenken. Er will die Zeit für uns einfrieren und uns die Gelegenheit 
geben, Inne zu halten, einander anzublicken, die Hände zu reichen. Ehe der Frühling wieder kommt und 
die Zeit sich die verlorenen Momente rasant wiederholt, in dem es die Welt aus der Erstarrung löst 
und verändern lässt.''\\
Renec setzte sich nun doch auf den Boden und beugte sich gespannt auf die Worte zu Arham vor.\\
``Früher war die Helle uns immer nahe. Sie war das Sternenlicht in der dunklen Nacht. Sie spendete 
Schatten, wenn die Sonne brannte. Sie erfüllte die Stille mit ihrer Musik, das Schweigen mit ihrer 
Stimme. Sie hob ihr Schwert im Kampf gegen die Löwen. Ihre Seele erfüllte das Herz unserer letzten 
Königin... und sie fiel mit ihr den Klauen zum Opfer. Sie blutete aus, verbirgt sich nun im 
Schatten des Feuergottes. Aber sie hat uns nicht verlassen!''\\
Arham breitete die Arme aus und drehte sich langsam im Kreis. ``Sie ist hier! Die Helle lebt''\\
``Die Helle lebt!'', erklang die zustimmenden Rufe.\\
``Und schon vor Jahren habe ich euch verkündet, dass sie uns eine neue Königin schicken wird, die 
unser Land in die Unabhängigkeit führen wird!''\\
``Und dann ist diese Königin aus einem Turm gesprungen!'', unterbrach Renec laut und erhob sich. Es 
war so verlockend sich in Arhams Worten zu verlieren. Er konnte Träume spinnen, einem die Nacht als 
taghell verkaufen. Und hätte er Sieva nicht erwähnt, dann hätte Renec sich damit begnügt ihn nur 
finster anzustarren.\\
Arham musterte ihn einen Moment lang und hob beschwichtigend die Hände. ``Ich habe nie behauptet, 
dass sie es sein würde. Die Frau, von der du sprichst, ist nur eine arme Seele, die vor dem 
Feuergott geflohen ist.''\\
Renec verdrehte die Augen und marschierte aus der Hütte. Kurz dachte er darüber nach, dieser 
Versammlung sofort den Rücken zu kehren, aber dann blieb er doch bei seinem Wallach stehen und 
ordnete ihm die Mähne. Sobald Arham seine Predigt beendet hatte, würden die Leute noch zusammen 
sitzen und plaudern. Manchmal spielten sie auf Flöten und Trommeln, eigentlich immer ging der 
Schnaps durch die Reihen. Renec wusste, dass dieser Zeitpunkt gekommen war, als Arham sich zu ihm 
gesellte.\\
``Es tut mir leid.''\\
``Sie wäre keine Königin geworden. Sie war nicht mal eine gute Gräfin.''\\
Arham legte den Kopf schief. ``Und trotzdem die Frau, die dein Herz besaß. Warum fühlst du dich 
angegriffen?''\\
Renec holte tief Luft. ``Ihr habt doch keine Ahnung, was auf uns zukommen wird. Ihr seid Bauern 
und Förster, Viehtreiber und Müller.''\\
``Weißt du es? Hast du schon in einem Krieg gekämpft?''\\
Renec schwieg verbissen.\\
``Ich will diesen Menschen Hoffnung geben, Renec. Die haben sie sich auch verdient. Du ebenso.''\\
``Ich wollte mit Sieva gehen. Ich hätte mein Land für sie verlassen. Sogar meine Göttin. Aber sie 
wollte nicht. Anstatt mit mir ein erfülltes Leben zu leben, sprang sie.''\\
``Und ließ dich zurück'', schlussfolgerte der Holzfäller.\\
``Was ist dein glorreicher Plan?'', fragte Renec sarkastisch.\\
``Hier ist nicht der richtige Ort für dieses Gespräch.'' Arham sah sich um und überlegte, was er 
sagen konnte. Dann fügte er hinzu: ``Wir müssen wissen, wer auf unserer Seite steht. Ohne, dass 
die Priester es merken. Ich habe mit einigen Leuten gesprochen. Und einige Briefe geschrieben. Für 
die brauche ich deine Unterschrift.''\\
Renec runzelte die Stirn. ``Meine? Wieso?'', fragte er misstrauisch.\\
``Du bist A'Riks Sohn.''\\
``Aber du wirst mir nicht sagen, was in diesen Briefen steht?''\\
``Wenn alles nach Plan geht, musst du es nicht wissen.''\\
``Und wenn dein Plan schief geht, werde ich wegen Verrat geköpft.''\\
``Eher gehängt, du bist immerhin nur ein Bastard.''\\
Renec verzog das Gesicht und schloss einen Moment die Augen. ``Ich bin mir nicht sicher, ob es das 
wert ist. Ich weiß einfach gerne, wofür ich hingerichtet werden könnte.''\\
Arham seufzte und flüsterte grinsend: ``Du bist ein lästiger Junge! Die Magie eines Predigers ist 
es, dass er mit Überraschungen arbeitet. Ich komme mit zur Burg, will mir das Mädchen mal selbst 
ansehen. Dann zeige ich dir die Briefe.''\\

``Oh, das erste was ich in der Stadt machen werde, ist auf den Markt zu gehen um Kleidung für den 
kleinen Schatz zu besorgen!'', plapperte Suja, während sie eine Decke faltete und in eine der 
Holztruhen legte. ``Darum kümmerst du dich gerade viel zu wenig! Du solltest viel mehr an das Kind 
denken.''\\
``Alles'', unterbrach Sarimé ihre Schwägerin: ``was ich gerade denke, hat mit diesem Kind zu 
tun!''\\
Suja betrachtete sie nachdenklich und schüttelte unzufrieden den Kopf. ``Aber es sind nicht die 
Gedanken, die eine werdende Mutter haben sollte. Du solltest dich fragen, welches Geschlecht es 
haben wird. Ob rotes Haar oder schwarz. Oh hoffen wir, wenn es ein Junge wird, dass er mehr nach 
seinem Vater kommt... ein kleiner Handwerker oder Kaufmann ist ja noch eine Sache, aber stell dir 
mal einen kleinen Grafen vor!'', kicherte Suja.\\
``Es ist egal, welches Geschlecht es hat. Es ist so oder so Evins Erbe.''
Die blonde Frau seufzte tief. ``Vielleicht bist du noch nicht reif genug dafür...''\\
\textit{Zu jung. Zu unreif. Zu naiv}, dachte Sarimé bitter und warf den Handspiegel in eine der 
offenen Truhen. ``Ich habe eine Grafschaft zu verwalten'', fauchte sie: ``Und ständig kommen mir 
irgendwelche Männer in die Quere, die mir sagen wollen, wie und was und wann ich dieses und jenes 
zu tun habe! Oh, ganz zu vergessen von dem Krieg und dem Feindesland, das hier ziemlich in der nähe 
ist! Und die Generäle, die nicht mal so tun, als würden sie mir Höflichkeit oder gar Respekt 
erweisen. Ach ja... und dann bin ich ja auch noch schwanger und werde in wenigen Wochen aufgebläht 
wie eine Sau sein! Und das kann man nicht schön reden, versuch es nicht einmal, Suja! Ich habe dich 
gesehen, als du schwanger warst. Und meine Stiefmutter auch viel zu oft. Da kann man faseln wie man 
will, dass das Leuchten des Glücks einen begleitet. Man sieht einfach nur fett und müde aus!''\\
Suja räusperte sich. ``Das sind nur die Stimmungen einer werdenden Mutter...''\\
Sarimé stampfte wütend auf. ``Ich weiß!''\\
Da bemerkte sie, dass Sujas Worte nicht an sie gerichtet waren und folgte ihrem Blick zur Tür. Dort 
stand ein drahtiger Mann in legerer Reisekleidung. Sarimé konnte auf dem ersten Blick nicht sagen, 
was genau es war, dass sein Äußeres exotisch aussehen ließ. Vielleicht war es auch nur die Ruhe in 
seinem Blick. ``Was platzt du hier herein?'', brauste sie auf: ``Ich hätte mich gerade umkleiden 
können!''\\
Jetzt zuckten seine Lippen zu einem flüchtigen lächeln. ``Ich soll Euch ausrichten, dass Rotan 
Àrell eingetroffen ist. Der neue Stallmeister.''\\
Sarimé kniff die Augen zusammen. Momentan wuselten viel zu viele Fremde in der Burg umher. ``Wo ist 
Renec? Er wollte dabei sein. Und wer bist du? Gehörst du zu den werten Priestern?''\\
Dass es einer von Samos Leuten war, konnte sie ausschließen. Ihr Hauptmann stellte ihr jede neue 
Wache persönlich vor. Er schüttelte den Kopf und erklärte: ``Ich stehe im Dienste des Generals.''\\
``Verstehe.'' Sarimé überlegte kurz und fügte hinzu: ``Ich hoffe dir ist klar, dass diese Räume 
meine Privatsphäre widerspiegeln. Was hier geschieht, was ich hier sage, verlässt mein Gemach auch 
nicht!''\\
Suja mischte sich ein. ``Genau. Sie ist immer nur hier zickig.''\\
Sarimé kniff die Lippen zusammen und atmete tief durch, ehe sie sich auf den Weg zum Stall machte.\\

Der Saleicaner verneigte sich tief vor seiner Gräfin und wiederholte erneut seinen Namen und welche 
Ehre es für ihn war. Sarimé erwiderte die Begrüßung höflich und blickte über den Hof. Dutzende 
Männer und Frauen wuselten herum. Einige gehörten zu den Arbeitern, die die Mauern ausbesserten. 
Andere packten schon eifrig für die Reise in die Stadt. Und wieder andere gehörten eindeutig zu 
Mi'Kaés Soldaten, die herumlungerten und lautstark scherzten. ``Und wo sind nun Ihre Pferde?'', 
fragte Sarimé: ``Ich würde gerne ein paar Beispiele Eurer Zucht sehen, bevor ich eine Entscheidung 
fälle.''\\
Mit diesen Worten nickte sie dem alternden Stallmeister zu, denn er hatte sie gebeten, dass sein 
potentieller Nachfolger gut geprüft werden solle.\\
Rotan Arell stammelte etwas haltlos umher und Sarimé betrachtete die drei Tiere hinter ihm. Fragend 
sah sie sich nach dem Stallmeister um.\\
``Nicht sehr eindrucksvoll'', bemerkte dieser: ``Wieso hat der Fuchs solche Furcht?''\\
``Die Stute ist noch nicht eingeritten'', murmelte Arell.\\
Der Stallmeister schien nicht sehr begeistert und schüttelte immer wieder den Kopf.\\
Sarimé wandte sich dem Saleicaner zu. ``Dann erzählt. Ich habe nicht viel Zeit für diese 
Angelegenheit.''\\
Der Mann straffte sich und nach einem Räuspern sagte er: ``Ergebnisse meiner Zucht wurden in ganz 
Merandila verkauft! Viele Tiere sind zum Heer gekommen. Einige auch in Euren Stall, Herrin...''\\
``Ich kenne Euren Namen und ich kannte einige Eurer Tiere'', warf der alte Meister ein: ``Aber was 
ist mit jetzt? Wo sind Eure Tiere?''\\
``Beim Heer'', platzte es aus ihm heraus: ``Meine Schwester ist Soldatin und kam mit einem guten 
Angebot.''\\
``Ihr habt alle Tiere verkauft? Alle?'', fragte Sarimé irritiert: ``Ich kenne mich ja nicht 
sonderlich in diesem Gebiet aus, aber wie wollt Ihr züchten, wenn Ihr keine Tiere mehr besitzt?''\\
``Ich habe noch ein Paar'', murmelte er verlegen.\\
Sarimé war es gleich, welche Tiere er noch besaß. Sie verstand auch ehrlich nicht, wieso das 
wichtig war, schließlich sollte er die Verantwortung über ihren Stall führen. Aber der Meister 
hatte sich klar genug mit seinen Worten ausgedrückt. Sie unterdrückte ein genervtes Seufzen, 
lächelte stattdessen und legte dem alten Stallmeister die Hand auf die Schulter. ``Wir werden Euch 
eine Antwort bald mitteilen, werter Herr Arell.''\\
Die junge Gräfin hakte sich bei dem Stallmeister unter und schlenderte in den Stall. Vorbei kam sie 
dabei an dem schwarzhaarigen Kerl, der in ihr Zimmer geplatzt war. Er striegelte gerade eine 
Fuchsstute. \textit{Sein Reittier und Arells Stute könnten Schwestern sein}, schmunzelte sie: 
\textit{Ebenso widerspenstig scheinen sie auch zu sein!}\\
Sarimé hatte schon das Geflüstert der Stallburschen über dieses Tier aufgeschnappt. Es hatte wohl 
schon einen Verletzten gegeben, der sich dem Tier genähert hatte und gleich darauf einen Huftritt 
abbekam.\\
``Also?'', seufzte sie: ``Hat er deinen Segen?''\\
``Ach Kind'', brummte der Stallmeister in einem vertraulichen Tonfall.\\
Das war ihr schon bei mehreren Bediensteten aufgefallen. Seit Evins Tod sprachen sie viele Leute 
persönlicher an, wenn sie nicht gerade Umgeben von Priestern und Soldaten war. Sarimé störte es 
nicht, im Gegenteil. Dadurch fühlte es sich an, als würde sie ihre Last nicht ganz alleine tragen. 
Außerdem hätte sie es sich bei dem alten Mann auch nicht getraut ihn zurecht zu weisen. Sie war 
unheimlich erleichtert, dass er sie anscheinend mochte und nicht so hart mit ihr umsprang wie mit 
manchen seiner Burschen und Knechte.\\
``Renec hat ihn vorgeschlagen, stimmt's?'', murmelte er: ``Und dann ist er nicht mal hier zur
Vorstellung.''\\
``Das hat mich auch überrascht'', bemerkte die Gräfin.\\
Im Gespräch verließen sie den Pferdestall und kamen zu den abgedunkelten Gehegen der Raubkatzen.\\
``Wie gesagt, ich kenne seinen Namen. Aber ich verstehe nicht, wieso er all seine Tiere verkaufen 
sollte, noch bevor er eine neue Stelle hat. Er konnte unmöglich wissen, dass man ihn bitten würde, 
sich dir vorzustellen'', überlegte er laut: ``Ich weiß, dass er gut mit Pferden umgehen kann.''\\
Sarimé betrachtete die Katzen. Schlanke, sehnige Tiere, die sich auf dem Boden und den Holzgerüsten 
reckelten. Das weiße Fell stand bauschig in alle Richtungen ab und die gelben Augen blinzelten 
träge. Manches Tier fühlte sich wohl durch ihre Anwesenheit gestört und der dünne Schwanz zuckte 
unruhig.\\
``Das klingt nach einem Aber.''\\
``Pferde ja. Falken und Katzen? Pff... ich glaube nicht, dass er mit diesen Tieren schon einmal zu 
tun hatte. Was anderes habe ich von Renec aber auch nicht erwartet. Es war mir klar, dass er nicht 
so weit denken würde.''\\
``Was ist also dein Vorschlag? Willst du Arell lehren?''\\
Der Stallmeister lachte spöttisch. ``Nein. Die Mühe mache ich mir auf meinen alten Tagen nicht 
mehr. Ich habe einige vielversprechende Schüler.''\\
Sarimé runzelte die Stirn. ``Und wieso hast du mir das nicht gesagt, bevor ich nach Arell schicken 
ließ?''\\
``Sie sind noch nicht so weit.''\\
Nun seufzte Sarimé doch frustriert. Die Gespräche mit dem alten Mann waren anstrengend.\\
``Also nimm Arell mit'', fügte er hinzu: ``Soll er sich um die Pferde kümmern, die mit nach Na'Rash 
kommen. Ich hätte sowieso keine Lust mehr auf diese Reise. Ich bleibe hier und kümmere mich um den 
Stall. Bringe meinen beiden Schülern alles bei. Wenn du dann irgendwann zurück kommst, sind sie 
Meister. Und ich räume mal die Zuchtbücher auf, damit Arell - falls er dann noch in deinem Diensten 
steht - auch mein Lebenswerk entziffern kann.''\\
Eine der Katzen fixierte Sarimé und sie erwiderte diesen Blick, während sie langsam nickte. ``In 
Ordnung. Dann zeig ihm die Pferde und wähle aus, welche mit kommen und welche hier bei dir bleiben. 
Und wenn du Renec siehst, sage ihm bitte, dass er ein unzuverlässiger Faulpelz ist.''\\
Der Stallmeister zeigte die wenigen Zähne die er noch besaß bei einem breiten Grinsen.\\
Als Sarimé den Weg zurück durch den Pferdestall ging, fiel ihr eine junge Frau auf, die bei Rabe 
herumlungerte. Die Gräfin musterte sie kurz. \textit{Es darf doch nicht sein, dass so viele fremde 
Gesichter in meiner Burg herumlaufen.}\\
``Guten Tag'', sprach sie die Frau an: ``Zu wem gehörst du?''\\
Erschrockene Augen starrten sie an. Ihr Mund öffnete und schloss sich wieder, ohne ein Wort zu 
sagen. Ungeduldig sah Sarimé sie an und wartete auf eine Antwort. ``Sprich. Den Priestern? Arell? 
Mi'Kae wird es ja wohl nicht sein.''\\
Sie blinzelte mehrmals und murmelte. ``Arell.''\\
Die junge Gräfin kniff die Augen zusammen. Es war eine Sache, wenn ihre Köchin oder ihr 
Stallmeister, die sie beide kannte seit sie als Braut hier her kam, vertraulich mit ihr umgingen. 
Aber das diese Fremde es nicht einmal als nötig erachtete, sich vorzustellen oder einen Satz mit 
ihr zu sprechen, reizte sie. \textit{Vielleicht wirklich die Hormone,} schoss es ihr durch den 
Kopf.\\
``Sie spricht kein Saleicanisch'', erklang Renecs Ruf, der gerade auf seinem Wallach herein 
geritten kam.\\
Irritiert sah sie vom Bastard zur Fremden. ``Welche Sprache denn dann?''\\
``Kasira'', erwiderte er nur, während er abstieg und mit einem schnellen Griff den Sattelgurt 
lockerte. ``Arell hat sie mitgebracht. Ich sah sie auf seinem Hof.''\\
``Willst du mir sagen, dass ich eine Kasira in meiner Burg habe?'', sagte Sarimé entsetzt.\\
Renec überlegte kurz und antwortete grinsend: ``Nein nein. Nur jemanden, der Kasira spricht.''\\
Er wandte sich der Frau zu und sprach mit ihr. Skeptisch lauschte Sarimé der Sprache des Feindes, 
die sie noch nie gehört oder auch nur als Schrift gesehen hatte. Aber ihr entging die Reaktion der 
Frau nicht. Sie blickte verlegen auf dem Boden, fummelte nervös mit ihren Fingern am Saum ihres 
Oberteils und als Renec endete, verneigte sie sich hastig und stammelte schnelle Worte.\\
Die Gräfin schüttelte verständnislos den Kopf und sah zu Renec.\\
``Ich habe sie darauf aufmerksam gemacht, wer du bist und dass sie sich vorstellen soll.''\\
``Hat gut funktioniert.''\\
Das war der Frau auch nicht entgangen. Sie deutete auf ihre Brust. ``Lavay'', sprach sie langsam 
und deutlich aus.\\
Sarimé musste lachen. So hatte sie sich immer als Kind die Kommunikation mit den Kolonien 
vorgestellt. Weiterhin schmunzelnd deutete sie auf sich selbst. ``Sarimé Sil'Vera!''\\
Die junge Frau erwiderte das Lächeln scheu.\\
``Hey Renec, ich soll dir von unserer Gräfin ausrichten, dass du ein fauler Nichtsnutz bist'', rief 
der Stallmeister über den Hof und zog etliche Blicke auf sich.\\
``Nein. Ich sagte, unnützer Faulpelz.''\\
``War es nicht doch eher ein unzuverlässiger Nichtsnutz?'', überlegte der Meister laut.\\
Zahlreiche Lacher folgten den Scherzen und Sarimé hatte genug davon, sich zu verstellen und lachte 
neckend mit. Ihr entging jedoch nicht, dass es ihre Leute waren, die den Scherz genossen. Von 
Mi'Kaes Soldaten und den Begleitern der Priester sah sie nur befremdliche Blicke. ``Einigen wir uns 
einfach darauf, dass ich ein Bastard bin?'', schlug Renec grinsend vor und nahm nun auch dem Wallach 
das Zaumzeug ab.\\
``Gerne doch. Der Form muss ja gewahrt werden, hier am Hofe Merandilas'', sprach Sarimé in einem 
leicht arroganten Tonfall, was einige in der Nähe wieder zum Kichern brachten.\\
Sie bemühte sich, wieder ernst zu werden und sagte zu Rotan Arell: ``Herzlichen Glückwunsch, Ihr 
habt den Segen des Meisters. Ihr werdet mit mir nach Na'Rash kommen, er wird Euch über die 
Einzelheiten aufklären. Ach ja. Ich nehme an, Ihr sprecht die Sprache Eurer Begleitung. Sagt ihr, 
ich lade sie heute Abend zu mir in meine Gemächer ein.''\\
Er nickte eifrig und sah sich nervös um. Schien noch nicht ganz einordnen zu können, was hier 
gerade vorgegangen war.\\

