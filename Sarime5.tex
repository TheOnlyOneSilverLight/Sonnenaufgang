\chapter{Inhalt: Diskussion der Verschwörer}

Leises Gemurmel erfüllte die kreisförmige Hütte. Zwischen den Spalten der Bretter drang das Licht 
des Morgens ebenso ein wie der Geruch nach feuchtem Laub und Regen. Der Morgen war klar - und dafür 
dankte Renec der Hellen aus tiefstem Herzen. Es war schrecklich bei Regen in dieser Hütte zu sein. 
Die Tropfen trommelten nervenzerreißend laut in ihrem unbeeinflussbaren Rhythmus auf das Dach. Der 
Wind peitschte durch die Wände und der Boden wurde zu einem Sumpf. Die Männer und Frauen in diesem 
Raum saßen auf mitgebrachten Sitzgelegenheiten, manch einer sogar auf dem feuchten Boden, und 
unterhielten sich innerhalb kleiner Gruppen. Renec stand jedoch am Rand, hatte die Arme verschränkt 
und blickte nachdenklich in die Mitte des Kreises. Seit seinem Eintreffen hatte er etliche 
freundliche Blicke erhalten, die ihn aufforderten, sich doch zu ihnen zu gesellen, aber er 
verharrte regungslos. Früher hatte er den Großteil der Leute hier als Freunde bezeichnet. Durch die 
Ereignisse der letzten Monate hatte sich seine Sichtweise etwas geändert. Die Dinge, die hier 
besprochen wurden, klangen zwar von solcher Wichtigkeit, aber letzten Endes hatten sie nichts mit 
der 
Realität zu tun. Der Krieg kam. Soldaten kamen. Und das von beiden Seiten Merandilas.\\
Samos, der Hauptmann der Wache, trat ein. Sein Blick ging über den Bastard hinweg und dann gesellte 
der Mann sich zu einer in den letzten Tagen immer größer werdenden Gruppe.\\
Diese Treffen hatten schon immer in leisen Kreisen stattgefunden. Geheim und mit Sorge. Diese Leute 
jedoch waren die Lauten. Sie flüsterten nicht mehr, wenn sie von Verrat sprachen. Sie verstummten 
nicht mehr, wenn die Andeutungen zu ernst wurden. Sie hatten keine Angst mehr in der Stimme, sondern 
Zorn und Ehrgeiz. Renec war stets mehr der Zuhörer gewesen, aber wenn Samos Gruppe nun die Stimme 
erhob, konnte er sich eine Erwiderung kaum noch verkneifen.\\
Plötzlich wurde es still, als die nächste Person die Hütte und den Kreis aus Menschen betrat. Es war 
ein unscheinbarer Mann und doch bewirkte seine Stimme so viel in den Menschen. Arham war einer der 
Holzfäller Talsmund und verbrachte seinen Tag mit durch den Wald zu streifen. Er hatte auch diese 
Hütte errichtet. Es war ungewohnt ihn ohne der schartigen Axt über seiner Schulter zu sehen. Seine 
Kluft war aber die übliche, geziert mit Schmutz des Waldes. Der Mann war vermutlich nahe seines 
vierzigsten Lebensjahrs und erste graue Haare schimmerten in dem unordentlichen dunklen Mopp auf 
seinem Kopf. Sein Bart war kurz und zerzaust, ein seltener Anblick in der von Saleicanern bestimmten 
Kultur.\\
Er neigte freundlich lächelnd den Kopf und sah sich um. Die Menschen lächelten schüchtern, wenn sich 
ihr Blick mit seinem Traf. Auch Renec, der weit außerhalb des Kreises stand, fanden seine Augen und 
über Renecs Lippen huschte ein flüchtiges Zucken. Er nickte Arham zu und trat etwas näher, weil er 
keines seiner Worte verpassen wollte.\\
Während die meisten Leute saßen, stand der Holzfäller in der Mitte. Schweigend bückte er sich und 
holte einige wenige Gegenstände aus seinem Beutel hervor. Eine schmale weiße Kerze folgte einem 
scharfkantigen Stück Eis. Die zarten Blätter einer frühen Winterblume legte er neben einem 
unförmigen Stein und als letztes kam ein weißes Raubkatzenfell dazu.\\
Renec verzog unwillkürlich das Gesicht. Die Raubkatzen lebten hier in der Gegend nicht mehr in der 
Wildnis, sondern ließen sich nur in der Zucht der Grafschaft finden. Wie auch immer er dieses Fell 
erhalten hatte, das Tier welches es einst zierte gehörte den Grafen Merandilas.\\
Arham begann gestikulierend zu erzählen. Seine ruhigen Worte erfüllten die Hütte und er schlenderte 
im Kreis umher um jeden der Anwesenden seine Aufmerksamkeit zu schenken. ``Der Winter wartet darauf, 
dass er sich über unser Land legen kann. Er will die Erde von der Vergangenheit reinigen und Tier 
und Mensch Frieden und Ruhe schenken. Er will die Zeit für uns einfrieren und uns die Gelegenheit 
geben, Inne zu halten, einander anzublicken, die Hände zu reichen. Ehe der Frühling wieder kommt und 
die Zeit sich die verlorenen Momente rasant wiederholt, in dem es die Welt aus der Erstarrung löst 
und verändern lässt.''\\
Renec setzte sich nun doch auf den Boden und beugte sich gespannt auf die Worte zu Arham vor.\\
``Früher war die Helle uns immer nahe. Sie war das Sternenlicht in der dunklen Nacht. Sie spendete 
Schatten, wenn die Sonne brannte. Sie erfüllte die Stille mit ihrer Musik, das Schweigen mit ihrer 
Stimme. Sie hob ihr Schwert im Kampf gegen die Löwen. Ihre Seele erfüllte das Herz unserer letzten 
Königin... und sie fiel mit ihr den Klauen zum Opfer. Sie blutete aus, verbirgt sich nun im 
Schatten des Feuergottes. Aber sie hat uns nicht verlassen!''\\
Arham breitete die Arme aus und drehte sich langsam im Kreis. ``Sie ist hier! Die Helle lebt''\\
``Die Helle lebt!'', erklang die zustimmenden Rufe.\\
``Und schon vor Jahren habe ich euch verkündet, dass sie uns eine neue Königin schicken wird, die 
unser Land in die Unabhängigkeit führen wird!''\\
``Und dann ist diese Königin aus einem Turm gesprungen!'', unterbrach Renec laut und erhob sich. Es 
war so verlockend sich in Arhams Worten zu verlieren. Er konnte Träume spinnen, einem die Nacht als 
taghell verkaufen. Und hätte er Sieva nicht erwähnt, dann hätte Renec sich damit begnügt ihn nur 
finster anzustarren.\\
Arham musterte ihn einen Moment lang und hob beschwichtigend die Hände. ``Ich habe nie behauptet, 
dass sie es sein würde. Die Frau, von der du sprichst, ist nur eine arme Seele, die vor dem 
Feuergott geflohen ist.''\\
Renec verdrehte die Augen und marschierte aus der Hütte. Kurz dachte er darüber nach, dieser 
Versammlung sofort den Rücken zu kehren, aber dann blieb er doch bei seinem Wallach stehen und 
ordnete ihm die Mähne. Sobald Arham seine Predigt beendet hatte, würden die Leute noch zusammen 
sitzen und plaudern. Manchmal spielten sie auf Flöten und Trommeln, eigentlich immer ging der 
Schnaps durch die Reihen. Renec wusste, dass dieser Zeitpunkt gekommen war, als Arham sich zu ihm 
gesellte.\\
``Es tut mir leid.''\\
``Sie wäre keine Königin geworden. Sie war nicht mal eine gute Gräfin.''\\
Arham legte den Kopf schief. ``Und trotzdem die Frau, die dein Herz besaß. Warum fühlst du dich 
angegriffen?''\\
Renec holte tief Luft. ``Ihr habt doch keine Ahnung, was auf uns zukommen wird. Ihr seid Bauern 
und Förster, Viehtreiber und Müller.''\\
``Weißt du es? Hast du schon in einem Krieg gekämpft?''\\
Renec schwieg verbissen.\\
``Ich will diesen Menschen Hoffnung geben, Renec. Die haben sie sich auch verdient. Du ebenso.''\\
``Ich wollte mit Sieva gehen. Ich hätte mein Land für sie verlassen. Sogar meine Göttin. Aber sie 
wollte nicht. Anstatt mit mir ein erfülltes Leben zu leben, sprang sie.''\\
``Und ließ dich zurück'', schlussfolgerte der Holzfäller.\\
``Dann sprich. Was ist dein glorreicher Plan?'', fragte Renec sarkastisch.\\
``Hier ist nicht der richtige Ort für dieses Gespräch.'' Arham sah sich um und überlegte, was er 
sagen konnte. Dann fügte er hinzu: ``Wir müssen wissen, wer auf unserer Seite steht. Ohne, dass 
die Priester es merken. Ich habe mit einigen Leuten gesprochen. Und einige Briefe geschrieben. Für 
die brauche ich deine Unterschrift.''\\
Renec runzelte die Stirn. ``Meine? Wieso?'', fragte er misstrauisch.\\
``Du bist A'Riks Sohn.''\\
``Aber du wirst mir nicht sagen, was in diesen Briefen steht?''\\
``Wenn alles nach Plan geht, musst du es nicht wissen.''\\
``Und wenn dein Plan schief geht, werde ich wegen Verrat geköpft.''\\
``Eher gehängt, du bist immerhin nur ein Bastard.''\\
Renec verzog das Gesicht und schloss einen Moment die Augen. ``Ich bin mir nicht sicher, ob es das 
wert ist. Ich weiß einfach gerne, wofür ich hingerichtet werden könnte.''\\
Arham seufzte und flüsterte grinsend: ``Du bist ein lästiger Junge! Die Magie eines Predigers ist 
es, dass er mit Überraschungen arbeitet. Ich komme mit zur Burg, will mir das Mädchen mal selbst 
ansehen. Dann zeige ich dir die Briefe.''\\

``Oh, das erste was ich in der Stadt machen werde, ist auf den Markt zu gehen um Kleidung für den 
kleinen Schatz zu besorgen!'', plapperte Suja, während sie eine Decke faltete und in eine der 
Holztruhen legte. ``Darum kümmerst du dich gerade viel zu wenig! Du solltest viel mehr an das Kind 
denken.''\\
``Alles'', unterbrach Sarimé ihre Schwägerin: ``was ich gerade denke, hat mit diesem Kind zu 
tun!''\\
Suja betrachtete sie nachdenklich und schüttelte unzufrieden den Kopf. ``Aber es sind nicht die 
Gedanken, die eine werdende Mutter haben sollte. Du solltest dich fragen, welches Geschlecht es 
haben wird. Ob rotes Haar oder schwarz. Oh hoffen wir, wenn es ein Junge wird, dass er mehr nach 
seinem Vater kommt... ein kleiner Handwerker oder Kaufmann ist ja noch eine Sache, aber stell dir 
mal einen kleinen Grafen vor!'', kicherte Suja.\\
``Es ist egal, welches Geschlecht es hat. Es ist so oder so Evins Erbe.''
Die blonde Frau seufzte tief. ``Vielleicht bist du noch nicht reif genug dafür...''\\
\textit{Zu jung. Zu unreif. Zu naiv}, dachte Sarimé bitter und warf den Handspiegel in eine der 
offenen Truhen. ``Ich habe eine Grafschaft zu verwalten'', fauchte sie: ``Und ständig kommen mir 
irgendwelche Männer in die Quere, die mir sagen wollen, wie und was und wann ich dieses und jenes 
zu tun habe! Oh, ganz zu vergessen von dem Krieg und dem Feindesland, das hier ziemlich in der nähe 
ist! Und die Generäle, die nicht mal so tun, als würden sie mir Höflichkeit oder gar Respekt 
erweisen. Ach ja... und dann bin ich ja auch noch schwanger und werde in wenigen Wochen aufgebläht 
wie eine Sau sein! Und das kann man nicht schön reden, versuch es nicht einmal, Suja! Ich habe dich 
gesehen, als du schwanger warst. Und meine Stiefmutter auch viel zu oft. Da kann man faseln wie man 
will, dass das Leuchten des Glücks einen begleitet, man sieht einfach nur fett und müde aus!''\\
Suja räusperte sich. ``Das sind nur die Stimmungen einer werdenden Mutter...''\\
Sarimé stampfte wütend auf. ``Ich weiß!''\\
Da bemerkte sie, dass Sujas Worte nicht an sie gerichtet waren und folgte ihrem Blick zur Tür. Dort 
stand ein drahtiger Mann in legerer Reisekleidung. Sarimé konnte auf dem ersten Blick nicht sagen, 
was genau es war, dass sein Äußeres exotisch aussehen ließ. Vielleicht war es auch nur die Ruhe in 
seinem Blick. ``Was platzt du hier herein?'', brauste sie auf: ``Ich hätte mich gerade umkleiden 
können!''\\
Jetzt zuckten seine Lippen zu einem flüchtigen lächeln. ``Ich soll Euch ausrichten, dass Rotan 
Àrell eingetroffen ist. Der neue Stallmeister.''\\
Sarimé kniff die Augen zusammen. Momentan wuselten viel zu viele Fremde in der Burg umher. ``Wo ist 
Renec? Er wollte dabei sein. Und wer bist du? Gehörst du zu den werten Priestern?''\\
Dass es einer von Samos Leuten war, konnte sie ausschließen. Ihr Hauptmann stellte ihr jede neue 
Wache persönlich vor. Er schüttelte den Kopf und erklärte: ``Ich bin im Dienste des 
Kommandanten.''\\
``Verstehe.'' Sarimé überlegte kurz und fügte hinzu: ``Ich hoffe dir ist klar, dass diese Räume 
meine Privatsphäre widerspiegeln. Was hier geschieht, was ich hier sage, verlässt mein Gemach auch 
nicht!''\\
Suja mischte sich ein. ``Genau. Sie ist immer nur hier zickig.''\\
Sarimé kniff die Lippen zusammen und atmete tief durch, ehe sie sich auf den Weg zum Stall machte.\\

Der Saleicaner verneigte sich tief vor seiner Gräfin und wiederholte erneut seinen Namen und welche 
Ehre es für ihn war. Sarimé erwiderte die Begrüßung höflich und blickte über den Hof. Dutzende 
Männer und Frauen wuselten herum. Einige gehörten zu den Arbeitern, die die Mauern ausbesserten. 
Andere packten schon eifrig für die Reise in die Stadt. Und wieder andere gehörten eindeutig zu 
Mi'Kaés Soldaten, die herumlungerten und lautstark scherzten. ``Und wo sind nun Ihre Pferde?'', 
fragte Sarimé: ``Ich würde gerne ein paar Beispiele Eurer Zucht sehen, bevor ich eine Entscheidung 
fälle.''\\
Mit diesen Worten nickte sie dem alternden Stallmeister zu, denn er hatte sie gebeten, dass sein 
potentieller Nachfolger gut geprüft werden solle.\\
Rotan Àrell stammelte etwas haltlos umher und Sarimé betrachtete die drei Tiere hinter ihm. Fragend 
sah sie sich nach dem Stallmeister um.\\
``Nicht sehr eindrucksvoll'', bemerkte dieser: ``Wieso hat der Fuchs solche Furcht?''\\
``Die Stute ist noch nicht eingeritten'', murmelte Àrell.\\
