\chapter{Die Segnung}

Sarimé stützte sich auf den marmorierten Fenstersims auf und ließ ihren Blick über die Stadt 
schweifen. Die Abendsonne würde bald hinter dem Horizont versinken. Die Tage wurden mittlerweile 
schon spürbar kürzer. Einzelne Sonnenstrahlen bündelten sich am Horizont und huschten um die Ecken 
und Kanten der Gebäude. Das Mädchen hatte freie Sicht auf den auf einem Hügel gelegenen Tempel, in 
dem in wenigen Minuten die Segnung stattfinden würde.\\
Die letzten Tage hatte Sarimé kaum eine ruhige Minute gehabt. Entweder war sie von Priestern oder 
Politikern umgeben. Ansonsten war Suja an ihrer Seite - und die ließ sich unmöglich abschütteln. 
Sarimé war dankbar die erfahrene Mutter an ihrer Seite zu haben. Schon allein die Vorstellung sich 
bei Fragen über die Schwangerschaft an die Priester wenden zu müssen, bereitete ihr Gänsehaut. Und 
das war noch das Beste an ihrer Schwägerin, Sarimé brauchte überhaupt nicht zu fragen, Suja sprach 
von sich aus immer wieder Themen an und erläuterte dies und das.\\
Einen Nachmittag hatte Sarimé sich jedoch frei genommen und war untergetaucht. Auch wenn Samos sich 
nicht hatte abhalten lassen, sie unauffällig zu begleiten. Sarimé hatte ihr Haar hinter einem 
Kopftuch verborgen und sich in einfache, farblose Kleidung einer Handwerkersfrau gehüllt. Durch die 
Gassen und Marktstraßen, durch Tavernen und Hinterhöfe war sie geschlendert und hatte Augen und 
Ohren offen gehalten. Und was sie alles gesehen und gehört hatte...\\
Flüche und Worte der Verdammung, Sorgen und Ängste. Wie sollte ein Kind Merandila im Krieg 
beschützen können? Die Gräfin war so dürr, war sie überhaupt schwanger? Und wenn, das bleiche 
Mädchen sieht doch jetzt schon krank aus, Osyma wird sie bei der Geburt zu sich holen. Wir können 
nur beten, dass der Erbe überlebt!\\
Sarimé hätte den Sprechern nur zustimmen können. Seltsamerweise tat es ihr gut, diese Worte zu 
hören. Es zeigte ihr, dass ihre Sorgen berechtigt waren. Sie war nicht die einzige, die fürchtete, 
die Geburt vielleicht nicht zu überleben. Die sich sorgte, wie sie gegen Kasir bestehen solle. Aber 
die verborgene Gräfin hörte auch viele andere Stimmen. Die Leute bemühten sich nicht mehr, ihren 
Unmut gegen die Priester zu verbergen. Laut schimpften sie, prügelten sich mit denen, die ihnen 
widersprachen. Wirte baten die Priester ihre Gaststätte zu verlassen, ehe etwas geschieht. Und dann 
noch das Geflüster an den Straßen. Worte, die eigentlich nicht gesagt werden sollten. Es war, als 
würde der Wind sie tragen. Sie huschten durch die schmalsten Gassen, über die Dächer der 
Mittelschicht. Sie verfolgten den Adel, deren Kutschen mit schnellen Pferden Na'Rash verließen.\\
\textit{Bald wird der König diese Worte hören. Die Priester wissen es vermutlich längst}, dachte 
Sarimé. Sorge erfüllte sie, aber auch ein seltsames Gefühl der Sturheit. Was, wenn diese Worte wahr 
sind? Was, wenn es wirklich ihre Bestimmung ist?\\
Das Volk teilte längst die Menschen in Saleicaner und Merandil. Adel und Handwerker. Das Feuer 
Osymas und die Wellen der Hellen.\\
``Die Helle lebt'', das waren diese Worte, die der Wind mit sich trug. Zart, wie das Knistern 
wärmender Flammen. Wie das Geräusch des Schnees, der bereits vom Himmel fiel, ohne auf der 
Erde liegen zu bleiben. ``Und sie ist mit uns.''\\
Sarimé presste ihre gefaltete Faust gegen die Stirn. Mit geschlossenen Augen murmelte sie: ``Sei 
mit mir...''\\
Die Tür fiel hinter ihr geräuschvoll ins Schloss und Sarimé seufzte. Die Momente der Stille waren 
vorbei.\\
``Du bist dumm und leichtsinnig!'', rief Renec: ``Wieso schleichst du dich aus dem Anwesen?''\\
``Weil ich die Gräfin Merandilas bin'', erwiderte Sarimé: ``Und deshalb brauche ich mich auch nicht 
zu rechtfertigen.''\\
Sie wandte sich ihm zu und war überrascht über sein nervöses Auftreten.\\
``Wir haben doch erst darüber gesprochen, in welcher Gefahr du dich befindest und dass du 
vorsichtig sein musst!''\\
Sarimé schwieg und kniff die Lippen zusammen. Sie hatte wirklich nicht vor, sich vor dem Bastard zu 
verteidigen. Stattdessen blickte sie in den Spiegel und betrachtete das zarte Kleid, welches die 
leichte Wölbung ihres Bauches verbarg. Sogar bei ihrer Kleiderwahl hatte sie mit Renec gestritten. 
Er meinte, sie solle zeigen, dass sie ein Kind in sich trug. Aber er verstand nicht, dass sich 
einfach noch nicht in der Lage fühle, stolz ihren wachsenden Bauch zu präsentieren. Bei der Farbe 
hatte er aber nicht abweichen wollen. Und so trug sie nun ein weißes Kleid, welches ihr Haar noch 
flammender und ihre Haut noch blasser wirken ließ. Auf Schmuck hatte sie verzichtet und das Haar 
war zu einem lockeren Zopf geflochten, welcher ihr über die Schulter nach vorn fiel. Sarimé fand 
sich selbst so zart. Ein hilfloses Kind, welches zweifelnd zwischen den Mächtigen hin und her sah.\\
\textit{Weiß ist nicht meine Farbe.}\\
``Wieso klopfst du mittlerweile nicht mal mehr an?'', fragte sie.\\
Renec eilte unruhig durch den Raum und sah sie nur kurz an. ``Ach... wozu? Ich werde dich schon 
nicht mit einem Geliebten erwischen. Außerdem glauben eh schon alle, dass ich derjenige bin.''\\
Sarimé funkelte ihn aus ihren grünen Augen an. ``Natürlich. Dein Ruf als Bettgefährte der Gräfin 
war schon vor mir da.''\\
``Warum so pikiert? Du warst es, dir mich geküsst hat'', spottete er.\\
Die junge Gräfin legte den Kopf schief und versuchte zu ergründen, woher dieses Verhalten von ihm 
nun kam. War er so nervös wegen des heutigen Abends, dass er sich nur mit spöttischen Worten zu 
verteidigen versuchte?\\
``Wieso sollte ich dir mein Herz schenken?'', erwiderte sie kalt: ``Du hast deinen Vater betrogen 
und belogen. Du hast dich in seine Familie eingeschlichen, nur um seine Gattin zu vögeln. Ich 
verzichte darauf. Außerdem hat der Windgeist dein Herz mit sich genommen.''\\
Renec blieb wie erstarrt stehen und sah sie schockiert an. Dann verfinsterte sich sein Gesicht. 
``Weiß du... ich wollte eigentlich nur kurz vorbei schauen, um dir etwas zu geben.''\\
Sarimé zuckte mit den Schultern. ``Dann tu das.''\\
Renec kramte ein Pergament aus seiner Jacke und warf es ihr vor die Füße. ``Ich hoffe, dass auch du 
heute Abend nur nervös bist und dich nicht wirklich in so ein Biest verwandelst'', erwiderte er.\\
Die junge Gräfin sah nur kurz auf den Brief und ignorierte ihn. Kurz biss sie sich auf die Zunge 
und antwortete dann: ``Es tut mir leid. Meine Worte waren zu hart. Aber ich will wirklich nicht, 
dass man mir unterstellt, du wärst mein Geliebter.''\\
``Ist die Vorstellung so gruselig?'' Immerhin war ein Hauch des vertrauten Spottes in seine Stimme 
zurückgekehrt.\\
Und wieder versuchte Sarimé ihn mit anderen Augen zu sehen. Sie wollte herausfinden, wie Sieva ihn 
wohl gesehen haben mag. ``Ich glaube immer noch nicht an die Liebe'', erwiderte sie: ``Aber ich 
sehe sie in deinen Augen, glaube ich. Wenn es um Sieva geht.''\\
``Nein'', erwiderte er: ``Das ist Hass. Sie hat sich umgebracht. Das verzeihe ich ihr nicht.''\\
Und wieder sah er so zornig und doch verletzt aus. Langsam trat sie auf ihn zu und berührte ihn 
sanft an der Schulter. ``Gibt es da wirklich einen Unterschied? Aber ich glaube mitterlweile, dass 
ich mir keine Sorgen machen muss, dass ich mich in dich verlieben könnte. Du hast keinen Platz in 
deinem Herzen frei.''\\
``Sorge?'' Er lachte leise und nahm ihre Hand in die seine. ``Weißt du, Sarimé... es gibt viele 
Leute in Merandila, die gerne sehen würden, dass du dich in mich verliebst.''\\
Sie blickte zu ihm auf und grinste schief. ``Ich weiß. Auch das habe ich gestern in der Stadt 
gehört. So viele hätten lieber den Bastard als Grafen. Ich bin nicht dumm, Renec. Auch wenn ich 
wohl kaum jemanden davon überzeugen kann. Was wird denn geplant? Sollst du charmant zu mir sein? 
Mein Herz stehlen? Den treu sorgenden Vater meines Kindes spielen, bis ich dir verfalle?''\\
``Wenn es dich nicht stört, könntest du ja einfach mitspielen. Gemeinsam ist es leichter zu 
lügen.''\\
``Willst du mir denn sagen, wer das ist, der uns als seine Schauspieler haben möchte? Willst du mir 
deine Geheimnisse verraten?'', fragte sie leise.\\
Sein Schweigen war Antwort genug und lachend schüttelte sie den Kopf. ``Nein. Und ich erzähle auch 
nicht die meinen. Bemühe dich aber ruhig, deinem Auftraggeber alles recht zu machen. Tue was du 
willst, ich werde mich nicht in dich verlieben. Ich weiß, welcher Geist durch deine Träume tanzt, 
wessen Blick du auf dich spürst und welcher Name durch deine Gedanken flüstert. Du kannst es Hass 
nennen, aber ich glaube dir nicht.''\\
Renec sah ihr weiterhin in die Augen und löste den Griff um ihre Hand nicht. ``Gut. Dann habe ich 
ja deinen Segen dabei, dich zu verführen. Gut wäre es, wenn ich das schriftlich bekommen würde. 
Nicht, dass du dann alles leugnest, wenn du dich doch in mich verliebt hast. Der Brief ist 
übrigens von deinen Vater. Ich habe ihn abgefangen, ehe Mi'Kae oder einer der Priester ihn sich 
schnappen konnte. Und ich habe ihn auch gelesen. Süßer Kosename, den du da hast.''\\
Sarimé erstarrte und wollte ihre Hand zurückziehen, aber er ließ es nicht zu. Stattdessen kam er 
näher und flüsterte in ihr Ohr. ``Er nennt dich Rose. Ich schätze, wegen deinem Haar? Rose 
Merandilas. Ich finde den Namen ziemlich passend. Er passt zu dir. Sei nicht überrascht, wenn in 
einigen Tagen auch diese Worte in den Straßen Na'Rashs geflüstert werden.''\\
``Lass mich los, Renec'', erwiderte sie und wich seinem Blick aus.\\
``Wieso hast du nicht gesagt, dass du gestern Geburtstag hattest? Deinen sechzehnten.''\\
``Weil alle dachten, ich wäre bereits sechzehn.''\\
``Ja. Weil du sonst noch nicht hättest heiraten dürfen. Weil mein Vater dich sonst noch nicht hätte 
ins Bett rufen dürfen. Du warst noch nicht bereit dafür und hast trotzdem gelogen. Vielleicht bist 
du doch nicht so schlau wie du denkst'', sagte er.\\
``Stell dir vor, ich wäre ehrlich gewesen. Dann wäre ich nun in Brom-Dalar, in meinem staubigen 
Zimmer. Vermutlich an einem armen Handwerker verheiratet'', antwortete die Gräfin: ``Wo ist der 
unterschied, ob ich mit fünfzehn von einem Grafen oder mit sechzehn von einem Schuster ins Bett 
gezwungen werde? Genau. Eine ganze Grafschaft, die mir gehört.''\\
``Du meins, die du verwaltest'', verbesserte Renec lächelnd.\\
``Natürlich'', stimmte sie zu.\\
``Darf ich dir trotzdem ein Geschenk geben?'', fragte er.\\
Das Mädchen nickte stumm. Sie hielt still, während er sich zu ihr beugte. Langsam. Wartete darauf, 
dass sie ein weiteres Mal zurückweichen würde. Sie blickten einander in die Augen, während sich 
ihre Lippen trafen.\\
``Darauf hätte ich dankend verzichten können'', flüsterte sie.\\
``Oh, das war nicht das Geschenk. Das war nur ein weiterer Schritt dich zu verführen. Das 
eigentliche Geschenk wirst du an den Gaben erkennen, die deine Gäste heute zum Fest bringen. Du 
kannst jeden deiner Gäste einen Freund nennen, wenn sein Geschenk die Farbe der Hellen hat. Das ist 
das Zeichen, mit dem sie dir zeigen werden, dass sie dich unterstützen.''\\
``Du schenkst mir also Freunde? Wie nett...''\\
``Habe ich eigentlich schon gesagt, wie hübsch du aussiehst? Ich mag auch deinen Bauch... er 
bedeutet nämlich, dass du vorerst nicht heimtückisch vergiftet wirst. Der Gedanke gefällt mir'', 
schmeichelte er.\\
``Geh jetzt'', forderte sie.\\
``Wirfst du sonst ein Buch nach mir?''\\
Sarimé lächelte breit und rief laut: ``Samos!''\\
Er vergingen keine drei Sekunden, da knallte die Tür gegen die Wand und der Hauptmann ihrer Wache 
stürzte mit gezückten Schwert in den Raum. Er stand zwar still, senkte die Klinge jedoch nicht, als 
er Renec erkannte.\\
Sarimé hob das Pergament vom Boden auf. ``Danke für den Brief, Renec. Nimm dir den Abend frei, ich 
möchte dich nicht im Tempel sehen.''\\
Renec sah sie mit offenem Mund an. ``Aber die Generäle...!''\\
``Werde ich mit General Mi'Kae empfangen. Ich wüsste nicht, wo ich dabei deine Dienste benötigen 
würde. Habe noch einen schönen Abend.''\\
Samos kickte die Tür mit einem Fuß hinter Renec zu und sah sich noch einmal gründlich im Raum um. 
``Irgendwann stech ich ihn ab'', murrte er.\\
``Nein. Wir vertrauen ihm.''\\
``Ach ja?''
Sarime betrachtete das gebrochene Wachssiegel des Briefs. ``Nur pass auf, dass er nicht mehr 
alleine in mein Gemach oder Arbeitszimmer kommt.''\\
``Also doch abstechen?''
Die Gräfin blickte ihn tadelnd an. ``Sei nicht so selbstsicher. Ich vermute, dass ist nur 
vorübergehend. Irgendjemand hat sich eingemischt. Er hat kein Interesse an mir. Nicht auf diese 
Art, die er gerade spielt. Ich muss noch herausfinden, wem er es vorspielt... wer ihn als Grafen 
will... es kann kein Saleicaner sein und auch kein Priester. Bastarde gehören nicht in solche 
Positionen. Ich habe keine Zeit mich auch noch damit zu beschäftigen.'' Sarimé seufzte schwer: 
``Reichen denn Generäle und Priester nicht?''\\
\textit{Vielleicht sollte ich einfach mitspielen... Es wäre leichter, wenn ich die Pflichten nicht 
alleine tragen muss. Ich muss ihn ja nicht wirklich lieben. Es gibt schlechtere Männer als ihn.}\\


