\chapter{Die Segnung}

Sarimé stützte sich auf den marmorierten Fenstersims auf und ließ ihren Blick über die Stadt 
schweifen. Die Abendsonne würde bald hinter dem Horizont versinken. Die Tage wurden mittlerweile 
schon spürbar kürzer. Einzelne Sonnenstrahlen bündelten sich am Horizont und huschten um die Ecken 
und Kanten der Gebäude. Das Mädchen hatte freie Sicht auf den auf einem Hügel gelegenen Tempel, in 
dem in wenigen Minuten die Segnung stattfinden würde.\\
Die letzten Tage hatte Sarimé kaum eine ruhige Minute gehabt. Entweder war sie von Priestern oder 
Politikern umgeben. Ansonsten war Suja an ihrer Seite - und die ließ sich unmöglich abschütteln. 
Sarimé war dankbar die erfahrene Mutter an ihrer Seite zu haben. Schon allein die Vorstellung sich 
bei Fragen über die Schwangerschaft an die Priester wenden zu müssen, bereitete ihr Gänsehaut. Und 
das war noch das Beste an ihrer Schwägerin, Sarimé brauchte überhaupt nicht zu fragen, Suja sprach 
von sich aus immer wieder Themen an.\\
Einen Nachmittag hatte Sarimé sich jedoch frei genommen und war untergetaucht. Auch wenn Samos sich 
nicht hatte abhalten lassen, sie unauffällig zu begleiten. Sarimé hatte ihr Haar hinter einem 
Kopftuch verborgen und sich in einfache, farblose Kleidung einer Handwerkersfrau gehüllt. Durch die 
Gassen und Marktstraßen, durch Tavernen und Hinterhöfe war sie geschlendert und hatte Augen und 
Ohren offen gehalten. Und was sie alles gesehen und gehört hatte...\\
Flüche und Worte der Verdammung, Sorgen und Ängste. Wie sollte ein Kind Merandila im Krieg 
beschützen können? Die Gräfin war so dürr, war sie überhaupt schwanger? Und wenn, das bleiche 
Mädchen sieht doch jetzt schon krank aus, Osyma wird sie bei der Geburt zu sich holen. Wir können 
nur beten, dass der Erbe überlebt!\\
Sarimé hätte den Sprechern nur zustimmen können. Seltsamerweise tat es ihr gut, diese Worte zu 
hören. Es zeigte ihr, dass ihre Sorgen berechtigt waren. Sie war nicht die Einzige, die fürchtete, 
die Geburt vielleicht nicht zu überleben. Die sich sorgte, wie sie gegen Kasir bestehen solle. Aber 
die verborgene Gräfin hörte auch viele andere Stimmen. Die Leute bemühten sich nicht mehr, ihren 
Unmut gegen die Priester zu verbergen. Laut schimpften sie, prügelten sich mit denen, die ihnen 
widersprachen. Wirte baten die Priester ihre Gaststätte zu verlassen, ehe etwas geschieht. Und dann 
noch das Geflüster an den Straßen. Worte, die eigentlich nicht gesagt werden sollten. Es war, als 
würde der Wind sie tragen. Sie huschten durch die schmalsten Gassen, über die Dächer der 
Mittelschicht. Sie verfolgten den Adel, deren Kutschen mit schnellen Pferden Na'Rash verließen.\\
\textit{Bald wird der König diese Worte hören. Die Priester wissen es vermutlich längst}, dachte 
Sarimé. Sorge erfüllte sie, aber auch ein seltsames Gefühl der Sturheit. Was, wenn diese Worte wahr 
sind? Was, wenn es wirklich ihre Bestimmung ist?\\
Das Volk teilte längst die Menschen in Saleicaner und Merandil. Adel und Handwerker. Das Feuer 
Osymas und das Licht der Hellen.\\
``Die Helle lebt'', das waren diese Worte, die der Wind mit sich trug. Zart, wie das Knistern 
wärmender Flammen. Wie das Geräusch des Schnees, der bereits vom Himmel fiel, ohne auf der 
Erde liegen zu bleiben. ``Und sie ist mit uns.''\\
Sarimé presste ihre gefaltete Faust gegen die Stirn. Mit geschlossenen Augen murmelte sie: ``Sei 
mit mir...''\\
Die Tür fiel hinter ihr geräuschvoll ins Schloss und Sarimé seufzte. Die Momente der Stille waren 
vorbei.\\
``Du bist dumm und leichtsinnig!'', rief Renec: ``Wieso schleichst du dich aus dem Anwesen? Ich 
dachte dafür hast du dir diese Kasira ins Haus geholt!''\\
``Weil ich die Gräfin Merandilas bin'', erwiderte Sarimé: ``Und deshalb brauche ich mich auch nicht 
zu rechtfertigen.''\\
Sie wandte sich ihm zu und war überrascht über sein nervöses Auftreten.\\
``Die Tempelwache streift durch die Straßen und schnappt sich jeden, dem sie auch nur einen Hauch 
von Ketzerei vorwerfen können!''\\
Sarimé schwieg und kniff die Lippen zusammen. \textit{Er hat Angst.}\\
Statt einer Erwiderung blickte sie in den Spiegel und betrachtete das zarte Kleid, welches die 
leichte Wölbung ihres Bauches verbarg. Sogar bei ihrer Kleiderwahl hatte sie mit Renec gestritten. 
Er meinte, sie solle zeigen, dass sie ein Kind in sich trug. Aber er verstand nicht, dass sich 
einfach noch nicht in der Lage fühlte, stolz ihren wachsenden Bauch zu präsentieren. Bei der Farbe 
hatte er aber nicht abweichen wollen. Und so trug sie nun ein weißes Kleid, welches ihr Haar noch 
flammender und ihre Haut noch blasser wirken ließ. Auf Schmuck hatte sie verzichtet und das Haar 
war zu einem lockeren Zopf geflochten. Sarimé fand sich selbst so zart. Ein hilfloses Kind, welches 
zweifelnd zwischen den Mächtigen hin und her sah.\\
\textit{Weiß ist nicht meine Farbe.}\\
``Wieso klopfst du mittlerweile nicht mal mehr an?'', fragte sie.\\
Renec eilte unruhig durch den Raum und sah sie nur kurz an. ``Ich dachte, darüber sind wir beide 
mittlerweile hinaus'', spottete er: ``Nach den Gerüchten sind wir eh Geliebte.''\\
Sarimé funkelte ihn aus ihren grünen Augen an. ``Natürlich. Dein Ruf als Bettgefährte der Gräfin 
war schon vor mir da.''\\
``Warum so pikiert? Du warst es, die mich geküsst hat'', lachte er und fuhr sich durch das Haar: 
``Oder diejenige, die sich nachts in mein Zimmer geschlichen hat.\\
Die junge Gräfin legte den Kopf schief und versuchte ihn ergründen. \textit{Ich mache ihn 
nervös...}\\
''Ich habe einen Brief für dich.'' Renec reichte ihr das zusammen gefaltete Pergament. ``Ist die 
Vorstellung, ich wäre dein Geliebter, so gruselig?'', fügte er hinzu und trat näher.\\
``Du hast keinen Platz in deinem Herzen für mich frei. Sie muss wirklich ein Windgeist gewesen 
sein, wenn sie dich immer noch umklammert'', erwiderte Sarimé und hob den Kopf, um Renec in die 
Augen blicken zu können.\\
``Und dann bist du die Wärme, die den Windgeist vertreibt.'' Er murmelte die Worte leise in ihr 
Haar und zog sie sanft in seine Arme.\\
``Es gibt nicht wenige Menschen in Na'Rash, die sich wünschen würden, dass ich dich liebe. Heirate. 
Zum Grafen mache'', flüsterte sie zurück: ``Ich habe ihr Wispern gehört.''\\
``Wenn es dich nicht stört, können wir ja nur so tun als ob'', schlug er leise vor und küsste ihre 
Stirn.\\
Sarimé hielt still. ``Willst du mir denn sagen, wer das ist, der das alles plant?''
Wie immer, wenn er nur durch eine Lüge den Ausweg finden würde, schwieg Renec. Sarimé vermutete 
mittlerweile, dass Renec einfach zu lange überlegen musste, bis ihm eine passable Antwort einfiel. 
Bestimmt schob sie ihn von sich und er ließ es widerstandslos geschehen. ``Nun, dann gibt es nichts 
mehr zu sagen.''\\
Eilig ergriff Renec ihre Hand um zu verhindern, dass sie sich völlig abwandte. ``Der Brief ist 
übrigens von deinen Vater. Ich habe ihn abgefangen, ehe Mi'Kae oder einer der Priester ihn sich 
schnappen konnte. Und ich habe ihn auch gelesen. Süßer Kosename, den du da hast.''\\
Sarimé erstarrte und wollte ihre Hand zurückziehen, aber er ließ es nicht zu. Stattdessen kam er 
näher und flüsterte in ihr Ohr. ``Er nennt dich Rose. Ich schätze, wegen deinem Haar? Rose 
Merandilas. Er passt zu dir. Sei nicht überrascht, wenn in einigen Tagen auch diese Worte in den 
Straßen Na'Rashs geflüstert werden.''\\
``Lass mich los, Renec'', erwiderte sie und wich seinem Blick aus.\\
``Wieso hast du nicht gesagt, dass du gestern Geburtstag hattest? Deinen sechzehnten.''\\
``Weil alle dachten, ich wäre bereits sechzehn.''\\
``Ja. Weil du sonst noch nicht hättest heiraten dürfen. Weil mein Vater dich sonst noch nicht hätte 
ins Bett rufen dürfen. Du warst noch nicht bereit dafür und hast trotzdem gelogen. Vielleicht bist 
du doch nicht so schlau wie du denkst'', sagte er.\\
``Stell dir vor, ich wäre ehrlich gewesen. Dann wäre ich nun in Brom-Dalar, in meinem staubigen 
Zimmer. Vermutlich an einem armen Handwerker verheiratet'', antwortete die Gräfin: ``Wo ist der 
Unterschied, ob ich mit fünfzehn von einem Grafen oder mit sechzehn von einem Schuster ins Bett 
gezwungen werde? Mein altes Blut würde im Schatten der Hauptstadt verrinnen. Meine Kinder durch 
Staub und Dreck kriechen. Der Unterschied ist eine ganze Grafschaft, die mir gehört.''\\
``Du meins, die du verwaltest'', verbesserte Renec lächelnd.\\
``Natürlich'', stimmte sie zu.\\
``Darf ich dir trotzdem ein Geschenk geben?'', fragte er.\\
Das Mädchen nickte stumm. Sie hielt still, während er sich zu ihr beugte. Langsam. Darauf 
wartend, dass sie ein weiteres Mal zurückweichen würde. Sie blickten einander in die Augen, während 
sich ihre Lippen trafen.\\
``Durch die Segnung deines Kindes, wird erst in den nächsten Tagen eine Andacht der Hellen 
stattfinden'', flüsterte er schnell und leise: ``Er hatte in den letzten Wochen viel zu tun.''\\
``Du nimmst mich also mit?'', fragte Sarimé überrascht.\\
Renec legte einen Finger an die Lippen und bedeutete ihr, leise zu sein. ``Und er hat auch ein 
Geschenk für dich.''\\
``Seinen eigenen Segen für das Kind?'', riet Sarimé.\\
``Die Helle hält von so etwas nichts. Die Seelen finden alleine ihren Weg.''\\
Sarimé runzelte die Stirn, als ihr bewusst wurde, wie wenig sie von dieser fremden Göttin wusste. 
``Also? Das Geschenk?''\\
``Freunde.''\\
Einen langen Moment sah sie ihn nur stumm an, dann lachte sie und trat zurück.\\
``Achte auf die Gaben im heiligen Feuer'', flüsterte er hastig: ``Einige Zeichen werden die Helle 
ehren.''\\
Sie trat zurück an das Fenster und blickte grimmig hinaus. Das ging ihr zu schnell. Es wurde zu 
viel verheimlicht. Ihr verdacht ruhte immer noch darauf, dass Ga'Leor dahinter steckte. An Verrat 
glaubte sie weniger. Zumindest nicht von Renecs Seite aus. Aber wer weiß, ob er nicht nur belogen 
wurde?\\
``Habe ich eigentlich schon gesagt, wie hübsch du aussiehst? Ich mag auch deinen Bauch... er 
bedeutet nämlich, dass du vorerst nicht heimtückisch vergiftet wirst. Der Gedanke gefällt mir'', 
schmeichelte er.\\
``Geh jetzt'', forderte sie.\\
``Wirfst du sonst ein Buch nach mir?''\\
Sarimé konnte nicht verhindern, dass er sie nun doch zum Lachen brachte. ``Samos'', fügte sie laut 
hinzu und wartete grinsend. Er vergingen keine drei Sekunden, da knallte die Tür gegen die Wand und 
der Hauptmann ihrer Wache stürzte mit gezückten Schwert in den Raum. Er stand zwar still, senkte die 
Klinge jedoch nicht, als er Renec erkannte.\\
Sarimé hob das Pergament vom Boden auf. ``Danke für den Brief, Renec. Hab noch einen schönen 
Tag!''\\
Mit einer Grimasse verließ der Bastard den Raum. Samos kickte die Tür mit einem Fuß hinter Renec zu 
und sah sich noch einmal gründlich um. 
``Irgendwann stech ich ihn ab'', murrte er.\\
``Nein.''\\
``Ach ja?''
Sarime betrachtete das gebrochene Wachssiegel des Briefs. ``Nur behalte ihm im Auge.''\\
``Also doch abstechen?''
Die Gräfin blickte ihn tadelnd an. ``Sei nicht so selbstsicher. Ich vermute, dass ist nur 
vorübergehend. Irgendjemand hat sich eingemischt.  Ich muss nur noch herausfinden, wer ihn als 
Grafen will... es kann kein Saleicaner sein und auch kein Priester.'' Sarimé seufzte schwer und ihr 
Blick wanderte zurück zum Fenster, erhaschte die Dächer der Stadt. \textit{Sollen sie hoffen und 
fluchen. Aber ich bin die Herrin Merandilas. Und das werde ich bleiben. Alleine.}

``Geschätzte Gräfin'', hallte Em'Hirs Stimme durch den Tempel.\\
Anders als in der Hauptstadt war diese Anlage sehr offen gestaltet. Große Fenster ließen tagsüber 
die Strahlen der Sonne ein und zeigten nachts die Sterne am Himmel. Es gab weniger Nischen, 
stattdessen runde Feuerplätze, an denen die Gläubigen gemeinsam den Predigten lauschen konnten. Zu 
diesem Festtag waren alle Plätze entzündet und Laternen erleuchteten auch die Wände und Decken des 
Tempels. Sarimé zog die Schultern fröstelnd hoch und bereute, sich nicht doch einen Pelz um die 
Schultern gelegt zu haben. Lief sie nahe an den Feuern vorbei, prickelte ihre Haut vor Hitze. 
Abseits davon bereitete die Kälte ihr Gänsehaut.\\
\textit{Während der Herbststurm meinen Gatten noch mit sich nahm, ist nun der Winter 
eingekehrt...}, dachte sie melancholisch. \textit{Werde ich meinen ersten Sommer in Merandila noch 
erleben?}\\
Sarimé setzte ein Lächeln auf, welches einer werdenden Mutter gerecht wurde. Lange Minuten hatte 
sie es mit Suja vor dem Spiegel geübt. Das Mädchen strahlte die Gäste an, welche sich verneigten 
und ihr Platz schafften um zum Podest des Hohepriesters zu gelangen. Sie entdeckte Mi'Kae, Pedan 
und die anderen Stadthalter. Silhe Basra knickste tief, ohne ihren Blick von ihr abzuwenden. Als 
Antwort neigte Sarimé nur lächelnd den Kopf und ihr Blick huschte auch über die zahlreichen Gäste, 
die sie nicht kannte.\\
Em'Hirs Hand war trocken und rau, als er ihr neben sich auf das Podest half. Sein Blick prüfend und 
berechnend, ehe er sich der Menge zuwandte und die Hände in den Himmel reckte. Er begann mit der 
üblichen Einleitung einer Andacht und rief um Osymas Aufmerksamkeit.\\
``Die Nacht ist bereits hereingebrochen und so hoffen wir, dass der Allmächtige uns seine 
Aufmerksamkeit schenken kann. Sieh hier her, Osyma, zu uns in den Tempel, der dir zu Ehren 
errichtet wurde. Dein Heim auf Erden. Deine Flammen brennen für dich. Sieh her und geleite uns. 
Höre uns. Führe uns.''\\
Silhe Basra hatte für die anschließenden Festlichkeiten einen Versammlungssaal der Schule 
herrichten lassen. Ebenso wie einen kleineren, ruhigen Raum in dem die Generäle sich treffen 
würden. Sarimé war sich nicht sicher, inwieweit die Priester von dieser Unterredung wussten. Noch 
hatte keiner ihr gegenüber etwas erwähnt.\\
\textit{Wieso brauchen diese mächtigen Götter eigentlich immer diese langen Reden?}, grübelte sie: 
textit{Nachdem was ich auf der Straße gehört habe, ist die Helle da auch nicht sachlicher.}\\
``Evin A'Rik war ein großer Mann. Er zog in zahlreiche Schlachten um unser Land und unsere Ehre 
zu verteidigen. Osyma segnete ihn mit einem Erben, der seinen Geist in sich tragen wird. Und heute 
sind wir hier, um Osyma zu bitten, auch den Erben unserer Grafschaft zu segnen.''\\
Em'Hir deutete auf das große Feuer, vor dem Podest. ``Zeigt den Flammen, welche Gaben ihr dem 
ungeborenem Leben im Leib unserer Gräfin schenken wollt. So dass Osyma wählen wird, ob er ihm 
diesen Segen gewehrt. Und lasst uns nicht vergessen, dass alles Leben verletzlich ist. Besonders 
Leben, welches noch nicht geboren ist. Lasst uns Osyma um Gnade bitten, für all unsere Sünden. 
Die Gefahr in unserer Welt ist so groß, dass wir sie niemals messen können. Unsere Feinde so 
zahlreich. Und doch ist unsere größte Gefahr, die Selbstzufriedenheit! Wir dürfen uns niemals 
sicher fühlen! Osyma schützt uns nur, wenn wir ihm Ehre bringen und seinen Willen erfüllen! 
Seine Strafe wird hart sein, wenn wir ihm nicht gerecht werden.''\\
Sarimés Mund wurde trocken, während sie Em'Hirs Drohungen lauschte, welche er so offen vor allen 
aussprach. Nur ein Narr würde nicht verstehen, dass er ihr drohte. Ihr und jedem, der dem 
Geflüster der Gassen lauschte. \\
Ihr Blick huschte zu den Tempelwachen, mit ihren polierten Klingen und ehrfürchtigen Blicken. Das 
ging weit über die üblichen bezahlte Arbeit hinaus. Vertraglich dienten sie dem Stadthalter 
Sakan. Er hatte sich geweigert sich ein weiteres mal mit seiner Gräfin an einen Tisch zu 
setzen. Als sie persönlich vor seinem Anwesen stand, wiesen seine Wachen sie ab. Ihr Herr war 
in das Gebet vertieft.\\
Sarimé besann sich eilig und sah zu dem Feuer. Einer nach dem anderen traten die Gäste vor und 
übergaben ihm ihre Geschenke. Hinter jedem steckte eine Gabe, die es symbolisierte. Eine Blume, für 
Schönheit. Ein Schwert für kriegerische Fähigkeiten. Ein Onyx, damit das Kind ein Junge wird. Oder 
Quarz für ein Mädchen. Gold für Verhandlungsfähigkeiten. Ein gesticktes Auge für Weitsicht. Ein 
Felsbrocken für Standhaftigkeit. Der Gräfin gingen Renecs Worte durch den Sinn und sie schüttelte 
kaum merklich den Kopf. Sie versuchte den Drang, die Bedeutung der Geschenke zu entschlüsseln, 
abzuschütteln. Stattdessen suchte sie nun nach jedem weißen Farbfleck, ehe Osymas Flammen es 
verzehrten. Welch interessante Wahl für ein Zeichen. Gib dem hungrigen Gott die Farbe seiner Rivalin 
zu fressen. Bedeutete das nun, dass Osyma aber auch ob siegte und all das, wofür die Helle stand, 
auffraß? Oder würde seine Macht erlöschen, wenn nur genug von ihrem Geist in sein Feuer überging?\\
Drei der sieben Generäle warfen ein Zeichen der Hellen in den Kreis. Sarimé kniff die Augen 
zusammen und versuchte sich ihre Gesichter einzuprägen. General Mi'Kae spendete ein dicke Decke, 
die qualmend verglühte. Sie war nicht weiß. \textit{Vermutlich hat er keine Ahnung, was Renec 
ausgeheckt hat.}\\
Mi'Kae sah zu ihr empor und fügte mit einem schiefen Grinsen hinzu: ``Die merandilischen Winter 
sollen doch so eisig sein.''\\
Sie tat, als hätte dieser Witz nicht schon längst seinen Reiz verloren und lachte kurz auf, ehe sie 
sich wie bei jedem anderen mit einem Nicken und Lächeln bedankte.\\
Silhe Basra verbrannte ein beschriebenes Pergament, sprach aber nichts dazu. Die Worte, die sie 
darauf geschrieben hatte, würden ihr Geheimnis bleiben, welches sie nur mit ihrem Gott teilte. 
Sarimé beobachtete ihre Mimik genau - versuchte eine Ahnung dieses Geheimnisses wahrzunehmen, aber 
vergebens. Das Gesicht der Rektorin glich einer Maske, welche den Inbegriff der distanzierten 
Höflichkeit widerspiegelte. Sarimé ertappte sich selbst dabei, wie sie unruhig ihr Gewicht von 
einem auf das andere Bein verlagerte und bemühte sich, das zu unterlassen. All ihre Pläne hingen 
von Basras Loyalität ab. Ihre Entscheidungen und Taten waren das Fundament, auf dem die Gräfin ihre 
Zukunft baute. Die Generäle, der Hohepriester, die Stadthalter. Sie alle waren die Säulen ihres 
Reichs, aber was brachten ihr wacklige Türme, die im Sumpf versinken würden?\\
``Ich habe auch ein Geschenk an mein Kind'', verkündete Sarimé, als der letzte Gast zurück getreten 
war. So viele Augen, die sich auf sie richteten und ihre bedächtigen Schritte verfolgten. Das 
Mädchen entdeckte die skeptischen und fragenden Blicke, auf der Suche nach einen Gegenstand, den 
sie in den Händen hielt. Aber da war nichts.\\
Sarimé trat dicht an die Flammen und dachte an die Nacht am Meer, als Wasser und Feuer um die 
Überreste ihres Gatten stritten. In ihren Ohren klangen noch die Töne der Feierlichkeiten wider. 
Und das Gefühl der Endgültigkeit, welches sich damals auf ihr Herz legte und nicht mehr gewichen 
war. Sarimé hatte lange überlegt, welche Botschaft sie ihren Widersachern an diesem heutigen Tag 
überbringen wollte und wie sie es gestalten würde. Sie hatte sich von beiden Gottheiten inspirieren 
lassen und richtete ihre Gedanken somit an den Krieger ebenso wie an die Helle.\\
Ihr Geschenk würde denjenigen, die Osyma verehrten zeigen, wie viel sie bereit war zu ertragen, zu 
kämpfen. Dass sie den Inbegriff der Ehre lebte und zur Schau tragen würde. Dass sie eine Löwin war, 
die ihr Junges verteidigte. Und die Merandil würden sich darauf berufen können, dass nur die Helle 
so viel Bescheidenheit und Opferbereitschaft ausstrahlen konnte.\\
Die Gräfin lächelte, während sie ihre linke Hand erhob. Ein Raunen ging durch die Menge und ihr 
Blick suchte Renecs, ehe sie ihre Hand in die Flammen hielt. Sie unterdrückte den Reflex, ihre Hand 
zurück zu ziehen und zählte innerlich bis drei. Dann brachen die Mauern ihrer Disziplin. Sarimé 
hatte keine Sekunde in ihren Überlegungen bezweifelt, welch Qualen die Verbrennungen sein würden, 
trotzdem klang ihr eigener Schrei seltsam fremd in ihren Ohren. Sie krümmte sich und verbarg die 
schmerzende Hand. Ein Stechen, ein Brennen, welches all ihr Denken einnahm. Wage nahm sie wahr, wie 
zwei Arme sie umfingen. Die Stimmen, die laut wurden. Basra stimmte eine Lobpreisung an, in die die 
Priester mit einfielen. Nur Em'Hir starrte Sarimé wortlos vom Podest aus an, als verfluche er sie.\\
``Geht schon'', knirschte Sarimé leise und versuchte, Renec fort zustoßen. Sie musste 
ihren Gästen aufrecht entgegen treten, sonst wäre ihre Opfer umsonst gewesen. Sonst würde ihre 
Botschaft ungehört verklingen. Der Bastard ließ sie nur widerwillig los. Während Sarimé ihre Hand 
gegen die Brust presste, hob sie das Kinn. Tränen glänzten in ihren Augen. Sie rief: ``Osyma, 
gewähre meinem Kind deinen Segen.''\\
Ein letztes Mal sah sie in die Menge, registrierte kaum merkliche Gestiken und vielsagenden 
Blicken, ehe sie hinzufügte: ``Und nachdem wir alle unsere Geschenke übergeben haben, ist nun die 
Zeit gekommen, zu tanzen und zu singen, zu trinken und zu speisen!''\\

``Wie hast du dir selbst eingeredet, dass du klug bist?'', murmelte Renec, während er das in kaltes 
Wasser getränkte Leinentuch in Streifen riss.\\
``Ich verrate dir nicht das Geheimnis meines Selbstvertrauens'', entgegnete Sarimé.\\
Es sollte spöttisch klingen, aber der Schmerz schwang deutlich in ihrer Stimme mit. Renec beugte 
sich ein weiteres mal über ihre Hand und legte vorsichtig den Verband an. Ein Stöhnen kam über ihre 
Lippen, so sehr sie es auch zu vermeiden versuchte.\\
``Es muss dir doch klar gewesen sein, wie das schmerzen wird!  Hätte ich gewusst, dass du so 
einer Dummheit vorhast, hätte ich dich aufgehalten'', antwortete er und befestigte den Verband: 
``Wusstest du es? Wieso hast du sie nicht aufgehalten?''
Samos hatte die Arme vor der Brust verschränkt und sah grimmig drein. Sarimé seufzte. ``Ihr beide 
seid schlimmer als Köter! Wenn ihr so weiter macht, besorge ich mir einen neuen Hauptmann der Garde 
und Bastarde gibt es auch genug in Saleica! `Ich muss los'', sagte sie entschieden und erhob sie.\\
Renec reichte ihr einen Krug mit Schnaps. ``Gegen die Schmerzen.''\\
Mit einem dünnen Lächeln leerte sie das Gefäß und musste sogleich husten. Immerhin überlagerte das 
Brennen ihres Rachens kurzfristig den Schmerz in ihrer Hand.\\
``Die beiden Generäle heißen Solvan und Arton'', erinnerte Renec sie.\\
``Wie lange hast du das geplant? Das mit dem Zeichen? Und woher willst du wissen, dass manche 
Geschenke nicht nur aus Zufall weiß waren?'', fragte Sarimé.\\
Er zuckte nur mit den Schultern. ``Nimm mich mit zur Besprechung und ich verrate es dir.''\\
Sarimé schüttelte nur den Kopf und ließ ihn stehen.\\

``Verzeiht meine Verspätung'', grüßte die Gräfin.\\
Die Generäle erhoben sich, als sie zum Tisch trat und verneigten sich wie es der Höflichkeit 
gehörte. Die Männer warteten, bis Sarimé sich auf einen der freien Stühle nieder ließ, ehe sie sich 
selbst wieder setzten. Mi'Kae räusperte sich: ``Ich denke, in diesem Fall kann man Euch keinen 
Vorwurf machen. Habt Ihr die Wunde sorgfältig behandeln lassen?''\\
Sarimé nickte nur und ging nicht weiter darauf ein. Es gab wichtigere Themen, die angesprochen 
werden mussten. Ihre Botschaft sollte still in den Köpfen der Gäste reifen.\\
``Wir haben bisher über die natürlichen Bedingungen der Grenze gesprochen und auf der Karte 
Bereiche markiert, die sich für Hinterhalte und Schlachten eignen würden'', informierte General 
Arton - eine weißhaariger Frau mit einer schmalen Narbe auf der Wange.\\
Sarimé bedankte sich mit einem Nicken und betrachtete die Karte, die vor ihnen auf dem runden Tisch 
ausgebreitet lag. Sie zeigte ganz Merandil, die Grenze zum Norden hin sowie die nördlichen Bereiche 
der umliegenden Grafschaften. ``Was ist mit den Versorgungswegen?'', erkundigte sie sich.\\
Einer der Generäle deutete auf mehrere Linien. ``Über diese Wege erhalten wir generell Nachschub. 
Sie führen momentan noch direkt zu unseren Garnisonen. Wenn unsere Heere sich in Bewegung setzen, 
bleibt also ein Großteil der Route gleich und variiert erst in den nördlicheren Gegenden.''\\
Das Mädchen sah in die Runde. ``Ich hörte von Deserteuren. Sind diese Gerüchte wahr?''\\
Mi'Kae räusperte sich. ``Nicht im wörtlichen Sinne... die Soldaten sind nicht von ihrem Dienst 
geflohen, sondern hatten es besonders eilig mit dem Krieg zu beginnen. Es gab einige Plünderungen 
in Kasir. Dörfer, deren Bewohner getötet wurden, sowie Patrouillen, die aufgehalten wurden.''\\
``Haben wir Löwen es nötig, gegen unbewaffnete Bauern zu kämpfen?'', fragte Sarimé kalt: ``Solange 
ihre Taten nicht auf Befehle beruhten, haben sie gegen das Wort ihrer Vorgesetzten verstoßen und 
gehören somit bestraft.''\\
``Ich habe bereits begonnen mich darum zu kümmern, dass die Verantwortlichen vor Gericht gestellt 
werden'', erklärte Mi'Kae.\\
``Ich möchte trotzdem, dass meinen Generälen bewusst ist, dass ich von solchen Taten kein weiteres 
Mal etwas hören will... das saleicanische Heer ist so gefürchtet, weil wir tapfere und gute 
Soldaten haben. Aber ein Heer lebt durch Disziplinen und Ordnung... ich denke, dass muss ich Ihnen 
als erfahrene Krieger nicht erklären.''\\
Die Männer schienen nicht erwartet zu haben, dass ihre junge Gräfin solche Worte aussprechen würde, 
nickten aber grimmig.\\
``Ansonsten habe ich nur ein weiteres Anliegen, welches ich mit Ihnen besprechen möchte'', fuhr 
Sarimé versöhnlicher fort: ``Ich baue auf Ihre Unterstützung und Erfahrung. Mir ist bewusst, dass 
ich noch sehr jung bin und bisher kaum etwas mit militärischen Belangen zu tun hatte. Deshalb 
korrigiert mich, falls meine Bitte töricht ist. Saleica hat drei stehende Heere in Merandila und 
zwei weitere in den anliegenden Grafschaften, welche unsere Landesgrenze mit unterstützen.''\\
Sarimé nickte den beiden betroffenen Generälen, welche erst vor wenigen Tagen mit einer kleinen 
Truppe in Na'Rash angelangt sind, zu. ``Wie weit sind Ihre Heere?'', erkundigte sie sich.\\
``Wir packen die Garnisonen gerade zusammen und werden in wenigen Wochen die Hälfte der Grafschaft 
durchquert haben'', erklärte er.\\
``Und was wäre Eure Bitte, Herrin?'', erkundigte sich General Solvan.\\
``Na'Rash ist die wichtigste Stadt in Merandila. Sie ist der Mittelpunkt des Handels, der Religion 
und der Bildung. Wir müssen die Grenze Saleicas verteidigen, das Feindesland erobern! Aber auch 
Na'Rash schützen. Ähnlich wichtig ist unser Hafen an der westlichen Küste, von dem aus unser 
Binnenverkehr erreichbar ist. General Mi'Kae vertraute mir in einem Gespräch an, dass Kundschafter 
bereits kasirische Soldatengruppen die Grenze überqueren sahen - auch wenn sie bisher noch keine 
aktive Kriegshandlung zeigten. Man kann es ihnen schlecht verdenken, immerhin haben saleicansiche 
Deserteure geplündert und gemordet.''\\
``Und Ihr wollt Kasir sagen, dass alles nur ein Versehen war? Ein Verbrechen? Dass die Soldaten 
hingerichtet werden und wir lieber noch ein paar Monate mit dem Krieg warten würden, bis unsere 
Heere auf Position sind?'', spottete General Arton und verschränkte die Arme vor der Brust.\\
``Nein. Ich will, dass bereits ein Heer sich um diese Truppen kümmert und die zwei bald ankommenden 
eine Position an der Grenze einnehmen. Aber wir wissen nicht, wie viele kasirische Soldaten bereits 
über die Grenze geschlüpft sind und vielleicht versuchen im Inneren der Grafschaft zu sabotieren. 
Unsere Versorgungsrouten sind die am Besten ausgebauten Straßen Merandilas, dass kann Kasir sich 
vermutlich auch denken. General Mi'Kae meinte, dass sie dort zuschlagen könnten. Fallen errichten, 
die uns zum Verhängnis werden, wenn es zu den ersten richtigen Schlachten kommen wird.''\\
Die Männer betrachteten nachdenklich die Karte und nickten widerstrebend.\\
``Ich möchte also zwei Heere im Inneren Merandilas haben, solange sie nicht an der Grenze gebraucht 
werden. Wäre jemand von Ihnen dazu bereit?''\\
General Arton und General Solvan tauschten flüchtige Blicke, ehe Letzterer sagte: ``Aber wenn es so 
weit kommt, dass wir an der Grenze gebraucht werden, brechen wir auf ohne auf Euren Segen zu 
warten, Herrin! Ihr sagtet schon, was wir alle hier denken. Ihr seid jung und unerfahren. Wir gehen 
Eurer bitte nach, solange es sinnvoll ist und brechen auf, sobald wir es als notwendig für unseren 
Krieg erachten. Einverstanden?''\\
Die Generäle diskutierten kurz, aber General Leinos - welcher mit seinem Heer als erstes an der 
Grenze sein würde - besiegelte es schließlich mit den Worten, dass es gar nicht so eine schlechte 
Idee sei, den Kasira erst einmal nicht ihre ganze Stärke vorzuführen. ``Nahe der Grenze lassen 
sich für uns vorteilhaftere Schlachten führen als im kasirischen Innland.''\\


