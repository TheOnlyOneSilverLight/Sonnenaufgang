\chapter{Königin}

Der Saum ihres weißen Kleides strich durch den Schutt und die Asche, die das Feuer zurück gelassen 
hatte. Die kühle Wintersonne zeigte sich nicht mehr und der fallende Schnee bedeckte die in Panik 
erstarrten Gesichter der Erstickten. Chaos hatte sich über die Stadt gelegt. Und nach dem Sturm 
folgte die Stille.\\
Die hölzernen Mauern waren nach der Befreiung durch Mi'Kaes Soldaten über Nacht abgebrannt. Das 
Dach eingestürzt. Die Verräter gejagt und die Überlebenden unter Silhe Basras Aufsicht versorgt.\\
Sarimé Sil'Vera stand in Mitten der Ruine. Vor ihren Füßen sah sie das fahle Gesicht eines 
Säuglings. Wortlos nahm sie sich den Pelz eines weißen Luchses von der Schulter und bedeckte das 
Kind.\\
``Sarimé, flüsterte Renec leise und griff nach ihrem Ellbogen.\\
Sie ließ ihn gewähren, ohne den Blick aus ihren grünen Augen vom Tod zu nehmen. Renec zog seine 
Hand zurück. Beklommen betrachtete er sie. Sie war noch immer so jung wie das Kind, dass damals im 
Hof der Burg stand. Nur hatte er sie falsch eingeschätzt. Da stand sie. Aufrecht. Statt Tränen und 
Kummer, lag Stolz und Zorn in ihren Augen.\\
''Wo ist Em'Hir?``\\
''Er wurde noch nicht gefunden. Mi'Kaé ist noch unterwegs``, antwortete Renec.\\
''Sakan?``\\
''Unter Gewahrsam``, murmelte Samos mit gesenktem Blick.\\
Sarimé nickte bedächtig. ''Wie geht es Lavay?``, fragte sie.\\
Sie hatte ihre Freundin in den Armen gehalten. Wie eine Ertrinkende hatte die Kasira sich an sie 
geklammert. Weinend, schluchzend, hustend.\\
''Sie ist in Eurem Anwesen.``\\
Die junge Gräfin warf einen letzten Blick auf die Toten. Hundertsiebzig Menschen waren in der Halle 
eingesperrt, als das Feuer entfacht wurde. Zweiunddreißig hatten es durch die Fenster geschafft, die 
zwei Soldaten geöffnet hatten. Weniger als vierzig hatten es nach draußen geschafft, als die 
Stadtwache eintraf und die Tore aufgebrochen wurden. Es war ein Wunder, dass Lavay lebte. Ein 
Zeichen?\\


Sarimé warf einen langen Blick aus dem Fenster ihres Gemachs. Vor ihr die Dächer Na'Rashs. Ihrer 
Stadt. Sie nickte, um ihre Gedanken zu bekräftigen, und wandte sich um. Lavay saß immer noch 
auf dem mit Fellen ausgelegten Bett. Renec stand aufrecht und beobachtete sie reglos. 
Den Kommandanten ihrer Garde hatte sie längst aus dem Zimmer geschickt, aber sie wusste, dass Samos 
es sich nicht nehmen lassen würde, persönlich vor der Tür zu wachen. Und dann kauerte da noch 
verletzt und zitternd Sakan. Nachdenklich musterte Sarimé seine Brandwunden. Die gerötete, teils 
verkohlte Haut. Das Feuer hatte die Muster auf der Haut ausgelöscht. Nur eine geringe Strafe zu 
all den Untaten, zu denen er in den letzten Jahren befohlen hatte. All die Dörfer, die brannten. 
All die Toten.\\
''Eure Worte haben den Tod vieler, vieler Merandil verursacht, verehrter Stadthalter Sakan``, brach 
Sarimé die Stille: ''Mein Wort wird dafür sorgen, dass Euer Leben hoffentlich viele, viele Merandil 
retten wird.``\\
Sie sah die Furcht in Renecs gleichgültiger Mimik. Die unverhohlene Neugierde in Lavays blauen 
Augen.\\
''Ich vertraue euch beiden die Zukunft Merandilas an``, fuhr Sarimé fort: ''Bald wird König Semric 
und der Hohepriester Brom-Dallars von diesem Zwischenfall erfahren. Sie werden einen Bürgerkrieg 
fürchten. Saleica kann sich keinen Krieg innerhalb der Grenze leisten. Und Merandila wird zu Staub 
zerschlagen, wenn wir zwischen den beiden Reichen stehen. Nach dem, was Sakan tat, werden die 
Merandil nicht mehr schweigen.``\\
Sarimés Blick huschte zu einer einsamen Teetasse, die vor weniger als einer Stunde hier zurück 
gelassen wurde. Aus dem Nichts war dieser Mann, ein Holzfäller wie er sagte, vor ihrer Tür 
gestanden. Er hatte über Tee geredet. Und über den Tod. Als er mit seinem Monolog endete, dankte er 
ihr für ihre Zeit und verschwand wieder.\\
''Ich habe also die Wahl. Fliehe ich in den Süden und jammere, dass die Merandil einen Bürgerkrieg 
anzetteln? Bleibe ich, kämpfe gegen diese Bauernrevolte und gewinne, hinterlasse aber eine 
angreifbare, zerstrittene Grafschaft und überlasse unseren Feinden im Norden viel zu viele 
Vorteile? Gebe ich dem Willen meines Volkes nach und begehre gegen Saleica auf? Sterben werden wir 
immer.``\\
''Nicht, wenn wir in die richtige Richtung weglaufen``, sagte Lavay auf Kasira.\\
Sarimé sah zu Renec, wartete auf seine Worte. Aber der Bastard schwieg.\\
''Ihr beide reist nach Norden``, befahl Sarimé: ''Findet den Feind. Findet deren König und 
sagt zu ihm: Die Königin Merandilas schenkt ihm diesen Priester. Lebend. Er war der zweitmächtigste 
Mann in Merandila. Stadthalter Na'Rashs. Vertrauter des Hohepriesters. Ich hoffe es reicht, um ein 
Gespräch zu eröffnen.``\\
''Das ist dumm!``, entschied Renec, während Lavay sich rücklings auf das Bett fallen ließ und an 
die Decke starrte. Er warf ihr einen hilfesuchenden Blick zu, doch die Kasira dachte gar nicht 
daran, ihn in seiner Meinung zu unterstützen.\\
''Wir wären Bettler, keine Gesandten! Wir haben nichts, was wir Kasir anbieten können! Sie wollten 
diesen Krieg nicht. Unsere Ländereien interessieren sie nicht! Unsere Ressourcen brauchen sie 
nicht! Und auf den Handel mit Merandila sind sie auch nicht angewiesen. Wir haben nichts, was wir 
ihnen bieten können.``\\
''Loyalität``, erwiderte Sarimé.\\
''Du willst Merandila also zu einer kasirischen Provinz machen?``\\
''Ich will Merandila retten. Ich will die Menschen retten. Ich will uns eine Zukunft erschaffen. 
Ansonsten wird das alles hier zu Ruinen verfallen.``\\
Renec kniff die Augen zusammen. ''Warum? Du bist Saleicanerin. Deine Familie führt zurück bis zu 
den ersten Eroberern. Dem ersten Clan. Deine Ahnen ritten an der Seite des ersten Königs.``\\
Sarimé strich sich eine gelockte Strähne hinter das Haar. Verblüfft sah sie ihn an, denn die 
Antwort schien ihr so offensichtlich. ''Es ist meine Pflicht.``\\
Sie hob ihr Handgelenk, zeigte das goldene Band aus Blüten, welches Evin ihr am Hochzeitstag 
anlegte. ''Vor welchem Gott auch immer, vor Priester und Volk und deinem Vater habe ich geschworen, 
dieses Land und seine Menschen zu schützen.``\\
Renec sah sie sprachlos an. Er hob die Hände, wollte sie berühren, doch fiel in sich zusammen. 
Stumm betrachtete er ihr rotes Haar, das ernste Gesicht, die sanfte Wölbung ihres Bauchs, die 
Brandnarben an ihrer Hand.\\
Die letzte Königin Merandilas ging ins Meer, hieß es. Und die Helle versprach den Menschen eine 
Nachfolgerin. Hier stand sie. Erwählt vom alten Blut Merandilas. Gezeichnet von Flammen. Gesegnet 
mit der Seele eines Kindes. Renec hatte immer an die Existenz der Hellen geglaubt. Er hatte ihre 
Zeichen gesehen. Und ihre Kälte gespürt.\\
Wenn Sarimé Sil'Vera nicht die Zukunft seiner Heimat war, wer wäre es dann?\\
Langsam sank er in die Knie, ergriff ihre Hand und starrte auf den Teppich unter sich. Er wollte 
die Worte sprechen. Den Eid. Ihr sein Leben schwören. Doch sie kam ihm mit ihren Worten zuvor.\\
''Ich werde nie Sieva sein, Renec. Ich werde nicht sehnsüchtig auf deine Besuche warten. Ich werde 
mich dir nie völlig offenbaren. Ich werde niemals von deiner Stimme und deinem Lachen abhängig 
sein. Ich werde nicht von deinen Küssen träumen, nicht deine Briefe an die Wand hängen oder in 
meinen Laken nach deinem Geruch suchen. Ich werde dich nicht heiraten, wie der Holzfäller oder 
wer auch immer er ist, plant. Ich werde dir nicht mein Herz schenken. Denn es gehört nicht mir. 
Es gehört meiner Pflicht. Eines hat mir deine Helle offenbar. Herrschen ist opfern. Ich werde dich 
opfern, wenn es so weit kommen sollte. Ich werde Lavay opfern. Ich werde mein Kind opfern. Ich 
opfere mich.``\\
Er küsste ihre Hand und erhob sich. Renec schaffte es nicht, ihr ein letztes Mal ins Gesicht zu 
blicken.\\

Da standen sie. In Reih und Glied. Die Uniformen gestärkt. Die Klingen der Speere glänzten im 
Mittagslicht. Die Schwerter zum salutieren erhoben. Der blaue Stoff erinnerte Ilia an die 
Farben des Himmels, bevor die Morgendämmerung einsetzt. Und auf der linken Schulter stieg sie 
auf, die mit Goldfaden gestickte Morgensonne. Das Löwenmaul weit aufgerissen, als wolle es den 
goldenen Ball mit einem Biss verschlucken.\\
Dies war kein Krieg, sondern eine gewonnene Schlacht. Es gab keinen Grund, lederne Panzer und 
Kettenrüstungen zu tragen. Sie hatten keine Angst. Sie waren sich sicher. Sie würden in den Palast 
marschieren und die Verräter jagen wie Ratten.\\
Ilia begegnete den Blick ihres Vaters, der neben einer handvoll weiteren in die Jahre gekommenen 
Männer und Frauen stand. Sie alle trugen die Uniform der aufgehenden Morgensonne und des 
brüllenden Löwen. Sie alle hatten mit ihren Taten Saleica Ruhm gebracht. Eroberer. Krieger. 
Bewahrer. Diese Offiziere führten Seeschlachten, Belagerungen und Massaker an. Und nun den Putsch.\\
Ilia war an den Gefangenen im Kerker vorbei gegangen. Durch die Elendsvirtel. Vorbei an Märkten und 
Werkstätten. Durch die Parks und über die Vergnügungspassagen. Ruhe lag über der Stadt. Fragende 
Blicke begleiteten sie und diese Soldaten hier würden die Antwort bringen.\\
Ilia schlüpfte aus ihren Schuhen und stieg mit der Hilfe eines Soldaten auf einen der 
Felsen im untersten Garten. So viele Köpfe und Klingen vor ihr. Das Schloss ragte hinter ihr auf, 
aber was zählte das schon, wenn die zahllosen Dächer Brom-Dallars vor ihr lagen? Saleica war zu 
lange der Haufen verwöhnter Adeliger, die jetzt gerade im Schloss tranken und fraßen und einen 
falschen König ehrten, gewesen. Saleica war nicht die Stille des Feuers im Tempel, die Gesänge 
der Priester. Weder Adel noch Priester machten die Stärke dieses Landes aus. Aber die Männer und 
Frauen, die für dieses Land bereit waren zu sterben.\\
''Ihr hattet einst einen starken König. Er eroberte mit euch Länder. Er ritt mit euch in die 
Schlacht. Er überquerte Meere mit euch! Er sicherte Saleicas Reichtum und Macht. Er schaffte durch 
eine Hochzeit Frieden mit Kasir. Und dann starb er durch die Hand eines Priesters.\\
Euch wurde einst eine tapfere Königin versprochen. Wild und furchtlos führte sie ihre Klingen und 
ihre Krieger. Sie kämpfte als junges Mädchen in den Kasernen. Raufte mit Rekruten und rannte mit den 
Pferden. Ihre Stimme war laut und ebenso klar wie ihr Lachen. Sie wollte euch das Reich der Löwen 
erschaffen. Sie versprach euch den Frieden eines Volkes, welches keine Feinde mehr hat. Und sie 
starb durch die Hand eines Priesters.``\\
Ilia schwieg einen Moment. Sie wählte ihre nächsten Worte weise. Seine Augen rief sie sich in 
Erinnerung. Sein Blick. Seine Konzentration, wenn er zeichnete oder ihrer Stimme lauschte.\\
''Ihr hattet einen König. Er war nicht König Kareen. Er war nicht Prinzessin Riolean. Er war ein 
Kind, zu früh gekrönt und zur Marionette erzogen. Er hat euch enttäuscht. Zu groß war die Last. Zu 
bedeutungsvoll das Erbe. Er konnte kein Reich der Löwen erschaffen. Und auch er starb durch die 
Hand eines Priesters.``\\
Sie sah hinab in das Gesicht ihres Vaters. Seine Miene blieb ernst und ohne Spur eines Zeichens. 
Ilia Ma'Sah holte tief Luft. ''König Semric, der goldene Löwe, ist tot. Saleicas letzter Herrscher 
wurde ermordet durch die Klinge eines Verräters. Einer Schlange, die sich in unsere Städte 
schlich. Ihr Nest in unseren Tempeln hatte. Ihre Taten mit unseren Gott rechtfertigte. Er 
zischte seine Worte in unserem Thronsaal und vergiftete unser Land. Aber wir dürfen eines 
nicht vergessen. Wir sind alle Löwen! Kein einzelner Mann und keine einzige Frau kann uns das Reich 
der Löwen bringen. Aber wir alle zusammen können es!``\\

Wie die Flut ergossen sich die Soldaten in den Palast. Sie schwemmten Verräter und Wachen nieder. 
Überschütteten sie mit Geröll und nahmen ihnen die Luft zum Atmen. Ilia sah fasziniert zu, wie die 
Angst die Gesichter des Adels verzerrte. Wie die Priester betend in die Knie gingen. Wie die Wachen 
brüllend auf ihre Krieger zu stürmten. Und wie sie alle von der Strömung ins Nichts gezerrt 
wurden.\\
Als sie den Saal betrat, in der vor wenigen Stunden die Hochzeit hätte stattfinden sollen, lag wieder 
Stille über dem Palast. Noch immer schmückten Bänder, Stoffe und Blumen den Saal. Blut glänzte in 
Lachen und trübte mit verschmierten Schlieren den Marmorboden.\\
Der schnell gekrönte und von Hisio-Mahar erwählte König stand inmitten seiner Leibgarde und 
schrie. Befehle und Flüche, Todesdrohungen und Gejammere. Die Soldaten der Morgensonne drängten 
immer weiter in den Saal. Stießen und töteten. Bis schließlich jemand den König erreichte, an 
den Haaren packte und in die Sehnen trat. ''Knie nieder vor deinen Richtern!``, herrschte seine 
Stimme durch den Saal. Ihr Vater trat mit einem Säbel vor, aber Ilia hob die Hand und ließ ihn 
innehalten. Bedächtig ging sie auf ihn zu und betrachtete die Krone auf dem dunklen Haar. Es war 
nicht Semrics Krone, sondern ein prunkvolles, klobiges Ding aus Gold und Juwelen. Der Mann 
war älter als sie. Halblanges Haar, vor Angst weit aufgerissene Augen. Sie erkannte kleine 
Tätowierungen an seinem Handgelenk. ''Wo ist Hisio-Mahar?``, fragte sie.\\
Hasserfüllt blickte er zu ihr auf und schwieg.\\
''Wie heißt er?``, fragte Ilia die Umstehenden.\\
Unschlüssige Blicke wurden getauscht, ehe einer der an die Wand gedrängten Adeligen antwortete: 
''Marek Hor'Le.``\\
Ilia legte den Kopf schief und strich sich eine Haarsträhne aus dem Gesicht. ''Ah... Sprössling der 
Nichte König Kareens, nicht wahr? Das weckt ganz entfernte Erinnerungen... die Nichte hat sich nach 
dem Tod des restlichen Familienzweiges auf eine der winzigen Inseln an der Westküste niedergelassen, 
oder? Du hättest dort bleiben sollen, Marek. Oder wie viel Geld hat Hisio-Mahar in deine Ausbildung 
gesteckt? Muss ein schlechtes Gefühl sein, eine Marionette zu sein. Und dann auch nur die für die 
Reserve. Aber du dachtest dir: Eines Tages könntest du König sein. Das war es dir wert, hm? Das war 
all die Jahre in Isolation, all die Gebete und Bestimmungen wert? Die glorreiche Hauptstadt. Hast du 
sie genossen?`` Ilia blickte verträumt und kicherte. Die Art Kichern, die nach ihren Erfahrungen 
Männer so lieben.\\
''Die Lichter in der Nacht. Die Tänze und Feste. Die Spiele und Turnire... Es tut mir leid, dass 
ich dir diese Träume nehmen werde...``\\
Ilia ließ den Säbel kreisen. Marek riss die Hände vor sein Gesicht. ''Nein!``, schrie er auf. Und 
er schrie noch weiter, als seine tätowierte Hand durch den Schwung des Säbels ein kurzes Stück 
durch die Luft fiel und dumpf auf dem Marmor aufkam. Er krümmte sich und presste den blutenden 
Stumpf an seine Brust. Über seine Schreie hinweg fixierte Ilia jedoch eine andere Person.\\
\textit{Sie ist schon verflucht hübsch}, dachte Ilia über diese Tatsache erbost, während sie Mihiki 
Sa Elrens braune Haut und das tiefschwarze Haar betrachtete. Das glänzende, feingeschwungene Diadem 
auf dem Haar, das Semric Ilia hatte aufsetzen wollen.\\
''Wo ist Hisio-Mahar?``, wiederholte Ilia schneidend.\\
Mareks Schreie war zu einem kläglichen Wimmern geworden. Sein Gesicht wurde immer bleicher, 
während weiter das Blut aus seinem Stumpf pulsierte. ''Weg!``, spuckte er aus. Seine Stimme klang 
schrill, zornig und panisch. Ilia kniff frustriert die Lippen zusammen. Der Säbel zuckte, tauchte 
ein in das weiche Fleisch, durchdrang Haut und Organe und hinterließ eine klaffende Wunde. Ilia 
reichte ihrem Vater die Klinge, ehe sie sich den Soldaten zuwandte. Sie suchte mit denen in der 
ersten Reihe Blickkontakt, ehe sie zum sprechen ansetzte. ''Wir müssen den Hohepriester finden. 
Jagd ihn. Und jeden seiner Getreuen. Die Zeit der Priester ist vorbei. Osyma will sein Land und 
seine Kinder ruhmreich sehen. Es wird Zeit, dass wir ihn wieder so Ehre erweisen, wie er es uns 
lehrte! Nicht durch Predigten und Gottesdienste! Die Priester standen kurz davor, Saleica in den 
Abgrund zu reisen! Sie hatten ihre Chance! Ihre Zeit ist vorbei! Lasst die ehrbaren Tage 
zurückkehren, die wir zu den Zeiten König Kareens hatten. Lasst uns wieder ehrbar werden! Findet 
Hisio-Mahar, damit er büßen wird.``\\
Die Soldaten reckten ihre vor Blut glänzenden Waffen und stimmten einen Jubel an. Schnell 
verklangen die Rufe jedoch, als Ilia sich abwandte. Die Anspannung der hunderten Menschen im Saal 
pulsierte spürbar durch den Raum.\\
Auf Ilias Zeichen hin, ließen die beiden Soldaten die Gesandte los. Mihiki Sa Elren ging in die 
Knie. Das schwarze Haar hing in Strähnen an ihr herab. Ihr Gewand prachtvoll, mit Gold und Juwelen 
besetzt. Sie versuchte Ilias Blick stand zu halten, aber je näher sie kam, desto mehr sank Mihiki 
in sich zusammen. Die Furcht fraß all ihre Schönheit. 
''Mihiki Sa Elren``, sagte sie ruhig; fast sanft. ''Du kamst nach Saleica, nach Brom-Dallar, um die 
Worte des Rats der Kolonien zu überbringen. Die Bitte der Gleichstellung und Einbürgerung. Du 
botest dich selbst als Pfand dafür an. Du hattest die Hoffnung, dass König Semric dich 
ehelichen würde. Hisio-Mahar versprach dir die Macht über Saleica. Aber zu diesem Zeitpunkt 
entglitt sie ihm bereits. Und ganz gleich, was du versuchtest, es änderte nichts daran, dass 
König Semric bereits das in dir sah, was du warst. Als eine in Gold gehüllte Schlange betratest 
du die Audienz. Du kannst dich schmücken oder ausziehen, aber es ändert nichts daran, dass du aus 
Schlamm und Dreck stammst.``\\
Ilia hob die Hand. Der Dolch, der Semric erstach. Sie hatte dieser Symbolik nicht widerstehen 
können. Bedächtig beugte sie sich vor und stieß der zitternden Frau das Messer zwischen die Rippen. 
''Wir sind Löwen``, antwortete Ilia: ''Wir erobern. Wir herrschen. Wir zertrampeln Schlangen. Diese 
Antwort wird dein Vater und der Rat von uns erhalten.``\\
Noch ehe die Gesandte ihren letzten Atemzug tat, hob Ilia Ma'Sah das blutige Messer. ''Jeder, der 
unser Brüllen hört! Alles, was das Sonnenlicht berührt!``\\
Und das Brüllen erklang.\\




