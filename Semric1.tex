
\chapter{Der saleicanische Löwe}
Er war elf gewesen. Ein schlaksiger, schüchterner Junge mit zerzaustem Haar. Ein Junge, dessen 
einziger Protest stets Schweigen war, da er genau wusste, jedes seiner vorgebrachten Argumente würde 
gegen ihn verwendet werden. Die Lehrer bezeichneten ihn als dumm, da er auf ihre Fragen nicht 
reagierte. Statt Sätze auswendig zu lernen, erträumte er sich Abenteuer und Forschungsreisen in 
fremde Länder. Für ihn war diese Welt viel erstrebenswerter als das, welches er im Schatten der 
älteren Schwester und des berüchtigten Vaters erlebte. Er liebte Geschichten, sog jede Erzählung und 
Legende in sich auf und sie wurden Teil seiner geheimen Traumwelt. In der Realität, in der er als 
dumm, unwichtig und nebensächlich galt, fühlte er sich wohl in der Rolle des Vergessenen. Das ließ 
ihm immerhin mehr Zeit für seine geheime Welt. Im war klar, dass das nicht ewig so weiter gehen 
würde. Wenn er erwachsen wäre, würde man ihm einen Posten zuteilen, der wichtig schien, auch wenn er 
es nicht war. Er würde ab und an in die Öffentlichkeit treten müssen, seiner Schwester Riolean 
höflich zunicken und wieder verschwinden. Immerhin würden ihn dann keine Lehrer und der Vater nicht 
mehr plagen. Keine Tadel, keine enttäuschten Blicke mehr.\\
Er war elf gewesen, als sein Vater und Riolean nicht mehr zurück kehrten. Auf dem zerzausten Haar 
des Vergessenen ruhte plötzlich die schwere Krone Saleicas. Heuchlerische Verwandte, Priester und 
Adel zerrten an jeder Faser, die sie in ihre Klauen bekamen. Er wurde zum Spielball der Politik und 
musste, während der Trauer um Schwester und Vater, eine Partei wählen.\\
Jetzt, zwölf Jahre später, blickte Semric in den Spiegel und fühlte sich viel zu oft immernoch wie 
ein kleiner Junge. Aber er war nicht mehr schweigsam, sondern verstummt. Nicht mehr schüchtern, 
stattdessen verbittert. Sein Blick glitt an seiner eigenen Spiegelung vorbei und betrachtete den 
Priester, der gemächlich im Sessel saß und eine Buchseite umblätterte. Einer der wenigen Momente, 
in denen Semric sich nicht zügelte, seine Verachtung zu verbergen. \\
\textit{Wem spiele ich es überhaupt vor? Der Aasfresser weiß es doch.}\\
Als hilfloser Junge wusste er schon, dass er weder dem Adel noch den Tanten, Onkeln und sonstigen 
Verwandten trauen konnte. Das hatte sein Vater, König Kareen ihm und Riolean beinahe täglich 
erzählt. Die Priester dagegen – zu den Zeiten König Kareens nicht mehr als Wanderprediger, 
Kräuterkundige und mittellose Leute – hatten ihre eigenen Waffen eingesetzt. Waffen, denen ein 
elfjähriger, verstörter Junge nichts entgegenzusetzen hatte. Trost, Hoffnung und Anteilnahme. 
Hisio-Mahar war der Schlimmste der ganzen Bande.\\
Semric gestand es sich ungern ein, aber seine 
damalige Schwäche hatte die Priester Saleicas erst zu der Machtposition gebracht, die sie jetzt 
innehatten. Aber was sollte er nun noch ändern? Er war ein miserabler König, das wusste er selbst 
genauso gut wie ganz Saleica. Er hatte das Land kein Stück weiter gebracht, die Rebellionen in den 
Kolonien mit viel zu hohen Verlusten niederwerfen lassen, und sonst? Was tat er anderes, als zu 
lächeln und zu nicken, seine Unterschriften unter die Pergamente zu setzen, die Hisio-Mahar ihm 
reichte?\\
Semric blickte sich selbst in die Augen. Er sah keinen Weg, irgendetwas zu ändern. Er hatte nie 
gelernt, ein König zu sein. Er wusste, dass der Adel ihn auslachte und sie ihre Bitten direkt an die 
Priester richteten. Semric wollte kein König sein, wollte er nie.\\
\textit{Riolean war die Strahlende. Sie hat mit vier Jahren bereits den gesamten Hof befehligt, mit 
sechs ritt sie auf die Jagd und mit acht erlegte sie einen Wolf. Ihr Pfeil traf ihn direkt ins Herz. 
Und das Gift der Rebellen zerfrasen ihr Herz.}\\
Er schüttelte kaum merklich den Kopf. Riolean war zehn Jahre älter gewesen als Semric und zu ihrer 
Lebzeit hatten sie kaum etwas gemein gehabt. Wenn die Familie unter sich war, ging es stets um ihre 
Heldentaten und wie glorreich sie einst herrschen wird. König Kareen tüftelte mit ihr an Strategien 
für weitere Eroberungen, der Stabilisierung der Kolonien und am einrichten weiterer Handelsrouten. 
Die Kolonien galten als Rioleans Spielwiese, dort durfte sie sich beweisen, bevor sie die Herrschaft 
über das eigentliche Juwel, Saleica, bekommen sollte. Deshalb sollte sie nach ihrem 21. Geburtstag 
den Vater dorthin begleiten. Es war das erste Mal überhaupt, dass König Kareen und Kronprinzessin 
Riolean die Kolonien betraten. Und sie kehrten beide nicht zurück.\\
\textit{Sie haben erwartet, gezähmten Tieren zu begegnen. Aber Menschen lassen sich nicht zähmen.}\\
Semric schluckte schwer und konnte den Blickkontakt mit seinem Spiegelbild nicht länger ertragen. 
\textit{Außer ich. Ich bin ein gekrönter Sklave.}\\
Erst nach dem Tod der beiden war Semric ihnen näher gekommen. Er las als Junge die Schriften seines 
Vaters, schien ihn erst jetzt wirklich kennenzulernen. Riolean begleitete ihn stets in seinen 
Gedanken. Ihre Stimme beschimpfte ihn, wenn er vor den kritischen Blicken der Priester zurückwich 
und sich lieber unter der Bettdecke versteckte, anstatt eine Ansprache zu halten. Auch jetzt 
verfolgte sie Semric noch, aber ihre Stimme war gnädiger geworden. Resignierter, wie er selbst. Es 
war, als würde Rioleans Geist versuchen, aus ihm doch noch einen annehmbaren Herrscher zu formen. 
Bei Paraden flüsterte sie, er sollte gerader sitzen, auch den armen Leuten in die Augen blicken, 
ihre Hände umfassen und tröstende Worte sprechen. Bei Verhandlungen forderte sie von ihm, lauter und 
sturer zu sein; bei Entscheidungen berechnender und direkter.\\
``Semric? Wir sollten noch einmal über diese Sache reden'', meldete sich Hisio-Mahar zu Wort, ohne 
seinen Blick von dem Buch zu wenden. Semric blieb stehen wo er war, betrachtete den Priester im 
Spiegel. Manchmal, so kam es ihm vor, schienen sich die verschlungenen Muster auf dessen Haut zu 
bewegen wie Giftschlangen. Zischend und lauernd auf ihre Chance wartend, um zu einem tödlichen Biss 
aufzuspringen. \textit{Dreh dich nicht um,} wisperten die Worte durch seinen Kopf. Die Worte, die 
schon der Junge als die Kritik und Ratschläge von Riolean zugeschrieben hatte. \textit{Erniedrige 
dich nicht noch weiter!}\\
Semric runzelte die Stirn und fixierte die Manschetten an seinem Gewand. ``Welche Sache?''\\
``Die mit Merandila und dem Mädchen.''\\
Semric versteifte sich augenblicklich. \textit{Nicht wieder nachgeben. Dieses mal nicht!}\\
``Ihr sprecht von Gräfin Sarimé Sil'Vera? Ich nehme an, ihr seid einander nicht so vertraut, dass 
Ihr die Gräfin so bezeichnen solltet.''\\
Hisio-Mahar blickte auf. Er war kein Widerwort gewohnt und Semric spürte den bohrenden Blick in 
seinem Rücken. Der Priester räusperte sich. ``Ich bin immer noch der Meinung, dass wir einen 
unserer Vertrauten das Mä... die Gräfin ehelichen lassen sollten.''\\
``Ihr habt Eure Meinung bereits vorgetragen. Ihr müsst Euch nicht wiederholen.''\\
``Und Mi'Kae... ich glaube nicht, dass wir ihm trauen sollten. Gesindel bleibt Gesindel, auch wenn 
es sich das Gesicht drei mal schrubbt.''\\
Nun drehte Semric sich um. Er bebte regelrecht. Der Hohepriester hatte einen Fehler begangen, den  
Kommandanten zu beleidigen. Semric kannte den Mann kaum, aber er hatte ihm schon durch seinen Blick 
gezeigt, was er von ihm als König hielt. Wer konnte es ihm verdenken? Als Mi'Kaé elf war, hatte er 
bereits die Entscheidung getroffen, Soldat zu werden. Semric hatte sich in diesem Alter hinter den 
Priestern versteckt. Mi'Kaes Lebensgeschichte hatte Semric aufgewühlt. Zum ersten Mal seit langer 
Zeit wieder Zweifel in ihm gesät. Wollte er das hier wirklich? War es vielleicht noch nicht zu 
spät, sich zu ändern? Wenn ein Mann aus der völligen Bedeutungslosigkeit zu einem Held werden 
konnte, warum dann nicht auch er selbst, der schon sein halbes Leben lang eine Krone trug?\\
Natürlich war es Hisio-Mahar nicht recht, dass der Kommandant nach Merandila reisen sollte. Aber 
vermutlich war er nur gegen dem Vorschlag, weil er vom Offizier Lerin stammte. Wäre der Priester 
selbst auf die Idee gekommen, wäre es der Wille Osymas gewesen. Lerin war der einzige, der sich 
damals nicht hatte vergraulen lassen. Der vermutlich beängstigend viel über die Lage des jungen 
Königs wusste, aber es offensichtlich noch nicht zu Semrics Nachteil ausgenutzt hatte. Wenn Lerin 
wollte, würde er auf genügend Ohren stoßen, die ihn zu einem Putsch folgen würden, daran zweifelte 
Semric nicht. Es gab Zeiten, da hatte er schlaflos am Fenster seines Zimmers gestanden und 
regelrecht darauf gewartet, dass die Spiegelungen im Glas zeigen würden, wie die Türe aufschwang. 
Lerin mit einer handvoll Männern – mehr würde es gewiss nicht benötigen – in das Zimmer trat, ihm 
seine gesamten Verfehlungen aufzählte und ihn schließlich köpfte. Obwohl Lerin mit den Jahren älter 
geworden war, traute Semric es ihm weiterhin zu. Zeitweise hatte er sogar gehofft, dass es endlich 
geschehen würde. Aber anscheinend gab es doch etwas, was der Offizier in ihm sah. Was ihm sein 
Weiterleben erlaubte.\\
\textit{Verlasse dich nicht noch länger darauf. Die Zeiten des Welpenschutzes sind auch in den 
Augen eines gutmütigen Mannes längst vorbei!}\\
``Geht jetzt!'', entschied Semric, bevor der Priester noch etwas sagen konnte. \\
Anscheinend war es einer der wenigen Momente, in denen Semrics Worte genug Autorität besaßen, 
sodass der Priester ihn ernst nahm. Aber vielleicht hatte Hisio-Mahar auch einfach nur noch etwas 
vor oder kein Interesse daran, mit ihm zu streiten. Schließlich erreichte er letztenendes seine 
Ziele. Der Priester war es gewohnt, immer zu gewinnen. Und er hatte Geduld. \\



Die Wellen des Meeres schlugen gegen die Hafenmauer. Die Flut hatte ihren Höhepunkt erreicht und 
salziger Schaum kletterte an dem steinernen Wall empor, streckte seine Finger aus und einzelne 
Tropfen schafften es, das Gebilde zu erklimmen. Semric trat eilig einen Schritt zurück und strich 
sich über seinen Wams. Er sah sich kurz beunruhigte um und warf einen längeren Blick zu der Stelle, 
an der die zwei Pferde angebunden waren. Nur die Umrisse der Tiere erkannte er in der Nacht. Und 
die reglose Gestalt seines Leibwächters. \\
Erhim sein Name. Der Mann – ungewöhnlich groß und breit für einen Mann aus den Kolonien – ist 
damals mit einem Schiff gekommen. Ebenso wie das Pferd als Geschenk für Riolean. Beide, Mann und 
Pferd, sollten auf Wunsch König Kareens für seine Kronprinzessin ausgebildet werden. Später, wenn 
sie aus den Kolonien zurückkehrte, sollte sie beide als Geschenk erhalten. \textit{Aber sie kehrten 
nicht wieder.}\\
Das Pferd war Semric ans Herz gewachsen, der Mann seit seiner Krönung zu seinem stummen Schatten 
geworden. Der junge König war erst wenige Male in eine Situation gekommen, in der er dem 
Leibwächter für seine Anwesenheit wirklich dankbar war. Das letzte Mal war, als er bei einer Parade 
vom Pferd fiel und die Menschenmasse durch ein paar betrunkene Soldaten in Panik verfallen war. 
Erhim hatte ihn vom Boden gepflückt und zurück zum Schloss gebracht. \textit{Wer weiß, welche 
Gefahren schon allein durch seine Anwesenheit nicht auftreten...} Rioleans mahnende Stimme weckte 
Bilder von verschleierten Attentätern und Meuchelmördern vor seinen Augen. Schaudernd wandt der 
junge König sich wieder der Kaimauer zu. Er blickte an ihr entlang zum Hafen. Es war ein kleiner 
Außenhafen für die Fischer, während die Handelsschiffe einen guten Kilometer entfernt ihre Ladung 
entluden. An dem flachen Steg ruhten also lediglich kleine Boote oder Segelschiffe im Wasser. Der 
kleine Hafen lag außerhalb der Stadtmauern und selbst bis zur Ansiedlung der Fischer musste man 
einige Minuten zu Fuß gehen. Ruhig war es also an diesem Abend und Semric lauschte den Geräuschen 
der Wellen. Als schnelle Schritte näher kamen, richtete er sich unwillkürlich auf. Seine Muskeln 
spannten sich an und einen Moment lang rechnete er damit, gleich eine Klinge an seinem Hals zu 
spüren. \textit{Verdient hättest du es,} spottete Riolean.\\
Aber dann erkannte Semric die deutlich weiblichen Umrisse. Eine schmale Gestalt, deren bauschiges 
Kleid um ihre Hüften wallte, eilte auf ihn zu. Besorgt machte er sich bemerkbar, gerade als die 
junge Frau stolperte. \\
``Vorsicht!''\\
So schmal sie auch war, ihr Gewicht zog ihn doch mit hinunter. Vorgebeugt stand er neben der Mauer 
und betrachtete interessiert das blasse Gesicht. Vor Schreck aufgerissene Augen starrten zu ihm 
empor. \\
``Aufstehen'', sagte er nur und richtete sich auf. Sein Griff blieb fest um ihre Hüften, bis sie 
wieder mit beiden Füßen auf dem Boden stand. Schnell hob sich die in ein Korsett geschnürte Brust 
der Dame. Ihre Aufmachung, ihr gepflegtes Äußeres und sogar ihre Haltung zeugten davon, dass sie 
dem Adel angehörte. Während sie nach Atem rang, sah sich Semric ein weiteres Mal um. Erhim war 
nicht mehr bei den Pferden, vermutlich deutlich näher gekommen. Aber er war auch jetzt nur ein 
Schatten und Semric konnte nicht genau sagen, wo der Mann in der Dunkelheit stand.\\
``Werdet Ihr verfolgt?'', fragte er und zückte sein Jagdmesser.\\
``Nein!'', rief sie aus und hob abwehrend die Hände: ``Nein... nein...''\\
``Ist wirklich alles in Ordnung?'', fragte er zweifelnd.\\
Ihr blondes Haar war halb in eine kunstvolle Frisur gesteckt, aber die meisten Strähnen hingen 
lockig heraus. Im Mondlicht konnte er die Farbe des Kleides nicht genau erkennen, vermutete aber 
ein tiefes schwarz. Die Farbe der Kohle, des Ursprungs und des Endes. Also entweder feierte sie 
gerade eine Geburt in ihrer Verwandtschaft oder war in Trauer über einen Verstorbenen.\\
``Seid Ihr alleine hier? Das ist kein Ort für eine Dame.''\\
Ihre Lippen verzogen sich flüchtig nach oben und sie lachte beschämt. ``Nein, da habt Ihr recht. 
Mein... mein Vater war bei mir. Wir haben ein Haus hier in der Nähe. Aber wir haben uns 
gestritten... Verzeiht, dass ich Euch so überrannt habe... es ziemt sich wirklich überhaupt 
nicht.''\\
Semric wusste nicht genau, was er zu der aufgeregten Frau sagen sollte und schwieg. Während sie 
sich über die Mauer beugte und hinab zu den Wellen sah, überlegte er, ob er ihr anbieten sollte, 
sie nachhause zu bringen. Eigentlich sollte er das tun, aber es war spät und bis zum Schloss würde 
er noch zwei Stunden reiten müssen. Hisio-Mahar hatte vermutlich längst gemerkt, dass er weg war.\\
``Das Meer ist so tröstend'', flüsterte sie.\\
``Ihr solltet nicht so weit über die Mauer...''\\
Sie schluchzte. ``Er ist weg, wisst Ihr? Weg. Er fehlt so sehr... mein Herz hat noch nie so weh 
getan. Und Vater... er versteht es nicht! Er sagt, ich bin selbst schuld!''\\
\textit{Also doch in Trauer...}\\ Vorsichtig berührte er sie an der Schulter um sie greifen zu 
können, sollte sie versuchen sich ins Meer zu stürzen. ``Wer?''\\
``Mein Verlobter'', schluchzte sie: ``Uns war viel zu wenig Zeit zusammen gewehrt worden!''
Ratlos sagte Semric: ``Es ist vermutlich schwer alleine zurück zu bleiben...''
Die Dame richtete sich ruckartig auf. ``Ja das ist es! Selbstverständlich! Was meint Ihr mit 
vermutlich?''\\
``Ich.. äh... ich meinte nur...''\\
``Dass das nur wieder das Gejammer eines Weibes ist? Oh, täuscht Euch bloß nicht!''\\
``Ihr missversteht mich!'', verteidigte Semric sich barsch.\\
Die Blonde strich sich das Haar zurück und schloss einen Moment die Augen. Dann lächelte sie 
einnehmend. ``Verzeiht, ich benehme mich fürchterlich. Danke für Eure Hilfe. Mein Name ist Ilia 
Ma'Sa. Und Euer Name?''\\
Semric starrte ihr Lächeln an und blinzelte mehrmals. Sie erkannte ihn nicht? Ihn, den von Osyma 
gekrönten Herrscher über Saleica und dessen zahlreichen Kolonien? ``Merio'', sagte er leise: 
``Merio Kaen...''\\
Sie runzelte die Stirn. ``Habe ich Euch schon einmal gesehen?''\\
``Ich weiß nicht...'', sagte er schnell: ``Aber gewiss hätte ich ein so hinreißendes Gesicht wie 
das Eure nicht vergessen.''\\
Mit Komplimenten brachte man die meisten Frauen vom Thema ab und so hoffte Semric auch jetzt. 
Dieser nächtliche Hafen war schon mehrmals zu seinem Zufluchtsort geworden. Hier hatte er sich vor 
dem Fluch der Krone versteckt und daher konnte er nun auch nicht zugeben, wer er war. 
``Ich bin Schreiber am Hof. Aber noch nicht lange... und ich assistiere eher meinem Lehrmeister als 
dass ich selbst etwas mache...'' Er lachte nervös. \\
``Ein Schreiber?'', wiederholte sie: ``Was schreibt Ihr so?''\\
Ohne lange darüber nachzudenken begann er eine Erzählung über seinen strengen Lehrmeister, über die 
Massen an Vorräten die er notieren und nachrechnen musste. Und darüber, wie er in seiner Freizeit 
versuchte ein eigenes Buch zu schreiben. Er erzählte von erfundenen Helden seiner nicht 
existierenden Romanen die Abenteuer bestritten und Heldentaten begannen. Je eindringlicher sie ihn 
aus ihren blauen Augen ansah, desto weniger konnte Semric aufhören. Erst als laute Rufe hörbar 
wurde, wandt sie den Blick ab und machte einen flüchtigen Knicks. ``Mein Vater. Er wird sauer sein, 
ich muss zurück! Wir sehen uns hier wieder Schreiber.''\\
Es war mehr eine Tatsache als eine Bitte und Semric konnte nur hinterher sehen, wie sie mit 
schnellen Schritten und wehendem Rock den Weg an der Hafenmauer entlang zurück lief.\\



``Das ist keine kleine Geschichte, mein König! Wir haben Krieg und wir können keine leichtsinnigen 
Plündereien erlauben! Diese Männer und Frauen haben den Befehl ihres Kommandanten ignoriert, nein 
sogar gegen seine ausdrücklichen Worte gehandelt. Das grenzt an Verrat. Sie haben sich unerlaubt 
vom Lager entfernt, sie sollten als Deserteure hingerichtet werden!'', zeterte Lerin.\\
Semric rührte sich keinen Millimeter in seinem Stuhl, blickte den älteren Mann nur an. Er hatte den 
Offizier noch nie so aufgebracht erlebt. Lerin war von seinem Stuhl aufgesprungen, die geballte 
Faust hielt er in der Luft und sogar seine Nasenflügel bebten. Hisio-Mahar beugte sich vor. Seine 
Hände falteten sich und kamen auf dem Tisch zur Ruhe. Und dann lächelte der Priester. ``Deserteure? 
Nein, keineswegs. Sie haben den Ruf Osymas vernommen und sind ihm gefolgt. Osyma will eine 
ruhmreiche Schlacht. Er will, dass sein Volk sich mehrt und sein Namen von weiteren Menschen voller 
Liebe ausgesprochen wird. Er möchte bekehren und so viele Menschen wie möglich in seine 
Herrlichkeit aufnehmen. Die Soldaten haben Ehre erlangt. Sie haben für Osyma gekämpft.'' 
Hisio-Mahar verzog verächtlich das Gesicht. ``Während ihr Kommandant sich ängstlich in seinem Zelt 
versteckte.''\\
``Ehre in der Schlacht? Sie haben Dörfer nieder gemacht, waffenlose Bauern getötet, Frauen und 
Kinder vergewaltigt! Sie haben das aus Zerstörungswut und Blutdurst getan!'', fluchte der Offizier. 
\\
``Mäßigt Euch gefälligst'', zischte Hisio-Mahar und funkelte seinen Gegenüber finster an. ``Wir 
haben nun mal Krieg, da kommt so etwas vor.''\\
Semric blickte auf die zahlreichen Kerben und Dellen des Holztisches vor sich. \textit{Mi'Kae würde 
nicht gegen Waffenlose kämpfen,} wisperte Riolean.\\
``Wollt ihr diese Männer wirklich hängen sehen, Lerin? Solltet Ihr nicht eine schützende Hand über 
die Soldaten haben? Ihr als oberster Offizier des Landes und Berater des Königs in militärischen 
Angelegenheiten?'', spottete der Priester.\\
Lerin schnappte nach Luft um empört zu antworten, als Semric aufstand. ``Ich werde einen Brief nach 
Merandila schicken. Zwei. Einen an den Kommandanten und ihm mitteilen, dass dieser Fall noch 
verfolgt wird.''\\
``Und der Andere?'', fragte Lerin skeptisch.\\
``An die Gräfin.''\\
Hisio-Mahar verging sein Lächeln. ``Sie ist ein Kind! Sie weiß vermutlich noch nicht einmal davon 
und hat genügend Probleme mit der Kleiderwahl und dem Geplauder mit Dienstmädchen!''\\
``Dann soll sie es erfahren. Wenn die Brief eintrifft, wird Mi'Kae bereits in der Festung 
angekommen sein. Er bekommt die Order, ihr bei militärischen Angelegenheiten zur Seite zu stehen. 
Immerhin sind wir im Krieg, wie Ihr, Hisio, bereits angesprochen habt.''\\
``Das war so nicht vereinbart'', rief Hisio und stand abrupt auf.\\
Im folgenden Moment schien der Priester mühsam nach Fassung zu ringen. Oh, wie Semric es genoss. Es 
kam immerhin nicht oft vor, dass der Hohepriester einmal nicht sein Schlangenlächeln zeigte.
``Der Hauptmann soll lediglich beobachten, ob das Mädchen geeignet ist und uns dann eine Nachricht 
übermitteln, damit wir einen passenden Gemahl für sie wählen können'', erklärte Hisio betont ruhig.
``Die Gräfin hat nach den merandilischen Recht, welches meine Großmutter ihnen bei ihrer 
Kapitulation zu sprach, keinen Zwang zu heiraten. Oder irre ich mich?'' Semric blickte den Offizier 
fest an.
Lerin zögerte kurz und schüttelte dann den Kopf. ``Nein, mein König. Die Gräfin Sil'Vera trägt 
A'Riks Erben unter ihrem Herzen und somit wird sie zur Merandil.''\\
``Unsinn'', murmelte der Priester.\\
``Nein'', entschied Lerin: ``Das ist Tradition. Sie beruht auf die Legenden der Helle, welche 
selbst als Fremde nach Merandil kam. Zwei Herzen Merandils, hieß es, musste sie erlangen. Die 
Geschichte erzählt, wie sie über Gletscher reiste und zum Meeresgrund schwamm, bis sie schließlich 
im Schein einer einzelnen Kerze das zweite Herz gewann.''\\
``Eine Metapher?'', fragte Semric interessiert, rückte seinen Stuhl ordentlich an den Tisch und gab 
Hisio-Mahar nebenbei ein kurzes Handzeichen. Es sollte ihm bedeuten, dass er gehen kann. Nach 
seinem kalten Blick zu urteilen, war ihm die Bedeutung auch keineswegs entgangen. Der Offizier 
blickte Hisio-Mahar, welcher grußlos und stumm das Ratszimmer verließ, hinterher. ``Ja, mein König. 
Die zwei Herzen waren Menschen. Das erste Herz wurde der Gemahl der Hellen, das Zweite war ihr 
erstgeborenes 
Kind.''\\


