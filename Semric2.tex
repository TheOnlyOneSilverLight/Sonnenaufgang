\chapter{Flucht vor dem Wind}

Die Sonne neigte sich bereits einem glühenden Flammenball gleich dem Horizont 
zu, als Semric wie so 
oft unruhig durch seine Gemächer schlich. Seine Schritte erzeugten kaum einen 
Laut. Er fühlte sie 
wieder Mal wie ein kleiner Junge, der darauf bedacht war, bloß keinen Lärm zu 
machen und niemandes 
Aufmerksamkeit zu erregen. Schließlich blieb er vor dem Spiegel stehen und sah 
sich selbst in die 
Augen. Er suchte nach etwas. Nach irgendetwas, worauf er stolz sein konnte. 
Etwas, was seiner 
Existenz einen Berechtigungsgrund geben konnte. Der junge König sah in seine 
Augen, deren Farbe den
Tiefen des Meeres glichen.\\
``Rioleans Augen'', murmelte er leise und streckte seine Hand nach dem 
Spiegelbild aus. \\
Die Augen, das war eines der wenigen Dinge, in denen die Geschwister sich 
wirklich gleich gewesen 
waren. Während er das blonde Haar der Mutter geerbt hatte, erhielt Riolean die 
schwarze 
Lockenpracht des Vaters. Ihr Teint war hell wie der Nebel, der über dem Meer 
aufstieg, seiner 
wirkte selbst im Winter sonnengebräunt. Semric trat so nah an den Spiegel 
heran, dass er nur noch 
das Blau seiner Augen wahrnahm. So konnte er sich einreden, dass seine 
Schwester doch nicht fort 
war. Er sah die Flammen noch vor sich. Seltsamerweise fand er den Gedanken, 
dass seine schöne, 
erhabene Schwester in den Flammen des Allmächtigen aufging und der Rauch ihre 
Seele in das andere 
Reich mitnahm, viel entsetzlicher als bei seinem Vater.\\
Sie waren Könige. Direkte Kinder des Gottes und daher wurden sie nicht in einem 
hölzernen Sarg 
verbrannt, sondern ihre toten Körper aufrecht an einen Pfahl gebunden. Es 
geschah damals an den 
Küsten der Kolonien, da die Leichen die Heimfahrt nicht so rein überstanden 
hätten, wie es das 
Anrecht für Könige war. Die Soldaten, die Eroberten und die Rebellen sollten 
sehen, wie die 
Kronprinzessin in ihrer strahlenden Schönheit von den Flammen ihres Gottes 
umarmt wurden. Auch wenn 
Semric nicht dabei war, verfolgte ihn das Abbild. Rioelan. Ihr Haar stand in 
Flammen. Das Feuer 
zerrte an ihrem Kleid. Der Wind ließ Funken tanzen und ihre Augen... ihre Augen 
waren geöffnet und 
das Herz in ihrer Brust keineswegs verstummt. \\
Semric blinzelte und sah sie vor sich. Ihr Blick stolz und hart, die Augen von 
einem frechen 
Funkeln beherrscht. Ein Lächeln zierte ihre Lippen und der Wind strich durch 
ihre Locken. Sie hob 
das Kinn und blickte zu ihm empor, die Arme vor der Brust verschränkt. \\
Semric rang nach Atem und taumelte zurück. Er krachte gegen einen Stuhl und 
spürte schließlich die 
Wand in seinem Rücken. Atemlos glitt er an dem rauen Gestein hinab, ohne seinen 
Blick von dem 
Spiegel wenden zu können. Er wagte nicht zu Blinzeln. Das Blinzeln würde sie 
wieder mitfort nehmen 
und er wäre wieder allein. \textit{Bruder.}\\
Überraschender Weise klang ihre Stimme in seinen Gedanken so sanft. Er schloss 
die Augen und 
erinnerte sich daran, dass er Atem holen sollte. Tief sog er die Luft ein und 
gab sie nach kurzen 
Momenten wieder preis. Nur wenn er alleine war, viel ihm auf, wie sehr sich 
wirklich Rioleans 
Stimme von seinen eigenen Gedanken unterschied. Nie hatte er es jemanden gewagt 
zu sagen, aus der 
Furcht, dass man ihn als wahnsinnig bezeichnen würde. Vielleicht war er das ja 
auch.\\
``Du bist tot'', flüsterte er leise.\\
Semric schlug die Augen auf, in der vollen Überzeugung, ihr Abbild würde wieder 
verschwunden sein. 
So wie stets. Stattdessen hatte die schlanke Gestalt im Spiegel eine Augenbraue 
skeptisch gehoben. 
\textit{Du wolltest mich sehen. Du wolltest Trost.} Stellte die Stimme sachlich 
fest. \\
Der König verharrte, schüttelte nur den Kopf. Aber der kühle Blick aus den 
Augen, die seinen 
eigenen so ähnlich waren, fixierte ihn weiterhin. Abrupt sprang er auf die 
Beine, griff sich noch 
im vorüber gehen seinen Mantel und stürmte aus dem Zimmer. Die Türe knallte 
gegen den Rahmen und 
blieb offen stehen, während Semric den Gang entlang hastete.\\

Erhim, welcher wohl gerade erst zu seiner abendlichen Schicht angetreten war, 
machte einen Satz, 
als wolle er ihn aufhalten. Aber schließlich verharrte seine Hand doch in der 
Luft und er räusperte 
sich verlegen. Es stand dem Leibwächter nicht zu, den König Saleicas, den 
goldenen Löwen, Osymas 
Kind, zu berühren. Semric bemerkte es kaum. Er war zu vertieft in seiner 
Flucht. Aber in jeder 
spiegelnden Oberfläche, die seinen Weg kreuzte, sah er sie. Zierspiegel zeigten 
ihr stolzes 
Gesicht. Auf polierten Kerzenständern funkelten ihre blauen Augen. \textit{Du 
kannst nicht fliehen.} 
Eine Ruhe lag in ihrer Stimme, eine Gewissheit, die ihn einen Schauer über den 
Rücken jagte. 
\textit{In deinen Adern fliehst das Blut von Königen. Das Blut des 
Allmächtigen.}\\
Ohne darauf zu achten, trugen seine Füße ihn zu einem der unauffälligen 
Seitenausgängen. Von dort 
aus waren es nur wenige Schritte über den dämmrigen Hof, bis er im Stall war. 
Dort war es bereits 
dunkel wie in der Nacht, denn die schmalen Fenster lagen zur Nordseite hin und 
den Knechten war es 
untersagt, in dem hölzernen Gebäude voller trockenem Stroh Fackeln zu 
entzünden. Nur eine kleine 
Laterne kennzeichnete den Eingangsbereich und eine weitere ruhte auf einem 
steinernen Untergrund. 
Dort nächtigten stets ein oder zwei Burschen, falls in der Nacht ein Pferd 
benötigt wurde oder 
eines der Tiere erkrankte. Auch an diesem Knecht ging Semric Grußlos vorüber. 
Mit großen Schritten 
näherte er sich einen der Ställe. Der Riegel klemmte und mit einem wütenden 
Laut schlug trat er mit 
dem Stiefel dagegen. Ajaran legte die Ohren an und wieherte erschrocken. Aber 
auch das zügelte 
Semrics Eile nicht. Er riss in einer Bewegung das lederne Zaumzeug von Hacken. 
Die Geste ließ den 
nahenden Stallknecht, der eben noch seine Hilfe anbieten wollte, zurück weichen 
und mit großen 
Augen zusehen, wie sein König mit wenigen geschickten Händen das Geschirr um 
den schmalen Kopf des 
Pferdes zog. Das fest schnurren der Riemen dagegen war eine weniger elegante 
Art, da Ajaran 
keineswegs brav stand. \\
\textit{Das Blut von Herrschern. Eroberern. Kriegern.}\\
Semric fluchte laut, warf die Zügel über den Hals des Tieres und zerrte mit 
einem kräftigen Ruck am 
Zügel. Der Hengst machte einen Satz aus der Box heraus und schon schwang sich 
Semric auf den 
blanken Rücken des Tieres. Er zog die Zügel fest an, ließ Ajaran keine 
Möglichkeit, Kopf oder Hals  
frei zu bewegen. Fest drückte Semric dem Tier die Fersen in die Flanken und 
duckte sich, als es mit 
weiteren, holprigen Sprüngen den Stall verließ.\\
Semric sah in den Augenwinkeln eine weitere, ebenfalls in einen Mantel gehüllte 
Gestalt. Erhim zog 
sich fluchend auf das bereits gesattelte Pferd eines Boten, welcher wohl gerade 
los reiten wollte. 
Ein weiteres Wiehern eines erschrockenen Pferdes und Hufe klapperten auf dem 
gepflasterten Weg. 
Erst als sie das Tor hinter sich ließen, gab Semric die Zügel frei. Er duckte 
sich tief hinab, 
seine Hände hielt er an den Seiten des sich nun streckenden Halses. Mit weit 
ausgreifenden 
Schritten flog Ajaran über erst kürzlich abgeerntete Felder. Staub und Stroh 
wirbelte auf und 
Semrics Mantel peitschte im Wind. \textit{Wovor fliehst du, Bruder?}, wisperte 
Riolean leise.\\
Ihre Stimme schien vom Wind getragen zu werden. Niemals, nicht einmal Ajarans 
schnellster Lauf, 
konnte den Wind hinter sich zurück lassen. ``Verschwinde!'', schrie er dem Wind 
entgegen. \\
Semric sah kaum was vor ihm lag, aber sobald Ajarans Lauf sich zu verlangsamen 
schien, trieb er den 
Hengst unerbittlich weiter. Er spürte nur noch den zerrenden Wind und die 
Muskeln des Pferdes unter 
sich. Semric hatte Mühe, sich auf dem Rücken des Tieres zu halten und klammerte 
sich mit seinen 
Knien fest. Er schwitzte ebenso sehr wie Ajaran. \\
Irgendwann, der Mond stieg bereits am Himmel empor, spürte er eisiges Wasser 
ans einen Waden. 
Tropfen spritzten empor, bedeckten sein Haar und seine Hände, hüllten seine 
Füße ein. Er war am 
Meer angelangt. Ajaran scheute vor den Wellen und stieg auf die Hinterhand. Die 
Zügel glitten durch 
seine feuchten Hände. Der König hätte zugreifen können. Mit Händen und Knien 
sich oben halten 
können. Es kam ihm vor, als würde die Zeit langsamer verrinnen, während er 
zusah, wie die Zügel ihm 
entglitten. Er sah noch Ajarans Hufe, welche sich vor dem tiefen Mond 
abzeichneten, durch die Luft 
wirbeln.
\textit{Mein Bruder}, wisperte sie leise. Bedauern lag in den Worten. 
\textit{Du kannst nicht vor 
dir selbst fliehen.}\\
Das Meer empfing ihm in seiner kalte Umarmung. Er drang ein in eine andere 
Welt. Eine Welt, in der 
das Mondlicht keinen Zutritt hatte und die Stille herrschte. Die Kälte drang 
sofort in seine 
Glieder. Eisige Klauen griffen nach seinem wild pochendem Herzen. Er spürte den 
Aufprall am Grund. 
Der grobe Sand des Strandes wirbelte auf und seine Finger suchten vergebens 
nach Halt. Aber nur 
halbherzig. Das Wasser wirbelte um ihn herum. Ein Huf donnerte neben seinem 
Handgelenk herab und 
verschwand sofort wieder. Dafür zog ein weiterer über ihn vorbei. 
\textit{Ajaran}, vermutete Semric 
träge. \\
Etwas packte ihm am Hemd. Ruckartig durchbrach sein Kopf die unruhige 
Wasseroberfläche. Semric sah 
seinem Retter lange ins Gesicht. Als würde er den Mann das erste Mal in seinem 
Leben sehen. Der 
König und sein Leibwächter standen bis zur Brust im Meer. Die Wellen wogen 
vorüber und beide 
starrten sich schwer Atem an, während das karge Mondlicht es nicht vermochte, 
die Finsternis zu 
verdrängen. \\
Die Kapuze des Mantels war längst verrutscht und Semric blickte in ein paar 
schwarze Augen. Das 
dunkle Haar seines Leibwächters war feucht vor Wasser und Schweiß. Bartstoppel 
zierten sein Kinn 
und die Wange. Ein untrügliches Zeichen, dass der Mann nicht viel von den 
Gebräuchen der 
saleicanischen Krieger hielt. \\
``Ihr seid ein Narr'', keuchte er schließlich und schob ihn zurück an Land. \\
Beide Männer taumelten erschöpft auf den festen Grund zu und fielen in den 
kalten Sand. Ajaran 
hatte sich einige Schritte entfernt zu Erhims Reittier gesellt und sog dessen 
Gelassenheit förmlich 
in sich ein, um sich selbst zu beruhigen. Männer wie Pferd tropften vor Nässe 
und rangen nach Luft. 
Semric lag flach neben seinem sonst so stillen und ruhigen Leibwächter. \\
``Warum'', stieß dieser ungehalten hervor: ``Warum will sich ein König aus 
Feuer nur ständig in das 
Meer stürzen!''\\
Semric hustete und begehrte auf. ``Gar nicht... wahr! Es war ein Unfall.''\\
``Erst jagt Ihr den Wind, richtet beinahe Euer Pferd zugrunde und dann sinkt 
Ihr wie ein Stein. Ihr 
habt es nicht einmal versucht!'' Erhim spie die Worte regelrecht aus, während 
er zum Himmel empor 
starrte. ``Und erzählt keine Lügen. Ich sehe Euren Blick, wenn Ihr am Hafen 
steht. Ich \emph{kenne} 
diesen Blick.''\\
Semric schwieg überrascht. Er wusste nicht, was er sagen sollte und auch sein 
Leibwächter schien 
sich wieder an seinen Stand zu erinnern. Schweigend lagen die Männer, die sich 
in den letzten zehn 
Jahren jeden Tag gesehen und doch kaum miteinander gesprochen hatten, 
nebeneinander und starrten in 
den Nachthimmel. Und Semric fragte sich zum ersten Mal, wer wohl dieser Mann 
war, der jede Nacht 
vor seinem Zimmer wachte und jeden seiner Ausflüge begleitete. Und wie gut 
dieser Mann, dieser 
Schatten vor seiner Türe, ihn wirklich kannte. Vielleicht war er der einzige 
Mann, der den König 
Saleicas wirklich kannte.\\
``Und?'', fragte der König schließlich: ``Magst du noch ein paar Beleidigungen 
loswerden? Jetzt ist 
die Gelegenheit.''\\
Erhims nasser Mantel raschelte, als er sich bewegte. Die folgende Stille sank 
wie ein erstickendes 
Tuch über Semric und er schluckte schwer. War es das Schicksal eines 
Herrschers, dass niemand mit 
ihm reden wollte? Würde Rioleans Stimme die einzige sein, die ihn verfluchte, 
beleidigte, tröstete 
und ermutigte? \\
``Die Sterne sind hier in deinem Land erloschen'', sagte der Schatten plötzlich 
mit einer rauen 
Stimme. ``Ich vermisse sie. Das Leuchten. Das Glühen. Und ihre Geschichten.''\\
Semric kniff die Augen zusammen und starrte in den Nachthimmel. Er erkannte 
keinen Unterschied zu 
sonstigen Nächten. ``Wovon sprichst du? Sehen die Sterne in den Kolonien anders 
aus?''\\
Ein Laut, der an einen knurrenden Hund erinnerte, kam aus Erhims Richtung. 
``Die Kolonien sind 
nicht ein Land. Sie sind nicht ein Volk. Sie teilen nicht die selbe Geschichte 
oder Kultur, Religion 
oder Vermächtnis. Das Einzige, was sie teilen, ist den Eroberer.''\\
Semric räusperte sich verlegen. ``Dann... also in deinem Land.''\\
Diesmal war ein heißeres Lachen die Reaktion. ``Meine Heimat Esvoros ist kein 
Land mehr. Es ist 
tot. Das Land wie das Volk, sogar der Glaube. Aber nicht die Löwen haben das zu 
verantworten. Nur 
die Sterne leuchten noch und sehen hinab auf das verwesende Land. Sterne in 
Farben, die nur aus dem 
Herzen kommen können. Die Seelen der Legenden sind in den Himmel aufgestiegen 
und haben sich dort 
zu Ruhe gelegt, um auf uns Sterbliche zurückzublicken. Sie erzählen 
Geschichten, wenn man genau 
hinsieht und lauscht. Die weiße Königin erzählt von ihrem Reich auf den Spitzen 
der Gebirge. Die 
rote Feder zeugt von dem brennenden Vogel, die sie einst zierte. Das grüne Auge 
zeigt sich nur 
Sterblichen, die würdig sind. Und die rote Nadel... die steht einfach da. Keine 
Ahnung, vielleicht 
gab es mal eine heldenhafte Näherin. Oder vielleicht ist es keine Nadel sondern 
ein Pfeil... Oder 
ein Stock.''\\
``Oder eine Linie'', pflichtete Semric gedankenverloren bei.\\
Erhims Lachen erklang so plötzlich und laut, dass der König zusammen zuckte. 
``Genau. Eine Linie! 
Aber es wird kalt. Schleichen wir uns nun zurück und hoffen, dass die Priester 
nichts 
mitbekommen?''\\
Ruckartig kam Semric auf die Beine und klopfte sich den Sand von seiner nassen 
Kleidung. Ein 
vergeblicher Versuch. ``Nein!'', fauchte er: ``Ich habe mich vor niemandem zu 
rechtfertigen!''\\
Erhim erhob sich langsam und mit bedacht. Er kratzte sich am Kinn und nickte. 
``Aber es ist 
kalt.''\\
Semric überlegte flüchtig, dann stapfte er auf Ajaran zu und ergriff die Zügel. 
``Es gibt genug 
Häuser des Adels hier in der Nähe. Es wird eine Ehre sein, ihrem König einen 
Tee servieren zu 
dürfen!''\\



Das Buch in ihren Händen war alt und das Papier dünn. Die Schrift war zierlich 
und elegant nach rechts geneigt. Und auch die Illustrationen waren detailreich, 
aber jeder Strich war präzise gesetzt und keiner wirkte überflüssig. Ilias 
Blick huschte gelangweilt über die Buchstaben und Satzzeichen. Es handelte sich 
um ein kasirisches Buch. Sie hatte gehofft, dass das lesen in der Fremdsprache 
eine interessantere Beschäftigung wäre, aber heute war schlicht einer dieser 
Tage, an denen überhaupt nichts interessant war. So wie der Tag davor. Wie 
jeder, seit ihr Vater sie aus der Hauptstadt geschickt hatte. \\
Es ist noch so früh am Abend, dachte sie verdrossen und warf einen 
sehnsüchtigen Blick zur Türe hin. Noch war es warm, bald würde der Herbst 
einziehen. Noch könnte man in luftigen Kleidern im mit Lampions beleuchteten 
Park flanieren. Noch könnte man einen dekorativen Fächer tragen, ohne das es 
lächerlich wirkte. Sie könnte tanzen und mit jungen – oder wenigstens reichen – 
Herren plaudern. Ebenso bedauerlich war, dass ihre neu angefertigten Kleider nun 
keinen Sinn mehr machten. Die Saison war bald um und die Mode hielt nie still! 
\\
Ilia seufzte frustriert und drehte sich auf dem Sofa herum. Sie lehnte sich 
gegen ein weiches Kissen und hob ihre Beine auf den Stoff. Nachdenklich 
betrachtete sie ihre helle Haut, die einzelnen Muttermale und ihre Füße. Wenn 
sie nicht heraus geputzt in der Hauptstadt tanzen konnte, so hat sie sich 
trotzdem eines ihrer luftigen Kleider angezogen. Sollten die Bediensteten sie 
doch verpetzen, was änderte das? Ilia war der Meinung, dass es gar nicht 
schlimmer kommen konnte. \\
Mit einem deutlich hörbaren Geräusch klappte sie das Buch zu und warf es auf 
den Boden. \\
``Das ist doch lächerlich!'', fauchte sie ihre Gesellschafterin an. Ein dummes 
Mädchen mit Sommersprossen und schüchternem Blick. Ilia konnte sie nicht ernst 
nehmen. \\
``Was will er erreichen?'', rief sie aus. \\
``Dass Ihr auf Euren Verlobten wartet'', murmelte das Mädchen und hielt in ihrer 
Stickarbeit inne. \\
Ilia machte eine wegwerfende Handbewegung. ``Jozah... Er ist süß und tapfer. Und 
er begehrt mich. Aber deshalb darf ich nicht tanzen und plaudern gehen? Und 
warum mag Vater ihn so? Ich hatte schon deutlich bessere Alternativen.'' \\
``Es ist Eure Strafe, weil Ihr ständig einen anderen Mann als Eure große Liebe 
vorstellt'', wiederholte das Mädchen die Worte des Hausherren. \\
Die blonde Frau warf ihr einen vernichtenden Blick zu. \\
``Bist du ein Papagei, der nur das sagen kann, was man ihm beibringt?'' \\
Das Mädchen schwieg und dafür war Ilia äußerst dankbar. Sie seufzte ein 
weiteres Mal und starrte finster auf das Muster des Wandteppichs. Sie mochte 
Jozah. Er war interessant. Und so anders als die typischen Männer. Bodenständig. 
Und jedes seiner Worte hatte einen Wert. Wenn er sagte, dass sie schön war, dann 
bedeutete das deutlich mehr als von anderen Junggesellen. Sie konnte sich 
vorstellen, ihn wirklich zu heiraten. Aber trotzdem würde sie nicht auf 
anregende Gesellschaft von anderen verzichten. \\
Jemand räusperte sich. Ilia blickte auf und sah den Knecht in der Türe stehen. 
``Was? Ist Vater schon wieder da? Ich dachte, er hätte Wichtiges zu Hause zu 
tun'', spottete sie. \\
``Nein, Herrin. Es klopften zwei nasse Herren eben an die Türe. Ich habe sie 
fortgeschickt, ehe sie eine leidliche Ausrede hervorbringen konnten. Ich wollte 
Euch nur informieren, dass draußen zwielichtige Gestalten sind.''
``Fortgeschickt? Warum?'' \\
``Ich denke, das war im Sinne Eures Vaters. Außerdem zu Eurer Sicherheit...'' \\
Ilia sprang auf die Beine. ``Ich bin eine mündige Frau und entscheide selbst, 
was für mich sicher ist'', entschied sie, ``Kein Bettler würde es wagen, an eines 
unserer Häuser zu klopfen. Willst du etwa, dass jemand dem Hause Ma'Sa 
vorwirft, es wäre unhöflich und kenne die Regeln der Gastfreundschaft nicht?'' \\
Sie war schon längst an ihm vorbei gelaufen und ihre Stimme hallte durch das 
Haus. An der Türe angelangt, riss sie diese auf und spähte hinaus. Der Hof 
wurde von einer Laterne beleuchtet und Ilia sah sich zwei Männern gegebüber. 
Sie hatten die Abweisung wohl nicht akzeptieren wollen und der schlankere der 
Beiden hatte bereits die Hand erhoben um erneut zu klopfen.
Einen Moment sahen sie sich einander an, dann lachte sie hell auf. \\
``Der Schreiber! Der Schreiber vom Hafen. Sind Sie ins Wasser gefallen?'' \\
Er schien wohl überrascht, dass er ihr gegenüberstand. Seine eben noch 
gekränkte Miene wandelte sich zu einem verblüfften Gesichtsausdruck. \\
``Kommt herein, kommt herein. Ich lasse Tee aufsetzen. Ich hoffe, es ist in 
Ordnung, dass wir kein Feuer schüren, ich finde es ist noch zu warm dafür. Die 
Kleidung wird auch so trocknen'', plapperte Ilia und trat zurück. Mit einem 
strahlenden Lächeln bat sie die Gäste herein. \\
\textit{Vielleicht wird der Abend doch noch interessant.}