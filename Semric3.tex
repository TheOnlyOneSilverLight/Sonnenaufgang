\chapter{Das Schicksal der Welt}

Seit seiner nächtlichen Flucht, die so abrupt im Meer endete, waren einige Tage ins Land 
gezogen. Tage, die Semric größtenteils im Bett verbrachte. Die Kälte des Meeres schien nicht von ihm 
weichen zu wollen und er lag zitternd unter mehreren Decken. Er war lange nicht mehr so krank 
gewesen und hatte fast schon vergessen wie schrecklich man sich dabei fühlt. Und doch genoss er es. 
Tage, in denen keine Priester ihn zu irgendwelchen Anlässen vorführten. Tage, in denen er nicht 
heuchlerisch dem Adel zu lächeln oder Bittsteller anhören musste. Diese Tatsache wurde nur dadurch 
getrübt, dass Semric genau wusste, dass diese Dinge auch ohne ihn geschahen.\\
\textit{Vielleicht hat der Hohepriester sich auch schon meine Krone aufgesetzt.}\\
Ein dumpfes Klopfen erklang und Semric drehte seinen Kopf zur Türe.\\
''Herein``, krächzte er erschöpft.\\
Er rechnete eigentlich nicht mehr mit Gästen und es war auch zu spät am Abend, als dass die 
Bediensteten ihre Arbeiten verrichten müssten. Der Arzt war am Nachmittag dagewesen und hatte 
nichts 
weiter als Bettruhe und Kräutertee verschrieben. Wer also konnte es sein, außer Hisio-Mahar?\\
\textit{Vielleicht ein paar Meuchelmörder.}\\
Die Türe ging ein großes Stück auf und der Mann blieb erst einmal nur stehen und sah sich um. \\
''Erhim?``, fragte Semric irritiert.\\
Sein Leibwächter nickte knapp, dann trat er ein und schloss behutsam die Türe hinter sich. Heute 
trug der Mann keinen Mantel, nur schlichte Stoffkleidung und an seinem Gürtel einen Dolch und ein 
Kurzschwert. Er schien nur wenige Jahre älter als sein König zu sein. Semric entsann sich an 
damals, 
als Erhim diese Stelle antrat. Ein schmächtiger, trotzig schauender junger Mann. \\
''Was willst du?``\\
Der Mann zuckte mit den Schultern. ''Ich glaube, die Gefahr, dass meinem König etwas zustößt ist 
hier im Zimmer höher als draußen vor der Tür.``\\
Semric setzte sich mühsam auf und lehnte sich gegen den Kopf des Bettes. Es fühlte sich falsch an 
vor Erhim hilflos im Bett zu liegen. So konnte er sich einer winzigen Illusion hingeben, dass er 
doch noch irgendwelche Autorität besaß. \\
''Das fällt dir jetzt erst auf?``, fragte Semric sarkastisch.\\
Erhim stand immer noch an der Tür und wirkte er überrascht. Dann trat wieder der grimmige 
Ausdruck in seine Augen. ''Dem Priester fehlt nur noch die Krone.``\\
''Keine Sorge. Hisio-Mahar will kein König sein. Er hält nur gerne die Stricke in der Hand. Er wird 
mich auch nicht umbringen.``\\
''Ein eingesperrter Vogel stirbt auch ohne Gift.``\\
Semric runzelte die Stirn. ''Was willst du?``\\
Bestimmt schloss er nun die Tür und trat näher. Semric blickte zu ihm auf, konnte aber nicht 
erahnen, was hinter den dunklen Augen versteckt lag. ``Warum wolltet Ihr Euch umbringen?'', 
fragte er ruhig.\\
''Bezahle ich dich eigentlich gut für deine Fürsorge?``, entgegnete Semric amüsiert.\\
Erhim stutzte kurz. Dann rief er übertrieben: ''Was? Ich könnte dafür bezahlt werden, jede Nacht 
vor 
Eurer Tür zu stehen?``\\
''Nur, wenn du mich ab und an aus dem Meer fischt.``\\
''Warum?``, sagte der Mann plötzlich wieder ernst: ''Man zwang mich einen Eid auf Euer Leben zu 
schwören. Es ist hilfreich, wenn Ihr Euch selbst nichts bedeutet.``\\
Semric deutete auf einen der Sessel in der Nähe seines Bettes und Erhim ließ sich darauf nieder. 
Jedoch nur auf der Kante der Sitzgelegenheit, bereit, jeder Zeit aufzuspringen.\\
''Wer hat dich gezwungen? Es gibt keine Sklaverei in Saleica!``, entschied Semric.\\
''Das sagt der Sklave mit der goldenen Krone``, murrte Erhim leise.\\
Der König rang erschrocken nach Luft, aber eher er etwas ausrufen konnte, sprach der Wächter schon 
weiter. ''Ich war 16. Ich gehörte einem Clan an, der in der Savanne lebte. Wir waren Nomaden und 
zogen umher. Ab und an besuchten wir größere Städte um Handel zu treiben. Und nun… Saleicaner 
halten nicht viel von Nomaden, scheint mir. In vielen Ländern Eurer Kolonien waren wir herzlich 
immer willkommen. Wir brachten interessante Dinge zum handeln und selbst wenn wir nichts 
dergleichen hatten, luden uns die Leute in ihre Häuser ein. Unsere Geschichten waren das Kostbarste 
was wir sammelten und wir waren nie geizig mit ihnen. Wir teilten sie gerne. Mein Vater sagte, es 
gibt unzählige Geschichten und unzählige Arten sie zu erzählen, verschenken wir sie also. Die Magie 
einer Geschichte erkennt man nur in den Augen ihrer Zuhörer. Oh… und den Tanz. Mein Volk liebte den 
Tanz. Vielleicht, weil der Tanz auch nur eine Möglichkeit ist, eine Geschichte zu erzählen.``\\
Semric sank zurück in seine Kissen und sah ihn aus halb geschlossenen Lidern an. \\
''Aber das heißt nicht, dass wir wehrlos waren. Wir Nomaden lebten gefährlich. Ob Raubtiere oder 
Räuber. Wir machten selbst aus dem Kämpfen etwas elegantes. Etwas, was Geschichten erzählt. Aber 
das reichte nicht. Die Weisen meines Clans wurden gefoltert. Danach schlachteten sie alle ab.``\\
''Wieso?``, fragte Semric flüsternd.\\
Erhim schwieg einen Moment versunken in seine Erinnerungen. Sein Blick war ins nichts gerichtet, 
als 
er weiter erzählte: ''Man sollte mehrere Geschichten nicht miteinander vermischen. Lasst mich erst 
meine weiter erzählen.``\\
''Aber ist es nicht deine, wenn sie der Grund ist, weshalb du nun hier stehst?``\\
Der Leibwächter schüttelte den Kopf. ''Nein. Der Grund war, dass ich kämpfte. An diesem Tag tötete 
ich das erste Mal einen Menschen. Und das letzte Mal, das sollte ich vielleicht auch dazu sagen. 
Ich war jung. Ich bereute. Rache ist ein unwürdiger Grund. Ich schlug um mich wie ein tollwütiger 
Hund. Ich war so unwürdig wie die Mörder meines Clans.``\\
Semric hob abwehrend seine Hand. ''Ich verstehe schon``, murmelte er: ''Du sprichst von meinem 
Volk. Meinen Soldaten.``\\
Erhim schüttelte den Kopf. ''Da war es noch das Volk Eures Vaters. Die Soldaten Eures Vaters. Und 
sie handelten nach seinem Befehl. Den Rest der Geschichte könnt Ihr Euch bestimmt vorstellen.``\\
''Du hast aufgegeben und die Soldaten feierten dich für deinen – wenn auch zwecklosen – Mut und 
deine Tapferkeit.``\\
''Ja. Die Hoffnung hatte mich verlassen. Ich hatte keinen Grund mehr zu kämpfen, alle die ich 
liebte waren tot. Sie schleppten mich zu einem Priester und er sah lächelnd auf mich herab. Ich 
werde dieses Lächeln nie vergessen. Er belächelte all das, was sie mir geraubt hatten. Er redete 
irgendeinen Unsinn davon, dass ich es würdig wäre, unter Osymas Licht zu leben.``\\
''Wenn du so gekämpft hast, wieso hast du dich zu einem Eid zwingen lassen?``\\
''Mein Volk ist der Meinung, dass man immer eine Wahl hat. Ich hätte meiner Familie ins nächste 
Leben folgen können, stattdessen schwor ich mein Leben einem Prinzen zu geben. Die Alternative war 
damals nicht erstrebenswert für mich.`` Erhim grinste breit. ''Außerdem wollte ich dabei sein, wenn 
die Welt untergeht!``\\
Der König runzelte die Stirn und wollte zu einer Frage ansetzen, aber da hob Erhim mahnend den 
Finger und fuhr fort. ''Da wären wir bei dem Wieso. Mein Volk wanderte durch viele Länder. Es 
sprach viele Sprachen, tanzte viele Tänze. Und es gibt ein Land… dort herrschen die Mahig-Na. Wir 
wussten es. Wir beteten zu ihnen, denn sie schienen göttlicher als alles andere, was wir je gesehen 
oder in Geschichten gehört hatten. Es gab nie viele Mahig-Na, denn ihr Leben ist lang. Es heißt, sie 
sind die Kinder von Glas und Edelstein. Wesen aus Licht und Schwärze, die sich in die Lüfte erheben 
können. Ihre Klauen können die Erde aufreißen und ihr Blut brennt. Es sind Wesen, die eigentlich 
nicht existieren sollten. Wesen, aus einer anderen Welt. Euer Vater wollte sie. Und mein Volk 
wusste, wo sie waren.``\\
''Und diese Mahig-Na werden den Weltuntergang bringen?``, fragte Semric skeptisch.\\
Erhim lachte. ''Nein. Wenn die Priester sie in Euren Namen einsetzen um Krieg zu führen und Leid 
über die Welt zu bringen, dann werden sich ihre Eltern aus der anderen Welt erheben. Dann wären die 
für den Weltuntergang verantwortlich.``\\
''Und hat mein Vater sie gefunden?``\\
''Ja. Einen. Die anderen Mahig-Na flohen, als die Menschen nahten. Sie wurden nicht erschaffen um 
uns zu töten. Vielleicht ist es ihnen sogar untersagt... Der Mahig-Na wurde eingekreist, war hin 
und her gerissen zwischen dem Gebot der Schöpfer und seinem Überlebenswillen. Eure Schwester führte 
das Kommando.``\\
Semric war gebannt von Erhims erzählungen. Er wargte kaum zu Atmen um auch ja keine Silbe zu 
überhören. Der Leibwächter erzählte, als wäre er selbst Zeuge dieser Taten gewesen, aber das wra 
unmöglich. ''Sie stand in der ersten Linie. Ihre Worte befahlen den Angriff und ihr Schwert vergoss 
heiliges Blut. Dafür wurde sie gerichtet. Aber zu spät. Das Geheimnis war gelüftet. Euer Vater 
kannte es. Die Priester kennen es. Und noch schlimmer, sie besitzen ein Herz der Schöpfer. Deshalb 
habe ich diesen Eid geleistet. Ich will sie sehen.``\\
''Die Mahig-Na?``, fragte der König verunsichert.\\
''Die Schöpfer!``\\
Semric sank erschöpft in sich zusammen. Sein Blick richtete sich zur Decke, während er über die 
Worte des Mannes nachdachte. \textit{Mahig-Na}, wisperte Riolean. Ihre Stimme klang fasziniert. 
\textit{Wir wollten sie sehen. Vater und ich. Deshalb waren wir in den Kolonien.}\\
Semric schüttelte kaum merklich den Kopf. Er konnte sich diese Wesen nicht vorstellen und wollte 
sie auch nicht sehen. Er wollte nicht darüber nachdenken, dass Erhims Worte einer Prophezeiung 
glichen.\\
\textit{Du bist ein feiger König. Wenn es solche Wesen gibt, dann sollten sie dir dienen und nicht 
den Priestern.}\\
''Wenn es solche Wesen gibt, dann sollten sie niemandem dienen``, flüsterte Semric.\\
Erhim nickte. ''Sie sind Götter. Wir sollten ihnen dienen.``\\

Der Morgen war längst gegraut und doch lag Stille über der Festung. Semric schloss den obersten 
Knopf des schlichten Hemdes und warf einen letzten Blick in den Spiegel. Das ungekämmte Haar 
verbarg er eilig unter der dunklen Mantelkapuze. Keine glorreiche Tarnung, aber er 
hatte nicht viel Zeit, wenn sein Ausflug vom Hohepriester unbemerkt bleiben sollte. Hisio-Mahar 
pflegte den Tag bei Sonnenaufgang mit einer Andacht im Tempel einzuläuten und schließlich den Morgen 
über in seinen Büro an Intrigen und Regierungsaufgaben zu verbringen. Selten hatte er sich vor 
Mittag bei Semric blicken lassen. Der König war als Langschläfer bekannt, ebenso wie die meisten 
Adeligen. Dass dies nur eine Lüge war, wusste vermutlich nicht mal Erhim, der in diesen Stunden nach 
seiner Nachtwache auch Schlaf nachholte. Wechselnde Wachen nahmen dann dessen Posten vor der Tür 
ein. Semric waren diese Stunden die liebsten an Tag. Stunden der Ruhe. Stunden, in denen er sich in 
Bücher vertiefen konnte und sogar Riolean meist still war. Aber heute würde er sich nicht in der 
Reisebiographie des Aros Enhelns verlieren.\\
\textit{Du wirst nicht durch das Fenster klettern!}, schalt Riolean ihn: \textit{Du bist der 
König!}\\
Semric gestand sich nur ungerne ein, dass sie recht hatte und verwarf den Plan. Also fasste er all 
seinen Stolz und verließ seine Gemächer durch die Tür. Er sah die Wache nicht an, als sie sich fast 
schon erschrocken zu ihm umwandte. Ein knapper Befehl und er ließ den Mann dort zurück. Er kam nur 
vereinzelten Bediensteten vorrüber, die sich nach einem Moment der Verblüffung eilig verneigten. 
Diesmal ließ er die Knechte Ajaran satteln und nahm die Zügel mit einem nicken entgegen. Semric 
wusste noch nicht, wohin sein Weg führen würde. Aber er schuldete Ajaran nach dem letzten wilden 
Ritt ein paar entspanntere Stunden in Zweisamkeit. Geschickt zog er sich in den Sattel, ordnete 
liebvoll Ajarans Mähne und gab ihm mit einem sanften Stupsen zu verstehen, dass es losgeht. Der 
Hengst blinzelte ebenso wie Semric der Morgensonne entgegen. Er schnaufte freudig und schüttelte 
seine Mähne, die wie Bronze schimmerte. Semric musste lachen. ''Du Angeber``, murmelte er dem Tier 
zu und passte seine Bewegungen dem Trabrhythmus an.\\
Die Straßen der Hauptstadt waren belebt vom einfachen Volk. Kaum ein Reiter begegnete ihm, dafür 
unmengen an Männern und Frauen, die geschäftig mit Körben und Karren auf dem Weg zum Markt waren. 
Kinder und Jugendliche, die auf dem Weg zu den öffentlichen Schulen waren und die ein oder andere 
Stadtwache, die gähnend auf ihre Ablösung wartete. Auf Ajarans Rücken konnte er über das hölzerne 
Tor der Kaserne blicken und beobachtete einen Moment wie die Rekruten Runde um Runde rannten, 
begleitet von der herrischen Stimme ihres Ausbilders. Semric sah zwei weißhaarige Männer aus einem 
der Gebäude kommen und im Gespräch vertieft auf das Tor zuschlendernd. Offizier Lerins stattliche 
Gestalt war kaum zu übersehen. Dagegen wirkte sein Begleiter nahezu schmächtig und mehr als einen 
Kopf kleiner. Der vergoldete Löwenkopf auf seinem Spazierstock fing das Licht der Morgensonne ein. 
Ajaran schnaufte und trat unruhig zwei Schritte zur Seite. Reflexartig griff Semric die Zügel 
fester und zischte mahnend. Jetzt waren die Männer nahe genug um Lerins Begleitung erkennen zu 
können.\\
Der König gab der Ungeduld des Pferdes nach und ließ es antraben, während er noch überlegte, was 
Vito Ma'Sah in der Kaserne zu suchen hatte. Vor allem lenkten diese Überlegungen seine Gedanken 
ganz unweigerlich auf dessen schöne Tochter.\\
Semric biss sich auf die Zunge und schüttelte den Kopf. Es war reiner Wahn, dass eine solche Frau 
wie Ilia - mit einem solchen Vater - ihn nicht erkannte. Warum sie dieses Spiel aber weiter trieb, 
konnte er sich nicht vorstellen. Und er hatte keinerlei Bedürfnis, es zu offenbaren.\\
Ajaran lief über die weiche Erde der Parkanlagen. Gärtner machten sich gerade daran, die Wege von 
Unkraut zu befreien. Semric nickte ihnen gedankenversunken zu und sah schon von weitem, Hisio-Mahar 
gefolgt von zwei Priestern, die mehr nach Kriegern als Predigern aussahen, auf die Wegkreuzung 
zukommen. Tief holte er Luft und hoffte, dass seine innere Unruhe sich nicht in Ajaran 
wiederspiegel würde. Der Hohepriester und seine Wachen verbeugten sich tief.\\
''Ein schöner Morgen, nicht wahr, mein König?``, grüßte Hisio-Mahar, während Ajaran ohne sein 
gemächliches Tempo zu ändern an ihm vorbei trottete.\\
Seine Begleiter fielen einige Schritte zurück, während der Hohepriester neben Semric blieb. ''Dank 
sei Osyma, dass Ihr so schnell genesen seid``, plauderte er weiter. Seine Umhänge raschelten leise 
bei jedem Schritt und Ajaran warf einige Male den Kopf herum um ihn misstrauisch zu beobachten. 
Semric schwieg verbissen und nickte nur.\\
''Ihr solltet mehr auf Euch achten, Hoheit. Mitternächtliche Badegänge, einsame Ausritte. Es könnte 
so viel passieren.``\\
Der junge König sah auf, starrte Hisio-Mahar einen Augenblick stumm an, ehe er sich wieder fasste 
und nach vorn sah. Natürlich hatte Semric erwartet, dass der Priester seinen heutigen Ausflug 
entdecken würde. Aber er konnte einach nichts von den Ritten zum Hafen wissen... \\
''Ich bitte Euch ergebenst, solche Risiken zu vermeiden, mein König``, fuhr er fort: ''Ich habe 
zwar einen Ersatz in der Hinterhand, aber den Umbruch der Regierung während ein Krieg in 
Aussicht steht... das würde ich gern vermeiden.``\\
"Ihr sprecht von Marek?'', riet Semric in die folgende Stille hinein.\\
Er konnte sich nur vage an seinen entfernten Vetter erinnern. Er war weit älter als Semric, hatte 
ihn aber nur zu wenigen offiziellen Anlässen gesehen. Der Letzte war die Einäscherung König Kareens 
und Prinzessin Rioleans. Aril Hor'Le - die Nichte König Kareens - hatte in ihrer Jugend schon die 
Hauptstadt verlassen, nachdem eine unschickliche Affaire zum Koch heraus kam. Semrics Vater 
schätzte den Koch sehr, daher schickte er statt ihn seine schwangere Nichte fort.\\
``Was treibt er denn so?'', fragte Semric. Er versuchte einen amüsierten Tonfall zu halten und 
seine Anspannung über die Drohung nicht zu zeigen.\\
``Er dient seit einigen Jahren auf einen der Inseln an der nördlichen Ostküste dem Allmächtigen.''\\
Semric nickte. So etwas hatte er sich schon gedacht. ``Und? Hat er sich schon eine Krone aus 
Korallen und Seetang gebastelt? Um bereit zu sein, falls Ihr mir die Fäden durchschneidet?''\\
Hisio-Mahar lachte. ``Vergleicht Ihr Euch mit einer Marionette? Ach... mein König... ich sehe Euch 
mehr wie einen winselnden Hund. In den letzten Wochen habt Ihr zwar ein paar mal nach meiner 
gütigen Hand geschnappt... aber um dem Einhalt zu gebieten bin ich ja nun hier.''\\
\textit{Sag etwas!}, forderte Riolean ihn scharf auf, aber Semric knirschte nur mit den Zähnen.\\
``Ich überlege sogar, Marek einzuladen'', sagte der Priester grüblerisch.\\
Semric schwieg verbissen.\\
Hisio-Mahar griff in die Zügel und brachte Ajaran zum stehen. Eindringlich blickte er zu seinem 
König hinauf und sagte leise: ``Aber soweit muss es nicht kommen, mein König. Es würde nicht 
reibungslos in meine Pläne passen, also tut uns beiden den Gefallen, und schnappt nicht mehr wie 
ein Welpe unnütz in die Luft. Unser Spiel hat doch viele Jahre gut funktioniert. Saleica blüht. Das 
Reich wird weiter wachsen. Der Wirtschaft geht es gut. Einzig ein paar ungläubige Bauern stehen im 
Weg. Was stört Euch plötzlich? Ist es wegen der Frau?''\\
``Lasst sie da raus!'', knurrte Semric.\\
Hisio-Mahar runzelte die Stirn. ``Dann haltet Euch fern von ihr. Ich möchte mich ungerne zum 
jetzigen Zeitpunkt mit dem alten Adel anlegen, aber ich werde es tun. Nun... vielleicht ist es an 
der Zeit für eine Eheschließung, wenn es Euch darum geht. Ich habe auch schon nach einer geeigneten 
Kandidatin gesucht und werde sie Euch in den nächsten Wochen vorstellen.''\\
Der König wollte aufbegehren, doch Hisio-Mahar schnitt ihm das Wort ab. ``Ihr seid nicht dumm, 
Semric. Ihr versteht Morddrohungen, wenn ich sie äußere. Haltet Euch fern von Ma'Sah. Sie ist eine 
zu hübsche Dame, um an ihrem eigenen Blut zu ersticken.''\\
