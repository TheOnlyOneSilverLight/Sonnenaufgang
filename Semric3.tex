\chapter{Das Schicksal der Welt}

Seit seiner nächtlichen Flucht, die so abrupt im Meer endete, waren einige Tage ins Land 
gezogen. Tage, die Semric größtenteils im Bett verbrachte. Die Kälte des Meeres schien nicht von ihm 
weichen zu wollen und er lag zitternd unter mehreren Decken. Er war lange nicht mehr so krank 
gewesen und hatte fast schon vergessen wie schrecklich man sich dabei fühlt. Und doch genoss er es. 
Tage, in denen keine Priester ihn zu irgendwelchen Anlässen vorführten. Tage, in denen er nicht 
heuchlerisch dem Adel zu lächeln oder Bittsteller anhören musste. Diese Tatsache wurde nur dadurch 
getrübt, dass Semric genau wusste, dass diese Dinge auch ohne ihn geschahen.\\
\textit{Vielleicht hat der Hohepriester sich auch schon meine Krone aufgesetzt.}\\
Ein dumpfes Klopfen erklang und Semric drehte seinen Kopf zur Türe.\\
''Herein``, krächzte er erschöpft.\\
Er rechnete eigentlich nicht mehr mit Gästen und es war auch zu spät am Abend, als dass die 
Bediensteten ihre Arbeiten verrichten müssten. Der Arzt war am Nachmittag dagewesen und hatte 
nichts 
weiter als Bettruhe und Kräutertee verschrieben. Wer also konnte es sein, außer Hisio-Mahar?\\
\textit{Vielleicht ein paar Meuchelmörder.}\\
Die Türe ging ein großes Stück auf und der Mann blieb erst einmal nur stehen und sah sich um. \\
''Erhim?``, fragte Semric irritiert.\\
Sein Leibwächter nickte knapp, dann trat er ein und schloss behutsam die Türe hinter sich. Heute 
trug der Mann keinen Mantel, nur schlichte Stoffkleidung und an seinem Gürtel einen Dolch und ein 
Kurzschwert. Er schien nur wenige Jahre älter als sein König zu sein. Semric entsann sich an 
damals, 
als Erhim diese Stelle antrat. Ein schmächtiger, trotzig schauender junger Mann. \\
''Was willst du?``\\
Der Mann zuckte mit den Schultern. ''Ich glaube, die Gefahr, dass meinem König etwas zustößt ist 
hier 
im Zimmer höher als draußen vor der Tür.``\\
Semric setzte sich mühsam auf und lehnte sich gegen den Kopf des Bettes. Es fühlte sich falsch an 
vor Erhim hilflos im Bett zu liegen. So konnte er sich einer winzigen Illusion hingeben, dass er 
doch noch irgendwelche Autorität besaß. \\
''Das fällt dir jetzt erst auf?``, fragte Semric sarkastisch.\\
Erhim stand immer noch an der Tür und wirkte er überrascht. Dann trat wieder der grimmige 
Ausdruck in seine Augen. ''Dem Priester fehlt nur noch die Krone.``\\
''Keine Sorge. Hisio-Mahar will kein König sein. Er hält nur gerne die Stricke in der Hand. Er wird 
mich auch nicht umbringen.``\\
''Ein eingesperrter Vogel stirbt auch ohne Gift.``\\
Semric runzelte die Stirn. ''Was willst du?``\\
Bestimmt schloss er nun die Tür und trat näher. Semric blickte zu ihm auf, konnte aber nicht 
erahnen, was hinter den dunklen Augen versteckt lag. ``Warum wolltet Ihr Euch umbringen?'', 
fragte er ruhig.\\
''Bezahle ich dich eigentlich gut für deine Fürsorge?``, entgegnete Semric amüsiert.\\
Erhim stutzte kurz. Dann rief er übertrieben: ''Was? Ich könnte dafür bezahlt werden, jede Nacht 
vor 
Eurer Tür zu stehen?``\\
''Nur, wenn du mich ab und an aus dem Meer fischt.``\\
''Warum?``, sagte der Mann plötzlich wieder ernst: ''Man zwang mich einen Eid auf Euer Leben zu 
schwören. Es ist hilfreich, wenn Ihr Euch selbst nichts bedeutet.``\\
Semric deutete auf einen der Sessel in der Nähe seines Bettes und Erhim ließ sich darauf nieder. 
Jedoch nur auf der Kante der Sitzgelegenheit, bereit, jeder Zeit aufzuspringen.\\
''Wer hat dich gezwungen? Es gibt keine Sklaverei in Saleica!``, entschied Semric.\\
''Das sagt der Sklave mit der goldenen Krone``, murrte Erhim leise.\\
Der König rang erschrocken nach Luft, aber eher er etwas ausrufen konnte, sprach der Wächter schon 
weiter. ''Ich war 16. Ich gehörte einem Clan an, der in der Savanne lebte. Wir waren Nomaden und 
zogen umher. Ab und an besuchten wir größere Städte um Handel zu treiben. Und nun… Saleicaner 
halten 
nicht viel von Nomaden, scheint mir. In vielen Ländern Eurer Kolonien waren wir herzlich immer
willkommen. Wir brachten interessante Dinge zum handeln und selbst wenn wir nichts dergleichen 
hatten, luden uns die Leute in ihre Häuser ein. Unsere Geschichten waren das Kostbarste was wir 
sammelten und wir waren nie geizig mit ihnen. Wir teilten sie gerne. Mein Vater sagte, es gibt 
unzählige Geschichten und unzählige Arten sie zu erzählen, verschenken wir sie also. Die Magie 
einer Geschichte erkennt man nur in den Augen ihrer Zuhörer. Oh… und den Tanz. Mein Volk liebte den 
Tanz. Vielleicht, weil der Tanz auch nur eine Möglichkeit ist, eine Geschichte zu erzählen.``\\
Semric sank zurück in seine Kissen und sah ihn aus halb geschlossenen Lidern an. \\
''Aber das heißt nicht, dass wir wehrlos waren. Wir Nomaden lebten gefährlich. Ob Raubtiere oder 
Räuber. Wir machten selbst aus dem Kämpfen etwas elegantes. Etwas, was Geschichten erzählt. Aber 
das reichte nicht. Die Weisen meines Clans wurden gefoltert. Danach schlachteten sie alle ab.``\\
''Wieso?``, fragte Semric flüsternd.\\
Erhim schwieg einen Moment versunken in seine Erinnerungen. Sein Blick war ins nichts gerichtet, 
als 
er weiter erzählte: ''Man sollte mehrere Geschichten nicht miteinander vermischen. Lasst mich erst 
meine weiter erzählen.``\\
''Aber ist es nicht deine, wenn sie der Grund ist, weshalb du nun hier stehst?``\\
Der Leibwächter schüttelte den Kopf. ''Nein. Der Grund war, dass ich kämpfte. An diesem Tag tötete 
ich das erste Mal einen Menschen. Und das letzte Mal, das sollte ich vielleicht auch dazu sagen. 
Ich war jung. Ich bereute. Rache ist ein unwürdiger Grund. Ich schlug um mich wie ein tollwütiger 
Hund. Ich war so unwürdig wie die Mörder meines Clans.``\\
Semric hob abwehrend seine Hand. ''Ich verstehe schon``, murmelte er: ''Du sprichst von meinem 
Volk. 
Meinen Soldaten.``\\
Erhim schüttelte den Kopf. ''Da war es noch das Volk Eures Vaters. Die Soldaten Eures Vaters. Und 
sie 
handelten nach seinem Befehl. Den Rest der Geschichte könnt Ihr Euch bestimmt vorstellen.``\\
''Du hast aufgegeben und die Soldaten feierten dich für deinen – wenn auch zwecklosen – Mut und 
deine 
Tapferkeit.``\\
''Ja. Die Hoffnung hatte mich verlassen. Ich hatte keinen Grund mehr zu kämpfen, alle die ich 
liebte waren tot. Sie schleppten mich zu einem Priester und er sah lächelnd auf mich herab. Ich 
werde dieses Lächeln nie vergessen. Er belächelte all das, was sie mir geraubt hatten. Er redete 
irgendeinen Unsinn davon, dass ich es würdig wäre, unter Osymas Licht zu leben.``\\
''Wenn du so gekämpft hast, wieso hast du dich zu einem Eid zwingen lassen?``\\
''Mein Volk ist der Meinung, dass man immer eine Wahl hat. Ich hätte meiner Familie ins nächste 
Leben 
folgen können, stattdessen schwor ich mein Leben einem Prinzen zu geben. Die Alternative war damals 
nicht erstrebenswert für mich.`` Erhim grinste breit. ''Außerdem wollte ich dabei sein, wenn die 
Welt 
untergeht!``\\
Der König runzelte die Stirn und wollte zu einer Frage ansetzen, aber da hob Erhim mahnend den 
Finger und fuhr fort. ''Da wären wir bei dem Wieso. Mein Volk wanderte durch viele Länder. Es 
sprach 
viele Sprachen, tanzte viele Tänze. Und es gibt ein Land… dort herrschen die Mahig-Na. Wir wussten 
es. Wir beteten zu ihnen, denn sie schienen göttlicher als alles andere, was wir je gesehen oder in 
Geschichten gehört hatten. Es gab nie viele Mahig-Na, denn ihr Leben ist lang. Es heißt, sie sind 
die Kinder von Glas und Edelstein. Wesen aus Licht und Schwärze, die sich in die Lüfte erheben 
können. Ihre Klauen können die Erde aufreißen und ihr Blut brennt. Es sind Wesen, die eigentlich 
nicht existieren sollten. Wesen, aus einer anderen Welt. Euer Vater wollte sie. Und mein Volk 
wusste, wo sie waren.``\\
''Und diese Mahig-Na werden den Weltuntergang bringen?``, fragte Semric skeptisch.\\
Erhim lachte. ''Nein. Wenn die Priester sie in Euren Namen einsetzen um Krieg zu führen und Leid 
über 
die Welt zu bringen, dann werden sich ihre Eltern aus der anderen Welt erheben. Dann wären die für 
den Weltuntergang verantwortlich.``\\
''Und hat mein Vater sie gefunden?``\\
''Ja. Einen. Die anderen Mahig-Na flohen, als die Menschen nahten. Sie wurden nicht erschaffen um 
uns zu töten. Vielleicht ist es ihnen sogar untersagt... Der Mahig-Na wurde eingekreist, war hin 
und her gerissen zwischen dem Gebot der Schöpfer und seinem Überlebenswillen. Eure Schwester führte 
das Kommando.``\\
Semric war gebannt von Erhims erzählungen. Er wargte kaum zu Atmen um auch ja keine Silbe zu 
überhören. Der Leibwächter erzählte, als wäre er selbst Zeuge dieser Taten gewesen, aber das wra 
unmöglich. ''Sie stand in der ersten Linie. Ihre Worte befahlen den Angriff und ihr Schwert vergoss 
heiliges Blut. Dafür wurde sie gerichtet. Aber zu spät. Das Geheimnis war gelüftet. Euer Vater 
kannte es. Die Priester kennen es. Und noch schlimmer, sie besitzen ein Herz der Schöpfer. Deshalb 
habe ich diesen Eid geleistet. Ich will sie sehen.``\\
''Die Mahig-Na?``, fragte der König verunsichert.\\
''Die Schöpfer!``\\
Semric sank erschöpft in sich zusammen. Sein Blick richtete sich zur Decke, während er über die 
Worte des Mannes nachdachte. \textit{Mahig-Na}, wisperte Riolean. Ihre Stimme klang fasziniert. 
\textit{Wir wollten sie sehen. Vater und ich. Deshalb waren wir in den Kolonien.}\\
Semric schüttelte kaum merklich den Kopf. Er konnte sich diese Wesen nicht vorstellen und wollte 
sie 
auch nicht sehen. Er wollte nicht darüber nachdenken, dass Erhims Worte einer Prophezeiung 
glichen.\\
\textit{Du bist ein feiger König. Wenn es solche Wesen gibt, dann sollten sie dir dienen und nicht 
den Priestern.}\\
''Wenn es solche Wesen gibt, dann sollten sie niemandem dienen``, flüsterte Semric.\\
Erhim nickte. ''Sie sind Götter. Wir sollten ihnen dienen.``\\

Es war Wochen her, dass der König das letzte Mal die Tempel Brom-Dalars besucht hatte. Und auch 
jetzt mied er die Zeiten in denen die Priester ihre Predigten hielten oder den Menschen zuhörten. 
Und es war seit fast zehn Jahren das erste Mal, dass Semric ohne Krone diese heiligen Hallen 
betrat.\\
\textit{Ich war damals noch so jung.}\\
Auch zu dieser frühen Morgenstunde, in der der Mond noch hell am Himmel stand, weilten einige 
Priester im Tempel. Sie saßen gebeugt über ihren Kerzen und murmelten oder schwiegen in sich 
hinein. 
Auch einige vom einfachen Volk sah man, die die offenen Kamine als warmen Schlafplatz nutzten. 
Semric durchquerte das Gebäude und sah dabei nur auf die Marmorfliesen unter sich. Schwarz war das 
Gestein, durchdrungen von weißen Linien, die ihn an Blitze in tiefer Nacht denken ließen. Keine 
Fliese glich der anderen und Semric hätte sich in diesem Abbild verlieren können. Aber dann hob er 
den Kopf und sah auf die Flammen. Er war in einer abgelegenen Nische angelangt. Das offene Feuer 
war recht klein, aber ruhig. Kein zuckender, wütender Tanz. Keine Feuerzungen, die sich nach dem 
Himmel richteten. Es spendete sanfte Wärme, sanftes Licht. Semric fiel auf die Knie. Die Kapuze 
bedeckte sein blondes Haar, die unscheinbare Kleidung seine Statur. Tränen füllten seine Augen und 
er zitterte und Erhims Frage kam ihm in den Sinn: \textit{Warum?}\\
''Herr``, keuchte er: ''Bist du bei mir?``\\
Er lauschte verkrampft nach einer Antwort, aber das einzige was er vernahm war das leise Knistern 
des Feuers und seinen pochenden Herzschlag. Semric kniff die Augen zusammen und verbarg sein 
Gesicht 
in seinen kalten Händen.\\
\textit{Osyma… bist du bei mir? Die Priester sagen, sie hören deine Worte… bitte, sprich auch zu 
mir… wenn ich doch von deinem Blut sein soll.}\\
Das Horchen auf eine Antwort blieb ohne Erfolg. Stattdessen spürte er die Leere in sich nur noch 
deutlicher. Semric ließ seinen Tränen nun laufen und ballte seine Hände zu Fäusten. \textit{Gibt es 
dich überhaupt oder bist du nur eine Geschichte der Priester?}\\
Jemand räusperte sich leise, aber Semric reagierte nicht und hoffte, dass der junge Novize einfach 
wieder verschwinden würde. Doch die Stimme klang zu rau um einem jungen Mann zu gehören. Er hielt 
weiterhin still, lauschte aber nun auf die Person. Er hatte nicht viel Hoffnung, dass es doch nur 
jemand war, der ihm priesterlichen Rat anbieten wollte.\\
''Herr``, sagte die Stimme schließlich leise.\\
Semric richtete sich auf. In seinem Gesicht zeichnete sich Erschöpfung und Frustration wieder. 
\textit{Bekomme ich nun doch deine Klinge zu spüren?}\\
''Lerin``, stellte Semric müde fest.\\
''Die Verkleidung ist brauchbar, aber euer Leibwächter hockt vor der Tür und guckt böse drein.``\\
''Er sollte nicht mitkommen.``\\
''Nun… dass muss schlimm für Euch sein, einen so treuen Untergebenen zu haben. Tröstet es Euch, 
dass 
es nur einer ist?``\\
Semric funkelte ihn böse an, erhob sich aber nicht. ''Was wollt Ihr? Mich umbringen? Dann los. 
Macht 
es, bevor Erhim es merkt, sonst könnte er in seiner Treue mich rächen wollen.``\\
Der Offizier seufzte nun doch, trat an den Rand der Nische und ließ sich auf einen der schmalen 
Bänke nieder. ''Ihr seid noch nie in den Tempel gekommen um zu Beten. Nur zu Feierlichkeiten.``\\
''Was wollt Ihr?``, wiederholte Semric und sah ihn eindringlich an.\\
\textit{Herr… dann soll es geschehen. Bitte, sei gnädig mit meiner Seele.}\\
''Ich glaube Osyma ist der falsche Gott für Euch``, sagte Lerin und deutete mit den Kinn auf die 
Flammen.\\
''Es gibt nur einen Gott und das ist der Allmächtige.``\\
Der Offizier zuckte nur mit der Schulter. ''Seid Ihr genesen?``, wechselte er das Thema: ''Es wäre 
langsam mal nötig. Die Priester toben wie von der Leine gelassene Jagdhunde. Kasir spuckt auf das 
Ultimatum - wie zu erwarten war. Sie widerholten ihre Forderung und rekrutieren bereits für ihre 
Armeen. Ihre Schiffe steuern nun einen ihrer kleinen Handelsposten an der Westküste an, statt in 
Na'Rash ihr Proviant aufzufüllen oder zu handeln. Sie handeln noch mit den Kolonien und deren 
freien Nachbarländern, aber wer weiß wie lang. Es wird gemunkelt, dass sie versuchen eine neue 
Rebellion in den Kolonien anzuzetteln um uns den Hintern heiß zu machen. In jeder Predigt feiern 
die Priester den Krieg, fordern beim Adel Geld und Güter für die Armee und vermutlich proben sie 
schon eine neue Siegeshymne. Hisio-Mahar hat drei seiner treusten Anhänger in die Kolonien 
geschickt. Angeblich um Osymas Geist dort zu festigen. Dafür haben sie drei Einheiten mit je 50 
Soldaten mitgenommen. Soldaten der Krone, nicht die privat gezahlten Schläger des Tempels.``\\
''Vielleicht werden sie  müde und dann kann ich das Chaos wieder aufräumen.``\\
Lerin lächelte flüchtig. ''Werdet Ihr es denn aufräumen?``\\
Semric legte seinen Kopf in den Nacken und atmete tief ein. ''Ich trage seit zehn Jahren die Krone 
Saleicas. Und doch hat mir nie jemand gezeigt wie man ein König ist. Ich finde, dadurch habe ich 
etwas Nachsicht verdient.``\\
''Ihr seid ein Herrscher. Die Leute verlassen sich darauf, dass Ihr sie leitet. Und dass Ihr 
das Richtige für sie entscheidet.``\\
''Was ist schon richtig?``, spottete Semric und schüttelte den Kopf. ''Hört zu, wieso kommt Ihr 
jetzt? 
Wieso nicht damals, vor zehn Jahren. Damals, als man es mir hätte beibringen müssen.``\\
Der Offizier schwieg einen Augenblick. ''Ja, das war ein Fehler, das sehe ich ein. Das Problem ist, 
dass es keine direkten Erben gibt und eure Vetter ziehe ich nicht in Betracht. Er steht länger als 
Ihr unter Hisio-Mahars Wort. Vermutlich ist er eine noch stumpfsinnigere Marionette als Ihr.``\\
''Nett, dass Ihr so offen zu mir seid.``\\
''Soll das Sarkasmus sein?``, fragte der Offizier: ''Ich dachte, Ihr fändet es wirklich besser, 
wenn 
ich offen zu Euch sprechen würde.``\\
Semric sah in die Flammen und wich damit den Blick des älteren Mannes aus. Dann nickte er jedoch. 
``Es herrscht Krieg, weil die Priester es wollen. Man kann es natürlich auch wirtschaftlich 
begründen. Oder einfach damit, dass es unserer Kultur entspricht, gerne mal zu erobern. Aber warum 
jetzt? Ich glaube, Hisio-Mahar hat irgendeinen Grund. Wir haben noch nie gegen Kasir gewonnen. 
Warum ist er so davon überzeugt, dass es diesmal anders sein wird?``\\
''Er sagte, Osyma habe zu ihm gesprochen. Der Allmächtige hat es befohlen.``\\
Lerin lachte leise. ''Ja, das ist immer eine gute Ausrede. Also habt Ihr nichts gegen den Krieg?``\\
''Keine Gegenargumente die gut genug wären. In den Kolonien ist es ruhig. Und er hat ja recht… als 
er 
den Krieg verkündete jubelten die Soldaten im ganzen Reich. Sie gieren nach Kampf.``\\
Lerin verzog sein Gesicht zu einer Grimasse und machte eine wegwerfende Handbewegung. ''Das ist die 
Langeweile. So was passiert nun mal wenn man ein so großes, stehendes Heer hat. Die Soldaten haben 
nichts zu tun, außer zu üben. Irgendwann wollen sie es natürlich auch anwenden.``\\
''Töten.``\\
''Für das Land, den Gott und die Ehre``, fügte Lerin hinzu: ''Aber wenn wir es wirklich so weit 
kommen 
lassen – und ich denke, wir haben keine andere Wahl – wird die Meinung schnell umschlagen. Wie 
viele, meint Ihr, waren wirklich schon einmal im Krieg? Nach der Sache in den Kolonien vor ein paar 
Jahren haben viele Soldaten ein Entlassungsgesuch gestellt. Geht hinaus auf die Straßen und fragt 
die Bettler, was sie vor einigen Jahren getan haben. Ich würde darauf wetten, dass jeder vierte in 
den Kolonien für seinen König gekämpft hat und es bereut, so dumm gewesen zu sein, dass alles sogar 
freiwillig getan zu haben.``\\
''Jozah Mi‘Kae bereut es nicht.``\\
''Ihr mögt diesen Mann wirklich, hm?``\\
Semric erhob sich und schüttelte den Kopf. ''Ich kenne ihn nicht. Vielleicht würde ich ihn mögen. 
Was 
tut das zur Sache?``\\
''Ich glaube, Jozah Mi‘Kae ist ein Mann, wie es nur wenige gibt. Ein Mann, der seiner Pflicht 
nachgeht, auch wenn er sich lieber verstecken würde. Jemand, der seine Versprechen hält. So wie 
Euer 
Leibwächter.``\\
Semric überging die letzten Worte des Offiziers und erwiderte: ''Ich hörte, noch jemand würde sehr 
pflichtbewusst handeln. Nur ist diese Person kein Mann.``\\
Lerin wirkte verblüfft. ''Die Priester haben etwas gutes über das Mädchen gesagt?``\\
''Nein, aber mein Käfig hat Fenster.``\\
Mit großen Schritten verließ der König die Nische, gefolgt von der hochaufragenden Gestalt 
des Offiziers. Auf seinem Rückweg durch den Tempel trug den Kopf hoch erhoben und seine 
Kapuze zurück geschlagen. Die Zeit des Versteckens musste ein Ende haben. \textit{Ich bin ihr 
König}, versuchte er sich selbst zu überzeugen: \textit{Ich bin der Löwe Saleicas.}\\
Blicke folgten ihm. Menschen verneigten sich eilig und Wispern erklang.\\
''Ihr scheint sie zu mögen. Also muss sie in Euren Augen etwas gut machen.``
''Ich will das Land schützen'', entgegnete Lerin: ``Einen erneuten Machtwechsel so kurz vor den 
ersten Schlachten kann nur negative Folgen haben. Wir müssen den Menschen eine Person geben, der 
sie vertrauen. An die sie ihre Hoffnungen richten können.``\\
''Ist Mi‘Kae schon dort?``\\
''Ja. Ich warte auf seinen ersten Bericht.``\\
Semric warf einen flüchtigen Blick in seine Richtung. ''Den Ihr mir selbstverständlich sofort 
übergeben werdet.``\\
''Selbstverständlich.``\\
Sie hatten das Eingangsportal bereits durchschritten, als der König sich noch einmal umwandte und 
abrupt stehen blieb. Leise fragte er: ''Und was wisst Ihr über Ilia Ma‘Sah?``\\
Mit gerunzelter Stirn antwortete Lerin: ''Ich kenne den Vater gut. Er war ein ausgezeichneter 
Offizier. Wie ich hatte er einen Platz im Rat eures Vaters. Ich bin mir ziemlich sicher, dass er 
heute an meiner Stelle Euer Berater wäre, wenn er nicht frühzeitig in Rente gegangen wäre.`` Lerin 
grinste flüchtig. ''Er hat die Zeit seines Ruhestandes weiter genutzt um Reichtümer anzuhäufen. 
Früher hatte er viele Verbindungen in die Kolonien. Das ist aber zurück gegangen und er hat sich 
auf die Freunde hier vor Ort konzentriert.``\\
''Und seine Tochter?``\\
Der ältere Mann zögerte kurz. ''Eine kluge, sehr schöne Frau die viel im gesellschaftlichen Leben 
der Oberschicht unterwegs ist.``\\



