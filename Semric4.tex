\chapter{Zwischen Pflichten und Sehnsucht}

Semric rutschte auf dem Tron umher. Egal wie teuer ein Stuhl war, nach einigen Stunden wurde wohl 
jeder unbequem, auch Vergoldete. Dort wo die Krone auf seinem Haupt saß, pochte ein unangenehmer 
Schmerz der, sobald er den Kopf bewegte, stärker zu werden schien. Aber er hielt den Blick gerade 
nach vorne Gerichtet um ja keine Schwäche vor den Priestern zu offenbaren. Hisio-Mahar schien etwas 
vorzuhaben, denn er war mit einem heiteren Lächeln aus dem Saal getreten um persönlich die neusten 
Gäste herein zu geleiten. Normalerweise warteten die Bittsteller Stunden, bis sie an der Reihe 
waren und eben solche Stunden hatte Semric nun auch schon hinter sich gebracht. Zwei Tage die Woche 
verbrachte er damit, wohlgekleideten Bettlern oder wütenden Anklägern zuzuhören.\\
\textit{Danach reicht es!}, entschied Semric genervt. Noch war es früh genug, um den Tag wenigstens 
einigermaßen angenehm ausklingen zu lassen. Seit einigen Tagen hatte er wieder zur Zeichenkohle 
gegriffen. Etwas, was er in den letzten Jahren kaum getan hatte. Nicht mehr, seit der Traum ein 
Froscher zu werden der seine eigenen Bücher illustrierte, unter der Last der Krone erdrückt wurde. 
Erhim hatte ihn auf die Idee gebracht. Semric wollte die Mahig-Na nach seinen Vorstellungen 
zeichnen und hatte in den letzten Tagen sich mit den Kohlestiften wieder vertraut gemacht, dass 
passende Papier gesucht und einfache Zeichnungen angefertigt. Seit dem ersten unerwarteten Besuchs 
des Leibwächters kehrte er nun beinahe jeden Abend ein und die Männer redeten. Er erzählte seinem 
König die Legenden seines Volkes und die Geschichten der Kolonien. Und Erhim hatte nicht 
übertrieben... wenn er zu erzählen begann, war es für Semric, als würde er in seine Worte 
eintauchen. Sogar Riolean schwieg in diesen Stunden.\\
Die Flügeltür öffnete sich schwungvoll und Hisio-Mahar trat mit einem überlegenen Grinsen ein. Es 
war eines der wenigen Mimiken, die nicht von seinen verschlungenen Tattos verzerrt wurde. Semric 
suchte den Grund dieses Grinsens bei den Gestalten hinter dem Hohepriester. Zwischen zwei weiteren 
hochrangingen Priestern lief eine schlanke Gestalt. Ihre Kleidung bestand aus farbenfrohen Stoffen, 
die ihren Körper in einer ihm unbekannten Wickeltechnik umhüllten. Ihre Augen glichen erloschener 
Kohle und ihre Haut schimmerte Bronzen.\\
\textit{Das ist mal viel Gold}, brummte Riolean und Semric blickte unwillkürlich auf den 
klimpernden schmuck. Armbänder, Ohrringe, Ketten um den bloßen Hals und den schlanken Fußknöcheln. 
Selbst in ihr rabenschwarzes Haar waren goldfarbene Fäden geflochten. \textit{Da will sich jemand 
einschleimen.} Riolean kicherte abwertend.\\
Semric ignorierte sie und wartete ruhig ab, bis die Priester mit der Frau näher heran gekommen 
waren. Sie knickste elegant und blickte ihn aus ihren dunklen Augen an. Semric runzelte wenig 
beeindruckt die Stirn. ``Sprich Priester'', forderte er und gönnte Hisio-Mahar nicht einmal einen 
irritierten Blick.\\
Der Hohepriester kniff die Lippen zusammen und säuselte: ``Ich stelle vor, Mihik Sa Elren. Die 
reizende Tochter des Elosas Or Elren, Verwalter des ersten Bezirks der Kolonien und Herr über die 
Städte Esral, Melras und Mihik. Die übrigens nach seinen Töchtern benannt sind.''\\
``Ich weiß wer Elosas Or Elren ist und welche Städte er führt'', kommentierte Semric: ``Wir stehen 
in regelmäßigen Briefkontakt, wie jeder meiner Verwalter in den Kolonien. Warum also steht seine 
Tochter vor mir, ohne dass er mir von diesem Vorhaben erzählte? Und was will sie mir sagen, was ihr 
Vater mir nicht schreiben könnte?''\\
``König Saleicas, Sohn des Osymas, erlaubt mir zu sprechen'', erklang ihre Stimme.\\
Semric hatte noch nicht viele Ausländer seine Sprache sprechen hören, abgesehen von Kasira und 
einer Handvoll Gesandten aus den Kolonien. Ihre Sprachmelodie klang völlig anders als der Akzent 
der Kasira.\\
Er nickte nur als Antwort und ließ Hisio-Mahar nicht aus den Augen. Das Gespräch mit dem Offizier 
im Tempel hatte ihm bewusst gemacht, dass dies nun seine letzte Chance war, sein Land wieder in den 
Griff zu bekommen. Er durfte den Priestern nicht mehr so viele Entscheidungen überlassen. Die 
Entscheidung, sich Hisio-Mahar entgegen zu stellen war ein Anfang, auch wenn es alles andere als 
leicht für Semric war. Er suchte nach der Hinterlist in den Augen des Priesters. Natürlich hatte er 
das eingefädelt und hinter seinen Rücken organisiert. Die Frage war, wieso er diese Gesandte ihm 
nun doch auf einem offiziellen Weg vorstellte, wenn er bis an diesen Punkt alles hinter seinem 
Rücken eingefädelt hatte.\\
``Der Rat, der sich aus den Bezirken eurer Kolonien zusammen setzt, kam zu den Entschluss, dass es 
eurem gesamtem Reich nur zu Gute kommen kann, wenn Ihr eine Vertreterin der Kolonien an Eurer Seite 
habt. Das würde den Worten der Rebellen ihre Macht nehmen. Mein Volk würde sich nicht mehr wie 
Sklaven fühlen, sondern als gleichberechtigte Bürger Saleicas. Die Kolonien sollen nicht mehr nur 
Kolonien sein, sondern Grafschaften und die Vertreter euren Segen als vor Osyma geweihte Grafen und 
Gräfinnen erhalten.''\\
\textit{Sowie die finanziellen Mittel, die damit bisher verbunden waren,} hüstelte Riolean.\\
Semrics Stirn legte sich tiefer in Falten. ``Wieso sollte ich die Kolonien einbürgern? Damit ist 
ein größerer Aufwand für mich verbunden. Auch finanziell gesehen. Abgesehen davon, dass ihr eure 
Religionen abschwören müsstet. Und das wiederum ist überhaupt nicht förderlich, wenn man 
Rebellionen verhindern will.''\\
``Man erzählte mir, dass Saleica einst ein kleines Land war. Seine jetzige Größe auf diesem 
Kontinent hat es nur erhalten, weil es andere Länder erobert habt. So wie die Länder meiner Heimat. 
Die Grafschaften waren auch einst Kolonien? Wieso wurden sie eingegliedert, aber meine Heimat 
nicht?''\\
Bisher hatte keiner der Bezirksverwalter dieses Thema angesprochen. Im Gegenteil, ein jeder von 
ihnen war bisher stets unterwürfig erschienen, aus dem schlichten Grund der Dankbarkeit, dass 
Saleica ihnen ihre Macht und ihr Geld gelassen hatte.\\
``Sie waren keine Kolonien'', entgegnete Semric bedächtig: ``Es gab auch keine Rebellionen. Und die 
Kulturen waren nicht so verschieden wie sie es in deiner Heimat sind. Außerdem geschah es alles mit 
gewissen zeitlichen Abständen. Die Grafschaft Merandila ist erst seit etwas mehr als Hundert Jahren 
ein Teil Saleicas.''\\
``Es muss ja auch nicht so schnell geschehn. Solche Angelegenheiten regeln sich über Jahrzehnte'', 
warf Hisio-Mahar ein: ``Aber mit Verlaubt, mein König, Ihr scheint einen wichtigen Teil ihrer Worte 
überhöhrt zu haben.''\\
Semric beugte sich vor und unterbrach den Priester. ``Was genau meint Ihr, Hohepriester? Dass 
Elosas Or Elren mir seine Tochter verkauft?''\\
Die Miene der schönen Frau versteinerte. ``Es gibt keine Sklaven in Saleica und auch nicht in 
dessen Kolonien.''\\
``Und doch spracht Ihr, dass dein Volk sich so fühlt.'', warf Semric ein: ``Aber gut, nach Eurem 
Protest zu urteilen, geschätzte Mihiki Sa Elren, seid Ihr also nicht nur durch die Worte Eures 
Vaters, der anderen Bezirksvorsitzenden oder meines Hohepriesters hier vor mir getreten um einen 
Heiratsantrag zu machen? Darf ich dann davon ausgehen, dass mein Charm Euch verführt hat? Meine 
Witze, die Euch ein Lachen entlocken und die vertrauten Gespräche?''\\
Alle Anwesenden im Saal starrten ihren König überrascht an. Noch keiner hatte je so sprechen hören, 
außer Erhim. Aber auch der Leibwächter sah einen flüchtigen Moment irritiert aus, ehe er zu grinsen 
begann und sich zurück halten musste, seinem Herrn nicht anerkennend auf die Schulter zu klopfen. 
Semric hatte den Plänen des Hohepriesters nicht nur widersprochen, sondern das sogar vor Zeugen 
getan.\\
``Ihr habt natürlich recht, Herr'', säuselte Hisio-Mahar mit zunehmender Schärfe in der Stimme: 
``Aber auch das muss nicht von Heute auf Morgen geschehen. Bevor Ihr die Kolonien beleidigt, 
solltet Ihr deren vielleicht zukünftigte Vertreterin kennenlernen. Dann könnt Ihr Euch 
entscheiden.''\\
``Natürlich'', stimmte Semric zu: ``Bis diese Entscheidung getroffen ist, könnt Ihr unserem Gast zu 
einigen Predigten mitnehmen. Damit sie auch weiß, was auf sie zukommt, wenn sie ihrem Glauben 
abschwört und ihr Leben dem allmächtigen Osyma widmet. Ein Gemach soll für Euch gerichtet werden, 
Mihik Sa Elren, und ich werde dann bald auf Euch zukommen. Wenn Ihr irgendwelche Wünsche habt, 
richtet sie ohne Zögern an die Bediensteten. Ich wünsche noch einen angenehmen Abend.''\\
Semric erhob sich, nickte den Anwesenden grüßen zu und verließ den Saal durch eine Seitentür, 
welche den kürzesten Weg zu seinen eigenen Zimmern verbarg. Erst als die schwere Holztür hinter ihm 
ins Schloss gefallen war, erlaubte er sich, nach Luft zu ringen.\\
``Das war gut'', bemerkt Erhim aufmunternd: ``Hättet Ihr Euch zu einer baldigen Hochzeit 
überrumpeln lassen, hätte ich es mir mit meinem Eid nochmal überlegt.''\\
``Du hälst also nichts von der Idee? Einige Argumente sind recht schlüssig, wenn man darüber 
nachdenkt.''\\
Erhim schüttelte den Kopf. ``Es tut nichts zu Sache, ob es sinnvoll wäre oder nicht. Wichtig ist, 
dass der Priester es will, weil es seinen Plänen zugute kommt. Er hat die Sache ausgehekt und ich 
würde vorschlagen, zu allem nein zu sagen, noch ehe er es vollständig ausgesprochen hat.''\\
Semric biss sich auf die Zunge und betrachtete seinen Leibwächter nachdenklich. Das war die 
Einstellung eines Kindes und genau das wollte er doch nun nicht mehr sein. Er beschloss Erhims 
Kommentar nicht zu erwidern.\\
``Aber... fals ich das sagen darf, habt Ihr ein eine Frau gedacht, während Ihr ihr die Abfuhr 
erteilt habt?''\\
Semric kniff die Augen zusammen. ``Die Frage muss ich dir nicht beantworten.''\\
``Wieso sind wir dann noch nicht auf dem Weg? Wenn nicht jetzt, wann dann? Es ist Wochen her, seit 
Ihr sie gesehen habt. Vielleicht irrt Ihr Euch und sie hat Euch gar nicht so verzaubert? Dann könnt 
Ihr auch dieser Mihiki Aufmerksamkeit schenken.''\\
``Ich werde mir dir nicht über den Zauber von Frauen sprechen'', rief Semric empört aus: ``Und 
außerdem haben es ehrbare Damen nicht gern, wenn man unangekündigt vor ihrer Tür steht.''\\
Erhim zuckte mit der Schulter. ``Da würde ich mir keine Sorgen machen. Was ich bisher  von diesem 
Land mitbekommen habe, lassen ehrbare Frauen auch keine durchnässten Fremden zu später Stunde in 
ihr Haus.''\\
