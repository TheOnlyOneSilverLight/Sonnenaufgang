\chapter{Zwischen Pflichten und Sehnsucht}

Semric rutschte auf dem Thron umher. Egal wie teuer ein Stuhl war, nach einigen Stunden wurde wohl 
jeder unbequem, auch Vergoldete. Dort wo die Krone auf seinem Haupt saß, pochte ein unangenehmer 
Schmerz der, sobald er den Kopf bewegte, stärker zu werden schien. Aber er hielt den Blick gerade 
nach vorne gerichtet um ja keine Schwäche vor den Priestern zu offenbaren. Hisio-Mahar schien etwas 
vorzuhaben, denn er war mit einem heiteren Lächeln aus dem Saal getreten um persönlich die neusten 
Gäste herein zu geleiten. Normalerweise warteten die Bittsteller Stunden, bis sie an der Reihe 
waren und eben solche Stunden hatte Semric nun auch schon hinter sich gebracht. Zwei Tage die Woche 
verbrachte er damit, wohlgekleideten Bettlern oder wütenden Anklägern zuzuhören.\\
\textit{Danach reicht es!}, entschied Semric genervt. Noch war es früh genug, um den Tag wenigstens 
einigermaßen angenehm ausklingen zu lassen. Seit einigen Tagen hatte er wieder zur Zeichenkohle 
gegriffen. Etwas, was er in den letzten Jahren kaum getan hatte. Nicht mehr, seit der Traum ein 
Forscher zu werden der seine eigenen Bücher illustrierte, unter der Last der Krone erdrückt wurde. 
Erhim hatte ihn auf die Idee gebracht. Semric wollte die Mahig-Na nach seinen Vorstellungen 
zeichnen und hatte in den letzten Tagen sich mit den Kohlestiften wieder vertraut gemacht, dass 
passende Papier gesucht und einfache Zeichnungen angefertigt. Seit dem ersten unerwarteten Besuchs 
des Leibwächters kehrte er nun beinahe jeden Abend ein und die Männer redeten. Er erzählte seinem 
König die Legenden seines Volkes und die Geschichten der Kolonien. Und Erhim hatte nicht 
übertrieben... wenn er zu erzählen begann, war es für Semric, als würde er in seine Worte 
eintauchen. Sogar Riolean schwieg in diesen Stunden.\\
Die Flügeltür öffnete sich schwungvoll und Hisio-Mahar trat mit einem überlegenen Grinsen ein. Es 
war eines der wenigen Mimiken, die nicht von seinen verschlungenen Tattoos verzerrt wurde. Semric 
suchte den Grund dieses Grinsen bei den Gestalten hinter dem Hohepriester. Zwischen zwei weiteren 
hochrangingen Priestern lief eine schlanke Gestalt. Ihre Kleidung bestand aus farbenfrohen Stoffen, 
die ihren Körper in einer ihm unbekannten Wickeltechnik umhüllten. Ihre Augen glichen erloschener 
Kohle und ihre Haut schimmerte wie Bronzen.\\
\textit{Das ist mal viel Gold}, brummte Riolean und Semric blickte unwillkürlich auf den 
klimpernden Schmuck. Armbänder, Ohrringe, Ketten um den bloßen Hals und den schlanken Fußknöcheln. 
Selbst in ihr rabenschwarzes Haar waren goldfarbene Fäden geflochten. \textit{Da will sich jemand 
einschleimen.} Riolean kicherte abwertend.\\
Semric ignorierte sie und wartete ruhig ab, bis die Priester mit der Frau näher heran gekommen 
waren. Sie knickste elegant und blickte ihn aus ihren dunklen Augen an. Semric runzelte wenig 
beeindruckt die Stirn. ``Sprich Priester'', forderte er und gönnte Hisio-Mahar nicht einmal einen 
irritierten Blick.\\
Der Hohepriester kniff die Lippen zusammen und säuselte: ``Ich stelle vor, Mihiki Sa Elren. Die 
reizende Tochter des Elosas Or Elren, Verwalter des ersten Bezirks der Kolonien und Herr über die 
Städte Esral, Melras und Mihik. Die übrigens nach seinen Töchtern benannt sind.''\\
``Ich weiß wer Elosas Or Elren ist und welche Städte er führt'', kommentierte Semric: ``Wir stehen 
in regelmäßigen Briefkontakt, wie jeder meiner Verwalter in den Kolonien. Warum also steht seine 
Tochter vor mir, ohne dass er mir von diesem Vorhaben erzählte? Und was will sie mir sagen, was ihr 
Vater mir nicht schreiben könnte?''\\
``König Saleicas, Sohn des Osymas, erlaubt mir zu sprechen'', erklang ihre Stimme.\\
Semric hatte noch nicht viele Ausländer seine Sprache sprechen hören, abgesehen von Kasira und 
einer Handvoll Gesandten aus den Kolonien. Ihre Sprachmelodie klang völlig anders als der Akzent 
der Kasira.\\
Er nickte nur als Antwort und ließ Hisio-Mahar nicht aus den Augen. Das Gespräch mit dem Offizier 
im Tempel hatte ihm bewusst gemacht, dass dies nun seine letzte Chance war, sein Land wieder in den 
Griff zu bekommen. Er durfte den Priestern nicht mehr so viele Entscheidungen überlassen. Die 
Entscheidung, sich Hisio-Mahar entgegen zu stellen war ein Anfang, auch wenn es alles andere als 
leicht für Semric war. Er suchte nach der Hinterlist in den Augen des Priesters. Natürlich hatte er 
das eingefädelt und hinter seinen Rücken organisiert. Die Frage war, wieso er diese Gesandte ihm 
nun doch auf einem offiziellen Weg vorstellte, wenn er bis an diesen Punkt alles hinter seinem 
Rücken eingefädelt hatte.\\
``Der Rat, der sich aus den Bezirken eurer Kolonien zusammen setzt, kam zu den Entschluss, dass es 
eurem gesamtem Reich nur zu Gute kommen kann, wenn Ihr eine Vertreterin der Kolonien an Eurer Seite 
habt. Das würde den Worten der Rebellen ihre Macht nehmen. Mein Volk würde sich nicht mehr wie 
Sklaven fühlen, sondern als gleichberechtigte Bürger Saleicas. Die Kolonien sollen nicht mehr nur 
Kolonien sein, sondern Grafschaften und die Vertreter euren Segen als vor Osyma geweihte Grafen und 
Gräfinnen erhalten.''\\
\textit{Sowie die finanziellen Mittel, die damit bisher verbunden waren,} hüstelte Riolean.\\
Semrics Stirn legte sich tiefer in Falten. ``Wieso sollte ich die Kolonien einbürgern? Damit ist 
ein größerer Aufwand für mich verbunden. Auch finanziell gesehen. Abgesehen davon, dass ihr eure 
Religionen abschwören müsstet. Und das wiederum ist überhaupt nicht förderlich, wenn man 
Rebellionen verhindern will.''\\
``Man erzählte mir, dass Saleica einst ein kleines Land war. Seine jetzige Größe auf diesem 
Kontinent hat es nur erhalten, weil es andere Länder erobert habt. So wie die Länder meiner Heimat. 
Die Grafschaften waren auch einst Kolonien? Wieso wurden sie eingegliedert, aber meine Heimat 
nicht?''\\
Bisher hatte keiner der Bezirksverwalter dieses Thema angesprochen. Im Gegenteil, ein jeder von 
ihnen war bisher stets unterwürfig erschienen, aus dem schlichten Grund der Dankbarkeit, dass 
Saleica ihnen ihre Macht und ihr Geld gelassen hatte.\\
``Sie waren keine Kolonien'', entgegnete Semric bedächtig: ``Es gab auch keine Rebellionen. Und die 
Kulturen waren nicht so verschieden wie sie es in deiner Heimat sind. Außerdem geschah es alles mit 
gewissen zeitlichen Abständen. Die Grafschaft Merandila ist erst seit etwas mehr als Hundert Jahren 
ein Teil Saleicas.''\\
``Es muss ja auch nicht so schnell geschehen. Solche Angelegenheiten regeln sich über Jahrzehnte'', 
warf Hisio-Mahar ein: ``Aber mit Verlaub, mein König, Ihr scheint einen wichtigen Teil ihrer Worte 
überhört zu haben.''\\
Semric beugte sich vor und unterbrach den Priester. ``Was genau meint Ihr, Hohepriester? Dass 
Elosas Or Elren mir seine Tochter verkauft?''\\
Die Miene der schönen Frau versteinerte. ``Es gibt keine Sklaven in Saleica und auch nicht in 
dessen Kolonien.''\\
``Und doch spracht Ihr, dass dein Volk sich so fühlt.'', warf Semric ein: ``Aber gut, nach Eurem 
Protest zu urteilen, geschätzte Mihiki Sa Elren, seid Ihr also nicht nur durch die Worte Eures 
Vaters, der anderen Bezirksvorsitzenden oder meines Hohepriesters hier vor mir getreten um einen 
Heiratsantrag zu machen? Darf ich dann davon ausgehen, dass mein Charme Euch verführt hat? Meine 
Witze, die Euch ein Lachen entlocken und die vertrauten Gespräche?''\\
Alle Anwesenden im Saal starrten ihren König überrascht an. Noch keiner hatte ihn je so sprechen 
hören, außer Erhim. Aber auch der Leibwächter sah einen flüchtigen Moment irritiert aus, ehe er zu 
grinsen begann und sich zurück halten musste, seinem Herrn nicht anerkennend auf die Schulter zu 
klopfen. Semric hatte den Plänen des Hohepriesters nicht nur widersprochen, sondern das sogar vor 
Zeugen getan.\\
``Ihr habt natürlich recht, Herr'', säuselte Hisio-Mahar mit zunehmender Schärfe in der Stimme: 
``Aber auch das muss nicht von Heute auf Morgen geschehen. Bevor Ihr die Kolonien beleidigt, 
solltet Ihr deren vielleicht zukünftige Vertreterin kennenlernen. Dann könnt Ihr Euch 
entscheiden.''\\
``Natürlich'', stimmte Semric zu: ``Bis diese Entscheidung getroffen ist, könnt Ihr unserem Gast zu 
einigen Predigten mitnehmen. Damit sie auch weiß, was auf sie zukommt, wenn sie ihrem Glauben 
abschwört und ihr Leben dem allmächtigen Osyma widmet. Ein Gemach soll für Euch gerichtet werden, 
Mihiki Sa Elren, und ich werde dann bald auf Euch zukommen. Selbstverständlich wird es auch einen 
gebührenden Empfang geben. Ein Ball... genau. Brom-Dalar braucht mal wieder einen Grund zu tanzen 
und welcher wäre passender als der Empfang einer geschätzten Vertreterin der Kolonien? Wenn Ihr 
irgendwelche Wünsche habt, richtet sie ohne Zögern an die Bediensteten. Ich wünsche noch einen 
angenehmen Abend.''\\
Semric erhob sich, nickte den Anwesenden grüßen zu und verließ den Saal durch eine Seitentür, 
welche den kürzesten Weg zu seinen eigenen Zimmern verbarg. Erst als die schwere Holztür hinter ihm 
ins Schloss gefallen war, erlaubte er sich, nach Luft zu ringen.\\
``Das war gut'', bemerkt Erhim aufmunternd: ``Hättet Ihr Euch zu einer baldigen Hochzeit 
überrumpeln lassen, hätte ich es mir mit meinem Eid nochmal überlegt. Und der Empfang tröstet 
sie vielleicht erst mal über das Nein hinweg... was ich bisher mitbekommen habe, lieben Frauen 
die Gelegenheit sich herauszuputzen. Jede Frau. Egal aus welchem Land. Und wenn sich meine 
Schwestern schon bei einem Lagerfeuer in der Dorfmitte den Kopf zerbrachen, was sie anziehen und 
mit wem sie tanzen, wie schlimm wird dass dann erst bei einem Ball im Palast des saleicanischen 
Königs?''\\
Semric stellte sich die mit Schmuck behangene Mihiki in einem Dorf aus schiefen Lehmhäusern vor, 
wie sie springend um ein Lagerfeuer tanzte und lachte schallend los. Immer noch grinsend sagte er 
dann jedoch: ``Du hältst also nichts von der Hochzeit? Einige Argumente sind recht schlüssig, wenn 
man darüber nachdenkt.''\\
Erhim schüttelte den Kopf. ``Es tut nichts zu Sache, ob es sinnvoll wäre oder nicht. Wichtig ist, 
dass der Priester es will, weil es seinen Plänen zugute kommt. Er hat die Sache ausgeheckt und ich 
würde vorschlagen, zu allem nein zu sagen, noch ehe er es vollständig ausgesprochen hat.''\\
Semric biss sich auf die Zunge und betrachtete seinen Leibwächter nachdenklich. Das war die 
Einstellung eines Kindes und genau das wollte er doch nun nicht mehr sein. Er beschloss Erhims 
Kommentar nicht zu erwidern.\\
``Aber... falls ich das sagen darf, habt Ihr an eine Frau gedacht, während Ihr ihr die Abfuhr 
erteilt habt?''\\
Semric kniff die Augen zusammen. ``Die Frage muss ich dir nicht beantworten.''\\
``Wieso sind wir dann noch nicht auf dem Weg? Wenn nicht jetzt, wann dann? Es Wochen her, seit 
Ihr sie gesehen habt. Vielleicht irrt Ihr Euch und sie hat Euch gar nicht so verzaubert? Dann könnt 
Ihr auch dieser Mihiki Aufmerksamkeit schenken.''\\
``Ich werde mir dir nicht über den Zauber von Frauen sprechen'', rief Semric empört aus: ``Und 
außerdem haben es ehrbare Damen nicht gern, wenn man unangekündigt vor ihrer Tür steht.''\\
Erhim zuckte mit der Schulter. ``Da würde ich mir keine Sorgen machen. Was ich bisher  von diesem 
Land mitbekommen habe, lassen ehrbare Frauen auch keine durchnässten Fremden zu später Stunde in 
ihr Haus.''\\

Ilia Ma'Sah saß in Gedanken verloren auf dem Bärenpelz und starrte in die Flammen des Kaminfeuers. 
Ihr Vater war wie die letzten Tage auch bei alten Militärkameraden unterwegs um mit ihnen über den 
nahenden Krieg mit Kasir zu philosophieren. Er war regelrecht aufgeblüht, als es verkündet wurde. 
Krieg! Dafür schlug das Herz ihres Vaters. Fast schon beeindruckt sah sie zu, wie er sogar seine 
Angewohnheit, bereits früh am Morgen seinen Tag mit einer Karaffe Wein zu starten, abgelegt hatte.\\
\textit{Ob er sich wohl einbildet noch ein Mal in eine Schlacht ziehen zu dürfen? In seinem 
Alter?}\\
Ilia strich sich nachdenklich die Feder immer wieder sanft über Wange und Lippen, während sie 
versuchte die richtigen Worte zu finden. Worte, an ihren Verlobten oben im Norden. Zum Glück war 
ihr Vater nicht da, er hätte sie schon längst gemaßregelt, dass sie die Feder gefälligst aus ihrem 
Gesicht nehmen soll. Dabei fühlte es sich so angenehm auf der Haut an. Wie die zarte Berührung 
eines Liebenden. Und solche Gefühle sollte sie doch wohl empfinden, wenn sie einen Brief an ihren 
Verlobten schreibt. Ilia bemühte sich so viel Ehrlichkeit und Liebe in ihre Worte zu füllen, wie 
ihr Herz nur konnte. Es war die dritte Fassung des Briefs, denn sie wagte nicht die Worte sich laut 
vorzusagen um eine passende Formulierung zu finden. Auch wenn sie alleine war, bis auf die Wache 
und den Knecht im Stall. Hier im Haus am Meer war es zu trostlos für Ilia. Sie sehnte sich nach dem 
Lärm der Stadt. Sie würde bei nächster Gelegenheit mit ihrem Vater sprechen. Er musste sie doch 
irgendwann wieder zurück in das Stadtanwesen lassen!\\
``Herrin'', räusperte sich die Wache, die plötzlich im Raum stand.\\
Ilia blickte ungeduldig auf und wedelte mit der Schreibfeder: ``Was ist?''\\
``Zwei Reiter wollen Euch besuchen.''\\
``Dann wimmle sie ab. Ich habe zu tun.''\\
Der Mann zupfte nervös an seiner zweckmäßigen Kleidung. ``Ähm... mir wäre es lieber, wenn Ihr das 
selbst tun würdet.''\\
``Wieso? Meinst du, potentielle Entführer lassen sich lieber von einer zarten Schönheit wegschicken 
als von einer bewaffneten, gut ausgebildeten - und wenn ich das betonen darf - gut bezahlten 
Wache?''\\
Der Mann schwieg, rührte sich aber nicht von der Stelle. Ilia runzelte die Stirn und schüttelte 
ungläubig den Kopf. Aber sie erhob sich, strich das Kleid glatt und trat zur Tür. Sie war noch 
geöffnet und die Besucher warteten höflich an der Schwelle. Ilia warf einen Blick über die Schulter 
des Mannes, der nur wenig größer war als sie selbst. Der Knecht hielt noch die Zügel der Pferde in 
der Hand und wartete ab, ob die Herren wieder fortgeschickt wurden oder er die Tiere in den Stall 
bringen sollte. Erst dann besah die Adelige ihren Besuch genauer. Mit einem ehrlich überraschtem 
Lächeln begegnete sie dem Blick des Schreibers Melior und knickste. ``Welch Überraschung'', sagte 
sie: ``Das kann nur Osyma gewesen sein, der Euch zu mir schicktet, werter Melior. Ein Schreiber ist 
genau das, was ich gerade brauche!''\\
Die Wache starrte die Besucher skeptisch an, hielt aber den Mund. Dafür wurde er immerhin äußerst 
reichlich bezahlt. Melior tippte sich höflich an den Hut und rang sich ein Lächeln ab. ``Oh, das 
passt ja gut...''\\
Ilia griff ihn am Ärmel und zog den Schreiber hinter sich her in den Salon. Gemütlich, aber etwas 
schicklicher als vorher noch, ließ sie sich wieder auf dem Boden nieder und ergriff das Pergament. 
``Ich würde Euch und Eurem Freund ja Tee anbieten, aber die Bedienstete hat heute ihren freien 
Tag.''\\
``Und Ihr könnt nicht kochen?'' fragte Meliors Begleiter.\\
Ilia hatte ihn bisher noch nicht sprechen hören, aber auch ihm schenkte sie ein charmantes Lächeln, 
während sie erwiderte: ``Doch. Aber ich will nicht.''\\
Die Männer sahen sie irritiert an. Sie klopfte auf das Fell neben sich. ``Außer Ihr seid am 
verdursten, Melior. Ansonsten wäre es wirklich nett, wenn Ihr mir bei diesem Brief helfen würde.''\\
Melior nahm den Hut ab und begann laut zu lachen. Die Dame legte den Kopf schief und betrachtete 
seine blonden Haarsträhnen und wie sein Anblick sich veränderte, während er so herzlich lachte. Es 
sah schön aus. Fast bereute sie, dass dieser Moment so schnell verstrich.\\
``Ihr seit eine unhöfliche Frau, Herrin'', lachte der Schreiber.\\
Ilia tippte sich mit der Feder an die Lippen. ``Nein. Ich bin nur ehrlich. Ehrlichkeit gibt es 
unter den Leuten meines Standes kaum noch, müsst ihr wissen. Als ob altes Blut vorraussetzt, dass 
jedes Wort welches man spricht, eine Lüge ist. Außerdem gibt es einen tieferen Grund... Freund des 
Schreibers, dessen Name mir noch nicht gesagt wurde. Als Kind traf mich kochendes Wasser und 
hinterließ seine Spuren. Deshalb hat selbst mein Verlobter noch nie meine linke Wade gesehen!''\\
Melior ließ sich auf das Fell nieder und nickte seinem Begleiter auffordernd zu, aber er hielt sich 
im Hintergrund und lehnte sich nur lässig an eine Kommode aus kasirischem Bergholz. Südhaft teuer, 
ein Geschenk zu ihrem fünfzehnten Geburtstag. Aber hässlich, deshalb hatte Ilia es nicht in das 
Stadtanwesen bringen lassen.\\
``Also los, Melior. Als Schreiber fällt Euch bestimmt gleich ein reizendes Gedicht über die 
unsterbliche Liebe zu meinem Verlobten ein. Los, zeigt was Ihr könnt!'', forderte Ilia kichernd auf 
und tränkte die Feder in Tinte, bereit, sofort los zu schreiben. Der blonde Schreiber jedoch sah 
etwas hilflos drein und blickte, ebenso wie sie noch vor wenigen Augenblicken, grübelnd in die 
Flammen. Ilia seufzte und senkte die Feder. ``Ich weiß. Ich grüble auch schon seit Stunden. 
Zumindest fühlt es sich an wie Stunden... man meint immer, dass so etwas einem doch leicht fallen 
müsse. Aber es ist schrecklich schwer Gefühle in Worte zu fassen!'' \\
``Weil es Gefühle schon vor den Worten gab'', murmelte Melior zu den Flammen.\\
``So eine Aussage hätte ich von einem Schreiber nicht erwartet! Kannst du mir nun helfen?''\\
Melior zog die Schultern hoch. ``Ehrlich gesagt, ich bin besser im Zeichnen als im Schreiben. 
Also... selbst schreiben, meine ich. Mir wird meistens diktiert...''\\
Ilia schüttelte den Kopf. ``Ich glaube, Ihr werdet in Eurem Beruf nicht sehr erfolgreich sein... 
wenn es dazu kommt, dass Ihr Hunger leidet... sprecht mich an, dann lasse ich für Euch kochen. Aber 
zeichnen? Dann zeichne mir etwas.''\\
Ilia sprang schwungvoll auf die Beine und Melior griff gerade noch nach der Feder um Tintenflecke 
zu vermeiden. ``Irgendwo hatte ich noch Kohle herumliegen... wartet, ich suche sie schnell! 
Pergament ist in der Kommode hinter Euch, Begleiter ohne Namen! Wenn Ihr so freundlich wärt.''\\
``Ein Porträt?'', fragte der Schreiber etwas überrumpelt.\\
Ilia reichte ihm die Kohlereste und überlegte einen Moment. \textit{Na... wieso nicht...}, dachte 
sie spontan und öffnete mit wenigen Griffen ihr Gewandt. Sie war unheimlich neugierig, wie ihr Gast 
wohl reagieren würde. ``Es ist immerhin für meinen Verlobten'', fügte sie hinzu und ließ sich auf 
dem Bärenfell nieder. Während die beiden Männer sie ungläubig anstarrten, probte Ilia einige 
verschiedene Posen und entschied sich schließlich dafür, auf dem Rücken liegen zu bleiben. Ihr 
blondes Haar breitete sich wie ein Fächer um sie herum und das Melior zugewandte Bein winkelte sie 
aufrecht an. ``Passt das so? Der Künstler muss schon etwas dazu sagen!''\\
``Äh...'', stammelte er.\\
``Joa, dass passt so schon'', warf sein Begleiter grinsend ein.\\
Ilia winkte ungeduldig mit der Hand. ``Dann los.''\\
Melior überlegte kurz und setzte sich dann in den Sessel um eine andere Perspektive auf die Frau zu 
haben. Und dann war es lange Zeit still. Ilia schloss die Augen und spürte die Wärme des 
Kaminfeuers neben sich. Sie lauschte auf die Geräusche der über das Papier kratzenden Kohle und 
versuchte sich auszumalen, wie ihr Abbild entstand. Irgendwann räusperte der Zeichner sich und 
murmelte leise ``Es ist fertig...''\\
Ilia erhob sich in einer geschmeidigen Bewegung, schlüpfte schnell in ihr Gewandt und trat auf 
Melior zu. Er reichte ihr das Papier und sah sie nervös an.\\
``Danke'', sagte sie und hielt seine Hand einen Moment länger als nötig.\\
``Ich glaube, wir sollten langsam heimkehren'', räusperte sich der namenlose Begleiter.\\
Der Schreiber nickte nur wortlos und Ilia tat es ihm gleich. ``Ich hoffe doch, wir sehen uns bald 
wieder'', fügte sie hinzu, als sie die Herren zur Tür begleitete. Erst als sie wieder alleine war, 
betrachtete sie die Zeichnung. Es war als wäre kein Strich zu viel und doch hauchten die zarten 
Schattierungen dem Bild Leben ein. Ihr Haar breitete sich wie ein Fächer um ihr Gesicht. Dann fiel 
ihr ihre Wade ins Auge und sie betrachtete, wie der Künstler mit feinen Strichen die Narben 
festgehalten hatte. Ihre Lippen wurden zu einem schmalen Strich. Ilia war schon oft Modell 
gestanden und nie hatte es ein Künstler gewagt diesen Makel fest zu halten. Sie hätte es Melior 
nicht zu getraut. Besonders nicht, nachdem sie ihm auch noch davon erzählte. Noch nie hatte sie ein 
so ehrliches Bild von sich selbst gesehen. Noch nie war sie jemandem begegnet, der zeigte, dass er 
sie sah wie sie war. Nachdenklich schlenderte sie zum Bärenfell zurück und nahm den angefangenen 
Brief wieder auf. Sie tunkte die Schreibfeder in frischer Tinte und begann zu schreiben:

\textit{Unsere Begegnungen erfüllten mich, Jozah. Du gabst mir ein Gefühl, welches mir noch keiner 
gab. Ich trage einen alten Namen. Einen der Ältesten unseres Reichs. Ich besitze Geld. Ich bin die 
einzige Erbin meines Vaters, einen altgedienten und ehrbaren Offiziers im Ruhestand. Ich bin 
kostbar. Und du hast mich so behandelt. Jeder meiner vorherigen Verehrer tat das. Aber jeder von 
ihnen gehörte dem Adel an und jeder von ihnen war mit Kostbarkeiten aufgewachsen. Sie wussten wie 
man mit ihnen umgeht und sie wussten, dass man Schätze ersetzen kann, wenn sie zerbrechen. Du 
nicht. Ich weiß, dass wenn du mich Kostbarkeit verlierst, es dir dein Herz brechen wird. Dafür 
liebe ich dich. Deshalb habe ich der Verlobung zugestimmt. Ich will jemanden, der mich nicht 
ersetzen kann.\\}
\textit{Aber er ist der König. Es tut mir leid.}\\

\textit{Osymas Segen sei mit dir,}
\textit{Ilia Ma'Sah}\\

Sie legte die Feder weg und las noch einmal die Zeilen. Nachdem die Tinte getrocknet war, faltete 
sie das Papier zusammen und versteckte es in ihrem Zimmer. Noch war nicht die Zeit, diesen Brief 
abzuschicken. Erst musste sie sich sicher sein, dass das Herz des Königs ihr gehörte.
