\chapter{Zwischen Pflichten und Sehnsucht}

Semric rutschte auf dem Thron umher. Egal wie teuer ein Stuhl war, nach einigen Stunden wurde wohl 
jeder unbequem, auch Vergoldete. Dort wo die Krone auf seinem Haupt saß, pochte ein unangenehmer 
Schmerz. Aber er hielt den Blick gerade nach vorn gerichtet um ja keine Schwäche vor den Priestern 
zu offenbaren. Seit er die Drohung zum ersten Mal ausgesprochen hatte, hatte sie für Semric ihr 
Unheil verloren. Riolean hatte schon recht, als sie in seinem Kopf grimmig geknurrt hatte. 
Hisio-Mahar hatte ihn bedroht. Ihn, seine Krone und sein Geburtsrecht. Und Ilia.\\
Noch wusste niemand davon und Semric bemühte sich, dass es so blieb. Er hatte Erhim nichts gesagt 
und sich von Ilia fern gehalten. Seit dem ersten unerwarteten Besuchs des Leibwächters kehrte er nun 
beinahe jeden Abend ein und die Männer redeten. Er erzählte dem König die Legenden seines Volkes 
und die Geschichten der Kolonien. Und Erhim hatte nicht übertrieben... wenn er zu erzählen begann, 
war es für Semric, als würde er in seine Worte eintauchen. Sogar Riolean schwieg in diesen 
Stunden.\\
Ansonsten hatte er sich bemüht, den Priestern aus den Weg zu gehen und seine Pflichten zu erfüllen. 
\textit{Was macht einen guten König aus?}, fragte er Riolean, während er dem letzten Bittsteller 
hinterher sah.\\
Die ganzen letzten Stunden schon versuchte er möglichst gerecht und zuvorkommend zu sein. Aber 
meistens jammerten die Leute einfach weiter wie kleine Kinder. Er hatte sich mit Generälen und 
einfachen Soldaten, mit Grafen und Handwerkern getroffen. \textit{Präsenz,} war Rioleans Antwort: 
\textit{Tatkraft. Stolz.}\\
Semric wartete darauf, dass die Tür sich erneut öffnete. Hisio-Mahar hatte bereits verraten, wer 
der letzte Gast sein würde. Ganz zufällig wohnten diesem Augenblick auch das Grafenpaar Ringens und 
Kantors bei. Ebenso wie Offizier Lerim und der Stadtrat.\\
Die Flügeltür öffnete sich schwungvoll und Hisio-Mahar trat mit einem überlegenen Grinsen ein. Es 
war eines der wenigen Mimiken, die nicht von seinen verschlungenen Tattoos verzerrt wurde. Semric 
suchte den Grund dieses Grinsen bei den Gestalten hinter dem Hohepriester. Zwischen zwei weiteren 
hochrangigen Priestern folgte eine schlanke Frau. Ihre Kleidung bestand aus farbenfrohen Stoffen, 
die ihren Körper in einer ihm unbekannten Wickeltechnik umhüllten. Ihre Augen glichen erloschener 
Kohle und ihre Haut schimmerte wie Bronze.\\
\textit{Das ist mal viel Gold}, brummte Riolean und Semric blickte unwillkürlich auf den 
klimpernden Schmuck. Armbänder, Ohrringe, Ketten um den bloßen Hals und den schlanken Fußknöcheln. 
Selbst in ihr rabenschwarzes Haar waren goldene Fäden geflochten. \textit{Da will sich jemand 
einschleimen.} Riolean kicherte abwertend.\\
Semric ignorierte sie und wartete ruhig ab, bis die Priester mit der Frau näher heran gekommen 
waren. Sie knickste elegant und blickte ihn aus ihren dunklen Augen an. Der Hohepriester 
verkündete: ``Ich stelle vor, Mihiki Sa Elren. Die reizende Tochter des Elosas Or Elren, Verwalter 
des ersten Bezirks der Kolonien und Herr über die Städte Esral, Melras und Mihik. Die übrigens nach 
seinen Töchtern benannt sind.''\\
Der junge König schwieg und sah sie nur an. Die Anwesenden um ihn herum begannen zu tuscheln, 
während die Fremde immer unruhiger wurde.
``König Saleicas, Sohn des Osymas, erlaubt mir zu sprechen'', erklang ihre Stimme schließlich.\\
Sie hatte einen rauen Akzent.\\
Er nickte nur als Antwort und ließ Hisio-Mahar nicht aus den Augen. Er sah selbstgefällig aus. 
Semric hätte ihm am liebsten vor die Füße gespuckt, aber stattdessen bemühte er sich, nicht all zu 
angespannt auszusehen. Er würde Ilias Leben nicht riskieren. Nicht für so etwas nichtiges wie eine 
Hochzeit.\\
``Der Rat, der sich aus den Bezirken eurer Kolonien zusammen setzt, kam zu den Entschluss, dass es 
Eurem gesamtem Reich nur zu Gute kommen kann, wenn Ihr eine Vertreterin der Kolonien an Eurer Seite 
habt. Das würde den Worten der Rebellen ihre Macht nehmen. Mein Volk würde sich nicht mehr wie 
Sklaven fühlen, sondern als gleichberechtigte Bürger Saleicas. Die Kolonien wünschen sich, zu 
Saleica zu gehören.''\\
Semric schwieg weiterhin. Es war schon seit längerem im Gespräch, erste einzelne Bezirke 
einzugliedern. Aber jedes Mal kam es zu erneuten Aufständen des einfachen Volks. Die, die ihren 
Reichtum behalten oder gar von den Eroberern erhalten hatten, plädierten dafür. Sie genossen die 
Handelsbeziehungen und Elosas Or Elren war nicht der Erste, der die Idee hatte, nach der Krone zu 
greifen. Riolean hatte auch tatsächlich vorgehabt sich in diese Richtung zu verloben. Momentan war 
es ruhig, aber Semric glaubte nicht daran, dass es so bleiben würde. Sie hatten seinen Vater und 
seine Schwester ermordet. In diesen Völkern brannte immer noch die Gier nach Rache, die Sehnsucht 
nach der alten Zeit. Nach Freiheit. Es war zu wenig Zeit vergangen, als dass die Menschen die Morde 
und Verbrechen der Eroberer vergessen hätten. \textit{Wenn doch sogar in Merandila der Zorn noch 
lebt,} grübelte er.\\
``Das heißt'', sagte er schließlich und das Tuscheln verstummte: ``Du verzichtest auf deine Heimat. 
Auf deine Traditionen. Auf deine Familie. Auf deine Götter?''\\
Mihiki Sa Elren warf einen zögernden Blick in die Richtung des Priesters und verneigte sich erneut 
vor Semric. ``Der größte Wunsch meines Volkes wäre, dass wir eins werden und einander gleichen. 
Aber ich weiß, dass es Zeit braucht, bist diese Knospe erblühen wird.''\\
``Und wie viel Blut wird diese Blume trinken müssen?'', entgegnete er und lehnte sich vor: ``Wie 
viele Klingen Herzen durchbohren? Wie viele Köpfe werden rollen und wie viele Kinder 
schreien? Dein Volk will keine Eingliederung, Mihiki Sa Elren. Zumindest kann ich deinen Worten 
nicht glauben, nachdem was ihr getan habt.''\\
Semric vernahm zustimmendes Gemurmel der anwesenden Adeligen. Aus den Augenwinkeln sah er Erhim 
kaum merklich nicken. Die Gesandte stattdessen fiel auf die Knie. Ihre Hände stützten sich auf dem 
steinernen Grund, ihr Haar verbarg ihre zitternden Schultern. ``Die Verbrecher haben gesühnt, 
Majestät. Der Mord an Eurem Vater und Eurer Schwester ist zehn Jahre her!''\\
Semric kratzte sich nachdenklich am Kinn. ``Prinzessin Riolean war deiner Heimat sehr angetan. 
Sie hätte euch diese Chance gegeben. Sie hätte eine Versöhnung angestrebt. Aber sie ist tot. Dein 
Volk mag diesen Vorfall irgendwann vergessen. Auch mein Volk könnte diese Tat vielleicht irgendwann 
ignorieren. Aber ich nicht. Sie war meine Schwester, ich denke, Ihr könnt das verstehen.''
Sie hob den Blick. ``Majestät. Es wäre ein erster Schritt. Irgendwann muss er getan werden.''\\
``Ein erster Schritt für deinen Vater und die Mächtigen der Kolonien. Ein weiterer Anlass zum Krieg 
für alle anderen.'' Er wandte sich an Hisio-Mahar: ``Wir sind bereits im Krieg. Was meint Ihr, 
Hohepriester, wäre es ratsam, eine zweite Front fern ab des Festlandes und unseren 
Versorgungsrouten zu riskieren? Unsere erfahrensten Soldaten sind bald auf dem Weg in den Norden. 
In den Kolonien stehen wie viele? Vier Garnisonen mit je 500 Soldaten zur Verfügung? Die eine 
Wegstrecke von Wochen auseinander entfernt sind?''\\
Hisio-Mahar räusperte sich für eine Erwiderung, doch Lerin trat vor. ``Durchschnittlich 400 Kämpfer 
pro Garnison, mein König. Inklusive Generäle und Kommandanten. Für Einzelheiten kann ich die 
Unterlagen holen lassen.''\\
Semric schüttelte den Kopf. ``Nicht nötig, danke.''\\
Er fing einen finsteren Blick von Hisio-Mahar auf, während er sich auf seinen Thron zurück lehnte 
und die junge Frau nachdenklich musterte. Er zögerte, versuchte abzuwägen, wie weit er noch gehen 
durfte, ohne den Hohepriester zu Taten zu provozieren.\\
``Wenn wir irgendwann Frieden mit den Kolonien wollen, müssen wir ihn sähen, Majestät!'', warf 
Hisio-Mahar ein: ``Es ehrt Euch, dass ihr König Kareen und Prinzessin Riolean stets in Eurem 
Herzen tragt, aber Ihr seid nicht nur ein einfacher Mann. Ihr seid der König Saleicas und müsst für 
die Zukunft handeln! Eure Bedenken sind natürlich klug und auch Vorsicht schadet nicht. Aber 
unterschätzt den Kampfgeist Eurer Soldaten nicht. Und Osymas Glanz ruht auf uns.''\\
``Gewiss. Aber auch Löwen können nicht auf zwei Kontinenten gleichzeitig kämpfen. Erhebt Euch, 
Mihiki Sa Elren.''\\
Die Gesandte gehorchte und sah entschlossen zu ihm auf. Auch Semric stand auf und trat die wenigen 
Stufen hinab. Er verneigte sich vor ihr und hauchte ihr einen Kuss auf den gebräunten Handrücken. 
``Ich heiße Euch in Brom-Dalar willkommen, Gesandte Mihiki Sa Elren'', begrüßte er sie förmlich: 
``Als Zeichen unserer Wertschätzung werden wir einen Ball zu Eurer Ankunft ausrichten. Auch würde 
ich Eure Teilnahme an Ratssitzungen über die Kolonien begrüßen um Eure Sicht der Dinge zu diesen 
Themen zu hören. Und dann'', fügte er mit einem Blick zu Hisio-Mahar hinzu: ``werden wir weiter 
sehen.''\\
Ihre Mundwinkel zuckten zu einem Lächeln, ehe sie ein weiteres mal in einen Knicks sank und 
wisperte: ``Danke für diese Ehre, Majestät.''\\


Ilia Ma'Sah saß in Gedanken verloren auf dem Bärenpelz und starrte in die Flammen des Kaminfeuers. 
Ihr Vater war wie die letzten Tage mit alten Militärkameraden unterwegs um über den 
nahenden Krieg zu philosophieren. Er war regelrecht aufgeblüht, als es verkündet wurde. 
Krieg! Dafür schlug das Herz ihres Vaters. Fast schon beeindruckt sah sie zu, wie er sogar seine 
Angewohnheit, bereits früh am Morgen seinen Tag mit einer Karaffe Wein zu beginnen, abgelegt 
hatte. Aber sie sah auch sich selbst wieder. Als Mädchen über ihren Büchern gebeugt, nur ein 
kurzer Abschiedsgruß und Klirren seines Kettenhemdes, ehe er zur Tür verschwand. Ihre Mutter hatte 
jedes Mal geweint. Ilia nicht. Sie hatte nur auf den leeren Raum, den er zurück ließ, gestarrt und 
gewartet.\\
\textit{Ob er sich wohl einbildet noch ein Mal in eine Schlacht ziehen zu dürfen? In seinem 
Alter?}\\
Ilia strich sich nachdenklich die Feder immer wieder sanft über Wange und Lippen, während sie 
versuchte die richtigen Worte zu finden. Worte, an ihren Verlobten oben im Norden. Sie standen 
im regen Austausch. Auch ihr Vater bekam Briefe von Jozah. Informativer natürlich als die ihren 
und das allein reichte schon um sie zu reizen. Sie war nicht mehr das kleine Kind, dass auf den 
Vater wartete. Sie würde nie wieder in diese Lethargie, welche ihr kostbare Jahre gestohlen 
hatte, verfallen. Nicht für Jozah.\\
Ilia bemühte sich so viel Ehrlichkeit und Liebe in ihre Worte zu füllen, wie ihr Herz nur konnte. Es 
war die dritte Fassung des Briefs, denn sie wagte nicht die Worte sich laut vor zusagen um eine 
passende Formulierung zu finden. Auch wenn sie alleine war, bis auf die Wache und den Knecht im 
Stall. Hier im Haus am Meer war es zu trostlos für Ilia. Sie sehnte sich nach dem Lärm der Stadt. 
Nach Geschichten und Menschen.\\
``Herrin'', räusperte sich die Wache, die plötzlich im Raum stand.\\
Ilia blickte ungeduldig auf und wedelte mit der Schreibfeder: ``Was ist?''\\
Er suchte nach den passenden Worte. ``Zwei Männer wollen Euch besuchen,'' sagte er ausweichend.\\
``Dann schick sie weg. Ich habe zu tun.''\\
Der Mann zupfte nervös an seiner zweckmäßigen Kleidung. ``Ähm... mir wäre es lieber, wenn Ihr das 
selbst tun würdet.''\\
``Wieso? Meinst du, potentielle Entführer lassen sich lieber von einer zarten Schönheit fortjagen 
als von einer bewaffneten, gut ausgebildeten - und wenn ich das betonen darf - gut bezahlten 
Wache?''\\
Der Mann schwieg, rührte sich aber nicht von der Stelle. Ilia runzelte die Stirn und schüttelte 
ungläubig den Kopf. Aber sie erhob sich, strich das Kleid glatt und trat zur Tür. Sie war noch 
geöffnet und die Besucher warteten höflich an der Schwelle. Ilia warf einen Blick über die Schulter 
des Mannes, der nur wenig größer war als sie selbst. Der Knecht hielt noch die Zügel der Pferde in 
der Hand und wartete ab, ob die Herren wieder fort geschickt wurden. Erst dann besah die Adelige 
ihren Besuch genauer. Mit einem ehrlich überraschtem Lächeln begegnete sie den blauen Augen des 
Mannes und knickste förmlich. ``Welch Überraschung'', sagte sie: ``Das kann nur Osyma gewesen sein, 
der Euch zu mir brachte, werter Melior. Ein Schreiber ist genau das, was ich gerade brauche!''\\
Die Wache starrte die Besucher skeptisch an, hielt aber den Mund. Dafür wurde er immerhin äußerst 
reichlich bezahlt. Melior tippte sich höflich an den Hut und rang sich ein Lächeln ab. ``Oh, das 
passt ja gut...''\\
Ilia griff ihn am Ärmel und zog den Schreiber hinter sich her in den Salon. Gemütlich, aber etwas 
schicklicher als vorher noch, ließ sie sich wieder auf dem Boden nieder und ergriff das Pergament. 
``Ich würde Euch und Eurem Freund ja Tee anbieten, aber die Bedienstete hat heute ihren freien 
Tag.''\\
``Und Ihr könnt kein Wasser kochen?'' fragte Meliors Begleiter.\\
Ilia hatte ihn bisher noch nicht sprechen hören, aber auch ihm schenkte sie ein charmantes Lächeln, 
während sie erwiderte: ``Doch. Aber ich will nicht.''\\
Die Männer sahen sie irritiert an. Sie klopfte auf das Fell neben sich. ``Außer Ihr seid am 
verdursten, Melior. Ansonsten wäre es wirklich nett, wenn Ihr mir bei diesem Brief helfen würde.''\\
Melior nahm den Hut ab und begann laut zu lachen. Die Dame legte den Kopf schief und betrachtete 
seine blonden Haarsträhnen und wie die Heiterkeit sein Äußeres veränderte- Fast bereute sie, dass 
dieser Moment so schnell verstrich.\\
``Ihr seid eine unhöfliche Frau, Herrin Ma'Sah'', lachte er.\\
Ilia tippte sich mit der Feder an die Lippen. ``Nein. Ich bin nur ehrlich. Ehrlichkeit gibt es 
unter den Leuten meines Standes kaum noch, müsst ihr wissen. Als ob altes Blut vorraussetzt, dass 
jedes Wort welches man spricht, eine Lüge ist. Außerdem gibt es einen tieferen Grund... Freund des 
Schreibers, dessen Name mir noch nicht gesagt wurde. Als Kind traf mich kochendes Wasser und 
hinterließ seine Spuren. Deshalb hat selbst mein Verlobter noch nie meine linke Wade gesehen!''\\
Melior ließ sich auf das Fell nieder und nickte seinem Begleiter auffordernd zu, aber er hielt sich 
im Hintergrund und lehnte sich nur lässig an eine Kommode aus kasirischem Bergholz. Sündhaft teuer, 
ein Geschenk zu ihrem fünfzehnten Geburtstag. Aber hässlich, deshalb hatte Ilia es nicht in das 
Stadtanwesen bringen lassen.\\
``Also los, Melior. Als Schreiber fällt Euch bestimmt gleich ein reizendes Gedicht über die 
unsterbliche Liebe zu meinem Verlobten ein. Los, zeigt was Ihr könnt!'', forderte Ilia kichernd auf 
und tränkte die Feder in Tinte, bereit, sofort los zu schreiben. Der blonde Schreiber jedoch sah 
etwas hilflos drein und blickte, ebenso wie sie noch vor wenigen Augenblicken, grübelnd in die 
Flammen. Ilia seufzte und senkte die Feder. ``Ich weiß. Ich grüble auch schon seit Stunden. 
Zumindest fühlt es sich an wie Stunden... man meint immer, dass so etwas einem doch leicht fallen 
müsse. Aber es ist schrecklich schwer Gefühle in Worte zu fassen!'' \\
``Weil es Gefühle schon vor den Worten gab'', murmelte Melior zu den Flammen.\\
``So eine Aussage hätte ich von einem Schreiber nicht erwartet! Kannst du mir nun helfen?''\\
Melior zog die Schultern hoch. ``Ehrlich gesagt, ich bin besser im Zeichnen als im Schreiben. 
Also... selbst schreiben, meine ich. Mir wird meistens diktiert...''\\
Ilia schüttelte den Kopf. ``Ich glaube, Ihr werdet in Eurem Beruf nicht sehr erfolgreich sein... 
wenn es dazu kommt, dass Ihr Hunger leidet... sprecht mich an, dann lasse ich für Euch kochen. Aber 
zeichnen? Dann zeichnet mir etwas.''\\
Ilia sprang schwungvoll auf die Beine und Melior griff gerade noch nach der Feder um Tintenflecke 
zu vermeiden. ``Irgendwo hatte ich noch Kohle herumliegen... wartet, ich suche sie schnell! 
Pergament ist in der Kommode hinter Euch, Begleiter ohne Namen! Wenn Ihr so freundlich wärt.''\\
``Ein Porträt?'', fragte der Schreiber etwas überrumpelt.\\
Ilia reichte ihm die Kohlereste und überlegte einen Moment. \textit{Na... wieso nicht...}, dachte 
sie spontan und öffnete mit wenigen Griffen ihr Gewandt. Sie war unheimlich neugierig, wie ihr Gast 
wohl reagieren würde. ``Es ist immerhin für meinen Verlobten'', fügte sie hinzu und ließ sich auf 
dem Bärenfell nieder. Während die beiden Männer sie ungläubig anstarrten, probte Ilia einige 
verschiedene Posen und entschied sich schließlich dafür, auf dem Rücken liegen zu bleiben. Ihr 
blondes Haar breitete sich wie ein Fächer um sie herum und das Melior zugewandte Bein winkelte sie 
aufrecht an. ``Passt das so? Der Künstler muss schon etwas dazu sagen!''\\
``Äh...'', stammelte er.\\
``Joa, dass passt so schon'', warf sein Begleiter grinsend ein.\\
Ilia winkte ungeduldig mit der Hand. ``Dann los.''\\
Melior überlegte kurz und setzte sich dann in den Sessel um eine andere Perspektive auf die Frau zu 
haben. Und dann war es lange Zeit still. Ilia schloss die Augen und spürte die Wärme des 
Kaminfeuers neben sich. Sie lauschte auf die Geräusche der über das Papier kratzenden Kohle und 
versuchte sich auszumalen, wie ihr Abbild entstand. Irgendwann räusperte der Zeichner sich und 
murmelte leise ``Es ist fertig...''\\
Ilia erhob sich in einer geschmeidigen Bewegung, schlüpfte schnell in ihr Gewandt und trat auf 
Melior zu. Er reichte ihr das Papier und sah sie nervös an.\\
``Danke'', sagte sie und hielt seine Hand einen Moment länger als nötig.\\
``Ich glaube, wir sollten langsam heimkehren'', räusperte sich der namenlose Begleiter.\\
Der Schreiber nickte nur wortlos und Ilia tat es ihm gleich. ``Ich hoffe doch, wir sehen uns bald 
wieder'', fügte sie hinzu, als sie die Herren zur Tür begleitete. Erst als sie wieder alleine war, 
betrachtete sie die Zeichnung. Es war als wäre kein Strich zu viel und doch hauchten die schwachen 
Schattierungen dem Bild Leben ein. Die Frau auf dem Bild hatte die Augen geschlossen und doch 
einen verträumten Ausdruck. Mit zarten und nur wenigen Strichen hatte er mit der schwarzen Kohle 
ihr helles Haar dargestellt. Dann fiel ihr ihre Wade ins Auge und sie betrachtete, wie der Künstler 
mit feinen Strichen die Narben festgehalten hatte. Ihre Lippen wurden zu einem schmalen Strich. Ilia 
war schon oft Modell gestanden und nie hatte es ein Künstler gewagt diesen Makel fest zu halten. Sie 
hätte es ihm nicht zu getraut. Noch nie hatte sie ein so ehrliches Bild von sich selbst gesehen. 
Jedes Muttermal, jede Narbe, jede Rundung ihres Körpers. Nachdenklich schlenderte sie zum Bärenfell 
zurück und nahm den angefangenen Brief wieder auf. Sie tunkte die Schreibfeder in frischer Tinte und 
begann zu schreiben:

\textit{Unsere Begegnungen erfüllten mich, Jozah. Du gabst mir ein Gefühl gezeigt, welches mir 
unbekannt war. Ich trage einen alten Namen. Einen der Ältesten unseres Reichs. Ich besitze Geld. 
Ich bin die einzige Erbin meines Vaters, einen altgedienten und ehrbaren Offiziers im Ruhestand. Ich 
bin kostbar. Du weißt das. Und bisher saß mich jeder so. Aber jeder meiner bisherigen Verehrer 
gehörte dem Adel an und jeder von ihnen war mit Kostbarkeiten aufgewachsen. Sie wussten wie man mit 
ihnen umgeht und sie wussten, dass man Schätze ersetzen kann, wenn sie zerbrechen. Du nicht. Ich 
weiß, dass wenn du mich Kostbarkeit verlierst, es dir dein Herz brechen wird. Dafür liebe ich dich. 
Deshalb habe ich der Verlobung doch noch zugestimmt. Ich dachte, wenn ich schon jemanden lieben 
muss, dann jemanden, der mich nicht ersetzen kann.\\}
\textit{Es tut mir Leid,}\\
\textit{aber er ist der König.}\\

\textit{Osymas Segen sei mit dir,}\\
\textit{Ilia Ma'Sah}\\

Sie legte die Feder weg und las noch einmal die Zeilen. Nachdem die Tinte getrocknet war, faltete 
sie das Papier zusammen und versteckte es in ihrem Zimmer. Noch war nicht die Zeit, diesen Brief 
abzuschicken. Erst musste sie sich sicher sein, dass das Herz des Königs ihr gehörte.
