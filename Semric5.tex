\chapter{Der Ball}

Es war einige Monate her, seit der große Saal das letzte Mal für eine prächtige Veranstaltung 
hergerichtet wurde. Semric hatte die Entscheidungsgewalt dafür abgegeben. Es war ihm gleich welche 
Farben die Blumen auf dem Tisch hatten oder welcher Fisch für den zweiten Gang aufgetragen wurde. 
Ehrim beschwerte sich kurz, als Semric sogar die Weinproben ablehnte und jemand anderem überließ. 
Zwei Mal rang er sich dazu durch, mit Mihiki Sa Elren einen Spaziergang durch die winterlichen 
Gärten zu unternehmen und drei Mal aß er mit ihr zu Abend. Die Gespräche waren oberflächlich und 
langweilig. Es fühlte sich viel zu gezwungen an. Jedes Wort und jede Bewegung der Frau schien 
sorgfältig überlegt. Einmal war er kurz davor ihr zu sagen, dass sie sich die Mühen sparen sollte. 
Falls er wirklich in die Bedrängnis kommen würde, sie heiraten zu müssen, dann ohne Liebe. Er 
empfand es als Unrecht, dass er sie ansah und sich wünschte, einen andere Frau würde hier neben ihm 
spazieren. Er kannte Mihiki nicht näher, aber fühlte sich nicht jede Frau von so etwas gekränkt? 
Erhim fand eine einfache Erklärung für dieses Problem. Eines Abends platzte er wieder Mal 
unangekündigt in Semrics Gemach und rief laut: ``Es ist die Nacktheit! Damit haben Frauen schon 
Kriege entschieden!''\\
Beschwichtigend hob der König die Hände und hoffte, sein Leibwächter würde die Stimme senken. Im 
Schloss hörte immer irgendjemand etwas. ``Wovon sprichst du?'', fragte er verärgert.\\
``Na, an wen denkst du denn, wenn ich \textit{nackt} sage?'', spottete Erhim: ``Man kann nicht 
leugnen. Ist schon eine bildschöne Frau. Immerhin verfolgt mich ihre Gestalt auch schon seit 
Wochen, dabei mag ich Blondinen gar nicht.''\\
``Brauchst du einen Abend frei? Vielleicht bist du nach einem Besuch in einem Bordell wieder bei 
klareren Gedanken!''\\
``Oh... aber wenn man es anfassen darf, ist es doch viel zu einfach!''\\
``Schluss jetzt!'', knurrte Semric.\\
``Natürlich. Ein Leibwächter sollte nicht die Liebste seines Herrn begehren. Verzeiht mir'', 
spottete Erhim: ``Sei dir sicher, mein Interesse ist rein körperlich. Nichts mit Gefühlen oder so. 
Da gehört sie ganz dir! Oh... und ihrem Verlobten.''\\
``Verschwinde!'', schimpfte Semric und Erhim verzog sich grinsend wieder auf seinen Posten vor der 
Tür.\\

Ilia dagegen konnte den Tag des Balls kaum erwarten. Es kam in den letzten Jahren nur noch selten 
vor, dass der Palast zum Tanz einlud. Früher soll es wohl öfters solche Veranstaltungen gegeben 
haben. Aber nicht nur die Gelegenheit endlich wieder unter Leute zu kommen erfüllte sie so mit 
Spannung, nein, sie hatte sich auch einiges für diesen Abend vorgenommen. Sie würde den König 
konfrontieren. Er konnte doch nicht wirklich glauben, dass sie nicht wusste, wer wer er war? 
Lächerlich.\\
Einen kurzen Moment hatte sie gezögert, als sie die Einladung in den Händen hielt. Wieso nicht 
dieses Spiel noch etwas weiter führen? Sie bezweifelte, dass König Semric auf die gleiche Art 
ungezwungen und heiter mit ihr umgehen würde, wenn er nicht die Rolle des Schreibers einnahm. Dann 
jedoch hatte sie die Worte ein weiteres Mal überflogen und den Grund für den Ball gesehen. Zu Ehren 
einer Repräsentantin aus den Kolonien. Ihr hübsches Gesicht verfinsterte sich bei diesen Worten. 
Sollte diese Fremde nicht mindestens das fünfzigste Lebensjahr überschritten haben, sah Ilia sie 
als Rivalin. Immerhin war König Semric in einem guten Heiratsalter für einen König und besaß 
keine Geschwister als potentielle Erben.\\
Ihr Vater war ebenfalls in der Einladung benannt und auch wenn er ihr den Besuch erst verwehren 
wollte, duldete sie das nicht. Wochenlang nun war sie auf seinen Wunsch hin im Anwesen 
außerhalb der Stadt geblieben, während er zu seinen Freunden und ehemaligen Kameraden in Brom-Dalar 
ritt. Wo sollte es sicherer sein als im Palast des Königs? Daher reiste sie schon einen Tag vor dem 
Ball zurück in die Hauptstadt, immerhin befanden sich dort ein Großteil ihrer Kleidung. Um ein 
neues anfertigen zu lassen war die Zeit zu knapp, außerdem hatte ihr Vater es verboten. Der Krieg 
kam und er meinte für so etwas musste man in Kriegszeiten kein unnötiges Geld ausgeben.\\
Ilia musste nicht überlegen, welches Kleid sie tragen würden. In solchen Dingen war sie stets sehr 
zielstrebig. Als ihr Vater bereits ungeduldig in der Kutsche wartete, warf sie noch einen letzten 
Blick in den Spiegel. Sie trug ihr Haar auf einer Seite geflochten und offen fiel es ihr dann über 
die linke Schulter. Es kringelte sich in sanften Wellen. Auf Haarschmuck hatte sie verzichtet um 
die Flechtung nicht zu überdecken. Das rote Gewand ließ die Schultern offen und war nach einer 
mittlerweile eher veralteten - aber von Ilia noch sehr geschätzten - Mode mit Stickereien aus 
Goldfäden verziert. Auf Schmuck jeglicher Art verzichtete sie. \\
Mit schnellen Schritten und gerafftem Rock eilte sie die Stufen des Anwesens hinunter und ließ sich 
vom Knecht in die Kutsche helfen. Dankbar schenkte sie ihm ein Lächeln und holte erst einmal tief 
Luft, als sie saß.\\
``So aufgeregt habe ich dich schon lange nicht mehr gesehen'', bemerkte ihr Vater.\\
``Ich freue mich darauf, den Palast besuchen zu dürfen. Und ich bin neugierig auf diese Gesandte. 
Hast du etwas über sie gehört?''\\
Der alte Mann grummelte leise. ``Viele Vermutungen und Gerüchte. Schön soll sie sein, wenn auch 
auf fremdartige Weise. Der König wurde öfters mit ihr gesehen.''\\
``Nun... er muss seine Gäste ja höflich behandeln'', murmelte Ilia gekränkt.\\
Ihr Vater beobachtete sie wachsam und ergriff ihre Hand. ``Mein Kind, du siehst heute bildschön 
aus.''\\
``Danke Vater. Deine Uniform steht dir auch ausgezeichnet'', erwiderte Ilia grinsend.\\
``Jozahs Brief hat dich erfreut? Ist das der Grund, warum du so strahlst?''\\
Ilias Lächelnd wankte keine Sekunde, während sie ihrem Vater in die Augen sah und den sanften Druck 
seiner Hände erwiderte. ``Es freut mich sehr, dass es ihm gut geht'', antwortete sie.\\
``Offizier Lerim teilte mir mit, dass unser König wohl auch zufrieden mit seinem ersten Berichten 
ist.''\\
Ilia legte den Kopf schief. ``Den Grund seiner Reise habt ihr mir alle verschwiegen, hm? Warum? Ist 
es etwas verbrecherisches?''\\
Ihr Vater lachte. ``Nein. Vielleicht dachten wir alle nur, es interessiere dich nicht.''\\
``Dann unterschätzt ihr mich alle'', konterte Ilia beharrlich lächelnd: ``Ich habe vor mit dieser 
Mihiki Elras oder wie auch immer sie heißt zu sprechen. Heißt es nicht, in den Kolonien wimmelt es 
nur so von giftigen Schlangen? Unser König braucht eine Löwin an seiner Seite, keine Schlange.''\\
``Und du willst ihn vor einer potentiellen Giftschlange beschützten?'', fragte ihr Vater skeptisch 
und ließ ihre Hand los: ``Wie kommst du darauf? Es steht dir nicht zu, einen Gast der Krone zu 
beleidigen!''\\
``Ach Vater'', seufzte Ilia: ``Du kennst mich nun 25 Jahre... glaubst du wirklich, ich würde mir 
ein persönliches Gespräch mit dem König entgehen lassen, wenn ich die Möglichkeit dazu hätte? Du 
sagtest, er ist Jozah sehr zugetan.''\\
``Worauf willst du hinaus?''\\
``Vater... stehst du hinter mir, egal was geschieht?''\\
Er betrachtete die Frau vor sich. So schnell war die Zeit vergangen. Vor wenigen Jahren noch ein 
kleines Mädchen mit wildem Blick, welches kaum still sitzen konnte. Wann war der Moment gekommen, 
in der er ihre Gedanken nicht mehr lesen konnte wie ein offenes Buch? Hatte er verlernt es zu sehen 
oder hatte sie gelernt sich zu verbergen? ``Du bist meine Tochter. Mein Herz'', antwortete er 
schließlich: ``Nur sei vorsichtig.''\\

Da der Ball zu Ehren des Gastes veranstaltet wurde, stand Mihiki Sa Elren direkt neben dem König 
Saleicas. Sie hatte sich sehr bemüht eine passende Erscheinung abzugeben. Immerhin sollte dies nur 
eine der ersten offiziellen Veranstaltungen werden, bei denen sie neben König Semric stand. Nach 
der momentanen saleicanischen Mode trug sie ein eng geschnittenes Kleid welches ihre schmale Figur 
betonte und keine Ärmel besaß, dafür aber hoch geschlossen war. Sie hatte es sich aus einem sanften 
goldfarbenen Stoff schneidern lassen und ihren Goldschmuck wieder üppig angelegt. Auch in ihr Haar 
waren wieder Goldfäden und Perlen geflochten. Sie wollte dem König zeigen, dass sie bereits in der 
Kleiderwahl darauf wert legte, sich an die saleicanischen Vorlieben anzupassen. Nur der Haarschmuck 
war eher eine Tradition ihrer Kultur. Immer wenn der junge König ihr einen Blick zu warf, lächelte 
sie ihn herzlich an.Sie versuchte auch ein Gespräch mit ihm zu beginnen, aber 
anscheinend war Sinn dieser Aufstellung, dass jeder hochrangige Gast persönlich vom König 
begrüßt wurde. Die Meisten handelte er mit einem höflichen Nicken oder einer freundlichen Bemerkung 
ab. Einige wenige kamen so nahe heran, dass er ihnen die Hand schüttelte oder den Damen einen Kuss 
auf die Hand hauchte. Die Leute schienen selbst zu wissen, wie weit sie sich ihrem König zur 
Begrüßung nähern sollten und nur selten ging die Initiative von ihm aus, die Gäste näher zu sich zu 
holen, auch wenn er immer höflich wirkte.\\
Um so verblüffter war sie, dass seine Mimik sich völlig veränderte, als das nächste Paar näher 
schritt. Ein alter Herr, der sich aufrecht hielt und einen kurzen Bart trug. Bei ihm eingehakt 
hatte sich eine junge Dame mit hellem Haar. Der Herr senkte wie die vielen Gäste zuvor den Blick, 
verharrte neben dem Kohlebecken und verneigte sich. Mihiki erkannte an der Verneigung, dass es wohl 
doch jemand mit hohem Rang war, denn er ging nicht so tief wie manch andere. Vielleicht lag das 
aber auch nur an seinem Alter. Erstaunlicher war das Verhalten der jungen Frau. Sie wandte den 
Blick nicht vom König ab, trat sogar noch herausfordernd einen Schritt näher.\\
Auch der Bedienstete, der die Gäste mit Namen vorstellte, zögerte irritiert. Einige Sekunden 
verstrichen ohne das einer der Beteiligten irgendetwas tat. Mihiki fühlte sich wie auf einer 
stummen Insel umgeben vom Lärm der versammelten Gesellschaft. Dann sank die Dame in einen eleganten 
Knicks, doch hielt den Blick weiterhin aufrecht. Der Bedienstete fasste sich wieder und beeilte sich zu 
verkünden: ``Vito Ma'Sah in Begleitung seiner Tochter Ilia Ma'Sah!''\\
Semrics Blick huschte von der Blondine zu dem Herrn und wieder zurück. ``Sehr erfreut'', murmelte 
er halblaut.\\
``Ich hoffe doch, dass ich mich auf einen Tanz freuen darf, mein König?'', fragte sie mit klarer, 
lauter Stimme. Ihr Vater zuckte regelrecht zusammen und hätte sie am liebsten gepackt und fort vom 
Podest gezogen. Mihiki kniff die Augen zusammen. Natürlich, die Saleicanerin war hübsch. Aber die 
meisten Frauen hier waren hübsch. Wieso reagierte der König so ganz anders als bei den anderen 
Gästen? Sie erwartete, dass er sie zu sich winken würde, stattdessen verließ er den Platz, den er 
seit einer Stunde inne hatte, und trat mit einem überraschtem Lachen auf sie zu.\\
``Tanzen?'', wiederholte er und ergriff ihre Hand zum Kuss. ``Ehrlich gesagt blamiere ich mich 
nicht gern!''\\
``Keine Sorge, Majestät. Mit mir an seiner Seite kann man sich nicht blamieren!''\\
Kurzentschlossen folgte Mihiki dem Herrscher und räusperte sich. ``Es freut mich, Euch kennen zu 
lernen!''\\
Der Herr verneigte sich angemessen, immerhin war sie der Grund für diesen Ball. Doch Ilia dachte 
nicht daran, vor der Fremden einen Knicks zu machen. Stattdessen griff sie nah ihrem Unterarm, was 
eine vertraute Begrüßung unter Freunden darstellt. ``Willkommen in Saleica'', sprach sie: ``Es 
freut mich zu sehen, dass Ihr Euch bereits bemüht an der Kultur unseres Landes beizuwohnen. Ach... 
eine Kultur in der die Mode immer im Wandel ist! Deshalb habe ich mich heute dafür entschlossen - 
auch im Anbetracht der immensen Kriegskosten die auf unser Land zukommen und in Gedenken an die 
Tapferkeit unserer Soldaten - diese Mode auszulassen.''\\
Mihiki sah sie geschockt an. Viel schlimmer als diese diskrete Beleidigung war die Reaktion der 
umstehenden Adeligen. Die Damen erröteten und zupfte unbeholfen an ihren Gewändern. Getuschelte 
Diskussionen kamen auf, viele nickende Köpfe und beschämte Blicke. 

``Ilia!'', raunte ihr Vater ihr zu uns zog sich an den Rand der Menge. Endlich war ein ruhiger 
Moment gekommen, ohne dass weitere ehemalige Kameraden oder Freunde ihn bequatschten. ``Was sollte 
das? Woher kennt dich der König? Auf dem Ball letztes Jahr hat er uns kaum eines Blickes 
gewürdigt!''\\
``Findest du es nicht auch lächerlich, wie die Fremde durch diese Hallen stolziert? Wie sie sich 
herausgeputzt hat in Gold? Als könnte die Farbe allein sie zu einer Löwin machen'', zischte Ilia 
und folgte Mihiki mit den Augen.\\
``Ich habe dir eine Frage gestellt!''\\
``Wir lernten uns eben kennen'', antwortete sie schlicht: ``Du hast gesagt, du stehst hinter 
mir!''\\
``Was hast du vor?'' Wo ist das Mädchen geblieben, dass ihm jedes Mal entgegen rannte, wenn er aus 
der Kaserne heimkehrte? Wo ist das Mädchen hin, dass lachend auf seinem Schoß saß oder artig 
vortrug, was es diese Woche vom Hauslehrer gelernt hatte? Das kasirische Alphabet, die 
saleicanischen alten Lieder und Tänze? Reiten und fechten? Vito spürte jetzt mehr als jemals 
zuvor, wie das Alter an ihm nagte. Er hatte alles gegeben um seine Tochter auf sein Erbe 
vorzubereiten. Ein so alter Name wie der ihre war auch eine Last. ``Kind'', seufzte er.\\
``Heute? Heute will ich nur mit dem König tanzen. Und dieser Schlange zeigen, dass sie niemals eine 
Löwin wird.''\\
Ilia sah auf und grüßte den näher kommenden Mann mit einem Lächeln. ``Begleiter, der mir noch nicht 
vorgestellt wurde!''\\
Erhim legte die rechte Hand auf sein Herz und verneigte sich. ``Dann erledigen wir das 
mal: Erhim, der Schatten des Königs, geschätzte Ilia Ma'Sah. Unsere Majestät lässt mich ausrichten, 
dass er sich gerade seinen Verpflichtungen noch nicht entbinden kann und ich Euch zum Tanz einladen 
darf.''\\
``Hm...'' Ilias Blick schweifte enttäuscht über die Menge, aber sie sah den König nicht. ``Kannst 
du denn tanzen, Schatten?''\\
Er lachte. ``Meine Tanzkünste retteten schon mein Leben.''\\
``Dann solltest du wohl besser mit unserem König üben.''\\
Er reichte ihr seine Hand. ``Ich bemühe mich bereits. Also, darf ich bitten?''\\

Es war ein langsamer Tanz und Ilia ließ kaum Raum zwischen ihrer beiden Körper. Schon allein aus 
dem Grund, weil sie Worte an ihn richten wollte, die niemand sonst mitbekommen sollte. Aber Erhim 
kam ihr zu vor. ``Er will wissen, was das zu bedeuten hat. Ob Ihr es von Anfang an wusstet.''\\
``Wirklich? Denkt er wirklich, dass eine Adelige, die ihr ganzes Leben in der Hauptstadt verbracht 
hat, ihren König nicht erkennen würde?''\\
``Jetzt im Nachhinein denkt er es vermutlich nicht mehr, nein. Warum habt Ihr nichts gesagt? Ihr 
sagtet doch, dass Ehrlichkeit Euch so wichtig ist.''\\
``Hat er dich mir geschickt, damit du überprüfen kannst, ob ich einen Dolch im Ausschnitt trage? 
Oder Gift im Ring? Oh... ich trage gar keinen Ring! Nicht einmal eine Kette, mit der ich ihn 
erwürgen könnte!''\\
Erhim sah sie aus dunklen Augen schweigend an und Ilia verdrehte die Augen. ``Siehst du? Deshalb 
habe ich nichts gesagt. Sobald alle Beteiligten wissen mit wem sie es zu tun haben, verhalten sie 
sich eben nicht mehr ehrlich! Letzte Woche noch hättest du über diesen Witz gelacht, da bin ich mir 
ziemlich sicher. Oder du hättest zumindest gesagt, du würdest dich was den Dolch an geht erst noch 
mal überzeugen müssen'', feigste Ilia und setzte zu einer komplizierten Drehung an. Wie sie 
vermutet hatte, kannte Erhim diese Tanzschritte nicht, aber das fiel gar nicht auf, denn er 
improvisierte ohne zu zögern und fiel nicht aus dem Takt. ``Du und der König scheinen sich nahe zu 
stehen. Also sage mir, stimmt mein Gefühl und er war als Schreiber viel mehr er selbst als 
jetzt mit Krone und Palast?''\\
Erhim antwortete auch darauf nicht und das war Ilia Bestätigung genug. Sie kannte natürlich das 
Gerede, dass der König Saleicas sich zu sehr von den Priestern herum schubsen ließ. Dass die 
Priester und nicht er den Krieg mit Kasir entschieden haben und was sonst noch alles an großen 
politischen Entscheidungen gefallen war. Sie überlegte, was sie noch zu Erhim sagen sollte. ``Ich 
mag hin'', murmelte sie schließlich: ``Ich mag ihn als Menschen. Als Schreiber. Als Zeichner. Als 
Tanzpartner... Meine Familie hat einen angesehenen Rang inne und daher gibt es keinen 
Grund wieso Semric nicht mit mir tanzen sollte.''\\
``Die Priester.''\\
Ilia lachte spöttisch: ``Ich habe keine Angst vor Priestern! Sollen sie mich doch verfluchen! Ich 
mag ihn- Semric. Glaubst du mir das nicht?''\\
``Doch. Aber ich glaube auch, dass Ihr die Krone mögt.''\\
``Warum auch nicht? Sollte das Schicksal es bereithalten, dann wäre es doch gut, wenn wenigstens 
eine vom Königspaar die Krone gerne trägt.''\\
``Soll ich ihm das genau so ausrichten?'', fragte Erhim spottend.\\
``Natürlich! Wort für Wort! Falls du eines vergisst, kann ich es Semric auch persönlich sagen. Ich 
steh zu meinem Wort, dass mir Ehrlichkeit wichtig ist. Zumindest gegenüber Leuten, die mir etwas 
bedeuten.''\\
Erhim ließ sie los und blieb mitten im Tanz stehen. Er nickte. ``In Ordnung. Ich richte es ihm 
aus.''\\
``Wir hätten ruhig noch den Tanz beenden können!'', rief sie ihm empört hinterher, als er sie 
mitten auf der Tanzfläche stehen ließ. Ein junger Mann ließ seine Partnerin los und räusperte sich. 
Ehe er etwas sagen konnte, schüttelte Ilia den Kopf. ``Nein, vergesst es. Das wäre zum einen 
ziemlich unhöflich und zum anderen seid Ihr nicht mal Ansatzweise so gut wie er!''\\
Sie bahnte sich einen Weg zurück zu ihrem Vater, der mit einigen Bekannten plauderte und gesellte 
sich mit einem freundlichen Lächeln, aber den Gedanken weit weg, zu ihnen.\\

