\chapter{Aufstieg}

Ilia durchschritt das Zimmer mit ruhigen Schritten. Ihre Füße trugen sie zum vergoldeten Spiegel, 
neben dem sie stehen blieb um die Titel im Bücherregal zu betrachten. Sie nahm eines der Bücher 
über Heilkräuter heraus. Ihre langen Finger strichen über die Seiten, während ihre Augen der 
Schrift folgten. Dann stellte sie es ohne ein Wort zurück. Ihre Hand glitt an der Wandtäfelung 
vorbei, während sie zum nächsten Gegenstand ihres Interesses weiter wanderte. Ilia hatte schon 
seine Garderobe betrachtet, seinen Degen, die Sammlung Porzellanfiguren die er noch aus seiner 
Kindheit aufbewahrt hatte und kam nun bei seinen Zeichnungen an. Semric saß in einem seiner Sessel 
und beobachtete sie zurückhaltend. Alle Dinge die sie in der letzten Stunde schweigend betrachtete, 
berührte sie auch. Das war ihm gleich als erstes aufgefallen. Als würde sie die Dinge, die seine 
Persönlichkeit ausmachten, nicht nur mit den Augen betrachten wollen, sondern auch spüren. 
Plötzlich wurde Semric nervös, was sie von seinen Zeichnungen halten sollte. Der Tisch war übersät 
von Pergament, Kohle und teurer, farbiger Kreide. Nur eine große Fläche zum Zeichnen war 
leer geräumt. Mitten darauf lag seine neuste Zeichnung. Ein Bild von Ilia. Er konnte nichts anderes 
mehr zeichnen als sie. Oder Dinge, die mit ihr im Zusammenhang standen. Vor wenigen Tagen erst 
hatte er ein Bildnis ihrer Augen. Dafür hatte er sich die teure Kreide bringen lassen. Aber ganz 
zufrieden war er nicht. Er hatte es nicht geschafft, den exakten Farbton ihrer Augen zu finden.\\
Schließlich legte sie die Zeichnung zurück und blickte ihn quer durch den Raum mit eben diesen 
Augen an. Obwohl mehrere Schritte zwischen ihnen lagen, begann Semrics Herz schneller zu schlagen 
und er befeuchtete sich nervös die Lippen.\\
Ihre Stimme klang sachlich, als sie die folgenden Worte sprach. ``Du liebst mich?''\\
Semric schwieg. Er wusste nicht, was er darauf antworten sollte, immerhin hatte sie seine 
Zeichnungen gesehen. Sie sprachen alles aus, was er im Herzen trug.\\
``Ich schließe daraus, dass du anders empfindest?'', antwortete er dann doch.\\
Ilia lächelte und kam langsam näher. ``Semric'', sagte sie sanft, kniete vor ihm nieder und ergriff 
seine Hand. ``Ich liebte schon viele Männer. Aber ich glaube, keiner liebte mich bisher so 
wie du es tust. Ich sagte dir, dass mir Ehrlichkeit wichtig ist. Also werde ich auch ehrlich zu dir 
sein. Ich liebe dich. Aber ich werde keine Mätresse sein. Kein Mädchen, dass still hinter dir 
stehen wird oder sich gar in deinen Zimmern versteckt.''\\
Ihre Augen leuchteten, während sie weiter sprach: ``Ich kann dir nicht versprechen, dass meine 
Liebe ewig sein wird. Und ich will nicht nur deine Gattin sein.''\\
Semric blickte schweigend auf sie herab und wünschte sich, sie würde nicht knien. Nicht sie, die er 
so bewunderte.\\
``Ich will deine Königin sein'', sagte sie leise: ``Ich will an deiner Seite stehen. Die Krone 
tragen. Die Priester in ihre Tempel sperren und diesen Krieg gegen Kasir mit dir gewinnen. Ich will 
mit dir die Kolonien beherrschen, unser Reich vergrößern. Ehre und Rum erlangen. Ich will, dass die 
ganze Welt vor uns Löwen erzittert.'' Ihre Lippen berührten seine Hand. ``Alle, die unser Brüllen 
hören. Alles, was das Sonnenlicht berührt.''\\
Es war nicht der erste Kuss in Semrics Leben, aber der erste mit Bedeutung. Während einer Atempause 
fragte er: ``Hast du mir gerade einen Antrag gemacht?''\\
``Oh ja. Und er war sehr viel romantischer als den letzten, den ich bekommen habe.''\\
Semric zuckte zurück. ``Verdammt. Mi'Kae, nicht wahr? Erhim hat es mir berichtet!''\\
Ilia setzte sich in den zweiten Sessel und strich ihr Kleid glatt. ``Mein Vater wollte es. Jozah 
ist sein Schützling, seit er damals über die Mauern der Kaserne geklettert ist. Es hat nichts zu 
bedeuten und ich habe ihm längst einen Brief geschrieben, in dem ich die Verlobung auflöse. Wenn 
ihm wirklich etwas an mir liegen würde, dann wäre er nicht nach Merandila geritten.''\\
Semric runzelte die Stirn. ``Er hatte keine Wahl. Ich habe es ihm befohlen.''\\
``Ein Mensch hat immer eine Wahl'', entgegnete Ilia: ``Was ist nun? Wirst du mich zu deiner Königin 
machen? Werden wir beide die Löwen Saleicas sein?''\\
``Das ist alles nicht so leicht wie es klingt'', wich Semric aus und blickte auf den Boden.\\
``Auch Hisio-Mahar ist nur ein Mensch. Er ist ein Priester, kein König. Und kein Löwe. Wir müssen 
ihm seine Macht nehmen.''\\
``Wie?''\\
``Kamst du je auf den Gedanken, einen neuen Rat zu erwählen?'', fragte sie direkt und blickte ihn 
forsch an: ``Und Hisio-Mahar nicht in diesen einzuladen?''\\
Semric biss sich auf die Unterlippe. Nein, auf diesen Idee ist er in all den Jahren nie gekommen. 
Er räusperte sich. ``An wen hast du gedacht?''\\
``Meinen Vater. Zum Beispiel. Er hat viele Freunde im Militär. Außerdem dachte ich an Sil'Vera.''\\
``Die junge Witwe?'', fragte Semric überrascht.\\
``Nein. Ihren Vater. Er saß bereits im Rat deines Vaters. Er ist völlig mittellos. Würdest du ihm 
zu neuer Größe verhelfen, den Namen seiner Familie retten, wäre er dir zu ewiger treue 
verpflichtet. Außerdem hat es seine Gründe gegeben, wieso König Kareen ihn berufen hat. Sein 
Verhandlungsgeschick hat damals in den Kolonien viel gebracht.''\\
``Der Gräfin würde es vermutlich auch gefallen'', murmelte Semric.\\
``Dann noch ein paar Vertreter aus den anderen Grafschaften. Nicht unbedingt die Erben des Titels, 
eher die später geborenen Kinder. Dadurch hättest du direkten Einblick in die Verwaltung deines 
Landes und sie könnten als Botschafter fungieren. Du hättest mehr Kontrolle über die Grafschaften 
und deren Herren. Und, bestenfalls, schaffst du es die Loyalität der Ratsmitglieder für dich zu 
gewinnen. Mach sie zu Freunden, die aus Dankbarkeit selbst ihre Brüder und Schwestern an dich 
verraten würden.''\\
Semric blickte sie sprachlos an. Da saß diese schöne Frau vor ihm, nippte an ihrem Pfefferminztee 
und traf innerhalb weniger Minuten politische Entscheidungen an die er nicht einmal im Traum 
gedacht hätte. \textit{Wie sähe Saleica aus, wenn Ilia als Prinzessin geboren worden wäre?}, dachte 
er unwillkürlich.
Rioleans Stimme wisperte gehässig: \textit{Ich hätte sie umbringen lassen. Eine solche Rivalin in 
der Erbfolge hätte ich nicht riskiert}\\
``Nun?'' Ilia hob eine Augenbraue.\\
``Ich werde darüber nachdenken'', murmelte Semric betreten: ``Sei so freundlich und notiere mir die 
Namen, die du vorschlägst.''\\
Sie schmunzelte, griff in ihr Dekolleté und holte ein gefaltetes Papier hervor. ``Natürlich'', 
erwiderte sie und überreichte ihm die Liste.\\
\textit{Du muss sie heiraten}, murmelte Riolean mit widerwilliger Anerkennung in der Stimme: 
\textit{Du hast gar keine andere Wahl.}\\
``Hast du auch schon eine Liste für die Verlobungsfeier?'', wechselte er das Thema.\\
Ihre Miene hellte sich auf. ``Nein. Da wollte ich Mihiki Sa Elren mit einbeziehen.''\\
``Wieso das?'', fragte Semric überrascht. Er hatte kaum noch einen Gedanken an die Botschafterin 
der Kolonien verschwendet.\\
``Weil sie mich eh schon hassen wird. Immerhin habe ich ihr den Gatten gestohlen. Sie kam doch 
offiziell, um die Verbindung zwischen den Kolonien und Saleica zu festigen. Die Krone wird sie 
nicht bekommen. Aber die Freundschaft einer Königin als Ausgleich ist doch ein interessantes 
Angebot.''\\
``Ich dachte, du könntest sie nicht leiden.''\\
Ilia lächelte charmant. ``Ach Semric... bei einer Freundschaft kommt es doch nicht darauf an, ob 
man sich mag!''\\
Sie goss ihnen beiden erneut Tee ein. ``Also. Die nächsten Tage werde ich die Verlobungsfeier 
planen. Überlege dir, wen du dir in deinem Rat vorstellen könntest. Die müssen auf jeden Fall alle 
eingeladen werden. Mein Vater hat sich bereits mit Offizier Lerin über die aktuelle Lage an der 
kasirischen Grenze ausgetauscht und stellt Kontakt zu den militärischen Oberhäuptern in den 
anliegenden Kolonien da. Wegen Versorgungstruppen und alles. Vielleicht wäre es für das Volk gut, 
wenn du eine Verkündung verfasst. Lass es nicht so aussehen, als stehle das Militär all ihre 
Vorräte und das Saatgut... formuliere es so, dass sie stolz darauf sein müssen, dass ihr Korn und 
ihr Vieh unsere tapferen Soldaten nährt.''\\
Ilia bemerkte Semrics nervösen Blick und legte ihm die Hand auf den Oberarm. ``Ich kann dir dabei 
gerne helfen'', sagte sie aufmunternd lächelnd.\\
``Sonst noch was?'', fragte er.\\
``Halte die Priester raus. Spricht nicht mit Hisio-Mahar und lass nicht zu, dass sein Wort vor 
deinem kommt. Alles weitere können wir später noch besprechen.'' Ihre Stimme wurde seidiger. Sie 
beugte sie vor und hauchte einen zarten Kuss auf seine Lippen. ``Der Tee ist kalt'', flüsterte sie: 
``Es gibt also keinen Grund mehr hier in den Sesseln zu sitzen.''\\
Ilia erhob sich und näherte sie rückwärts dem Bett, während sie ihm verführerisch anlächelte und 
mit geschickten Fingern den Verschluss ihres Kleides öffnete.\\
