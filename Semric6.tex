\chapter{Aufstieg}

Ilia durchschritt das Zimmer mit ruhigen Schritten. Ihre Füße trugen sie zum vergoldeten Spiegel, 
neben dem sie stehen blieb um die Titel im Bücherregal zu betrachten. Sie nahm eines der Bücher 
über Heilkräuter heraus. Ihre langen Finger strichen über die Seiten, während ihre Augen der 
Schrift folgten. Dann stellte sie es ohne ein Wort zurück. Ihre Hand glitt an der Wandtäfelung 
vorbei, während sie zum nächsten Gegenstand ihres Interesses weiter wanderte. Ilia hatte schon 
seine Garderobe betrachtet, seinen Degen, die Sammlung Porzellanfiguren die er noch aus seiner 
Kindheit aufbewahrt hatte und kam nun bei seinen Zeichnungen an. Semric saß in einem seiner Sessel 
und beobachtete sie zurückhaltend. Alle Dinge die sie in der letzten Stunde schweigend betrachtete, 
berührte sie auch. Das war ihm gleich als erstes aufgefallen. Als würde sie die Dinge, die seine 
Persönlichkeit ausmachten, nicht nur mit den Augen betrachten wollen, sondern auch spüren. 
Plötzlich wurde Semric nervös, was sie von seinen Zeichnungen halten würde. Der Tisch war übersät 
von Pergament, Kohle und teurer, farbiger Kreide. Nur eine große Fläche zum Zeichnen war 
leer geräumt. Mitten darauf lag seine neustes Werk. Ein Bild von Ilia. Er konnte nichts anderes 
mehr zeichnen als sie. Oder Dinge, die mit ihr im Zusammenhang standen. Vor wenigen Tagen erst 
hatte er ein Bildnis ihrer Augen fertiggestellt. Dafür hatte er sich die teure Kreide bringen 
lassen. Aber ganz zufrieden war er nicht. Er hatte es nicht geschafft, den exakten Farbton ihrer 
Iris zu finden.\\
Schließlich legte sie die Zeichnung zurück und blickte ihn quer durch den Raum mit eben diesen 
Augen an. Obwohl mehrere Schritte zwischen ihnen lagen, begann Semrics Herz schneller zu schlagen 
und er befeuchtete sich nervös die Lippen.\\
Ihre Stimme klang sachlich, als sie die folgenden Worte sprach. ``Du liebst mich?''\\
Semric schwieg. Er wusste nicht, was er darauf antworten sollte, immerhin hatte sie seine 
Zeichnungen gesehen. Sie sprachen alles aus, was er im Herzen trug.\\
``Ich schließe daraus, dass du anders empfindest?'', antwortete er dann doch.\\
Ilia lächelte und kam langsam näher. ``Semric'', sagte sie sanft, kniete vor ihm nieder und ergriff 
seine Hand. ``Ich liebte schon viele Männer. Aber ich glaube, keiner brauchte mich bisher so 
wie du es tust. Ich sagte dir, dass mir Ehrlichkeit wichtig ist. Also werde ich auch ehrlich zu dir 
sein. Ich liebe dich. Aber ich werde keine Mätresse sein. Kein Mädchen, dass still hinter dir 
stehen wird oder sich gar in deinen Zimmern versteckt.''\\
Ihre Augen leuchteten, während sie weiter sprach: ``Ich kann dir nicht versprechen, dass meine 
Liebe ewig sein wird. Und ich will nicht nur deine Gattin sein.''\\
Semric blickte schweigend auf sie herab und wünschte sich, sie würde nicht knien. Nicht sie, die er 
so bewunderte.\\
``Ich will deine Königin sein'', sagte sie leise: ``Ich will an deiner Seite stehen. Die Krone 
tragen. Die Priester in ihre Tempel sperren und diesen Krieg gegen Kasir mit dir gewinnen. Ich will 
mit dir die Kolonien beherrschen, unser Reich vergrößern. Ehre und Rum erlangen. Ich will, dass die 
ganze Welt vor uns Löwen erzittert.'' Ihre Lippen berührten seine Hand. ``Alle, die unser Brüllen 
hören. Alles, was das Sonnenlicht berührt.''\\
Sie beugte sich zu ihm hoch. Es war nicht der erste Kuss in Semrics Leben, aber der erste mit 
Bedeutung. Während einer Atempause fragte er: ``Hast du mir gerade einen Antrag gemacht?''\\
``Oh ja. Und er war sehr viel romantischer als den letzten, den ich bekommen habe.''\\
Semric zuckte zurück. ``Verdammt. Mi'Kae, nicht wahr? Erhim hat es mir berichtet!''\\
Ilia setzte sich in den zweiten Sessel und strich ihr Kleid glatt. ``Mein Vater wollte es. Jozah 
ist sein Schützling, seit er damals über die Mauern der Kaserne geklettert ist. Es hat nichts zu 
bedeuten und ich habe ihm längst einen Brief geschrieben, in dem ich die Verlobung auflöse. Wenn 
ihm wirklich etwas an mir liegen würde, dann wäre er nicht nach Merandila geritten.''\\
Semric runzelte die Stirn. ``Er hatte keine Wahl. Ich habe es ihm befohlen.''\\
``Ein Mensch hat immer eine Wahl'', entgegnete Ilia: ``Was ist nun? Wirst du mich zu deiner Königin 
machen? Werden wir beide die Löwen Saleicas sein?''\\
``Das ist alles nicht so leicht wie es klingt'', wich Semric aus und blickte auf den Boden.\\
``Auch Hisio-Mahar ist nur ein Mensch. Er ist ein Priester, kein König. Und kein Löwe. Wir müssen 
ihm seine Macht nehmen.''\\
``Wie?''\\
``Kamst du je auf den Gedanken, einen neuen Rat zu erwählen?'', fragte sie direkt und blickte ihn 
forsch an: ``Und Hisio-Mahar nicht in diesen einzuladen?''\\
Semric biss sich auf die Unterlippe. Nein, auf diesen Idee ist er in all den Jahren nie gekommen. 
Er räusperte sich. ``An wen hast du gedacht?''\\
``Meinen Vater. Zum Beispiel. Er hat viele Freunde im Militär. Außerdem dachte ich an Sil'Vera.''\\
``Die junge Witwe?'', fragte Semric überrascht.\\
``Nein. Ihren Vater. Er saß bereits im Rat deines Vaters. Er ist völlig mittellos. Würdest du ihm 
zu neuer Größe verhelfen, den Namen seiner Familie retten, wäre er dir zu ewiger treue 
verpflichtet. Außerdem hat es seine Gründe gegeben, wieso König Kareen ihn berufen hat. Sein 
Verhandlungsgeschick hat damals in den Kolonien viel gebracht.''\\
``Der Gräfin würde es vermutlich auch gefallen'', murmelte Semric.\\
``Dann noch ein paar Vertreter aus den anderen Grafschaften. Nicht unbedingt die Erben des Titels, 
eher die später geborenen Kinder. Dadurch hättest du direkten Einblick in die Verwaltung deines 
Landes und sie könnten als Botschafter fungieren. Du hättest mehr Kontrolle über die Grafschaften 
und deren Herren. Und, bestenfalls, schaffst du es die Loyalität der Ratsmitglieder für dich zu 
gewinnen. Mach sie zu Freunden, die aus Dankbarkeit selbst ihre Brüder und Schwestern an dich 
verraten würden. Natürlich auch noch Vertreter aus den Kolonien. Immerhin haben sie uns in den 
letzten Tagen deutlich gemacht, dass sie nach den Jahren der Besetzung mehr von Saleicas 
Herrscher erwarten. Über eine Eingliederung sollten wir uns aber erst Gedanken machen, wenn wir 
die Konfrontation mit Kasir hinter uns haben. Ich bin mir sicher, eine höflich formulierte 
Verkündung wird der Rat in den Kolonien berücksichtigen müssen. Andererseits können wir auch das 
mit Soldaten lösen. Es wird nur schwieriger, an zwei Fronten gleichzeitig zu kämpfen.''\\
Semric blickte sie sprachlos an. Da saß diese schöne Frau vor ihm, nippte an ihrem Pfefferminztee 
und traf innerhalb weniger Minuten politische Entscheidungen an die er nicht einmal im Traum 
gedacht hätte. \textit{Wie sähe Saleica aus, wenn Ilia als Prinzessin geboren worden wäre?}, dachte 
er unwillkürlich.\\
Rioleans Stimme wisperte gehässig: \textit{Ich hätte sie umbringen lassen. Eine solche Rivalin in 
der Erbfolge hätte ich nicht riskiert}\\
``Nun?'' Ilia hob eine Augenbraue.\\
``Ich werde darüber nachdenken'', murmelte Semric betreten: ``Sei so freundlich und notiere mir die 
Namen, die du für den Rat vorschlägst.''\\
Sie schmunzelte, griff in ihr Dekolleté und holte ein gefaltetes Papier hervor. ``Natürlich'', 
erwiderte sie und überreichte ihm die Liste.\\
\textit{Du musst sie heiraten}, murmelte Riolean mit widerwilliger Anerkennung in der Stimme: 
\textit{Du brauchst sie. Du hast gar keine andere Wahl.}\\
``Hast du auch schon eine Liste für die Hochzeit?'', wechselte er das Thema.\\
Ihre Miene hellte sich auf. ``Nein. Da wollte ich Mihiki Sa Elren mit einbeziehen.''\\
``Wieso das?'', fragte Semric überrascht. Er hatte kaum noch einen Gedanken an die Botschafterin 
der Kolonien verschwendet.\\
``Weil sie mich eh schon hassen wird. Immerhin habe ich ihr den König gestohlen. Sie kam doch 
offiziell, um die Verbindung zwischen den Kolonien und Saleica zu festigen. Die Krone wird sie 
nicht bekommen. Aber die Freundschaft einer Königin als Ausgleich ist doch ein interessantes 
Angebot.''\\
``Ich dachte, du könntest sie nicht leiden.''\\
Ilia lächelte charmant. ``Ach Semric... bei einer Freundschaft kommt es doch nicht darauf an, ob 
man sich mag!''\\
Sie goss ihnen beiden erneut Tee ein. ``Also. Die nächsten Tage werde ich die Hochzeit 
planen. Überlege dir, wen du dir in deinem Rat vorstellen könntest. Die müssen auf jeden Fall alle 
eingeladen werden. Mein Vater hat sich bereits mit Offizier Lerin über die aktuelle Lage an der 
kasirischen Grenze ausgetauscht und stellt Kontakt zu den militärischen Oberhäuptern in den 
anliegenden Kolonien her. Wegen Versorgungstruppen und alles. Vielleicht wäre es für das Volk gut, 
wenn du eine Verkündung verfasst. Lass es nicht so aussehen, als stehle das Militär all ihre 
Vorräte und das Saatgut... formuliere es so, dass sie stolz darauf sein müssen, dass ihr Korn und 
ihr Vieh unsere tapferen Soldaten nährt. Wobei es nicht zu huldvoll klingen soll... das würde 
zu sehr Hisio-Mahars Reden gleichen. Eine Mischung aus Stolz, Ehrlichkeit und Ernsthaftigkeit.''\\
Ilia bemerkte Semrics nervösen Blick und legte ihm die Hand auf den Oberarm. ``Ich kann dir dabei 
gerne helfen'', sagte sie aufmunternd lächelnd.\\
``Sonst noch was?'', fragte er.\\
``Halte die Priester raus. Spricht nicht mit Hisio-Mahar und lass nicht zu, dass sein Wort vor 
deinem kommt. Alles weitere können wir später noch besprechen.'' Ihre Stimme wurde seidiger. Sie 
beugte sie vor und hauchte einen zarten Kuss auf seine Lippen. ``Der Tee ist kalt'', flüsterte sie: 
``Es gibt also keinen Grund mehr hier in den Sesseln zu sitzen.''\\
Ilia erhob sich und näherte sie rückwärts dem Bett, während sie ihm verführerisch anlächelte und 
mit geschickten Fingern den Verschluss ihres Kleides öffnete.\\

Die Schritte der beiden Damen hallten durch den leeren Ballsaal. Wo vor wenigen Tagen noch getanzt, 
getrunken und gelacht wurde, herrschte nun die Stille vor. Ohne die Dekoration, die Menschen und 
die Musik wirkte der Saal wie ein Loch in der Zeit. Er verschluckte alles.\\
Ilia legte den Kopf in den Nacken und betrachtete die Raumdecke, das Podium und die Wände. Sie 
hatte vor, letztere mit Stoffen zu verkleiden.\\
``Was findet Saleica so an Löwen?'', fragte Mihiki Sa Elren plötzlich und betrachtete eine der 
vielen Säulen. Im Gestein befanden sich die Umrisse eines wütenden, brüllenden Löwen. ``In manchen 
Savannen gibt es sie in den Kolonien'', fügte sie hinzu: ``Sie schlafen meistens. Oder fressen.''\\
``In Saleica gab es sie einst in freier Wildnis. Mittlerweile nur noch in der königlichen Zucht. 
Raubkatzen sind begehrt unter dem Adel, sie gelten als Statussymbol. Aber nur wer vom königlichen 
Blut ist, darf sie züchten lassen. Deshalb haben nun viele Adelshäuser andere Raubkatzen. Merandila 
ist zum Beispiel für seine weißen Luchse bekannt. Ringen für eine fabelhafte Zucht der Pumas. Ah... 
und eine Freundin hat sich letztens erst einen Gepard für die Jagdveranstaltungen im kommenden 
Sommer gekauft. Nun ja, die größte Ehre, die es in diesem Land gibt, ist von der königlichen 
Familie einen Löwen geschenkt zu bekommen. Es heißt, Prinzessin Riolean hatte ihr Lieblingstier mit 
in die Kolonien gebracht um dort ihren zukünftigen Verlobten die Aufwartung zu machen. Was wohl aus 
dem Tier geworden ist...''\\
``Aber warum Löwen?'', wiederholte Mihiki Sa Elren und strich sich eine schwarze Haarsträhne hinter 
das Ohr. Sie trug ein Kleid nach der saleicanischen Mode und kam sich darin unheimlich steif und 
gefangen vor. Der Stoff war schwer um den Winter entgegen zu wirken. Auch wenn in den südlichen 
Gefilden Saleicas selten Schnee fiel, war die Temperatur doch sehr viel niedriger als in ihrer 
Heimat.\\
Ilia seufzte. ``Das ist eine lange Geschichte...''\\
Mihiki Sa Elren blickte sie eindringlich an und wartete stur auf eine Antwort. Sie war sich nicht 
sicher, woher das plötzliche Wohlwollen der saleicanischen Edeldame kam, aber sie wollte es ihr 
nicht so leicht machen. Immerhin war deutlich geworden, dass diese Blondine die Frau war, die 
der König an seiner Seite haben wollte. Am liebsten wäre Mihiki Sa Elren sofort wieder auf ihr 
Schiff gestiegen und in die Heimat gesegelt, aber der Hohepriester Hisio-Mahar hatte sie gebeten, 
ihren Besuch noch nicht abzubrechen. Mihiki war zornig und fühlte sich verraten, hatte man ihr doch 
etwas völlig anderes versprochen. In seinen Briefen hatte der Priester es so dargestellt, als würde 
König Semric sie noch innerhalb weniger Wochen zur Königin machen. Und jetzt sollte sie die Hofdame 
dieser Frau spielen?\\
Ilia lächelte nachsichtig und begann zu erzählen: ``Einst herrschte im ganzen Land Chaos. Die 
Menschen waren kriegerische Stämme, die nicht über den Horizont hinaus blickten. Unser erster König 
Riol Sa'Leica war der Sohn eines Stammesführers. Er war mutig, tapfer und stolz. Deshalb hielt er 
sich auch nicht an die Friedensverträge, die sein Vater mit den umliegenden Stämmen ausgehandelt 
hatte. Er machte sich mit wenigen Freunden auf den Weg und wollte einen Stammeshäuptling des nachts 
töten. So weit kam er aber nicht. Der Häuptling stellte ihm eine Falle und statt einen schlafenden 
Mann fand sich Riol plötzlich in einem Zelt mit einem mächtigen Löwen vor. Seine primitiven Waffen 
konnten nichts gegen das mächtige Tier ausrichten. Aber anstatt zu fliehen oder heulend zu sterben, 
ging er mit seiner Keule auf den Löwen los. Ein Schlag der Pranke genügte um Riol zu Boden zu 
werfen und das Fleisch aufzuschlitzen. Der Löwe beugte sich über ihm um sein Genick zu brechen, 
aber immer noch schlug Riol - nun mit bloßen Fäusten - auf das Raubtier ein. Der Löwe hielt ganz 
still und starrte ihn aus goldenen Augen an, während er darauf wartete, dass der lächerliche Mensch 
aufgab. Es vergingen lange Minuten, ehe der Löwe sich zurückzog, sich gelassen auf den Boden nieder 
ließ und weiterhin wartete. Diesmal, dass der törichte Junge verbluten würde. Aber Riol starb nicht 
so schnell. Und schließlich hörte er die Stimme des Löwen.''\\
Mihiki hob eine Augenbraue und blickte sie skeptisch an. Ilia sich sich jedoch nicht aus der Ruhe 
bringen und fuhr fort: ``Der Löwe erzählte ihm, dass er seinen Mut bewunderte. Dass er ihn für 
einen ehrbaren Menschen hält. Den Ersten, den er je gesehen hat. Deshalb dürfe er weiter leben. Der 
Löwe berührte Riol mit seiner Schnauze und die Wunden schlossen sich. Ich, Osyma der Allmächtige, 
segne dich. Erringe in meinem Namen Ehre und dein Blut wird niemals aussterben. Riol Sa'Leica 
verneigte sich vor dem Löwen und kehrte heim. Innerhalb eines Jahres hatte er die südliche Küste 
des Landes erobert und in ein Reich vereint. Er krönte sich, wählte den Löwen als seinen Führer und 
erbaute die ersten, wackligen Tempel zu Ehren Osymas.''\\
``Aha'', erwiderte Mihiki und schüttelte nur den Kopf. Das klang nur wie eines der Märchen, welche 
in ihrer Heimat die Kinder erzählt bekamen. Darauf gründet sich die ganze Nation? Lächerlich.\\
``Dann wird das Thema der Feier wieder der Löwe sein? Wie immer?'', fragte sie spöttisch.\\
``Ich wollte dich nach einen Thema fragen. Wie werden Hochzeiten in den Kolonien gefeiert?''\\
Mihiki lächelte grimmig und schüttelte den Kopf. ``So viele Kulturen lassen sich nicht 
zusammenfassen. Und ich kenne die meisten Bräuche nicht.''\\
``Tanz!''\\
Das Wort hallte durch den leeren Saal, gefolgt von den schnellen Schritten. Erhim zeigte mit einem 
breiten Grinsen seine Zähne, als er näher kam, verneigte sich noch im gehen vor Ilia und Mihiki. 
Ohne inne zu halten, ergriff er Ilias Hand, zog sie in eine elegante Drehung und vollführte eine 
kurze Folge von Tanzschritten. ``Tanz verbindet alle Menschen in den Kolonien!''\\
Mihiki runzelte die Stirn und musterte ihn skeptisch. ``Im ersten Bezirk zeigt sich Tradition auf 
anderer Art und Weise. Deutlich anders!''\\
``Pah. Vielleicht seid ihr versucht es den langweiligen Saleicanern nachzumachen!'', spottete 
Erhim.\\
Ilia lachte. ``Langweilig? Wir? Hast du schon mal das Nachtleben in der Hauptstadt genossen? Wir 
gelten als die Temperamentvollsten diesseits des Meeres!''\\
``Oh, ich kann nur allen Göttlichen danken, dass das Meer unsere Kontinente trennt! Es stimmt 
schon, dass viele Länder in euren Kolonien unterschiedliche Bräuche haben. Einen Mittelweg findet 
man jedoch. Ich schlage vor, eine Tanzvorführung die die Liebesgeschichte des Paares zeigt.''\\
Ilia hob eine Augenbraue und überlegte kurz. ``Du machst aber keine Oper daraus.''\\
Erhim funkelte sie fast schon erbost an. ``Ein guter Tanz braucht keine Worte!''\\
``Wie hast du es nur zu einem Leibwächter geschafft'', rätselte sie und schüttelte den Kopf. ``Also 
gut, ich überlasse es dir. Du wirst schon die richtigen Momente auffangen.''\\
``Ich schätze, eine nackte Blondine darf nicht auftreten?''\\
Mihiki blickte verwirrt von einem zum anderen, aber Ilia zuckte nur mit den Schultern. ``Wenn du 
eine findest, die meiner Schönheit nahe kommt. Dann das Essen. Die Dekoration. Die Einladungen! Es 
gibt noch so viel zu tun. Werte Mihiki Sa Elren, wenn Ihr als Repräsentantin des Rats hier 
seid, solltet Ihr vielleicht auch irgendetwas repräsentatives beitragen.''\\
Mihiki rang sich ein Lächeln ab. ``Wenn unsere Geschmäcker sich einig sind, könnte ich die 
Dekoration übernehmen.''\\
``Was schwebt Euch vor?''\\
``Wasser wäre eine Abwechslung zu den ständigen Flammen. Dazu seidene Stoffe in Blautönen.''\\
Ilia nickte zustimmend. Eine Abwechslung würde wirklich gut tun. Und es würde ein Zeichen an die 
Priester sein.\\
``Ich muss mich jetzt leider verabschieden, aber ich hoffe doch, Ihr leistet mir Gesellschaft beim 
Abendessen, geschätzte Mihiki Sa Elren?''\\
Mihiki knickste, wie es Brauch war in dieser Kultur. ``Sehr gern.''\\


In den folgenden Tagen wurde die Nachricht der baldigen Hochzeit des Königs verkündet. Ilia 
besuchte alte Freunde ihrer Familie, knüpfte neue Kontakte, besuchte den Tempel und das einfache 
Volk. Man sah sie auf dem Pferdemarkt mit den Züchtern scherzen und feilschen. Bei den Fischern 
bestellte sie persönlich einige Netze zur Dekoration und gab ebenfalls einen großzügig entlohnten 
Auftrag für frischen Fisch für das Bankett. Vom rauschenden Fest in der Kaserne wurde nur hinter 
vorgehaltener Hand geflüstert. Offensichtlich war nur, dass die zukünftige Braut im Morgengrauen 
umringt von einigen Soldaten mit zerknitterten Kleid und einer zerstörten Frisur laut lachend zum 
Palast begleitet wurde. Der König kehrte noch später zurück als seine Zukünftige.\\
Auch über Mihiki Sa Elren wurde in den kommenden Wochen viel erzählt. Bei beinahe jeden dieser 
Ereignisse war sie dabei. Sie galt bei den Bürgern der Hauptstadt als zu brav und leise. Die Leute 
beobachteten sie irritiert und misstrauisch. Nicht nur, dass sie aus den Kolonien stammte, sondern 
ihr gegensätzliches Verhalten war es. Auch gingen die Gerüchte um, dass diese Fremde mit der Bitte 
nach Saleica kam, die Braut des Königs zu werden. Diese stille Fremde sollte die Löwin 
Saleicas sein?\\
Der Adel Brom-Dallars machte üble Scherze und hielten sich nicht zurück. Innerhalb weniger Tage 
galt das gesamte Volk der Kolonien als prüde und so schleimig wie Schnecken. Wieso sollten sie 
sonst eine in Gold gehüllte Stumme als Repräsentantin schicken? Egal wie Mihiki sich bemühte, sie 
fand keine Gnade vor dem Adel.\\
Ilia genoss die Aufmerksamkeit. Sie nannte viele Personen ihre Freunde. Leute, auf deren Festen sie 
schon als kleines Mädchen tanzte. Und seit ihr Name mit dem des Königs in Verbindung stand, kamen 
jeden Tag immer mehr Einladungen und Briefe. Nicht nur der Adel, die ganze Stadt blühte auf. Es war 
endlich an der Zeit, dass der König sich eine Braut wählte. Es war endlich an der Zeit, dass 
Saleica wieder von zwei Löwen regiert wurde. Das Land brauchte den Glanz einer Liebe, deren Feuer 
selbst die dunklen Tage des Krieges erhellen würde. Die Soldaten hefteten sich rote Tücher in den 
Gürtel, als Zeichen, dass sie für die baldige Königin Ehre erlangen wollten.\\
Es war nicht nur ihr Charme und die Leidenschaft, mit denen Ilia bereits das Herz des Adels besaß. 
Ihr Name, ihr Geld, die Ehre, die ihr Vater erlangte. Die Leute buhlten um ihre Aufmerksamkeit, so 
ging doch das Gerücht um, dass die baldige Braut mit dem König plante, einen neuen Rat zu eröffnen. 
Mit Ilia hatte der Adel die Chancen, neue Positionen zu erhalten. Die Machtverteilung des Landes 
könnte sich rapide verändern. Und was hatte die Repräsentantin zu bieten? Die Menschen flüsterten. 
Es war kein Geheimnis, dass der Hohepriester von der plötzlichen Verbindung zwischen den Häusern 
Sa'Leica und Ma'Sah nicht gerade erfreut war.\\


