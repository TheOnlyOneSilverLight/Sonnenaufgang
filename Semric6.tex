\chapter{Aufstieg}

Ilia durchquerte das Zimmer mit ruhigen Schritten. Ihre Füße trugen sie zum vergoldeten Spiegel, 
neben dem sie stehen blieb um die Titel im Bücherregal zu betrachten. Sie nahm eines der Bücher 
über Heilkräuter heraus. Ihre langen Finger strichen über die Seiten, während ihre Augen der 
Schrift folgten. Dann stellte sie es ohne ein Wort zurück. Ihre Hand glitt an der Wandtäfelung 
vorbei, während sie zum nächsten Gegenstand ihres Interesses wanderte. Ilia hatte schon 
seine Garderobe betrachtet, seinen Degen, die Sammlung Porzellanfiguren die er noch aus seiner 
Kindheit aufbewahrt hatte und kam nun bei seinen Zeichnungen an. Semric saß in einem seiner Sessel 
und beobachtete sie zurückhaltend. Alle Dinge die sie in der letzten Minuten schweigend 
betrachtete, berührte sie auch. Das war ihm gleich als erstes aufgefallen. Als würde sie die Dinge, 
die seine Persönlichkeit ausmachten, nicht nur mit den Augen betrachten wollen, sondern auch 
spüren. Plötzlich wurde Semric nervös, was sie von seinen Zeichnungen halten würde. Der Tisch war 
übersät von Pergament, Kohle und teurer, farbiger Kreide. Nur eine große Fläche zum Zeichnen war 
leer geräumt. Mitten darauf lag sein neustes Werk. Ein Bild von Ilia. Er konnte nichts anderes 
mehr zeichnen als sie. Oder Dinge, die mit ihr im Zusammenhang standen. Vor wenigen Tagen erst 
hatte er ein Bildnis ihrer Augen fertiggestellt. Dafür hatte er sich die teure Kreide bringen 
lassen. Aber ganz zufrieden war er nicht. Er hatte es nicht geschafft, den exakten Farbton ihrer 
Iris zu finden.\\
Schließlich legte sie die Zeichnung zurück und blickte ihn quer durch den Raum mit eben diesen 
Augen an. Obwohl mehrere Schritte zwischen ihnen lagen, begann Semrics Herz schneller zu schlagen 
und er befeuchtete sich nervös die Lippen.\\
Ihre Stimme klang sachlich, als sie die folgenden Worte sprach. ``Du liebst mich.''\\
Semric schwieg. Er wusste nicht, was er darauf antworten sollte, immerhin hatte sie seine 
Zeichnungen gesehen. Sie sprachen alles aus, was er im Herzen trug.\\
``Ich schließe daraus, dass du anders empfindest?'', antwortete er dann doch.\\
Ilia lächelte und kam langsam näher. ``Semric'', sagte sie sanft, kniete vor ihm nieder und ergriff 
seine Hand. ``Ich liebte schon viele Männer. Aber ich glaube, keiner empfand bisher so 
wie du es tust. Ich sehe dich, Semric. Dich. Den Zeichner. Den Träumer. Du bist ganz anders, als 
der König, den ich vorher kannte. Du warst am Hafen du selbst und diesen Mann finde ich sehr 
interessant. Aber du bist nicht nur das. Du bist der König und ich werde keine Mätresse sein. 
Kein dummes Ding, dass still hinter dir stehen wird oder sich gar in deinem Zimmern versteckt.''\\
Ihre Augen leuchteten, während sie weiter sprach: ``Ich kann dir nicht versprechen, dass meine 
Liebe ewig sein wird. Und ich will nicht nur deine Gattin sein.''\\
Semric blickte schweigend auf sie herab und wünschte sich, sie würde nicht knien. Nicht sie, die er 
so bewunderte.\\
``Du willst die Krone'', sagte er leise.\\
Ilia nickte und lächelte listig. ``Ja. Das ist mir in den letzten Wochen bewusst geworden. Und 
auch, dass ich in der Lage wäre, sie zu erkämpfen. Das Volk ist aufgebraucht. Die Soldaten 
frustriert. Die Priester beschäftigt. Der Adel erwartet es nahezu. Ich könnte es schaffen und weder 
du noch Hisio-Mahar könnten mich aufhalten. Aber das ist nicht nötig, denn ich will deine Königin 
sein'', sagte sie leise: ``Ich will an deiner Seite stehen. Die Priester in ihre Tempel sperren und 
diesen Krieg gegen Kasir mit dir gewinnen. Ich will mit dir die Kolonien beherrschen, unser Reich 
vergrößern. Ehre und Ruhm erlangen. Ich will, dass die ganze Welt vor uns Löwen erzittert.'' Ihre 
Lippen berührten seine Hand. ``Alle, die unser Brüllen hören. Alles, was das Sonnenlicht 
berührt.''\\
Sie beugte sich zu ihm hoch. Es war nicht der erste Kuss in Semrics Leben, aber der erste mit 
Bedeutung. Während einer Atempause fragte er: ``Hast du mir gerade einen Antrag gemacht?''\\
``Oh ja. Und er war sehr viel romantischer als der Letzten, den ich bekommen habe.''\\
Semric zuckte zurück. ``Verdammt. Mi'Kae, nicht wahr? ''\\
Ilia setzte sich in den zweiten Sessel und strich ihr Kleid glatt. ``Mein Vater wollte es. Jozah 
ist sein Schützling, seit er damals über die Mauern der Kaserne geklettert ist. Es hat nichts zu 
bedeuten und ich habe ihm längst einen Brief geschrieben, in dem ich die Verlobung auflöse.''\\
Semric runzelte die Stirn. ``Er hatte keine Wahl. Ich habe es ihm befohlen. Er ist Soldat und 
ich sein König.''\\
``Ein Mensch hat immer eine Wahl'', entgegnete Ilia: ``Und ich habe mich entschieden. Was ist nun? 
Wirst du mich zu deiner Königin machen? Werden wir beide die Löwen Saleicas sein?''\\
``Das ist alles nicht so leicht wie es klingt'', wich Semric aus und blickte auf den Boden.\\
``Auch Hisio-Mahar ist nur ein Mensch. Er ist ein Priester, kein König. Und kein Löwe. Wir müssen 
ihm seine Macht nehmen.''\\
``Wie?''\\
``Kamst du je auf den Gedanken, einen neuen Rat zu wählen?'', fragte sie direkt und blickte ihn 
forsch an: ``Und Hisio-Mahar nicht in diesen einzuladen?''\\
Semric biss sich auf die Unterlippe. Nein, auf diesen Idee ist er in all den Jahren nie gekommen. 
Er räusperte sich. ``An wen hast du gedacht?''\\
``Meinen Vater. Zum Beispiel. Er hat viele Freunde im Militär. Außerdem dachte ich an Sil'Vera.''\\
``Die junge Witwe?'', fragte Semric überrascht.\\
``Nein. Ihren Vater. Er saß bereits im Rat deines Vaters. Er ist völlig mittellos. Würdest du ihm 
zu neuer Größe verhelfen, den Namen seiner Familie retten, wäre er dir zu ewiger Treue 
verpflichtet. Außerdem hat es seine Gründe gegeben, wieso König Kareen ihn berufen hat. Sein 
Verhandlungsgeschick hat damals in den Kolonien viel gebracht.''\\
``Der Gräfin würde es vermutlich auch gefallen'', murmelte Semric.\\
``Dann noch ein paar Vertreter aus den anderen Grafschaften. Nicht unbedingt die Erben des Titels, 
eher die später geborenen Kinder. Dadurch hättest du direkten Einblick in die Verwaltung deines 
Landes und sie könnten als Botschafter fungieren. Du hättest mehr Kontrolle über die Grafschaften 
und deren Herren. Und, bestenfalls, schaffst du es die Loyalität der Ratsmitglieder für dich zu 
gewinnen. Mach sie zu Freunden, die aus Dankbarkeit selbst ihre Brüder und Schwestern an dich 
verraten würden. Natürlich auch noch Vertreter aus den Kolonien. Immerhin haben sie uns in den 
letzten Tagen deutlich gemacht, dass sie nach den Jahren der Besetzung mehr von Saleicas 
Herrscher erwarten. Über eine Eingliederung sollten wir uns aber erst Gedanken machen, wenn wir 
die Konfrontation mit Kasir hinter uns haben. Andererseits können wir auch das 
mit Soldaten lösen. Es wird nur schwieriger, an zwei Fronten gleichzeitig zu kämpfen.''\\
Semric blickte sie sprachlos an. Da saß diese schöne Frau vor ihm, nippte an ihrem Rotwein 
und traf innerhalb weniger Minuten politische Entscheidungen an die er nicht einmal im Traum 
gedacht hätte. \textit{Wie sähe Saleica aus, wenn Ilia als Prinzessin geboren worden wäre?}, dachte 
er unwillkürlich.\\
Rioleans Stimme wisperte gehässig: \textit{Ich hätte sie umbringen lassen. Eine solche Rivalin in 
der Erbfolge hätte ich nicht riskiert}\\
``Nun?'' Ilia hob eine Augenbraue.\\
``Ich werde darüber nachdenken'', murmelte Semric betreten: ``Sei so freundlich und notiere mir die 
Namen, die du für den Rat vorschlägst.''\\
Sie schmunzelte, griff in ihr Dekolleté und holte ein gefaltetes Papier hervor. ``Natürlich'', 
erwiderte sie und überreichte ihm die Liste.\\
\textit{Du musst sie heiraten}, murmelte Riolean mit widerwilliger Anerkennung in der Stimme: 
\textit{Du brauchst sie. Du hast keine andere Wahl.}\\
``Hast du auch schon eine Liste für die Hochzeit?'', wechselte er das Thema.\\
Ihre Miene hellte sich auf. ``Nein. Da wollte ich Mihiki Sa Elren mit einbeziehen.''\\
``Wieso das?'', fragte Semric überrascht. Er hatte kaum noch einen Gedanken an die Botschafterin 
der Kolonien verschwendet.\\
``Weil sie mich eh schon hassen wird. Immerhin habe ich ihr den König gestohlen. Die Krone wird sie 
nicht bekommen, aber die Freundschaft einer Königin als Ausgleich ist doch ein interessantes 
Angebot.''\\
``Ich dachte, du könntest sie nicht leiden.''\\
Ilia lächelte charmant. ``Ach Semric... bei einer Freundschaft kommt es doch nicht darauf an, ob 
man sich mag!''\\
Sie goss ihnen beiden erneut Wein ein. ``Also, falls es eine Hochzeit geben wird: Überlege dir, wen 
du dir in deinem Rat vorstellen könntest. Die müssen auf jeden Fall alle eingeladen werden. Mein 
Vater hat sich bereits mit Offizier Lerin über die aktuelle Lage an der kasirischen Grenze 
ausgetauscht und stellt Kontakt zu den militärischen Oberhäuptern in den anliegenden Kolonien her. 
Wegen Versorgungstruppen und alles. Vielleicht wäre es für das Volk gut, wenn du eine Verkündung 
verfasst. Lass es nicht so aussehen, als stehle das Militär all ihre Vorräte und das Saatgut... 
formuliere es so, dass sie stolz darauf sein müssen, dass ihr Korn und ihr Vieh unsere tapferen 
Soldaten nährt. Wobei es nicht zu huldvoll klingen soll... das würde zu sehr Hisio-Mahars Reden 
gleichen. Eine Mischung aus Stolz, Ehrlichkeit und Ernsthaftigkeit.''\\
Ilia bemerkte Semrics nervösen Blick und legte ihm die Hand auf den Oberarm. ``Ich kann dir dabei 
gerne helfen'', sagte sie aufmunternd lächelnd.\\
``Sonst noch was?'', fragte er.\\
``Halte die Priester raus. Sprich nicht mit Hisio-Mahar und lass nicht zu, dass sein Wort vor 
deinem kommt. Alles weitere können wir später noch besprechen.'' Ihre Stimme wurde seidiger. Sie 
beugte sie vor und hauchte einen zarten Kuss auf seine Lippen. ``Der Wein ist leer'', flüsterte sie: 
``Es gibt also keinen Grund mehr hier in den Sesseln zu sitzen.''\\
Ilia erhob sich und näherte sie rückwärts dem Bett, während sie ihm verführerisch anlächelte und 
mit geschickten Fingern den Verschluss ihres Kleides öffnete.\\

Er hatte sie angefleht sich weiterhin nur heimlich zu treffen, aber Ilia dachte nicht daran. 
Sie spazierte zu jeder möglichen Tageszeit durch den Palast, traf ihn in den Gärten, besuchte 
die Versammlungen der Bittsteller im Thronsaal und lieh sich sogar sein liebstes Pferd. Bis eines 
Abends im Fechtsaal die Magd zitternd vor ihr in die Knie ging.\\
Ilias Haar klebte ihr verschwitzt an der Stirn. Langsam ließ sie die Spitze ihres Degens sinken und 
verließ die Kampfstellung um sich dem Mädchen zuzuwenden. Ihr Vater, der sie zum wiederholten Male 
in der letzten Stunde besiegt hatte, tat es ihr gleich und fragte ungeduldig: ``Was ist denn? Schon 
wieder schwanger?''\\
Die Angestellte war nur am zittern und schluchzen, kaum ein Wort war zu verstehen. Ilia war gar 
nicht nach trösten zu Mute. Sie war frustriert, weil sie es trotz seines Alters nicht schaffte, 
ihrem Vater gleich zu kommen. Sie war enttäuscht, weil Semric sie heute nicht empfangen hatte und 
empört über Jozahs beleidigende Antwort auf die Auflösung der Verlobung. Aber es war deutlich, dass 
das Mädchen verstört war und es nicht schaffen würde, sich selbst zu beruhigen. Ilia kniete sich 
neben ihr und legte tröstend die Arme um sie. ``Shhhh'', wisperte sie ihr ins Ohr: ``Es ist alles 
gut.''\\
``Nein!'', tief die Weinende aus und vergrub ihren Kopf an Ilias Schulter. Ihre Tränen benetzten 
den groben Stoff des Trainingshemdes.\\
Vito Ma'Sah seufzte ungeduldig, stellte beide Degen in die Halterung zurück und nahm sich eines der 
Weingläser vom Tablett. ``Nein!'', quietschte die Magd wieder auf und der Adelige hielt in der 
Bewegung inne. Vater und Tochter tauschten bedeutungsvolle Blicke, während die Tränen weiter 
flossen. Ilia ließ sie los und holte tief Luft.\\
``Verzeiht mir!'', krächzte die Angestellte: ``Ich...''\\
Der Schlag war so kräftig, dass das Mädchen das Gleichgewicht verlor und Ilia selbst noch ein 
schmerzhaftes Pochen in der Hand spürte. Vito Ma'Sah kippte den vergifteten Wein in einen 
Blumenkübel. ``Dann ist das ja entschieden'', sagte er: ``Ich lass die Kutsche aufzäumen. Du 
packst. Wir werden deine Cousine in Ringen besuchen.''\\
Ilia hörte ihm nicht zu. Sie stand auf und verließ schon den Übungsraum. Die weichen Sohlen ihrer 
Lederstiefel erzeugten kein Geräusch, während ihre Füße sie die Treppe hinunter führten und über 
einen kleinen Seitengang direkt in den Hauseigenen Stall brachten. Mit geschickten, schnellen 
Fingern befestigte Ilia das Zaumzeug und schnalzte mit der Zunge. Der Schimmel folgte ihr artig in 
den Hof.\\
``Ilia!'', schimpfte ihr Vater aus dem Fenster: ``Es ist zu gefährlich geworden. Wir müssen 
fliehen!''\\
``Das tue ich'', erwiderte sie ruhig und schwang sich auf den bloßen Rücken der Stute.\\

Und dann stand sie dort. Verschwitzt und in farbloser Trainingskleider. Ihr Haar hatte sich aus dem 
Zopf gelöst. Ihre Augen funkelten voller Zorn und Entschlossenheit. Die geladenen Gäste - Das 
Grafenpaar aus Ringen, der Hohepriester mit zwei Begleitern, Offizier Lerim und seine Frau, Mihiki 
Sa Elren und der Rektor der Schule - sahen von ihren Tellern auf. Fisch, golden angebraten und 
auf frisch gebackenen Brot serviert. Dazu klarer heller Wein und ein Salat. Erhim hatte sie 
eingelassen und sah nun distanziert und neugierig wie alle anderen Gäste zu. Semric erhob sich 
verblüfft und die Gesandte neben ihm tat es ihm gleich. Ilia fixierte mit einem finsteren Blick 
Mihikis Hand auf Semrics Arm.\\
``Was soll das?'', fluchte der Rektor und versuchte sich einen vor Schreck verschütteten Weinfleck 
vom Wams zu wischen.\\
Ilia ignorierte die Leute bewusst und eilte auf den König zu. Es war egal, was sie nun sagen würde. 
Sie hatte bereits alles wichtige in der Stille der Zweisamkeit ausgesprochen. Aber jetzt war es 
nötig, dass er etwas für die Öffentlichkeit antwortete. Also baute sie sich nur zur vollen 
Größe vor ihm auf, würdigte Mihikis Hand keines Blickes und sah Semric fest in die Augen. ``Mein 
König. Wollt Ihr mich, Ilia Ma'Sah, heiraten?''\\
Es blieb still im Speisesaal. Nur Mihiki stieß ein schrilles Kichern aus. Der Versuch schlug fehl. 
Niemand tat es ihr gleich und machte sich über den Moment lustig. Im Gegenteil. Blicke wurden 
getauscht und auch das ein oder andere vielsagende Stirnrunzeln.\\
Ein weiterer Moment der Stille folgte in dem Ilia geduldig darauf wartete, dass Semric mit seinen 
inneren Dämonen focht, ehe er sich schließlich ein Lächeln abrang und nickte. Ilia erwiderte das 
Schmunzeln und reichte ihm ihre Hand für einen Kuss, um das Versprechen zu besiegeln. ``Dann will 
ich nicht weiter stören, Liebster'', fügte sie hinzu: ``Sei so gut und besuche mich morgen zum 
Frühstück in meinem Anwesen um meinem Vater deine Aufwartung zu machen.''\\
Semrics Lächeln wurde breiter. ``Ich freue mich'', sagte er und nickte zum Abschied, ehe er sich 
wieder zu seinen Tischgästen setzte und so tat, als hätte es keine Störung gegeben.\\
Ilia war zufrieden mit ihrem Auftritt, grüßte ein letztes Mal mit einem leichten Knicks in die 
Runde und verließ den Raum. Erhim heftete sich ihr an die Fersen.\\
``Teilen wir uns ein Pferd oder gebt Ihr mir zwei Minuten um eines satteln zu lassen?'', fragte 
er.\\
``Ich habe nichts gegen kuscheln'', antwortete Ilia: ``Ich vermute, ich werde dich eh nicht mehr 
los, Schatten?''\\
``Jetzt nicht mehr. Ist etwas passiert?''\\
Ilia nickte nur grimmig und zog sich auf ihr Reitpferd. Sie rutschte und gab Erhim die Zeit, 
ebenfalls aufzusteigen. Seine Arme schlossen sich um ihre Hüfte und sie trieb die Stute in eine 
ruhige Gangart an. Jetzt hatte sie es nicht mehr eilig.\\
``Fast.''\\
``Er hätte Euch auch so eine halbe Garnison als Wache vor die Tür stellen können.''\\
``Wenn meine eigene Angestellte sich bezahlen lässt, dann auch Soldaten. Weißt du, Schatten... eine 
Geliebte lässt sich viel leichter loswerden als eine Verlobte. Als eine zukünftige Königin. Schon 
jetzt weiß es jeder Bedienstete im Palast. Morgen früh weiß es ganz Brom-Dallar. Und in einer Woche 
das ganze Land.''\\

Ilia floh in die Öffentlichkeit. Sie organisierte die baldige Hochzeit, bestärkte alte 
Bekanntschaften und knüpfte neue. Das Volk sah sie auf dem Pferdemarkt mit Züchtern scherzen und 
feilschen. Auf dem Fischmarkt am Hafen ließ sie sich von einem blinden Fischer erklären, wie man 
Netze flocht. Bei Gottesdiensten umarmte sie Bettler und Waisen. Sie veranstaltete für ihren Vater 
ein Bankett im Palast mit ehemaligen Soldaten und Generälen die er noch aus seiner Dienstzeit 
kannte. Vom rauschenden Fest in der Kaserne wurde nur hinter vorgehaltener Hand geflüstert. Bekannt 
war nur, dass die zukünftige Braut und der König im Morgengrauen umringt von einigen Soldaten 
lachend zum Palast begleitet wurde.\\
Der Adel bemühte sich nicht, seine Absichten zu verbergen. Ein jeder erkannte die Möglichkeiten, 
die Ilia bot und buhlte um ihre Aufmerksamkeit und Freundschaft. Das Brautpaar wurde zu Jagden 
eingeladen, auf Landsitze, zum Abendessen und sogar auf ein Segelschiff.\\
Auch das einfache Volk blühte auf. Die Kinder spielten Königspaar, die Altern erinnerten sich 
verträumt an die letzte königliche Hochzeit vor mehr als dreißig Jahren zurück. Das Land brauchte 
den Glanz einer Liebe, deren Feuer selbst die dunklen Tage des Krieges erhellen würde. Die Soldaten 
und Stadtwachen hefteten sich rote Tücher in den Gürtel, als Zeichen, dass sie für die baldige 
Königin Ehre erlangen wollten und die Zahl der Rekruten stieg.\\
Und man flüsterte, dass Ilia auf das Angebot mehrere Soldaten, die für sie in den Dienst 
treten wollten, nur antwortete: ``Eine Löwin hat ihre eigenen Klauen.''
