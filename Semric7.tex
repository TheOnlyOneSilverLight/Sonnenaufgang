\chapter{Hört die Löwen brüllen}

``Früher, da musste ich dich noch auf die Schultern heben, damit in den Spiegel in unserem Salon 
schauen konntest'', seufzte Vito Ma'Sah und betrachtete seine Tochter.\
Er saß in einem der Sessel und beobachtete sie, während sie auf ihr Spiegelbild starrte. 
``Was suchst du im Spiegel?'', fragte er schließlich.\\
``Die Königin eines der mächtigsten Länder in der bekannten Welt.'' Ihre Stimme war ein 
leises Wispern und er überlegte, wann seine Tochter das letzte Mal so unsicher gewesen war. Das 
musste am Feuer ihrer Mutter gewesen sein. Damals, als Ilia kaum sieben Jahre alt war und verstand, 
dass sie tot war.\\
``Saleica hat seit vielen Jahren keine Königin mehr gesehen.''\\
``Wie war sie? Königin Hasea?''\\
Ilia suchte in ihren grünen Augen nach etwas, das königlich aussah. In ihrem Haar. Ihrem Lächeln. 
Schließlich kniff sie die Lippen wieder zusammen und wartete auf die Worte ihres Vaters.\\
``Ach... nimm sie dir nicht als Beispiel. Sie war keine Saleicanerin. Sie bemühte sich, aber ihr 
gefiel unsere Kultur nicht. Sie beschwerte sich über die Musik, über die Kunst. Und am meisten über 
das Wetter. Die längste Zeit der Ehe verbrachte sie in den nördlichen Grafschaften. Und König 
Kareen in den Kolonien. Als versuchten sie, so viel Abstand wie möglich zwischen sich zu bringen. 
Sie blieb nicht einmal für ihre Kinder in der Hauptstadt. Beteiligte sich nicht an der 
Regierung.''\\
\textit{Sie führte das Leben, welches Semric auch gerne hätte}, dachte Ilia.\\
``Sie kam aus Kasir, nicht wahr?'', murmelte sie und strich ihren Rock aus seidigen Stoff 
glatt. ``Deshalb herrschte auch so lange Waffenstillstand zwischen unseren Ländern. Sie war die 
Schwester des kasirischen Königs und er verkaufte sie für den Frieden.''\\
Vito Ma'Sah nickte. ``Kareen war es egal, welche Frau die Mutter seiner Kinder sein würde. Und 
zwei Kriege gleichzeitig wollte er nicht. Sein Blick hatte sich immer schon nach Süden gerichtet, 
als suche er etwas. Hasea wurde akzeptiert. Ihre Tochter verehrt. Riolean... sie war wie die 
aufgehende Sonne für das Volk. So viel Kraft, Ehrgeiz, Feuer. Und obwohl ihre Mutter aus Kasir 
stammte, verkündete sie schon mit jungen Jahren, dass auch bald die Zeit kommen würde, die Insel 
unter einer Krone zu vereinen. Wer weiß, wie die Welt aussehen würde, wenn Riolean die Gelegenheit 
gehabt hätte, das Zepter zu führen. Du erinnerst das Volk an Riolean. Ich denke, du erinnerst auch 
unseren König an sie.''\\
Ilia schüttelte den Kopf. ``Sie wird stets als Kriegerin beschrieben. Ich will kein Schlachtfeld 
betreten.'' Sie verzog das Gesicht zu einer Grimasse: ``Ich muss Hasea zustimmen. Ich mag keinen 
Krieg. Aber er ist gut für die Wirtschaft. Und die Soldaten brauchen es. Vielleicht wäre es ein 
Ansatz, nach diesem Krieg die stehenden Heere zu dezimieren. Sie fressen so viele Ressourcen. Aber 
es müsste alles sehr langsam gehen, so stur wie die Menschen sind. Es gibt noch so viele Länder, zu 
denen wir Kontakte knüpfen sollten. Neue Handelsbeziehungen. Waren, die wir noch nie gesehen 
haben.''\\
``Ach Kind'', seufzte Vito: ``Unterschätze diesen Krieg nicht. Konzentriere dich erst einmal 
darauf.''\\
Ilia wandte sich ihm zu und lachte. ``Ist das dein letzter Ratschlag, bevor ich heirate? Nicht 
etwa Worte wie: Folge deinem Herzen! Gehe auf Osymas Wegen! Werde glücklich, meine Tochter!''\\
Vito griff zu einem Brief und faltete ihn auf. ``Hast du eigentlich von Mi'Kae gehört?''\\
Ein weiteres Mal sah Ilia in den Spiegel und löste eines der Stoffbänder in ihrem Haar. 
Es war zu viel des Guten. ``Nein. Er hat mir nicht geantwortet.''\\
``Ich habe ihr seinen letzten Bericht, von vor ein paar Tagen. Er schrieb, dass eine 
Unterredung mit den Generälen stattfand, bei der Segnung des Ungeborenen.''\\
``Da musste er meinen Brief schon erhalten haben. Erwähnte er mich?''\\
Vito seufzte wieder. ``Ich mag ihn. Ich finde es nicht gut, dass du so mit ihm gespielt hast!''\\
Ilia funkelte ihren Vater herausfordernd an. Schon längst war die Unsicherheit in ihrer Haltung 
verschwunden. ``Ich habe nicht gespielt. Ich habe dir nachgegeben. Du wolltest das doch! Du hast 
mich nicht einmal gefragt! Und warum sollte ich als Soldatenfrau mein Dasein verbringen? Der 
Name Ma'Sah ist für größeres Bestimmt. Ich könnte Einfluss haben. Macht. Die Welt verändern.''\\
``Ich weiß. Und ich verstehe.'' Vito erhob sich, trat auf seine Tochter zu und küsste sie auf die 
Stirn. ``Pass auf dich auf, mein Kind.''\\
``Du stehst an meiner Seite, Vater?''\\
Vito nickte ernst. ``Immer.''\\


Obwohl sich so viele Menschen im Flur befanden, waren Ilias Schritte das einzige Geräusch. Dutzende 
Adelige säumten ihren Weg, der nach der Tradition im obersten Garten begonnen hatte. Ihr Kleid - 
farblich passend zu dem einzelnen Saphir, der an einer zarten Goldkette um ihren Hals lag, wehte um 
ihre Beine. Goldene Stickereien rafften den Stoff um ihre Taille. In ihrer Hand hielt sie ein 
Armband- ein Gegenstück ihrer Halskette. Dieser Saphir war seit Generationen in Familienbesitz 
und er würde das Band zwischen ihr und ihrem Gatten besiegeln. Ihr Haar trug Ilia offen. Keine 
kunstvolle Frisur sollte von dem Diadem ablenken, welches sie in wenigen Minuten krönen würde. 
Während dem Gang zum Saal, ging sie schweigend an den Gästen vorüber. Sie wurde ebenso schweigend 
empfangen. Ilia erkannte bereits das geöffnete Portal, welches in den großen Saal führte. Sie 
schritt schwebend über den Teppich. Registrierte das Nicken ihres Vaters, sein flüchtiges Lächeln. 
Sie musste sich zwingen, ihre Gesichtszüge neutral zu halten. Immerhin sollte sie die Löwin 
Saleicas werden. Es wurde Ernsthaftigkeit und Stolz bei dieser Zeremonie erwartet. Ilia 
durchschritt das Portal. Sie hatte keinen Blick für die Tischreihen, auf denen unzählige 
Köstlichkeiten gerichtet waren. Auch die Dekoration, die Blumen und die Musiker erlangten nicht 
ihre Aufmerksamkeit. Stattdessen verharrte sie reglos und starrte auf die Stelle, an der Semric 
stehen sollte. Ilias Blick huschte zu Hisio-Mahar. Der Hohepriester lächelte nicht und doch wirkte 
er ungeheuer triumphierend. Die Stille war gebrochen. Aber Ilia nahm das Wispern nicht wahr. Sie 
wirbelte herum. Sah ihren Vater in die Augen. Überflog die Menschenmenge, suchte nach Semrics 
Erscheinung. Und schließlich traf sich ihr Blick mit Erhims. Ilia öffnete den Mund, um die Frage 
auszusprechen, da hatte Erhim bereits die umstehenden Gäste grob aus den Weg gestoßen und rannte 
aus dem Saal. Ilia wollte so viel sagen, so viel fragen, aber sie blieb stumm. Sie fühlte sich wie 
unter Wasser. Das Wispern der Menge wurde lauter und doch verstand sie nichts. Ihre Augen wanderten 
erneut von ihrem Vater, der nun auf sie zu eilte, zum Hohepriester, der vor den gesegneten Flammen 
und dem Diadem stand. \\
\textit{Nein}, dachte sie: \textit{Das würde er mir nicht antun. Er liebt mich.}\\
Wachleute versuchten die Menge zu beruhigen. Sie riefen zur Ordnung, versuchten sich Gehör zu 
verschaffen. Ilia schwieg immer noch, aber erwachte aus ihrer Lähmung. Ihre Faust umschloss immer 
noch fest das Armband, während sie ihren Rock raffte und ebenfalls aus dem Saal rannte. Vorbei an 
dem Diadem, welches ihr Ziel gewesen war.\\

Ilia Ma'Sah blieb in der Tür stehen. Schweißtropfen rannten ihr über die Stirn. Ihr Haar war 
zerzaust. Ihr Ärmel am Saum eingerissen, weil sie an einem Treppengeländer hängen geblieben war. Es 
gab nur ein dumpfes Geräusch, als das teure Saphirarmband aus ihren Finger glitt und federnd auf 
dem Teppichboden ankam. Eben dieser Teppich, den Ilia am Vorabend noch als weicher als das Bett 
bezeichnet hatte. Ilia trat einen Schritt weiter in den Raum ein. Ihr Schuh hinterließ einen 
Abdruck in der Blutlache. Sie musste jedoch weiter hinein um der Spur zu folgen. Ein umgestoßener 
Sessel versperrte ihr die Sicht. Ein weiterer Schritt.\\
Und dann sah sie sein blondes Haar. Sein verzerrtes Gesicht, weiß wie frisch gefallener Schnee. Und 
die Angst in seinen Augen, ehe die Lieder sich schlossen. Seine Hand presste sich noch auf den 
dunklen Fleck an seinem Bauch. Auf der Lehnte des Sessels prangte ein blutiger Handabdruck. Semric 
musste sich an ihm festgehalten hatten, während er fiel. Ilia hielt den Atem an, während sie den 
Blick abwandte und stattdessen zum Bett sah. In eben diesem saß nackt - nur nachlässig in eine der 
dünnen Tagesdecken gehüllt - Mihiki Sa Elren. Ihr Haar war offen und zerzaust, wie Ilia ihres nur 
nach einer Liebesnacht kannte. In ihrem Blick lag eine Kälte, die Ilia erneut erstarren ließ, als 
Mihiki zu schreien begann.\\
Wachen stürzten in den Raum. Hände griffen nach Ilia, zerrten sie zur Seite, hielten sie 
umklammert. Hisio-Mahar betrat das Gemach. Er sah auf den König, der vor seinen Füßen am Boden lag 
und seine Miene blieb starr wie Stein.\\
``Lasst mich los'', flüsterte Ilia heißer und begann sich zu wehren. ``Lasst mich los!'', schrie 
sie nun gegen Mihikis Schreie an. Die Repräsentantin, die eben noch so kühl zu ihr empor gesehen 
hatte, schrie nun mit Panik in der Stimme. Sie hielt die Decke bis ans Kinn hochgezogen. Ihre Augen 
waren weit aufgerissen und ihre Worte kreischten durch das Gemach. ``Haltet sie fest! Lasst sie 
nicht los! Sie will mich umbringen! Sie hat den König erstochen! Sie hat den König erstochen!''\\
Weitere Männer stürmten in den Raum. Darunter auch Erhim, der an Semrics Seite stürmte und sich 
über ihn beugte. Der Erste, der dem Verletzten zur Hilfe kam. Ilia verstand die Worte, die Erhim 
murmelte, nicht. Aber sie hatte diesen Mann noch nie flehen gehört.\\
Hisio-Mahar ließ sich in diesem Chaos nicht dazu herab, seine Stimme zu erheben. Ein einfaches 
Handzeichen genügte und die Wachen zerrten Ilia hinaus. Sie hatten dem Priester schon immer 
gehorcht. Warum sollten sie es jetzt nicht tun, während ihr König sterbend am Boden lag?\\
``Warum kommst du erst jetzt?'', schrie Ilia und versuchte sich aus den Griffen zu lösen: 
``Erhim, warum erst jetzt?''\\

``Ich dachte, er wäre zu den Ställen gegangen. Ich dachte, er hätte sich Ajaran geschnappt und 
wollte abhauen...''\\
Ilia blickte Erhim nicht an. Sie kehrte ihm den Rücken zu und verbarg ihre zitternden Hände. 
Stattdessen betrachtete sie die Kerben und Unebenheiten im Gestein. In Wahrheit wollte sie einfach 
nicht die Zellengitter sehen. Das würde es real machen. Es stank nach faulem Stroh. Ratten huschten 
an den Wänden entlang und das einzige Licht spendeten Fackeln. Ilia stand seit Stunden in der Mitte 
der Zelle. Noch war sie rein. Noch war sie sauber. Zu erhaben für diesen Ort. Wenn sie ihrem Kummer 
nachgab, dieser Zelle nachgab, dann würde sie auch so aussehen, als würde sie hier her gehören.\\
\textit{Ich bin eine Löwin. Ich bin eine Königin}, wiederholte sie stumm.\\
Erhim umklammerte das Gitter von der anderen Seite, wie um ihr möglichst nahe zu sein. Er machte 
sich nicht die Mühe, leise zu sprechen. ``Ich dachte nicht... sie klagen dich an. Diese Hündin 
klagt dich an. Dieser dreckige Priester... sie sagen, Semric hätte sie in sein Bett geholt. Und du 
bist auf ihn losgegangen, als du sie erwischt hast.''\\
Ilia schüttelte den Kopf. ``Welche Beweise meinen sie zu haben?''\\
``Mihiki Sa Elren als Zeugin. Zwei Wachen und ein Priester in unterschiedlichen Räumen sagten, sie 
haben Semric dich auslachen gehört. Und das Messer trägt dein Siegel.''\\
``Das Messer gehört meinem Vater'', fasste Ilia zusammen.\\
``Das reicht. Hisio-Mahar hat es nicht zum Hohepriester und Regenten des Reichs geschafft, weil er 
gut sticken kann. Den Leuten war immer klar, dass er herrscht, nicht Semric. Solange Semric nicht 
widerspricht, werden sie seinen Befehlen gehorchen.''\\
Ilia wandte sich um. Sie wagte kaum zu fragen. ``Er lebt also?''\\
Erhim kniff die Lippen zusammen. ``Wäre dem nicht so, wäre ich ihm längst gefolgt. Ich habe einen 
Schwur geleistet.''\\
``Nicht sehr erfolgreich. Wieso widerspricht Semric nicht?''\\
``Er ist nicht wach. Er hat viel Blut verloren.''\\
``Wird er überleben?''\\
Erhim brachte die nächsten Worte nur mühsam hervor. ``Die Heiler sagen, dies wird sich in den 
nächsten Tagen entscheiden.''\\
``Sie werden mich hinrichten'', sagte sie leise.\\
Erhims Schweigen war Antwort genug. ``Geh'', flüsterte sie: ``Geh an die Seite des Königs. Du bist 
der Einzige, den er noch hat.''\\
Als sie alleine war, füllten Tränen ihre Augen. Sie weinte um die Krone, die sie hätte haben 
können. Um das Leben, welches sie sich erträumte. Um den Mann, in dessen Armen sie sich zum ersten 
Mal in ihrem Leben nicht einsam fühlte. Um die verpassten Chancen und die Nächte, die sie sinnlos 
verschwendet hatte. Und über ihre Naivität, dass sie damit nicht gerechnet hatte. Sie 
verabschiedete sich von allem, was sie verloren hatte. Und dann versiegten ihre Tränen. ``Ich bin 
eine Löwin'', wisperte sie und umfasste den Saphir um ihren Hals. ``Löwen weinen nicht. Sie 
brüllen. Sie zerfetzen. Sie gewinnen''\\


Das Podium befand sich im untersten Garten. Selbst das einfache Volk sollte dem Folgenden 
beiwohnen. Hisio-Mahar stand dort, umgeben von weiteren hochrangigen Priestern. Er hob seine mit 
Ringen besetzten Hand und die Menschen verstummten. Adel und Bürger. Handwerker und Bauern. 
Kaufleute und Bettler. Sie alle standen nebeneinander auf dem überfüllten Platz, drängten sich eng 
aneinander um auch nichts zu verpassen. Doch es wurde nicht still. Rufe. Fragen. Flüche.\\
Der Hohepriester legte der zitternden Mihiki Sa Elren einen Arm um die Schulter. Ihre Augen waren 
gerötet vom weinen. ``Diese Tochter unter Osymas Sonne ist heute hier, um euch - das Volk Saleicas 
- von eurem König zu erzählen.''\\
Die Menschen verstummten und starrten Mihiki an. Sie schluchzte kurz auf und verbarg ihr Gesicht, 
ehe sie mit zitternder Stimme verkündete: ``Unser geliebter König ist immer noch nicht gerettet. 
Er kämpft gegen Fieber und die tiefe Wunde. Ich wache seit Tagen an seinem Bett und bete zu Osyma, 
dass er uns den edelsten seiner Söhne lässt. Ich flehe, dass das Leben ihn nicht verlassen wird. 
Aber die Heiler wagen es nicht, mir Versprechungen zu machen.''\\
Ein Raunen ging durch die Menge. Mihiki fuhr fort mit ihren Erzählungen. Schweifte ab, welch 
schreckliche Dämonen die Hand der Attentäterin geführt haben mögen und dass Osyma sie in diesen 
schweren Stunden nicht verlassen würde. Es war ein Fischer, nahe am Podium, der sie unterbrach. 
``Wer ist diese Göre?'', schrie er an seine Kameraden gewandt. ``Als ich das letzte Mal mit der 
Verlobten des Königs sprach, hatte ihr Haar die Farbe der Sonne!''\\
Weitere Leute riefen nach der Braut des Königs. Ein Tumult brach los und die Stadtwache versuchte 
für Ordnung zu sorgen. Hisio-Mahar trat einen Schritt vor und rief: ``Ihr wollt seine Braut? Sie 
wird kommen. Deshalb sind wir heute hier. Ilia Ma'Sah, Tochter des Vito Ma'Sah, Verlobte König 
Semrics, wird angeklagt und verurteilt unseren König niedergestochen zu haben. Sie konnte nicht 
ertragen, dass unser König sich von ihr abwandte. Er kam nicht zur Hochzeit, weil er eigentlich 
seine wahre Braut hinführen wollte. Mihiki Sa Elren, Tochter des Elosas Or Elren, Ratsmitglied des 
ersten Bezirks der Kolonien und Braut des saleicanischen Königs!''
Soldaten zerrten Ilia auf das Podium. Drei Tage war sie im Kerker eingesperrt. Erhims Besuch war 
das einzige Mal, dass jemand kam, abgesehen von einer alten Frau, die ihr einen Wasserkrug gab. Die 
Stunden in der Dunkelheit hatten Ilia gezeichnet. Sie zitterte vor Schwäche. Ihr Haar hing wirr 
herab und den Rock hatte sie sich selbst abgerissen, um sich auf etwas setzten zu können, was nicht 
von Rattenkot überseht war. Ihr Blick schweifte ängstlich und erschöpft über die Menge. Sie sah 
Mihikis Lächeln nicht oder den Blick des Hohepriesters. Aber sie sah den Bottich voll Salzwasser.\\
``Das Feuer ist Osymas Geschenk an uns. Es ist heilig. Es segnet unsere Geburt, unsere Hochzeit und 
unseren Tod. Eine Verräterin ist diesem nicht würdig. Auch eine Klinge, wie sie eine verwendete um 
unseren König zu töten, ist es nicht wert sich mit ihrem dreckigen Blut zu beflecken. Ich, 
Hisio-Mahar, von Osymas Gnaden zum Hohepriester Saleicas und Brom-Dalars gesegnet, verkünde das 
Urteil. Tod durch Ertrinken.''\\
Die Soldaten zerrten sie näher an den Bottich und Ilia schüttelte panisch den Kopf. Aber sie war zu 
schwach um sich zu wehren oder zu entwinden. Eine Hand drückte ihren Kopf unter Wasser. Sie wollte 
noch schreien, da drang das kalte, brennende Wasser in ihren Mund ein. Ihr Winden wurde zu 
einem Zucken. Und dann riss jemand sie zurück. Das Wasser rann aus ihrem langem Haar und tropfte zu 
Boden, während sie keuchend nach Luft schnappte. Der Arm, der sie nun hielt, stützte sie anstatt 
sie festzuhalten. Ilia blinzelte und sah in die Augen ihres Vaters. ``Ich stehe immer an deiner 
Seite'', murmelte er dir zu.\\
Ilia richtete sich auf und strich sich das Haar aus dem Gesicht. Fassungslos sah sie sich um. 
Prügeleien zwischen der Stadtwache und dem Volk waren ausgebrochen. Männer und Frauen in der 
normalen Uniform hatten das Podium gestürmt. Ihre Schultern zierte das saleicanische Siegel - den 
brüllenden Löwen. Aber jedoch umgeben von einer aufgehenden Sonne.\\
``Dein Vater ist nicht mehr der Jüngste'', murmelte er ihr ins Ohr: ``Aber dafür hab ich doch 
trotzdem eine ganz passablen Putsch in so kurzer Zeit auf die Beine gestellt. Hisio-Mahar hat 
geheimgehalten, wann er sein Urteil vollstrecken will und der Kerker war dermaßen bewacht... mehr 
als der König. Drei meiner Leute sind bei dem Versuch gestorben, dich zu finden.''\\
Ilia dachte flüchtig an Erhim, verwarf den Gedanken dann jedoch. ``Was... passiert hier?'', fragte 
sie verwirrt.\\
``Das liegt allein in deiner Hand.''\\
Sie sah sich auf dem Podium um. Erkannte, wie Mihiki und die Priester von den Klingen der Soldaten 
in Schach gehalten wurden. Mihiki wirkte wie ein erschrockenes Reh, während Hisio-Mahars Augen vor 
Zorn funkelten. Ilia richtete sich auf und sah zu der aufgewühlten Menge. Ein Soldat kam ihr zu 
Hilfe und brüllte: ``Seid still! Die Braut des Königs will zu euch sprechen!''\\
Es half wirklich. Selbst die Wachen - welche auch nur ihre Befehle ausführten - gierten danach 
etwas zu hören. Ilias ersten Worte waren fast nur ein Krächzen. Sie hustete kurz und versuchte so 
laut wir möglich zu sprechen. ``Saleica ist ein Reich, welches seit Generationen von tapferen 
Leuten bevölkert wird. Nicht nur die Kämpfer. Nein, jeder einzelne von euch. Und ihr habt ebenso 
tapfere Herrscher verdient. Herrscher, die euch führen. Die euch direkt in die Augen sehen und 
schwören, ihr Leben für euer Wohl zu opfern. Riolean wäre so eine Herrscherin geworden. Doch die 
Kolonien haben sie euch entrissen. Sie haben euch um eure Königin betrogen.''\\
Wütende Stimmen erklangen, die ihre Worte bekräftigten.\\
``Semric ist nicht Riolean'', fuhr Ilia fort: ``Auch er wurde um eine Königin betrogen. Wir wissen 
alle, warum unsere Kronprinzessin dort war. Hisio-Mahar hatte König Kareen verführt. Es war sein 
Plan. Aber er kehrte ohne König und ohne Prinzessin zurück. Stattdessen nahm er die Regenschaft in 
Namen eines Prinzen, der alleine und viel zu jung war. Und er gab sie bis heute nicht zurück.''\\
Ilia suchte die Blicke der Menschen und versuchte ihre Worte so ehrlich zu wählen, wie es möglich 
war. Die einfachen Leute hatten schon lange genug von den schleimigen Worten der Priester. ``Semric 
hat viele von euch enttäuscht. Aber diesen Verrat hat er nicht verdient. Ich schwöre euch, die 
Zeiten werden sich ändern.'' Sie schwieg einen Moment, weil ihr die Stimme versagte. Ilia 
schluckte ein paar mal versuchte weiter zu reden: ``Wir sind im Krieg mit Kasir. Einem Feind, 
direkt vor uns. Nicht auf der anderen Seite des Meeres! Wir müssen eine vereinigte Nation sein um 
diesem Feind entgegen treten zu können. Wir sind Saleica. Wir sind Löwen! Alles, was das 
Sonnenlicht berührt. Alle, die unser Brüllen hören!''\\
Die Menschen waren nicht mehr zu halten. Sie reckten ihre Fäuste in den Himmel und schrien: 
``Alles, was das Sonnenlicht berührt! Alle, die unser Brüllen hören!''\\
Ilias Stimme versagte. Sie flüsterte ihrem Vater ins Ohr: ``Bring mich zu Semric. Und die Schlange 
und den Priester mit in die Festung.''\\


