\chapter{Hört die Löwen brüllen}

``Früher, da musste ich dich noch auf die Schultern heben, damit du in den Spiegel schauen konntest'', seufzte Vito Ma'Sah und betrachtete seine Tochter.\
Er saß in einem der Sessel und beobachtete sie, während sie auf ihr Spiegelbild starrte. 
``Was suchst du im Spiegel?'', fragte er schließlich.\\
``Die Königin eines der mächtigsten Länder in der bekannten Welt.'' Ihre Stimme war ein 
leises Wispern und er überlegte, wann seine Tochter das letzte Mal so unsicher gewesen war. Das 
musste am Feuer ihrer Mutter gewesen sein. Damals, als Ilia kaum sieben Jahre alt war und verstand, 
dass sie tot war.\\
``Saleica hat seit vielen Jahren keine Königin mehr gesehen.''\\
``Wie war sie? Königin Hasea?''\\
Ilia suchte in ihren Augen nach etwas, das königlich aussah. In ihrem Haar. Ihrem Lächeln. 
Schließlich kniff sie die Lippen wieder zusammen und wartete auf die Worte ihres Vaters.\\
``Ach... nimm sie dir nicht als Beispiel. Sie war keine Saleicanerin. Sie bemühte sich, aber ihr 
gefiel unsere Kultur nicht. Sie beschwerte sich über die Musik, über die Kunst. Und am meisten über 
das Wetter. Die längste Zeit der Ehe verbrachte sie in den nördlichen Grafschaften. Und König 
Kareen in den Kolonien. Als versuchten sie, so viel Abstand wie möglich zwischen sich zu bringen. 
Sie blieb nicht einmal für ihre Kinder in der Hauptstadt. Beteiligte sich nicht an der 
Regierung.''\\
\textit{Sie führte das Leben, welches Semric auch gerne hätte}, dachte Ilia.\\
``Sie kam aus Kasir, nicht wahr?'', murmelte sie und strich ihren Rock aus seidenen Stoff 
glatt. ``Deshalb herrschte auch so lange Waffenstillstand zwischen unseren Ländern. Sie war die 
Schwester des kasirischen Königs und er verkaufte sie für den Frieden.''\\
Vito Ma'Sah nickte. ``Kareen war es egal, welche Frau die Mutter seiner Kinder sein würde. Und 
zwei Kriege gleichzeitig wollte er nicht. Sein Blick hatte sich immer schon nach Süden gerichtet, 
als suche er etwas. Hasea wurde akzeptiert. Ihre Tochter verehrt. Riolean... sie war wie die 
aufgehende Sonne für das Volk. So viel Kraft, Ehrgeiz, Feuer. Und obwohl ihre Mutter aus Kasir 
stammte, verkündete sie schon mit jungen Jahren, dass auch bald die Zeit kommen würde, die Insel 
unter einer Krone zu vereinen. Wer weiß, wie die Welt aussehen würde, wenn Riolean die Gelegenheit 
gehabt hätte, das Zepter zu führen. Du erinnerst das Volk an Riolean. Ich denke, du erinnerst auch 
unseren König an sie.''\\
Ilia schüttelte den Kopf. ``Sie wird stets als Kriegerin beschrieben. Ich will kein Schlachtfeld 
betreten.'' Sie verzog das Gesicht zu einer Grimasse: ``Ich muss Hasea zustimmen. Ich mag keinen 
Krieg. Aber er wäre momentan gut für das Land. Die Soldaten brauchen ihn. Vielleicht wäre es ein 
Ansatz, nach diesem Krieg die stehenden Heere zu dezimieren. Sie fressen so viele Ressourcen. Aber 
es müsste alles sehr langsam gehen, so stur wie die Menschen sind. Es gibt noch so viele Länder, zu 
denen wir Kontakte knüpfen sollten. Neue Handelsbeziehungen. Waren, die wir noch nie gesehen 
haben.''\\
``Ach Kind'', seufzte Vito: ``Unterschätze diesen Krieg nicht. Konzentriere dich erst einmal 
darauf.''\\
Ilia wandte sich ihm zu und lachte. ``Ist das dein letzter Ratschlag, bevor ich heirate? Nicht 
etwa Worte wie: Folge deinem Herzen! Gehe auf Osymas Wegen! Werde glücklich, meine Tochter!''\\
Vito griff zu einem Brief und faltete ihn auf. ``Hast du eigentlich von Jozah gehört?''\\
Ein weiteres Mal sah Ilia in den Spiegel und löste eines der Stoffbänder in ihrem Haar. 
Es war zu viel des Guten. ``Nein. Er hat mir nicht geantwortet.''\\
``Ich habe hier seinen letzten Bericht, von vor ein paar Tagen. Er schrieb, dass eine 
Unterredung mit den Generälen stattfand, bei der Segnung des Ungeborenen.''\\
``Da musste er meinen Brief schon erhalten haben. Erwähnte er mich?''\\
Vito seufzte wieder. ``Ich mag ihn. Ich finde es nicht gut, dass du so mit ihm gespielt hast!''\\
Ilia funkelte ihren Vater herausfordernd an. Schon längst war die Unsicherheit in ihrer Haltung 
verschwunden. ``Ich habe nicht gespielt. Ich habe dir nachgegeben. Du hast 
mich nicht einmal gefragt! Und warum sollte ich als Soldatenfrau mein Dasein verbringen? Der 
Name Ma'Sah ist für größeres Bestimmt. Ich kann Einfluss haben. Macht. Die Welt verändern.''\\
``Ich weiß. Und ich verstehe.'' Vito erhob sich, trat auf seine Tochter zu und küsste sie auf die 
Stirn. ``Pass auf dich auf, mein Kind.''\\
``Du stehst an meiner Seite, Vater?''\\
Vito nickte ernst. ``Immer.''\\


Obwohl sich so viele Menschen im Flur befanden, waren Ilias Schritte das einzige Geräusch. Dutzende 
Adelige säumten ihren Weg, der nach der Tradition im obersten Garten begonnen hatte. Ihr Kleid - 
farblich passend zu dem einzelnen Saphir, der an einer zarten Goldkette um ihren Hals lag, wehte um 
ihre Beine. Goldene Stickereien rafften den Stoff um ihre Taille. In ihrer Hand hielt sie ein 
Armband- ein Gegenstück ihrer Halskette. Dieser Saphir war seit Generationen in Familienbesitz 
und er würde das Band zwischen ihr und ihrem Gatten besiegeln. Ihr Haar trug Ilia offen. Keine 
kunstvolle Frisur sollte von dem Diadem ablenken, welches sie in wenigen Minuten krönen würde. 
Während dem Gang zum Saal, schritt sie schweigend an den Gästen vorüber. Sie wurde ebenso schweigend 
empfangen. Ilia erkannte bereits das geöffnete Portal, welches in den großen Saal führte. Sie 
schwebte über den Teppich, registrierte das Nicken ihres Vaters, sein flüchtiges Lächeln. 
Die Adelige musste sich zwingen, ihre Gesichtszüge neutral zu halten. Immerhin sollte sie die Löwin 
Saleicas werden. Es wurde Ernsthaftigkeit und Stolz bei dieser Zeremonie erwartet. Ilia 
durchschritt das Portal. Sie hatte keinen Blick für die Tischreihen, auf denen unzählige 
Köstlichkeiten gerichtet waren. Auch die Dekoration, die Blumen und die Musiker erlangten nicht 
ihre Aufmerksamkeit. Stattdessen verharrte sie reglos und starrte auf die Stelle, an der Semric 
stehen sollte. Ilias Augen suchten Hisio-Mahar. Der Hohepriester lächelte nicht und doch wirkte 
er ungeheuer triumphierend. Die Stille war gebrochen. Aber Ilia nahm das Wispern nicht wahr. Sie 
wirbelte herum. Sah ihren Vater in die Augen, überflog die Menschenmenge, suchte nach Semrics 
Erscheinung. Und schließlich traf sich ihr Blick mit Erhims. Ilia öffnete den Mund, um die Frage 
auszusprechen, da hatte Erhim bereits die umstehenden Gäste grob aus den Weg gestoßen und rannte 
aus dem Saal. Ilia wollte so viel sagen, so viel fragen, aber sie blieb stumm. Sie fühlte sich wie 
unter Wasser. Das Wispern der Menge wurde lauter und doch verstand sie nichts. Ihre Augen wanderten 
erneut von ihrem Vater, der nun auf sie zu eilte, zum Hohepriester, der vor den gesegneten Flammen 
und dem Diadem stand. \\
\textit{Nein}, dachte sie: \textit{Das würde er mir nicht antun. Er liebt mich. Er braucht mich!}\\
Wachleute versuchten die Menge zu beruhigen. Sie riefen zur Ordnung, versuchten sich Gehör zu 
verschaffen. Ilia schwieg immer noch, aber erwachte aus ihrer Lähmung. Ihre Faust umschloss 
noch fest das Armband, während sie ihren Rock raffte und ebenfalls aus dem Saal rannte. Vorbei an 
dem Diadem, welches ihr Ziel gewesen war.\\

Ilia Ma'Sah blieb in der Tür stehen. Schweißtropfen rannten ihr über die Stirn. Ihr Haar war 
zerzaust. Ihr Ärmel am Saum eingerissen, weil sie an einem Treppengeländer hängen geblieben war. Es 
gab nur ein dumpfes Geräusch, als das teure Hochzeitband aus ihren Finger glitt und federnd auf 
dem Teppichboden aufkam. Eben dieser Teppich, den sie am Vorabend noch als weicher als das Bett 
bezeichnet hatte. Ilia trat einen Schritt weiter in den Raum ein. Ihr Schuh hinterließ einen 
Abdruck in der Blutlache. Sie musste jedoch weiter hinein um der Spur zu folgen. Ein umgestoßener 
Sessel versperrte ihr die Sicht. Ein weiterer Schritt.\\
Und dann sah sie sein blondes Haar. Sein verzerrtes Gesicht, weiß wie frisch gefallener Schnee. Und 
die Angst in seinen Augen, ehe die Lieder sich schlossen. Seine Hand presste sich noch auf den 
dunklen Fleck an seinem Bauch. Auf der Lehne des Sessels prangte ein blutiger Handabdruck. Semric 
musste sich an ihm festgehalten hatten, während er fiel. Ilia hielt den Atem an, während sie den 
Blick abwandte und stattdessen zum Bett sah. In eben diesem saß nackt - nur nachlässig in eine der 
dünnen Tagesdecken gehüllt - Mihiki Sa Elren. Ihr Haar war offen und zerzaust, wie Ilia ihres nur 
nach einer Liebesnacht kannte. In ihrem Blick lag eine Kälte, die Ilia erneut erstarren ließ, als 
Mihiki zu schreien begann.\\
Wachen stürzten in den Raum. Hände griffen nach Ilia, zerrten sie zur Seite, hielten sie 
umklammert. Hisio-Mahar betrat das Gemach. Er sah auf den König, der vor seinen Füßen am Boden lag 
und seine Miene blieb starr wie Stein.\\
``Lasst mich los'', flüsterte Ilia heißer und begann sich zu wehren. ``Lasst mich los!'', schrie 
sie nun gegen Mihikis Schreie an. Die Repräsentantin, die eben noch so kühl zu ihr empor gesehen 
hatte, schrie nun mit Panik in der Stimme. Sie hielt die Decke bis ans Kinn hochgezogen. Ihre Augen 
waren weit aufgerissen und ihre Worte kreischten durch das Gemach. ``Haltet sie fest! Sie will mich umbringen! Sie hat den König erstochen! Sie hat den König erstochen!''\\
Weitere Männer stürmten in den Raum. Darunter auch Erhim, der an Semrics Seite stürmte und sich 
über ihn beugte. Der Erste, der dem Verletzten zur Hilfe kam. Ilia verstand die Worte, die Erhim 
murmelte, nicht. Aber sie hatte diesen Mann noch nie flehen gehört.\\
Hisio-Mahar ließ sich in diesem Chaos nicht dazu herab, seine Stimme zu erheben. Ein einfaches 
Handzeichen genügte und die Wachen zerrten Ilia hinaus. Sie hatten dem Priester schon immer 
gehorcht. Warum sollten sie es jetzt, während ihr König sterbend am Boden lag, nicht tun?\\
``Warum kommst du erst jetzt?'', schrie Ilia und versuchte sich aus den Griffen zu lösen: 
``Erhim, warum erst jetzt?''\\

Ilia blickte den Hohepriester nicht an. Sie kehrte ihm den Rücken zu und verbarg ihre zitternden 
Hände. Stattdessen betrachtete sie die Kerben und Unebenheiten im Gestein. In Wahrheit wollte sie 
einfach nicht die Zellengitter sehen. Das würde es real machen. Es stank nach faulem Stroh. Ratten 
huschten an den Wänden entlang und das einzige Licht spendeten Fackeln. Ilia stand in 
der Mitte der Zelle. Noch war sie rein. Noch war sie sauber. Zu erhaben für diesen Ort. Wenn sie 
ihrem Kummer nachgab, dieser Zelle nachgab, dann würde sie auch so aussehen, als würde sie hier her 
gehören.\\
\textit{Ich bin eine Löwin. Ich bin eine Königin}, wiederholte sie stumm.\\
Sie sah den Schatten, den das Licht der Fackel an die Wand warf. Der Umriss des kahlen Schädels. 
Die Falten des Umhangs. Er machte sich nicht die Mühe, leise zu sprechen. ``Einen Moment, einen 
flüchtigen Moment, habe ich mir wirklich Sorgen gemacht,'' erklärte Hisio-Mahar: ``Da kam diese 
blonde Frau mit ihrem hübschen Lächeln und großen Brüsten und eroberte die Stadt, den Adel, das 
Volk. Einen Moment habe ich wirklich an die göttliche Fügung gezweifelt. Aber ich erkannte, es war 
alles wieder nur eine Prüfung. Der Allmächtige will wissen, ob ich bereit bin, sein Reich auf 
Erden zu führen. Ob ich ihm würdig bin oder nur ein Feigling, der betend auf Rettung wartet. Ich 
bin bereit. Ich bin würdig. Deshalb kann ich auch leider nicht selbst Euer Urteil sprechen, 
geschätzte Dame.''\\
Ilia schüttelte den Kopf. ``Ich dachte, Ihr würdet versuchen mich zu töten. Aber wenn Semric 
stirbt, zerreißen eure Fäden..''\\
Er lachte trocken. ``Ihr hattet also doch Angst? Obwohl Ihr so bescheiden Leibwächter abgelehnt 
habt? Es war mein Fehler. Es war zu spät. Hätte ich Euch getötet, dann hätte sich nichts zu meinen 
Gunsten geändert. Ihr habt Semric die Augen geöffnet.''\\
``Ist er tot?'', fragte Ilia leise. Sie schaffte es nicht, das Zittern in ihrer Stimme zu 
verbergen.\\
``Bald. Ich werde nicht darauf warten. Die Zeit ist gekommen, mich den wirklich wichtigen Dingen 
zuzuwenden. Und diese liegen nicht in Saleica. Noch nicht.''\\
Nun wandte sich Ilia doch um. Überrascht sah sie ihn an, zeigte ihre Verwirrung offen. ``Wovon 
sprecht Ihr?''\\
``Als ob ich Euch das verraten würde. König Inhem wird Eure Verurteilung und Hinrichtung leiten. 
Gleich das Erste, was er nach seiner Krönung heute Abend erledigen wird.''\\
``Inhem? Inhem El'Kera?''\\
``Der Nächste in der Thronreihe. Er kam heute Morgen im Palast an. Unser neuer König genoss lieber die Flammen des 
Tempels als die Gesellschaft des restlichen Adels'', spottete Hisio-Mahar und wandte sich mit den 
Worten: ``Ach... ich wollte Euch noch mitteilen, dass Ihr selbstverständlich Osymas Gnade übergeben 
werdet. Um Eure Seele zu reinigen und vielleicht läd er Euch dann doch an seinen Tisch. Immerhin 
ist es sehr tapfer, seinen König zu ermorden.''\\
\textit{Sie werden mich verbrennen...}\\
Als sie alleine war, füllten Tränen ihre Augen. Sie weinte um die Krone, die sie hätte haben 
können. Um das Leben, welches sie sich erträumte. Um den Mann, in dessen Armen sie sich zum ersten 
Mal in ihrem Leben nicht einsam fühlte. Um die verpassten Chancen und die Nächte, die sie sinnlos 
verschwendet hatte. Und über ihre Naivität, dass sie damit nicht gerechnet hatte. Sie 
verabschiedete sich von allem, was sie verloren hatte. Und dann versiegten ihre Tränen. ``Ich bin 
eine Löwin'', wisperte sie und umfasste den Saphir um ihren Hals. ``Löwen weinen nicht. Sie 
brüllen. Sie zerfetzen. Sie gewinnen''\\


Er war Nomade. Er beherrschte zahllose Sprachen, den tödlichen Tanz und den Zauber des Erzählens. 
Und er kannte die Zeichen des Todes.\\
Semric lag nun in seinem zerwühlten Bett. Niemand war gekommen um nach dem König zu sehen. Keine 
Priester, keine Diener, keine Soldaten. Nur er war hier und wachte am Sterbebett des Löwen.\\
Erhim hatte versucht die Wunde abzubinden, aber das Blut färbte den Stoff schnell ein. So viel 
Blut. Er kniff die Augen zusammen, presste seine zitternden Hände auf die Wunde und versuchte die 
Tatsache, dass er genau wusste, wie tödlich die Wunde war, zu verdrängen. Er murmelte in der 
Sprache seiner Mutter, flehte zu den Mahig-Na. Zu den Schöpfern selbst. Zu Osyma. Zu jedem 
verdammtem göttlichen Wesen, was es wohl geben mag. Egal. Nur irgendjemand sollte etwas tun!\\
Das letzte Mal hatte Erhim geweint, als er auf dem Schiff stand und am Horizont seine Heimat 
verschwand. Und davor, als er den leblosen, bleichen Körper seines ertrunkenen Bruders in den Armen 
hielt. Er wollte nicht wieder etwas oder jemanden verlieren, der so wichtig war.\\ ``Es darf nicht 
sein'', knurrte er in der Sprache seiner Mutter und verstärkte den Druck.\\
``Lass los.''\\
Vor Überraschung riss Ehrim den Kopf hoch und lockerte wirklich kurz seinen Griff. Weiteres Blut 
sickerte durch den Verband und er stemmte sich wieder fest auf die Wunde. ``Verschwinde, Weib!'', 
knurrte der Leibwächter und starrte finster auf das viele Blut.\\
Die Priesterin war alt. Und trotz ihres Alters zeichneten sich kaum Muster auf ihrer Haut ab. Warum 
schickte Hisio-Mahar eine alte, ranglose Vettel? Sollte die Großmutter das Schicksal des Königs 
besiegeln? Erhim verzog das Gesicht. Selbst im letzten Handgriff noch eine Beleidigung.\\
Ihr Haar war weiß und lang. Der Blick aus wässrigen blauen Augen auf die Szenerie gerichtet. Aus 
den Augenwinkeln sah er ihren Buckel, ihre Falten, ihre dünnen Glieder. Und doch hallte in ihrer 
Stimme etwas nach. ``Lass los.''\\
Wieder hob Erhim den Blick. Suchte in ihren Augen, was er in ihrer Stimme nicht erkennen konnte. So 
klar. So hell. So mächtig.\\
``Wer bist du?'', fragt er und kniff die Augen misstrauisch zusammen.\\
Ihr Blick blieb kühl und berechnend. ``Die Antwort auf dein Flehen.''\\
Erhim leckte sich über die trockenen Lippen und überlegte fieberhaft. ``Du bist eine Priesterin!'', 
stieß er schließlich hervor.\\
Sie hob ihre faltigen, klauenartigen Hände und betrachtete sie interessiert. ``Anscheinend.''\\
``Was hast du vor?'', fauchte Erhim: ``Wer bist du?''\\
Er hätte ihr am liebsten die dürre Kehle aufgeschlitzt, aber dafür würde er seinen Griff lockern 
müssen. Im Messerwerfen war er miserabel.\\
``Lass los.''\\
Die Tür flog auf und es knallte, als das Holz gegen die Wand stieß. Offizier Lerin war trotz seines 
Alters noch eine stattliche Erscheinung, was jedoch auch an dem Zweihänder liegen könnte.\\
Was auch immer der Offizier vorgehabt hatte, als er die Tür aufstieß und mit gezogener Waffe 
eintrat. Vielleicht war er gekommen, um es zu beenden. Vielleicht um zu retten. Erhim würde es nie 
erfahren, denn der Mann hielt inne. Die Alte wandte sich ihm zu. Sie sagte nichts. Ihre Miene zeigte 
keine Regung. Langsam sank die Klinge zu Boden.\\
``Lass los'', wiederholte sie ruhig.\\
Ein Gedanke keimte in ihm. Er zögerte nur kurz und fragte flüsternd: ``Bist du einer der 
Schöpfer?''\\
``Ich bin nur eine Dienerin. Ich werde dein Flehen erhören. Und du wirst meines erhören. Wir sind 
alle nur Diener. Bete, dass die Schöpfer es niemals als nötig erachten werden, sich uns zu 
offenbaren.''\\
``Was?''\\
``Lass los.''\\
Er sah sie an. Er wagte es nicht, in Semric bleiches Gesicht zu schauen, als er die Hände hob und 
das Blut ungehindert aus seinem Körper floss. Es blieb still. Bis Semric vor Schmerz stöhnte. 
Erhim starrte auf den Verband der das Blut nicht hatte halten können. Er wagte nicht, den 
durchtränkten Stoff zu berühren. Aber er sah, dass Semrics Miene sich entspannter. Der Schmerz war 
gewichen. Einen furchtbaren Moment lang dachte Erhim, sein Herr - sein König - wäre gestorben. Aber 
sein Brustkorb hob sich regelmäßig. Erhims tastenden Finger fanden den pulsierenden Herzschlag am Handgelenk.\\
``Reist nach Süden'', fuhr die Alte fort, als wäre eben kein Wunder geschehen: ``Über das Meer. 
Durch die Savannen und Berge. Findet die Mahig-Na. Zeigt ihnen dies.'' Sie hielt einen kantigen 
Stein empor. So schwarz, dass er alles Licht zu verschlingen schien.\\
``Was ist das?'', fragte Erhim krächzend und sah verwirrt vom Offizier zur Priesterin.\\
``Nichts von dieser Welt'', warf ihre Antwort: `Aber es sollte genügen. Hoffe ich... Bete ich.''\\
Erhim schüttelte den Kopf und rief: ``Du altes Weib bist verrückt!''\\
Zum ersten Mal veränderte sich die Miene der Frau und zeigte ein faltiges Schmunzeln. ``Nun... 
dann lass es. Ich habe es nicht nötig, dich zu überzeugen. Es ist nicht meine Seele, die 
verklingen wird, wenn die Schöpfer erscheinen. Ich werde in Merandila gebraucht'', ihre letzten 
Worte richtete sie an den mittlerweile knienden Offizier: ``Und du auch.''\\
Erhim öffnete den Mund, wollte weitere Fragen stellen, doch da sank das Weib in sich zusammen. Sie 
fiel hart auf den Boden. Ihr Blick verschleierte, wurde vom Alter getrübt. Keuchend und rasselnd 
holte sie Luft. Ihre dürren Finger griffen sich an die Brust und dann verstummte ihr Herz.\\
Der Offizier brach die Stille. Er hob in einer feierlichen Geste an die Tote die Hand zum Gruß und 
murmelte wie zu sich selbst, aber voller Überzeugung: ``Die Helle lebt!''\\







