\chapter{Prolog}

Die Sonne strahlte seit Stunden auf das Land und ließ alles in Hitze erglühen. Auch der warme Wind 
verschaffte ihrer verschwitzten Haut keine Kühlung. Sie liebte es. Das feuchte Fell des Pferdes 
unter ihren Fingern. Die vor Kraft zitternden Muskeln. Der wirbelnde rote Staub unter den Hufen. 
Der Geruch nach Blut und Schweiß. Kilha lag vor ihr, die endlose Steppe hinter ihr.\\
\textit{Meine Stadt}, dachte Riolean überlegen: \textit{Meine Steppe.}\\
Ihr Blick wanderte über ihre Leute. Männer und Frauen, die einst als artige Soldaten in Reih und 
Glied standen und nun als ihre wilden Löwen über das Land herfielen. \textit{Mein Land.}\\
Der Schimmel ihres Vaters näherte sich. Sie betrachtete die goldene Krone auf seinem blonden Haupt, 
sein kantiges, unrasiertes Gesicht und das stolze Funkeln in den blauen Augen. Das liebevollste 
Lächeln zu dem er fähig war, schenkte er ihr. ``Du solltest heimkehren, Vater'', raunte sie ihm zu: 
``Kehre heim in dein Königreich. Dies hier ist meins!''\\
Er sah sie nur skeptisch an und verkniff sich ein Lächeln. ``Wir haben gefunden, was wir 
suchten.''\\
Riolean sah herab in ihre Hand. Es war unnötig, denn sie spürte das Pulsieren der Scherbe. Sie 
strahlte Hitze aus, wie alles hier in dieser Gegend. Und doch so schwarz, als würde es das Licht 
verschlucken wollen. Mit einem Schauern erinnerte Riolean sich an die Wesen in der Höhle. Ihr 
Brüllen. Die Hitze. Der Gestank nach brennendem Fleisch. Es war ein angenehmer Schauer. ``Und du 
kannst nicht hier bleiben'', fügte er streng hinzu: ``Du bist meine Erbin. Das hier ist nur Dreck 
und Staub. Wir haben, was wir suchten.''
Riolean betrachtete diesen König eines so großen und starken Landes nachdenklich. Ob er auch nur 
eine Ahnung hatte, wie sehr sie ihn verabscheute? Für all das, was er ihrer Mutter angetan 
hatte? Ihrem Bruder? Ihrem Volk? Sie war sich sicher, dass ihre Mutter sie geliebt hätte, wenn 
sie ihre Kindheit bei ihr hätte verbringen dürfen. Sie wäre als ihre Mutter gestorben, nicht als 
eine Fremde, die schnell vergessen wurde. \textit{Ich werde nicht mehr zurück gehen. Ich habe 
meine Heimat gefunden. Im Staub und im Dreck. In den Savannen und Bergen}, dachte sie: \textit{Und 
mein Bruder wird mich vergessen.} Mit der freien Hand hob sie ihren Säbel. ``Alles, was das 
Sonnenlicht berührt! Jeder, der unser Brüllen hört!''\\
Ihre Krieger jubelten. Riolean trieb ihr Pferd an und sie galoppierten in einem breiten Haufen 
auf die Stadt zu. Breite Lehmbauten mit flachen Dächern. Wahllos angeordnet, wie ein kleines 
Dorf das immer weiter gewachsen war und nun einem Labyrinth aus zahllosen Leben glich. Als sie die 
ersten Häuser passierten, parierte sie in den Trab und begegnete den Blicken der Menschen. Dunkle 
Augen sahen sie an und Riolean wusste, was sie sahen. Eine junge Frau in der Blüte ihres Lebens. 
Die schwarzen Locken von Wind und Schweiß zerzaust. Kaum gebändigt durch die schmale Krone. In ihren 
Augen lag der Triumph und der Sieg. Sie war die Prinzessin Saleicas. Sie würde die Königin der 
Kolonien sein, sie zu ihrem Land machen. \textit{Scheiß auf Saleica! Scheiß auf Osyma!}
Und um das diesen Menschen zu beweisen, trugen ihre Männer das Zeichen ihres Sieges. Der Kopf so 
groß wie ein Schaf, die gelben Augen starrten ins Nichts. Sein Maul noch im Todesschrei 
aufgerissen, offenbarte die scharfen Zähne, die elf Leben nahmen, bevor es selbst zu Grunde ging. 
Ins seinem Leben hatte die schuppige Haut geschimmert wie Rubin, aber nun war aller Glanz 
verblasst. Farblos und stumpf wie diese Wüste aus Kies und Dreck, die die Oase umgab.
\textit{Ich habe ihren Gott getötet}, dachte die Kriegerin und genoss diese Worte.\\
Ein raues Murmeln begleitete sie durch die Gassen von Kilha. Augen, die sie verfolgten. Flüche und 
Gebete die in der rau klingenden Sprache geflüstert wurden. Und dann die schockierte Stille, wenn 
sie den Kopf des Gottes sahen. Zum ersten Mal in ihrem Leben spürten die Menschen Kilhas Kälte, bei 
dem Anblick des Wesens. Es war die Angst, die sie lähmte. Und es war die Angst, die einige Männer 
und Frauen antrieb. Angst, durch die sie ihre Waffen hoben. Nicht vor Prinzessin Riolean und den 
Eroberern. Nicht davor zu sterben oder zu leiden. Angst vor der Endgültigkeit.\\
Riolean sah es in ihren Augen, als ihr Pferd stehen blieb und sie auf die zahlreichen Menschen sah, 
die ihr den Weg versperrten. Sie sah ihre zitternden Hände. Ihre Tränen. Ihre Verzweiflung. Und die 
Klingen, Knüppel und Schleudern. Sie sah ihren Fehler. Riolean hob ihren Säbel und ihre Löwen 
brüllten.